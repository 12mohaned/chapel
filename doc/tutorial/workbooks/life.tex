\documentclass[10pt]{article}

\usepackage{fullpage}
\usepackage{times}
\usepackage{listings}
\lstdefinelanguage{chapel}
  {
    morekeywords={
      and, array, atomic,
      begin, bool, break,
      call, class, cobegin, complex, config, const, constructor, continue,
      def, distribute, do, domain,
      else, enum, except,
      for, forall,
      goto,
      if, imag, implements, in, int, inout, _invariant, iterator,
      let, like,
      module,
      nil, not,
      on, or, ordered, otherwise, out,
      param, _private, private, public,
      real, record, _release, repeat, return,
      select, serial, single, subtype, sync
      then, to, type, typeselect,
      uint, union, until, _unordered,
      var, _view,
      when, where, while, with,
      yield
    },
    sensitive=false,
    mathescape=false,
    morecomment=[l]{//},
    morecomment=[s]{/*}{*/},
    morestring=[b]",
}

\lstset{
    basicstyle=\footnotesize\tt,
    keywordstyle=\bf,
    commentstyle=\em,
    showstringspaces=false,
    flexiblecolumns=false,
    numbers=left,
    numbersep=5pt,
    numberstyle=\tiny,
    numberblanklines=false,
    stepnumber=0
  }

\newcommand{\chpl}[1]{\lstinline[language=chapel,basicstyle=\normalsize\tt,keywordstyle=]!#1!}

\lstnewenvironment{chapel}{\lstset{language=chapel,xleftmargin=2pc}}{}


\pagestyle{empty}

\begin{document}
\lstset{language=chapel}

\section*{Game of Life Exercise}

Conway's Game of Life is played on a \chpl{n} by \chpl{n} grid.  Each
space is either alive (a cell) or dead (empty).  On subsequent
iterations, the cells are updated according to the following rules:
\begin{itemize}
\item A live cell dies if it has fewer than two neighbors. (starvation)
\item A live cell dies if it has more than three neighbors. (overcrowded)
\item A live cell with two or three neighbors survives.
\item An empty space becomes a live cell if it has exactly three live
  neighbors. (birth)
\end{itemize}

\noindent
Using life.chpl as a starting point, fill in the main loop.  The
random initialization and output procedures are already written.  When
you are finished, you should be able to run your code to see something
like:
\begin{quote}
\begin{footnotesize}
\begin{verbatim}
> a.out --n=3 --p=40
Initial Grid
+---+
|  o|
|oo |
|o o|
+---+
Iteration 1
+---+
| o |
|o o|
|o  |
+---+
Iteration 2
+---+
| o |
|o  |
| o |
+---+
Iteration 3
+---+
|   |
|oo |
|   |
+---+
Iteration 4
+---+
|   |
|   |
|   |
+---+
Stable Grid
\end{verbatim}
\end{footnotesize}
\end{quote}

\end{document}
