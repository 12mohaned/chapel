
\section{Introduction}\label{s:intro}
Chapel is an emerging parallel language being developed by Cray Inc. with
the goal of improving programmer productivity on large-scale systems as
well as on the desktop. It has been developed with the goal being to a
large extent the  standard HPCesque battery of programs which in the large
majority of cases do not involve heavy string based processing, and up
until recently this held true with the vast majority of HPC applications.

However, with the event of big data, and especially looking at iterative
programs we see that in many cases these programs can
benefit from an in memory MapReduce as demonstrated in \cite{zaharia2010spark},
in which the files data is read in and worked on the node that hosts that data 
- retaining the data on that node until it is told that it is no longer needed - 
and doing MapReduce operations using these nodes.

We assert that Chapel could work quite well in this framework. In this
document we explore how to go about adding MapReduce operations to Chapel as 
well as how this might benefit the user not only in terms of speed, but also 
in terms of user productivity.

%We also explore another approach: a MapReduce-mapreduce type operation in which data
%aggregation is done in the first MapReduce, at the end of which all the data that is
%needed is mapped to different nodes locally in memory, at which point we can then run
%a regular program in parallel over this data. This gives performance improvements not
%only in iterative tasks, but also in tasks that require large amounts of contextual
%information (since the reduce at the end of the first MapReduce can condense and
%give us all the information that we need). Not only does this give us the opportunity to then
%explore greater degrees of parallelism as well as faster communication between nodes,
%but we can also utilize non-standard architectures (GPU, MIC) to our benefit. 
%
%\tz{think: supercomputer hooked up to a Hadoop cluster}
