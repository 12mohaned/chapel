

%\documentclass[preprint]{sigplanconf}
\documentclass[10pt]{article}
\usepackage{graphicx}
\usepackage{epstopdf}
\usepackage{float}
\usepackage{hyperref}

% The following \documentclass options may be useful:
%
% 10pt          To set in 10-point type instead of 9-point.
% 11pt          To set in 11-point type instead of 9-point.
% authoryear    To obtain author/year citation style instead of numeric.


%%====================================================================================================

% \usepackage{url}
% \usepackage{code}
% \usepackage{graphicx}
% \usepackage{enumerate}
% \usepackage{latexsym}
% \usepackage{amsmath}
% \usepackage{amssymb}
% \usepackage{amstext}

\usepackage{comment}
\usepackage{graphicx}
\usepackage{moresize}

% \usepackage{hyperref}
% \hypersetup{colorlinks,%
%             citecolor=black,%
%             filecolor=black,%
%             linkcolor=black,%
%             urlcolor=blue,%
%             pdftex}

\usepackage{url}
% \usepackage{multicol}
% \usepackage{subfigure}

%%====================================================================================================
%%  Listings package and configuration:

\usepackage{listings}
% \usepackage{listings-1.4/listings}
% \usepackage{listings-1.4/lstdoc}
% \usepackage{listings-1.4/lstlang1}
% \usepackage{listings-1.4/lstmisc}
% \input{listings-1.4/listings.sty}

\ExecuteOptions{letter}

\newenvironment{commentenv}[1]{\begin{list}{}{}\item[]{\sc [#1:}
}{{\rm {\sc End of comment.]}} \end{list}}

\long\gdef\comment#1#2{{\color{red} [\textsc{#1}: \textsf{#2}]}}
%\long\gdef\comment#1#2{}

% \newcommand{\code}[1]{\lstinline[basicstyle=\sffamily]{#1}}
\newcommand{\func}[1]{\lstinline[basicstyle=\sffamily]{#1()}}
\newcommand{\keyword}[1]{\emph{#1}}

\newcommand{\concept}[1]{{\sc #1}}

% ----------------------------------------

\lstdefinestyle{basic}{showstringspaces=false,columns=fullflexible,language=C++,escapechar=@,xleftmargin=1pc,%
%basicstyle=\small\bfseries\itshape,%
%keywordstyle=\underbar,
basicstyle=\small\sffamily,
commentstyle=\mdseries,
moredelim=**[is][\color{white}]{~}{~},
morekeywords={concept,model,require,where},
literate={->}{{$\rightarrow\;$}}1 {<-}{{$\leftarrow\;$}}1 {=>}{{$\Rightarrow\;$}}1,
}

\usepackage{color}
\definecolor{darkgreen}{rgb}{0,0.5,0}
\definecolor{darkred}{rgb}{0.5,0,0}

\lstloadlanguages{C++}
\lstnewenvironment{code}
    {\lstset{}%
      \csname lst@SetFirstLabel\endcsname}
    {\csname lst@SaveFirstLabel\endcsname}
    \lstset{
      language=C++,
%      basicstyle=\ssmall\ttfamily,
      basicstyle=\footnotesize\ttfamily,
      flexiblecolumns=true,
%      basicstyle=\ssmall\ttfamily,
      flexiblecolumns=false,
      numbers=left,
      basewidth={0.5em,0.45em},
      keywordstyle=\color{blue},
      commentstyle=\color{darkgreen},
      stringstyle=\ttfamily\color{darkred},
      escapechar=\@,
      xleftmargin=2mm,
      morekeywords={cilk_for,cilk_spawn,cilk_sync, ivar_payload_t,
      __cilkrts_ivar, ivar_payload_t,
      align, atomic,
      begin, bool, break, by,
      class, cobegin, coforall, complex, config,
      const, continue,
      delete, dmapped, do, domain,
      else, enum, extern, export,
        false, for, forall,
          if, imag, in, index,
          inline, inout, int,
          iter,
          label, lambda,
          let, local,
          locale,
          module,
          new,
          nil,
          on,
          opaque,
          otherwise,
          out,
          param,
          proc,
          range,
          real,
          record,
          reduce,
          ref,
          return,
          scan,
          select,
          serial,
          single,
          sparse,
          string,
          subdomain,
          sync,
          then,
          true,
          type,
          uint,
          union,
          use,
          var,
          when,
          where,
          while,
          yield,
          zip},
      %      literate={dotdotdot}{{$\ldots$}}3
        literate={...}{{$\dots$}}3
        %               {cilk_spawn}{\bf{\texttt cilk\_spawn}}8
        %               {cilk_spawn}{{cilkpawn}}10
        %
        % literate={+}{{$+$}}1 {/}{{$/$}}1 {*}{{$*$}}1 % {=}{{$=$}}1
        %          {>}{{$>$}}1 {<}{{$<$}}1 {\\}{{$\lambda$}}1
        %          {\\\\}{{\char`\\\char`\\}}1
        %          {->}{{$\rightarrow$}}2 {>=}{{$\geq$}}2 {<-}{{$\leftarrow$}}2
        %          {<=}{{$\leq$}}2 {=>}{{$\Rightarrow$}}2 
        %          {\ .}{{$\circ$}}2 {\ .\ }{{$\circ$}}2
        %          {>>}{{>>}}2 {>>=}{{>>=}}2 {=<<}{{=<<}}2
        %          {|}{{$\mid$}}1               
        %          {dotdotdot}{{$\ldots$}}3
    }


%\input{code.sty}

%%====================================================================================================


\makeatletter

\renewenvironment{thebibliography}[1]
{\section*{\refname
  \@mkboth{\MakeUppercase\refname}{\MakeUppercase\refname}}%
    \list{\@biblabel{\@arabic\c@enumiv}}%
    {\settowidth\labelwidth{\@biblabel{#1}}%
      \leftmargin\labelwidth
        \advance\leftmargin\labelsep
        \@openbib@code
        \usecounter{enumiv}%
        \let\p@enumiv\@empty
        \renewcommand\theenumiv{\@arabic\c@enumiv}}%
        \sloppy\clubpenalty4000\widowpenalty4000%
        \sfcode`\.\@m}
{\def\@noitemerr
  {\@latex@warning{Empty `thebibliography' environment}}%
  \endlist}

  \def\Box{{\ \vbox{\hrule\hbox{%                                            
    \vrule height1.3ex\hskip0.8ex\vrule}\hrule
  }}\par}

% \input{squeeze}



\if{0}
%%%%%%%%%%%%%%%%%%%%%%%%%%%%%%%%%%%%%%%%%%%%%%%%%%%%%%%
% A subfigure environment that can hold verbatim text %
%%%%%%%%%%%%%%%%%%%%%%%%%%%%%%%%%%%%%%%%%%%%%%%%%%%%%%%

\newbox\subfigbox % Create a box to hold the subfigure. 
\makeatletter 
\newenvironment{subfloat}% % Create the new environment. 
{\def\caption##1{\gdef\subcapsave{\relax##1}}% 
  \let\subcapsave=\@empty % Save the subcaption text. 
    \let\sf@oldlabel=\label 
    \def\label##1{\xdef\sublabsave{\noexpand\label{##1}}}% 
    \let\sublabsave\relax % Save the label key. 
    \setbox\subfigbox\hbox 
    \bgroup}% % Open the box... 
{\egroup % ... close the box and call \subfigure. 
  \let\label=\sf@oldlabel 
    \subfigure[\subcapsave]{\box\subfigbox}}% 
    \makeatother 
    \fi{}

    %%====================================================================================================
    %%   EXTRA DEFS FOR THIS PROPOSAL:


    %% ============================================================

    \newcommand{\nix}[1]{\textcolor{darkred}{#1}}
    \newcommand{\fixme}[1]{{\bf\textcolor{darkred}{#1}}}

    % % [2011.07.09] Hmm.  Having a problem with the code environment not being teletype.
    % \newenvironment{mycode}
    % {% This is the begin code
      %   \noindent
        %   \begin{code}\noindent\vspace{-1mm}
      %   \footnotesize
        % }
        % %  \begin{code}\noindent}
        % {% This is the end code
          %   \end{code} 
          % }

          % \newenvironment{inlinecode}
          % {
            %   \begin{center}
            %   \begin{minipage}{4in}
            %   \begin{mycode}
            % } 
            % {
              %   \end{mycode}
              %   \end{minipage}
              %   \end{center}
              %   \vspace{-1.5ex}
              % }


              % \newcommand{\myhref}[2]{\href{#1}{\underline{#2}}}
              \newcommand{\myhref}[2]{\href{#1}{\underbar{\smash{#2}}}}

              %% ================================================================================

              %% Proper nouns:

              \newcommand{\ccilk}{Concurrent Cilk}
              \newcommand{\eg}{\textit{e.g. }}
              \newcommand{\ie}{\textit{i.e. }}


              %% ================================================================================



\usepackage{color}

\newcommand{\tz}[1]{\pgwrapper{TZ}{#1}}

\InputIfFileExists{activateeditingmarks}{
}{
    \def\noeditingmarks{}
}

\ifx\noeditingmarks\undefined
   \definecolor{comment-red}{rgb}{0.8,0,0}
   \newcommand{\textred}[1]{\textcolor{comment-red}{#1}}
   \newcommand{\pgwrapper}[2]{\textred{#1: #2}}
   \newcommand{\note}[1]{\begin{itemize}\item{\textcolor{blue}{#1}}\end{itemize}}
   \newcommand{\new}[1]{\textcolor{blue}{#1}}
   \newcommand{\const}[1]{\textred{#1}}

   %\definecolor{mygrey}{rgb}{0.6,0.6,0.6}
   \definecolor{mygrey}{rgb}{0.7,0.7,0.7}

   %%  I don't know a way based on user or host name, so...
   %%  instead I just signal on a file, activategreybg.tex which I 
   %%  don't then checkin.
   \InputIfFileExists{activategreybg}{
       \definecolor{comment-red}{rgb}{0.5,0,0}
       \pagecolor{mygrey}
   }{} 

\else
   \newcommand{\textred}[1]{#1}
   \newcommand{\pgwrapper}[2]{}
   \newcommand{\note}[1]{}
   \newcommand{\new}[1]{#1}
%   \newcommand{\new}[1]{\textcolor{blue}{#1}}
   \newcommand{\const}[1]{#1}
\fi





\begin{document}
%\conferenceinfo{} {}
% \CopyrightYear{}
% \copyrightdata{}

%\titlebanner{Working Draft -- NON-FINAL VERSION}
% \titlebanner{preprint - Submitted to ICFP 2012}   % These are ignored unless
%\preprintfooter{OOPSLA 2013 Submission}  % 'preprint' option specified.

\title{Adding in memory MapReduce operations and IO interfaces to Chapel}
% \title{Lightweight Parallel Frameworks for Legacy Languages Should Provide Concurrency Too}
%\subtitle{So many things, so little time...}

\author{Tim Zakian \and Brad Chamberlain \and Michael Ferguson}

\maketitle

\begin{abstract}

%% 1. State the problem
%% 2. Say why it’s an interesting problem
%% 3. Say what your solution achieves
%% 4. Say what follows from your solution

As big data becomes more and more prevalent in day-to-day computing,
there is a need to be able to deal with large sets of data. However
conventional systems such as Hadoop and Hive rely heavily on a divide and conquer method distributed throughout a
cluster of computers. This method has been proven to be extremely efficient
at a number of tasks, however this method can lead to problems when dealing
with data in a contextual, iterative or interactive environment. In this
document we not only offer an implementation of a framework that allows in memory MapReduce
operations in Chapel and shows the benefits that can be achieved this way,
but also offers up an API so that it can integrate seamlessly with other
filesystems. 
\end{abstract}


% \category{D.3.2}{Concurrent, Distributed, and Parallel Languages}{}

% \terms
% Design, Languages, Performance

% \keywords
% Work-stealing, Composability, Haskell, GPU

% ================================================================================
%  Use the @ symbol for simple inline code within prose:
% \makeatactive
%\lstMakeShortInline[]@


% ================================================================================
%% Main Sections:


\section{Introduction}\label{s:intro}
Chapel is an emerging parallel language being developed by Hewlett Packard
Enterprise (formerly by Cray Inc.) with
the goal of improving programmer productivity on large-scale systems as
well as on the desktop. It has been developed with the goal being to a
large extent the  standard HPCesque battery of programs which in the large
majority of cases do not involve heavy string based processing, and up
until recently this held true with the vast majority of HPC applications.

However, with the event of big data, and especially looking at iterative
programs we see that in many cases these programs can
benefit from an in memory MapReduce as demonstrated in \cite{zaharia2010spark},
in which the files data is read in and worked on the node that hosts that data 
- retaining the data on that node until it is told that it is no longer needed - 
and doing MapReduce operations using these nodes.

We assert that Chapel could work quite well in this framework. In this
document we explore how to go about adding MapReduce operations to Chapel as 
well as how this might benefit the user not only in terms of speed, but also 
in terms of user productivity.

%We also explore another approach: a MapReduce-mapreduce type operation in which data
%aggregation is done in the first MapReduce, at the end of which all the data that is
%needed is mapped to different nodes locally in memory, at which point we can then run
%a regular program in parallel over this data. This gives performance improvements not
%only in iterative tasks, but also in tasks that require large amounts of contextual
%information (since the reduce at the end of the first MapReduce can condense and
%give us all the information that we need). Not only does this give us the opportunity to then
%explore greater degrees of parallelism as well as faster communication between nodes,
%but we can also utilize non-standard architectures (GPU, MIC) to our benefit. 
%
%\tz{think: supercomputer hooked up to a Hadoop cluster}


\section{API}\label{s:API}

There are four APIs exposed. 
\begin{itemize}
\item A Systems Level API that is the glue code between the Chapel runtime
and the filesystem. ({\tt qio\_plugin\_hdfs.c/h})
\item A user-level IO API that interfaces with the system API, as well as
provides the interface for the MapReduce API.
({\tt HDFS.chpl})
\item A way to parse from the IO interface in Chapel to records in Chapel.
({\tt RecordParser.chpl})
\item A MapReduce API that interfaces with the IO system and
RecordParser in Chapel. ({\tt HDFSiterator.chpl})
\end{itemize}
%
\subsection{Runtime API}
Throughout this section, the names that we use in the description of the function pointer are the names of those
function pointers in the {\tt qio\_file\_functions\_t} struct.
The Runtime API currently consists of Several functions:
\begin{itemize}
\item {\tt readv} Implements a function that has the same semantics as {\tt readv} in
\href{http://pubs.opengroup.org/onlinepubs/9699919799/}{POSIX.1-2008}. It takes
in a user defined filesystem struct which contains all the information that the file system needs
in order to work. The {\tt iovec} argument is already allocated.
\begin{lstlisting}
typedef qioerr (*qio_readv_fptr)
                (void*, // plugin file pointer
                const struct iovec*,   
                // Data to write into
                int,                    
                // number of elements in iovec
                ssize_t*);              
                // Amount that was written into iovec
\end{lstlisting}
\item {\tt writev} Implements the same sematics as {\tt writev} in 
\href{http://pubs.opengroup.org/onlinepubs/9699919799/}{POSIX.1-2008}. 
\begin{lstlisting}
typedef qioerr (*qio_writev_fptr) 
                (void*, // plugin fp
                const struct iovec*,      
                // data to write from
                int,                      
                // Number of elements in iovec
                ssize_t*);                
                // Amount written on return
\end{lstlisting}
\item {\tt preadv} Implements a function which does a positional read of the file 
and puts the results in the {\tt iovec} argument which is already allocated. 
\begin{lstlisting}
typedef qioerr (*qio_preadv_fptr) 
                (void*, // plugin fp
                const struct iovec*,      
                // Data to write into
                int,                      
                // number of elements in iovec
                off_t,                    
                // Offset to read from
                ssize_t*);                
                // Amount that was written into iovec
\end{lstlisting}
\item {\tt pwritev} does a positional write of the file that is referenced by the plugin
filepointer from the contents stored in the argument to {\tt iovec}. It returns the
number of bytes written on return.
\begin{lstlisting}
typedef qioerr (*qio_pwritev_fptr) 
                (void*,//plugin fp
                const struct iovec*,      
                // data to write from
                int,                      
                // Number of elements in iovec 
                off_t,                    
                // offset to write 
                ssize_t*);                
                // Amount written on return
\end{lstlisting}
\item {\tt seek} Implements the semantics of {\tt lseek} in 
\href{http://pubs.opengroup.org/onlinepubs/9699919799/}{POSIX.1-2008}.
\begin{lstlisting}
typedef qioerr (*qio_seek_fptr)
                (void*,  // plugin fp
                 off_t,  // offset to seek from
                 int,    // Amount to seek
                 off_t*);// Offset on return from seek
\end{lstlisting}
\item {\tt filelength} Returns the length of the file in bytes that is referenced by the
plugin filepointer.
\begin{lstlisting}
typedef qioerr (*qio_filelength_fptr)
                (void*,     // file information 
                 int64_t*); // length on return
\end{lstlisting}
\item {\tt getpath} Returns the path to the file that is referenced by the plugin
filepointer.
\begin{lstlisting}
typedef qioerr (*qio_getpath_fptr)
                (void*, // file information
                 const char**);// string/path on return
\end{lstlisting}
\item {\tt open} Opens the file on the configured filesystem pointed to by the path passed in as
the second argument. The plugin filepointer (which is user defined) is finished being
populated here. The user also at this point needs to also set the flags for the file
(for more information see ~\ref{s:flags} and ~\ref{s:hints}). {\tt open} expects a configured filesystem
passed in as the last argument to the function.
\begin{lstlisting}
typedef qioerr (*qio_open_fptr)
                (void**,  // the plugin fp on return
                 const char*, // pathname to file
                 int*, // flags out
                 mode_t,  // mode
                 qio_hint_t, // Hints for opening the file
                 void*);  // The configured filesystem
\end{lstlisting}
\item {\tt close} Closes the file pointed to by the plugin filepointer. Has the same
semantics as {\tt close} in
\href{http://pubs.opengroup.org/onlinepubs/9699919799/}{POSIX.1-2008}.
\begin{lstlisting}
typedef qioerr (*qio_close_fptr)
                (void*); // file fp
\end{lstlisting}
\item {\tt fsync} Provides the same functionality as {\tt fsync} in 
\href{http://pubs.opengroup.org/onlinepubs/9699919799/}{POSIX.1-2008}.
\begin{lstlisting}
typedef qioerr (*qio_fsync_fptr)
                (void*); // file information
\end{lstlisting}
\item {\tt getcwd} replicates the semantics of {\tt getcwd} in 
\href{http://pubs.opengroup.org/onlinepubs/9699919799/}{POSIX.1-2008}.
\begin{lstlisting}
typedef qioerr (*qio_getcwd_fptr)
                (void*, // file information
                 const char**); // path on return
\end{lstlisting}
\item The struct {\tt qio\_file\_functions\_t} represents all the functions needed within
QIO in order to implement the functionality needed for file IO in Chapel. This is
loaded into the QIO representation of the file at initialization and the user is
responsible for populating this struct before passing it into the QIO runtime code.
\begin{lstlisting}
typedef struct qio_file_functions_s {
  qio_writev_fptr  writev; 
  qio_readv_fptr   readv;

  qio_pwritev_fptr pwritev;
  qio_preadv_fptr  preadv;

  qio_close_fptr   close;
  qio_open_fptr    open;

  qio_seek_fptr   seek;

  qio_filelength_fptr filelength;
  qio_getpath_fptr getpath;

  qio_fsync_fptr fsync;
  qio_getcwd_fptr getcwd;

  void* fs; // Holds the configured filesystem

} qio_file_functions_t;
\end{lstlisting}
\end{itemize}
Where the {\tt void* fs} in {\tt qio\_file\_functions\_t} holds the configured file system. This way we can support
calling functions that are not dependent upon opening the file on a file system (\eg
calling {\tt getpath} or {\tt getcwd}). 

The various types of information needed by the file system in order to read and write
files is passed around as a user defined struct in a {\tt void*} (the plugin fp). This way we can support
any filesystem since we no longer have to worry about the number of
arguments to these functions. Therefore the library writer is responsible for packing
the arguments into - and extracting the arguments from - the plugin fp. 

The library writer is also responsible for writing the wrapper functions
around the calls to the filesystem so that it conforms with this API.
This way we can report appropriate errors as well as supporting as many
filesystems as possible.

The only other function needed in order to create an interface with QIO is to
implement a function that populates the {\tt qio\_file\_functions\_t} struct (the number of
arguments to such a function can be arbitrary and user defined). Hereafter we will
call this function {\tt create\_qio\_functions}. This will then be an
interface to the module level code. The way in which we pass this information through to the QIO runtime is via the function
in runtime called {\tt qio\_file\_open\_access\_usr} which is called by {\tt open} in the Chapel
module level code and is the last argument to the function
(the rest of the arguments are the same as {\tt qio\_file\_open\_access}). 

This now brings us to a discussion of the module level API to the runtime.

The module-level API for the runtime is almost trivial and depends only on
{\tt create\_qio\_functions}. The way the library writer would interface with the runtime 
at the module level would be along the lines of 
\begin{lstlisting}
extern proc create_qio_functions(...):
                 qio_file_functions_t;

proc myOpen(...): file {
  var err: syserr = ENOERR;
  ...
  var fsfns = create_qio_functions(...);
  ...
  err = qio_file_open_access(..., fsfns);
  ...
  }
\end{lstlisting}

\subsection{User API for HDFS}
\subsubsection{Types}
The types defined by the HDFS module are as follows:

\begin{lstlisting}
record hdfsChapelFile {
  var files: [rcDomain] file;
}
\end{lstlisting}
Which is a wrapper around a replicated array of files; one per locale. (\ie a
``Global file'' in a sense)

\begin{lstlisting}
record hdfsChapelFileSystem {
  var home: locale;
  var _internal_file: [rcDomain] c_ptr; 
  // contains hdfsFS
}
\end{lstlisting}
This is almost the same as {\tt hdfsChapelFile}, except this time instead of replicating
a file across each locale, it replicates the configured file system across each locale.

\begin{lstlisting}
record hdfsChapelFile_local {
  var home: locale = here;
  var _internal_:qio_locale_map_ptr_t 
      = QIO_LOCALE_MAP_PTR_T_NULL;
}
\end{lstlisting}
Represents a mapping of a localeId to a specific byte range in the file.

\begin{lstlisting}
record hdfsChapelFileSystem_local {
  var home: locale;
  var _internal_: c_ptr;
}
\end{lstlisting}
Represents a configured file system pointer. 

\subsubsection{Functions}
\begin{lstlisting}
hdfsChapelConnect(name: string, port: int): fs;
\end{lstlisting}
Connects to HDFS with name {\tt name} and port {\tt port} and replicates across all locales
on the machine. This way there is a valid way to reference this other then from the
locale it was called on.

\begin{lstlisting}
fs.hdfsChapelDisconnect();
\end{lstlisting}
Disconnects (on each locale) from the file system {\tt fs} connected to
by {\tt hdfsChapelConnect}

\begin{lstlisting}
fs.hdfsOpen(filename: string, 
                flags: iomode): hdfsChapelFile;
\end{lstlisting}
Opens a file with path {\tt pathname} and in mode {\tt flags} on each locale from the file system
that was connected to via {\tt hdfsConnect}. The only possible iomodes are
{\tt iomode.r} and {\tt iomode.cw} (due to HDFS constraints).

\begin{lstlisting}
hdfsChapelFile.hdfsClose();
\end{lstlisting}
Closes the files created by {\tt fs.hdfsOpen}.

\begin{lstlisting}
hdfsChapelFile.getLocal: file;
\end{lstlisting}
Returns the file for the current locale that you are on when this function is called.

\begin{lstlisting}
hdfs_chapel_connect(path:string, port: int): 
                  hdfsChapelFileSystem_local;
\end{lstlisting}
Same as {\tt hdfsChapelConnect} except that this only creates a valid file system on the
locale it was on when it was called.

\begin{lstlisting}
hdfsChapelFileSystem_local.hdfs_chapel_disconnect();
\end{lstlisting}
Disconnects from HDFS.

\begin{lstlisting}
getHosts(f: file);
\end{lstlisting}
Returns a C array of structs of the form 
\begin{lstlisting}
{(locale_id, start_byte, length), ...}
\end{lstlisting}
which can be accessed via 
\begin{lstlisting}
getLocaleBytes(g: hdfsChapelFile_locale, i: int);
\end{lstlisting}
The only other things added to the current IO functionality is a convenience function
\begin{lstlisting}
hdfsChapelFile.hdfsReader(...): channel;
\end{lstlisting}
Which takes in the same arguments as the standard {\tt file.reader} function.

After this, all other functionality is supported. An example of this is:
\begin{lstlisting}{chapel}
use HDFS;

var gfl: hdfsChapelFile;
var hdfs: fs;

hdfs = hdfsChapelConnect("default", 0);
gfl  = hdfs.hdfsOpen("/tmp/advo.txt", iomode.r);

for loc in Locales {
  on loc {
    var r = gfl.hdfsReader(start=50);
    // same as:
    // var r = gfl.getLocal.reader(start=50);
    var str: string;
    r.readline(str);
    writeln("on locale ", here.id, " string: " + str);
    r.close();
  }
}
on Locales[2] {
  gfl.hdfsClose();
}

on Locales[1] {
  hdfs.hdfsChapelDisconnect();
}

/* outputs:

   on locale 0 string: 0325

   on locale 1 string: 0325

   on locale 2 string: 0325

   on locale 3 string: 0325

 */
\end{lstlisting}

\subsection{Record parser API}
The API for the record parser works with the IO interface in Chapel and
\emph{is not} dependent upon using HDFS or anything else except for the
functions provided in {\tt IO.chpl} and regexp support. The API is as follows:
\begin{lstlisting}
new recordReader(type recordType, reader: channel, 
                 regex: string): recordReader;
\end{lstlisting}
Creates a record reader that parses into the record of type {\tt recordType}
from the channel {\tt reader} using the regexp {\tt regex}
\begin{lstlisting}
new recordReader(type recordType, 
                 reader: channel): recordReader;
\end{lstlisting}
The same as the first one, however this time the regex is inferred from the
field names in the record {\tt recordType}. The regex created this way is very
lax in terms of how much whitespace there is between records. This could lead to naming problems as
well as there might be problems parsing into the record. 

\begin{lstlisting}
recordReader.get(): recordType
\end{lstlisting}
Returns one record and advances the position in the file to the end of
where it read. Will return error if it cannot return one.

\begin{lstlisting}
recordReader.stream(): iter(recordType)
\end{lstlisting}
Returns a stream of records of type {\tt recordType} until the regex no longer
matches. At end, leaves the channel position at the place where it read to.  

An example of how to use this is as follows:
\begin{lstlisting}
use RecordParser;

var fl = open("test.txt", iomode.r);
var ch = fl.reader();

record Test {
  var name: string;
  var id:   int;
}

var regex = "Name: (.*)\\s*Id: (.*)\\n\\n";

var M = new RecordReader(Test, ch, regex);

writeln("get() = ", M.get());

writeln("Now testing stream()");

for m in M.stream {
  writeln(m);
}

ch.close();
fl.close();

/*  Outputs: 
    get() = (name = one, id = 1)
    Now testing stream
    (name = two, id = 2)
    (name = three, id = 3)

    For test.txt = 
    Name: one
    Id: 1

    Name: two
    Id: 2

    Name: three
    Id: 3
*/
\end{lstlisting}

\subsection{MapReduce API}
The API here is the simplest of them all and consists of only one function
\begin{lstlisting}
HDFSiter(path: string, type recordType,
         regex: string): iter(RecordType)
\end{lstlisting}
This is a leader-follower iterator that is locale aware in terms of data
locality (\eg  if blocks 0 and 1 reside on locales 0 and 1 respectively,
it will read on and work on locales 0 and 1 while using those blocks).

Many times what things might look like is:
\begin{lstlisting}
use HDFSiterator;

record someRecord {
  var id1: real;
  var id2: real;
}

var regex = "ID1(.*)\\s*ID2(.*)\\n";

forall r in HDFSiter("/tmp/test.txt", someRecord, regex) {
  <do something with r in here>
}
\end{lstlisting}

\subsection{Flags in QIO}\label{s:flags}
The flags for QIO are fairly straightforward and consist of:
\begin{itemize}
\item {\tt QIO\_FDFLAG\_READABLE} Specifies that this file has been opened in a mode that
supports reading. 
\item {\tt QIO\_FDFLAG\_WRITABLE} Specifies that this file has been opened in a mode that
supports writing
\item {\tt QIO\_FDFLAG\_SEEKABLE} Specifies that this file is seekable.
\end{itemize}

\subsection{Hints in QIO}\label{s:hints}
There are 5 types of hints:
\begin{itemize}
\item {\tt IOHINT\_NONE} Normal operation. We expect to use this most of the time.
\item {\tt QIO\_HINT\_RANDOM} We expect random access to this file.
\item {\tt QIO\_HINT\_SEQUENTAL} We expect sequential access to this file.
\item {\tt QIO\_HINT\_CACHED} We expect the entire file to be cached and/or pulled in all
at once.
\item {\tt QIO\_HINT\_PARALLEL} We expect many channels to work on this file concurrently.
\end{itemize}

\section{Examples}\label{s:examples}

The examples here show how one might go about interfacing with the various APIs
provided in section ~\ref{s:API} and while they are not supposed to be comprehensive, the
hope is that this section will provide enough guidance to get started.

\subsection{System API}
In this section we'll walk through a simple example of implementing {\tt preadv} for
HDFS.

\begin{lstlisting}
qioerr hdfs_preadv (void* file, 
                    const struct iovec *vector, 
                    int count, off_t offset, 
                    ssize_t* num_read_out)
{
  ssize_t got;
  ssize_t got_total;
  qioerr err_out = 0;
  int i;

  STARTING_SLOW_SYSCALL;

  err_out = 0;
  got_total = 0;
  for(i = 0; i < count; i++) {
    got = hdfsPread(to_hdfs_file(file)->fs,
                    to_hdfs_file(file)->file, 
                    offset + got_total, 
                    (void*)vector[i].iov_base, 
                    vector[i].iov_len);
    if( got != -1 ) {
      got_total += got;
    } else {
      err_out = qio_mkerror_errno();
      break;
    }
    if(got != (ssize_t)vector[i].iov_len ) {
      break;
    }
  }

  if( err_out == 0 && 
    got_total == 0 && 
    sys_iov_total_bytes(vector, count) != 0 ) 
    err_out = qio_int_to_err(EEOF);

  *num_read_out = got_total;

  DONE_SLOW_SYSCALL;

  return err_out;
}
\end{lstlisting}
In lines 1-5, we take in the file pointer for our filesystem as a {\tt void*}, a vector
of preallocated {\tt iovec} buffers, the number of these vectors, the offset to read from and the
output to tell the runtime how much we were able to read. {\tt qioerr} is a QIO defined
struct, and consists of an error number field and a message field (\ie an {\tt int} and a
{\tt const char*}) and is a {\tt syserr} in Chapel code.

In lines 11-30 we call {\tt STARTING\_SLOW\_SYSCALL} which at this time does nothing - and
is meant to signify that we are doing a system call that might block - however in the
future this might do something to migrate threads, and therefore errno's might go
away if you tried accessing them after {\tt DONE\_SLOW\_SYSCALL}. Inside the for loop we
simply populate our vector of buffers until we cannot read anymore, or until we
have filled up all the buffers. 

In lines 32-42 we check to see if we have read anything, if we don't have an error
and we didn't read anything and the vectors total length is not 0, then we have
encountered an unexpected EOF, and set the error to {\tt EEOF}. We then simply set the
amount we've read and return.

\subsection{Other APIs}

In this example, we walk through the implementation of the HDFS leader-follower
iterator which uses the User API along with the RecordParser API. The code for the
leader is:
\begin{lstlisting}

iter HDFSiter(param tag: iterKind, 
              path: string, type rec, regex: string)
  where tag == iterKind.leader {

    type hdfsInfo = 2*int(64);

    var workQueue: [LocaleSpace] domain(hdfsInfo);

    // ----- Create a file per locale ----
    var hdfs = hdfsChapelConnect("default", 0);
    var fl   = hdfs.hdfsOpen(path, iomode.r);

    // -------- Get locales for blocks ------
    var fll = fl.getLocal();
    var (hosts, t) = getHosts(fll);

    for j in 0..t-1 {
      var h = getLocaleBytes(hosts, j);
      writeln(h);
      var r: hdfsInfo;
      r(1) = h.start_byte;
      r(2) = h.len;
      workQueue[h.locale_id] += r;
    }

    coforall loc in Locales {
      on loc {
        forall block in workQueue[loc.id] {
          var rr = fl.hdfsReader(start=block(1));
          var N  = new RecordReader(rec, rr, regex);
          for n in N.stream_num(block(1), block(2)) {
            yield n;
          }
        }
      }
    }

    fl.hdfsClose();
    hdfs.hdfsChapelDisconnect();
  }
\end{lstlisting}

In lines 7-11 We first create an associative domain over our locales with tuples of
the form {\tt (start\_byte, length)} (in essence creating a work-queue for each locale),
we then connect to HDFS and open files on each of our locales.

In lines 14-24 we then get our local file and get a C array of structs of
the form {\tt (locale\_id, start\_byte, length)} and the number of elements in that array are returned as the
second element of the tuple. We then go through each element in the array and add it
to our work-queue for that locale.

In lines 26-35, we go on each locale and in parallel run through the work queue for that
locale, returning records in parallel.




\section{Future Work}\label{s:future}

Future work in this area has a couple different areas.

\subsection{Locality}

Right now, it is up to the programmer to create the ability to query where data is
stored as well how they wish to represent this data and make it viewable in Chapel.
However in the future it would be nice to have a locality API much like the current 
file IO API and therefore make it easier for the library writer to interface with Chapel 
level programs in such a way that they can leverage as much benefit from a different 
filesystem as possible. 

Also, it would be nice to have a way to deal only with data on a
certain locale when dealing with regular filesystems (such as LUSTRE, or other RAID 0
type filesystems) the goal being so you could do something along the lines of 
\begin{lstlisting}
// Connect to, and open a filesystem here
...
forall stripe in file.stripes by 3 {
  // Do something with every third stripe
}
\end{lstlisting}
Which gives the ability to easily reason about and use concurrency in RAID
0 type filesystems.

\subsection{Replication of Files}
Right now due to the way HDFS is structured, when {\tt hdfsOpen} is called, it creates a
file per locale. This is not ideal, but at least at this point the only way to be
able to handle all cases in HDFS since we do not know beforehand where the various
blocks of the file will reside.

On other filesystems, this could be different, and therefore the possibility to
minimize the number of files created becomes a possiblity. As well as this also might
allow us to represent this information in a more compact and meaningful way. 

Along with these options is the option to make this file creation and replication
lazy in the sense that files will only be created on a locale iff a file is needed
on that locale. Otherwise, that locale will not get a file. This coupled with
caching the file once we've opened it on a locale would also offer a speed
improvement.


\section{Integrating with Other File Systems}\label{s:OFS}

To integrate with other filesystems, the user must modify a couple areas. For
example if we wanted to add Ceph \cite{ceph}, we would need to edit and change the following:
\begin{itemize}
\item Create a plugin {\tt qio\_plugin\_ceph.c} that is placed under {\tt runtime/src/qio/foreignFS/ceph}
\item Add the header file {\tt qio\_plugin\_ceph.h} to {\tt runtime/include/qio}
\item Add a {\tt Makefile.foreignFS-ceph} to {\tt runtime/etc} that will link in what you need at the compile time of the program (this will normally include include and libraries).
\item You will need to add and replicate (renaming appropriately)  the various
makefiles in {\tt runtime/src/qio/foreignFS/hdfs}
\item After this you should be able to make the {\tt ceph} plugin by setting {\tt
CHPL\_FOREIGN\_FS=ceph} and remaking the runtime.
\end{itemize}




%\input{progmodel}
%% \input{semantics}
%\input{implementation}
%\input{evaluation}
%%\input{related}
%\input{conclusion}

\newpage
% \appendix

\section{C+MPI Version of Random Access Update Loop}\label{randmpi}
\begin{C}
static void
Power2NodesMPIRandomAccessUpdate(u64Int logTableSize,
                                 u64Int TableSize,
                                 u64Int LocalTableSize,
                                 u64Int MinLocalTableSize,
                                 u64Int GlobalStartMyProc,
                                 u64Int Top,
                                 int logNumProcs,
                                 int NumProcs,
                                 int Remainder,
                                 int MyProc,
                                 s64Int ProcNumUpdates,
                                 MPI_Datatype INT64_DT)
{
  s64Int i, j;
  int proc_count;

  s64Int SendCnt;
  u64Int Ran;
  s64Int WhichPe;
  u64Int LocalOffset;
  int logLocalTableSize = logTableSize - logNumProcs;
  int NumberReceiving = NumProcs - 1;
  MPI_Request inreq, outreq = MPI_REQUEST_NULL;
  u64Int inmsg;
  int bufferBase;

  MPI_Status finish_statuses[NumProcs];
  MPI_Request finish_req[NumProcs];

  MPI_Status status;
  MPI_Status ignoredStatus; /* Cray X1 doesn't have MPI_STATUS_IGNORE */
  int have_done;

  int pe;
  int pendingUpdates;
  int maxPendingUpdates;
  int localBufferSize;
  int peUpdates;
  int recvUpdates;
  Bucket_Ptr Buckets;

  pendingUpdates = 0;
  maxPendingUpdates = MAX_TOTAL_PENDING_UPDATES;
  localBufferSize = LOCAL_BUFFER_SIZE;
  Buckets = HPCC_InitBuckets(NumProcs, maxPendingUpdates);

  /* Initialize main table */
  for (i=0; i<LocalTableSize; i++)
    HPCC_Table[i] = i + GlobalStartMyProc;

  /* Perform updates to main table.  The scalar equivalent is:
   *
   *     u64Int Ran;
   *     Ran = 1;
   *     for (i=0; i<NUPDATE; i++) {
   *       Ran = (Ran << 1) ^ (((s64Int) Ran < 0) ? POLY : 0);
   *       Table[Ran & (TABSIZE-1)] ^= Ran;
   *     }
   */

  SendCnt = ProcNumUpdates; /*  SendCnt = (4 * LocalTableSize); */
  Ran = HPCC_starts (4 * GlobalStartMyProc);

  pendingUpdates = 0;
  i = 0;

  MPI_Irecv(&LocalRecvBuffer, localBufferSize, INT64_DT,
            MPI_ANY_SOURCE, MPI_ANY_TAG, MPI_COMM_WORLD, &inreq);

  while (i < SendCnt) {

    /* receive messages */
    do {
      MPI_Test(&inreq, &have_done, &status);
      if (have_done) {
        if (status.MPI_TAG == UPDATE_TAG) {
          MPI_Get_count(&status, INT64_DT, &recvUpdates);
          bufferBase = 0;
          for (j=0; j < recvUpdates; j ++) {
            inmsg = LocalRecvBuffer[bufferBase+j];
            HPCC_Table[inmsg & (LocalTableSize-1)] ^= inmsg;
          }

        } else if (status.MPI_TAG == FINISHED_TAG) {
          /* we got a done message.  Thanks for playing... */
          NumberReceiving--;
        } else {
          abort();
        }
        MPI_Irecv(&LocalRecvBuffer, localBufferSize, INT64_DT,
                  MPI_ANY_SOURCE, MPI_ANY_TAG, MPI_COMM_WORLD, &inreq);
      }
    } while (have_done && NumberReceiving > 0);


    if (pendingUpdates < maxPendingUpdates) {
      Ran = (Ran << 1) ^ ((s64Int) Ran < ZERO64B ? POLY : ZERO64B);
      WhichPe = (Ran >> logLocalTableSize) & (NumProcs - 1);
      if (WhichPe == MyProc) {
        LocalOffset = (Ran & (TableSize - 1)) - GlobalStartMyProc;
        HPCC_Table[LocalOffset] ^= Ran;
      }
      else {
        HPCC_InsertUpdate(Ran, WhichPe, Buckets);
        pendingUpdates++;
      }
      i++;
    }

    else {
      MPI_Test(&outreq, &have_done, MPI_STATUS_IGNORE);
      if (have_done) {
        outreq = MPI_REQUEST_NULL;
        pe = HPCC_GetUpdates(Buckets, LocalSendBuffer, localBufferSize, &peUpdates);
        MPI_Isend(&LocalSendBuffer, peUpdates, INT64_DT, (int)pe, UPDATE_TAG,
                  MPI_COMM_WORLD, &outreq);
        pendingUpdates -= peUpdates;
      }
    }

  }


  /* send remaining updates in buckets */
  while (pendingUpdates > 0) {

    /* receive messages */
    do {
      MPI_Test(&inreq, &have_done, &status);
      if (have_done) {
        if (status.MPI_TAG == UPDATE_TAG) {
          MPI_Get_count(&status, INT64_DT, &recvUpdates);
          bufferBase = 0;
          for (j=0; j < recvUpdates; j ++) {
            inmsg = LocalRecvBuffer[bufferBase+j];
            HPCC_Table[inmsg & (LocalTableSize-1)] ^= inmsg;
          }
        } else if (status.MPI_TAG == FINISHED_TAG) {
          /* we got a done message.  Thanks for playing... */
          NumberReceiving--;
        } else {
          abort();
        }
        MPI_Irecv(&LocalRecvBuffer, localBufferSize, INT64_DT,
                  MPI_ANY_SOURCE, MPI_ANY_TAG, MPI_COMM_WORLD, &inreq);
      }
    } while (have_done && NumberReceiving > 0);


    MPI_Test(&outreq, &have_done, MPI_STATUS_IGNORE);
    if (have_done) {
      outreq = MPI_REQUEST_NULL;
      pe = HPCC_GetUpdates(Buckets, LocalSendBuffer, localBufferSize, &peUpdates);
      MPI_Isend(&LocalSendBuffer, peUpdates, INT64_DT, (int)pe, UPDATE_TAG,
                MPI_COMM_WORLD, &outreq);
      pendingUpdates -= peUpdates;
    }

  }

  /* send our done messages */
  for (proc_count = 0 ; proc_count < NumProcs ; ++proc_count) {
    if (proc_count == MyProc) { finish_req[MyProc] = MPI_REQUEST_NULL; continue; }
    /* send garbage - who cares, no one will look at it */
    MPI_Isend(&Ran, 1, INT64_DT, proc_count, FINISHED_TAG,
              MPI_COMM_WORLD, finish_req + proc_count);
  }

  /* Finish everyone else up... */
  while (NumberReceiving > 0) {
    MPI_Wait(&inreq, &status);
    if (status.MPI_TAG == UPDATE_TAG) {
      MPI_Get_count(&status, INT64_DT, &recvUpdates);
      bufferBase = 0;
      for (j=0; j < recvUpdates; j ++) {
        inmsg = LocalRecvBuffer[bufferBase+j];
        HPCC_Table[inmsg & (LocalTableSize-1)] ^= inmsg;
      }

    } else if (status.MPI_TAG == FINISHED_TAG) {
      /* we got a done message.  Thanks for playing... */
      NumberReceiving--;
    } else {
      abort();
    }
    MPI_Irecv(&LocalRecvBuffer, localBufferSize, INT64_DT,
              MPI_ANY_SOURCE, MPI_ANY_TAG, MPI_COMM_WORLD, &inreq);
  }

  MPI_Waitall( NumProcs, finish_req, finish_statuses);

  /* Be nice and clean up after ourselves */
  HPCC_FreeBuckets(Buckets, NumProcs);
  MPI_Cancel(&inreq);
  MPI_Wait(&inreq, &ignoredStatus);

  /* end multiprocessor code */
}
\end{C}

{ \small
\bibliographystyle{abbrv}
%\bibliographystyle{abbrvnat}
\bibliography{refs}
}
%% The bibliography should be embedded for final submission.
%
%\begin{thebibliography}{}
%\softraggedright

%\bibitem[Smith et~al.(2009)Smith, Jones]{smith02}
%P. Q. Smith, and X. Y. Jones. ...reference text...

%\end{thebibliography}

% \appendix

\section{C+MPI Version of Random Access Update Loop}\label{randmpi}
\begin{C}
static void
Power2NodesMPIRandomAccessUpdate(u64Int logTableSize,
                                 u64Int TableSize,
                                 u64Int LocalTableSize,
                                 u64Int MinLocalTableSize,
                                 u64Int GlobalStartMyProc,
                                 u64Int Top,
                                 int logNumProcs,
                                 int NumProcs,
                                 int Remainder,
                                 int MyProc,
                                 s64Int ProcNumUpdates,
                                 MPI_Datatype INT64_DT)
{
  s64Int i, j;
  int proc_count;

  s64Int SendCnt;
  u64Int Ran;
  s64Int WhichPe;
  u64Int LocalOffset;
  int logLocalTableSize = logTableSize - logNumProcs;
  int NumberReceiving = NumProcs - 1;
  MPI_Request inreq, outreq = MPI_REQUEST_NULL;
  u64Int inmsg;
  int bufferBase;

  MPI_Status finish_statuses[NumProcs];
  MPI_Request finish_req[NumProcs];

  MPI_Status status;
  MPI_Status ignoredStatus; /* Cray X1 doesn't have MPI_STATUS_IGNORE */
  int have_done;

  int pe;
  int pendingUpdates;
  int maxPendingUpdates;
  int localBufferSize;
  int peUpdates;
  int recvUpdates;
  Bucket_Ptr Buckets;

  pendingUpdates = 0;
  maxPendingUpdates = MAX_TOTAL_PENDING_UPDATES;
  localBufferSize = LOCAL_BUFFER_SIZE;
  Buckets = HPCC_InitBuckets(NumProcs, maxPendingUpdates);

  /* Initialize main table */
  for (i=0; i<LocalTableSize; i++)
    HPCC_Table[i] = i + GlobalStartMyProc;

  /* Perform updates to main table.  The scalar equivalent is:
   *
   *     u64Int Ran;
   *     Ran = 1;
   *     for (i=0; i<NUPDATE; i++) {
   *       Ran = (Ran << 1) ^ (((s64Int) Ran < 0) ? POLY : 0);
   *       Table[Ran & (TABSIZE-1)] ^= Ran;
   *     }
   */

  SendCnt = ProcNumUpdates; /*  SendCnt = (4 * LocalTableSize); */
  Ran = HPCC_starts (4 * GlobalStartMyProc);

  pendingUpdates = 0;
  i = 0;

  MPI_Irecv(&LocalRecvBuffer, localBufferSize, INT64_DT,
            MPI_ANY_SOURCE, MPI_ANY_TAG, MPI_COMM_WORLD, &inreq);

  while (i < SendCnt) {

    /* receive messages */
    do {
      MPI_Test(&inreq, &have_done, &status);
      if (have_done) {
        if (status.MPI_TAG == UPDATE_TAG) {
          MPI_Get_count(&status, INT64_DT, &recvUpdates);
          bufferBase = 0;
          for (j=0; j < recvUpdates; j ++) {
            inmsg = LocalRecvBuffer[bufferBase+j];
            HPCC_Table[inmsg & (LocalTableSize-1)] ^= inmsg;
          }

        } else if (status.MPI_TAG == FINISHED_TAG) {
          /* we got a done message.  Thanks for playing... */
          NumberReceiving--;
        } else {
          abort();
        }
        MPI_Irecv(&LocalRecvBuffer, localBufferSize, INT64_DT,
                  MPI_ANY_SOURCE, MPI_ANY_TAG, MPI_COMM_WORLD, &inreq);
      }
    } while (have_done && NumberReceiving > 0);


    if (pendingUpdates < maxPendingUpdates) {
      Ran = (Ran << 1) ^ ((s64Int) Ran < ZERO64B ? POLY : ZERO64B);
      WhichPe = (Ran >> logLocalTableSize) & (NumProcs - 1);
      if (WhichPe == MyProc) {
        LocalOffset = (Ran & (TableSize - 1)) - GlobalStartMyProc;
        HPCC_Table[LocalOffset] ^= Ran;
      }
      else {
        HPCC_InsertUpdate(Ran, WhichPe, Buckets);
        pendingUpdates++;
      }
      i++;
    }

    else {
      MPI_Test(&outreq, &have_done, MPI_STATUS_IGNORE);
      if (have_done) {
        outreq = MPI_REQUEST_NULL;
        pe = HPCC_GetUpdates(Buckets, LocalSendBuffer, localBufferSize, &peUpdates);
        MPI_Isend(&LocalSendBuffer, peUpdates, INT64_DT, (int)pe, UPDATE_TAG,
                  MPI_COMM_WORLD, &outreq);
        pendingUpdates -= peUpdates;
      }
    }

  }


  /* send remaining updates in buckets */
  while (pendingUpdates > 0) {

    /* receive messages */
    do {
      MPI_Test(&inreq, &have_done, &status);
      if (have_done) {
        if (status.MPI_TAG == UPDATE_TAG) {
          MPI_Get_count(&status, INT64_DT, &recvUpdates);
          bufferBase = 0;
          for (j=0; j < recvUpdates; j ++) {
            inmsg = LocalRecvBuffer[bufferBase+j];
            HPCC_Table[inmsg & (LocalTableSize-1)] ^= inmsg;
          }
        } else if (status.MPI_TAG == FINISHED_TAG) {
          /* we got a done message.  Thanks for playing... */
          NumberReceiving--;
        } else {
          abort();
        }
        MPI_Irecv(&LocalRecvBuffer, localBufferSize, INT64_DT,
                  MPI_ANY_SOURCE, MPI_ANY_TAG, MPI_COMM_WORLD, &inreq);
      }
    } while (have_done && NumberReceiving > 0);


    MPI_Test(&outreq, &have_done, MPI_STATUS_IGNORE);
    if (have_done) {
      outreq = MPI_REQUEST_NULL;
      pe = HPCC_GetUpdates(Buckets, LocalSendBuffer, localBufferSize, &peUpdates);
      MPI_Isend(&LocalSendBuffer, peUpdates, INT64_DT, (int)pe, UPDATE_TAG,
                MPI_COMM_WORLD, &outreq);
      pendingUpdates -= peUpdates;
    }

  }

  /* send our done messages */
  for (proc_count = 0 ; proc_count < NumProcs ; ++proc_count) {
    if (proc_count == MyProc) { finish_req[MyProc] = MPI_REQUEST_NULL; continue; }
    /* send garbage - who cares, no one will look at it */
    MPI_Isend(&Ran, 1, INT64_DT, proc_count, FINISHED_TAG,
              MPI_COMM_WORLD, finish_req + proc_count);
  }

  /* Finish everyone else up... */
  while (NumberReceiving > 0) {
    MPI_Wait(&inreq, &status);
    if (status.MPI_TAG == UPDATE_TAG) {
      MPI_Get_count(&status, INT64_DT, &recvUpdates);
      bufferBase = 0;
      for (j=0; j < recvUpdates; j ++) {
        inmsg = LocalRecvBuffer[bufferBase+j];
        HPCC_Table[inmsg & (LocalTableSize-1)] ^= inmsg;
      }

    } else if (status.MPI_TAG == FINISHED_TAG) {
      /* we got a done message.  Thanks for playing... */
      NumberReceiving--;
    } else {
      abort();
    }
    MPI_Irecv(&LocalRecvBuffer, localBufferSize, INT64_DT,
              MPI_ANY_SOURCE, MPI_ANY_TAG, MPI_COMM_WORLD, &inreq);
  }

  MPI_Waitall( NumProcs, finish_req, finish_statuses);

  /* Be nice and clean up after ourselves */
  HPCC_FreeBuckets(Buckets, NumProcs);
  MPI_Cancel(&inreq);
  MPI_Wait(&inreq, &ignoredStatus);

  /* end multiprocessor code */
}
\end{C}


\end{document}
