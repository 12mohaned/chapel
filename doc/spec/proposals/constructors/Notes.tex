\section{Usage Notes}

This appendix contains notes on usage, related to certain design choices.

\subsection{Running Code Prior to Construction}

In some cases, when creating a new object it is desirable to run code prior to
construction.  No specific provision has been made in the specification for constructors
to support this.  In particular, the initializer list supports the initialization of each
field individually (in an arbitrary order), but does not support initializing many of them
as the result of one function.  How can this case be handled?

In Chapel, this is especially easy, given its support for generic functions and
overloading.  The solution is to wrap object creation within a factory method.  The one
restriction that must currently be obeyed is that the result of such a factory method must
be described by a single type.  If the result type is not generic (i.e. it is a single
static type) it may still be polymorphic (i.e. the returned object may have one of several
runtime types).  

For example:
\begin{chapel}
class B { }
class D1 : B { }
class D2 : B { }

proc D_factory(i:int) : B {
  select i {
    when 1 do return new D1();
    when 2 do return new D2();
    otherwise compilerError("Bad D_factory argument.");
  }
}
\end{chapel}
\noindent
This example is as simple as possible.  But can easily be imagined that \chpl{D_factory}
accepts multiple arguments and performs nontrivial computations, passing the appropriate
arguments to the constructor corresponding to each case.  Common code can be placed above
the select statement, and case-specific code within the statement controlled by each
\chpl{when}.

The only potential down-side of this approach as compared to providing more general
semantics within the constructor itself is that --- at least on the surface --- the
initializers for some fields must be built up outside the constructor and then passed in
as arguments.  That in turn may appear to necessitate a copy.  

Many of these apparent drawbacks may be dismissed due to the fact that the same copy would
be required even if the initializer value were generated within the constructor itself.
The remaining cases may be answered with the eventual inclusion in the compiler
implementation of copy-avoidance.  Specifically, if the coder avoids giving a name to the
initializer expression passed to a constructor, the compiler may complete the
initialization of the corresponding field with nothing more expensive than a pointer-copy
operation.

\vspace{6pt}

The result type may also be generic.  That observation gives the user considerably more
flexibility.  It means that the types of the results of specific cases within the factory
method must have the same name.  Otherwise, they need have little in common.  A wrapper
type containing a type field can handle the most general case.

\begin{chapel}
record Q { }
record R { }
record S { type T; }

proc S_factory(i:int) : S {
  select i {
    when 1 do return new S(T=Q);
    when 2 do return new S(T=R);
    otherwise compilerError("Bad S_factory argument.");
  }
}
\end{chapel}
\noindent
Here again, the example is as simple as possible.  However, it is easy to imagine
considerable computation preceding the call to new, and those arguments being fed to the
constructor for each type being constructed.

\subsection{Escape of Partially-Constructed Objects}

There was some discussion around having a second body in the definition of the constructor
--- the first body taking the object from a field-initialized state to a
fully-constructed state, the second containing actions that are permissible on the
fully-constructed object (e.g. calling methods within the same type).

I believe the discussion lapsed without coming to a conclusion.  However, it may now be
argued that the person writing a constructor should know the point at which an object of
that type is fully-initialized.  At the same time, without external knowledge the compiler has no way to determine
whether the division between the two bodies is correct.  Therefore, there is no need set this point off specially
in the syntax.  

Carrying this argument further, the author of a constructor should know fully the
interface and effects of any function he calls during construction (more so than the
application programmer).  The problem of preventing the early escape of a
partially-constructed object is pushed back on the author.  The class author himself must
be the best judge of when it is safe to publish an object of that type.  

One can see that the same problem arises whether a constructor has one body or two.  In
the latter case, the author must be aware of the danger that calling a certain routine
poses if he chooses to place a call to it in the second body.  That establishes the
premise that he knows well enough how to make the calls in the right order so as to
prevent early escape.  

The availability of a second body may help the author consider the order in which things
must be done, but it cannot force him to do this correctly.  For that matter, there is
nothing special about the number 2 in this context.  We may as well support any number of bodies,
including zero.
