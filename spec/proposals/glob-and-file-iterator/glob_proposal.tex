\documentclass{article}
\title{File and Glob iterator proposal}

%%====================================================================================================

% \usepackage{url}
% \usepackage{code}
% \usepackage{graphicx}
% \usepackage{enumerate}
% \usepackage{latexsym}
% \usepackage{amsmath}
% \usepackage{amssymb}
% \usepackage{amstext}

\usepackage{comment}
\usepackage{graphicx}
\usepackage{moresize}

% \usepackage{hyperref}
% \hypersetup{colorlinks,%
%             citecolor=black,%
%             filecolor=black,%
%             linkcolor=black,%
%             urlcolor=blue,%
%             pdftex}

\usepackage{url}
% \usepackage{multicol}
% \usepackage{subfigure}

%%====================================================================================================
%%  Listings package and configuration:

\usepackage{listings}
% \usepackage{listings-1.4/listings}
% \usepackage{listings-1.4/lstdoc}
% \usepackage{listings-1.4/lstlang1}
% \usepackage{listings-1.4/lstmisc}
% \input{listings-1.4/listings.sty}

\ExecuteOptions{letter}

\newenvironment{commentenv}[1]{\begin{list}{}{}\item[]{\sc [#1:}
}{{\rm {\sc End of comment.]}} \end{list}}

\long\gdef\comment#1#2{{\color{red} [\textsc{#1}: \textsf{#2}]}}
%\long\gdef\comment#1#2{}

% \newcommand{\code}[1]{\lstinline[basicstyle=\sffamily]{#1}}
\newcommand{\func}[1]{\lstinline[basicstyle=\sffamily]{#1()}}
\newcommand{\keyword}[1]{\emph{#1}}

\newcommand{\concept}[1]{{\sc #1}}

% ----------------------------------------

\lstdefinestyle{basic}{showstringspaces=false,columns=fullflexible,language=C++,escapechar=@,xleftmargin=1pc,%
%basicstyle=\small\bfseries\itshape,%
%keywordstyle=\underbar,
basicstyle=\small\sffamily,
commentstyle=\mdseries,
moredelim=**[is][\color{white}]{~}{~},
morekeywords={concept,model,require,where},
literate={->}{{$\rightarrow\;$}}1 {<-}{{$\leftarrow\;$}}1 {=>}{{$\Rightarrow\;$}}1,
}

\usepackage{color}
\definecolor{darkgreen}{rgb}{0,0.5,0}
\definecolor{darkred}{rgb}{0.5,0,0}

\lstloadlanguages{C++}
\lstnewenvironment{code}
    {\lstset{}%
      \csname lst@SetFirstLabel\endcsname}
    {\csname lst@SaveFirstLabel\endcsname}
    \lstset{
      language=C++,
%      basicstyle=\ssmall\ttfamily,
      basicstyle=\footnotesize\ttfamily,
      flexiblecolumns=true,
%      basicstyle=\ssmall\ttfamily,
      flexiblecolumns=false,
      numbers=left,
      basewidth={0.5em,0.45em},
      keywordstyle=\color{blue},
      commentstyle=\color{darkgreen},
      stringstyle=\ttfamily\color{darkred},
      escapechar=\@,
      xleftmargin=2mm,
      morekeywords={cilk_for,cilk_spawn,cilk_sync, ivar_payload_t,
      __cilkrts_ivar, ivar_payload_t,
      align, atomic,
      begin, bool, break, by,
      class, cobegin, coforall, complex, config,
      const, continue,
      delete, dmapped, do, domain,
      else, enum, extern, export,
        false, for, forall,
          if, imag, in, index,
          inline, inout, int,
          iter,
          label, lambda,
          let, local,
          locale,
          module,
          new,
          nil,
          on,
          opaque,
          otherwise,
          out,
          param,
          proc,
          range,
          real,
          record,
          reduce,
          ref,
          return,
          scan,
          select,
          serial,
          single,
          sparse,
          string,
          subdomain,
          sync,
          then,
          true,
          type,
          uint,
          union,
          use,
          var,
          when,
          where,
          while,
          yield,
          zip},
      %      literate={dotdotdot}{{$\ldots$}}3
        literate={...}{{$\dots$}}3
        %               {cilk_spawn}{\bf{\texttt cilk\_spawn}}8
        %               {cilk_spawn}{{cilkpawn}}10
        %
        % literate={+}{{$+$}}1 {/}{{$/$}}1 {*}{{$*$}}1 % {=}{{$=$}}1
        %          {>}{{$>$}}1 {<}{{$<$}}1 {\\}{{$\lambda$}}1
        %          {\\\\}{{\char`\\\char`\\}}1
        %          {->}{{$\rightarrow$}}2 {>=}{{$\geq$}}2 {<-}{{$\leftarrow$}}2
        %          {<=}{{$\leq$}}2 {=>}{{$\Rightarrow$}}2 
        %          {\ .}{{$\circ$}}2 {\ .\ }{{$\circ$}}2
        %          {>>}{{>>}}2 {>>=}{{>>=}}2 {=<<}{{=<<}}2
        %          {|}{{$\mid$}}1               
        %          {dotdotdot}{{$\ldots$}}3
    }


%\input{code.sty}

%%====================================================================================================


\makeatletter

\renewenvironment{thebibliography}[1]
{\section*{\refname
  \@mkboth{\MakeUppercase\refname}{\MakeUppercase\refname}}%
    \list{\@biblabel{\@arabic\c@enumiv}}%
    {\settowidth\labelwidth{\@biblabel{#1}}%
      \leftmargin\labelwidth
        \advance\leftmargin\labelsep
        \@openbib@code
        \usecounter{enumiv}%
        \let\p@enumiv\@empty
        \renewcommand\theenumiv{\@arabic\c@enumiv}}%
        \sloppy\clubpenalty4000\widowpenalty4000%
        \sfcode`\.\@m}
{\def\@noitemerr
  {\@latex@warning{Empty `thebibliography' environment}}%
  \endlist}

  \def\Box{{\ \vbox{\hrule\hbox{%                                            
    \vrule height1.3ex\hskip0.8ex\vrule}\hrule
  }}\par}

% \input{squeeze}



\if{0}
%%%%%%%%%%%%%%%%%%%%%%%%%%%%%%%%%%%%%%%%%%%%%%%%%%%%%%%
% A subfigure environment that can hold verbatim text %
%%%%%%%%%%%%%%%%%%%%%%%%%%%%%%%%%%%%%%%%%%%%%%%%%%%%%%%

\newbox\subfigbox % Create a box to hold the subfigure. 
\makeatletter 
\newenvironment{subfloat}% % Create the new environment. 
{\def\caption##1{\gdef\subcapsave{\relax##1}}% 
  \let\subcapsave=\@empty % Save the subcaption text. 
    \let\sf@oldlabel=\label 
    \def\label##1{\xdef\sublabsave{\noexpand\label{##1}}}% 
    \let\sublabsave\relax % Save the label key. 
    \setbox\subfigbox\hbox 
    \bgroup}% % Open the box... 
{\egroup % ... close the box and call \subfigure. 
  \let\label=\sf@oldlabel 
    \subfigure[\subcapsave]{\box\subfigbox}}% 
    \makeatother 
    \fi{}

    %%====================================================================================================
    %%   EXTRA DEFS FOR THIS PROPOSAL:


    %% ============================================================

    \newcommand{\nix}[1]{\textcolor{darkred}{#1}}
    \newcommand{\fixme}[1]{{\bf\textcolor{darkred}{#1}}}

    % % [2011.07.09] Hmm.  Having a problem with the code environment not being teletype.
    % \newenvironment{mycode}
    % {% This is the begin code
      %   \noindent
        %   \begin{code}\noindent\vspace{-1mm}
      %   \footnotesize
        % }
        % %  \begin{code}\noindent}
        % {% This is the end code
          %   \end{code} 
          % }

          % \newenvironment{inlinecode}
          % {
            %   \begin{center}
            %   \begin{minipage}{4in}
            %   \begin{mycode}
            % } 
            % {
              %   \end{mycode}
              %   \end{minipage}
              %   \end{center}
              %   \vspace{-1.5ex}
              % }


              % \newcommand{\myhref}[2]{\href{#1}{\underline{#2}}}
              \newcommand{\myhref}[2]{\href{#1}{\underbar{\smash{#2}}}}

              %% ================================================================================

              %% Proper nouns:

              \newcommand{\ccilk}{Concurrent Cilk}
              \newcommand{\eg}{\textit{e.g. }}
              \newcommand{\ie}{\textit{i.e. }}


              %% ================================================================================

\begin{document}
\maketitle
\lstMakeShortInline[]|

The goal of this proposal is to show the interface that we are thinking of implementing
(we being Brad and Tim). The intention of this iterator is to yield the directories
and or files reachable from a given directory and is meant to model in some way,
the semantics of |glob|, |wordexp|, and |listdir| (and functions along these lines).

%args
%----
%startdir
%pattern
%? types (can we avoid?)
%? skipdirs (intention: use to skip .svn directories, e.g. -- or could more
             %specifically have a skipSCMDirs concept)

%flags
%-----
%sorted=? (C glob defaults to true;
           %cl glob defaults to false;
                      %python: not sure)
%recursive=false
%expand=false
%dirs=false
%files=true
%dotfiles=false
%? symbolic links=false
%? chunked=false (intention would be to return an assoc array mapping dir
                  %names to file lists)
%? warnings/verbose (intention would be to optionally print out problems)
%??? append slash to dir names=true (would one ever want to do this?)
\section{User facing interface}
The interface to the user would be the following:
\begin{lstlisting}
iter glob(pattern   : string = "",
          startdir  : string = "",
          recursive : int    = 0,
          files     : bool   = true,
          sorted    : bool   = false,
          dirs      : bool   = false,
          expand    : bool   = false,
          dotfiles  : bool   = false) : string
\end{lstlisting}
where
\begin{itemize}
\item |pattern| is a valid |glob| or |wordexp| glob pattern. More information on what
exactly a valid glob pattern (or wordexp pattern) is, can be found on the |man(3)|
page for |glob| and |wordexp|.
\item |sorted| tells us whether we should return a sorted output. These would be
sorted at the directory level, and using the \emph{full path} (starting from
|startdir|) of the string yielded from te iterator.
\item |dirs| Says whether we should yield directories or not.
\item |files| Says whether we should yield files or not.
\item |expand| Says whether we should expand environment variables, or run shell
commands in |pattern| (a la |wordexp|). For more information on this, see the
|man(3)| page for |wordexp|.
\item |startdir| directory to start searching from.
%\item |recursive| tells us whether we should recurse into subdirectories or not.
\item |recursive| tells us whether we should be recursive or not. If |recursive| is
set to $0$ we do not recurse into subdirectories, if |recursive| is set to $-1$ we
recurse into all subdirectories. Otherwise, we recurse to a depth of |recursive| num.
Whether or not we should do this or use a |bool| instead is an open
issue. (see ~\ref{s:open_issues})
\item |dotfiles| Should we yield dotfiles or not? By default, we do not yield
dotfiles (or recurse into dot directories).
\end{itemize}

We would (of course) also expose a parallel version of this iterator.

Other, additional flags have been discussed for the possible future. For a discussion
on what these are see ~\ref{s:open_issues}.

The general thinking that we had when designing this interface was the following:
\emph{I want to...}
\begin{itemize}
\item ``iterate over dirs'' $\to$ |glob(dirs=true, rec=<num>)|
\item ``give me everything'' $\to$ |glob(rec=<num>)|
\item ``give me wordexp'' $\to$ |glob(pattern=..., expand=true)|
\item ``give me C glob'' $\to$ |glob(pattern=..., sorted=true)|
\item ``give me Python glob'' $\to$ |glob(pattern=...)|
\item ``give me command-line glob''  $\to$ |glob(pattern=..., sorted=false)|
\end{itemize}

\section{Open Issues}\label{s:open_issues}
The following are flags that we have discussed possibly putting in, however we are
seeking opinions as to whether or not we should add these in.

\begin{itemize}
\item Is there a better name for this iterator other than |glob|. (We're open to
suggestions).
\item Support sorting for parallel invocations of the iterator. Since, in this case
we would either have to synchronize at the directory level, or sort everything all at
once.
\item A flag to drop the final slash when yielding directories.
\item A warning or verbose flag. The idea behind this flag would be to warn you about
certain underlying problems. For instance, if we were using |glob| with |expand = true|
and  |pattern = "$CHPL_HOME/*"| and we had not set |$CHPL_HOME|, we would generate a
warning that we were using an undefined environment variable in expansion.
%\item A |chunked| flag. The idea behind this being that we would return ``directory''
%chunks out to the user as domains or arrays. For an example of what we are
%envisioning this looking like see ~\ref{s:examples}.
\item Add a |skipdirs| flag. The idea behind this being that if we wanted
to avoid certain directory or file names (\eg |.svn| or |.git|), we would set this
flag to the string we wanted to skip.
\item Add a |symlinks| flag to say whether or not we should follow symbolic
links. The main problems that we see with are (1) it add another flag, and (2) we
could get ourselves into a possible infinite loop if a symlink points to a directory
that contains the directory we are currently in.
\item Should, |recursive| be a boolean. In this case we have two possible proposals:
\begin{itemize}
\item We simply support recursive or non-recursive searching. The depth to which we
should recur is not specifiable.
\item We add a |depth| flag that specifies the depth that we should recur to.
(defaults to $0$)
\end{itemize}
\end{itemize}

\section{Examples}\label{s:examples}
\subsection{Example 1}
\begin{lstlisting}
// Yield all subdirectories of the current directory
// Won't recurse (so we only get the children of of the current dir)
// think: ls -l | egrep '^d'
for dir in glob(dirs=true, files=false) {
  writeln(dir);
}
\end{lstlisting}

\subsection{Example 2}
\begin{lstlisting}
// Yield all subdirectories of the current directory
// In this case we get all subdirectories
// think: ls -l -R | egrep '^d'
for dir in glob(dirs=true, files=false, recursive=-1) {
  writeln(dir);
}
\end{lstlisting}

\subsection{Example 3}
\begin{lstlisting}
// Yield all subdirectories of the current directory
// In this case we get all subdirectories up to depth 10
for dir in glob(dirs=true, files=false, recursive=10) {
  writeln(dir);
}
\end{lstlisting}

\subsection{Example 4}
\begin{lstlisting}
// Yield all files in the current directory
// think: ls -l | egrep -v '^d'
for fl in glob() {
  writeln(fl);
}
\end{lstlisting}

\subsection{Example 5}
\begin{lstlisting}
// Yield all files in and children of the current directory
// think: ls
for fl in glob(dirs=true) {
  writeln(fl);
}
\end{lstlisting}

\subsection{Example 6}
\begin{lstlisting}
// Yeild all .c files in the current directory
// think: ls *.c
for fl in glob(pattern="*.c") {
  writeln(fl);
}
\end{lstlisting}

\subsection{Example 7}
\begin{lstlisting}
// Yield all files in the current directory in sorted order.
for fl in glob(sorted=true) {
  writeln(fl);
}
\end{lstlisting}

\subsection{Example 8}
\begin{lstlisting}
// Yield all .c files from this directory down
// think: ls -R *.c
for fl in glob(pattern="*.c", recursive=-1) {
  writeln(fl);
}
\end{lstlisting}

\subsection{Example 9}
\begin{lstlisting}
// Yield all [a-p] directories from this directory down
// think: ls -R [a-p]
for fl in glob(pattern="[a-p]", dirs=true, recursive=-1) {
  writeln(fl);
}
\end{lstlisting}

\subsection{Example 10}
\begin{lstlisting}
// Yield all [a-p] directories from this directory down. Yield them in sorted order.
for fl in glob(pattern="[a-p]", dirs=true, recursive=-1, sorted=true) {
  writeln(fl);
}
\end{lstlisting}

\subsection{Example 11}
\begin{lstlisting}
// Yield all [a-p] directories from the $CHPL_HOME directory down. Yield them in sorted order.
for fl in glob(pattern="$CHPL_HOME/[a-p]", dirs=true,
               recursive=-1, sorted=true, expand=true) {
  writeln(fl);
}
\end{lstlisting}

\subsection{Example 12}
\begin{lstlisting}
// Yield all [a-p] directories from the $CHPL_HOME directory down. Ignore all
// .git directories and files. Yield them in sorted order.
for fl in glob(pattern="$CHPL_HOME/[a-p]", dirs=true,
               recursive=-1, sorted=true, expand=true, skipdirs=".git") {
  writeln(fl);
}
\end{lstlisting}

%\subsection{Example 13}
%\begin{lstlisting}
%// Yield all [a-p] directories from this directory down. For each directory we
%// enter, return an array/domain of all files that were found in that directory
%for fl in glob(pattern="[a-p]", recursive=-1, chunked=true) {
%// do something with the array of files handed back...
%}
%\end{lstlisting}
\end{document}
