% This is a first draft and is likely to change!

\section{The {\bf noinit} Initializer}
\label{noinit}

This section describes the proposed syntax and semantics for the \chpl{noinit}
initializer.

\subsection{Syntax}
\label{noinit_Syntax}

The special initializer \chpl{noinit} may appear anywhere an \sntx{initializer-expression}
may appear: in a variable or field declaration or as the initialization expression in an
initializer-list.  To include \chpl{noinit} in the syntax, we replace \sntx{expression}
with \sntx{initializer-expression} in those context and define the latter as:
\begin{syntax}
initializer-expression:
  expression
  `noinit'
\end{syntax}

\subsection{Semantics}
\label{noinit_Semantics}

When the variable or field being initialized is of a scalar type (integer, real, imaginary,
complex, bool, enum), presence of the \chpl{noinit} initializer disables the normal
default initialization. 

When the variable or field being initialized is of class or record type, the \chpl{noinit}
initializer is internally replaced by a call to the constructor for the type of the
destination variable or field, passing the argument ``\chpl{noinit = true}''.  Hence the
two declarations below are exactly equivalent: 
\begin{chapel}
var r1:R = noinit;
var r2:R = new R(noinit = true);
\end{chapel}
\noindent
The \chpl{noinit} initializer is essentially syntactical sugar for a specific constructor
invocation.  Note that when \chpl{noinit} is supplied for a variable or field of class
type, \emph{more} initialization will be performed than if the class variable had no
initializer at all and was thus default-initialized to \chpl{nil}.  However, that behavior
is consistent with the interpretation of the \chpl{noinit} initializer as a constructor call.

When the variable or field is of tuple type, the
\chpl{noinit = true} argument is passed to the constructor for each of its elements.


\subsection{Discussion}
\label{noinit_Discussion}

A number of advantages accompany defining the \chpl{noinit} feature in this way:
\begin{itemize}
\item It is easy to implement.
\item It is easy to describe in the specification.
\item When explained as a textual substitution, the feature is easy to comprehend.
\item The effect of \chpl{noinit} is under the control of the record author.
\end{itemize}

This last point is particularly important, because even when \chpl{noinit} is used to
initialize a record, the resulting object must be initialized to a point that it obeys the
basic invariants pertaining to the assignment operator for that record type.

By using the \chpl{noinit} initializer in the initializer-list of a record or class
constructor, the sub-objects of a record or class can be left in an uninitialized or
partially-initialized state (as defined by that field's type).  This permits the effect of
the \chpl{noinit} flag to be defined recursively.  Moreover, it provides a path by which
most of the initialization of a record can be inhibited --- should the record author deem
that desirable.  

The example in the next subsection shows that careful coding can result in record
initialization that overrides the language default behavior and leaves some or all of the
fields partially or completely uninitialized.  The way this is achieved under the current
proposal is rather verbose, so we might consider alternate syntax that inhibits field
initialization by default rather than having it on by default.  We
might have an alternate keyword to indicate that only the named fields are participate in
field-initialization in a constructor.  As a strawman, consider \chpl{with only}
\begin{chapel}
ctor R(x : real)
with only (r = x)
{}
\end{chapel}
In the record type R defined below, this would leave the fields \chpl{i} and \chpl{u}
unitialized and set the value of field \chpl{r} to x.  It then becomes very easy to write
a constructor that will perform no initialization at all:
\begin{chapel}
ctor R() with only () {}
\end{chapel}

An alternate idea would be to give the noinit constructor its own character.  In that
case, the \chpl{noinit} initializer means ``call the noinit constructor passing zero
arguments''.  In the context of a noinit constructor, the with-clause would have a
slightly modified meaning: field names not mentioned explicitly would be initialized
through a call to their noinit constructors, recursively.  Contrast this with the standard
with-clause, wherein if a field name is unmentioned it is initialized using that field's
class-scope or record-scope initial value.


\subsection{Examples}
\label{noinit_Examples}

For example, in:
\begin{chapel}
record R {
  var i: int = -7;
  var r: real = 3.1416;
  var u: uint = 42;

  ctor (param noinit = false) 
  with (i = int(noinit=noinit), r = real(noinit=noinit), u = uint(noinit=noinit)) 
  {}
  ctor (x : real) with (noinit, r = x, noinit) {}
}

var r1:R = noinit;
var r2:R;
var r3:R(2.71828);

writeln(r1);
writeln(r2);
writeln(r3);
\end{chapel}
\noindent
the declaration of \chpl{r1} will provide space for \chpl{r1} but perform no initialization
whatsoever.  That is, default initialization of its fields will be inhibited.  In
contrast, the declaration of \chpl{r2} will result in all three fields being
default-initialized.  (The \chpl{noinit} flag is passed as false to each scalar type's
constructor, but since no value is supplied, the default value for that type is used
instead.)  The declaration of \chpl{r3} will result in fields \chpl{i} and
\chpl{u} remaining uninitialized while \chpl{r} is set to the real value supplied.  The
output of the program should be
\begin{chapelprintoutput}
(i = <rj>, r = <rj>, u = <rj>)
(i = 0, r = 0.0, u = 0)
(i = <rj>, r = 2.71828, u = <rj>)
\end{chapelprintoutput}
where \chpl{<rj>} means ``random junk''.

