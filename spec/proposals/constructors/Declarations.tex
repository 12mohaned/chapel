\section{Declarations}
\label{Declarations}

This section describes the proposed syntax for variable and field declarations, including
optional initializers and their impact upon object initialization.  This proposal does not
address the behavior of the \sntx{no-initialization-part} with respect to class and record
types.  That behavior is the subject of a separate design.

The present proposal seeks to address the sometimes surprising semantics arising from
initialization depending on whether the type is specified.  It also seeks to define
initialization in terms of construction, and remove unnecessary data motion as much as
possible.

Thus, the proposal mainly deals with implementation details.  The proposed changes should
not affect the functional behavior of programs: they should affect only performance.  The
observable changes will affect the total amount of memory allocated and freed, the amount
of processing time spent in data motion and the number of times copy-constructors and
copy-assignment operators are called.  But filtering out a program's introspection
w.r.t. these quantities, its output should remain unaffected.

\subsection{Syntax}

The proposed syntax is unchanged from the existing syntax.

\subsection{Semantics}

The proposed semantic changes are detailed below.  The description is in terms of declared
variables, but it applies in the same way to fields declared within a class or record.

\subsubsection{Normalize Type Inference}

The current implementation conflates value and type representation to the extent that
\emph{representing the declared type of a variable requires creating a default-valued
  object of that type}.  This seems almost self-referential, and probably accounts for
much of the complexity surrounding initialization in the compiler, and the extra code
emitted when a variable's type is specified explicitly.

At a high level, the proposal here is to perform variable type inference separately from
initialization.  That is, if a variable (or the return type of a function) has a declared type, that type will be represented internally as a type only.  It
will not have a value associated with it and it will not entail the creation of a default
value of that type as a stand-in for the type itself.

This will require representing the type of a variable separately from its
initializer.  As it happens, there is already room in the AST for this representation.
But the normalization performed in \chpl{fix_def_expr()} removes both the \chpl{exprType}
and \chpl{init} fields from a variable declaration and replaces them with flattened AST.
The flattened AST creates a temporary object that has the value of the initializer
expression (if present) or an invocation of \chpl{PRIM_INIT} on the type expression
provided.  That temporary is copied into the declared variable by assignment.  

In contrast, we propose that at least the exprType portion of the representation be left in
place, at least until resolution is complete.
The expression representing an \chpl{exprType} must evaluate to a type.  Once type
resolution is complete, it can be discarded.  

Another observation is that only the type information conveyed by such an expression is
important.  Any values used in the representation of that expression may be discarded.
Unfortunately, it is rather glib to say so.  Representation of coercions between
fundamental types is currently tightly coupled with their value representation.  This
seemingly small change in the internal representation will in fact be quite extensive.

\subsubsection{Unify Initialization}

In the current implementation, there is a step where the value is default-initialized
prior to its value being established with an initializer if present.  If the view is taken
that the declaration itself always results in a constructor call (the copy-constructor if
an initializer is supplied and the no-args constructor if not), then default
initialization becomes moot and should be removed.

As part of this proposal, the existing \chpl{_defaultOf()} feature will be preserved.  The
observation is that a call to \chpl{_defaultOf()} is effectively a constructor call.  This
is backed up by the fact that the compiler-supplied implementation of \chpl{_defaultOf()}
actually calls the compiler-supplied zero-args constructor.  

After the type of a variable is established, it will either have an initialization
expression or not.  If the initialization expression is missing, the compiler will add a
call to \chpl{_defaultOf()} to supply the default value corresponding to that type.  If the
\chpl{=noinit} initializer is supplied, then the special actions specified for
that initializer are taken.  Otherwise, the initializer provided explicitly in the code is
used.

Like the \chpl{exprType} expression, it may be desirable to leave the initializer attached
to the declaration rather than smashing it flat.  But in this case, I think either
representation will serve: there is no pressing need to un-flatten the representation.

\subsubsection{Collapse Constructor Calls}

In the current implementation, constructor calls allocate and initialize (for class types)
or simply initialize (for record types) an unnamed object of the corresponding type.  
For class types, the resulting object is copied into the named variable using assignment.
But since assignment of class variables is rigidly defined by the language to be a bitwise
copy of a reference (pointer) to that unnamed object, the copy is not expensive.

For record types, the story is different.  In general, the semantics of assignment are not
defined if the left operand is in an uninitialized state.  So using assignment to
initialize the named variable is flat wrong.  If the initializer expression is a named
object (see below) then at worst copy-construction should be used.  But for constructor
calls we can do even better by eliminating the copy entirely.

Secondly, it is wasteful of effort to gin up
an anonymous record object, copy its contents into a named record object and then throw
the original away.  It would be better to express a constructor as a method that
initializes the memory associated with the named variable in situ.  Allocation and
deallocation of the unnamed temporary could be avoided, as well as the verbatim copy.

Therefore, the proposal is to represent constructors as methods and the calls to them as
method invocations.  This will directly support initializing record objects in situ.  For
symmetry, it is desirable (though less urgent) to pull the allocation part of class
object creation (via a \chpl{new} expression) outside of the constructor.  That would make
the internal structure of class and record constructors identical.  

Where an explicit initializer or constructor call is not provided, it is similarly desirable to have
\chpl{_defaultOf()} defined as a method, for the same reasons as described above for
constructors.  In the current implementation \chpl{_defaultOf()} works like a constructor
--- its default implementation for record types even devolving to a constructor call.
Maintaining that symmetry will keep the specification and implementation of
\chpl{_defaultOf()} simple.

\subsubsection{Use Copy-Construction, Not Assignment}

The cases discussed so far involve no initializer, the \chpl{=noinit} initializer and an
initializer that looks like a constructor call (i.e. \chpl{= new T(...)}.  The remaining
case is where the initializer expression is an expression coercible to the type of the
named variable.\footnote{In the case that the type of the named variable is not specified
  explicitly, it will be inferred from the computed type of the initializer expression.
  The inferred type of the variable will of course always match the type of the
  initializer expression exactly, so the latter is always trivially coercible to the
  former.}

As explained above, assignment in the context of initialization is illegal, because the
left operand is not yet initialized.  Copy-construction should be used instead.  The
proposal here is to simply substitute a copy-constructor call where assignment is inserted
today.  For fundamental and class types, copy-construction, assignment and bit-wise moves are
semantically equivalent, so in that realm it becomes merely a name change.  The change
will only be of consequence as it applies to records and record-based types (e.g. tuples
and ranges).

\subsection{Discussion}

Using copy-construction in place of assignment does not currently make any difference as
long as the compiler-provided versions of those methods are used.  However, for
user-defined record types, the assumption of prior initialization of the left operand of
assignment can be violated, and such a violation may cause incorrect behavior.  I believe
we already have one test case that fails (only in multi-locale testing) due to this
problem (in privatized arrays, because the default privatization index is not distinguished
from a valid one).  More such problems may appear as greater use is made of records to
support ad hoc memory management schemes.  The simple solution is to ``do it
right''.

Normalizing type inference will get rid of the extra type variable initialization code.
After which, unifying the initialization code just means adding a \chpl{_defaultOf} call
wherever no explicit initializer is provided.  Collapsing constructor calls will provide
some nice code size reduction and performance improvement.

\subsection{Examples}

An example of the odd behavior associated with an explicit type as well as constructor
call in an initializer is given by test/types/records/hilde/newUserDefaultCtor.chpl:
\begin{chapel}
record R
{
  proc R() { writeln("Called R()."); }
  proc ~R() { writeln("Called ~R()."); }
}

var r:R = new R();

writeln("Done.");
\end{chapel}
The output of this program is:
\begin{chapelprintoutput}
Called R().
Called R().
Called ~R().
Done.
Called ~R().
\end{chapelprintoutput}
\noindent
whereas what is expected is:
\begin{chapelprintoutput}
Called R().
Done.
Called ~R().
\end{chapelprintoutput}


The difference between calling assignment versus calling the copy-constructor should be
made apparent by instrumenting both in a test case.

The other changes --- separating out type inference and handling initialization uniformly
--- can probably only be detected by examining the AST.  The supporting change of
representing and calling constructors as methods would, for the most part only be visible
in the AST.  Although (as mentioned above) total memory usage statistics would also be
expected to change.  The performance benefits of initializing variables in situ might be
exposed by calling a function that performs initialization a couple million times and
comparing the total elapsed time before and after.

