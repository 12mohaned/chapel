\section{Constructors}
\label{Constructors}

This section describes the proposed syntax and semantics for constructors.

\subsection{Syntax}

A constructor performs initialization of an object.  On entry, a
constructor may assume that each of its fields has been default-initialized according to the
rules for that type, as specified in Section 8.1.1 of the Chapel Language
Specification.\footnote{Default-initialization may be overridden for an individual type by
  defining an explicit \chpl{_defaultOf(type t)} for that type.}
Default-initialization is supplied by the compiler, so it has no corresponding syntactical
element.\footnote{A separate proposal is being developed for a ``no-init'' capability,
  whereby default-initialization may be overridden in whole or in part.}

\begin{note}
The default-initialization value for most fundamental types is chosen such that a 
\chpl{memset( , 0, );} call would have the desired effect.  This could be quite useful in
initializing contiguous memory occupied by the representation of a class, record or tuple.
This property does not hold for the more complex types: range, domain, array, sync, single
or atomic.  However, since all of these types are implemented in module code,
zero-initialization can still be used for the default-initialization part of their
definition.
\end{note}

Default-initialization is followed by field-initialization.  Field-initialization
constructs each field in an object, in an order specified in the Semantics
(\rsec{Constructor_Semantics} subsection below.  The syntactical elements influencing
field-initialization are the field declarations and the initializers appearing in the
\sntx{initializer-list}.  Field-initialization is complete before control enters the body
of the constructor.

The body of the constructor performs the actual construction of the object.  Since
field-initialization is already complete by the time the body of the constructor is
entered, it is quite common for a constructor body to be empty.

The syntax of a constructor may thus be set down as:
\begin{syntax}
constructor-declaration-statement:
  linkage-specifier[OPT] `ctor' type-binding constructor-name[OPT] argument-list 
    where-clause[OPT] initializer-clause[OPT] constructor-body

constructor-name:
  identifier

initializer-clause:
  `with' ( initializer-list )

initializer-list:
  initializer
  initializer-list , initializer

initializer:
  field-name = expression

field-name:
  identifier

constructor-body:
  statement
\end{syntax}

The linkage specifier (if present) is \chpl{inline} \chpl{extern} or \chpl{export}.  Since
in general the backend implementation does not support the object model, the
\chpl{extern} option probably does not make much sense at present.  Constructors are
expected to be statically-linked methods, so it should be possible for an external program
to call a constructor exported using the
\chpl{export} linkage-specifier through a stereotyped method interface.  The meaning of
the \chpl{inline} specifier should be self-evident.

The keyword \chpl{ctor} introduces a constructor as distinguished from a procedure or
iterator.  A special keyword was chosen to support the arbitrary naming of constructors --
which would be a deviation from the language as currently specified.  At present, the name
of the constructor must match the type name, so there is no need for a special keyword:
the compiler can infer from its name that a function is a constructor.

The \sntx{constructor-name} may be omitted, in which case the default
constructor name is used internally.  The default constructor name is an
implementation-defined reserved name that is unique within the (class) scope in
which it is defined.  If the \sntx{constructor-name} is
the same as the class name, it is replaced internally by the default constructor name.
\begin{rationale}
Programmers familiar with the C++ convention can use the class name to nominate
constructors.  However, the creation of generic classes is simplified if the
constructor name is invariant across classes.  Making the constructor name
optional serves this purpose by associating the empty string with the same set
of overloaded constructors.
\end{rationale}

Constructor declarations do not have a parentheses-less form.
\begin{rationale}
Normal methods may have either a single parenthesesless version or overloaded
versions taking various argument lists.  In typical usage, we expect overloaded
constructors to be common.  Also, the parenthesesless form would have the same
behavior as the default constructor --- that is, it would perform default
initialization for all fields.  But the compiler-generated constructor already
does this (if called with an empty argument list).  Given that the
parenthesesless form is redundant, and potentially takes flexibility away from
the user, there is no advantage to supporting it.
\end{rationale}

If present, the \sntx{initializer-clause} contains a comma-separated list of
\sntx{initializer}s.  Each field-name in the initializer list must correspond to the name
of one of the fields declared in that class or record, or in one of its base types
(recursively).  Each initializer contains a field-name followed by an
\chpl{=} followed by a new-expression for value fields or a type expression for type fields.
Each initializer results in the initialization of that field by
calling the constructor corresponding to the type of that field, passing in the arguments
supplied.  In this context, the \chpl{=} does not mean assignment; rather, it implies
initialization.

\begin{rationale}
Since a new-expression already provides syntax for creating a (class or record) object
passing an arbitrary argument list to the constructor, this behavior can be co-opted for
the purposes of initialization -- thus avoiding a radical syntax change while supporting
the required semantics of the added \sntx{initializer-clause}.  

In terms of implementation, it is understood that initialization of a variable or field
using a new-expression does not mean that an anonymous symbol is first built up and then
copied verbatim into the named memory location.  Rather, the new-expression merely
specifies how the named variable is to be constructed in-place.
\end{rationale}

Within an \sntx{initializer-clause}, the name of each field may appear at most
once.  A field name may contain any number of \chpl{super .} qualifiers to distinguish an
inherited field from a field of the same name in the most-derived type.

It is illegal to declare a return type for a constructor; the return type is
implicitly \chpl{void}.  This also means that if a \chpl{return} statement
appears within the \sntx{constructor-body}, it must be of the form that does not
return an expression.

Within an \sntx{initializer-clause}, the names of all formal arguments in that constructor
declaration are in scope.  All module-level variables may also be referenced.  In
addition, any field names associated with that constructor's type or any of its base types
are in scope.  However, if a field name is used in the list of expressions passed as
actual arguments to a field-initializer, the field-initializer associated with that name
must appear to the left of its use.  It is illegal to reference \chpl{this} within the
\sntx{initializer-clause}.

Within the body of a constructor, the names of all formals, field names and visible
module-level variables are visible, as well as any variables defined within the body
itself.


\subsection{Binding}

In general, a constructor call has the form:
\begin{syntax}
constructor-call-expression:
  `new' type-name ( named-expression-list[OPT] )
\end{syntax}
The \sntx{type-name} must name a type.  The named type may be generic.
The set of visible constructors are all constructors defined for that type.  Usual
generic expansion rules are applied to the set of visible functions to select a set of
candidate constructors.  Any \chpl{where} clauses are applied to further reduce the set of
candidates.  Disambiguation is applied to select the best candidate from among those
remaining.  It is a program error if there is no unique best candidate.  Otherwise, the
call is bound to that unique best candidate.

Actual arguments in the call are bound to formal arguments in the constructor according to
the usual argument binding rules (including the substitution of default formals for
omitted actual arguments).

\subsection{Semantics}
\label{Constructor_Semantics}

When a constructor is executed, formal initialization occurs first, followed by field
initialization and then construction.  Formal initialization consists of initializing the
formal arguments in the constructor from the passed-in actual arguments using the normal
argument-passing rules.  Construction consists of executing the body of the
constructor.  Field initialization is detailed below.

Field initialization starts with the first initializer in the initializer-list and
proceeds from left to right.  Initialization of the first field in the list completes
before initialization of the second field begins and so on.  The expressions used in any
named-expression passed as an actual argument to a field-initializer call may consist of
any expression that is valid in that scope (including function calls and other
expressions).  In particular, the names of any or all formal arguments of the constructor
as well as the names of any fields that have already been initialized may be used.  The
use of the name of any field that has not yet been initialized is illegal\footnote{This
  can be reported as a syntax error, since the error condition can be detected
  statically.}

After the initializer-list has been processed, some fields may remain uninitialized.
These fields are initialized using the default initialization specified for them in the
class or record declaration.  The fields of a class or record are processed in lexical
order.  If not already initialized, the initializer associated with the field declaration
(if any) is used.  Otherwise, the field is default-initialized via a call to its
default-constructor.

\begin{chapelexample}{InitializationOrder.chpl}
\begin{chapel}
record R {
  var a : int;
  var b : int = 3;  // Same as var b = 3 or var b:int = new int(3);
  var c : int;
  var d = new int(4);
  ctor() with (c = 4, d = c + 1) {}
  ctor(_c) with (a = 2, c = _c, d = a**b) {}
}
\end{chapel}
\end{chapelexample}
In this example, the first constructor is legal while the second is not.  In the first
one, \chpl{c} is first initialized to \chpl{4},
then \chpl{d} is initialized to \chpl{5}.  It is legal to use \chpl{c} in this context
because it has already been initialized.  Because initializers for \chpl{a} and \chpl{b}
were not listed in the \sntx{initializer-clause}, they are initialized as specified by
their field declarations in the record itself.  In lexical order, \chpl{a} appears first.
Because no explicit initializer is called, the default constructor for type \chpl{int} is
called on \chpl{a}, which has the effect of setting \chpl{a} to zero.  The declaration for
\chpl{b} is equivalent to a copy-constructor call for type \chpl{int}.  It sets the value
of \chpl{b} to \chpl{3}.

The second example would be flagged as containing a coding error.  In the initializer
list, the field name \chpl{b} is accessed in the initialization expression for \chpl{d}
before \chpl{b} has been initialized.  This could be repaired by initializing \chpl{b}
explicitly before the initializer for \chpl{d} is encountered.  We might consider special
syntax, where mentioning just the name of a field would mean ``use the field-default
initializer for this field.  In which case
\begin{chapel}
  ctor(_c) with (a = 2, c = _c, b, d = a**b) {}
\end{chapel}
\noindent
would mean, ``initialize \chpl{a} to \chpl{2}, \chpl{c} to \chpl{_c}, \chpl{b} to \chpl{3}
and then \chpl{d} to \chpl{8}.''

\subsection{Discussion}

The existing initialization syntax for variable and field initialization can also be
employed in the initializer list.  Because they specify initialization, the
proposed semantics are different from those of the existing implementation.  At present,
assuming that \chpl{R} names a record type
\begin{chapel}
var r:R = new R(args,...);
\end{chapel}
\noindent
is actually implemented by first creating a variable \chpl{r} and default-initializing
this according to the definition of \chpl{R}.  It then proceeds to construct an anonymous
R, passing in the argument list \chpl{args,...}.  The value of the resulting anonymous
record is then copied into \chpl{r} using the assignment operator that can accept an
object of type R as it left- and right-hand arguments.

The default-initialization of \chpl{r} and its
redefinition through assignment are both undesirable.  We really want variable and field
definitions to construct the corresponding object in-situ.  In this view, a variable
declared without an initializer should be interpreted as a call to the default (no-args)
constructor for that variable's type.  Thus
\begin{chapel}
var r:R;
\end{chapel}
\noindent
should be interpreted roughly as
\begin{chapel}
r.R();
\end{chapel}
That is, the a constructor for type \chpl{R} is called as a method, specifying \chpl{r} as
the receiver and passing no arguments.

A declaration with an \sntx{initializer-part} should be interpreted as a constructor call
in which only one argument is passed.  If the type of the variable (or field) and the initializer
expression are the same (as will always be true if the type of the variable for field is
inferred), then the method called will be the copy-constructor for that type.  The two
types may differ as long as there is a constructor that will accept a single argument of
the initialization expression's type.  (If not, then an unresolved call error will be
reported.)
\begin{chapel}
var r = s;
\end{chapel}
This says that r is initialized by calling the copy-constructor for the type of \chpl{s}.
Assuming that \chpl{s} is of type \chpl{R}, this would be roughly equivalent to
\begin{chapel}
r.R(s);
\end{chapel}

As a special case, the syntax
\begin{chapel}
var r = new R(args,...);
\end{chapel}
should not construct an anonymous R and call the copy-constructor to then initialize
\chpl{r}.  Rather, it should call the constructor for \chpl{R} with the given argument
list to initialize \chpl{r} in situ.  This would be roughly equivalent to:
\begin{chapel}
r.R(args,...);
\end{chapel}

It is desirable that the transformation of the two forms involving \chpl{=} into in-place
constructor calls take place before resolution.  In that way, construction of variables
and fields will be called as a method, and all standard steps of resolution (including
scope resolution, visible method selection, generic instantiation, candidate selection and
argument binding) used in a consisten way.

\subsection{Examples}

As a practical example, we can consider an explicit initializer for one of the array
implementation types.  At present, the code relies upon the compiler-supplied all-fields
constructor and the special initialize function.  These two pieces can be combined in an
explicit constructor, to make the semantics of initializing an object of that type
somewhat more obvious.

The field-declaration part of the that class is given by:
\begin{chapel}
  class DefaultRectangularArr: BaseArr {
    type eltType;
    param rank : int;
    type idxType;
    param stridable: bool;
  
    var dom : DefaultRectangularDom(rank=rank, idxType=idxType,
                                           stridable=stridable);
    var off: rank*idxType;
    var blk: rank*idxType;
    var str: rank*chpl__signedType(idxType);
    var origin: idxType;
    var factoredOffs: idxType;
    var data : _ddata(eltType);
    var shiftedData : _ddata(eltType);
    var noinit_data: bool = false;

    ...
  }
\end{chapel}
That would make the compiler-generated all-fields constructor look like:
\begin{chapel}
class DefaultRectangularArr : BaseArr {
  ctor (type eltType,
        param rank : int,
        type idxType,
        param stridable : bool = false,
        dom = new DefaultRectangularDom(rank=rank, idxType=idxType,
                                         stridable=stridable),
        off : rank*idxType,
        blk : rank*idxType,
        str : rank*chpl__signedType(idxType),
        origin : idxType,
        factoredOffs : idxType,
        data : _ddata(eltType),
        noinit_data : bool = false)
    with (eltType = eltType,
          rank = rank,
          idxType = idxType,
          stridable = stridable,
          dom = dom,
          off = off,
          blk = blk,
          str = str,
          origin = origin,
          factoredOffs = factoredOffs,
          data = data,
          noinit_data = noinit_data) {}
}
\end{chapel}

\begin{openissue}
The example is written this way because I realize there is no way to use a qualified name
for the constructor but omit the name itself so that the compiler-default constructor name
is used.  I think the empty string would still work in this context, but it looks a bit
awkward to say:
\begin{chapel}
ctor DefaultRectangularArr.( ... ) with ( ... ) {}
\end{chapel}
It is worth considering alternatives.
\end{openissue}

The \chpl{initialize()} routine is called implicitly by each constructor, including
compiler-generated ones.
The \chpl{initialize()} routine for this class is given by:
\begin{chapel}
    proc initialize() {
      if noinit_data == true then return;
      for param dim in 1..rank {
        off(dim) = dom.dsiDim(dim).alignedLow;
        str(dim) = dom.dsiDim(dim).stride;
      }
      blk(rank) = 1:idxType;
      for param dim in 1..(rank-1) by -1 do
        blk(dim) = blk(dim+1) * dom.dsiDim(dim+1).length;
      computeFactoredOffs();
      var size = blk(1) * dom.dsiDim(1).length;
      data = _ddata_allocate(eltType, size);
      initShiftedData();
    }
\end{chapel}


\begin{chapel}
ctor DefaultRectangularArr.DefaultRectangularArr(type eltType,
                                                 param rank : int,
                                                 type idxType,
                                                 param stridable : bool = false,
                                                 dom : _domain)
  with (eltType = eltType, 
        rank = rank, 
        idxType = idxType, 
        stridable = stridable,
        dom = dom) {
    if noinit_data == true then return;
    for param dim in 1..rank {
      off(dim) = dom.dsiDim(dim).alignedLow;
      str(dim) = dom.dsiDim(dim).stride;
    }
    blk(rank) = 1:idxType;
    for param dim in 1..(rank-1) by -1 do
      blk(dim) = blk(dim+1) * dom.dsiDim(dim+1).length;
    computeFactoredOffs();
    var size = blk(1) * dom.dsiDim(1).length;
    data = _ddata_allocate(eltType, size);
    initShiftedData();
}
\end{chapel}
Or more simply, the body of this constructor could be replaced by an explicit call to
\chpl{initialize()}.  According to the current design, implicit calls to
\chpl{initialize()} would be eliminated, so in classes that currently exist for which such
a function is defined, it would have to be called explicitly.

The semantics of this constructor are as follows:
\begin{enumerate}
\item Formal arguments are initialized by their corresponding actuals.
\item Default values are supplied for unbound formals (e.g. if an actual corresponding to
  \chpl{stridable} is missing, then \chpl{false} is used).
\item The initializations specified in the \chpl{with} clause are processed
  left-to-right.  The only value field specified in the \chpl{with} clause is \chpl{dom};
  this gets initialized with the domain object that was passed in as an argument.
\item Any value fields left uninitialized at that point are initialized to their default
  values as specified in the class declaration.  Specifically, \chpl{off}, \chpl{blk},
  \chpl{str}, \chpl{origin}, \chpl{factoredOffs}, \chpl{data} and \chpl{shiftedData} are
  all initialized to the default value of their respective types.  The
  \chpl{noinit_data} field is then initialized to \chpl{false}, as specified in the class
  declaration.
\item Then, the statements in the body of the constructor are executed.
\end{enumerate}

One possible call that would bind to this constructor is taken from the existing
implementation of \chpl{DefaultRectangularDom.dsiBuildArray(type eltType)}:
\begin{chapel}
      return new DefaultRectangularArr(eltType=eltType, rank=rank, idxType=idxType,
                                      stridable=stridable, dom=this);
\end{chapel}


%Notes:
% Brad favors keeping the ``new'' syntax for creating a value of a given type and using
% this to initialize a field or variable.  The compiler should recognize this idiom early
% in translation and replace it with in-place initialization of the same variable.  That is,
%   var r = new R(<arg-list>);
% actually devolves to something like
%   <stack-allocate r>
%   r.R(<arg-list>)
% which means that the memory for r gets initialized in-place.
%
% The ``new'' keyword is not necessary for type-initializers, because types are not
% values.  Probably ditto for param arguments, but in their case due to the fact that
% construction is not required: They are either literals or expressions that can be folded
% down to literals.
%
% As a rough swag, Brad is OK with the syntax naming the field only in order to get the
% default initialization defined in the record or class declaration.
%

