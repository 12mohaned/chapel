\section{Introduction}

The original design for Chapel anticipated that memory management would be handled
by garbage collection.  However, efficient distributed-memory garbage collection
algorithms have not been forthcoming.  In the mean time, the desire to support a growing
number of users by keeping memory leakage within practical bounds has forced the
implementation to provide some means for reclaiming memory.  

Since memory reclamation was not part of the original design, its implementation is somewhat
ad hoc.  A review of the memory management elements in use in the Chapel compiler revealed
that most of the necessary elements are present in some form.  But since these have not
been applied uniformly across all types, memory is still leaked.

To provide the user full control over memory management, the compiler needs to be modified
to call UMM support routines at all the necessary places in the code.  These consist
of two particular constructors (the zero-args constructor and the copy-constructor),
assignment and the destructor.  It is hoped that by applying these calls in a consistent
manner, type-specific handling of memory management can be removed and the compiler
rendered simpler.

Since the Chapel implementation of user-managed memory was added after the initial design,
the specification needs to be updated to integrate this concept with the descriptions of
object creation, copying and destruction.  This proposal restates the object
model as it relates to the current Chapel implementation, and makes recommendations on how
the existing specification should be updated.  In addition, it provides recommendations for
syntactical and semantical modifications that may make that model more usable.

The remainder of this document is organized as follows: Section \rsec{Objects} discusses
the object life-cycle and provides a brief overview of the User-Managed Memory model.  It
also contains general information common to the other sections.  Section
\rsec{Declarations} presents the proposed syntax and semantics for variable and field
declarations outside the context of a constructor definition, highlighting how this
differs from the current implementation.  Section \rsec{Constructors} presents the
proposed syntax and semantics for constructors, highlighting how this differs from the
current implementation.

Appendices are provided for quick reference.  Appendix \rsec{Changes} provides a summary of the
proposed changes.  Appendix \rsec{Examples} gives examples including declarations for the
fundamental types provided by Chapel.

