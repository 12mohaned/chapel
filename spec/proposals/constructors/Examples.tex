\section{Examples}
\label{Examples}

This section contains specific examples, showing how the proposal would be expanded
relative to the fundamental types provided by the Chapel language.

\subsection{Record Variables}

When \chpl{T} is a record type, we expect the following Chapel code
\begin{chapel}
  var s:R;
  var t:R = noinit;
  var u:R = new R();
  var v:R = new R(8, 9, 10.0);
  var w = new R();
  var x = new R(9, 16, 0.5625);
\end{chapel}
\noindent
to produce the corresponding AST:
\begin{numberedchapel}
  var s:R
  (call _defaultOf s) // Calls R._defaultOf()
  var t:R
  (call _construct_R t) // Calls R.R(noinit=true)
  var u:R
  (call _construct_R u) // Calls R.R()
  var v:R
  (call _construct_R v 8 9 10.0) // Calls R.R(8, 9, 10.0)
  var w:R
  (call _construct_R w) // Calls R.R()
  var x:R
  (call _construct_R x 9 16 0.5625) // Call R.R(9, 16, 0.5625)
\end{numberedchapel}
The compiler-supplied definition for \chpl{_defaultOf} for this type is expected to look
like:
\begin{chapel}
  proc R._defaultOf() {
    this.();
  }
\end{chapel}
The implementation might supply a default constructor that looks like:
\begin{chapel}
  ctor R(param noinit:bool, u:uint = 7, i:int = -2, r:real = 0.0)
    with (u = new uint(noinit, u),
          i = new int(noinit, i),
          r = new real(noinit, real)) 
  { }
\end{chapel}
\noindent
The nested constructor calls in the \sntx{initializer-clause} pass along the noinit flag
to the constructor for that field type.  Those in turn will replace the operand
field with the input or defaulted formal if \chpl{noinit} is \chpl{false} and otherwise do
nothing.

We note here that the execution of an initialization in the \sntx{initializer-clause} is
unconditional, so there is no way to control whether a field is initialized explicitly
vs. falling back to default initialization.  The puzzle can be solved by using a where
clause to implement the effects of the \chpl{noinit} flag.  So another way to write the
above is:
\begin{chapel}
  ctor R(param noinit=false, u:uint = 7, i:int = -2, r:real = 0.0)
    where noinit == false
    with (u = u, i = i, r = r)
  { }

  ctor R(param noinit=false, u:uint = 7, i:int = -2, r:real = 0.0)
    where noinit == true
    with (u = noinit, i = noinit, r = noinit)
  { }
\end{chapel}
\noindent
In this case, the \chpl{noinit} field initializer inhibits initialization of the
corresponding field.  

There is still the question of how to default to the field-default value when an explicit
argument is not supplied to that formal in the constructor call.  This is a more general
problem related to determining whether the corresponding argument has been supplied.
Providing a property with the formal to indicate this would support that, but to pick up
the default value for each field would still require writing the full matrix
possibilities for arguments supplied or not.  The question of a better syntax is left open.

With the \chpl{_defaultOf} and constructors defined above inlined in the AST and scalar
replacement applied, the result would be:
\begin{numberedchapel}
  var s:R
  ('.=' s u 7)
  ('.=' s i -2)
  ('.=' s r 0.0)

  var t:R

  var u:R
  ('.=' u u 7)
  ('.=' u i -2)
  ('.=' u r 0.0)

  var v:R
  ('.=' v u 8)
  ('.=' v i 9)
  ('.=' v r 10.0)

  var w:R
  ('.=' w u 7)
  ('.=' w i -2)
  ('.=' w r 0.0)

  var x:R
  ('.=' v u 9)
  ('.=' v i 16)
  ('.=' v r 0.5625)
\end{numberedchapel}


\subsection{Class Variables}
When initializing a variable of class type, the following forms are possible:
\begin{chapel}
  var c:C; c = new C(false);
  var d:C = new C(true);
  var e = new C(false);
  var f:C = noinit;
\end{chapel}
\noindent
According to the proposal, these would result in the following AST:
\begin{numberedchapel}
  var c:C
  ('move' c nil)
  var tmp:C
  ('move' tmp ('alloc' sizeof(C)))
  (call _construct_C tmp) // Calls C(false)
  ('move' c tmp)

  var d:C
  ('move' d ('alloc' sizeof(C)))
  (call _construct_C d) // Calls C(true)

  var e:C
  ('move' e ('alloc' sizeof(C)))
  (call _construct_C e) // Calls C(false)

  var f:C
  ('move' f nil)
\end{numberedchapel}
\noindent
Note that the primitive \chpl{'move'} in the above code is intended to mean a bit-wise
copy.  It corresponds to whatever primitive operation in the implmentation accomplishes
this, so it is not tied to the deprecated \chpl{PRIM_MOVE} primitive.  Reference to its
replacement \chpl{PRIM_ASSIGN} was avoided, because its representation in AST logging is
\chpl{'='}, which suggests assignment.  Assignment in general means more than a
bit-wise copy, since it may also entail releasing resources currently held by the LHS.

One may imagine that after \chpl{PRIM_MOVE} has been deprecated and removed from the
compiler, the \chpl{PRIM_ASSIGN} primitive may be renamed as \chpl{PRIM_MOVE} and its
representation in the AST log changed to \chpl{'move'}.  That is the implementation to
which this document is speaking.


\subsection{Distribution Variables}

In the current implementation, distributions are implemented as classes, so we expect the
creation of distribution variables to follow the pattern outlined for class types above.
Thus, for the Chapel code:
\begin{chapel}
var a: Block(rank=2);
var d: Block(rank=2) = noinit;
var b = new Block({2..5, 2..5});
var c: Block(rank=2) = new Block({1..4, 1..4});
\end{chapel}
\noindent
we expect the intermediate AST:
\begin{numberedchapel}
  var a:Block(2,int(64))
  (call _defaultOf a) // Calls Block(2,int(64))._defaultOf()

  var d:Block(2,int(64))
  (call _construct_Block d) // Calls Block(2,int(64))(noinit = true)

  var b:Block(2,int(64))
  var tmp0:domain(2,int(64),false)
  var tmp1:range(int(64),bounded,false)
  (call _construct_range tmp1 2 5)
  var tmp2:range(int(64),bounded,false)
  (call _construct_range tmp2 2 5)
  (call _construct_domain_2_int64_t_F tmp0 tmp1 tmp2)
  (call _construct_Block b tmp0) // Calls Block(2,int(64))(boundingBox = tmp0)

  var c:Block(2,int(64))
  var tmp0:domain(2,int(64),false)
  var tmp1:range(int(64),bounded,false)
  (call _construct_range tmp1 1 4)
  var tmp2:range(int(64),bounded,false)
  (call _construct_range tmp2 1 4)
  (call _construct_domain_2_int64_t_F tmp0 tmp1 tmp2)
  (call _construct_Block c tmp0) // Calls Block(2,int(64))(boundingBox = tmp0)
\end{numberedchapel}
As for classes, as show above, the action of \chpl{_defaultOf} on a distribution class is
to set \chpl{this} to \chpl{nil}.

I am not sure what actions would be taken here when \chpl{noinit} was passed as
\chpl{true}.  It is up to the author for that distribution to do something sensible in
that case.  It seems like initializing the environment arguments \chpl{dataPar...} can be
done early at low cost.  Other parts of the distribution need to be filled in before it is
used.  This can be done piecemeal, or through an initialization-finalization method.  In
either case, the finalization of the initialization must be done explicitly in the client code.

At present, we may lack the syntax to perform a pointer replacement of a \chpl{nil}
array pointer with a full-fledged array.  Practically, this can be accomplished by
inventing a new pointer-copy primitive and calling that.  This leaves the user view of
array assignment as being an element-by-element replacement undisturbed while supporting
partial initialization of objects containing arrays.  


\subsection{Array Variables}

The process through which an array variable is constructed is similar to that for a
domain.  The runtime type info variable has a field that contains a copy of the domain
information. The type of the structure is fully known after resolution: it contains only
one field whose type is exactly \chpl{domain(2,int(64),false)}.

The call to \chpl{chpl_convertRuntimeTypeToValue()} produces a result of the array type,
which is the element type \chpl{int(64)} attached to the domain type above.  The runtime
type info variable also conveys the dimensions of the domain.  (Note that a
fully-constructed domain object would convey the same information.)

In the current implementation, the function \chpl{chpl__buildArrayRuntimeType()} calls
\chpl{dom.buildArray(eltType)}, which attaches the element type to the passed-in domain
and returns a fully-constructed array.  The \chpl{buildArray()} function in turn calls
\chpl{dsiBuildArray} on the value extracted from the domain representation.  The value
field is the actual domain object, which is wrapped in the \chpl{_domain} record to support
privatization.  The actual work of building the array is done in \chpl{dsiBuildArray}; the
call to \chpl{_newArray()} merely updates the fields in the container record.

The function \chpl{dsiBuildArray} calls down to the array constructor for the
corresponding domain type.  For a default rectangular array, this is
\chpl{DefaultRectangularArr}.  The generic arguments \chpl{rank}, \chpl{idxType} and
\chpl{stridable} are copied from the domain on which \chpl{dsiBuildArray} is called.  The
elementType is supplied as an argument, and a pointer to the domain is passed to the
array constructor as the \chpl{dom} argument.

There is no explicit constructor for the \chpl{DefaultRectangularArr} as it stands, so a
wrapper for the all-fields constructor is created to set up the fields in the
\chpl{DefaultRectangularArr} object.  The generic fields include everything except the
\chpl{dom} argument.  The remaining fields in the object are value fields.  Their types
are mostly dependent upon the generic fields.  Thus, the generic fields must all be
resolved before the types of the remaining fields can be established.

The \chpl{initialize()} procedure is called implicitly as part of the all-fields
constructor implementation.  In this routine, if the field \chpl{noinit_data} is true,
then allocation and initialization of the \chpl{_ddata} representation is skipped.  This
is used in \chpl{dsiReindex} and \chpl{dsiSlice} to create an alias of that representation
(rather than creating it anew).  Otherwise, the call to \chpl{_ddata_allocate} takes care
of both allocating and initializing the array data.  We note that at this point, the
allocation and initialization could be separated if, for example, the algorithm could
guarantee that initialization were performed later.

At this point, the proposal is to keep the construction of arrays largely intact.  In
order to carry out the deprecation of the \chpl{initialize} function, however, it would be
necessary to rewrite the pertinent forms of array constructors so that they could call the
\chpl{initialize()} function explicitly.  

Under this proposal, the explicit constructor for a \chpl{DefaultRectangularArr} would
look like:
\begin{chapel}
  ctor DefaultRectangularArr(type eltType, param rank:int, type idxType,
                             param stridable:bool, dom: _domain)
    with (eltType = eltType, rank = rank, idxType = idxType,
          stridable = stridable, dom = dom) {
      initialize();
    }
\end{chapel}
\noindent
As noted above, the four generic arguments establish the type of the object being
constructed.  Once the type is instantiated, the types of all of the fields in that type
is determined.  Current behavior is to default-initialize those fields.  The same action
will be taken in the constructor shown.  The type of the domain argument was abbreviated
as \chpl{_domain}.  However, given the arguments to the constructor, this type could be
expanded as \chpl{DefaultRectangularDom(rank=rank, idxType=idxType, stridable=stridable)},
just as it is in the class declaration.

Aside from the layers that are added by the DSI abstraction and the privatization
mechanism, the representation of arrays are mostly class-based.  As has been mentioned
above, it should be possible to replace the runtime type implementation with appropriate
introspection in the compiler.  That done, in the actual implementation, the domain field
in an array could be represented by a base class that was the common ancestor of all
domain times.  No further information would be required in the representation of the
domain field in an array.  On the other hand, the domain object actually attached has a
runtime type associated with it.  This supports the polymorphic behavior required to
uniformly represent arrays based on disparate domain types.  I believe that dynamic
dispatch is already used to handle this polymorphic behavior, so there would be no
performance loss associated with removing an explicit type from the domain field.

To implement a \chpl{noinit} capability that initializes everything except the values of
the array elements, it would be necessary to supply a boolean to control whether the call
to \chpl{init_elts} is made.  That flag would have to be propagate back up the call
hierarchy outlined above, and the \chpl{noinit} initializer translated into that flag by
the compiler at the place where \chpl{chpl__convertRuntimeTypeToValue()} is currently
called.

\subsection{Domain Variables}

I suspect (though I still don't understand it fully) that the ``runtime type'' mechanism ---
as used with domain variable declarations --- provides the means to create a variable with
a partial type, and have the rest of the type information filled in later in the code.  It
modifies the compiler's normal type resolution machinery, to back-patch the concrete type
of the variable after the initially-lacking information is supplied.

This technique has the advantages that the full type of the may be resolved statically, and
the representation of the type itself remains flat.  It has the disadvantage that it adds
complexity and special-case code to the compiler handle this specialized type inference.
It also adds bulk to the AST and even the generated code, so the efficiency advantages of
the flat representation become questionable.

An alternative approach would be to partition the representation of the type, so that
information supplied in the initial declaration was stored in the top-level object.
Information that was supplied later (e.g. in an initializer) would be stored in a
contained object that could be polymorphic.  To represent the variable type information,
the top-level structure need only contain a class variable.  The object attached to that
class variable at run time would determine the full type of the domain being represented.

Certain code in the compiler needs to inspect elements of the full static type of a
domain.  Here is where the special-case code must come into play.  The compiler needs to
know how to traverse the AST to extract element types that are added post-facto.  I think
that this kind of introspection follows a much more logical path --- meaning that it would
be easier to understand and maintain.  Another large benefit of this approach is that code
implementing the top-level \chpl{domain} type need not be expanded for each concrete type
supported.  The sizes of the AST and generated code would shrink accordingly.  

Example AST given below assumes that the full concrete type of the domain has been
computed by the compiler without specifying the method used.   Any AST elements related to
that type computation are omitted. It also assumes that the
existing ``flat'' representation is used.  However, for clarity, we assume that the actual
construction of the object has been factored --- specifically, the call to the
convertRuntimeTypeToValue is used only to establish the type of the variable; construction
of the object is performed through a separate call.

In that case, the Chapel code:
\begin{chapel}
  var D:domain(2);
  var E = {1..2, 3..4};
  var F:domain(2) = {6..7, 8..9};
  var G:domain(2) = noinit;
\end{chapel}
\noindent
will give rise to the following AST:
\begin{numberedchapel}
  var D:domain(2,int(64),false)
  (call _defaultOf D) // Calls domain(2,int(64),false)._defaultOf()

  var E:domain(2,int(64),false)
  var tmp1:range(int(64),bounded,false)
  var tmp2:range(int(64),bounded,false)
  (call _construct__range_int64_t_bounded_F tmp1 1 2)
  (call _construct__range_int64_t_bounded_F tmp2 3 4)
  (call _construct__domain_2_int64_t_F F tmp1 tmp2) // Calls domain(2,int(64),false)(tmp0)

  var F:domain(2,int(64),false)
  var tmp1:range(int(64),bounded,false)
  var tmp2:range(int(64),bounded,false)
  (call _construct__range_int64_t_bounded_F tmp1 6 7)
  (call _construct__range_int64_t_bounded_F tmp2 8 9)
  (call _construct__domain_2_int64_t_F F tmp1 tmp2) // Calls domain(2,int(64),false)(tmp0)

  var G:domain(2,int(64),false)
  (call _construct__domain_2_int64_t_F G) // Calls domain(2,int(64),false)(noinit=true)
\end{numberedchapel}
\noindent
For the domain labelled \chpl{D}, the type constructor supplies default values for the
\chpl{idxType} and \chpl{stridable} type parameters.  These are \chpl{int(64)} and
\chpl{false}, respectively.  It then calls the default (i.e. zero-args) constructor
through \chpl{_defaultOf()}.

There is no difference between the initialization code for variables \chpl{E} and
\chpl{F}.  The internal difference is that the compiler infers the full type of \chpl{E} from
its initializer, whereas for \chpl{F} the rank is provided explicitly: the compiler only
has to infer the \chpl{idxType} and \chpl{stridable} type parameters.  The \chpl{idxType}
it infers from the initializer; for \chpl{stridable}, it uses the default value
(\chpl{false}).

Initialization for \chpl{G} passes the \chpl{noinit} flag to a constructor that can
receive it (most likely the ``zero-args'' constructor).

The compiler-generated implementation of \chpl{_defaultOf()} for domain types is expected
to look like:
\begin{chapel}
  proc domain(2,int(64),false)._defaultOf() {
    this.(); // Call the zero-args constructor.
  }
\end{chapel}
The default constructor could be written explicitly as:
\begin{chapel}
  ctor domain(2,int(64),false)(param noinit=false) {
  }
\end{chapel}
\noindent
Since the \sntx{initializer-clause} is empty, this basically means ``use the default value
for every field''.  For this domain type, it ends up setting the two ranges to empty
ranges and setting the \chpl{dist} field to \chpl{nil}.  Note that the \chpl{noinit} flag
is ignored.  This means that the domain is initialized in the same way whether or not the
\chpl{noinit} initializer is supplied.  Also, it means that the \chpl{pragma "noinit"}
need not be specified on concrete domain implementations.

At least one of the constructors for a rectangular domain would expand to:
\begin{chapel}
  ctor domain(2,int(64),false)(r1:range, r2:range) {
    with (ranges = (r1,r2))
  }
\end{chapel}
\noindent
which means that the supplied range arguments should be used to replace the ranges in the
domain structure.  Since it is not mentioned in the \sntx{initializer-clause}, the
\chpl{dist} field is set to \chpl{nil}.

When these definitions are inlined in the above code and scalar replacement applied, the
resulting AST is expected to look like:
\begin{numberedchapel}
  var D:domain(2,int(64),false)
  var tmp0:range(int(64),bounded,false)
  (call _defaultOf tmp0)
  ('.=' D ranges.x0 tmp0)
  var tmp1:range(int(64),bounded,false)
  (call _defaultOf tmp1)
  ('.=' D ranges.x1 tmp1)
  ('.=' D dist nil)

  var E:domain(2,int(64),false)
  var tmp1:range(int(64),bounded,false)
  var tmp2:range(int(64),bounded,false)
  (call _construct__range_int64_t_bounded_F tmp1 1 2)
  (call _construct__range_int64_t_bounded_F tmp2 3 4)
  ('.=' E ranges.x0 tmp1)
  var tmp1:range(int(64),bounded,false)
  ('.=' E ranges.x1 tmp2)
  ('.=' E dist nil)
  
  var F:domain(2,int(64),false)
  var tmp1:range(int(64),bounded,false)
  var tmp2:range(int(64),bounded,false)
  (call _construct__range_int64_t_bounded_F tmp1 5 6)
  (call _construct__range_int64_t_bounded_F tmp2 7 8)
  ('.=' F ranges.x0 tmp1)
  var tmp1:range(int(64),bounded,false)
  ('.=' F ranges.x1 tmp2)
  ('.=' F dist nil)
  
  var G:domain(2,int(64),false)
  var tmp0:range(int(64),bounded,false)
  (call _defaultOf tmp0)
  ('.=' G ranges.x0 tmp0)
  var tmp1:range(int(64),bounded,false)
  (call _defaultOf tmp1)
  ('.=' G ranges.x1 tmp1)
  ('.=' G dist nil)
\end{numberedchapel}
An interesting observation here is that the AST still forces us to create an unnamed range
literal and then copy it into the appropriate field in the domain.  If the AST instead
supported the creation of a field reference, then fields in a structure could be updated
in-place.  This could be used in some places to avoid bitwise copies (such as that
performed by the \chpl{PRIM_SET_FIELD} primitive.  In that case, for example,
initialization of the first range on lines 2-4 above could be replaced by:
\begin{chapelcode}
  var tmp0:ref 2*range(int(64),bounded,false)
  ('move' tmp0 ('.' D ranges)) // Here '.' means extract field reference.
  var tmp1:ref range(int(64),bounded,false)
  ('move' tmp1 ('.' tmp0 x0))
  (call _defaultOf tmp1)
\end{chapelcode}


\subsection{Sync and Single Variables}

For the Chapel code:
\begin{chapel}
  var a: sync int;
  var b: sync int = noinit;
  var c: sync int = n;

  var d: single int;
  var e: single int = noinit;
  var f: single int = n;
\end{chapel}
\noindent
the proposed implementation would give the following AST:
\begin{numberedchapel}
  var a:_syncvar(int(64))
  (call _defaultOf a) // Calls _syncvar(int(64))._defaultOf().

  var b:_syncvar(int(64))
  (call _construct__syncvar_int64_t b) // Calls _syncvar(int(64))(noinit=true).

  var c:_syncvar(int(64))
  (call _construct__syncvar_int64_t c n) // Calls _syncvar(int(64))(i:int).

  var d:_singlevar(int(64))
  (call _defaultOf a) // Calls _singlevar(int(64))._defaultOf().

  var e:_singlevar(int(64))
  (call _construct__singlevar_int64_t b) // Calls _singlevar(int(64))(noinit=true).

  var f:_singlevar(int(64))
  (call _construct__singlevar_int64_t c n) // Calls _singlevar(int(64))(i:int).
\end{numberedchapel}
\noindent
This rendition assumes that the \chpl{_syncvar} and \chpl{_singlevar} types have been
rewritten as records to take advantage of automatic allocation and deallocation within the
compiler.  Blank intent for those types would have to be adjusted to \chpl{ref} to
preserve the existing semantics.  A concrete \chpl{ref} intent makes no attempt to read 
the value stored in the sync var (e.g. using \chpl{readFE}).  The AST for the current
class-based implementation would include explicit allocation for each of the variables and the
assumption that they are properly deallocated at the end of the scope in which they are declared.
Mention of a class variable in the AST includes an implicit dereferencing, consistent with
current behavior.

The compiler-generated \chpl{_defaultOf()} for the sync type would look like:
\begin{chapel}
  proc _syncvar(int(64))._defaultOf() {
    this.(); // Call the zero-args constructor
  }
\end{chapel}
\noindent
which serves just fine.  The code for the constructor that can accept zero arguments might
look like:
\begin{chapel}
  ctor _syncvar(?t).(param noinit=false, val=t._defaultOf()) {
    with (value=val)
    __primitive("sync_init", this);
  }
\end{chapel}
The current implementation uses the special pragma ``ignore noinit'', which is no longer
needed because the noinit flag is simply ignored.

When constructors are implemented to the point of supporting \sntx{initializer-clause}s,
the pragma ``omit from constructor'' will also be unnecessary because the all-fields
constructor would not always be called from an explicit constructor as it is today.
Instead, field-initialization would be covered by the semantics of the
\sntx{initializer-clause}.  The all-fields constructor would be called only if invoked
explicitly from user code (impossible if any user-defined constructor is present), or to
implement the semantics of \chpl{_defaultOf()}.

The implementation for single variable would, of course, parallel the implementation for
sync vars.  When the \chpl{_defaultOf()} and the constructor calls are inlined, the above as would
simplify to:
\begin{numberedchapel}
  var a:_syncvar(int(64))
  ('.=' a val 0)
  ('sync_init' a)

  var b:_syncvar(int(64))
  ('.=' b val 0)
  ('sync_init' b)

  var c:_syncvar(int(64))
  ('.=' c val n)
  ('sync_init' c)

  var d:_singlevar(int(64))
  ('.=' d val 0)
  ('single_init' d)

  var e:_singlevar(int(64))
  ('.=' e val 0)
  ('single_init' e)

  var f:_singlevar(int(64))
  ('.=' f val n)
  ('single_init' f)
\end{numberedchapel}


\subsection{Tuple Variables}

For the Chapel code:
\begin{chapel}
  var a:2*real;
  var b:2*real = noinit;
  var d:2*real = (12.0,13.3);
  var e = d;
\end{chapel}
\noindent
the proposed implementation would give rise the the following AST:
\begin{numberedchapel}
  var a:2*real(64)
  (call _defaultOf a) // Calls 2*real(64)._defaultOf().

  var b:2*real(64)
  (call _construct_2_real64_t b) // Calls 2*real(64).2*real(64)(noinit=true).

  var d:2*real(64)
  (call _construct_2_real64_t d 12.0 13.3)

  var e:2*real(64)
  (call _construct_2_real64_t e d) // Calls the copy-constructor.
\end{numberedchapel}
The compiler-generated \chpl{_defaultOf()} for a 2-tuple of reals would look like:
\begin{chapel}
  proc 2*real(64)._defaultOf() {
    this.(); // Call the zero-args constructor (see below).
  }
\end{chapel}
\noindent
This raises a bit of a design issue: The compiler-generated version of \chpl{_defaultOf()}
wants to call a constructor as a method on \chpl{this}, but we don't really have syntax
for that.  I normally expect \sntx{`new'~type-name} on a record type to produce an
anonymous object, so adding a \chpl{`new'} keyword is not the right choice.  I chose to
omit the type name, but supplying it explicitly is equally valid.  

The problem is that calling a constructor on an object that has already been initialized
(as we expect all objects to be) is normally illegal.  Even going so far as to supply
syntax for calling a constructor on an existing object seems like bending the rules.  On
the other hand, we know (special case) that the operand of a \chpl{_defaultOf()} method is
uninitialized, so at least in that context the constructor call is semantically correct.

The syntax problem is created by the existence of \chpl{_defaultOf()}, which is really
just a synonym for a call to the zero-args constructor.  After more of the constructor
story is in place, we can consider deprecating \chpl{_defaultOf()} as an approach to
avoiding the syntax problem.  In the mean time, it is acceptable to treat the body of
\chpl{_defaultOf()} as a special case, and allow constructor calls (only on \chpl{this})
in that syntactical context only.

The code for the compiler-generated zero-args constructor (supporting the noinit
initializer) would look like:
\begin{chapel}
  ctor 2*real(64)(param noinit = false) {
    if noinit then { }
    else { this.x0 = 0.0; this.x1 = 0.0; }
  }
\end{chapel}

The code for the compiler-generated all-fields constructor (called on line 8) would look
like:
\begin{chapel}
  ctor 2*real(64)(a0:real(64), a1:real(64)) {
    this.x0 = a0; this.x1 = a1;
  }
\end{chapel}
After inlining and scalar substitution, the AST would reduce to:
\begin{numberedchapel}
  var a:2*real(64)
  ('.=' a x0 0.0)
  ('.=' a x1 0.0)

  var b:2*real(64)

  var d:2*real(64)
  ('.=' a x0 12.0)
  ('.=' a x1 13.3)

  var e:2*real(64)
  ('.=' a x0 ('.' d x0))
  ('.=' a x1 ('.' d x1))
\end{numberedchapel}


\subsection{Scalar Variables}

For scalar types, we would expect declarations of the form:
\begin{chapel}
// type T = int(64);
   var a:T;
   var b:T = noinit;
   var c:T = new T("42");
   var d = 13; // Initializer expression e has type T.
   var e = d; // Same.
\end{chapel}
\noindent
to produce the following AST (after resolution):
\begin{numberedchapel}
    var a:int(64)
    (call _defaultOf a) // Calls int(64)._defaultOf().

    var b:int(64)
    (call _construct_int64_t b) // Calls b._construct_int64_t(noinit=true).

    var c:int(64)
    var tmp:int(64)
    (call _construct_int64_t tmp "42") // Calls tmp._construct_int64_t(s:c_string).
    (call _construct_int64_t c tmp) // Calls c._construct_int64_t(i:int(64)).
    // === OR === //
    (call _construct_int64_t c "42") // Calls c._construct_int64_t(s:c_string) in situ.

    var d:int(64)
    (call _construct_int64_t d 13) // Calls the copy-constructor.

    var e:int(64)
    (call _construct_int64_t e d) // Calls the copy-constructor.
\end{numberedchapel}
This looks a lot more complex than one would expect, but that is for consistency with more
complex types.  Scalar types are really a special case because they exhibit value
semantics and have trivial constructors, destructors and assignment operators.  When that
is true, copy-construction is equivalent to assignment is equivalent to a move operation.

The definition of the various constructors provides the semantics you would expect.  On
line 2, the \chpl{_defaultOf()} method on a 64-bit integer replaces the LHS with a zero of
that size.  If \chpl{=noinit} is a valid initializer for an object of type \chpl{int(64)},
then at least one of its constructors will accept a \chpl{param bool} to control its
actions.  For example, the default- and copy-constructor for int(64) could be covered by:
\begin{chapel}
  ctor int(64)(val = 0, param noinit = false) {
    if noinit then { }
    else { this = val; }
  }
\end{chapel}
\noindent
The constructor does not contain a field-initializer list, because the int(64) type has no
sub-fields.

The promotion constructor from type \chpl{c_string} to type \chpl{int(64)} does not
currently exist, but would presumably do something logical,
like converting a text representation of an integer into the internal representation of an
integer and then overwriting the receiver with that value, for example:
\begin{chapel}
  ctor int(64)(s:c_string) {
    this = __primitive("convert_c_string_to_int64", s);
  }
\end{chapel}
\noindent
The first form (lines 9-10) would be without the optimization that collapses out
unnecessary copy-constructors: the second form (line 12) would be with that optimization
applied.  The last two cases are simple copy-constructor calls.

After the compiler has inlined the calls to \chpl{_defaultOf()} and the various
constructors, and replaced scalar assignment with move (or assign) primitives, the above
AST would devolve to:
\begin{numberedchapel}
    var a:int(64)
    ('move' a 0)

    var b:int(64)

    var c:int(64)
    ('move' c ('cast' 0x42)) // Cast from a char literal to int(64).

    var d:int(64)
    ('move' d 13)

    var e:int(64)
    ('move' e d)
\end{numberedchapel}

