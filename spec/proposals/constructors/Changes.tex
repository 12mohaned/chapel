\section{List of Proposed Changes}
\label{Changes}

This appendix contains a list of the proposed changes to the Chapel specification,
gleaned from the body of the proposal.  This is followed by a prioritized list of tasks to
be pursued in their implementation.

\subsection{Proposed Implementation Changes}
Here is a synopsis of the proposed changes, in approximate priority order.
\begin{itemize}[leftmargin=2.5cm]
\item[\bf Use Copy-Constructors] -- in place of assignment in initialization.  This is a UMM requirement.
\item[\bf Rename initCopy] -- For consistency, mostly.
\item[\bf Simplify defaultOf] -- Rework the implementation of \chpl{_defaultOf()} so it takes
  precedence over the user-defined default constructor (and calls it unless overridden).
\item[\bf Unify Initialization] -- Reduce the initialization logic to a single conditional:
  Use the explicit initializer if present; otherwise, call \chpl{_defaultOf()}.
\item[\bf Separate Allocation] -- Pull allocation out of class constructors and get rid of the
  \chpl{meme} argument passed to the all-fields constructor.
\item[\bf Constructors as Methods] -- To support in-place initialization.
\item[\bf Add Initializer Clause] -- To control field initialization in the context of a
  constructor.
\item[\bf Remove Zero-Initialization] -- Assuming the rest of the model is implemented as
  specified, this is just dead code.
\item[\bf Collapse Copy-Constructors] -- Where the argument is a new-expression containing
  an object of the same type.  This is just an optimization.
\item[\bf Move RTTV to Modules] -- Compiler creation of runtime type values is hidden and hard
  to understand.  This would move those functions out to module code verbatim, to aid
  maintainability.
\item[\bf Normalize Type Inference] -- Separate type inference in initializations and function
  return value construction from the creation of initialization values.  Type information
  would be carried through to resolution rather than being squashed out of the AST during
  normalization.  Advantages: Simplify type inference and separate type inference from
  initialization.  Disadvantages: Major implementation change.
\item[\bf Add ctor keyword] -- and optional naming, to better support generic
  programming.
\item[\bf Field Extraction] -- Some AST could be simplified if a field-extraction primitive
  were supported.
\end{itemize}

\subsection{Proposed Changes to the Specification}
Proposed changes to the specification are listed below:
\begin{itemize}[itemsep=0pt]
\item Review initialization story w.r.t. fundamental types to ensure that it is accurate
  under the new mode.
\item Add description of the initializer list to Classes and Records chapters.
\item Describe visibility of the all-fields constructor (hidden if any user-defined
  constructors are present.
\item Revise the description of initialization of Classes and Records in terms of a call to
  either \chpl{_defaultOf()} or the copy-constructor.
\item Describe in an implementor's note that copy-construction can be collapsed out when
  the initializer expression is a constructor call.
\item Review the description of allocation w.r.t. class object construction, and revise if
  necessary.  It should allow the implementation to pull allocation outside of class
  constructors (including the all-fields constructor).
\item Review the initialization story for Classes and Records and ensure that it does not
  require default- or zero-initialization to be applied before field initialization.
\item Deprecate the \chpl{initialize} method.
\end{itemize}

\subsection{Open Questions}

\begin{itemize}[itemsep=0pt]
\item Should we keep the \chpl{with} keyword and parentheses?
\item Provide syntax for unnamed constructor introduced with a qualifier.
\item Syntactic sugar for piping constructor args through to like-named fields.
\item Add syntax to explicitly take field default (just the field name).
\end{itemize}


\newpage
\subsection{Implementation Priorities}
This section presents the implementation tasks in tabular form.

\begin{tabular}{|r|c|c|l|}
\hline
Task & Need & Cost & Explanation \\ \hline\hline
Use Copy-Constructors & High & Low & Required for UMM. Change AST generation and chase \\
 & & & down and eliminate internal dependencies. \\ \hline
Rename initCopy & Med & Med & More consistent object model increases usability, chances \\
 & & & for adoption.  Cost here is in understanding the distinction between \\
 & & & initCopy and autoCopy and coming up with a constructor model that \\
 & & & accommodates both.  Plus the usual internal dependencies.  Additional\\
 & & & benefit is removing special-case code (pragma "init copy fn", etc.). \\ \hline
Simplify defaultOf & Med & Low & Rewrite \chpl{_defaultOf()} implementation to \\
 & & & give it priority over the user-defined zero-args constructor \\
 & & & for consistency and simplicity in the code.\\ \hline
Unify Initialization & Med & Low & Mostly in place and a clean-up task suggested to \\
 & & & Lydia.  Enables the removal of zero-initialization. \\ \hline
Separate Allocation & Med & Low & Means getting rid of the \chpl{meme} argument. \\
 & & & Makes all constructors uniform and enables the \\
 & & & Constructors as Methods task. \\ \hline
Constructors as Methods & Med & High & More consistent object model. Simplifies \\
 & & & initialization. Supports in-place initialization (performance). \\
 & & & Major change in how initialization is done. \\
 & & & May depend on normalized type inference. \\ \hline
Add Initializer Clause & High & Med & Supports UMM, object model, ref fields in \\
 & & & objects. New code, so relatively easy compared to rework. \\
 & & & Waiting until after constructors are methods will save having \\
 & & & to revise this part. Guaranteed initialization in the new \\
 & & & model depends on it, but can be worked around in the current impl. \\ \hline
Remove Zero-Initialization & Low & Low & Compiler perf. Redundant zero-init should be \\
 & & & removed by CVP, but the extra code bloats the AST, slows the compiler. \\ \hline
Collapse Copy-Constructors & Low & Med & Optimization. Faster construction of some complex \\
 & & & types e.g. arrays. Can let need drive the priority on this one. \\ \hline
Move RTTV to Modules & Med & Low & Maintainability. Current init path for arrays/domains \\
 & & & is obscured by compiler generation of the RTTV function. \\
 & & & Exposing in module code will simplify both compiler and impl. of \\
 & & & those types, make impl of optimizations and noinit simpler. \\ \hline
Normalize Type Inference & High & High & Reverse very bad design decision. Complicates \\
 & & & much of initialization and resolution. Big change, but probably \\
 & & & smaller than Constructors as Methods. Low priority due to risks \\
 & & & involved, but may be bumped based on need. \\ \hline
Field Extraction & Low & Low & Enable further optimization of the AST using in-place \\
 & & & updates of record fields. \\ \hline
Add \chpl{ctor} Keyword & Low & Med & Better support for generic programming.  Low current need. \\ \hline
\end{tabular}

