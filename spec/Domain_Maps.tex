\sekshun{Domain Maps}
\label{Domain_Maps}
\index{domain maps}

\index{mapped!domain maps}
A domain map specifies the implementation of the domains and arrays
that are \emph{mapped} using it. That is, it defines how domain indices
and array elements are mapped to locales, how they are stored in
memory, and how operations such as accesses, iteration, and slicing
are performed.  Each domain and array is mapped using some domain map.

\index{layouts (see also domain maps, layouts)}
\index{distributions (see also domain maps, distributions)}
\index{domain maps!layout}
\index{domain maps!distribution}
A domain map is either a \emph{layout} or a \emph{distribution}.
A layout describes domains and arrays that exist on a single locale,
whereas a distribution describes domains and arrays that are
partitioned across multiple locales.

A domain map is represented in the program with an instance of
a \emph{domain map class}.
Chapel provides a set of standard domain map classes.
Users can create domain map classes as well.

Domain maps are presented as follows:
\begin{itemize}

\item domain maps for domain types \rsec{Domain_Maps_For_Types},
      domain values \rsec{Domain_Maps_For_Values}, and
      arrays \rsec{Domain_Maps_For_Arrays}

\item domain maps are not retained upon domain assignment
      \rsec{Domain_Maps_Not_Assigned}

\item standard layouts and distributions, such as Block and Cyclic,
are documented under \emph{Standard Library}
in Cray Chapel online documentation here:
\\ %formatting
\mbox{$$ $$ $$} %indent
% Should this be "craychapel"?
\url{https://chapel-lang.org/docs/modules/layoutdist.html}

\item specification of user-defined domain maps is forthcoming;
please refer to the \emph{Domain Map Standard Interface} page
under \emph{Technical Notes}
in Cray Chapel online documentation here:
\\ %formatting
\mbox{$$ $$ $$} %indent
% Should this be "craychapel"?
\url{https://chapel-lang.org/docs/technotes/dsi.html}

\end{itemize}


\section{Domain Maps for Domain Types}
\label{Domain_Maps_For_Types}
\index{domain maps for domain types}
\index{types!domains!domain maps for}

Each domain type has a domain map associated with it.
This domain map is used to map all domain values of this type
(\rsec{Domain_Maps_For_Values}).

If a domain type does not have a domain map specified for it
explicitly as described below,
a default domain map is provided by the Chapel implementation.
Such a domain map will typically be a layout that maps the entire domain
to the locale on which the domain value is created or
the domain or array variable is declared.
% or: "the locale on which the current task is running, i.e., 'here'"

\begin{craychapel}
The default domain map provided by the Cray Chapel compiler
is such a layout. The storage for the representation of a domain's
index set is placed on the locale where the domain variable is declared.
The storage for the elements of arrays declared over domains with
the default map is placed on the locale where the array variable
is declared.
Arrays declared over rectangular domains with this default map
are laid out in memory in row-major order.
\end{craychapel}

\pagebreak
\index{dmap value}
\index{dmapped clause}
\index{domain maps!dmap value}
\index{domain maps!dmapped clause}
A domain map can be specified explicitly by
providing a \emph{dmap value} in a \chpl{dmapped} clause:

\begin{syntax}
mapped-domain-type:
  domain-type `dmapped' dmap-value

dmap-value:
  expression
\end{syntax}

A dmap value consists of an instance of a domain map class
wrapped in an instance of the predefined record \chpl{dmap}.
The domain map class is chosen and instantiated by the user.
% The above sentence strive to emphasize that here the users need to make a
% choice according to their needs, vs. 'dmap' which is prescribed by the lang.
\chpl{dmap} behaves like a generic record with a single generic field,
which holds the domain map instance.

\begin{example}
The code
\begin{chapel}
use BlockDist;
var MyBlockDist: dmap(Block(rank=2));
\end{chapel}
declares a variable capable of storing dmap values
for a two-dimensional Block distribution.
The Block distribution is described in more detail
in the standard library documentation. See:
\\ %formatting
\mbox{$$ $$ $$} %indent
% Should this be "craychapel"?
\url{https://chapel-lang.org/docs/modules/dists/BlockDist.html}
\end{example}

\begin{example}
The code
\begin{chapel}
use BlockDist;
var MyBlockDist: dmap(Block(rank=2)) = new dmap(new Block({1..5,1..6}));
\end{chapel}
creates a dmap value wrapping a two-dimensional Block distribution with a
bounding box of \chpl{\{1..5, 1..6\}} over all of the locales.
\end{example}

\begin{example}
The code
\begin{chapel}
use BlockDist;
var MyBlockDist = new dmap(new Block({1..5,1..6}));
type MyBlockedDom = domain(2) dmapped MyBlockDist;
\end{chapel}
defines a two-dimensional rectangular domain type
that is mapped using a Block distribution.
\end{example}

The following syntactic sugar is provided within the \chpl{dmapped} clause.
If a \chpl{dmapped} clause starts with the name of a domain map class,
it is considered to be a constructor expression as if preceded by
\chpl{new}. The resulting domain map instance is wrapped in a newly-created
instance of \chpl{dmap} implicitly.

\begin{example}
The code
\begin{chapel}
use BlockDist;
type BlockDom = domain(2) dmapped Block({1..5,1..6});
\end{chapel}
is equivalent to
\begin{chapel}
use BlockDist;
type BlockDom = domain(2) dmapped new dmap(new Block({1..5,1..6}));
\end{chapel}
\end{example}


\section{Domain Maps for Domain Values}
\label{Domain_Maps_For_Values}
\index{domain maps!for domain values}
\index{values!domains!domain maps for}

A domain value is always mapped using the domain map of that value's type.
The type inferred for a domain literal (\rsec{Rectangular_Domain_Values})
has a default domain map.

\begin{example}
In the following code
\begin{chapel}
use BlockDist;
var MyDomLiteral = {1..2,1..3};
var MyBlockedDom: domain(2) dmapped Block({1..5,1..6}) = MyDomLiteral;
\end{chapel}
\chpl{MyDomLiteral} is given the inferred type of the domain literal
and so will be mapped using a default map.
MyBlockedDom is given a type explicitly, in accordance to which
it will be mapped using a Block distribution.
\end{example}

A domain value's map can be changed explicitly with a \chpl{dmapped} clause,
in the same way as a domain type's map.

\begin{syntax}
mapped-domain-expression:
  domain-expression `dmapped' dmap-value
\end{syntax}

\begin{example}
In the following code
\begin{chapel}
use BlockDist;
var MyBlockedDomLiteral1 = {1..2,1..3} dmapped new dmap(new Block({1..5,1..6}));
var MyBlockedDomLiteral2 = {1..2,1..3} dmapped Block({1..5,1..6});
\end{chapel}
both \chpl{MyBlockedDomLiteral1} and \chpl{MyBlockedDomLiteral2}
will be mapped using a Block distribution.
\end{example}


\section{Domain Maps for Arrays}
\label{Domain_Maps_For_Arrays}
\index{domain maps!for arrays}
\index{arrays!domain maps}

Each array is mapped using the domain map of the domain
over which the array was declared.

\begin{example}
In the code
\begin{chapel}
use BlockDist;
var Dom: domain(2) dmapped Block({1..5,1..6}) = {1..5,1..6};
var MyArray: [Dom] real;
\end{chapel}
the domain map used for \chpl{MyArray} is the Block
distribution from the type of \chpl{Dom}.
\end{example}

\section{Domain Maps Are Not Retained upon Domain Assignment}
\label{Domain_Maps_Not_Assigned}
\index{domain maps!domain assignment}
\index{domains!assignment}
\index{assignment!domain}

Domain assignment (\rsec{Domain_Assignment}) transfers only the index
set of the right-hand side expression. The implementation of the
left-hand side domain expression, including its domain map, is
determined by its type and so does not change upon a domain assignment.

\begin{example}
In the code
\begin{chapel}
use BlockDist;
var Dom1: domain(2) dmapped Block({1..5,1..6}) = {1..5,1..6};
var Dom2: domain(2) = Dom1;
\end{chapel}
\chpl{Dom2} is mapped using the default distribution, despite
\chpl{Dom1} having a Block distribution.
\end{example}

\begin{example}
In the code
\begin{chapel}
use BlockDist;
var Dom1: domain(2) dmapped Block({1..5,1..6}) = {1..5,1..6};
var Dom2 = Dom1;
\end{chapel}
\chpl{Dom2} is mapped using the same distribution as \chpl{Dom1}.
This is because the declaration of \chpl{Dom2} lacks an explicit
type specifier and so its type is defined to be the type of its
initialization expression, \chpl{Dom1}. So in this situation
the effect is that the domain map does transfer upon
an initializing assignment.
\end{example}
