\sekshun{Classes}
\label{Classes}

Classes are an abstraction of a data structures.where the storage
location is allocated independent of the scope of the variable of
class type.  Each call to the constructor creates a new data object
and returns a reference to the object.  Storage is reclaimed
automatically as described in~\rsec{Automatic_Memory_Management}.

\subsection{Class Declarations}
\label{Class_Declarations}

A class is defined with the following syntax:
\begin{syntax}
class-declaration-statement:
  `class' identifier inherit-type-list[OPT] {
    class-statement-list }

inherit-expression-list:
  class-type
  class-type , inherit-expression-list

class-statement-list:
  class-statement
  class-statement class-statement-list

class-statement:
  type-declaration-statement
  function-declaration-statement
  variable-declaration-statement
\end{syntax}
A \sntx{class-declaration-statement} defines a new type symbol
specified by the identifier.  Classes inherit data and functionality
from other classes if the \sntx{inherit-type-list} is specified.
Inheritance is described in~\rsec{Inheritance}.

The body of a class declaration consists of a sequence of statements
where each of the statements either defines a variable, called a
field, a function, called a method, or a type.

If a class contains a type alias or a parameter, the class is generic.
Generic classes are described in~\rsec{Generics}.

\subsubsection{Class Fields}
\label{Class_Fields}

Variables and constants declared within class declarations define
fields within that class.  (Parameters make a class generic.)  Fields
define the storage associated with a class.

\begin{example}
The code
\begin{chapel}
class Actor {
  var name: string;
  var age: uint;
}
\end{chapel}
defines a new class type called \chpl{Actor} that has two fields: the
string field \chpl{name} and the unsigned integer field \chpl{age}.
\end{example}

\subsubsection{Class Field Accesses}
\label{Class_Field_Accesses}

The field in a class is accessed via a member access expression as
described in~\rsec{Member_Access_Expressions}.  Fields in a class can
be modified via an assignment statement where the left-hand side of
the assignment is a member access.
\begin{example}
Given a variable \chpl{anActor} of type \chpl{Actor}, defined above,
the code
\begin{chapel}
var s: string = anActor.name;
anActor.age = 27;
\end{chapel}
reads the field \chpl{name} and assigns the value to the variable
\chpl{s}, and assigns the storage location in the object
\chpl{anActor} associated with the field \chpl{age} the value
\chpl{27}.
\end{example}

\subsection{Class Assignment}
\label{Class_Assignment}

Classes are assigned by reference.  After an assignment from one
variable of class type to another, the variables reference the same
storage location.

\subsection{Class Constructors}
\label{Class_Constructors}

This section is forthcoming.

\subsubsection{The Default Constructor}
\label{The_Default_Constructor}

This section is forthcoming.

\subsubsection{User-Defined Constructors}
\label{User-Defined_Constructors}

This section is forthcoming.

\subsubsection{Ambiguities in Constructor Calls}
\label{Ambiguities_in_Constructor_Calls}

This section is forthcoming.

\subsection{Class Methods}
\label{Class_Methods}

This section is forthcoming.

\subsubsection{Class Method Declarations}
\label{Class_Method_Declarations}

Methods are functions that are bound to a class.  They are declared
with the following syntax:
\begin{syntax}
method-declaration-statement:
  `def' type-binding function-name argument-list[OPT] var-clause[OPT]
    return-type[OPT] where-clause[OPT] block-level-statement

type-binding:
  identifier .
\end{syntax}

If a method is declared within the lexical scope of a class, record,
or union, the type binding can be omitted and is taken to be the
innermost class, record, or union that the method is defined in.

\subsubsection{The {\em this} Reference}
\label{The_em_this_Reference}

This section is forthcoming.

\subsubsection{Class Method Calls}
\label{Class_Method_Calls}

This section is forthcoming.

\subsubsection{Class Methods without Parentheses}
\label{Class_Methods_without_Parentheses}

This section is forthcoming.

\subsubsection{The {\em this} Method}
\label{The_em_this_Method}

This section is forthcoming.

\subsection{Getters and Setters}
\label{Getters_and_Setters}

This section is forthcoming.

\subsubsection{Default Getters and Setters}
\label{Default_Getters_and_Setters}

This section is forthcoming.

\subsubsection{User-Defined Getters and Setters}
\label{User-Defined_Getters_and_Setters}

This section is forthcoming.

\subsection{Inheritance}
\label{Inheritance}

This section is forthcoming.

\subsubsection{Derived Class Definition}
\label{Derived_Class_Definition}

This section is forthcoming.

\subsubsection{Accessing Base Class Fields}
\label{Accessing_Base_Class_Fields}

This section is forthcoming.

\subsubsection{Derived Class Constructors}
\label{Derived_Class_Constructors}

This section is forthcoming.

\subsubsection{Shadowing Base Class Fields}
\label{Shadowing_Base_Class_Fields}

This section is forthcoming.

\subsubsection{Overriding Base Class Methods}
\label{Overriding_Base_Class_Methods}

This section is forthcoming.

\subsubsection{Inheriting from Multiple Classes}
\label{Inheriting_from_Multiple_Classes}

This section is forthcoming.

\subsection{Class Promotion of Scalar Functions}
\label{Scalar Promotion}

\begin{implementation}
In the current implementation, a class can be defined to promote
scalar functions by defining an iterator in the class
named \chpl{this} and specifying a return type.  The return type
indicates the type that the class promotes.  The body of
the \chpl{this} iterator is ignored.  The class must also implement
the iterator interface as described in~\rsec{Iterator_Interface}.
\end{implementation}

This section is forthcoming.

\subsection{Nested Classes}
\label{Nested_Classes}

\begin{implementation}
Nested classes are not yet supported.
\end{implementation}

This section is forthcoming.

\subsection{Automatic Memory Management}
\label{Automatic_Memory_Management}

This section is forthcoming.
