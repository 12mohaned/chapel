\sekshun{Classes}
\label{Classes}
\index{class@\chpl{class}}
\index{classes}

Classes are an abstraction of a data structure where the storage
location is allocated independent of the scope of the variable of
class type.  Each call to the constructor creates a new data object
and returns a reference to the object.  Storage is reclaimed
automatically as described in~\rsec{Automatic_Memory_Management}.

\subsection{Class Declarations}
\label{Class_Declarations}
\index{classes!declarations}

A class is defined with the following syntax:
\begin{syntax}
class-declaration-statement:
  `class' identifier class-inherit-type-list[OPT] {
    class-statement-list }

class-inherit-expression-list:
  class-type
  class-type , inherit-expression-list

class-statement-list:
  class-statement
  class-statement class-statement-list

class-statement:
  type-declaration-statement
  function-declaration-statement
  variable-declaration-statement
\end{syntax}
A \sntx{class-declaration-statement} defines a new type symbol
specified by the identifier.  Classes inherit data and functionality
from other classes if the \sntx{inherit-type-list} is specified.
Inheritance is described in~\rsec{Inheritance}.

The body of a class declaration consists of a sequence of statements
where each of the statements either defines a variable (called a
field), a function (called a method), or a type.

If a class contains a type alias or a parameter, the class is generic.
Generic classes are described in~\rsec{Generics}.

\subsection{Class Assignment}
\label{Class_Assignment}
\index{classes!assignment}

Classes are assigned by reference.  After an assignment from one
variable of class type to another, the variables reference the same
storage location.

\subsection{Class Fields}
\label{Class_Fields}
\index{classes!fields}

Variables and constants declared within class declarations define
fields within that class.  (Parameters make a class generic.)  Fields
define the storage associated with a class.

\begin{example}
The code
\begin{chapel}
class Actor {
  var name: string;
  var age: uint;
}
\end{chapel}
defines a new class type called \chpl{Actor} that has two fields: the
string field \chpl{name} and the unsigned integer field \chpl{age}.
\end{example}

\subsubsection{Class Field Accesses}
\label{Class_Field_Accesses}
\index{member access}

The field in a class is accessed via a member access expression as
described in~\rsec{Member_Access_Expressions}.  Fields in a class can
be modified via an assignment statement where the left-hand side of
the assignment is a member access.
\begin{example}
Given a variable \chpl{anActor} of type \chpl{Actor}, defined above,
the code
\begin{chapel}
var s: string = anActor.name;
anActor.age = 27;
\end{chapel}
reads the field \chpl{name} and assigns the value to the variable
\chpl{s}, and assigns the storage location in the object
\chpl{anActor} associated with the field \chpl{age} the value
\chpl{27}.
\end{example}

\subsection{Class Methods}
\label{Class_Methods}
\index{classes!methods}

A method is a function or iterator that is bound to a class.  A method
is called by passing an instance of the class to the method via a
special syntax that is similar to a field access.

\subsubsection{Class Method Declarations}
\label{Class_Method_Declarations}

Methods are declared with the following syntax:
\begin{syntax}
method-declaration-statement:
  `def' type-binding function-name argument-list[OPT] var-clause[OPT]
    return-type[OPT] where-clause[OPT] block-level-statement
  `iterator' type-binding function-name argument-list[OPT] var-clause[OPT]
    return-type[OPT] where-clause[OPT] block-level-statement

type-binding:
  identifier .
\end{syntax}
If a method is declared within the lexical scope of a class, record,
or union, the type binding can be omitted and is taken to be the
innermost class, record, or union that the method is defined in.

\subsubsection{Class Method Calls}
\label{Class_Method_Calls}

A method is called by using the member access syntax as described
in~\rsec{Member_Access_Expressions} where the accessed expression is
the name of the method.

\begin{example}
A method to output information about an instance of the \chpl{Actor}
class can be defined as follows:
\begin{chapel}
def Actor.print() {
  writeln("Actor ", name, " is ", age, " years old");
}
\end{chapel}
This method can be called on an instance of the \chpl{Actor}
class, \chpl{anActor}, by writing \chpl{anActor.print()}.
\end{example}

\subsubsection{The {\em this} Reference}
\label{The_em_this_Reference}
\index{this@\chpl{this}}

The instance of a class is passed to a method using special syntax.
It does not appear in the argument list to the method.  The
reference \chpl{this} is an alias to the instance of the class on
which the method is called.

\begin{example}
Let class \chpl{C}, method \chpl{foo}, and function \chpl{bar} be
defined as
\begin{chapel}
class C {
  def foo() {
    bar(this);
  }
}
def bar(c: C) { }
\end{chapel}
Then given an instance of \chpl{C} called \chpl{c}, the method
call \chpl{c.foo()} results in a call to \chpl{bar} where the argument
is \chpl{c}.
\end{example}

\subsubsection{The {\em this} Method}
\label{The_em_this_Method}
\index{this@\chpl{this}}
\index{classes!indexing}

A method declared with the name \chpl{this} allows a class to be
``indexed'' similarly to how an array is indexed.  Indexing into a
class has the semantics of calling a method on the class
named \chpl{this}.  There is no other way to call a method
called \chpl{this}.  The \chpl{this} method must be declared with
parentheses even if the argument list is empty.

\begin{example}
In the following code, the \chpl{this} method is used to create a
class that acts like a simple array that contains three integers
indexed by 1, 2, and 3.
\begin{chapel}
class ThreeArray {
  var x1, x2, x3: int;
  def this(i: int) var {
    select i {
      when 1 do return x1;
      when 2 do return x2;
      when 3 do return x3;
    }
    halt("ThreeArray index out of bounds: ", i);
  }
}
\end{chapel}
\end{example}

\subsection{Class Constructors}
\label{Class_Constructors}
\index{classes!constructors}

A class constructor is defined by declaring a method with the same
name as the class.  The constructor is used to create instances of the
class.  When the constructor is called, memory is allocated to store a
class instance.

\subsubsection{The Default Constructor}
\label{The_Default_Constructor}

A default constructor is automatically created for every class in the
Chapel program.  This constructor is defined such that it has one
argument for every field in the class.  Each of the arguments has a
default value.

The default constructor is very useful but its generality in terms of
having one argument for each field all of which have default values
makes it slightly difficult for the user to create their own
constructor.  It is expected that in many simple cases, the default
constructor will be all that is necessary.

\begin{example}
Given the class
\begin{chapel}
class C {
  def x: int;
  def y: real = 3.14;
  def z: string = "Hello, World!";
}
\end{chapel}
then instances of the class can be created by calling the default
constructor as follows:
\begin{itemize}
\item The call \chpl{C()} is equivalent to \chpl{C(0,3.14,"Hello, World!")}.
\item The call \chpl{C(2)} is equivalent to \chpl{C(2,3.14,"Hello, World!")}.
\item The call \chpl{C(z="")} is equivalent to \chpl{C(0,3.14,"")}.
\item The call \chpl{C(0,0.0,"")} is equivalent to \chpl{C(0,0.0,"")}.
\end{itemize}
\end{example}

\subsection{Getters and Setters}
\label{Getters_and_Setters}
\index{classes!getters}
\index{classes!setters}

All field accesses are resolved via getter and setter methods that are
defined in the class with the same name as the field.  A setter is
defined as an explicit setter
function~(\rsec{Explicit_Setter_Functions}).  Default getters and
setters are defined that simply access or set the field if the user
does not define their own.

\begin{example}
In the code
\begin{chapel}
class C {
  var x: int;
  def =x(value: int) {
    if value < 0 then
      halt("x assigned negative value");
    x = value;
  }
}
\end{chapel}
a setter is defined for field \chpl{x} that ensures that \chpl{x} is
never assigned a negative value.
\end{example}

\subsection{Inheritance}
\label{Inheritance}
\index{inheritance}
\index{classes!inheritance}

A ``derived'' class can inherit from one or more other classes by
specifying those classes, the base classes, following the name of the
derived class in the declaration of the derived class.  When
inheriting from multiple base classes, only one of the base classes
may contain fields.  The other classes can only define methods.  Note
that a class can still be derived from a class that contains fields
which is itself derived from a class that contains fields.

\subsubsection{Accessing Base Class Fields}
\label{Accessing_Base_Class_Fields}

A derived class contains data associated with the fields in its base
classes.  The fields can be accessed in the same way that they are
accessed in their base class unless the getter or setter method is
overridden in the derived class, as discussed
in~\rsec{Overriding_Base_Class_Methods}.

\subsubsection{Derived Class Constructors}
\label{Derived_Class_Constructors}

Derived class constructors automatically call the default constructor
of the base class.  There is an expectation that a more standard way
of chaining constructor calls will be supported.

\subsubsection{Shadowing Base Class Fields}
\label{Shadowing_Base_Class_Fields}

A field in the derived class can be declared with the same name as a
field in the base class.  Such a field shadows the field in the base
class in that it is always referenced when it is accessed in the
context of the derived class.  There is an expectation that there will
be a way to reference the field in the base class but this is not
defined at this time.

\subsubsection{Overriding Base Class Methods}
\label{Overriding_Base_Class_Methods}
\index{dynamic dispatch}

If a method in a derived class is declared with the identical
signature as a method in a base class, then it is said to override the
base class's method.  Such a method is a candidate for dynamic
dispatch in the event that a variable that has the base class type
references a variable that has the derived class type.

The identical signature requires that the names, types, and order of
the formal arguments be identical.

\subsubsection{Inheriting from Multiple Classes}
\label{Inheriting_from_Multiple_Classes}
\index{multiple inheritance}

\begin{status}
Multiple inheritance is not yet supported.
\end{status}

A class can be derived from multiple base classes provided that only
one of the base classes contains fields either directly or from base
classes that it is derived from.  The methods defined by the other
base classes can be overridden.

\subsection{Nested Classes}
\label{Nested_Classes}
\index{classes!nested}

A class defined within another class is a nested class.

Nested classes can refer to fields and methods in the outer class
implicitly or explicitly with an \chpl{outer} reference.

\subsection{Automatic Memory Management}
\label{Automatic_Memory_Management}
\index{automatic memory management}

\begin{status}
Memory allocated to store class objects is not yet reclaimed.
\end{status}

Memory associated with class instances is reclaimed automatically when
there is no way for the current program to reference this memory.  The
programmer does not need to free the memory associated with class
instances.
