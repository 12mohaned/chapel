\sekshun{Locality and Distribution}
\label{Locality_and_Distribution}

\begin{implementation}
Programs can currently only run on a single locale.  The abstractions
described here are not yet implemented.
\end{implementation}

Chapel provides high-level abstractions that allow programmers to
exploit locality by defining the affinity of data and computation.
This is accomplished by associating both data objects and computations
with abstract {\em locales}. To provide a higher-level mechanism,
Chapel allows a mapping from domains to locales to be specified. This
mapping is called a {\em distribution} and it guides that placement of
variables associated with arrays and the placement of subcomputations
defined over the domain.

\index{local}
\index{remote}
Throughout this section, the term {\em local} refers to data that is
associated with the locale that a computation is running on and {\em
remote} refers to data that is not. We assume that there is some
overhead associated with accessing data that may be remote compared to
data known to be local.

\subsection{Locales}
\label{Locales}
\index{locales}

A locale abstracts a processor or node in a parallel computer system,
or the basic component in the computer system where local memory can
be accessed uniformly.

\subsubsection{The Locale Type}
\label{The_Locale_Type}
\index{locale@\chpl{locale}}

The identifier \chpl{locale} is a primitive type that abstracts a
locale as described above.  Both data and computations can be
associated with a value of locale type. The only operators defined
over locales are the equality and inequality comparison operators.

\subsubsection{Predefined Locales Array}
\label{Predefined_Locales_Array}
\index{Global@\chpl{Global}}
\index{Locales@\chpl{Locales}}
\index{numLocales@\chpl{numLocales}}
\index{execution environment}

A predefined configuration variable defines the {\em execution
environment} for a program.  This environment is defined by the
following definitions:
\begin{chapel}
config const numLocales: int;
const Locales: [1..numLocales] locale;
const Global: locale;
\end{chapel}
The environment consists of constants which are fixed when the program
begins execution.  The variable \chpl{Global} holds a special value
of \chpl{locale} type that can be distinct from the values stored
in \chpl{Locales}. This value is used to denote an object or
computation that has no defined affinity.

When a program starts, a single thread is executing.  It is running on
the locale given by \chpl{Locales(1)}.

\subsubsection{Querying the Locale of a Variable}
\label{Querying_the_Locale_of_a_Variable}

Every variable \chpl{v} is associated with some locale which can be
queried using the following syntax:
\begin{syntax}
locale-access:
  expression . `locale'
\end{syntax}
When the \sntx{expression} is a class type, the locale is where the
object is located rather than where the \sntx{expression} may be
located.

\subsection{Specifying Locales for Computation}
\label{Specifying_Locales_for_Computation}

This is a stub.  This portion of the document does not exist.

\subsubsection{On}
\label{On}

The on statement controls on which locale a computation or data should
be placed.  The syntax of the on statement is given by
\begin{syntax}
on-statement:
  `on' expression `do' statement
  `on' expression block-level-statement
\end{syntax}

\subsubsection{On and Forall Loops}
\label{On_and_Forall_Loops}

This is a stub.  This portion of the document does not exist.

\subsubsection{On and Iterators}
\label{On_and_Iterators}

This is a stub.  This portion of the document does not exist.

\subsection{Distributions}
\label{Distributions}

This is a stub.  This portion of the document does not exist.

\subsubsection{Distributed Domains}
\label{Distributed_Domains}

This is a stub.  This portion of the document does not exist.

\subsubsection{Distributed Arrays}
\label{Distributed_Arrays}

This is a stub.  This portion of the document does not exist.

\subsubsection{Undistributed Domains and Arrays}
\label{Undistributed_Domains_and_Arrays}

This is a stub.  This portion of the document does not exist.

\subsection{Standard Distributions}
\label{Standard_Distributions}

This is a stub.  This portion of the document does not exist.

\subsubsection{Block Distribution}
\label{Block_Distribution}

This is a stub.  This portion of the document does not exist.

\subsubsection{Cyclic Distribution}
\label{Cyclic_Distribution}

This is a stub.  This portion of the document does not exist.

\subsubsection{BlockCyclic Distribution}
\label{BlockCyclic_Distribution}

This is a stub.  This portion of the document does not exist.

\subsubsection{Cut Distribution}
\label{Cut_Distribution}

This is a stub.  This portion of the document does not exist.

\subsection{User-Defined Distributions}
\label{User-Defined_Distributions}

This is a stub.  This portion of the document does not exist.
