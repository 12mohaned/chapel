\sekshun{Generics}
\label{Generics}

Chapel supports generic functions and types that are parameterizable
over both types and parameters.  The generic functions and types look
similar to non-generic functions and types already discussed.

\subsection{Generic Functions}
\label{Generic_Functions}
\index{functions!generic}
\index{generics!functions}

A function is generic if any of the following conditions hold:
\begin{itemize}
\item
Some formal argument is specified with an intent of \chpl{type} or
\chpl{param}.
\item
Some formal argument has no specified type and no default value.
\item
Some formal argument is specified with a queried type.
\item
The type of some formal argument is a generic type, e.g., \chpl{List}.
Queries may be inlined in generic types, e.g., \chpl{List(?eltType)}.
\item
The type of some formal argument is an array type where either the
element type is queried or omitted or the domain is queried or
omitted.
\end{itemize}
These conditions are discussed in the next sections.

\subsubsection{Formal Type Arguments}
\label{Formal_Type_Arguments}
\index{intents!type@\chpl{type}}

If a formal argument is specified with intent \chpl{type}, then a type
must be passed to the function at the call site.  A copy of the
function is instantiated for each unique type that is passed to this
function at a call site.  The formal argument has the semantics of a
type alias.
\begin{example}
The following code defines a function that takes two types at the call
site and returns a 2-tuple where the types of the components of the
tuple are defined by the two type arguments and the values are
specified by the types default values.
\begin{chapelpre}
% build2tuple.chpl
\end{chapelpre}
\begin{chapel}
def build2Tuple(type t, type tt) {
  var x1: t;
  var x2: tt;
  return (x1, x2);
}
\end{chapel}
This function is instantiated with ``normal'' function call syntax
where the arguments are types:
\begin{chapel}
var t2 = build2Tuple(int, string);
t2 = (1, "hello");
\end{chapel}
\begin{chapelpost}
writeln(t2);
\end{chapelpost}
\begin{chapeloutput}
(1, hello)
\end{chapeloutput}
\end{example}

\subsubsection{Formal Parameter Arguments}
\label{Formal_Parameter_Arguments}
\index{intents!param@\chpl{param}}

If a formal argument is specified with intent \chpl{param}, then a
parameter must be passed to the function at the call site.  A copy of
the function is instantiated for each unique parameter that is passed
to this function at a call site.  The formal argument is a parameter.
\begin{example}
The following code defines a function that takes an integer parameter
\chpl{p} at the call site as well as a regular actual argument of
integer type \chpl{x}.  The function returns a homogeneous tuple of
size \chpl{p} where each component in the tuple has the value of
\chpl{x}.
\begin{chapelpre}
% fillTuple.chpl
\end{chapelpre}
\begin{chapel}
def fillTuple(param p: int, x: int) {
  var result: p*int;
  for param i in 1..p do
    result(i) = x;
  return result;
}
\end{chapel}
\begin{chapelpost}
writeln(fillTuple(3,3));
\end{chapelpost}
\begin{chapeloutput}
(3, 3, 3)
\end{chapeloutput}
The function call \chpl{fillTuple(3, 3)} returns a 3-tuple where each
component contains the value \chpl{3}.
\end{example}

\subsubsection{Formal Arguments without Types}
\label{Formal_Arguments_without_Types}
\index{formal arguments!without types}

If the type of a formal argument is omitted, the type of the formal
argument is taken to be the type of the actual argument passed to the
function at the call site.  A copy of the function is instantiated for
each unique actual type.
\begin{example}
The example from the previous section can be extended to be generic on
a parameter as well as the actual argument that is passed to it by
omitting the type of the formal argument \chpl{x}.  The following code
defines a function that returns a homogeneous tuple of size \chpl{p}
where each component in the tuple is initialized to \chpl{x}:
\begin{chapelpre}
% fillTuple2.chpl
\end{chapelpre}
\begin{chapel}
def fillTuple(param p: int, x) {
  var result: p*x.type;
  for param i in 1..p do
    result(i) = x;
  return result;
}
\end{chapel}
\begin{chapelpost}
var x = fillTuple(3, 3.14);
writeln(x);
writeln(typeToString(x.type));
\end{chapelpost}
\begin{chapeloutput}
(3.14, 3.14, 3.14)
3*real
\end{chapeloutput}
In this function, the type of the tuple is taken to be the type of the
actual argument.  The call \chpl{fillTuple(3, 3.14)} returns a 3-tuple
of real values \chpl{(3.14, 3.14, 3.14)}.  The return type is
\chpl{(real, real, real)}.
\end{example}

\subsubsection{Formal Arguments with Queried Types}
\label{Formal_Arguments_with_Queried_Types}
\index{formal arguments!queried types}

If the type of a formal argument is specified as a queried type, the
type of the formal argument is taken to be the type of the actual
argument passed to the function at the call site.  A copy of the
function is instantiated for each unique actual type.  The queried
type has the semantics of a type alias.
\begin{example}
The example from the previous section can be rewritten to use a
queried type for clarity:
\begin{chapelpre}
% fillTuple3.chpl
\end{chapelpre}
\begin{chapel}
def fillTuple(param p: int, x: ?t) {
  var result: p*t;
  for param i in 1..p do
    result(i) = x;
  return result;
}
\end{chapel}
\begin{chapelpost}
var x = fillTuple(3, 3.14);
writeln(x);
writeln(typeToString(x.type));
\end{chapelpost}
\begin{chapeloutput}
(3.14, 3.14, 3.14)
3*real
\end{chapeloutput}
\end{example}

\subsubsection{Formal Arguments of Generic Type}
\label{Formal_Arguments_of_Generic_Type}
\index{formal arguments!generic types}

If the type of a formal argument is a generic type, the type of the
formal argument is taken to be the type of the actual argument passed
to the function at the call site with the constraint that the type of
the actual argument is an instantiation of the generic type.  A copy
of the function is instantiated for each unique actual type.
\begin{example}
The following code defines a function \chpl{writeTop} that takes an
actual argument that is a generic stack
(see~\rsec{Example_Generic_Stack}) and outputs the top element of the
stack.  The function is generic on the type of its argument.
\begin{chapel}
def writeTop(s: Stack) {
  write(s.top.item);
}
\end{chapel}
\end{example}

Types and parameters may be queried from the top-level types of formal
arguments as well.  In the example above, the formal argument's type
could also be specified as \chpl{Stack(?type)} in which case the
symbol \chpl{type} is equivalent to \chpl{s.itemType}.

Note that generic types which have default values for all of their
generic fields, \emph{e.g. range}, are not generic when simply
specified and require a query to mark the argument as generic.  For
simplicity, the identifier may be omitted.
\begin{example}
The following code defines a class with a type field that has a
default value.  Function \chpl{f} is defined to take an argument of
this class type where the type field is instantiated to the default.
Function \chpl{g}, on the other hand, is generic on its argument
because of the use of the question mark.
\begin{chapel}
class C {
  type t = int;
}
def f(c: C) {
  // c.type is always int
}
def g(c: C(?)) {
  // c.type may not be int
}
\end{chapel}
\end{example}

\index{where@\chpl{where}!implicit}
The generic type may be specified with some queries and some exact
values.  Thesse exact values result in \emph{implicit where clauses}
for the purpose of function resolution.
\begin{example}
Given the class definition
\begin{chapel}
class C {
  type t;
  type tt;
}
\end{chapel}
then the function definition
\begin{chapel}
def f(c: C(?t,real)) {
  // body
}
\end{chapel}
is equivalent to
\begin{chapel}
def f(c: C(?t,?tt)) where tt == real {
  // body
}
\end{chapel}
\end{example}
For tuples with query arguments, an implicit where clause is always
created to ensure that the size of the actual tuple matches the
implicitly specified size of the formal tuple.
\begin{example}
The function definition
\begin{chapel}
def f(tuple: (?t,real)) {
  // body
}
\end{chapel}
is equivalent to
\begin{chapel}
def f(tuple: (?t,?tt)) where tuple.size == 2 && tt == real {
  // body
}
\end{chapel}
\end{example}

\index{integral@\chpl{integral}}
\index{numeric@\chpl{numeric}}
\index{enumerated@\chpl{enumerated}}
The generic types \chpl{integral}, \chpl{numeric} and \chpl{enumerated}
are generic types that can only be instantiated with, respectively, the
signed and unsigned integral types, all of the numeric types, and
enumerated types.

\subsubsection{Formal Arguments of Generic Array Types}
\label{Formal_Arguments_of_Generic_Array_Types}
\index{formal arguments!array types}

If the type of a formal argument is an array where either the domain
or the element type is queried or omitted, the type of the formal
argument is taken to be the type of the actual argument passed to the
function at the call site.  If the domain is omitted, the domain of
the formal argument is taken to be the domain of the actual argument.

A queried domain may not be modified via the name to which it is bound
(see~\rsec{Association_of_Arrays_to_Domains} for rationale).

\subsection{Function Visibility in Generic Functions}
\label{Function_Visibility_in_Generic_Functions}
\index{generics!function visibility}

Function visibility in generic functions is altered depending on the
instantiation.  When resolving function calls made within generic
functions, the visible functions are taken from any call site at which
the generic function is instantiated for each particular
instantiation.  The specific call site chosen is arbitrary and it is
referred to as the \emph{point of instantiation}.

For function calls that specify the module
explicitly~(\rsec{Explicit_Naming}), an implicit use of the specified
module exists at the call site.

\begin{example}
Consider the following code which defines a generic
function \chpl{bar}:
\begin{chapelpre}
% point_of_instantiation.chpl
\end{chapelpre}
\begin{chapel}
module M1 {
  record R {
    var x: int;
    def foo() { }
  }
}

module M2 {
  def bar(x) {
    x.foo();
  }
}

module M3 {
  use M1, M2;
  def main() {
    var r: R;
    bar(r);
  }
}
\end{chapel}
\begin{chapeloutput}
\end{chapeloutput}
In the function \chpl{main}, the variable \chpl{r} is declared to be
of type \chpl{R} defined in module \chpl{M1} and a call is made to the
generic function \chpl{bar} which is defined in module \chpl{M2}.
This is the only place where \chpl{bar} is called in this program and
so it becomes the point of instantiation for \chpl{bar} when the
argument \chpl{x} is of type \chpl{R}.  Therefore, the call to
the \chpl{foo} method in \chpl{bar} is resolved by looking for visible
functions from within \chpl{main} and going through the use of
module \chpl{M1}.
\end{example}

If the generic function is only called indirectly through dynamic
dispatch, the point of instantiation is defined as the point at which
the derived type (the type of the implicit \chpl{this} argument) is
defined or instantiated (if the derived type is generic).

\begin{rationale}
Visible function lookup in Chapel's generic functions is handled
differently than in C++'s template functions in that there is no split
between dependent and independent types.

Also, dynamic dispatch and instantiation is handled differently.
Chapel supports dynamic dispatch over methods that are generic in some
of its formal arguments.

Note that the Chapel lookup mechanism is still under development and
discussion.  Comments or questions are appreciated.
\end{rationale}

\subsection{Generic Types}
\label{Generic_Types}
\index{generics!types}
\index{types!generic}
\index{generics!classes}
\index{classes!generic}
\index{generics!records}
\index{records!generic}

Generic types are generic classes and generic records.
A class or record is generic if it contains one or more
\index{generics!fields}
\index{fields!generic}
generic fields. A generic field is one of:
\begin{itemize}
\item a specified or unspecified type alias,
\item a parameter field, or
\item a \chpl{var} or \chpl{const} field that has no type and no initialization
expression.
\end{itemize}

\mbox{} % push the following line to the next page

For each generic field, the class or record is parameterized over:
\begin{itemize}
\item the type bound to the type alias,
\item the value of the parameter field, or
\item the type of the \chpl{var} or \chpl{const} field, respectively.
\end{itemize}
Correspondingly, the class or record is instantiated with a set
of types and parameter values, one type or value per generic field.

% Here are the aspects to be defined for each kind of generic field:
% - what it makes the class/record generic over
% - the type constructor arg that gets created
% - the default constructor arg that gets created
% - the requirements on the corresponding user-defined constructor arg
% - for each of the above args:
%    - what kind of actual it accepts (type, param, value)
%    - what is the semantics;
%      i.e. how it corresponds to the class/record's genericity
%    - what is the arg's default, if any
% 
% In the presentation below, some of these aspects are discussed
% in the field-kind-specific subsections, some in the constructor-specific
% subsections, some in both.  I.e. there is an overlap between
% field-kind and constructor subsections; that should be OK but feel free
% to clean up.
% 
% It would be cool to summarize that in a table
% (one dimension: field kinds; the other dimension: aspects).

\subsubsection{Type Aliases in Generic Types}
\label{Type_Aliases_in_Generic_Types}
\index{type aliases!in classes or records}
\index{fields!type alias}

If a class or record defines a type alias, the class or record
is generic over the type that is bound to that alias.
% Type aliases defined in a class or a record can be unspecified type
% aliases; type aliases that are not bound to a type.  If a class or
% record contains an unspecified type alias, the aliased type must be
% specified whenever the type is used.
Such a type alias is accessed as if it were a field;
similar to a parameter field, it cannot be assigned
except in its declaration.

The type alias becomes an argument with intent \chpl{type} to
the default constructor (\rsec{Generic_Default_Constructors})
for its class or record. This makes the default constructor generic.
The type alias also becomes an argument with intent \chpl{type} to
the type constructor (\rsec{Type_Constructors}).
If the type alias declaration binds it to a type, that type
becomes the default for these arguments, otherwise they have no defaults.

The class or record is instantiated by binding the type alias
to the actual type passed to the corresponding argument of
a user-defined (\rsec{Generic_User_Constructors})
or default constructor or type constructor.
If that argument has a default, the actual type can be omitted, in
which case the default will be used instead.

\begin{example}
The following code defines a class called \chpl{Node} that implements
a linked list data structure.  It is generic over the type of the
element contained in the linked list.
\begin{chapelpre}
% NodeClass.chpl
\end{chapelpre}
\begin{chapel}
class Node {
  type eltType;
  var data: eltType;
  var next: Node(eltType);
}
\end{chapel}
\begin{chapelpost}
var n: Node(real) = new Node(real, 3.14);
writeln(n.data);
writeln(n.next);
writeln(typeToString(n.next.type));
\end{chapelpost}
\begin{chapeloutput}
3.14
nil
Node(real)
\end{chapeloutput}
The call \chpl{new Node(real, 3.14)} creates a node in the linked list
that contains the value \chpl{3.14}.  The \chpl{next} field is set to
nil.  The type specifier \chpl{Node} is a generic type and cannot be
used to define a variable.  The type specifier \chpl{Node(real)}
denotes the type of the \chpl{Node} class instantiated over
\chpl{real}.  Note that the type of the \chpl{next} field is specified
as \chpl{Node(eltType)}; the type of \chpl{next} is the same type as
the type of the object that it is a field of.
\end{example}

\subsubsection{Parameters in Generic Types}
\label{Parameters_in_Generic_Types}
\index{parameters!in classes or records}
\index{fields!parameter}

If a class or record defines a parameter field, the class or record
is generic over the value that is bound to that field.
% A parameter defined in a class or record is accessed as if it were a
% field.  This access returns a parameter.  
The parameter becomes an argument with intent \chpl{param} to the
default constructor (\rsec{Generic_Default_Constructors})
for that class or record.  This makes the default
constructor generic.  The parameter also becomes an argument
with intent \chpl{param} to the type  constructor (\rsec{Type_Constructors}).
If the parameter declaration has an initialization expression, that expression
becomes the default for these arguments, otherwise they have no defaults.

The class or record is instantiated by binding the parameter
to the actual value passed to the corresponding argument of
a user-defined (\rsec{Generic_User_Constructors})
or default constructor or type constructor.
If that argument has a default, the actual value can be omitted, in
which case the default will be used instead.

\begin{example}
The following code defines a class called \chpl{IntegerTuple} that is
generic over an integer parameter which defines the number of
components in the class.
\begin{chapelpre}
% IntegerTuple.chpl
\end{chapelpre}
\begin{chapel}
class IntegerTuple {
  param size: int;
  var data: size*int;
}
\end{chapel}
\begin{chapelpost}
var x = new IntegerTuple(3);
writeln(x.data);
\end{chapelpost}
\begin{chapeloutput}
(0, 0, 0)
\end{chapeloutput}
The call \chpl{new IntegerTuple(3)} creates an instance of the
\chpl{IntegerTuple} class that is instantiated over parameter
\chpl{3}.  The field \chpl{data} becomes a 3-tuple of integers.  The
type of this class instance is \chpl{IntegerTuple(3)}.  The type
specified by \chpl{IntegerTuple} is a generic type.
\end{example}

\subsubsection{Fields without Types}
\label{Fields_without_Types}
\index{fields!variable and constant, without types}
\index{variables!in classes or records}
\index{constants!in classes or records}

If a \chpl{var} or \chpl{const} field in a class or record has no specified type or
initialization expression, the class or record is generic over the
type of that field.  The field becomes an argument with blank intent to
the default constructor (\rsec{Generic_Default_Constructors}).
That argument has no specified type and no default
value. This makes the default constructor generic.
The field also becomes an argument with \chpl{type} intent and no default
to the type constructor (\rsec{Type_Constructors}).
Correspondingly, an actual value must always be passed to the default
constructor argument and an actual type to the type constructor argument.

The class or record is instantiated by binding the type of the field
to the type of the value passed to the corresonding argument
of a user-defined (\rsec{Generic_User_Constructors}) or default constructor.
When the type constructor is invoked, the class or record is instantiated
by binding the type of the field to the actual type passed to
the corresponding argument.

\begin{example}
The following code defines another class called \chpl{Node} that
implements a linked list data structure.  It is generic over the type
of the element contained in the linked list.  This code does not
specify the element type directly in the class as a type alias but
rather omits the type from the \chpl{data} field.
\begin{chapelpre}
% fieldWithoutType.chpl
\end{chapelpre}
\begin{chapel}
class Node {
  var data;
  var next: Node(data.type) = nil;
}
\end{chapel}
A node with integer element type can be defined in the call to the
constructor.  The call \chpl{new Node(1)} defines a node with the
value \chpl{1}.  The code
\begin{chapel}
var list = new Node(1);
list.next = new Node(2);
\end{chapel}
\begin{chapelpost}
writeln(list.data);
writeln(list.next.data);
\end{chapelpost}
\begin{chapeloutput}
1
2
\end{chapeloutput}
defines a two-element list with nodes containing the values \chpl{1}
and \chpl{2}.  The type of each object could be specified
as \chpl{Node(int)}.
\end{example}

\subsubsection{The Type Constructor}
\label{Type_Constructors}
\index{generics!type constructor}
\index{constructors!type constructors}

A type constructor is automatically created for each class or record.
A type constructor is a type function (\rsec{Type_Functions}) that has
the same name as the class or record.  It takes one argument per the
class's or record's generic field, including fields inherited from the
superclasses, if any.
The formal argument has intent \chpl{type} for a type alias field and is a
parameter for a parameter field. It accepts the type to be bound
to the type alias and the value to be bound to the parameter, respectively.
For a generic \chpl{var} or \chpl{const} field, the corresponding
formal argument also has intent \chpl{type}. It accepts the type
of the field, as opposed to a value as is the case for a parameter field.
The formal arguments occur in the same order as the fields are
declared and have the same names as the corresponding fields.
Unlike the default constructor, the type constructor has only
those arguments that correspond to generic fields.

A call to a type constructor accepts actual types and parameter values
and returns the type of the class or record that is instantiated
appropriately for each field
(\rsec{Type_Aliases_in_Generic_Types}, \rsec{Parameters_in_Generic_Types},
\rsec{Fields_without_Types}).
\index{generics!instantiated type}
Such an instantiated type must be used as the type of a variable,
array element, non-generic formal argument, and in other cases
where uninstantiated generic class or record types are not allowed.

When a generic field declaration has an initialization expression
or a type alias is specified, that initializer becomes the default value
for the corresponding type constructor argument.  Uninitialized
fields, including all generic \chpl{var} and \chpl{const} fields,
and unspecified type aliases result in arguments with no defaults;
actual types or values for these arguments must always be provided
when invoking the type constructor.

\subsubsection{Generic Methods}
\label{Generic_Methods}
\index{generics!methods}

All methods bound to generic classes or records, including
constructors, are generic over the implicit \chpl{this} argument.
This is in addition to being generic over any other argument that is generic.

\subsubsection{The Default Constructor}
\label{Generic_Default_Constructors}
\index{generics!constructors, default}
\index{constructors!default!for generic classes or records}

The default constructor for a class or record (\rsec{The_Default_Constructor})
is generic over each argument that corresponds to a generic field, as
specified above.
The argument has intent \chpl{type} for a type alias field and is a
parameter for a parameter field. It accepts the type to be bound
to the type alias and the value to be bound to the parameter, respectively.
This is the same as for the type constructor.
For a generic \chpl{var} or \chpl{const} field, the corresponding
formal argument has the blank intent and accepts the value
for the field to be initialized with. The type of the field
is inferred automatically to be the type of the initialization value.

The default values for the generic arguments of the default constructor
are the same as for the type constructor (\rsec{Type_Constructors}).
For example, the arguments corresponding to the generic \chpl{var}
and \chpl{const} fields, if any, never have defaults, so the corresponding
actual values must always be provided.

\subsubsection{User-Defined Constructors}
\label{Generic_User_Constructors}
\index{generics!constructors, user-defined}
\index{constructors!user-defined!for generic classes or records}

If a generic field of a class does not have an initialization expression
or a type alias is unspecified, each user-defined constructor for that
class must provide a formal argument whose name
matches the name of the field.

If the name of a formal argument in a user-defined constructor matches the name
of a generic field that does not have an initialization
expression, is a type alias, or is a parameter field, the field is
automatically initialized at the beginning of the constructor invocation
to the actual value of that argument.
This is done by passing that formal argument to the implicit invocation
of the default constructor at the start of the user-defined constructor
(\rsec{User_Defined_Constructors}).

%%  The following story is nicer but it's not how it is implemented:
%If the name of a formal argument in a class constructor
%matches the name of a generic field, the field is automatically initialized
%to the actual value for that argument upon the constructor invocation.
%If the generic field does not have an initialization expression,
%such a matching formal argument must be provided in each constructor
%for that class.

\begin{example}
In the following code:
\begin{chapelpre}
% constructorsForGenericFields.chpl
\end{chapelpre}
\begin{chapel}
class MyGenericClass {
  type t1;
  param p1;
  const c1;
  var v1;
  var x1: t1; // this field is not generic

  type t5 = real;
  param p5 = "a string";
  const c5 = 5.5;
  var v5 = 555;
  var x5: t5; // this field is not generic

  def MyGenericClass(c1, v1, type t1, param p1) { }
  def MyGenericClass(type t5, param p5, c5, v5, x5,
                     type t1, param p1, c1, v1, x1) { }
}  // class MyGenericClass

var g1 = new MyGenericClass(11, 111, int, 1);
var g2 = new MyGenericClass(int, "this is g2", 3.3, 333, 3333,
                            real, 2, 222, 222.2, 22);
\end{chapel}
\begin{chapelpost}
writeln(g1);
writeln(g2);
\end{chapelpost}
\begin{chapeloutput}
{p1 = 1, c1 = 11, v1 = 111, x1 = 0, p5 = a string, c5 = 5.5, v5 = 555, x5 = 0.0}
{p1 = 2, c1 = 222, v1 = 222.2, x1 = 0.0, p5 = this is g2, c5 = 5.5, v5 = 555, x5 = 0}
\end{chapeloutput}
The arguments \chpl{t1}, \chpl{p1}, \chpl{c1}, and \chpl{v1} are
required in all constructors for \chpl{MyGenericClass}. They can appear
in any order. Both \chpl{MyGenericClass} constructors initialize the
corresponding fields implicitly because these fields do not have initialization
expressions. The second constructor also initializes implicitly
the fields \chpl{t5} and \chpl{p5} because they are a type field
and a parameter field. It does not initialize the fields \chpl{c5}
and \chpl{v5} because they have initialization expressions, or
the fields \chpl{x1} and \chpl{x5} because they are not generic fields
(even though they are of generic types).
\end{example}

\begin{openissue}
The design of constructors, especially for generic classes, is
under development, so the above specification may change.
\end{openissue}

\subsection{Where Expressions}
\label{Where_Expressions}
\index{where@\chpl{where}}
\index{generics!where@\chpl{where}}

The instantiation of a generic function can be constrained by {\em
where clauses}.  A where clause is specified in the definition of a
function~(\rsec{Function_Definitions}).  When a function is
instantiated, the expression in the where clause must be a parameter
expression and must evaluate to either \chpl{true} or \chpl{false}.
If it evaluates to \chpl{false}, the instantiation is rejected and the
function is not a possible candidate for function resolution.
Otherwise, the function is instantiated.
\begin{example}
Given two overloaded function definitions
\begin{chapelpre}
% whereClause.chpl
\end{chapelpre}
\begin{chapel}
def foo(x) where x.type == int { writeln("int"); }
def foo(x) where x.type == real { writeln("real"); }
\end{chapel}
\begin{chapelpost}
foo(3);
foo(3.14);
\end{chapelpost}
\begin{chapeloutput}
int
real
\end{chapeloutput}
the call foo(3) resolves to the first definition because when the
second function is instantiated the where clause evaluates to false.
\end{example}

\subsection{User-Defined Compiler Diagnostics}
\label{User_Defined_Compiler_Errors}
\index{compiler diagnostics!user-defined}
\index{compiler errors!user-defined}
\index{compiler warnings!user-defined}
\index{compilerError}
\index{compilerWarning}

The special compiler diagnostic function calls \chpl{compilerError}
and \chpl{compilerWarning} generate compiler diagnostic of the
indicated severity if the function containing these calls may be
called when the program is executed and the function call is not
eliminated by parameter folding.

The compiler diagnostic is defined by the actual arguments which must
be string parameters.  The diagnostic points to the spot in the Chapel
program from which the function containing the call is called.
Compilation halts if a \chpl{compilerError} is encountered whereas it
will continue after encountering a \chpl{compilerWarning}.

Note that when a variable function is called in a context where the
implicit \chpl{setter} argument is true or false, both versions of the
variable function are resolved by the compiler.  Consequently,
the \chpl{setter} argument cannot be effectively used to guard a
compiler diagnostic statements.

\begin{example}
The following code shows an example of using user-defined compiler
diagnostics to generate warnings and errors:
\begin{chapelpre}
% compilerDiagnostics.chpl
\end{chapelpre}
\begin{chapel}
def foo(x, y) {
  if (x.type != y.type) then
    compilerError("foo() called with non-matching types: ", 
                  typeToString(x.type), " != ", typeToString(y.type));
  writeln("In 2-argument foo...");
}

def foo(x) {
  compilerWarning("1-argument version of foo called");
  writeln("In generic foo!");
}
\end{chapel}
\begin{chapelpost}
foo(3.4);
foo("hi");
foo(1, 2);
foo(1.2, 3.4);
foo("hi", "bye");
\end{chapelpost}
\begin{chapeloutput}
compilerDiagnostics.chpl:12: warning: 1-argument version of foo called
compilerDiagnostics.chpl:13: warning: 1-argument version of foo called
In generic foo!
In generic foo!
In 2-argument foo...
In 2-argument foo...
In 2-argument foo...
\end{chapeloutput}

The first routine generates a compiler error whenever the compiler
encounters a call to it where the two arguments have different types.
It prints out an error message indicating the types of the arguments.
The second routine generates a compiler warning whenver the compiler
encounters a call to it.

Thus, if the program foo.chpl contained the following calls:

\begin{numberedchapel}
foo(3.4);
foo("hi");
foo(1, 2);
foo(1.2, 3.4);
foo("hi", "bye");
foo(1, 2.3);
foo("hi", 2.3);
\end{numberedchapel}

\noindent compiling the program would generate output like:

\begin{commandline}
foo.chpl:1: warning: 1-argument version of foo called with type: real
foo.chpl:2: warning: 1-argument version of foo called with type: string
foo.chpl:6: error: foo() called with non-matching types: int != real
\end{commandline}

\end{example}

\subsection{Example: A Generic Stack}
\label{Example_Generic_Stack}
\begin{chapelpre}
% genericStack.chpl
\end{chapelpre}
\begin{chapel}
class MyNode {
  type itemType;              // type of item
  var item: itemType;         // item in node
  var next: MyNode(itemType); // reference to next node (same type)
}

record Stack {
  type itemType;             // type of items
  var top: MyNode(itemType); // top node on stack linked list

  def push(item: itemType) {
    top = new MyNode(itemType, item, top);
  }

  def pop() {
    if isEmpty then
      halt("attempt to pop an item off an empty stack");
    var oldTop = top;
    top = top.next;
    return oldTop.item;
  }

  def isEmpty return top == nil;
}
\end{chapel}
\begin{chapelpost}
var s: Stack(int);
s.push(1);
s.push(2);
s.push(3);
while !s.isEmpty do
  writeln(s.pop());
\end{chapelpost}
\begin{chapeloutput}
3
2
1
\end{chapeloutput}
