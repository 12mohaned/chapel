\sekshun{Standard Distributions}
\label{Standard_Distributions}

The following table lists distributions standard to the Chapel
language:
\begin{center}
\begin{tabular}{|l|l|l|}
\hline
{\bf Distribution} & {\bf Module} & {\bf Supported Domain Types} \\
\hline
\chpl{Block} & \chpl{BlockDist} & Arithmetic \\
\chpl{Cyclic} & \chpl{CyclicDist} & Arithmetic \\
\hline
\end{tabular}
\end{center}

\begin{rationale}
Why supply any standard distributions?  A main design goal of Chapel
requires that the standard distributions be defined using the same
mechanisms available to Chapel programmers wishing to define their own
distributions or layouts~(\rsec{User_Defined_Domain_Maps}).  That way
there shouldn't be a necessary performance cost associated with
user-defined domain maps.  Nevertheless, distributions are an integral
part of the Chapel language which would feel incomplete without a good
set of standard distributions.  It is hoped that many distributions
will begin as user-defined domain maps and later become part of the
standard set of distributions.
\end{rationale}

\subsection{The Standard Block Distribution}
\label{Block_Dist}

The standard Block distribution is defined in the module called
\chpl{BlockDist}.  It is parameterized by the rank and index type of
the domains that it can support.  Thus domains of different ranks or
different index types must be distributed with different Block
distributions.

The Block class constructor is defined as follows:
\begin{chapel}
def Block(boundingBox: domain,
          targetLocales: [] locale = Locales, 
          dataParTasksPerLocale = $\mbox{{\it value in global config const of the same name}}$,
          dataParIgnoreRunningTasks = $\mbox{{\it value in global config const of the same name}}$,
          dataParMinGranularity = $\mbox{{\it value in global config const of the same name}}$,
          param rank = boundingBox.rank,
          type idxType = int)
\end{chapel}

The argument \chpl{boundingBox} is a non-distributed domain defining a
bounding box used to partition the space of all indices across an
array of target locales.  For Block distributions of rank $d$, given
a bounding box
\begin{chapel}
[$l_1$..$h_1$, $\ldots$, $l_d$..$h_d$]
\end{chapel}
and an array of target locales defined over the domain
\begin{chapel}
[$0$..$n_1$, $\ldots$, $0$..$n_d$]
\end{chapel}
then a Block distribution maps an index $i$ to a locale by the
following formula which computes the $k$\emph{th} component of an
index into the array of target locales from the $k$\emph{th} component
of $i$:
\[
\left\{
  \begin{array}{ll}
    0 & \mbox{if $i_k < l_k$} \\
    \left\lfloor\dfrac{n_k (i_k - l_k)}{h_k - l_k + 1}\right\rfloor & \mbox{if $i_k \geq l_k$ and $i_k \leq h_k$} \\
    n_k-1 & \mbox{if $i_k > h_k$} \\
  \end{array}
\right.
\]

The argument \chpl{targetLocales} is a non-distributed array
containing the target locales to which this distribution partitions
indices and data.  The rank of \chpl{targetLocales} must match the
rank of the distribution, or be one.  If the rank of
\chpl{targetLocales} is one, a greedy heuristic is used to reshape the
array of target locales so that it matches the rank of the
distribution and each dimension contains an approximately equal number
of indices.

The arguments \chpl{dataParTasksPerLocale},
\chpl{dataParIgnoreRunningTasks}, and \chpl{dataParMinGranularity} set
the knobs that are used to control intra-locale data parallelism for
Block-distributed domains and arrays in the same way that the global
configuration constants of these names control data parallelism for
ranges and default-distributed domains and
arrays~\rsec{data_parallel_knobs}.

The rank and index type of the Block distribution are inferred from
the \chpl{boundingBox} argument unless explicitly set.

\begin{example}
The following code sets the elements of a Block-distributed array of
integers to the ID of the locale on which each element exists.
%\begin{chapelpre}
%\end{chapelpre}
\begin{chapel}
use BlockDist;

const Space = [1..8, 1..8];
const MyBlock: dmap(Block(rank=2)) = new dmap(new Block(boundingBox=Space));
const D: domain(2) dmapped MyBlock = Space;
var A: [D] int;

forall a in A do
  a = a.locale.id;

writeln(A);
\end{chapel}
The Block distribution \chpl{MyBlock} is specified with an explicit
type that sets the rank of the distribution to two and uses the
default dimensional index type \chpl{int}.  It is initialized by
setting the bounding box to the value of the non-distributed domain
\chpl{Space}.  Note that the named arguments \chpl{rank} and
\chpl{boundingBox} are used only for clarity and may be omitted.

Domains \chpl{D} and \chpl{A} are distributed via \chpl{MyBlock}.  The
computation occurs in the forall loop where each array element is set
to the ID of the locale on which that element exists.  When run on 6
locales, the output of this code is:
\begin{chapel}
0 0 0 0 1 1 1 1
0 0 0 0 1 1 1 1
0 0 0 0 1 1 1 1
2 2 2 2 3 3 3 3
2 2 2 2 3 3 3 3
2 2 2 2 3 3 3 3
4 4 4 4 5 5 5 5
4 4 4 4 5 5 5 5
\end{chapel}
\end{example}

\subsection{The Standard Cyclic Distribution}
\label{Cyclic_Dist}

This section is forthcoming.
