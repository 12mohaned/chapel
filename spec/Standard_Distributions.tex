\sekshun{Standard Distributions}
\label{Standard_Distributions}

Standard distributions include the following:
\begin{itemize}
\item The block distribution \chpl{Block}
\item The cyclic distribution \chpl{Cyclic}
\item The block-cyclic distribution \chpl{BlockCyclic}
\end{itemize}

\begin{openissue}
There is an expectation that the number of standard distributions will
grow dramatically.
\end{openissue}

A design goal is that all standard distributions are defined using the
same mechanisms available to Chapel programmers wishing to define
their own distributions~(\rsec{User_Defined_Domain_Maps}).

\subsubsection{The Standard Block Distribution}
\label{Block_Dist}

The standard Block distribution is defined in the module
called \chpl{BlockDist}.  This module must be explicitly used.

The Block distribution is parameterized by the rank and index type of
the domains for which it will support.  Thus domains of different
ranks or different index types must be distributed with different
distributions.

The Block class constructor is defined as follows:
\begin{chapel}
def Block(param rank: int,
          type idxType = int,
          bbox: domain(rank, idxType),
          targetLocales: [] locale = thisRealm.Locales, 
          tasksPerLocale: int = 0)
\end{chapel}

The argument \chpl{bbox} is a non-distributed domain defining a
bounding box used to partition the space of all indices across the
array of target locales.  The indices inside the bounding box are
partitioned ``evenly'' across the locales and indices outside the
bounding box are mapped to the same locale as the nearest index inside
the bounding box.

The argument \chpl{targetLocales} is a non-distributed array
containing the target locales to which this distribution partitions
indices and data.  The rank of \chpl{targetLocales} must match the
rank of the distribution, or be one.  If the rank
of \chpl{targetLocales} is one, it will be factored and reshaped so
that it matches the rank of the distribution.  A greedy heuristic is
used to ``evenly'' do the factoring.

The argument \chpl{tasksPerLocale} is an integer that specifies the
number of tasks to use on each locale when iterating in parallel over
a Block-distributed domain or array.

\subsubsection{The Standard Cyclic Distribution}
\label{Cyclic_Dist}

This section is forthcoming.

\subsubsection{The Standard Block-Cyclic Distribution}
\label{Block_Cyclic_Dist}

This section is forthcoming.
