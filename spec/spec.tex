\documentclass[10pt,oneside,titlepage]{spec}
\usepackage[T1]{fontenc}
\usepackage{amsmath}
\usepackage{amssymb}
\usepackage{color}
\usepackage{times}
%\usepackage{fullpage}
\usepackage{graphicx}
\usepackage{listings}
\usepackage{longtable}
\usepackage[nottoc]{tocbibind}
\usepackage{multirow}
%
% These are special environments for adding extra information about
% code snippets which can be later extracted and used to generate test
% codes for automated testing.
%
% During LaTeX compilation, the environments defined in this file throw
% away all text within the scope of the environment, with the
% exception of 'chapelprintoutput' which prints the output (and is
% also extracted for testing purposes).
%
% Usage:
%
% - chapelexample (REQUIRED) {f.chpl}
%   This marks the start of a test.  This environment requires a
%   single argument that is the name of the Chapel test program.  This
%   filename will appear in the spec.
%
% - chapelpre
%   Any Chapel code in this scope is put *before* the code in the
%   chapel|chapelcode scope.
%
% - chapelcode|chapel
%   This is the part of the code that is in the spec.
%
% - chapelnoprint
%   This is the part of the code that goes in the test with chapelcode
%   and chapel, but does not appear in the spec.
%   
% - chapelpost
%   Any Chapel code in this scope is put *after* the code in the
%   chapel|chapelcode scope.
%
% - chapelfuture
% - chapelcompopts
% - chapelexecopts
%   The lines in these scopes are put directly into the appropriate file.
%
% - chapeloutput|chapelprintoutput (REQUIRED)
%   These environment provide the test output (.good files).  There can be
%   multiple such environments, and the filename is specified by a LaTeX
%   style comment preceding the contents of the output.  The
%   'chapelprintoutput' scope is also outputted in the spec itself and
%   thus may contain LaTeX formatting (see GENERAL CAVEATS below)
%   
% - chapelwideoutput
%   Provides the test output for no-local tests, if that differs from the
%   normal test output.  The content of this environment is dumped into a
%   <test>.no-local.good file, along with a copy of the content of 
%   chplprintoutput.
%

%
% GENERAL CAVEATS:
%
% - Because the chapelprintoutput environment must used LaTeX
%   formatting, the script that extracts the tests must removed any
%   LaTeX specific formatting.
%
% - Using a backslash or other special LaTeX characters may also be
%   needed (e.g., \_ or \#) in the other environments for LaTex
%   parsing purposes.  Such characters are considered fragile and may
%   lead to unexpected results.
%

%
% Gobble up the text in this new box.  The text in each environment is
% dropped on the floor during LaTeX compilation.
%
\newsavebox{\teststuff}

%
% Any additional lines needed for the code snippet to run/compile
% (before and after the chapel code segment)
%
%\newenvironment{chapelpre} {\begin{lrbox}{\teststuff}
%\begin{minipage}{6in}}
%{\end{minipage}\end{lrbox}}

%\DefineVerbatimEnvironment{chapelpre}{Verbatim}{}
\newenvironment{chapelpre}{BLOCK-test-chapelpre}{}

%\newenvironment{chapelnoprint} {\begin{lrbox}{\teststuff}
%\begin{minipage}{6in}}
%{\end{minipage}\end{lrbox}}

%\DefineVerbatimEnvironment{chapelnoprint}{Verbatim}{}
\newenvironment{chapelnoprint}{BLOCK-test-chapelnoprint}{}

%\newenvironment{chapelpost} {\begin{lrbox}{\teststuff}
%\begin{minipage}{6in}}
%{\end{minipage}\end{lrbox}}

%\DefineVerbatimEnvironment{chapelpost}{Verbatim}{}
\newenvironment{chapelpost}{BLOCK-test-chapelpost}{}


%
% .future file
%
%\newenvironment{chapelfuture} {\begin{lrbox}{\teststuff}
%\begin{minipage}{6in}}
%{\end{minipage}\end{lrbox}}

%\DefineVerbatimEnvironment{chapelfuture}{Verbatim}{}
\newenvironment{chapelfuture}{BLOCK-test-chapelfuture}{}

%
% .compopts file
%
%\newenvironment{chapelcompopts} {\begin{lrbox}{\teststuff}
%\begin{minipage}{6in}}
%{\end{minipage}\end{lrbox}}

%\DefineVerbatimEnvironment{chapelcompopts}{Verbatim}{}
\newenvironment{chapelcompopts}{BLOCK-test-chapelcompopts}{}

%
% .execopts file
%
%\newenvironment{chapelexecopts} {\begin{lrbox}{\teststuff}
%\begin{minipage}{6in}}
%{\end{minipage}\end{lrbox}}

\DefineVerbatimEnvironment{chapelexecopts}{Verbatim}{}
\newenvironment{chapelexecopts}{BLOCK-test-chapelexecopts}{}


%
% .good file
% To get more than one file, use a LaTeX style comment to name the
% .good file
%
%\newenvironment{chapeloutput} {\begin{lrbox}{\teststuff}
%\begin{minipage}{6in}}
%{\end{minipage}\end{lrbox}}

\DefineVerbatimEnvironment{chapeloutput}{Verbatim}{}
\newenvironment{chapeloutput}{BLOCK-test-chapeloutput}{}

%
% .no-local.good file
% (The naming feature mentioned above does not yet work, so this
% environment is a Q&D way to get a .no-local.good file.)
%
%\newenvironment{chapelwideoutput} {\begin{lrbox}{\teststuff}
%\begin{minipage}{6in}}
%{\end{minipage}\end{lrbox}}

%\DefineVerbatimEnvironment{chapelwideoutput}{Verbatim}{}
\newenvironment{chapelwideoutput}{BLOCK-test-chapelwideoutput}{}

%
% .prediff file
%
%\newenvironment{chapelprediff} {\begin{lrbox}{\teststuff}
%\begin{minipage}{6in}}
%{\end{minipage}\end{lrbox}}

%\DefineVerbatimEnvironment{chapelprediff}{Verbatim}{}
\newenvironment{chapelprediff}{BLOCK-test-chapelprediff}{}

%
% .good file that is printed in the text of the Spec
% To get more than one file, use a LaTeX style comment to name the
% .good file
%
%\lstnewenvironment{chapelprintoutput} 
% (See chapel_listing.tex for the implementation.)

%\DefineVerbatimEnvironment{chapeloutputname}{Verbatim}{}
\newenvironment{chapeloutputname}{BLOCK-test-chapeloutputname}{}

\lstdefinelanguage{chapel}
  {
    morekeywords={
      and, array, atomic,
      begin, bool, break,
      call, class, cobegin, complex, config, const, constructor, continue,
      def, distribute, do, domain,
      else, enum, except,
      for, forall,
      goto,
      if, imag, implements, in, int, inout, _invariant, iterator,
      let, like,
      module,
      nil, not,
      on, or, ordered, otherwise, out,
      param, _private, private, public,
      real, record, _release, repeat, return,
      select, serial, single, subtype, sync
      then, to, type, typeselect,
      uint, union, until, _unordered,
      var, _view,
      when, where, while, with,
      yield
    },
    sensitive=false,
    mathescape=false,
    morecomment=[l]{//},
    morecomment=[s]{/*}{*/},
    morestring=[b]",
}

\lstset{
    basicstyle=\footnotesize\tt,
    keywordstyle=\bf,
    commentstyle=\em,
    showstringspaces=false,
    flexiblecolumns=false,
    numbers=left,
    numbersep=5pt,
    numberstyle=\tiny,
    numberblanklines=false,
    stepnumber=0
  }

\newcommand{\chpl}[1]{\lstinline[language=chapel,basicstyle=\normalsize\tt,keywordstyle=]!#1!}

\lstnewenvironment{chapel}{\lstset{language=chapel,xleftmargin=2pc}}{}

\lstdefinelanguage{syntax}
  {
    sensitive=false,
    mathescape=true,
    basewidth = 0.50em,
    fontadjust=true,
    columns=fullflexible,
    basicstyle=\small,
    keywordstyle=\footnotesize\ttfamily,
    commentstyle=\footnotesize\ttfamily,
    identifierstyle=\small\slshape,
    moredelim=[is][\small\bf]{`}{'},
    literate={-}{{\ttfamily -}}{1}
             {[OPT]}{{{\scriptsize $_{opt}$}}}{2}
  }

\lstnewenvironment{syntax}{\lstset{language=syntax,xleftmargin=2pc}}{}

\newcommand{\sntx}[1]{\lstinline[language=syntax]!#1!}


%% High section numbers require different number widths
\usepackage[titles]{tocloft}
\setlength{\cftchapnumwidth}{1.3em} % Wide enough for a chapter number.
\setlength{\cftsecnumwidth}{3.2em}  % Same as cftsubsecnumwidth:
\setlength{\cftsubsecnumwidth}{3.2em} % Wide enough for three digits and two dots.
%\setlength{\cftsubsubsecnumwidth}{5.4em}
\setlength{\cftsecindent}{1.3em}    % cftchapnumwidth
\setlength{\cftsubsecindent}{1.3em} % cftchapnumwidth
\setlength{\cftsubsubsecindent}{4.5em} % cftchapnumwidth + cftsecnumwidth

% This can be re-enabled in order to aid in spec editing/reviewing.
%% bring in todonotes and make sure we have enough space in the margins. Also
%% note, even side margin is changed below since I couldn't figure out how to
%% force notes to the right side.
%\usepackage[paperwidth=9.5in, paperheight=11in]{geometry}
%\setlength{\marginparwidth}{1.8in}
%\usepackage[colorinlistoftodos, textwidth=1.8in, shadow]{todonotes}

\usepackage{ifpdf}
\ifpdf
\usepackage[pdftex,
            bookmarks,
            plainpages=false,
            breaklinks,
            pdftitle={Chapel Language Specification},
            pdfauthor={Cray Inc, 901 Fifth Avenue Suite 1000, Seattle, WA 98164},
            pdfsubject={Chapel High Productivity Language}
           ]{hyperref}
\else
\usepackage[ps2pdf]{hyperref}
\fi

\newcommand{\ie}{\emph{i.e.}}
\newcommand{\eg}{\emph{e.g.}}

\newenvironment{TODO} {
\begin{quote}
{\it TODO:}
}{
\end{quote}
}

\newenvironment{example}{
\begin{quote}
{\it Example}.
}{
\end{quote}
}

\newenvironment{chapelexample}[1]{
\begin{quote}
{\it Example (#1)}.
}{
\end{quote}
}

\newenvironment{note}{
\begin{quote}
{\it Implementors' note}.
}{
\end{quote}
}

\newenvironment{rationale}{
\begin{quote}
{\it Rationale}.
}{
\end{quote}
}

\newenvironment{openissue}{
\begin{quote}
{\it Open issue}.
}{
\end{quote}
}

\newenvironment{future}{
\begin{quote}
{\it Future}.
}{
\end{quote}
}

\newenvironment{craychapel}{
\begin{quote}
{\it Cray's Chapel Implementation}.
}{
\end{quote}
}

\newenvironment{suggestionbox}{
\begin{quote}
{\it Suggestions?}
}{
\end{quote}
}

\newcommand{\rsec}[1]
           {\S\ref{#1}}

% courtesy: http://www.iam.ubc.ca/~newbury/tex/page-set-up.html
\newcommand{\sekshun}[1]
           {
             \chapter{#1}
             \markboth{Chapel Language Specification}{#1}
           }

\oddsidemargin 0.0in
% not needed with onesided version \evensidemargin 0.5in
\textwidth 6.5in
\headheight 0.2in
\topmargin 0in
\headsep 0.3in
\textheight 8.5in

\makeindex
\title{Chapel Language Specification\\Version 0.982}

\author{Cray Inc\\
901 Fifth Avenue, Suite 1000\\
Seattle, WA 98164}

\date{October 6, 2016}

\setcounter{tocdepth}{3}

\begin{document}

\pagestyle{empty}
%\pagenumbering{alph}

\ifpdf
\pdfbookmark[1]{Title}{titlepage}
\fi
\maketitle

\setcounter{page}{2}
\null\vfill
\noindent
\begin{center}
\copyright 2016 Cray Inc.
\end{center}

\cleardoublepage
\include{tm}
\cleardoublepage

\pagestyle{myheadings}
\markboth{Chapel Language Specification}{Chapel Language Specification}
%\pagenumbering{roman}

\ifpdf
\pdfbookmark[1]{Table of Contents}{tablecontents}
\fi
\tableofcontents

\cleardoublepage

\pagestyle{myheadings}
%\pagenumbering{arabic}

\setlength{\parindent}{0in}
\setlength{\parskip}{4mm plus2mm minus1mm}

\sekshun{Scope}
\label{Scope}

Chapel is a new parallel programming language that is under
development at Cray Inc. in the context of the DARPA High Productivity
Language Systems initiative and the DARPA High Productivity Computing
Systems initiative.

This document specifies the Chapel language.  It is a work in progress
and is not definitive.  In particular, it is not a standard.

\cleardoublepage
\sekshun{Notation}
\label{Notation}

Special notations are used in this specification to denote Chapel code
and to denote Chapel syntax.

Chapel code is represented with a fixed-width font where keywords are
bold and comments are italicized.
\begin{example}
\begin{chapel}
for i in D do   // iterate over domain D
  writeln(i);   // output indices in D
\end{chapel}
\end{example}

Chapel syntax is represented with standard syntax notation in which
productions define the syntax of the language.  A production is
defined in terms of non-terminal ({\it italicized}) and terminal
(non-italicized) symbols.  The complete syntax defines all of the
non-terminal symbols in terms of one another and terminal symbols.

A definition of a non-terminal symbol is a multi-line construct.  The
first line shows the name of the non-terminal that is being defined
followed by a colon.  The next lines before an empty line define the
alternative productions to define the non-terminal.
\begin{example}
The production
\begin{syntax_donotcollect}
bool-literal:
  `true'
  `false'
\end{syntax_donotcollect}
defines \sntx{bool-literal} to be either the symbol \sntx{`true'} or
\sntx{`false'}.
\end{example}
In the event that a single line of a definition needs to break across
multiple lines of text, more indentation is used to indicate that it
is a continuation of the same alternative production.

As a short-hand for cases where there are many alternatives that
define one symbol, the first line of the definition of the
non-terminal may be followed by ``one of'' to indicate that the single
line in the production defines alternatives for each symbol.
\begin{example}
The production
\begin{syntax_donotcollect}
unary-operator: one of
  + - ~ !
\end{syntax_donotcollect}
is equivalent to
\begin{syntax_donotcollect}
unary-operator:
  +
  -
  ~
  !
\end{syntax_donotcollect}
\end{example}

As a short-hand to indicate an optional symbol in the definition of a
production, the subscript ``opt'' is suffixed to the symbol.
\begin{example}
The production
\begin{syntax_donotcollect}
formal:
  formal-tag identifier formal-type[OPT] default-expression[OPT]
\end{syntax_donotcollect}
is equivalent to
\begin{syntax_donotcollect}
formal:
  formal-tag identifier formal-type default-expression
  formal-tag identifier formal-type
  formal-tag identifier default-expression
  formal-tag identifier
\end{syntax_donotcollect}
\end{example}

\cleardoublepage
\sekshun{Organization}
\label{Organization}

This specification is organized as follows:
\begin{itemize}

\item
Chapter~\ref{Scope}, Scope, describes the scope of this specification.

\item
Chapter~\ref{Notation}, Notation, introduces the notation that is used
throughout this specification.

\item
Chapter~\ref{Organization}, Organization, describes the contents of
each of the chapters within this specification.

\item
Chapter~\ref{Acknowledgments}, Acknowledgments, offers a note of
thanks to people and projects.

\item
Chapter~\ref{Language_Overview}, Language Overview, describes Chapel
at a high level.

\item
Chapter~\ref{Lexical_Structure}, Lexical Structure, describes the
lexical components of Chapel.

\item
Chapter~\ref{Types}, Types, describes the types in Chapel and defines
the primitive and enumerated types.

\item
Chapter~\ref{Variables}, Variables, describes variables and constants
in Chapel.

\item
Chapter~\ref{Conversions}, Conversions, describes the legal implicit
and explicit conversions allowed between values of different types.
Chapel does not allow for user-defined conversions.

\item
Chapter~\ref{Expressions}, Expressions, describes the non-parallel
expressions in Chapel.

\item
Chapter~\ref{Statements}, Statements, describes the non-parallel
statements in Chapel.

\item
Chapter~\ref{Modules}, Modules, describes modules in Chapel., Chapel
modules allow for name space management.

\item
Chapter~\ref{Functions}, Functions, describes functions and function
resolution in Chapel.

\item
Chapter~\ref{Tuples}, Tuples, describes tuples in Chapel.

\item
Chapter~\ref{Classes}, Classes, describes reference classes in Chapel.

\item
Chapter~\ref{Records}, Records, describes records or value classes in
Chapel.

\item
Chapter~\ref{Unions}, Unions, describes unions in Chapel.

\item
Chapter~\ref{Ranges}, Ranges, describes ranges in Chapel.

\item
Chapter~\ref{Domains}, Domains, describes domains in Chapel.  Chapel
domains are first-class index sets that support the description of
iteration spaces, array sizes and shapes, and sets of indices.

\item
Chapter~\ref{Arrays}, Arrays, describes arrays in Chapel.  Chapel arrays are
more general than in most languages including support for
multidimensional, sparse, associative, and unstructured arrays.

\item
Chapter~\ref{Iterators}, Iterators, describes iterator functions.

\item
Chapter~\ref{Generics}, Generics, describes Chapel's support for
generic functions and types.

\item
Chapter~\ref{Input_and_Output}, Input and Output, describes support
for input and output in Chapel, including file input and output..

\item
Chapter~\ref{Task_Parallelism_and_Synchronization}, Task Parallelism
and Synchronization, describes task-parallel expressions and
statements in Chapel as well as synchronization constructs and the atomic
statement.

\item
Chapter~\ref{Data_Parallelism}, Data Parallelism, describes
data-parallel expressions and statements in Chapel including
reductions and scans, whole array assignment, and promotion.

\item
Chapter~\ref{Locales}, Locales, describes constructs for managing
locality and executing Chapel programs on distributed-memory systems.

\item
Chapter~\ref{Domain_Maps}, Domain Maps, describes
Chapel's \emph{domain map} construct for defining the layout of
domains and arrays within a single locale and/or the distribution of
domains and arrays across multiple locales.

\item
Chapter~\ref{User_Defined_Reductions_and_Scans}, User-Defined
Reductions and Scans, describes how Chapel programmers can define
their own reduction and scan operators.

\item
Chapter~\ref{User_Defined_Domain_Maps}, User-Defined Domain Maps,
describes how Chapel programmers can define their own domain maps to
implement domains and arrays.

\item
  Chapter~\ref{Memory_Consistency_Model}, Memory Consistency Model,
  describes Chapel's rules for ordering the reads and writes performed
  by a program's tasks.

\item
Chapter~\ref{Interoperability} describes Chapel's interoperability
features for combining Chapel programs with code written in different
languages.

\item
Chapter~\ref{Standard_Modules}, Standard Modules, describes the
standard modules that are provided with the Chapel language.

\item
Chapter~\ref{Standard_Distributions}, Standard Distributions,
describes the standard distributions (multi-locale domain maps) that
are provided with the Chapel language.

\item
Chapter~\ref{Standard_Layouts}, Standard Layouts, describes the
standard layouts (single locale domain maps) that are provided with
the Chapel language.

\item
Appendix~\ref{Syntax}, Collected Lexical and Syntax Productions,
contains the syntax productions listed throughout this specification
in both alphabetical and depth-first order.

\end{itemize}

\cleardoublepage
This is a stub.  This portion of the document does not exist.

\cleardoublepage
\sekshun{Language Overview}
\label{Language_Overview}

In HPC applications, the current dominant parallel programming paradigm 
is characterized by a localized
view of the computation combined with explicit control
over message passing, as exemplified by a combination
of Fortran or C/C++ with MPI. Such a fragmented memory
model provides the programmer with full control over data
distribution and communication, at the expense of productivity,
conciseness, and clarity.

Chapel is a new parallel programming language that 
strives to improve the programmability of parallel computer systems.
It provides a higher level of expression 
than current parallel languages do and it improves the separation between 
algorithmic expression and data structure implementation details. 

Chapel supports a global-view parallel programming model at a high level by 
supporting abstractions for data parallelism, task parallelism, and nested parallelism. 
It supports optimization for the locality of data and computation in the program 
via abstractions for data distribution and data-driven placement of subcomputations. 
It supports code reuse and generality via object-oriented concepts and generic 
programming features. While Chapel borrows concepts from many preceding languages, 
its parallel concepts are most closely based on ideas from High-Performance Fortran 
(HPF), ZPL, and the Cray MTA's extensions to Fortran/C. 

The key features of the Chapel language for productive parallel programming are as 
follows: 
\begin{itemize}
\item {\bf Locale type} - an opaque type used for organizing and referring to 
units of machine locality.
\item {\bf Domains} - first-class index sets that can potentially be distributed 
between multiple locales.  Domains are Chapel's primary vehicle for global-view 
data parallelism.
\item {\bf Arrays} - generalized support for distributed data aggregates, including 
dynamic multidimensional rectilinear arrays, potentially strided and/or sparse 
in each dimension; associative arrays; set- and graph-based arrays.
\item {\bf User-defined distributions} - a capability for users to specify the 
low-level distributed implementation of domains and arrays orthogonally 
to the computations that operate on these concepts.
\item {\bf \chpl{forall} loops and iterators} - concepts for specifying parallel 
iteration in a manner that separates algorithm and implementation.
\item {\bf Index types} - types representing domain indices to support code 
clarity and bounds-checking optimizations.    
\item {\bf User-defined reductions and scans} - a framework for expressing parallel 
prefix operations over data aggregates cleanly and efficiently.
\item {\bf \chpl{cobegin} and \chpl{begin} statements} - statement types for 
supporting task-parallel computations.
\item {\bf Sync and single-assignment variables} - variable types that support 
synchronization between parallel tasks.
\item {\bf Atomic sections} - compound statements that support atomic execution 
from the perspective of other threads.
\item {\bf \chpl{On} clauses} - specifications that support explicit placement of 
data values and computation on the machine's locales.
\item {\bf Value and reference classes} - object-oriented software containers 
that support encapsulation of state and the separation of interfaces from 
implementations.
\item {\bf Function and operator overloading, multiple dispatch, pass-by-argument name, 
default argument values} - concepts that support modern and productive function call 
capabilities.
\item {\bf Type variables and latent types} - capabilities for writing algorithms 
independently of types to support code reuse, exploratory programming, and 
generic functions and data structures.
\item {\bf Modules} - software containers for namespace management and programming 
in-the-large.
\item {\bf Other features for productive programming} - tuples, type-safe unions, 
sequences, etc.
\end{itemize}

\subsection{Motivating Principles}
\label{Motivating_Principles}

Chapel pushes the state-of-the-art in parallel programming
by focusing on productivity and not just performance. In particular
Chapel combines the goal of highest possible
object code performance with that of programmability
by supporting a high level interface resulting in
shorter time-to-solution and reduced application development
cost. The design of Chapel is guided by four key
areas of programming language technology: global-view programming,
locality-awareness, object-orientation, and generic
programming.

\subsubsection{Global View Programming Model}
\label{Global_View_Programming_Model}

Parallel programming models can be divided into two types of models:
{\em fragmented} and {\em global-view}.  Fragmented programming models
require programmers to express algorithms on a task-by-task basis so that
the tasks can execute in parallel.  Global-view programming models
allow programmers to express a parallel algorithm as a whole, similar to
a serial algorithm.  The compiler and runtime libraries identify and assign
tasks to run in parallel across processors.  Chapel provides a global-view 
programming model.

The global-view programming model is is able:
\begin{itemize}
\item to operate on distributed data structures monolithically as 
though they were local to the executing thread's memory, and 
\item to express parallelism within a single source text without requiring 
multiple executables to be run simultaneously. 
\end{itemize}
While the global-view programming model can be implemented on any distributed memory 
machine, specific architectures provide an ideal target for this model.  
Global-view models map particularly well to architectures that 
support a global address space, DGAS and PGAS memory segments, a high performance 
network, lightweight synchronization, and latency-tolerant processors.  This 
synergy results in improved performance as compared to implementations on less-
productive architectures.

We believe that the dominance of the fragmented programming model is the primary
inhibitor of parallel programmability today, and therefore recommend that new
productivity-oriented languages focus on supporting a global view of parallel
programming.  





\subsubsection{Locality Aware Programming}
\label{Locality_Aware_Programming}

Locality-aware programming, in the style of HPF and
ZPL, provides distribution of shared data structures without
requiring a fragmentation of control structure. The programmer
reasons about load-balance and locality by specifying
the placement of data objects and threads.

\subsubsection{Object-Oriented Programming}
\label{Object-Oriented_Programming}

Object-oriented programming helps in managing complexity
by separating common function from specific implementation
to facilitate reuse.

\subsubsection{Generic Programming}
\label{Generic_Programming}

Generic programming and type-inference simplify the
type systems presented to users. High-performance computing
requires type systems to provide data structure details
that allow for efficient implementation. Generic programming
avoids the need for explicit specification of such
details when they can be inferred from the source or from
specialization of program templates.

\subsection{Basic Language Features}
\label{Basic_Language_Features}

Chapel is an imperative programming language.  The basic concepts of the
language should be familiar to users of C, Fortran, Java, Modula and Ada.
However, the syntax of the Chapel language does not directly build upon any of 
these existing languages.   Programmers should start afresh when programming in 
Chapel and not be limited to the constructs of existing languages.

\subsubsection{Getting Started}

Consider the classic first program.  The program file,
\chpl{helloworld.chpl} contains:
\begin{chapel}
def main() {
  writeln("hello, world");
}
\end{chapel}

The syntax of this simple program somewhat resembles C.  There are a
few items to note.
The program contains one module, \chpl{helloworld}, which is implicitly
named from the name of the file.  The keyword \chpl{def} indicates that the
definition of a function follows.

To compile and run this program, execute the following
commands at the system prompt:
\begin{verbatim}
> chpl helloworld.chpl
> ./a.out
\end{verbatim}
The following output is shown:
\begin{verbatim}
> hello, world
\end{verbatim}

\subsubsection{Programs and Modules}
\label{Programs_and_Modules}

All Chapel code is organized using \emph{modules} which serve as code
containers to help manage code complexity as programs grow in size.
One module may ``\chpl{use}'' another, giving it access to that
module's public global symbols.  In the following example, the
standard Chapel module \chpl{Types} is used.  This module contains
the \chpl{numBits} function that returns the size of a Chapel numeric
type in bits. 
%% test:  modulesexample.chpl
\begin{chapel}
use Types;

writeln("The default size of Chapel integers is ",numBits(int)," bits.");
\end{chapel}
The output of this example is:
\begin{verbatim}
The default size of Chapel integers is 32 bits.
\end{verbatim}

For convenience in exploratory programming,
explicit module declarations are not required.  If code is specified 
without a module declaration, the code's
filename is used as the module name for the code that it contains.

All Chapel programs must define a single subroutine named
\fnname{main()} that specifies the entry point for the program.  
This entry point is executed by a single logical thread.

Chapel provides standard modules for bit level operations, computing
random numbers and quering the system time.  See~\rsec{Standard_Modules}
for more details about these modules and how to use them.

\subsubsection{Data Types and Variables}
\label{Data_Types_and_Variables}

The following example demonstrates some variable declarations in Chapel.
\begin{chapel}
config const n = 10;

var x = 1.0,
    y = n:real,
    z: real;
\end{chapel}

The constant \chpl{n} can be set at runtime, as indicated by \chpl{config}, 
or it is set to its default value of \chpl{10}.  It is inferred to be of 
type \chpl{int} from this integer default value.  Similarly, \chpl{x} and 
\chpl{y} are inferred to be of type \chpl{real}.  The variable \chpl{x}
is initialized to \chpl{1.0} and the variable \chpl{y} is initialized to
the value of \chpl{n}, converted from an integer value to a real value.
The variable \chpl{z} has an explicit type declaration.  
Because \chpl{z} is not initialized, it has a default intial value of \chpl{0.0}.

Variable declarations in Chapel include the kind of variable, the variable's
name, type and initial value.  A variable's initialization may be omitted, 
in which case it will be initialized to an value dependent on its definition for 
safety (\eg, ``zero'' for numerical types).  Alternatively, a variable's type 
may be omitted, in which case it will be inferred from its
initializer.

There are three kinds of variables in Chapel specified by the following 
keywords:  \chpl{var}, \chpl{const},
and \chpl{param}.  The optional keyword \chpl{config} may precede
any of these variable keywords.  The \chpl{var} keyword indicates that a
variable is truly ``variable'' and may be modified throughout its
lifetime.  The \chpl{const} keyword indicates that a variable is a
constant, meaning that it \emph{must} be initialized and that its
value cannot change during its lifetime.  Unlike many languages,
Chapel's constant initializers need not be known at compile-time.  The
\chpl{param} keyword is used to define a \emph{parameter}, which is a
compile-time constant.  Parameter values are
required in certain language contexts, such as when specifying a
scalar type's bit-width or an array's rank.  In other contexts,
parameter values can be used to assert to the compiler that a
variable's value is known and unchanging.

Labeling a variable declaration with the optional \chpl{config}
keyword allows its value to be specified on the command line of the
compiler-generated executable (for \chpl{config const} and
\chpl{config var} declarations), or on the command-line of the
compiler itself (for \chpl{config param} declarations).

Variable declarations may also be specified in a variety of
comma-separated ways which allow multiple variables to share the same
variable-kind, type definition or initializer.  

Chapel has support for boolean, integer and floating point primitive types,
including the support for unsigned integers and complex types.  There is
also support for strings as primitive types.  The following table
lists the set of primitive types.  For each type, the default size and
all possible sizes are given.

\begin{center}
\begin{tabular}{|l|l|l|}
\hline
{\bf Type} & {\bf Default Size} & {\bf Supported Sizes}  \\
\hline
\begin{chapel}
int
\end{chapel}
& 32 bits & 
\begin{chapel}
int(8)
int(16) 
int(32) 
int(64) 
\end{chapel} \\
\hline
\begin{chapel}
uint
\end{chapel}
& 32 bits & 
\begin{chapel}
uint(8)
uint(16) 
uint(32) 
uint(64) 
\end{chapel} \\
\hline
\begin{chapel}
real
\end{chapel}
& 64 bits & 
\begin{chapel}
real(32)
real(64)
real(128)
\end{chapel} \\
\hline
\begin{chapel}
imag
\end{chapel} 
& 64 bits & 
\begin{chapel}
imag(32) 
imag(64)
imag(128)
\end{chapel} \\
\hline
\begin{chapel}
complex
\end{chapel}
& 128 bits & 
\begin{chapel}
complex(64)
complex(128)
complex(256)
\end{chapel} \\
\hline
\begin{chapel}
bool
\end{chapel} 
& 1 bit & 
\begin{chapel}
bool
\end{chapel}  \\
\hline
\begin{chapel}
string
\end{chapel}
& unbounded & 
\begin{chapel}
string 
\end{chapel} \\
\hline
\end{tabular}
\end{center}

%% Should locale be listed in the above table?

Chapel is a type-safe language.  When assigning from one type to another, explicit 
casts are often required by the compiler.  Details of implicit and explicit
conversions are discussed in~\rsec{Conversions}.

%% Not sure how much to say about coercions and conversions here.

Beyond these primitive types, there is support for enumerated types, tuples and
unions.  Additionally, arrays, domains, sequences, classes and records are
used in variable declarations as type definitions.

Chapel supports the ability to created named type definitions using
the \chpl{type} keyword.  Like parameter variables, type definitions
must be known at compile-time.  The example below demonstrates a use of a type
definition.

\begin{chapel}
type elemType = real(32);
var alpha: elemType;
\end{chapel}

The first line of the example code defines the identifier 
\typename{elemType} to be
an alias for a \chpl{real(32)}---Chapel's 32-bit floating point type.
The identifier \typename{elemType} may be used to specify a variable's
definition or anywhere else that a type is allowed.  

\subsubsection{Statements and Expressions}
\label{Statements_and_Expressions}

Examples of Chapel statements are given in the following table.

\begin{center}
\begin{tabular}{|l|l|}
\hline
{\bf Statement} & {\bf Example} \\
\hline
Block Statement &
\begin{chapel} % test:  block.chpl
var tau, s, c: real;
const a = 2.0, b = 5.5;
const b = 5.5;
{
tau = -a/b;
s = 1/sqrt(1 + tau*tau);
c = s*tau;
}
writeln("Givens rotation = ", s, " ", c);
\end{chapel} \\
\hline
Expression Statement & 
\begin{chapel} % test:  expstmt.chpl
var denom = 1.0;
var x: real;

testForZero(denom);
testForZero(x);
testForZero(0.0);

def testForZero(x: real) {
  if (x == 0.0) then halt("Value is zero.");
  else writeln("Non-zero value.  Continuing.");   
}
\end{chapel} \\
\hline
Assignment Statement & 
\begin{chapel} % test: assign.chpl
var i: int;

i = 0;
i = i + 1;
i += 1;
writeln(i);
\end{chapel} \\
\hline
Conditional Statement &
\begin{chapel} % test:  cond.chpl
const D = [1..5];
var x, y: [D] real;
var alpha = 2.0;

[i in D] y(i) = 3.0*i;
scale(x, y, alpha);
writeln(x);

def scale(x, y, alpha: real) {
  if (x.numElements != y.numElements) {
    writeln("Error:  Input vectors are not the same length.");
    return;
  }
  if (alpha == 0.0) {
    x = 0.0;
  } else if (alpha == 1.0) {
    x = y;
  } else {
    x = alpha*y;
  }
}
\end{chapel} \\
\hline
Select Statement &
\begin{chapel} % test:  select.chpl
const D = [1..5];
var A: [D] real;

[i in D] A(i) = i;

writeln(getvalue("first",A));
writeln(getvalue("last",A));
writeln(getvalue("middle",A));

def getvalue(pos:string,y) {
  var x = 0.0;
  select pos {
    when "first" do x = y(1);
    when "last" do x = y(y.numElements);
    when "middle" do x = y((y.numElements/2):int + y.numElements%2);
    otherwise writeln("Unrecognized element position");
  }
  return x;
}
\end{chapel} \\
\hline
\end{tabular}

\begin{tabular}{|l|l|}
\hline
{\bf Statement} & {\bf Example} \\
\hline
While and Do While Loops &
\begin{chapel} % test: while.chpl
var t = 11;

writeln("Scope of do while loop:");
do {
  t += 1;
  writeln(t);
} while (t <= 10);

t = 11;
writeln("Scope of while loop:");
while (t <= 10) {
  t += 1;
  writeln(t);
}
\end{chapel} \\
\hline
For Loop &
\begin{chapel} % test: for.chpl
const D = [1..5];
var A: [D] real;

[i in D] A(i) = -i*i;
writeln(norm1(A));

def norm1(x) {
  var norm = 0.0;
  for i in x.domain {
    norm += abs(x(i));
  }
  return norm;
}
\end{chapel} \\
\hline
Use Statement &
\begin{chapel} % test:  use.chpl
use Time;
var programTimer:Timer;

programTimer.start();
writeln("Write one line.");
programTimer.stop();
writeln(programTimer.accumulated);
\end{chapel} \\
\hline
Type Select Statement &
\begin{chapel} % test:  typeselect.chpl
var x = 32, y = 15.5;
var z: int(8);
var coord = (0.0,0.0);
var yes: bool;

writetype(x);
writetype(y);
writetype(z);
writetype(coord);
writetype(yes);
writetype("no");

def writetype(x) {
  type select x {
    when int do writeln("Integer type");
    when uint do writeln("Unsigned integer type");
    when real do writeln("Real type");
    when complex do writeln("Complex type");
    when string do writeln("String type");
    when bool do writeln("Boolean type");
    otherwise writeln("Non-primitive type");
  }
}
\end{chapel} \\
\hline
Empty Statement &
\begin{chapel}
;
\end{chapel} \\
\hline
\end{tabular}
\end{center}

\begin{center}
\begin{tabular}{|l|l|}
\hline
{\bf Expression} & {\bf Example} \\
\hline
Query Expression &
\begin{chapel} % query.chpl
writeln(sumOfThree(1,2,3));
writeln(sumOfThree(4.0,5.0,3.0));

def sumOfThree(x: ?t, y:t, z:t):t {
   var sum: t;

   sum = x + y + z;
   return sum;
}
\end{chapel} \\
\hline
Casts &
\begin{chapel} % casts.chpl
var x, y: complex;
x = 2.56 + 9.0i;
y = (3.12, 8.7): complex;
var z = (4.2, 6.1);

writeln(x);
writeln(y);
writeln(z);

var m = 2: int(64);
var n = 2;
var i = 1;
var j = 1;

while (n > 0) do {
  n *= 2;
  i += 1;
}
while (m > 0) do {
  m *= 2;
  j += 1;
}

writeln("For 32-bit integers, 2 ** (",i,") overflows.");
writeln("For 64-bit integers, 2 ** (",j,") overflows.");
\end{chapel} \\
\hline
Let Expression &
\begin{chapel} % let.chpl
quadsol(3.0,8.0,5.0);
quadsol(3.0,4.0,5.0);

def quadsol(a:real, b:real, c:real) {
  writeln("The solution of ",a,"x^2 + ",b,"x + ",c," = 0 is:");
  if (b*b > 4.0*a*c) {
    var x:  (real, real);

    x = let temp1 = sqrt(b*b - 4.0*a*c), temp2 = 2.0*a in
        ((-b + temp1)/temp2, (-b - temp1)/temp2);

    writeln(x);
  } else {
    var x: (complex, complex);

    x = let temp1 = sqrt(4.0*a*c - b*b)/(2.0*a), temp2 = -b/(2.0*a) in
        ((temp2,temp1):complex,(temp2,-temp1):complex);

    writeln(x);
  }
}
\end{chapel} \\
\hline
Conditional Expression &
\begin{chapel} % condexp.chpl
writehalf(8);
writehalf(21);
writehalf(1000);

def writehalf(i: int) {
  var half = if (i % 2) then i/2 +1 else i/2;
  writeln("Half of ",i," is ",half);
}
\end{chapel} \\
\hline
\end{tabular}
\end{center}

\subsubsection{Structured Data Types}
\label{Structured_Data_Types}

\subsubsection{Functions and Methods}
\label{Functions_and_Methods}

\subsubsection{Sequences and Iterators}
\label{Sequences_and_Iterators}


\subsubsection{Arrays and Domains}
\label{Arrays_and_Domains}

In Chapel, arrays are reference types that are declared using domains.
A domain is a first-class representation of an index space, potentially 
defined to be distributed across multiple locales.   All arrays
declared with a particular domain are indexed and distributed according 
to that domain's specifications.  

The following example shows three arrays that are declared to be
vectors of length \chpl{m} and then used to compute and store a 
scaled addition.
\begin{chapel}
const VectorD: domain(1) = [1..m];
var A, B, C: [VectorD] real;

A = B + alpha * C;
\end{chapel}
The first line declares a constant named \chpl{VectorD} that
is defined to be a \chpl{domain} that is  
1-dimensional, describing indices $\{ 1, 2, \ldots, m \}$.
The next line uses the \chpl{VectorD} domain to declare three
arrays \chpl{A}, \chpl{B}, and \chpl{C} of type \chpl{real}.  The
domain's index set defines the size and shape of these arrays. 

The final line uses whole-array syntax to specify the elementwise 
multiplications, additions, and assignments.  In this case since
all three arrays are declared with the same domain, the compiler
knows that the arrays are the right shape and size to successfully
compute the array addition and can generate the appropriate elementwise
additions.

Whole-array operations like this one are implicitly parallel, if the \chpl{VectorD}
domain were distributed across a set of processors.  For example, 
a block distribution of \chpl{VectorD} would be specified as follows.    
\begin{chapel}
const VectorD: domain(1) distributed(Block) = [1..m];
\end{chapel}
Each processor would perform the operations for the array elements that it owns, 
as defined by \chpl{VectorD}'s distribution since that was the domain
used to define all three arrays.

Arrays may be multi-dimensional if declared with multi-dimensional domains,
and they may be of any type.  
%% Need to qualify previous statement.
Since arrays are reference types, they are passed by reference to functions
where they may be modified and remain modified upon return.  However, assigning
from one array to another merely copies the values from one to the other.  The
two arrays each continue to reference individual arrays. 
%% Is previous statement correct?

The above example uses arithmetic domains and arrays.  Domains may
also be sparse, indefinite, enumerated or opaque.  Subdomains may be
defined to specify a subset of the domain's indices, as in the case of
inner, non-boundary points of a grid.  See~\rsec{Domains_and_Arrays} for
a complete description of domains and arrays.
 
\subsection{Parallel Features}
\label{Parallel_Features}


\subsubsection{Data Parallel Constructs}
\label{Data_Parallel_Constructs}


\subsubsection{Task Parallel Constructs}
\label{Task_Parallel_Constructs}


\subsubsection{Exploiting Data Locality}
\label{Exploiting_Data_Locality}


\subsubsection{Synchronizing and Serializing Tasks}
\label{Synchronizing_and_Serializing_Tasks}


\subsection{Data Distributions}
\label{Data_Distributions}


\cleardoublepage
\sekshun{Lexical Structure}
\label{Lexical_Structure}

This is a stub.  This portion of the document does not exist.

\subsection{Programs}
\label{Programs}

This is a stub.  This portion of the document does not exist.

\subsection{Comments}
\label{Comments}

This is a stub.  This portion of the document does not exist.

\subsection{White Space}
\label{White_Space}

This is a stub.  This portion of the document does not exist.

\subsection{Case Sensitivity}
\label{Case_Sensitivity}

This is a stub.  This portion of the document does not exist.

\subsection{Tokens}
\label{Tokens}

This is a stub.  This portion of the document does not exist.

\subsubsection{Identifiers}
\label{Identifiers}

This is a stub.  This portion of the document does not exist.

\subsubsection{Keywords}
\label{Keywords}

This is a stub.  This portion of the document does not exist.

\subsubsection{Literals}
\label{Literals}

This is a stub.  This portion of the document does not exist.

\subsubsection{Operators and Punctuation}
\label{Operators_and_Punctuation}

This is a stub.  This portion of the document does not exist.

\subsubsection{Grouping Tokens}
\label{Grouping_Tokens}

This is a stub.  This portion of the document does not exist.

\subsection{Compile-Time Conditionals}
\label{Compile-Time_Conditionals}

This is a stub.  This portion of the document does not exist.

\subsection{User-Defined Compiler Errors}
\label{User-Defined_Compiler_Errors}

This is a stub.  This portion of the document does not exist.

\cleardoublepage
\sekshun{Types}
\label{Types}

Chapel is a statically typed language with a rich set of types.  These
include a set of predefined primitive types, enumerated types,
locale types, structured types (classes, records, unions, tuples),
data parallel types (ranges, domains, arrays), and synchronization
types (sync, single).

% This section defines the primitive
% types, enumerated types, and type aliases.  

The syntax of a type is as follows:

\begin{syntax}
type-specifier:
  primitive-type
  enum-type
  locale-type
  structured-type
  dataparallel-type
  synchronization-type
\end{syntax}

Programmers can define their own enumerated types, classes, records,
unions, and type aliases using type declaration statements:

\begin{syntax}
type-declaration-statement:
  enum-declaration-statement
  class-declaration-statement
  record-declaration-statement
  union-declaration-statement
  type-alias-declaration-statement
\end{syntax}

These statements are defined in Sections \rsec{Enumerated_Types},
\rsec{Class_Declarations}, \rsec{Record_Declarations},
\rsec{Union_Declarations}, and \rsec{Type_Aliases}, respectively.

\section{Primitive Types}
\label{Primitive_Types}
\index{types!primitive}

The primitive types include the following types: \chpl{void}, chpl{bool},
\chpl{int}, \chpl{uint}, \chpl{real}, \chpl{imag}, \chpl{complex},
\chpl{string}, and \chpl{locale}.  These primitive types are defined
in this section.

The primitive types are summarized by the following syntax:
\begin{syntax}
primitive-type:
  `void'
  `bool' primitive-type-parameter-part[OPT]
  `int' primitive-type-parameter-part[OPT]
  `uint' primitive-type-parameter-part[OPT]
  `real' primitive-type-parameter-part[OPT]
  `imag' primitive-type-parameter-part[OPT]
  `complex' primitive-type-parameter-part[OPT]
  `string'

primitive-type-parameter-part:
  ( integer-parameter-expression )

integer-parameter-expression:
  expression
\end{syntax}

If present, the parenthesized \sntx{integer-parameter-expression} must
evaluate to a compile-time constant of integer type.  See~\rsec{Compile-Time_Constants}

\begin{openissue}
There is an expectation of future support for larger bit width
primitive types depending on a platform's native support for those
types.
\end{openissue}

\subsection{The Void Type}
\label{The_Void_Type}
\index{void@\chpl{void}}

The \chpl{void} type is used to represent the lack of a value, for
example when a function has no arguments and/or no return type.  

There may be storage associated with a value of type \chpl{void}, in which
case its lifetime obeys the same rules as a value of type \chpl{int}.

\subsection{The Bool Type}
\label{The_Bool_Type}
\index{bool@\chpl{bool}}

Chapel defines a logical data type designated by the symbol
\chpl{bool} with the two predefined values \chpl{true} and
\chpl{false}.  This default boolean type is stored using an
implementation-defined number of bits.  A particular number of bits
can be specified using a parameter value following the \chpl{bool}
keyword, such as \chpl{bool(8)} to request an 8-bit boolean value.
Legal sizes are 8, 16, 32, and 64 bits.

%% The relational operators return values of \chpl{bool} type and the
%% logical operators operate on values of \chpl{bool} type.

Some statements require expressions of \chpl{bool} type and Chapel
supports a special conversion of values to \chpl{bool} type when used
in this context~(\rsec{Implicit_Statement_Bool_Conversions}).

\subsection{Signed and Unsigned Integral Types}
\label{Signed_and_Unsigned_Integral_Types}
\index{uint@\chpl{uint}}
\index{int@\chpl{int}}

The integral types can be parameterized by the number of bits used to
represent them.  Valid bit-sizes are 8, 16, 32, and 64.  
The default signed integral type, \chpl{int}, and the
default unsigned integral type, \chpl{uint}, are 32 bits.

The integral types and their ranges are given in the following table:

\begin{center}
\begin{tabular}{|l|r|r|}
\hline
{\bf Type} & {\bf Minimum Value} & {\bf Maximum Value} \\
\hline
{\tt int(8)} & -128 & 127 \\
{\tt uint(8)} & 0 & 255 \\
{\tt int(16)} & -32768 & 32767 \\
{\tt uint(16)} & 0 & 65535 \\
{\tt int(32)}, {\tt int} & -2147483648 & 2147483647 \\
{\tt uint(32)}, {\tt uint} & 0 & 4294967295 \\
{\tt int(64)} & -9223372036854775808 & 9223372036854775807 \\
{\tt uint(64)} & 0 & 18446744073709551615 \\
\hline
\end{tabular}
\end{center}

The unary and binary operators that are pre-defined over the integral
types operate with 32- and 64-bit precision.  Using these operators on
integral types represented with fewer bits results in a coercion
according to the rules defined in~\rsec{Implicit_Conversions}.

\begin{openissue}
There is on going discussion on whether the default size of the
integral types should be changed to 64 bits.
\end{openissue}


\subsection{Real Types}
\label{Real_Types}
\index{real@\chpl{real}}

Like the integral types, the real types can be parameterized by the
number of bits used to represent them.  The default real
type, \chpl{real}, is 64 bits.  The real types that are supported are
machine-dependent, but usually include \chpl{real(32)} (single
precision) and \chpl{real(64)} (double precision) following the IEEE
754 standard.  

\subsection{Imaginary Types}
\label{Imaginary_Types}
\index{imaginary@\chpl{imaginary}}

The imaginary types can be parameterized by the number of bits used to
represent them.  The default imaginary type, \chpl{imag}, is 64 bits.
The imaginary types that are supported are machine-dependent, but
usually include \chpl{imag(32)} and \chpl{imag(64)}.

\begin{rationale}
The imaginary type is included to avoid numeric instabilities and
under-optimized code stemming from always coercing real values to
complex values with a zero imaginary part.
\end{rationale}

\subsection{Complex Types}
\label{Complex_Types}
\index{complex@\chpl{complex}}

Like the integral and real types, the complex types can be
parameterized by the number of bits used to represent them.  A complex
number is composed of two real numbers so the number of bits used to
represent a complex is twice the number of bits used to represent the
real numbers.  The default complex type, \chpl{complex}, is 128 bits;
it consists of two 64-bit real numbers.  The complex types that are
supported are machine-dependent, but usually
include \chpl{complex(64)} and \chpl{complex(128)}.

The real and imaginary components can be accessed via the methods
\chpl{re} and \chpl{im}.  The type of these components is real.
See~\rsec{Math} for math routines for complex types.

\begin{example}
Given a complex number \chpl{c} with the value \chpl{3.14+2.72i}, the
expressions \chpl{c.re} and \chpl{c.im} refer to \chpl{3.14}
and \chpl{2.72} respectively.
\end{example}

\subsection{The String Type}
\label{The_String_Type}
\index{string@\chpl{string}}

Strings are a primitive type designated by the symbol \chpl{string}
comprised of ASCII characters.  Their length is unbounded.
See~\rsec{Standard} for routines for manipulating strings.


\begin{openissue}
There is an expectation of future support for fixed-length strings.
\end{openissue}

\begin{openissue}
There is an expectation of future support for different character
sets, possibly including internationalization.
\end{openissue}

\section{Enumerated Types}
\label{Enumerated_Types}
\index{enumerated types}

Enumerated types are declared with the following syntax:

\begin{syntax}
enum-declaration-statement:
  `enum' identifier { enum-constant-list }

enum-constant-list:
  enum-constant
  enum-constant , enum-constant-list[OPT]

enum-constant:
  identifier init-part[OPT]

init-part:
  = expression
\end{syntax}

The enumerated type can then be referenced by its name, as summarized
by the following syntax:

\begin{syntax}
enum-type:
  identifier
\end{syntax}

An enumerated type defines a set of named constants that can be
referred to via a member access on the enumerated type.
These constants are treated as parameters of integral type.  Each
enumerated type is a distinct type. If the \sntx{init-part} is
omitted, the \sntx{enum-constant} has an integral value one higher
than the previous \sntx{enum-constant} in the enum, with the first
having the value \chpl{1}.

\begin{chapelexample}{enum.chpl}
The code
\begin{chapel}
enum statesman { Aristotle, Roosevelt, Churchill, Kissinger }
\end{chapel}
defines an enumerated type with four constants.  The function
\begin{chapel}
proc quote(s: statesman) {
  select s {
    when statesman.Aristotle do
       writeln("All paid jobs absorb and degrade the mind.");
    when statesman.Roosevelt do
       writeln("Every reform movement has a lunatic fringe.");
    when statesman.Churchill do
       writeln("A joke is a very serious thing.");
    when statesman.Kissinger do
       { write("No one will ever win the battle of the sexes; ");
         writeln("there's too much fraternizing with the enemy."); }
  }
}
\end{chapel}
\begin{chapelnoprint}
for s in statesman.Aristotle..statesman.Kissinger do
  quote(s:statesman);
\end{chapelnoprint}
\begin{chapeloutput}
All paid jobs absorb and degrade the mind.
Every reform movement has a lunatic fringe.
A joke is a very serious thing.
No one will ever win the battle of the sexes; there's too much fraternizing with the enemy.
\end{chapeloutput}
outputs a quote from the given statesman.  Note that enumerated
constants must be prefixed by the enumerated type and a dot.
\end{chapelexample}


\section{Locale Types}
\label{Locale_Types}
\index{types!locale}

Locale types are summarized by the following syntax:

\begin{syntax}
locale-type:
  `locale'
\end{syntax}

The \chpl{locale} type is defined in~\rsec{The_Locale_Type}.

\begin{openissue}
We expect to support \emph{realms} as another locale type.
\end{openissue}

\section{Structured Types}
\label{Structured_Types}
\index{types!structured}

The structured types are summarized by the following syntax:

\begin{syntax}
structured-type:
  class-type
  record-type
  union-type
  tuple-type
\end{syntax}
% in README.firstClassFns: function-type

Classes are discussed in \rsec{Classes}.  Records are discussed
in \rsec{Records}.  Unions are discussed in \rsec{Unions}.  Tuples are
discussed in \rsec{Tuples}.

\subsection{Class Types}

The class type defines a type that contains variables and constants,
called fields, and functions, called methods.  Classes are defined
in~\rsec{Classes}.  The class type can also contain type aliases and
parameters.  Such a class is generic and is defined
in~\rsec{Generic_Types}.

\subsection{Record Types}

The record type is similar to a class type; the primary difference is
that a record is a value rather than a reference.  Records are defined
in~\rsec{Records}.

\subsection{Union Types}

The union type defines a type that contains one of a set of variables.
Like classes and records, unions may also define methods.  Unions are
defined in~\rsec{Unions}.

\subsection{Tuple Types}

A tuple is a light-weight record that consists of one or more
anonymous fields.  If all the fields are of the same type, the tuple
is homogeneous.  Tuples are defined in~\rsec{Tuples}.

\section{Data Parallel Types}
\label{Data_Parallel_Types}
\index{types!dataparallel}

The data parallel types are summarized by the following syntax:

\begin{syntax}
dataparallel-type:
  range-type
  domain-type
  mapped-domain-type
  array-type
  index-type
\end{syntax}

Ranges and their index types are discussed in \rsec{Ranges}.
Domains and their index types are discussed in \rsec{Domains}.
Arrays are discussed in \rsec{Arrays}.

\subsection{Range Types}

A range defines an integral sequence of some integral type.  Ranges
are defined in \rsec{Ranges}.

\subsection{Domain, Array, and Index Types}
\label{Domain_and_Array_Types}

A domain defines a set of indices. An array defines a set of
elements that correspond to the indices in its domain.
A domain's indicies can be of any type.
Domains, arrays, and their index
types are defined in \rsec{Domains} and \rsec{Arrays}.

\section{Synchronisation Types}
\label{Synchronisation_Types}
\index{types!synchronisation}

The synchronization types are summarized by the following syntax:

\begin{syntax}
synchronization-type:
  sync-type
  single-type
\end{syntax}

Sync and single types are discussed in \rsec{Sync_Variables}
and \rsec{Single_Variables}.

\section{Type Aliases}
\label{Type_Aliases}
\index{type aliases}

Type aliases are declared with the following syntax:
\begin{syntax}
type-alias-declaration-statement:
  `config'[OPT] `type' type-alias-declaration-list ;

type-alias-declaration-list:
  type-alias-declaration
  type-alias-declaration , type-alias-declaration-list

type-alias-declaration:
  identifier = type-specifier
  identifier
\end{syntax}
A type alias is a symbol that aliases the type specified in the
\sntx{type-part}.  A use of a type alias has the same meaning as using
the type specified by \sntx{type-part} directly.

If the keyword \chpl{config} precedes the keyword \chpl{type}, the
type alias is called a configuration type alias.  Configuration type
aliases can be set at compilation time via compilation flags or other
implementation-defined means.  The \chpl{type-specifier} in the
program is ignored if the type-alias is alternatively set.

The \sntx{type-part} is optional in the definition of a class or
record.  Such a type alias is called an unspecified type
alias. Classes and records that contain type aliases, specified or
unspecified, are generic~(\rsec{Type_Aliases_in_Generic_Types}).

\begin{openissue}
There is on going discussion on whether a type alias is a new
type or simply an alias.  The former should enable redefinition of
default values, identity elements, etc.
%hilde
% Would inheritance work?
\end{openissue}

\cleardoublepage
\sekshun{Variables}
\label{Variables}

A variable is a symbol that represents memory.  Chapel is a
statically-typed, type-safe language so every variable has a type that
is known at compile-time and the compiler enforces that values
assigned to the variable can be stored in that variable as specified
by its type.

\subsection{Variable Declarations}
\label{Variable_Declarations}
\index{variables!declarations}

Variables are declared with the following syntax:
\begin{syntax}
variable-declaration-statement:
  `config'[OPT] variable-kind variable-declaration ;

variable-kind: one of
  `param' `const' `var'

variable-declaration-list:
  variable-declaration
  variable-declaration , variable-declaration-list

variable-declaration:
  identifier-list type-part[OPT] initialization-part
  identifier-list type-part

identifier-list:
  identifier
  identifier , identifier-list

type-part:
  : type
  : synchronization-type type

initialization-part:
  = expression
\end{syntax}
A \sntx{variable-declaration-statement} is used to define one or more
variables.  If the statement is a top-level module statement, the
variables are global; otherwise they are local.  Global variables are
discussed in~\rsec{Global_Variables}.  Local variables are discussed
in~\rsec{Local_Variables}.

The optional keyword \chpl{config} specifies that the variables are
configuration variables, described in
Section~\rsec{Configuration_Variables}.

The \sntx{variable-kind} specifies whether the variables are
parameters (\chpl{param}), constants (\chpl{const}), or regular
variables (\chpl{var}).  Parameters are compile-time constants whereas
constants are runtime constants.  Both levels of constants are
discussed in~\rsec{Constants}.

Multiple variables can be defined in the same variable declaration
list.  All variables defined in the same \sntx{identifier-list} are
defined to have the same type and initialization expression.

The \sntx{type-part} of a variable declaration specifies the type of
the variable.  It is optional if the \sntx{initialization-part} is
specified.  If the \sntx{type-part} is omitted, the type of the
variable is inferred using local type inference described
in~\rsec{Local_Type_Inference}.

The \sntx{initialization-part} of a variable declaration specifies an
initial expression to assign to the variable.  If
the \sntx{initialization-part} is omitted, the variable is initialized
to a default value described in~\rsec{Default_Initialization}.

\subsubsection{Default Initialization}
\label{Default_Initialization}
\index{variables!default initialization}

If a variable declaration has no initialization expression, a variable
is initialized to the default value of its type.  The default values
are as follows:
\begin{center}
\begin{tabular}{|l|l|}
\hline
{\bf Type} & {\bf Default Value} \\
\hline
{\tt bool} & {\tt false} \\
{\tt int(*)} & {\tt 0} \\
{\tt uint(*)} & {\tt 0} \\
{\tt real(*)} & {\tt 0.0} \\
{\tt imag(*)} & {\tt 0.0i} \\
{\tt complex(*)} & {\tt 0.0 + 0.0i} \\
{\tt string} & {\tt ""} \\
enums & first enum constant \\
classes & {\tt nil} \\
records & default constructed record \\
sequences & empty sequence \\
arrays & elements are default values \\
tuples & components are default values \\
\hline
\end{tabular}
\end{center}

\subsubsection{Local Type Inference}
\label{Local_Type_Inference}
\index{type inference}

If the type is omitted from a variable declaration, the type of the
variable becomes the type of the initialization expression.

\subsection{Global Variables}
\label{Global_Variables}
\index{variables!global}

Variables declared in statements that are in a module but not in a
function or block within that module are global variables.  Global
variables can be accessed anywhere within that module after the
declaration of that variable.  They can also be accessed in other
modules that use that module.

\subsection{Local Variables}
\label{Local_Variables}
\index{variables!local}

Local variables are variables that are not global.  Local variables
are declared within block statements.  They can only be accessed
within the scope of that block statement (including all inner nested
block statements and functions).

A local variable only exists during the execution of code that lies
within that block statement.  This time is called the lifetime of the
variable.  When execution has finished within that block statement,
the local variable and the storage it represents is removed.
Variables of class type are the sole exception.  Constructors of class
types create storage that is not associated with any scope.  Such
storage is managed automatically as discussed
in~\rsec{Automatic_Memory_Management}.

\subsection{Constants}
\label{Constants}

Constants are divided into two categories: parameters, specified with
the keyword \chpl{param}, are compile-time constants and constants,
specified with the keyword \chpl{const}, are runtime constants.

\subsubsection{Compile-Time Constants}
\label{Compile-Time_Constants}
\index{constants!compile-time}
\index{param@\chpl{param}}
\index{parameters}

A compile-time constant or parameter must have a single value that is
known statically by the compiler.  Parameters are restricted to
primitive and enumerated types.

Parameters can be assigned expressions that are parameter expressions.
Parameter expressions are restricted to the following constructs:
\begin{itemize}
\item
 Literals of primitive type.
\item
 Parenthesized parameter expressions.
\item
 Casts of parameter expressions to primitive or enumerated types.
\item
 Applications of the unary operators \verb@+@, \verb@-@, \verb@!@,
 and \verb@~@ on operands that are bool or integral parameter
 expressions.
\item
 Applications of the binary operators \verb@+@, \verb@-@, \verb@*@, \verb@/@, \verb@%@, \verb@**@, \verb@&&@, \verb@||@, \verb@!@, \verb@&@, \verb@|@, \verb@^@, \verb@~@, \verb@<<@, \verb@>>@, \verb@==@, \verb@!=@, \verb@<=@, \verb@>=@, \verb@<@, and \verb@>@ on operands that are bool or integral parameter expressions.
\item
 The conditional expression where the condition is a parameter and the
 then- and else-expressions are parameters.
\end{itemize}

There is an expectation that parameters will be expanded to more types
and more operations, and that functions that return parameters will be
introduced, in the future.

\subsubsection{Runtime Constants}
\label{Runtime_Constants}
\index{constants!runtime}
\index{const@\chpl{const}}

Constants, as opposed to parameters, do not have the restrictions that
are associated with parameters.  Constants can be any type.  They
require an initialization expression and contain the value of that
expression throughout their lifetime.

Variables of class type that are constants are constant references.
The fields of the class can be modified, but the variable always
points to the object that it was initialized to reference.

\subsection{Configuration Variables}
\label{Configuration_Variables}
\index{variables!configuration}
\index{config@\chpl{config}}

If the keyword \chpl{config} precedes the
keyword \chpl{var}, \chpl{const}, or \chpl{param}, the variable,
constant, or parameter is called a configuration variable,
configuration constant, or configuration parameter respectively.  Such
variables, constants, and parameters must be global.

The initialization of these variables can be set via implementation
dependent means, such as command-line switches or environment
variables.  The initialization expression in the program is ignored if
the initialization is alternatively set.

\index{parameters!configuration}
Configuration parameters are set during compilation time via
compilation flags or other implementation dependent means.
\begin{example}
A configuration parameter is set via a compiler flag.  It may be used
to control the target that is being compiled.  For example, the code
\begin{chapel}
config param target: string = "XT3";
\end{chapel}
sets a string parameter \chpl{target} to \chpl{"XT3"}.  This can be
checked to compile different code for this target.
\end{example}

\cleardoublepage
\sekshun{Conversions}
\label{Conversions}
\index{conversions}

A \emph{conversion} converts an expression of one type to another type,
possibly changing its value.
\index{conversions!source type}
\index{conversions!target type}
We refer to these two types the \emph{source} and \emph{target} types.
Conversions can be either
implicit~(\rsec{Implicit_Conversions}) or
explicit~(\rsec{Explicit_Conversions}).


\section{Implicit Conversions}
\label{Implicit_Conversions}
\index{conversions!implicit}

An \emph{implicit conversion} is a conversion that occurs implicitly,
that is, not due to an explicit specification in the program.
Implicit conversions occur at the locations in the program listed below.
Each location determines the target type.
The source and target types of an implicit conversion must be allowed.
They determine whether and how the expression's value changes.

Implicit conversions are not applied when initializing \chpl{ref} or
\chpl{type} values or for actual arguments passed to \chpl{ref} or
\chpl{type} formal arguments.

\index{conversions!implicit!occurs at}
An implicit conversion occurs at each of the following program locations:

\begin{itemize}
\item In an assignment, the expression on the right-hand side of
      the assignment is converted to the type of the variable
      or another lvalue on the left-hand side of the assignment.

\item The actual argument of a function call or an operator is converted
      to the type of the corresponding formal argument, if the formal's
      intent is \chpl{param}, \chpl{in}, \chpl{const in}, or an abstract intent
      (\rsec{Abstract_Intents}) with the semantics of
      \chpl{in} or \chpl{const in}.

% MPF: This rule doesn't seem to be implemented right now,
%      but rather reflects ideal language design.
\item If the formal argument's intent is \chpl{out}, the formal argument
      is converted to the type of the corresponding actual argument
      upon function return.

\item The return or yield expression within a function without a \chpl{ref}
      return intent is converted to the return type of that function.

\item The condition of a conditional expression,
      conditional statement, while-do or do-while loop statement
      is converted to the boolean type~(\rsec{Implicit_Statement_Bool_Conversions}).
      A special rule defines the allowed source types and
      how the expression's value changes in this case.
\end{itemize}

\index{conversions!implicit!allowed types}
Implicit conversions \emph{are allowed} between
the following source and target types,
as defined in the referenced subsections:

\begin{itemize}
\item numeric and boolean types~(\rsec{Implicit_NumBool_Conversions}),
\item class types~(\rsec{Implicit_Class_Conversions}),
\item integral types in the special case when the expression's value
      is a compile-time constant~(\rsec{Implicit_Compile_Time_Constant_Conversions}), and
\item from an integral or class type to \chpl{bool}
      in certain cases~(\rsec{Implicit_Statement_Bool_Conversions}).
\end{itemize}

In addition,
an implicit conversion from a type to the same type is allowed for any type.
Such conversion does not change the value of the expression.

% TODO: If an implicit conversion is not allowed, it is an error.

Implicit conversion is not transitive. That is, if an implicit conversion
is allowed from type \chpl{T1} to \chpl{T2} and from \chpl{T2} to \chpl{T3},
that by itself does not allow an implicit conversion
from \chpl{T1} to \chpl{T3}.

\subsection{Implicit Numeric and Bool Conversions}
\label{Implicit_NumBool_Conversions}

\index{conversions!numeric}
\index{conversions!implicit!numeric}
Implicit conversions among numeric types are allowed when
all values representable in the source type can also be represented
in the target type, retaining their full precision.
%
%REVIEW: vass: I did not understand the point of the following,
% so I am commenting it out for now.
%When the implicit conversion is from an integral to a real type, source
%types are converted to type \chpl{int} before determining if the
%conversion is valid.
%
In addition, implicit conversions from
types \chpl{int(64)} and \chpl{uint(64)} to types \chpl{real(64)}
and \chpl{complex(128)} are allowed, even though they may result in a loss of
precision.

%REVIEW: hilde
% Unless we are supporting some legacy behavior, I would recommend removing this
% provision.  A loss of precision is a loss of precision, so I would favor
% consistent behavior that does not lead to surprising results.  EVERY ``if''
% costs money: which is to say that if a behavior can be described simply, it can
% be implemented simply.

\begin{rationale}
We allow these additional conversions because they are an important
convenience for application programmers. Therefore we are willing to
lose precision in these cases. The largest real and complex types
are chosen to retain precision as often as as possible.
\end{rationale}

\index{conversions!boolean}
\index{conversions!implicit!boolean}
Any boolean type can be implicitly converted to any other boolean type,
retaining the boolean value.
Any boolean type can be implicitly converted to any integral type
by representing \chpl{false} as 0 and \chpl{true} as 1,
except (if applicable)
a boolean cannot be converted to \chpl{int(1)}.
% Rationale: because 1 cannot be represented by \chpl{int(1)}.

\begin{rationale}
We disallow implicit conversion of a boolean to
a real, imaginary, or complex type because of the following.
We expect that the cases where such a conversion is needed
will more likely be unintended by the programmer.
Marking those cases as errors will draw the programmer's attention.
If such a conversion is actually desired, a cast \rsec{Explicit_Conversions}
can be inserted.
\end{rationale}

Legal implicit conversions with numeric and boolean types
may thus be tabulated as follows:

\begin{center}
\begin{tabular}{l|llllll}
& \multicolumn{6}{c}{Target Type} \\ [4pt]

Source Type  & bool($t$) & uint($t$) & int($t$) & real($t$) & imag($t$) & complex($t$) \\  [3pt]

\cline{1-7} \\

bool($s$)    & all $s,t$ & all $s,t$   & all $s$; $2 \le t$ & & & \\ [7pt]

uint($s$)    & & $s \le t$ & $s < t$   & $s \le mant(t)$   & & $s \le mant(t/2)$   \\ [7pt]

uint(64)     & &           &           & real(64)          & & complex(128)        \\ [7pt]

int($s$)     & &           & $s \le t$ & $s \le mant(t)+1$ & & $s \le mant(t/2)+1$ \\ [7pt]

int(64)      & &           &           & real(64)          & & complex(128)        \\ [7pt]

real($s$)    & & & & $s \le t$ &           & $s \le t/2$ \\ [7pt]

imag($s$)    & & & &           & $s \le t$ & $s \le t/2$ \\ [7pt]

complex($s$) & & & &           &           & $s \le t$   \\ [5pt]

\end{tabular}
\end{center}
Here, $mant(i)$ is the number of bits in the (unsigned) mantissa of
the $i$-bit floating-point type.\footnote{For the IEEE 754 format,
$mant(32)=24$ and $mant(64)=53$.}
%
Conversions for the default integral and real types (\chpl{uint},
\chpl{complex}, etc.) are the same as for their
explicitly-sized counterparts.

\subsection{Implicit Compile-Time Constant Conversions}
\label{Implicit_Compile_Time_Constant_Conversions}
\index{conversions!numeric!parameter}
\index{conversions!implicit!parameter}

The following implicit conversion of a parameter is allowed:
\begin{itemize}
\item A parameter of type \chpl{int(64)} can be implicitly converted
to \chpl{int(8)}, \chpl{int(16)}, \chpl{int(32)}, or any unsigned integral type if the
value of the parameter is within the range of the target type.
\end{itemize}

\subsection{Implicit Statement Bool Conversions}
\label{Implicit_Statement_Bool_Conversions}
\index{conversions!boolean!in a statement}
\index{conversions!implicit!boolean}

In the condition of an if-statement, while-loop, and do-while-loop,
the following implicit conversions to \chpl{bool} are supported:
\begin{itemize}
\item An expression of integral type is taken to be false if it is zero and is true otherwise.
\item An expression of a class type is taken to be false if it is nil and is true otherwise.
\end{itemize}

\section{Explicit Conversions}
\label{Explicit_Conversions}
\index{conversions!explicit}

Explicit conversions require a cast in the code.  Casts are defined
in~\rsec{Casts}.  Explicit conversions are supported between more
types than implicit conversions, but explicit conversions are not
supported between all types.

The explicit conversions are a superset of the implicit conversions.
In addition to the following definitions,
an explicit conversion from a type to the same type is allowed for any type.
Such conversion does not change the value of the expression.

\subsection{Explicit Numeric Conversions}
\label{Explicit_Numeric_Conversions}
\index{conversions!numeric}
\index{conversions!explicit!numeric}

Explicit conversions are allowed from any numeric type or boolean to bytes or
string, and vice-versa.

% A valid \chpl{bool} value behaves like a single unsigned bit.  
When a \chpl{bool} is converted to a \chpl{bool}, \chpl{int}
or \chpl{uint} of equal or larger size, its value is zero-extended to fit the
new representation.  When a \chpl{bool} is converted to a
smaller \chpl{bool}, \chpl{int} or \chpl{uint}, its most significant
bits are truncated (as appropriate) to fit the new representation.

When a \chpl{int}, \chpl{uint}, or \chpl{real} is converted to a \chpl{bool}, the result is \chpl{false} if the number was equal to 0 and \chpl{true} otherwise.
% This has the odd effect that a bool stored in a signed one-bit bitfield would
% change sign without generating a conversion error.  But its subsequent
% conversion back to a bool would yield the original value.
% In regard to supporting bitfields: Be careful what you wish for.

% The source type determines whether a value is zero- or sign-extended.
When an \chpl{int} is converted to a larger \chpl{int} or \chpl{uint}, its value is
sign-extended to fit the new representation.  
When a \chpl{uint} is converted to a larger \chpl{int} or \chpl{uint}, its value
is zero-extended.
When an \chpl{int} or \chpl{uint} is converted to an \chpl{int} or \chpl{uint}
of the same size, its binary representation is unchanged.
When an \chpl{int} or \chpl{uint} is converted to a smaller \chpl{int}
or \chpl{uint}, its value is truncated to fit the new representation.

\begin{future}
There are several kinds of integer conversion which can result in a loss of
precision.  Currently, the conversions are performed as specified, and no error
is reported.  In the future, we intend to improve type checking, so the user can
be informed of potential precision loss at compile time, and actual precision
loss at run time.  Such cases include:
%
% An exception is thrown if the source value cannot be represented in the target type.
When an \chpl{int} is converted to a \chpl{uint} and the original value is
negative;
When a \chpl{uint} is converted to an \chpl{int} and the sign bit of the result
is true;
When an \chpl{int} is converted to a smaller \chpl{int} or \chpl{uint} and any
of the truncated bits differs from the original sign bit;
%
When a \chpl{uint} is converted to a smaller \chpl{int} or \chpl{uint} and any
of the truncated bits is true;
\end{future}

\begin{rationale}
For integer conversions, the default behavior of a program should be to produce
a run-time error if there is a loss of precision.  Thus, cast expressions not only
give rise to a value conversion at run time, but amount to an assertion
that the required precision is preserved.  Explicit conversion procedures would be
available in the run-time library so that one can perform explicit conversions
that result in a loss of precision but do not generate a run-time diagnostic.
\end{rationale}

When converting from a \chpl{real} type to a larger \chpl{real} type, the
represented value is preserved.  When converting from a \chpl{real} type to a
smaller \chpl{real} type, the closest representation in the target type is
chosen.\footnote{When converting to a smaller real type, a loss of precision is \emph{expected}.
Therefore, there is no reason to produce a run-time diagnostic.}

When converting to a \chpl{real} type from an integer type, integer types
smaller than \chpl{int} are first converted to \chpl{int}.  Then, the closest
representation of the converted value in the target type is chosen.  The exact
behavior of this conversion is implementation-defined.

When converting from \chpl{real($k$)} to \chpl{complex($2k$)}, the original
value is copied into the real part of the result, and the imaginary part of the
result is set to zero.  When converting from a \chpl{real($k$)} to
a \chpl{complex($\ell$)} such that $\ell > 2k$, the conversion is performed as
if the original value is first converted to \chpl{real($\ell/2$)} and then
to \chpl{$\ell$}.

The rules for converting from \chpl{imag} to \chpl{complex} are the same as for
converting from real, except that the imaginary part of the result is set using
the input value, and the real part of the result is set to zero.

\subsection{Explicit Tuple to Complex Conversion}
\label{Explicit_Tuple_to_Complex_Conversion}
\index{conversions!tuple to complex}
\index{conversions!explicit!tuple to complex}

A two-tuple of numerical values may be converted to a \chpl{complex} value.  If
the destination type is \chpl{complex(128)}, each member of the two-tuple must
be convertible to \chpl{real(64)}.  If the destination type
is \chpl{complex(64)}, each member of the two-tuple must be convertible
to \chpl{real(32)}.  The first member of the tuple becomes the real part of the
resulting complex value; the second member of the tuple becomes the imaginary
part of the resulting complex value.

\subsection{Explicit Enumeration Conversions}
\label{Explicit_Enumeration_Conversions}
\index{conversions!enumeration}
\index{conversions!explicit!enumeration}

Explicit conversions are allowed from any enumerated type to any
\chpl{bytes} or \chpl{string} and vice-versa, and include \chpl{param} conversions.
For enumerated types that are either concrete or semi-concrete
(\rsec{Enumerated_Types}), conversions are supported between the
enum's constants and any numeric type or \chpl{bool},
including \chpl{param} conversions.  For a semi-concrete enumerated
type, if a conversion is attempted involving a constant with no
underlying integer value, it will generate a compile-time error for
a \chpl{param} conversion or an execution-time error otherwise.

When the target type is an integer type, the value is first converted to its
underlying integer type and then to the target type, following the rules above
for converting between integers.

When the target type is a real, imaginary, or complex type, the value
is first converted to its underlying integer type and then to the
target type.

When the target type is \chpl{bool}, the value is first converted to its
underlying integer type.  If the result is zero, the value of the \chpl{bool}
is \chpl{false}; otherwise, it is \chpl{true}.

When the target type is \chpl{bytes} or \chpl{string}, the value becomes the
name of the enumerator.

When the source type is \chpl{bool}, enumerators corresponding to the values 0
and 1 in the underlying integer type are selected, corresponding to input values
of \chpl{false} and \chpl{true}, respectively.

%REVIEW: hilde
% As with default values for variables of enumerated types, I am pushing for the
% simplest implementation -- in which the conversion does not actually change
% the stored value.  This means that it may be possible for an enumerated variable
% to assume a value that does not correspond to any of its enumerators.  Further
% encouragement to always supply a default clause in your switch statements!

When the source type is a real or integer type, the value is converted to the
target type's underlying integer type.  

The conversion from \chpl{complex} and \chpl{imag} types to an enumerated type is not
permitted.

When the source type is \chpl{bytes} or \chpl{string}, the enumerator whose name
matches value of the input is selected.  If no such enumerator exists, an
\chpl{IllegalArgumentError} is thrown.

\subsection{Explicit Class Conversions}
\label{Explicit_Class_Conversions}
\index{conversions!class}
\index{conversions!explicit!class}

An expression of static class type \chpl{C} can be explicitly
converted to a class type \chpl{D} provided that \chpl{C} is derived
from \chpl{D} or \chpl{D} is derived from \chpl{C}.

When at run time the source expression refers to an instance of \chpl{D}
or it subclass, its value is not changed.  Otherwise, the cast fails and
the result depends on whether or not the destination type is nilable. If
the cast fails and the destination type is not nilable, the cast
expression will throw a \chpl{classCastError}. If the cast fails and the
destination type is nilable, as with \chpl{D?}, then the result will be
\chpl{nil}.

In some cases a new variant of a class type needs to be computed that has
different nilability or memory management strategy. Supposing that
\chpl{T} represents a class type, then these casts may compute a new type:

\begin{itemize}
\item
\chpl{T:owned} - new management is \chpl{owned}, nilability from \chpl{T}

\item
\chpl{T:shared} - new management \chpl{shared}, nilability from \chpl{T}

\item
\chpl{T:borrowed} - new management \chpl{borrowed}, nilability from \chpl{T}

\item
\chpl{T:unmanaged} - new management \chpl{unmanaged}, nilability from \chpl{T}

\item
\chpl{T:class} - non-nilable type with specific concrete or generic management from \chpl{T}

\item
\chpl{T:class?} - nilable type with specific concrete or generic management from \chpl{T}

\item
\chpl{T:owned class} - non-nilable type with \chpl{owned} management
\item
\chpl{T:owned class?} - nilable type with \chpl{owned} management

\item
\chpl{T:shared class} - non-nilable type with \chpl{shared} management
\item
\chpl{T:shared class?} - nilable type with \chpl{shared} management

\item
\chpl{T:borrowed class} - non-nilable type with \chpl{borrowed} management
\item
\chpl{T:borrowed class?} - nilable type with \chpl{borrowed} management

\item
\chpl{T:unmanaged class} - non-nilable type with \chpl{unmanaged} management
\item
\chpl{T:unmanaged class?} - nilable type with \chpl{unmanaged} management

\end{itemize}

\subsection{Explicit Range Conversions}
\label{Explicit_Range_Conversions}
\index{conversions!range}
\index{conversions!explicit!range}

An expression of stridable range type can be explicitly converted
to an unstridable range type, changing the stride to 1 in the process.

\subsection{Explicit Domain Conversions}
\label{Explicit_Domain_Conversions}
\index{conversions!domain}
\index{conversions!explicit!domain}

An expression of stridable domain type can be explicitly converted
to an unstridable domain type, changing all strides to 1 in the process.

\subsection{Explicit String to Bytes Conversions}
\label{Explicit_String_to_Bytes_Conversions}
\index{conversions!string to bytes}
\index{conversions!explicit!string to bytes}

An expression of \chpl{string} type can be explicitly converted to a
\chpl{bytes}. However, the reverse is not possible as a \chpl{bytes} can contain
arbitrary bytes. Instead, \chpl{bytes.decode()} method should be used to produce
a \chpl{string} from a \chpl{bytes}.

\subsection{Explicit Type to String Conversions}
\label{Explicit_Type_to_String_Conversions}
\index{conversions!type to string}
\index{conversions!explicit!type to string}

A type expression can be explicitly converted to a \chpl{string}. The resultant
\chpl{string} is the name of the type.

\begin{chapelexample}{explicit-type-to-string.chpl}
For example:
\begin{chapel}
var x: real(64) = 10.0;
writeln(x.type:string);
\end{chapel}
\begin{chapeloutput}
real(64)
\end{chapeloutput}
This program will print out the string \chpl{"real(64)"}.
\end{chapelexample}

\cleardoublepage
\sekshun{Expressions}
\label{Expressions}
\index{expressions}

Chapel provides the following expressions:

\begin{syntax}
expression:
  literal-expression
  nil-expression
  variable-expression
  enum-constant-expression
  call-expression
  iteratable-call-expression
  member-access-expression
  constructor-call-expression
  query-expression
  cast-expression
  lvalue-expression
  parenthesized-expression
  unary-expression
  binary-expression
  let-expression
  if-expression
  for-expression
  forall-expression
  reduce-expression
  scan-expression
  module-access-expression
  tuple-expression
  tuple-expand-expression
  locale-access-expression
  mapped-domain-expression
\end{syntax}
% in README.firstClassFns: lambda-declaration-expression

Individual expressions are defined in the remainder of this chapter
and additionally as follows:

\begin{itemize}
\item forall, reduce, and scan \rsec{Data_Parallelism}
\item module access \rsec{Explicit_Naming}
\item tuple and tuple expand \rsec{Tuples}
\item locale access \rsec{Querying_the_Locale_of_a_Variable}
\item mapped domain \rsec{Domain_Maps}
\item constructor calls \rsec{Class_New}
\item \chpl{nil} \rsec{Class_nil_value}
\end{itemize}

\section{Literal Expressions}
\label{Literal_Expressions}
\index{literal expressions}
\index{expressions!literal}

A literal value for any of the predefined
types~(\rsec{Primitive_Type_Literals}) is a literal expression.
Literal expressions are given by the following syntax:
\begin{syntax}
literal-expression:
  bool-literal
  integer-literal
  real-literal
  imaginary-literal
  string-literal
  range-literal
  domain-literal
  array-literal
\end{syntax}

\section{Variable Expressions}
\label{Variable_Expressions}
\index{expressions!variable}

A use of a variable, constant, parameter, or formal argument, is
itself an expression.  The syntax of a variable expression is given
by:
\begin{syntax}
variable-expression:
  identifier
\end{syntax}

\section{Enumeration Constant Expression}
\label{Enumeration_Constant_Expression}
\index{expressions!enumeration constant}

A use of an enumeration constant is itself an expression.  Such a
constant must be preceded by the enumeration type name.  The syntax of
an enumeration constant expression is given by:
\begin{syntax}
enum-constant-expression:
  enum-type . identifier
\end{syntax}

For an example of using enumeration constants,
see~\rsec{Enumerated_Types}.

\section{Parenthesized Expressions}
\label{Parenthesized_Expressions}
\index{expressions!parenthesized}

A \sntx{parenthesized-expression} is an expression that is delimited
by parentheses as given by:
\begin{syntax}
parenthesized-expression:
  ( expression )
\end{syntax}
Such an expression evaluates to the expression.  The parentheses are
ignored and have only a syntactical effect.

\section{Call Expressions}
\label{Call_Expressions}
\index{function calls}
\index{expressions!call}

Functions and function calls are defined in~\rsec{Functions}.

\section{Indexing Expressions}
\label{Indexing_Expressions}
\index{indexing}
\index{expressions!indexing}

Indexing, for example into arrays, tuples, and domains,
has the same syntax as a call expression.
 
Indexing is performed by an implicit invocation of the \chpl{this} method
on the value being indexed,
passing the indices as the actual arguments.

\section{Member Access Expressions}
\label{Member_Access_Expressions}
\index{member access}
\index{expressions!member access}

Member access expressions provide access to a field or invoke a method
of an instance of a class, record, or union.
They are defined in \rsec{Class_Field_Accesses} and
\rsec{Class_Method_Calls}, respectively.

\begin{syntax}
member-access-expression:
  field-access-expression
  method-call-expression
\end{syntax}

\section{The Query Expression}
\label{The_Query_Expression}
\index{expressions!type query}
\index{? (type query)@\chpl{?} (type query)}
\index{operators!? (type query)@\chpl{?} (type query)}

A query expression is used to query a type or value within a formal
argument type expression.  The syntax of a query expression is given
by:
\begin{syntax}
query-expression:
  ? identifier[OPT]
\end{syntax}
Querying is restricted to querying the type of a formal argument, the
element type of a formal argument that is an array, the domain of a
formal argument that is an array, the size of a primitive type, or a
type or parameter field of a formal argument type.

The identifier can be omitted.  This is useful for ensuring the
genericity of a generic type that defines default values for all of
its generic fields when specifying a formal argument as discussed
in~\rsec{Formal_Arguments_of_Generic_Type}.

\begin{chapelexample}{query.chpl}
The following code defines a generic function where the type of the
first argument is queried and stored in the type alias \chpl{t} and
the domain of the second argument is queried and stored in the
variable \chpl{D}:
\begin{chapelnoprint}
{ // }
\end{chapelnoprint}
\begin{chapel}
proc foo(x: ?t, y: [?D] t) {
  for i in D do
    y[i] = x;
}
\end{chapel}
\begin{chapelnoprint}
// {
var x = 1.5;
var y: [1..4] x.type;
foo(x, y);
writeln(y);
}
\end{chapelnoprint}
This allows a generic specification of assigning a
particular value to all elements of an array.  The value and the
elements of the array are constrained to be the same type.  This
function can be rewritten without query expression as follows:
\begin{chapelnoprint}
{ // }
\end{chapelnoprint}
\begin{chapel}
proc foo(x, y: [] x.type) {
  for i in y.domain do
    y[i] = x;
}
\end{chapel}
\begin{chapelnoprint}
// {
var x = 1.5;
var y: [1..4] x.type;
foo(x, y);
writeln(y);
}
\end{chapelnoprint}
\begin{chapeloutput}
1.5 1.5 1.5 1.5
1.5 1.5 1.5 1.5
\end{chapeloutput}
\end{chapelexample}

There is an expectation that query expressions will be allowed in more
places in the future.

\section{Casts}
\label{Casts}
\index{casts}
\index{expressions!cast}
\index{: (cast)@\chpl{:} (cast)}
\index{operators!: (cast)@\chpl{:} (cast)}

A cast is specified with the following syntax:
\begin{syntax}
cast-expression:
  expression : type-specifier
\end{syntax}
The expression is converted to the specified type.  A cast expression invokes
the corresponding explicit conversion~(\rsec{Explicit_Conversions}).  A
resolution error occurs if no such conversion exists.

\section{LValue Expressions}
\label{LValue_Expressions}
\index{lvalues}
\index{expressions!lvalue}

An {\em lvalue} is an expression that can be used on the left-hand
side of an assignment statement or on either side of a swap statement,
that can be passed to a formal argument of a function that
has \chpl{out}, \chpl{inout} or \chpl{ref} intent, or that can be returned by a
variable function.  Valid lvalue expressions include the following:
\begin{itemize}
\item
 Variable expressions.
\item
 Member access expressions.
\item
 Call expressions of variable functions.
\item
 Indexing expressions.
\end{itemize}

LValue expressions are given by the following syntax:
\begin{syntax}
lvalue-expression:
  variable-expression
  member-access-expression
  call-expression
  parenthesized-expression
\end{syntax}
The syntax is less restrictive than the definition above.  For
example, not all \sntx{call-expression}s are lvalues.

\section{Precedence and Associativity}
\label{Operator_Precedence_and_Associativity}
\index{operators!precedence}
\index{operators!associativity}
\index{expressions!precedence}
\index{expressions!associativity}

The following table summarizes operator and expression precedence and
associativity.  Operators and expressions listed earlier have higher
precedence than those listed later.
\begin{center}
\begin{tabular}{|l|l|l|}
\hline
{\bf Operator} & {\bf Associativity} & {\bf Use} \\
\hline
\verb@.@ & \multirow{3}{*}{left} & member access \\
\verb@()@ & & function call or access \\
\verb@[]@ & & function call or access \\
\hline
\verb@new@ & right & constructor call \\
\hline
\verb@:@ & left & cast \\
\hline
\verb@**@ & right & exponentiation \\
\hline
\verb@reduce@ & \multirow{3}{*}{left} & reduction \\
\verb@scan@ & & scan \\
\verb@dmapped@ & & domain map application \\
\hline
\verb@!@ & \multirow{2}{*}{right} & logical negation \\
\verb@~@ & & bitwise negation \\
\hline
\verb@*@ & \multirow{3}{*}{left} & multiplication \\
\verb@/@ & & division \\
\verb@%@ & & modulus \\
\hline
unary \verb@+@ & \multirow{2}{*}{right} & positive identity \\
unary \verb@-@ & & negation \\
\hline
\verb@+@ & \multirow{2}{*}{left} & addition \\
\verb@-@ & & subtraction \\
\hline
\verb@<<@ & \multirow{2}{*}{left} & left shift \\
\verb@>>@ & & right shift \\
\hline
\verb@<=@ & \multirow{4}{*}{left} & less-than-or-equal-to comparison \\
\verb@>=@ & & greater-than-or-equal-to comparison \\
\verb@<@ & & less-than comparison \\
\verb@>@ & & greater-than comparison \\
\hline
\verb@==@ & \multirow{2}{*}{left} & equal-to comparison \\
\verb@!=@ & & not-equal-to comparison \\
\hline
\verb@&@ & left & bitwise/logical and \\
\hline
\verb@^@ & left & bitwise/logical xor \\
\hline
\verb@|@ & left & bitwise/logical or \\
\hline
\verb@&&@ & left & short-circuiting logical and \\
\hline
\verb@||@ & left & short-circuiting logical or \\
\hline
\verb@..@ & left & range construction \\
\hline
\verb@in@ & left & forall expression \\
\hline
\verb@by@ & \multirow{2}{*}{left} & range/domain stride application \\
\verb@#@ & & range count application \\
\hline
\verb@if then else@ & \multirow{5}{*}{left} & conditional expression \\
\verb@forall do@ & & forall expression \\
\verb@[ ]@ & & forall expression \\
\verb@for do@ & & for expression \\
\verb@sync single@ & & sync and single type \\
\hline
\verb@,@ & left & comma separated expressions \\
\hline
\end{tabular}
\end{center}

\begin{rationale}
In general, our operator precedence is based on that of the C family
of languages including C++, Java, Perl, and C\#.  We comment on a few
of the differences and unique factors here.

We find that there is tension between the relative precedence of
exponentiation, unary minus/plus, and casts.  The following three
expressions show our intuition for how these expressions should be
parenthesized.

\begin{center}
\begin{tabular}{lcl}
\chpl{-2**4} & wants & \chpl{-(2**4)} \\
\chpl{-2:uint} & wants & \chpl{(-2):uint} \\
\chpl{2:uint**4:uint} & wants & \chpl{(2:uint)**(4:uint)} \\
\end{tabular}
\end{center}

Trying to support all three of these cases results in a
circularity---exponentiation wants precedence over unary minus, unary
minus wants precedence over casts, and casts want precedence over
exponentiation.  We chose to break the circularity by making unary
minus have a lower precedence.  This means that for the second case
above:

\begin{center}
\begin{tabular}{lcl}
\chpl{-2:uint} & requires & \chpl{(-2):uint} \\
\end{tabular}
\end{center}

We also chose to depart from the C family of languages by making unary
plus/minus have lower precedence than binary multiplication, division,
and modulus as in Fortran.  We have found very few cases that
distinguish between these cases.  An interesting one is:

\begin{center}
\begin{tabular}{l}
\chpl{const minint = min(int(32));}\\
\chpl{...-minint/2...}
\end{tabular}
\end{center}

Intuitively, this should result in a positive value, yet C's
precedence rules results in a negative value due to asymmetry in
modern integer representations.  If we learn of cases that argue in
favor of the C approach, we would likely reverse this decision in
order to more closely match C.

We were tempted to diverge from the C precedence rules for the binary
bitwise operators to make them bind less tightly than comparisons.
This would allow us to interpret:

\begin{center}
\begin{tabular}{lcl}
\chpl{a | b == 0} & as & \chpl{(a | b) == 0} \\
\end{tabular}
\end{center}

However, given that no other popular modern language has made this
change, we felt it unwise to stray from the pack.  The typical
rationale for the C ordering is to allow these operators to be used as
non-short-circuiting logical operations.

One final area of note is the precedence of reductions.  Two common
cases tend to argue for making reductions very low or very high in the
precedence table:

\begin{center}
\begin{tabular}{lcl}
\chpl{max reduce A - min reduce A} & wants & \chpl{(max reduce A) - (min reduce A)} \\
\chpl{max reduce A * B} & wants & \chpl{max reduce (A * B)} \\
\end{tabular}
\end{center}

The first statement would require reductions to have a higher
precedence than the arithmetic operators while the second would
require them to be lower.  We opted to make reductions have high
precedence due to the argument that they tend to resemble unary
operators.  Thus, to support our intuition:

\begin{center}
\begin{tabular}{lcl}
\chpl{max reduce A * B} & requires & \chpl{max reduce (A * B)} \\
\end{tabular}
\end{center}

This choice also has the (arguably positive) effect of making the
unparenthesized version of this statement result in an aggregate value
if A and B are both aggregates---the reduction of A results in a
scalar which promotes when being multiplied by B, resulting in an
aggregate.  Our intuition is that users who forget the parenthesis
will learn of their error at compilation time because the resulting
expression is not a scalar as expected.

\end{rationale}

\section{Operator Expressions}
\label{Binary_Expressions}
\label{Unary_Expressions}
\index{expressions!operator}

\index{operators!unary}
\index{expressions!unary operator}
The application of operators to expressions is itself an expression.
The syntax of a unary expression is given by:
\begin{syntax}
unary-expression:
  unary-operator expression

unary-operator: one of
  + - ~ !
\end{syntax}

\index{operators!binary}
\index{expressions!binary operator}
The syntax of a binary expression is given by:
\begin{syntax}
binary-expression:
  expression binary-operator expression

binary-operator: one of
  + - * / % ** & | ^ << >> && || == != <= >= < > `by' #
\end{syntax}

The operators are defined in subsequent sections.

\section{Arithmetic Operators}
\label{Arithmetic_Operators}
\index{operators!arithmetic}

This section describes the predefined arithmetic operators.  These
operators can be redefined over different types using operator
overloading~(\rsec{Function_Overloading}).

For each operator, implicit conversions are applied to the operands of
an operator such that they are compatible with one of the function
forms listed, those listed earlier in the list being given
preference.  If no compatible implicit conversions exist, then a
compile-time error occurs.  In these cases, an explicit cast is required.

All integral arithmetic operators are implemented over integral types
of size 32 and 64 bits only.  For example, adding two 8-bit integers
is done by first converting them to 32-bit integers and then adding
the 32-bit integers.  The result is a 32-bit integer.

\subsection{Unary Plus Operators}
\label{Unary_Plus_Operators}
\index{+ (unary)@\chpl{+} (unary)}
\index{operators!+ (unary)@\chpl{+} (unary)}

The unary plus operators are predefined as follows:
\begin{chapel}
proc +(a: int(8)): int(8)
proc +(a: int(16)): int(16)
proc +(a: int(32)): int(32)
proc +(a: int(64)): int(64)

proc +(a: uint(8)): uint(8)
proc +(a: uint(16)): uint(16)
proc +(a: uint(32)): uint(32)
proc +(a: uint(64)): uint(64)

proc +(a: real(32)): real(32)
proc +(a: real(64)): real(64)

proc +(a: imag(32)): imag(32)
proc +(a: imag(64)): imag(64)

proc +(a: complex(64)): complex(64)
proc +(a: complex(128)): complex(128)
\end{chapel}
For each of these definitions, the result is the value of the operand.

\subsection{Unary Minus Operators}
\label{Unary_Minus_Operators}
\index{operators!negation}
\index{- (unary)@\chpl{-} (unary)}
\index{operators!- (unary)@\chpl{-} (unary)}

The unary minus operators are predefined as follows:
\begin{chapel}
proc -(a: int(8)): int(8)
proc -(a: int(16)): int(16)
proc -(a: int(32)): int(32)
proc -(a: int(64)): int(64)

proc -(a: real(32)): real(32)
proc -(a: real(64)): real(64)

proc -(a: imag(32)): imag(32)
proc -(a: imag(64)): imag(64)

proc -(a: complex(64)): complex(64)
proc -(a: complex(128)): complex(128)
\end{chapel}
For each of these definitions that return a value, the result is the
negation of the value of the operand.  For integral types, this
corresponds to subtracting the value from zero.  For real and
imaginary types, this corresponds to inverting the sign.  For complex
types, this corresponds to inverting the signs of both the real and
imaginary parts.

It is an error to try to negate a value of type \chpl{uint(64)}.  Note
that negating a value of type \chpl{uint(32)} first converts the type
to \chpl{int(64)} using an implicit conversion.

\subsection{Addition Operators}
\label{Addition_Operators}
\index{operators!addition}
\index{+@\chpl{+}}
\index{operators!+@\chpl{+}}

The addition operators are predefined as follows:
\begin{chapel}
proc +(a: int(8), b: int(8)): int(8)
proc +(a: int(16), b: int(16)): int(16)
proc +(a: int(32), b: int(32)): int(32)
proc +(a: int(64), b: int(64)): int(64)

proc +(a: uint(8), b: uint(8)): uint(8)
proc +(a: uint(16), b: uint(16)): uint(16)
proc +(a: uint(32), b: uint(32)): uint(32)
proc +(a: uint(64), b: uint(64)): uint(64)

proc +(a: real(32), b: real(32)): real(32)
proc +(a: real(64), b: real(64)): real(64)

proc +(a: imag(32), b: imag(32)): imag(32)
proc +(a: imag(64), b: imag(64)): imag(64)

proc +(a: complex(64), b: complex(64)): complex(64)
proc +(a: complex(128), b: complex(128)): complex(128)

proc +(a: real(32), b: imag(32)): complex(64)
proc +(a: imag(32), b: real(32)): complex(64)
proc +(a: real(64), b: imag(64)): complex(128)
proc +(a: imag(64), b: real(64)): complex(128)

proc +(a: real(32), b: complex(64)): complex(64)
proc +(a: complex(64), b: real(32)): complex(64)
proc +(a: real(64), b: complex(128)): complex(128)
proc +(a: complex(128), b: real(64)): complex(128)

proc +(a: imag(32), b: complex(64)): complex(64)
proc +(a: complex(64), b: imag(32)): complex(64)
proc +(a: imag(64), b: complex(128)): complex(128)
proc +(a: complex(128), b: imag(64)): complex(128)
\end{chapel}
For each of these definitions that return a value, the result is the
sum of the two operands.

It is a compile-time error to add a value of type \chpl{uint(64)} and
a value of type \chpl{int(64)}.

Addition over a value of real type and a value of imaginary type
produces a value of complex type.  Addition of values of complex type
and either real or imaginary types also produces a value of complex
type.

\subsection{Subtraction Operators}
\label{Subtraction_Operators}
\index{operators!subtraction}
\index{-@\chpl{-}}
\index{operators!-@\chpl{-}}

The subtraction operators are predefined as follows:
\begin{chapel}
proc -(a: int(8), b: int(8)): int(8)
proc -(a: int(16), b: int(16)): int(16)
proc -(a: int(32), b: int(32)): int(32)
proc -(a: int(64), b: int(64)): int(64)

proc -(a: uint(8), b: uint(8)): uint(8)
proc -(a: uint(16), b: uint(16)): uint(16)
proc -(a: uint(32), b: uint(32)): uint(32)
proc -(a: uint(64), b: uint(64)): uint(64)

proc -(a: real(32), b: real(32)): real(32)
proc -(a: real(64), b: real(64)): real(64)

proc -(a: imag(32), b: imag(32)): imag(32)
proc -(a: imag(64), b: imag(64)): imag(64)

proc -(a: complex(64), b: complex(64)): complex(64)
proc -(a: complex(128), b: complex(128)): complex(128)

proc -(a: real(32), b: imag(32)): complex(64)
proc -(a: imag(32), b: real(32)): complex(64)
proc -(a: real(64), b: imag(64)): complex(128)
proc -(a: imag(64), b: real(64)): complex(128)

proc -(a: real(32), b: complex(64)): complex(64)
proc -(a: complex(64), b: real(32)): complex(64)
proc -(a: real(64), b: complex(128)): complex(128)
proc -(a: complex(128), b: real(64)): complex(128)

proc -(a: imag(32), b: complex(64)): complex(64)
proc -(a: complex(64), b: imag(32)): complex(64)
proc -(a: imag(64), b: complex(128)): complex(128)
proc -(a: complex(128), b: imag(64)): complex(128)
\end{chapel}
For each of these definitions that return a value, the result is the
value obtained by subtracting the second operand from the first
operand.

It is a compile-time error to subtract a value of type \chpl{uint(64)}
from a value of type \chpl{int(64)}, and vice versa.

Subtraction of a value of real type from a value of imaginary type,
and vice versa, produces a value of complex type.  Subtraction of
values of complex type from either real or imaginary types, and vice
versa, also produces a value of complex type.

\subsection{Multiplication Operators}
\label{Multiplication_Operators}
\index{operators!multiplication}
\index{operators!*@\chpl{*}}
\index{*@\chpl{*}}

The multiplication operators are predefined as follows:
\begin{chapel}
proc *(a: int(8), b: int(8)): int(8)
proc *(a: int(16), b: int(16)): int(16)
proc *(a: int(32), b: int(32)): int(32)
proc *(a: int(64), b: int(64)): int(64)

proc *(a: uint(8), b: uint(8)): uint(8)
proc *(a: uint(16), b: uint(16)): uint(16)
proc *(a: uint(32), b: uint(32)): uint(32)
proc *(a: uint(64), b: uint(64)): uint(64)

proc *(a: real(32), b: real(32)): real(32)
proc *(a: real(64), b: real(64)): real(64)

proc *(a: imag(32), b: imag(32)): real(32)
proc *(a: imag(64), b: imag(64)): real(64)

proc *(a: complex(64), b: complex(64)): complex(64)
proc *(a: complex(128), b: complex(128)): complex(128)

proc *(a: real(32), b: imag(32)): imag(32)
proc *(a: imag(32), b: real(32)): imag(32)
proc *(a: real(64), b: imag(64)): imag(64)
proc *(a: imag(64), b: real(64)): imag(64)

proc *(a: real(32), b: complex(64)): complex(64)
proc *(a: complex(64), b: real(32)): complex(64)
proc *(a: real(64), b: complex(128)): complex(128)
proc *(a: complex(128), b: real(64)): complex(128)

proc *(a: imag(32), b: complex(64)): complex(64)
proc *(a: complex(64), b: imag(32)): complex(64)
proc *(a: imag(64), b: complex(128)): complex(128)
proc *(a: complex(128), b: imag(64)): complex(128)
\end{chapel}
For each of these definitions that return a value, the result is the
product of the two operands.

It is a compile-time error to multiply a value of type \chpl{uint(64)} and
a value of type \chpl{int(64)}.

Multiplication of values of imaginary type produces a value of real
type.  Multiplication over a value of real type and a value of
imaginary type produces a value of imaginary type.  Multiplication of
values of complex type and either real or imaginary types produces a
value of complex type.

\subsection{Division Operators}
\label{Division_Operators}
\index{operators!division}
\index{/@\chpl{/}}
\index{operators!/@\chpl{/}}

The division operators are predefined as follows:
\begin{chapel}
proc /(a: int(8), b: int(8)): int(8)
proc /(a: int(16), b: int(16)): int(16)
proc /(a: int(32), b: int(32)): int(32)
proc /(a: int(64), b: int(64)): int(64)

proc /(a: uint(8), b: uint(8)): uint(8)
proc /(a: uint(16), b: uint(16)): uint(16)
proc /(a: uint(32), b: uint(32)): uint(32)
proc /(a: uint(64), b: uint(64)): uint(64)

proc /(a: real(32), b: real(32)): real(32)
proc /(a: real(64), b: real(64)): real(64)

proc /(a: imag(32), b: imag(32)): real(32)
proc /(a: imag(64), b: imag(64)): real(64)

proc /(a: complex(64), b: complex(64)): complex(64)
proc /(a: complex(128), b: complex(128)): complex(128)

proc /(a: real(32), b: imag(32)): imag(32)
proc /(a: imag(32), b: real(32)): imag(32)
proc /(a: real(64), b: imag(64)): imag(64)
proc /(a: imag(64), b: real(64)): imag(64)

proc /(a: real(32), b: complex(64)): complex(64)
proc /(a: complex(64), b: real(32)): complex(64)
proc /(a: real(64), b: complex(128)): complex(128)
proc /(a: complex(128), b: real(64)): complex(128)

proc /(a: imag(32), b: complex(64)): complex(64)
proc /(a: complex(64), b: imag(32)): complex(64)
proc /(a: imag(64), b: complex(128)): complex(128)
proc /(a: complex(128), b: imag(64)): complex(128)
\end{chapel}
For each of these definitions that return a value, the result is the
quotient of the two operands.

It is a compile-time error to divide a value of type \chpl{uint(64)} by
a value of type \chpl{int(64)}, and vice versa.

Division of values of imaginary type produces a value of real type.
Division over a value of real type and a value of imaginary type
produces a value of imaginary type.  Division of values of complex
type and either real or imaginary types produces a value of complex
type.

When the operands are integers, the result (quotient) is also an integer.  If \chpl{b}
does not divide \chpl{a} exactly, then there are two candidate quotients $q1$ and $q2$
such that $b * q1$ and $b * q2$ are the two multiples of \chpl{b} closest to \chpl{a}.
The integer result $q$ is the candidate quotient which lies closest to zero.

\subsection{Modulus Operators}
\label{Modulus_Operators}
\index{operators!modulus}
\index{\%@\chpl{\%}}
\index{operators!\%@\chpl{\%}}

The modulus operators are predefined as follows:
\begin{chapel}
proc %(a: int(8), b: int(8)): int(8)
proc %(a: int(16), b: int(16)): int(16)
proc %(a: int(32), b: int(32)): int(32)
proc %(a: int(64), b: int(64)): int(64)

proc %(a: uint(8), b: uint(8)): uint(8)
proc %(a: uint(16), b: uint(16)): uint(16)
proc %(a: uint(32), b: uint(32)): uint(32)
proc %(a: uint(64), b: uint(64)): uint(64)
\end{chapel}
For each of these definitions that return a value, the result is the
remainder when the first operand is divided by the second operand.

The sign of the result is the same as the sign of the dividend \chpl{a}, and the
magnitude of the result is always smaller than that of the divisor \chpl{b}.
For integer operands, the \chpl{\%} and \chpl{/} operators are related by the
following identity:
\begin{chapel}
var q = a / b;
var r = a % b;
writeln(q * b + r == a);    // true
\end{chapel}

It is a compile-time error to take the remainder of a value of
type \chpl{uint(64)} and a value of type \chpl{int(64)}, and vice
versa.

There is an expectation that the predefined modulus operators will be
extended to handle real, imaginary, and complex types in the future.

\subsection{Exponentiation Operators}
\label{Exponentiation_Operators}
\index{operators!exponentiation}
\index{**@\chpl{**}}
\index{operators!**@\chpl{**}}

The exponentiation operators are predefined as follows:
\begin{chapel}
proc **(a: int(8), b: int(8)): int(8)
proc **(a: int(16), b: int(16)): int(16)
proc **(a: int(32), b: int(32)): int(32)
proc **(a: int(64), b: int(64)): int(64)

proc **(a: uint(8), b: uint(8)): uint(8)
proc **(a: uint(16), b: uint(16)): uint(16)
proc **(a: uint(32), b: uint(32)): uint(32)
proc **(a: uint(64), b: uint(64)): uint(64)

proc **(a: real(32), b: real(32)): real(32)
proc **(a: real(64), b: real(64)): real(64)
\end{chapel}
For each of these definitions that return a value, the result is the
value of the first operand raised to the power of the second operand.

It is a compile-time error to take the exponent of a value of
type \chpl{uint(64)} by a value of type \chpl{int(64)}, and vice
versa.

There is an expectation that the predefined exponentiation operators
will be extended to handle imaginary and complex types in the future.

\section{Bitwise Operators}
\label{Bitwise_Operators}
\index{operators!bitwise}

This section describes the predefined bitwise operators.  These
operators can be redefined over different types using operator
overloading~(\rsec{Function_Overloading}).

\subsection{Bitwise Complement Operators}
\label{Bitwise_Complement_Operators}
\index{operators!bitwise!complement}
\index{\~@\chpl{\~}}
\index{operators!\~@\chpl{\~}}

The bitwise complement operators are predefined as follows:
\begin{chapel}
proc ~(a: bool): bool

proc ~(a: int(8)): int(8)
proc ~(a: int(16)): int(16)
proc ~(a: int(32)): int(32)
proc ~(a: int(64)): int(64)

proc ~(a: uint(8)): uint(8)
proc ~(a: uint(16)): uint(16)
proc ~(a: uint(32)): uint(32)
proc ~(a: uint(64)): uint(64)
\end{chapel}
For each of these definitions, the result is the bitwise complement of
the operand.

\subsection{Bitwise And Operators}
\label{Bitwise_And_Operators}
\index{operators!bitwise!and}
\index{&@\chpl{&}}
\index{operators!&@\chpl{&}}

The bitwise and operators are predefined as follows:
\begin{chapel}
proc &(a: bool, b: bool): bool

proc &(a: int(?w), b: int(w)): int(w)
proc &(a: uint(?w), b: uint(w)): uint(w)

proc &(a: int(?w), b: uint(w)): uint(w)
proc &(a: uint(?w), b: int(w)): uint(w)
\end{chapel}
For each of these definitions, the result is
computed by applying the logical and operation to the bits of the
operands.

Chapel allows mixing signed and unsigned integers of the same size
when passing them as arguments to bitwise and.
In the mixed case the result is of the same size as the arguments
and is unsigned.
No run-time error is issued, even if the apparent sign changes as the
required conversions are performed.

\begin{rationale}
The mathematical meaning of integer arguments
is discarded when they are passed to bitwise operators.
Instead the arguments are treated simply as bit vectors.
The bit-vector meaning is preserved when converting
between signed and unsigned of the same size.
The choice of unsigned over signed as the result type in the mixed case
reflects the semantics of standard C.
\end{rationale}

\subsection{Bitwise Or Operators}
\label{Bitwise_Or_Operators}
\index{operators!bitwise!or}
\index{|@\chpl{|}}
\index{operators!|@\chpl{|}}

The bitwise or operators are predefined as follows:
\begin{chapel}
proc |(a: bool, b: bool): bool

proc |(a: int(?w), b: int(w)): int(w)
proc |(a: uint(?w), b: uint(w)): uint(w)

proc |(a: int(?w), b: uint(w)): uint(w)
proc |(a: uint(?w), b: int(w)): uint(w)
\end{chapel}

For each of these definitions, the result is
computed by applying the logical or operation to the bits of the
operands.
Chapel allows mixing signed and unsigned integers of the same size
when passing them as arguments to bitwise or.
No run-time error is issued, even if the apparent sign changes as the
required conversions are performed.

\begin{rationale}
The same as for bitwise and (\rsec{Bitwise_And_Operators}).
\end{rationale}

\subsection{Bitwise Xor Operators}
\label{Bitwise_Xor_Operators}
\index{operators!bitwise!exclusive or}
\index{^@\chpl{^}}
\index{operators!^@\chpl{^}}

The bitwise xor operators are predefined as follows:
\begin{chapel}
proc ^(a: bool, b: bool): bool

proc ^(a: int(?w), b: int(w)): int(w)
proc ^(a: uint(?w), b: uint(w)): uint(w)

proc ^(a: int(?w), b: uint(w)): uint(w)
proc ^(a: uint(?w), b: int(w)): uint(w)
\end{chapel}

For each of these definitions, the result is
computed by applying the XOR operation to the bits of the operands.
Chapel allows mixing signed and unsigned integers of the same size
when passing them as arguments to bitwise xor.
No run-time error is issued, even if the apparent sign changes as the required
conversions are performed.

\begin{rationale}
The same as for bitwise and (\rsec{Bitwise_And_Operators}).
\end{rationale}

\section{Shift Operators}
\label{Shift_Operators}
\index{operators!shift}
\index{<<@\chpl{<<}}
\index{operators!<<@\chpl{<<}}
\index{>>@\chpl{>>}}
\index{operators!>>@\chpl{>>}}

This section describes the predefined shift operators.  These
operators can be redefined over different types using operator
overloading~(\rsec{Function_Overloading}).

The shift operators are predefined as follows:
\begin{chapel}
proc <<(a: int(8), b): int(8)
proc <<(a: int(16), b): int(16)
proc <<(a: int(32), b): int(32)
proc <<(a: int(64), b): int(64)

proc <<(a: uint(8), b): uint(8)
proc <<(a: uint(16), b): uint(16)
proc <<(a: uint(32), b): uint(32)
proc <<(a: uint(64), b): uint(64)

proc >>(a: int(8), b): int(8)
proc >>(a: int(16), b): int(16)
proc >>(a: int(32), b): int(32)
proc >>(a: int(64), b): int(64)

proc >>(a: uint(8), b): uint(8)
proc >>(a: uint(16), b): uint(16)
proc >>(a: uint(32), b): uint(32)
proc >>(a: uint(64), b): uint(64)
\end{chapel}
The type of the second actual argument must be any integral type.

The \chpl{<<} operator shifts the bits of \chpl{a} left by the
integer \chpl{b}.  The new low-order bits are set to zero.

The \chpl{>>} operator shifts the bits of \chpl{a} right by the
integer \chpl{b}.  When \chpl{a} is negative, the new high-order bits
are set to one; otherwise the new high-order bits are set to zero.

The value of \chpl{b} must be non-negative.

\section{Logical Operators}
\label{Logical_Operators}
\index{operators!logical}

This section describes the predefined logical operators.  These
operators can be redefined over different types using operator
overloading~(\rsec{Function_Overloading}).

\subsection{The Logical Negation Operator}
\label{Logical_Negation_Operators}
\index{operators!logical!not}
\index{\!@\chpl{!}}
\index{operators!\!@\chpl{!}}

The logical negation operator is predefined as follows:
\begin{chapel}
proc !(a: bool): bool
\end{chapel}
The result is the logical negation of the operand.

\subsection{The Logical And Operator}
\label{Logical_And_Operators}
\index{operators!logical!and}
\index{&&@\chpl{&&}}
\index{operators!&&@\chpl{&&}}

The logical and operator is predefined over bool type.  It returns
true if both operands evaluate to true; otherwise it returns false.
If the first operand evaluates to false, the second operand is not
evaluated and the result is false.
%% hilde sez: In the interest of supporting parallel execution, we should leave
%% unspecified whether the right operand is evaluated.
%% Where sufficient processing resources are available, it is faster on average
%% to evaluate both the left and right operands and perform the conjunction or
%% disjunction than to block until the value of the left operand is known and
%% only then commence to evaluate the right operand.

The logical and operator over expressions \chpl{a} and \chpl{b} given
by
\begin{chapel}
a && b
\end{chapel}
is evaluated as the expression
\begin{chapel}
if isTrue(a) then isTrue(b) else false
\end{chapel}

The function \chpl{isTrue} is predefined over bool type as follows:
\begin{chapel}
proc isTrue(a:bool) return a;
\end{chapel}
Overloading the logical and operator over other types is accomplished
by overloading the \chpl{isTrue} function over other types.

\subsection{The Logical Or Operator}
\label{Logical_Or_Operators}
\index{operators!logical!or}
\index{||@\chpl{||}}
\index{operators!||@\chpl{||}}


The logical or operator is predefined over bool type.  It returns
true if either operand evaluate to true; otherwise it returns false.
If the first operand evaluates to true, the second operand is not
evaluated and the result is true.

The logical or operator over expressions \chpl{a} and \chpl{b} given
by
\begin{chapel}
a || b
\end{chapel}
is evaluated as the expression
\begin{chapel}
if isTrue(a) then true else isTrue(b)
\end{chapel}

The function \chpl{isTrue} is predefined over bool type as described
in~\rsec{Logical_And_Operators}.  Overloading the logical or operator
over other types is accomplished by overloading the \chpl{isTrue}
function over other types.

\section{Relational Operators}
\label{Relational_Operators}
\index{operators!relational}

This section describes the predefined relational operators.  These
operators can be redefined over different types using operator
overloading~(\rsec{Function_Overloading}).

\subsection{Ordered Comparison Operators}
\label{Ordered_Comparison_Operators}

\index{operators!less than}
\index{<@\chpl{<}}
\index{operators!<@\chpl{<}}
The ``less than'' comparison operators are predefined over numeric
types as follows:
\begin{chapel}
proc <(a: int(8), b: int(8)): bool
proc <(a: int(16), b: int(16)): bool
proc <(a: int(32), b: int(32)): bool
proc <(a: int(64), b: int(64)): bool

proc <(a: uint(8), b: uint(8)): bool
proc <(a: uint(16), b: uint(16)): bool
proc <(a: uint(32), b: uint(32)): bool
proc <(a: uint(64), b: uint(64)): bool

proc <(a: real(32), b: real(32)): bool
proc <(a: real(64), b: real(64)): bool

proc <(a: imag(32), b: imag(32)): bool
proc <(a: imag(64), b: imag(64)): bool
\end{chapel}
The result of \chpl{a < b} is true if \chpl{a} is less than \chpl{b};
otherwise the result is false.

\index{operators!greater than}
\index{>@\chpl{>}}
\index{operators!>@\chpl{>}}
The ``greater than'' comparison operators are predefined over numeric
types as follows:
\begin{chapel}
proc >(a: int(8), b: int(8)): bool
proc >(a: int(16), b: int(16)): bool
proc >(a: int(32), b: int(32)): bool
proc >(a: int(64), b: int(64)): bool

proc >(a: uint(8), b: uint(8)): bool
proc >(a: uint(16), b: uint(16)): bool
proc >(a: uint(32), b: uint(32)): bool
proc >(a: uint(64), b: uint(64)): bool

proc >(a: real(32), b: real(32)): bool
proc >(a: real(64), b: real(64)): bool

proc >(a: imag(32), b: imag(32)): bool
proc >(a: imag(64), b: imag(64)): bool
\end{chapel}
The result of \chpl{a > b} is true if \chpl{a} is greater
than \chpl{b}; otherwise the result is false.

\index{operators!less than or equal}
\index{<=@\chpl{<=}}
\index{operators!<=@\chpl{<=}}
The ``less than or equal to'' comparison operators are predefined over
numeric types as follows:
\begin{chapel}
proc <=(a: int(8), b: int(8)): bool
proc <=(a: int(16), b: int(16)): bool
proc <=(a: int(32), b: int(32)): bool
proc <=(a: int(64), b: int(64)): bool

proc <=(a: uint(8), b: uint(8)): bool
proc <=(a: uint(16), b: uint(16)): bool
proc <=(a: uint(32), b: uint(32)): bool
proc <=(a: uint(64), b: uint(64)): bool

proc <=(a: real(32), b: real(32)): bool
proc <=(a: real(64), b: real(64)): bool

proc <=(a: imag(32), b: imag(32)): bool
proc <=(a: imag(64), b: imag(64)): bool
\end{chapel}
The result of \chpl{a <= b} is true if \chpl{a} is less than or equal
to \chpl{b}; otherwise the result is false.

\index{operators!greater than or equal}
\index{>=@\chpl{>=}}
\index{operators!>=@\chpl{>=}}
The ``greater than or equal to'' comparison operators are predefined
over numeric types as follows:
\begin{chapel}
proc >=(a: int(8), b: int(8)): bool
proc >=(a: int(16), b: int(16)): bool
proc >=(a: int(32), b: int(32)): bool
proc >=(a: int(64), b: int(64)): bool

proc >=(a: uint(8), b: uint(8)): bool
proc >=(a: uint(16), b: uint(16)): bool
proc >=(a: uint(32), b: uint(32)): bool
proc >=(a: uint(64), b: uint(64)): bool

proc >=(a: real(32), b: real(32)): bool
proc >=(a: real(64), b: real(64)): bool

proc >=(a: imag(32), b: imag(32)): bool
proc >=(a: imag(64), b: imag(64)): bool
\end{chapel}
The result of \chpl{a >= b} is true if \chpl{a} is greater than or
equal to \chpl{b}; otherwise the result is false.

The ordered comparison operators are predefined over strings as follows:
\begin{chapel}
proc <(a: string, b: string): bool
proc >(a: string, b: string): bool
proc <=(a: string, b: string): bool
proc >=(a: string, b: string): bool
\end{chapel}
Comparisons between strings are defined based on the ordering of the
character set used to represent the string, which is applied
elementwise to the string's characters in order.


\subsection{Equality Comparison Operators}
\label{Equality_Comparison_Operators}
\index{operators!equality}
\index{==@\chpl{==}}
\index{operators!==@\chpl{==}}
\index{"!=@\chpl{"\"!=}}
\index{operators!"!=@\chpl{"\"!=}}

The equality comparison operators \chpl{==} and \chpl{\!=} are predefined over bool and the
numeric types as follows:
\begin{chapel}
proc ==(a: int(8), b: int(8)): bool
proc ==(a: int(16), b: int(16)): bool
proc ==(a: int(32), b: int(32)): bool
proc ==(a: int(64), b: int(64)): bool

proc ==(a: uint(8), b: uint(8)): bool
proc ==(a: uint(16), b: uint(16)): bool
proc ==(a: uint(32), b: uint(32)): bool
proc ==(a: uint(64), b: uint(64)): bool

proc ==(a: real(32), b: real(32)): bool
proc ==(a: real(64), b: real(64)): bool

proc ==(a: imag(32), b: imag(32)): bool
proc ==(a: imag(64), b: imag(64)): bool

proc ==(a: complex(64), b: complex(64)): bool
proc ==(a: complex(128), b: complex(128)): bool

proc !=(a: int(8), b: int(8)): bool
proc !=(a: int(16), b: int(16)): bool
proc !=(a: int(32), b: int(32)): bool
proc !=(a: int(64), b: int(64)): bool

proc !=(a: uint(8), b: uint(8)): bool
proc !=(a: uint(16), b: uint(16)): bool
proc !=(a: uint(32), b: uint(32)): bool
proc !=(a: uint(64), b: uint(64)): bool

proc !=(a: real(32), b: real(32)): bool
proc !=(a: real(64), b: real(64)): bool

proc !=(a: imag(32), b: imag(32)): bool
proc !=(a: imag(64), b: imag(64)): bool

proc !=(a: complex(64), b: complex(64)): bool
proc !=(a: complex(128), b: complex(128)): bool
\end{chapel}
The result of \chpl{a == b} is true if \chpl{a} and \chpl{b} contain
the same value; otherwise the result is false.  The result of \chpl{a
\!= b} is equivalent to \chpl{\!(a == b)}.

The equality comparison operators are predefined over classes as
follows:
\begin{chapel}
proc ==(a: object, b: object): bool
proc !=(a: object, b: object): bool
\end{chapel}
The result of \chpl{a == b} is true if \chpl{a} and \chpl{b} reference
the same storage location; otherwise the result is false.  The result
of \chpl{a \!= b} is equivalent to \chpl{\!(a == b)}.

Default equality comparison operators are generated for records if the
user does not define them.  These operators are described
in~\rsec{Record_Comparison_Operators}.

\index{== (string)@\chpl{==} (string)}
\index{operators!== (string)@\chpl{==} (string)}
\index{"!= (string)@\chpl{"\"!=} (string)}
\index{operators!"!= (string)@\chpl{"\"!=} (string)}
The equality comparison operators are predefined over strings as
follows:
\begin{chapel}
proc ==(a: string, b: string): bool
proc !=(a: string, b: string): bool
\end{chapel}
The result of \chpl{a == b} is true if the sequence of characters
in \chpl{a} matches exactly the sequence of characters in \chpl{b};
otherwise the result is false.  The result of \chpl{a \!= b} is
equivalent to \chpl{\!(a == b)}.

\section{Miscellaneous Operators}
\label{Miscellaneous_Operators}

This section describes several miscellaneous operators.  These
operators can be redefined over different types using operator
overloading~(\rsec{Function_Overloading}).

\subsection{The String Concatenation Operator}
\label{The_String_Concatenation_Operator}
\index{operators!string concatenation}
\index{operators!concatenation!string}
\index{operators!+ (string)@\chpl{+} (string)}

The string concatenation operator \chpl{+} is predefined over numeric, boolean,
and enumerated types with strings. It casts its operands to string type and
concatenates them together.

\begin{chapelexample}{string-concat.chpl}
The code
\begin{chapelnoprint}
var i:int = 3;
writeln(
\end{chapelnoprint}
\begin{chapel}
"result: "+i
\end{chapel}
\begin{chapelnoprint}
);
\end{chapelnoprint}
\begin{chapeloutput}
result: 3
\end{chapeloutput}
where \chpl{i} is an integer appends the string representation of \chpl{i} to the
string literal \chpl{"result: "}.  If \chpl{i} is \chpl{3}, then the resulting string
would be \chpl{"result: 3"}.
\begin{chapelnoprint}
\end{chapelnoprint}
\end{chapelexample}

\subsection{The By Operator}
\label{The_By_Operator}
\index{by@\chpl{by}}
\index{operators!by@\chpl{by}}

The operator \chpl{by} is predefined on ranges and rectangular domains.
It is described in~\rsec{By_Operator_For_Ranges} for ranges
and~\rsec{Domain_Striding} for domains.

\subsection{The Range Count Operator}
\label{The_Range_Count_Operator}
\index{operators!range!count}
\index{#@\chpl{#}}
\index{operators!#@\chpl{#}}

The operator \chpl{#} is predefined on ranges. It is described
in ~\rsec{Count_Operator}.

\section{Let Expressions}
\label{Let_Expressions}
\index{let@\chpl{let}}
\index{operators!let@\chpl{let}}

A let expression allows variables to be declared at the expression
level and used within that expression.  The syntax of a let expression
is given by:
\begin{syntax}
let-expression:
  `let' variable-declaration-list `in' expression
\end{syntax}
The scope of the variables is the let-expression.
\begin{chapelexample}{let.chpl}
Let expressions are useful for defining variables in the context of
an expression.  In the code
\begin{chapelnoprint}
  var a = 4;
  var b = 5;
  var l =
\end{chapelnoprint}
\begin{chapel}
  let x: real = a*b, y = x*x in 1/y
\end{chapel}
the value determined by \chpl{a*b} is computed and converted to type
real if it is not already a real.  The square of the real is then
stored in \chpl{y} and the result of the expression is the reciprocal
of that value.
\begin{chapelnoprint}
  ;
  writeln(l);
\end{chapelnoprint}
\begin{chapeloutput}
0.0025
\end{chapeloutput}
\end{chapelexample}

\section{Conditional Expressions}
\label{Conditional_Expressions}
\index{conditional expressions}
\index{expressions!conditional}
\index{expressions!if-then-else}
\index{if@\chpl{if}}
\index{then@\chpl{then}}
\index{else@\chpl{else}}

A conditional expression is given by the following syntax:
\begin{syntax}
if-expression:
  `if' expression `then' expression `else' expression
  `if' expression `then' expression
\end{syntax}
The conditional expression is evaluated in two steps.  First, the
expression following the \chpl{if} keyword is evaluated.  Then, if the
expression evaluated to true, the expression following the \chpl{then}
keyword is evaluated and taken to be the value of this expression.
Otherwise, the expression following the \chpl{else} keyword is
evaluated and taken to be the value of this expression.  In both
cases, the unselected expression is not evaluated.

The `else' clause can be omitted only when the conditional expression
is nested immediately inside a for or forall expression.  Such an expression
is used to filter predicates as described
in~\rsec{Filtering_Predicates_For} and~\rsec{Filtering_Predicates_Forall},
respectively.

\begin{chapelexample}{condexp.chpl}
This example shows how if-then-else can be used in a context in which an
expression is expected.
\begin{chapel}
writehalf(8);
writehalf(21);
writehalf(1000);

proc writehalf(i: int) {
  var half = if (i % 2) then i/2 +1 else i/2;
  writeln("Half of ",i," is ",half); 
}
\end{chapel}
\begin{chapelprintoutput}
Half of 8 is 4\\
Half of 21 is 11\\
Half of 1000 is 500\\
\end{chapelprintoutput}
\end{chapelexample}

\section{For Expressions}
\label{For_Expressions}
\index{for@\chpl{for}}
\index{expressions!for@\chpl{for}}

A for expression is given by the following syntax:
\begin{syntax}
for-expression:
  `for' index-var-declaration `in' iteratable-expression `do' expression
  `for' iteratable-expression `do' expression
\end{syntax}
The for expression executes a for loop (\rsec{The_For_Loop}),
evaluates the body expression on each iteration of the loop,
and returns the resulting values as a collection.
The size and shape of that collection
are determined by the iteratable-expression.

\subsection{Filtering Predicates in For Expressions}
\label{Filtering_Predicates_For}
\index{for@\chpl{for}!filtering predicates}
\index{expressions!for@\chpl{for}!filtering predicates}

A conditional expression that is immediately enclosed in a for
expression and does not require an else clause filters the iterations of the for expression.
The iterations for which the condition does not hold
are not reflected in the result of the for expression.

\begin{chapelexample}{yieldPredicates.chpl}
The code
\begin{chapel}
var A = for i in 1..10 do if i % 3 != 0 then i;
\end{chapel}
\begin{chapelpost}
writeln(A);
\end{chapelpost}
\begin{chapeloutput}
1 2 4 5 7 8 10
\end{chapeloutput}
declares an array A that is initialized to the integers between
1 and 10 that are not divisible by 3.
\end{chapelexample}

\cleardoublepage
\sekshun{Statements}
\label{Statements}

\index{statement}

Chapel is an imperative language with statements that may have side
effects.  Statements allow for the sequencing of program execution.
They are as follows:
\begin{syntax}
statement:
  block-statement
  expression-statement
  assignment-statement
  swap-statement
  conditional-statement
  select-statement
  while-do-statement
  do-while-statement
  for-statement
  label-statement
  break-statement
  continue-statement
  param-for-statement
  return-statement
  yield-statement
  module-declaration-statement
  function-declaration-statement
  method-declaration-statement
  type-declaration-statement
  variable-declaration-statement
  remote-variable-declaration-statement
  use-statement
  type-select-statement
  empty-statement
  parallel-statement
  on-statement
\end{syntax}

The declaration statements are discussed in the sections that define
what they declare.  Module declaration statements are defined
in~\rsec{Modules}.  Function declaration statements are defined
in~\rsec{Functions}.  Method declaration statements are defined
in~\rsec{Class_Methods}.  Type declaration statements are defined
in~\rsec{Types}.  Variable declaration statements are defined
in~\rsec{Variables}.  Remote variable declaration statements are
defined in~\rsec{remote_variable_declarations}.  Tuple variable
declaration statements are defined
in~\rsec{Variable_Declarations_in_a_Tuple}.  Return statements are
defined in~\rsec{The_Return_Statement}.  Yield statements are defined
in~\rsec{The_Yield_Statement}.  The \sntx{parallel-statement} consists
of statements that create or limit parallelism.  These statements are
described in~\rsec{Task_Parallelism_and_Synchronization}
and~\rsec{Data_Parallelism}.  The \sntx{on-statement} is defined
in~\rsec{On}.  The compiler error statement is defined
in~\rsec{User_Defined_Compiler_Errors}.

\subsection{Blocks}
\label{Blocks}

\index{block}

A block is a statement or a possibly empty list of statements that
form their own scope.  A block is given by
\begin{syntax}
block-statement:
  { statements[OPT] }

statements:
  statement
  statement statements
\end{syntax}

Variables defined within a block are local
variables~(\rsec{Local_Variables}).

The statements within a block are executed serially unless the block
is in a cobegin statement~(\rsec{Cobegin}).

\subsection{Expression Statements}
\label{Expression_Statements}

\index{expression statement}
\index{expression!as a statement}
The expression statement evaluates an expression solely for side
effects. The syntax for an expression statement is given by
\begin{syntax}
expression-statement:
  expression ;
\end{syntax}

\subsection{Assignment Statements}
\label{Assignment_Statements}
\index{assignment}

An assignment statement assigns the value of an expression to another
expression that can appear on the left-hand side of the operator, for
example, a variable.  Assignment statements are given by

\index{=@\chpl{=}}
\index{+=@\chpl{+=}}
\index{-=@\chpl{-=}}
\index{*=@\chpl{*=}}
\index{/=@\chpl{/=}}
\index{\%=@\chpl{\%=}}
\index{**=@\chpl{**=}}
\index{&=@\chpl{&=}}
\index{|=@\chpl{|=}}
\index{^=@\chpl{^=}}
\index{||=@\chpl{||=}}
\index{&&=@\chpl{&&=}}
\index{<<=@\chpl{<<=}}
\index{>>=@\chpl{>>=}}
\begin{syntax}
assignment-statement:
  lvalue-expression assignment-operator expression

assignment-operator: one of
   = += -= *= /= %= **= &= |= ^= &&= ||= <<= >>=
\end{syntax}

The expression on the left-hand side of the assignment operator must
be a valid lvalue~(\rsec{lvalue}).  It is evaluated before the
expression on the right-hand side of the assignment operator, which
can be any expression.

The assignment operators that contain a binary operator as a prefix is
a short-hand for applying the binary operator to the left and
right-hand side expressions and then assigning the value of that
application to the already evaluated left-hand side.  Thus, for
example, \chpl{x += y} is equivalent to \chpl{x = x + y} where the
expression \chpl{x} is evaluated once.

In a compound assignment, a cast to the type on the left-hand side is
inserted before the simple assignment if the operator is a shift or
both the right-hand side expression can be assigned to the left-hand
side expression and the type of the left-hand side is a primitive
type.

\begin{rationale}
This cast is necessary to handle \chpl{+=} where the type of the
left-hand side is, for example, \chpl{int(8)} because the \chpl{+}
operator is defined on \chpl{int(32)}, not \chpl{int(8)}.
\end{rationale}

Values of one primitive or enumerated type can be assigned to another
primitive or enumerated type if an implicit coercion exists between
those types~(\rsec{Implicit_Conversions}).

The validity and semantics of assigning between
classes~(\rsec{Class_Assignment}), records~(\rsec{Record_Assignment}),
unions~(\rsec{Union_Assignment}), tuples~(\rsec{Tuple_Assignment}),
ranges~(\rsec{Range_Assignment}),
domains~(\rsec{Domain_Assignment}), and arrays~(\rsec{Array_Assignment})
is discussed in these later sections.

\subsection{The Swap Statement}
\label{The_Swap_Statement}
\index{swap!statement}
\index{swap!operator}
The swap statement indicates to swap the values in the expressions
on either side of the swap operator.  Since both expressions are assigned
to, each must be a valid lvalue expression~(\rsec{lvalue}).
\begin{syntax}
swap-statement:
  lvalue-expression swap-operator lvalue-expression

swap-operator:
  <=>
\end{syntax}

To implement the swap operation, the compiler uses temporary variables
as necessary.

\begin{example}
The following swap statement
\begin{chapel}
var a, b: real;

a <=> b;
\end{chapel}
is semantically equivalent to:
\begin{chapel}
const t = b;
b = a;
a = t;
\end{chapel}
\end{example}

\subsection{The Conditional Statement}
\label{The_Conditional_Statement}

\index{if@\chpl{if}}
\index{then@\chpl{then}}
\index{else@\chpl{else}}
\index{conditional!statement}
The conditional statement allows execution to choose between two
statements based on the evaluation of an expression of \chpl{bool}
type. The syntax for a conditional statement is given by
\begin{syntax}
conditional-statement:
  `if' expression `then' statement else-part[OPT]
  `if' expression block-statement else-part[OPT]

else-part:
  `else' statement
\end{syntax}

A conditional statement evaluates an expression of bool type. If the
expression evaluates to true, the first statement in the conditional
statement is executed.  If the expression evaluates to false and the
optional else-clause exists, the statement following the
\chpl{else} keyword is executed.

If the expression is a parameter, the conditional statement is folded
by the compiler. If the expression evaluates to true, the first
statement replaces the conditional statement. If the expression
evaluates to false, the second statement, if it exists, replaces the
conditional statement; if the second statement does not exist, the
conditional statement is removed.

\index{conditional statement!dangling else}
If the statement that immediately follows the optional \chpl{then}
keyword is a conditional statement and it is not in a block, the
else-clause is bound to the nearest preceding conditional statement
without an else-clause.

Each statement embedded in the {\em conditional-statement} has its own
scope whether or not an explicit block surrounds it.

\subsection{The Select Statement}
\label{The_Select_Statement}

\index{select@\chpl{select}}
\index{when@\chpl{when}}

The select statement is a multi-way variant of the conditional
statement.  The syntax is given by:
\begin{syntax}
select-statement:
  `select' expression { when-statements }

when-statements:
  when-statement
  when-statement when-statements

when-statement:
  `when' expression-list `do' statement
  `when' expression-list block-statement
  `otherwise' statement

expression-list:
  expression
  expression , expression-list
\end{syntax}
The expression that follows the keyword \chpl{select}, the select
expression, is compared with the list of expressions following the
keyword \chpl{when}, the case expressions, using the equality
operator \chpl{==}.  If the expressions cannot be compared with the
equality operator, a compile-time error is generated.  The first case
expression that contains an expression where that comparison
is \chpl{true} will be selected and control transferred to the
associated statement.  If the comparison is always \chpl{false}, the
statement associated with the keyword \chpl{otherwise}, if it exists,
will be selected and control transferred to it.  There may be at most
one \chpl{otherwise} statement and its location within the select
statement does not matter.

Each statement embedded in the {\em when-statement} has its own scope
whether or not an explicit block surrounds it.

\subsection{The While and Do While Loops}
\label{The_While_and_Do_While_Loops}

\index{while loops}
\index{while@\chpl{while}}

There are two variants of the while loop in Chapel.  The syntax of the
while-do loop is given by:
\begin{syntax}
while-do-statement:
  `while' expression `do' statement
  `while' expression block-statement
\end{syntax}
The syntax of the do-while loop is given by:
\begin{syntax}
do-while-statement:
  `do' statement `while' expression ;
\end{syntax}
In both variants, the expression evaluates to a value of type \chpl{bool}
which determines when the loop terminates and control continues with
the statement following the loop.

The while-do loop is executed as follows:
\begin{enumerate}
\item The expression is evaluated.
\item If the expression evaluates to \chpl{false}, the statement is
  not executed and control continues to the statement following the
  loop.
\item If the expression evaluates to \chpl{true}, the statement is
  executed and control continues to step 1, evaluating the expression
  again.
\end{enumerate}

The do-while loop is executed as follows:
\begin{enumerate}
\item The statement is executed.
\item The expression is evaluated.
\item If the expression evaluates to \chpl{false}, control continues
  to the statement following the loop.
\item If the expression evaluates to \chpl{true}, control continues to
  step 1 and the the statement is executed again.
\end{enumerate}
In this second form of the loop, note that the statement is executed
unconditionally the first time.

\subsection{The For Loop}
\label{The_For_Loop}

\index{for@\chpl{for}}
\index{for loops}

The for loop iterates over ranges, domains, arrays, iterators, or any
class that implements an iterator named \chpl{these}.  The syntax of
the for loop is given by:
\begin{syntax}
for-statement:
  `for' loop-control-part loop-body-part

loop-control-part:
  index-expression `in' iterator-expression
  iterator-expression

loop-body-part:
  `do' statement
  block-statement

index-expression:
  expression

iterator-expression:
  expression
\end{syntax}

The index-expression declares new variables for the scope of the loop.
It may specify a new identifier.  Alternatively, the index-expression
may specify multiple identifiers grouped using a tuple notation in
order to destructure the values returned by the iterator expression,
as described in~\rsec{Indices_in_a_Tuple}.

The index-expression is optional and may be omitted if the indices do
not need to be referenced in the loop.

If the iterator-expression is a tuple delimited by parentheses, the
components of the tuple must support iteration, e.g., a tuple of
arrays, and those components are iterated over using a zipper
iteration defined in~\rsec{Zipper_Iteration}.  If the
iterator-expression is a tuple delimited by brackets, the components
of the tuple must support iteration and these components are iterated
over using a tensor product iteration defined
in~\rsec{Tensor_Product_Iteration}.

\subsubsection{Zipper Iteration}
\label{Zipper_Iteration}
\index{zipper iteration}

When multiple iterators are iterated over in a zipper context, on each
iteration, each expression is iterated over, the values are returned
by the iterators in a tuple and assigned to the index, and the
statement is executed.

The shape of each iterator, the rank and the extents in each
dimension, must be identical.

\begin{example}
The output of
\begin{chapel}
for (i, j) in (1..3, 4..6) do
  write(i, " ", j, " ");
\end{chapel}
is ``1 4 2 5 3 6 ''.
\end{example}

\subsubsection{Tensor Product Iteration}
\label{Tensor_Product_Iteration}
\index{tensor product iterator}
When multiple iterators are iterated over in a tensor product context,
they are iterated over as if they were nested in distinct for loops.
There is no constraint on the iterators as there is in the zipper
context.

\begin{example}
The output of
\begin{chapel}
for (i, j) in [1..3, 4..6] do
  write(i, " ", j, " ");
\end{chapel}
is ``1 4 1 5 1 6 2 4 2 5 2 6 3 4 3 5 3 6 ''. The statement is
equivalent to
\begin{chapel}
for i in 1..3 do
  for j in 4..6 do
    write(i, " ", j, " ");
\end{chapel}
\end{example}

\subsubsection{Parameter For Loops}
\label{Parameter_For_Loops}

\index{for loops!parameters}
\index{for@\chpl{for}}
\index{param@\chpl{param}}

Parameter for loops are unrolled by the compiler so that the index
variable is a parameter rather than a variable.  The syntax for a
parameter for loop statement is given by:
\begin{syntax}
param-iterator-expression:
  range-literal
  range-literal `by' integer-literal

param-for-statement:
  `for' `param' identifier `in' param-iterator-expression `do' statement
  `for' `param' identifier `in' param-iterator-expression block-statement
\end{syntax}
Parameter for loops are restricted to iteration over range literals
with an optional by expression where the bounds and stride must be
parameters.  The loop is then unrolled for each iteration.

\subsection{The Label, Break, and Continue Statements}
\label{Label_Break_Continue}
\index{label@\chpl{label}}
\index{break@\chpl{break}}
\index{continue@\chpl{continue}}

The label-statement is used to name a specific loop which can then
be the target of a break- or continue-statement.  If a break-
or continue-statement has no label, the target is the lexically
inner-most loop. Labels can only be given to for-, while-do- and
do-while-statements.

The syntax for label, break, and continue statements is given by:
\begin{syntax}
label-statement:
  `label' identifier statement

break-statement:
  `break' identifier[OPT] ;

continue-statement:
  `continue' identifier[OPT] ;
\end{syntax}

If a break-statement is encountered, control will be transferred to
after the associated loop.  If a continue-statement is encountered,
control will be transferred to the end of the associated loop, but
still inside the loop.  Break-statements cannot be used to break out of
parallel loops.  Neither break- nor continue-statements can
cross out of cobegin-, coforall-, begin-, or sync-statements.

\begin{example}
In the following code, the index of the first element in each row of
\chpl{A} that is equal to \chpl{findVal} is printed.  Once a match is
found, the continue statement is executed causing the outer loop to
move to the next row.
\begin{chapel}
label outer for i in 1..n {
  for j in 1..n {
    if A[i, j] == findVal {
      writeln("index: ", (i, j), " matches.");
      continue outer;
    }
  }
}
\end{chapel}
\end{example}

\subsection{The Use Statement}
\label{The_Use_Statement}
\index{use@\chpl{use}}
\index{modules!using}

The use statement makes symbols in modules available without accessing
them via the module name.  The syntax of the use statement is given
by:
\begin{syntax}
use-statement:
  `use' module-name-list ;

module-name-list:
  module-name
  module-name , module-name-list

module-name:
  identifier
  module-name . module-name
\end{syntax}
The use statement makes symbols in each listed module's scope available
from the scope where the use statement occurs.

Symbols injected by a use statement are at an outer scope from those
defined directly in the scope where the use statement occurs, but at a
nearer scope than symbols defined in the scope containing the scope where
the use statement occurs.

If used modules themselves use other modules, symbols are scoped according
the depth of use statements followed to find them. It is an error for two
variables, types, or modules to be defined at the same depth.

\begin{openissue}
There is an expectation that this statement will be extended to allow
the programmer to restrict which symbols are 'used' as well as to
rename symbols that are used.
\end{openissue}

\subsection{The Type Select Statement}
\label{The_Type_Select_Statement}

\index{type select statements}

A type select statement has two uses.  It can be used to determine the
type of a union, as discussed
in~\rsec{The_Type_Select_Statement_and_Unions}.  In its more general
form, it can be used to determine the types of one or more values
using the same mechanisms used to disambiguate function definitions.
It syntax is given by:
\begin{syntax}
type-select-statement:
  `type' `select' expression-list { type-when-statements }

type-when-statements:
  type-when-statement
  type-when-statement type-when-statements

type-when-statement:
  `when' type-list `do' statement
  `when' type-list block-statement
  `otherwise' statement

expression-list:
  expression
  expression , expression-list

type-list:
  type-specifier
  type-specifier , type-list
\end{syntax}

Call the expressions following the keyword \chpl{select}, the select
expressions.  The number of select expressions must be equal to the
number of types following each of the \chpl{when} keywords.  Like the
select statement, one of the statements associated with a \chpl{when}
will be executed.  In this case, that statement is chosen by the
function resolution mechanism.  The select expressions are the actual
arguments, the types following the \chpl{when} keywords are the types
of the formal arguments for different anonymous functions.  The
function that would be selected by function resolution determines the
statement that is executed.  If none of the functions are chosen, the
the statement associated with the keyword \chpl{otherwise}, if it
exists, will be selected.

As with function resolution, this can result in an ambiguous
situation.  Unlike with function resolution, in the event of an
ambiguity, the first statement in the list of when statements is
chosen.

\subsection{The Empty Statement}
\label{The_Empty_Statement}

An empty statement has no effect.  The syntax of an empty statement is
given by
\begin{syntax}
empty-statement:
  ;
\end{syntax}

\cleardoublepage
\sekshun{Modules}
\label{Modules}
\index{modules}

Chapel supports modules to manage name spaces.  A program consists of
one or more modules.  Every symbol, including variables, functions,
and types, is associated with some module.

Module definitions are described in~\rsec{Module_Definitions}.  The
relation between files and modules is described
in~\rsec{Implicit_Modules}.  Nested modules are described
in~\rsec{Nested_Modules}.  The visibility of a module's symbols by
users of the module is described in~\rsec{Visibility_Of_Symbols}.  The execution
of a program and module initialization is described
in~\rsec{Program_Execution}.

\section{Module Definitions}
\label{Module_Definitions}
\index{module@\chpl{module}}
\index{modules!definitions}

A module is declared with the following syntax:
\begin{syntax}
module-declaration-statement:
  privacy-specifier[OPT] `module' module-identifier block-statement

privacy-specifier:
  `private'
  `public'

module-identifier:
  identifier
\end{syntax}

A module's name is specified after the \chpl{module} keyword.
The \sntx{block-statement} opens the module's scope.  Symbols defined
in this block statement are defined in the module's scope and are
called \emph{top-level module symbols}.  The visibility of a module is
defined by its \sntx{privacy-specifier}~(\rsec{Visibility_Of_A_Module}).

Module declaration statements must be top-level statements within a
module.  A module that is declared within another module is called a
nested module~(\rsec{Nested_Modules}).

\section{Files and Implicit Modules}
\label{Implicit_Modules}
\index{modules!and files}

Multiple modules can be defined in the same file and need not bear any
relation to the file in terms of their names.

\begin{chapelexample}{two-modules.chpl}
The following file contains two explicitly named modules
(\rsec{Explicit_Naming}), MX and MY.
\begin{chapel}
module MX {
  var x: string = "Module MX";
  proc printX() {
    writeln(x);
  }
}

module MY {
  var y: string = "Module MY";
  proc printY() {
    writeln(y);
  }
}
\end{chapel}
\begin{chapelpost}
MX.printX();
MY.printY();
\end{chapelpost}
\begin{chapeloutput}
Module MX
Module MY
\end{chapeloutput}
Module MX defines top-level module symbols x and printX, while MY
defines top-level module symbols y and printY.
\end{chapelexample}

For any file that contains top-level statements other than module
declarations, the file itself is treated as the module declaration.
In this case,
\index{implicit modules}
\index{modules!implicit}
the module is implicit and takes its name from the base filename.  In
particular, the module name is defined as the remaining string after
removing the \chpl{.chpl} suffix and any path specification from the
specified filename.  If the resulting name is not a legal Chapel
identifier, it cannot be referenced in a use statement.

\begin{chapelexample}{implicit.chpl}
The following file, named implicit.chpl, defines an implicitly named
module called implicit.
\begin{chapel}
var x: int = 0;
var y: int = 1;

proc printX() {
  writeln(x);
}
proc printY() {
  writeln(y);
}
\end{chapel}
\begin{chapelpost}
printX();
printY();
\end{chapelpost}
\begin{chapeloutput}
0
1
\end{chapeloutput}
Module implicit defines the top-level module symbols x, y, printX, and
printY.
\end{chapelexample}


\section{Nested Modules}
\label{Nested_Modules}
\index{modules!nested}

A nested module is a module that is defined within another module, the
outer module.  Nested modules automatically have access to all of the
symbols in the outer module.  However, the outer module needs to
explicitly use a nested module to have access to its symbols.

A nested module can be used without using the outer module by
explicitly naming the outer module in the use statement.
\begin{chapelexample}{nested-use.chpl}
The code
\begin{chapelpre}
module libsci {
  writeln("Initializing libsci");
  module blas {
    writeln("\\tInitializing blas");
  }
}
module testmain { // used to avoid warnings
}
\end{chapelpre}
\begin{chapel}
use libsci.blas;
\end{chapel}
\begin{chapeloutput}
Initializing libsci
	Initializing blas
\end{chapeloutput}
uses a module named \chpl{blas} that is nested inside a module
named \chpl{libsci}.
\end{chapelexample}

Files with both module declarations and top-level statements result in
nested modules.

\begin{chapelexample}{nested.chpl}
The following file, named nested.chpl, defines an
implicitly named module called nested, with nested modules
MX and MY.
\begin{chapel}
module MX {
  var x: int = 0;
}

module MY {
  var y: int = 0;
}

use MX, MY;

proc printX() {
  writeln(x);
}

proc printY() {
  writeln(y);
}
\end{chapel}
\begin{chapelpost}
printX();
printY();
\end{chapelpost}
\begin{chapeloutput}
0
0
\end{chapeloutput}
\end{chapelexample}


\section{Access of Module Contents}
\label{Access_Of_Module_Contents}
\index{modules!access}

A module's contents can be accessed by code outside of that module
depending on the visibility of the module
itself~(\rsec{Visibility_Of_A_Module}) and the visibility of each
individual symbol~(\rsec{Visibility_Of_Symbols}).  This can be done
via explicit naming~(\rsec{Explicit_Naming}) or the use
statement~(\rsec{Using_Modules}).

\subsection{Visibility Of A Module}
\label{Visibility_Of_A_Module}
\index{modules!access}

A module defined at file scope is visible anywhere. The visibility of a nested
module is subject to the rules of~\rsec{Visibility_Of_Symbols}. There,
the nested module is considered a "symbol defined at the top level
scope" of its outer module.

\subsection{Visibility Of A Module's Symbols}
\label{Visibility_Of_Symbols}
\index{modules!access}

A symbol defined at the top level scope of a module is \emph{visible}
from outside the module when the \sntx{privacy-specifier} of its
definition is \chpl{public} or is omitted (i.e. by default). When a
symbol defined at the top level scope of a module is declared
\chpl{private}, it is not visible outside of that module. A
symbol's visibility inside its module is controlled by normal lexical
scoping and is not affected by its \sntx{privacy-specifier}.  A
module's visible symbols are accessible via explicit
naming~(\rsec{Explicit_Naming}) or the use
statement~(\rsec{Using_Modules}) only where the module's symbol is
visible~(\rsec{Visibility_Of_A_Module}).

\subsection{Explicit Naming}
\label{Explicit_Naming}
\index{modules!explicitly named}

All publicly visible top-level module symbols can be named explicitly
with the following syntax:
\begin{syntax}
module-access-expression:
  module-identifier-list . identifier

module-identifier-list:
  module-identifier
  module-identifier . module-identifier-list

\end{syntax}
This allows two variables that have the same name to be distinguished
based on the name of their module.  Using explicit module naming in a
function call restricts the set of candidate functions to those in the
specified module.

If code refers to symbols that are defined by multiple modules, the
compiler will issue an error.  Explicit naming can be used to
disambiguate the symbols in this case.

\begin{openissue}
It is currently unspecified whether the
first-named module is always at the outermost module level scope, or whether a
scope-search mechanism is used starting at the scope containing the
usage.
\end{openissue}

\begin{chapelexample}{ambiguity.chpl}
In the following example,
\begin{chapel}
module M1 {
  var x: int = 1;
  var y: int = -1;
  proc printX() {
    writeln("M1's x is: ", x);
  }
  proc printY() {
    writeln("M1's y is: ", y);
  }
}
 
module M2 {
  use M3;
  use M1;

  var x: int = 2;

  proc printX() {
    writeln("M2's x is: ", x);
  }

  proc main() {
    M1.x = 4;
    M1.printX();
    writeln(x);
    printX(); // This is not ambiguous
    printY(); // ERROR: This is ambiguous
  }
}

module M3 {
  var x: int = 3;
  var y: int = -3;
  proc printY() {
    writeln("M3's y is: ", y);
  }
}
\end{chapel}
\begin{chapeloutput}
ambiguity.chpl:22: In function 'main':
ambiguity.chpl:27: error: ambiguous call 'printY()'
ambiguity.chpl:34: note: candidates are: printY()
ambiguity.chpl:7: note:                 printY()
\end{chapeloutput}
The call to printX() is not ambiguous because M2's definition shadows
that of M1.  On the other hand, the call to printY() is ambiguous
because it is defined in both M1 and M3.  This will result in a
compiler error.
\end{chapelexample}

\subsection{Using Modules}
\label{Using_Modules}
\index{modules!using}

If a module is visible to the scope in which accessing its symbols is desirable,
then a use statement on that module may be employed.  Use statements
make a module's visible symbols available without requiring them to be
prefixed by the module's name.  For information about use statements in general,
see~\rsec{The_Use_Statement}.

If a type is specified in the \sntx{limitation-clause}, then the type's fields
and methods are treated similarly to the type name.  These fields and methods
cannot be specified in a \sntx{limitation-clause} on their own.

% We need to figure out what to do about functions that return types which due
% to the limitation-clause are not visible without prefix.


\subsection{Module Initialization}
\label{Module_Initialization}
\index{modules!initialization}

Module initialization occurs at program start-up.  All top-level
statements in a module other than function and type declarations are
executed during module initialization.

\begin{chapelexample}{init.chpl}
In the code,
\begin{chapelpre}
proc foo() {
    return 1;
}
\end{chapelpre}
\begin{chapel}
var x = foo();       // executed at module initialization
writeln("Hi!");      // executed at module initialization
proc sayGoodbye {
  writeln("Bye!");   // not executed at module initialization
}
\end{chapel}
\begin{chapeloutput}
Hi!
\end{chapeloutput}
The function foo() will be invoked and its result assigned to x.  Then
``Hi!'' will be printed.
\end{chapelexample}

Module initialization order is discussed
in~\rsec{Module_Initialization_Order}.


\section{Program Execution}
\label{Program_Execution}
\index{program execution}
\index{program initialization}

Chapel programs start by initializing all modules and then executing
the main function~(\rsec{The_main_Function}).

\subsection{The {\em main} Function}
\label{The_main_Function}

\index{main@\chpl{main}}
\index{functions!main@\chpl{main}}
The main function must be called \chpl{main} and must have zero
arguments.  It can be specified with or without parentheses.  In any
Chapel program, there is a single main function that defines the
program's entry point.  If a program defines multiple potential entry
points, the implementation may provide a compiler flag that
disambiguates between main functions in multiple modules.

\begin{craychapel}
In the Cray Chapel compiler implementation, the \emph{--
--main-module} flag can be used to specify the module from which the
main function definition will be used.
\end{craychapel}

\begin{chapelexample}{main-module.chpl}
Because it defines two \chpl{main} functions, the following code will yield an
error unless a main module is specified on the command line.
\begin{chapel}
module M1 {
  const x = 1;
  proc main() {
    writeln("M", x, "'s main");
  }
}
 
module M2 {
  use M1;

  const x = 2;
  proc main() {
    M1.main();
    writeln("M", x, "'s main");
  }
}
\end{chapel}
\begin{chapelcompopts}
--main-module M1 \# main\_module.M1.good
--main-module M2 \# main\_module.M2.good
\end{chapelcompopts}
If M1 is specified as the main module, the program will output:
\begin{chapelprintoutput}{main_module.M1.good}
M1's main
\end{chapelprintoutput}
If M2 is specified as the main module the program will output:
\begin{chapelprintoutput}{main_module.M2.good}
M1's main
M2's main
\end{chapelprintoutput}
Notice that main is treated like just another function if it is not in
the main module and can be called as such.
\end{chapelexample}

\index{exploratory programming}

%subsubsection{Programs with a Single Module}
%% \label{Programs_with_a_Single_Module}

To aid in exploratory programming, a default main function is
created if the program does not contain a user-defined main function.  The
default main function is equivalent to
\begin{chapel}
proc main() {}
\end{chapel}

\begin{chapelexample}{no-main.chpl}
The code
\begin{chapel}
writeln("hello, world");
\end{chapel}
\begin{chapeloutput}
hello, world
\end{chapeloutput}
is a legal and complete Chapel program.  The startup code for a Chapel program
first calls the module initialization code for the main module and then
calls \chpl{main()}.  This program's initialization function is the top-level
writeln() statement.  The module declaration is taken to be the entire file,
as described in~\rsec{Implicit_Modules}.
\end{chapelexample}


\subsection{Module Initialization Order}
\label{Module_Initialization_Order}
\index{modules!initialization order}

Module initialization is performed using the following algorithm.

Starting from the module that defines the main function, the modules named in
its use statements are visited depth-first and initialized in post-order.  If a
use statement names a module that has already been visited, it is not visited a
second time.  Thus, infinite recursion is avoided.

Modules used by a given module are visited in the order in which
they appear in the program text.  For nested modules, the
parent module and its uses are initialized before the nested module and its uses.

\begin{chapelexample}{init-order.chpl}
The code
\begin{chapel}
module M1 {
  use M2.M3;
  use M2;
  writeln("In M1's initializer");
  proc main() {
    writeln("In main");
  }
}

module M2 {
  use M4;
  writeln("In M2's initializer");
  module M3 {
    writeln("In M3's initializer");
  }
}

module M4 {
  writeln("In M4's initializer");
}
\end{chapel}
prints the following
\begin{chapelprintoutput}{}
In M4's initializer
In M2's initializer
In M3's initializer
In M1's initializer
In main
\end{chapelprintoutput}
M1, the main module, uses M2.M3 and then M2, thus M2.M3 must be
initialized.  Because M2.M3 is a nested module, M4 (which is used by
M2) must be initialized first.  M2 itself is initialized, followed by
M2.M3.  Finally M1 is initialized, and the main function is run.
\end{chapelexample}

\cleardoublepage
\sekshun{Functions}
\label{Functions}
\index{functions}

This section defines functions.  Methods and iterators are functions
and most of this section applies to them as well.  They are defined
separately in~\rsec{Iterators} and~\rsec{Class_Methods}.

\subsection{Function Definitions}
\label{Function_Definitions}
\index{functions!syntax}

\index{def@\chpl{def}}
Functions are declared with the following syntax:
\begin{syntax}
function-declaration-statement:
  `def' function-name argument-list[OPT] var-param-clause[OPT]
    return-type[OPT] where-clause[OPT] block-level-statement

function-name:
  identifier
  operator-name

operator-name: one of
  + - * / % ** ! == <= >= < > << >> & | ^ ~

argument-list:
  ( formals[OPT] )

formals:
  formal
  formal , formals

formal:
  formal-tag identifier formal-type[OPT] default-expression[OPT]
  formal-tag identifier formal-type[OPT] variable-argument-expression

formal-type:
  : type
  : TQUESTION identifier

default-expression:
  = expression

variable-argument-expression:
  ... expression
  ... TQUESTION identifier

formal-tag: one of
  in out inout param type

var-param-clause:
  `var'
  `const'
  `param'

where-clause:
  `where' expression
\end{syntax}

Operator overloading is supported in Chapel on the operators listed
above under operator name.  Operator and function overloading is
discussed in~\rsec{Function_Overloading}.

The intents \chpl{in}, \chpl{out}, and \chpl{inout} are discussed
in~\rsec{Intents}.  The formal tags \chpl{param} and \chpl{type} make
a function generic and are discussed in~\rsec{Generics}.  If the
formal argument's type is elided, generic, or prefixed with a question
mark, the function is also generic and is discussed
in~\rsec{Generics}.

Default expressions allow for the omission of actual arguments at the
call site, resulting in the implicit passing of a default value.
Default values are discussed in~\rsec{Default_Values}.

Functions do not require parentheses if they have no arguments.  Such
functions are described in~\rsec{Functions_without_Parentheses}.

Return types are optional and are discussed in~\rsec{Return_Types}.

Functions can take a variable number of arguments.  Such functions are
discussed in~\rsec{Variable_Length_Argument_Lists}.

The optional \sntx{var-param-clause} defines a variable function,
discussed in~\rsec{Variable_Functions}, or a parameter function,
discussed in~\rsec{Parameter_Functions}.  By default, a function call
cannot be treated as an lvalue and is constant.  This may be
explicitly specified via the keyword~\chpl{const}.

The optional where clause is only applicable if the function is
generic.  It is discussed in~\rsec{Where_Expressions}.

\subsection{The Return Statement}
\label{The_Return_Statement}
\index{return@\chpl{return}}

The return statement can only appear in a function.  It exits that
function, returning control to the point at which that function was
called.  It can optionally return a value.  The syntax of the return
statement is given by
\begin{syntax}
return-statement:
  `return' expression[OPT] ;
\end{syntax}

\begin{example}
The following code defines a function that returns the sum of three
integers:
\begin{chapel}
def sum(i1: int, i2: int, i3: int)
  return i1 + i2 + i3;
\end{chapel}
\end{example}

\subsection{Function Calls}
\label{Function_Calls}
\index{function calls}

Functions are called in call expressions described
in~\rsec{Call_Expressions}.  The function that is called is resolved
according to the algorithm described in~\rsec{Function_Resolution}.

\subsection{Formal Arguments}
\label{Formal_Arguments}
\index{formal arguments}

Chapel supports an intuitive formal argument passing mechanism.  An
argument is passed by value unless it is a class, array, or domain in
which case it is passed by reference.

Intents~(\rsec{Intents}) result in potential assignments to temporary
variables during a function call.  For example, passing an array by
intent \chpl{in}, a temporary array will be created.

\subsubsection{Named Arguments}
\label{Named_Arguments}
\index{named arguments}
\index{formal arguments!naming}

A formal argument can be named at the call site to explicitly map an
actual argument to a formal argument.

\begin{example}
In the code
\begin{chapel}
def foo(x: int, y: int) { ... }

foo(x=2, y=3);
foo(y=3, x=2);
\end{chapel}
named argument passing is used to map the actual arguments to the
formal arguments.  The two function calls are equivalent.
\end{example}

Named arguments are sometimes necessary to disambiguate calls or
ignore arguments with default values.  For a function that has many
arguments, it is sometimes good practice to name the arguments at the
call-site for compiler-checked documentation.

\subsubsection{Default Values}
\label{Default_Values}
\index{default values}
\index{formal arguments!defaults}

Default values can be specified for a formal argument by appending the
assignment operator and a default expression the declaration of the
formal argument.  If the actual argument is omitted from the function
call, the default expression is evaluated when the function call is
made and the evaluated result is passed to the formal argument as if
it were passed from the call site.

\begin{example}
In the code
\begin{chapel}
def foo(x: int = 5, y: int = 7) { ... }

foo();
foo(7);
foo(y=5);
\end{chapel}
default values are specified for the formal arguments \chpl{x}
and \chpl{y}.  The three calls to \chpl{foo} are equivalent to the
following three calls where the actual arguments are
explicit: \chpl{foo(5, 7)}, \chpl{foo(7, 7)}, and \chpl{foo(5, 5)}.
Note that named arguments are necessary to pass actual arguments to
formal arguments but use default values for arguments that are
specified earlier in the formal argument list.
\end{example}

\subsection{Intents}
\label{Intents}
\index{intents}

Intents allow the actual arguments to be copied to a formal argument
and also to be copied back.

\subsubsection{The Blank Intent}
\label{The_Blank_Intent}

If the intent is omitted, it is called a blank intent.  In such a
case, the value is copied in using the assignment operator.  Thus
classes are passed by reference and records are passed by value.
Arrays and domains are an exception because assignment does not apply
from the actual to the formal.  Instead, arrays and domains are passed
by reference.

With the exception of arrays, any argument that has blank intent
cannot be assigned within the function.

\subsubsection{The In Intent}
\label{The_In_Intent}
\index{in@\chpl{in}}
\index{intents!in@\chpl{in}}

If \chpl{in} is specified as the intent, the actual argument is copied
to the formal argument as usual, but it may also be assigned to within
the function.  This assignment is not reflected back at the call site.

If an array is passed to a formal argument that has \chpl{in} intent,
a copy of the array is made via assignment.  Changes to the elements
within the array are thus not reflected back at the call site.
Domains cannot be passed to a function via the \chpl{in} intent.

\subsubsection{The Out Intent}
\label{The_Out_Intent}
\index{out@\chpl{out}}
\index{intents!out@\chpl{out}}

If \chpl{out} is specified as the intent, the actual argument is
ignored when the call is made, but after the call, the formal argument
is assigned to the actual argument at the call site.  The actual
argument must be a valid lvalue.  The formal argument can be assigned
to and read from within the function.

The formal argument cannot not be generic and is treated as a variable
declaration.  Domains cannot be passed to a function via
the \chpl{out} intent.

\subsubsection{The Inout Intent}
\label{The_Inout_Intent}
\index{inout@\chpl{inout}}
\index{intents!inout@\chpl{inout}}

If \chpl{inout} is specified as the intent, the actual argument is
both passed to the formal argument as if the \chpl{in} intent applied
and then copied back as if the \chpl{out} intent applied.  The formal
argument can be generic and takes its type from the actual argument.
Domains cannot be passed to a function via the \chpl{inout} intent.
The formal argument can be assigned to and read from within the
function.

\subsection{Return Types}
\label{Return_Types}
\index{return@\chpl{return}!types}

A function can optionally return a value.  If the function does not
return a value, then no return type can be specified.  If the function
does return a value, the return type is optional.

\subsubsection{Explicit Return Types}
\label{Explicit_Return_Types}

If a return type is specified, the values that the function returns
via return statements must be assignable to a value of the return
type.  For variable functions~(\rsec{Variable_Functions}), the return
type must match the type returned in all of the return statements
exactly.

\subsubsection{Implicit Return Types}
\label{Implicit_Return_Types}
\index{type inference!of return types}

If a return type is not specified, it will be inferred from the return
statements.  Given the types that are returned by the different
statements, if exactly one of those types can be a target, via
implicit conversions, of every other type, then that is the inferred
return type.  Otherwise, it is an error.  For variable
functions~(\rsec{Variable_Functions}), every return statement must
return the same exact type and it becomes the inferred type.

\subsection{Variable Functions}
\label{Variable_Functions}
\index{functions!as lvalues}

A variable function is a function that can be assigned a value.  Note
that a variable function does not return a reference.  That is, the
reference cannot be captured.

A variable function is specified by following the argument list with
the \chpl{var} keyword.  A variable function must return an lvalue.

When a variable function is called on the left-hand side of an
assignment statement or in the context of a call to a formal argument
by out or inout intent, the lvalue that is returned by the function is
assigned a value.

Variable functions support an implicit argument \chpl{setter} of type
bool.  If the variable function is called in a context such that the
returned lvalue is assigned a value, the argument \chpl{setter}
is \chpl{true}; otherwise it is \chpl{false}.  This argument is useful
for controlling different behavior depending on the call site.

\begin{example}
The following code creates a function that can be interpreted as a
simple two-element array where the elements are actually global
variables:
\begin{chapel}
var x, y = 0;

def A(i: int) var {
  if i < 0 || i > 1 then
    halt("array access out of bounds");
  if i == 0 then
    return x;
  else
    return y;
}
\end{chapel}
This function can be assigned to in order to write to the ``elements''
of the array as in
\begin{chapel}
A(0) = 1;
A(1) = 2;
\end{chapel}
It can be called as an expression to access the ``elements'' as in
\begin{chapel}
writeln(A(0) + A(1));
\end{chapel}
This code outputs the number \chpl{3}.

The implicit \chpl{setter} argument can be used to ensure, for
example, that the second element in the pseudo-array is only assigned
a value if the first argument is positive.  To do this, the line
\begin{chapel}
if setter && i == 1 && x <= 0 then
  halt("cannot assign value to A(1) if A(0) <= 0");
\end{chapel}
\end{example}

\subsection{Parameter Functions}
\label{Parameter_Functions}
\index{functions!as parameters}

A parameter function is a function that returns a parameter
expression.  It is specified by following the function's argument list
by the keyword \chpl{param}.  It is often, but not necessarily,
generic.

It is a compile-time error if a parameter function does not return a
parameter expression.  The result of a parameter function is computed
during compilation and the result is inlined at the call site.

\begin{example}
In the code
\begin{chapel}
def sumOfSquares(param a: int, param b: int) param
  return a**2 + b**2;

var x: sumOfSquares(2, 3)*int;
\end{chapel}
the function \chpl{sumOfSquares} is a parameter function that takes
two parameters as arguments.  Calls to this function can be used in
places where a parameter expression is required.  In this example, the
call is used in the declaration of a homogeneous and so is required to
be a parameter.
\end{example}.

\subsection{Function Overloading}
\label{Function_Overloading}
\index{functions!overloading}
\index{operators!overloading}

Functions that have the same name but different argument lists are
called overloaded functions.  Function calls to overloaded functions
are resolved according to the algorithm in~\rsec{Function_Resolution}.

Operator overloading is achieved by defining a function with a name
specified by that operator.  The operators that may be overloaded are
listed in the following table:

\begin{center}
\begin{tabular}{|l|l|}
\hline
{\bf arity} & {\bf operators} \\
\hline
unary & \verb@+ - ! ~@ \\
binary & \verb@+ - * / % ** ! == <= >= < > << >> & | ^ @ \\
\hline
\end{tabular}
\end{center}

The arity and precedence of the operator must be maintained when it is
overloaded.  Operator resolution follows the same algorithm as
function resolution.

\subsection{Function Resolution}
\label{Function_Resolution}

Given a function call, the function that the call resolves to is
determined according to the following algorithm:
\begin{itemize}
\item
Identify the set of visible functions.  A visible function is any
function with the same name that satisfies the criteria
in~\rsec{Identifying_Visible_Functions}.
\item
From the set of visible functions, determine the set of candidate
functions.  A function is a candidate if the function is a valid
application of the function call's actual arguments as determined
in~\rsec{Determining_Candidate_Functions}.  A compiler error occurs if
there are no candidate functions.
\item
From the set of candidate functions, the most specific function is
determined.  The most specific function is a candidate function that
is more specific than every other candidate function.  If there is no
function that is more specific than every other candidate function,
the function call is ambiguous and a compiler error occurs.  The term
{\em more specific function} is defined
in~\rsec{Determining_More_Specific_Functions}.
\end{itemize}.

\subsubsection{Identifying Visible Functions}
\label{Identifying_Visible_Functions}
\index{functions!visible}

A function is a visible function to a function call if the name of the
function is the same as the name of the function call and the function
is defined or used in a lexical outer scope.

\index{functions!with class arguments}
Additionally, functions that have arguments of class type are
considered globally visible and so are always visible regardless of
the location of their definition.

\subsubsection{Determining Candidate Functions}
\label{Determining_Candidate_Functions}
\index{functions!candidates}

A function is a candidate function if there is a {\em valid mapping}
from the function call to the function and each actual argument is
mapped to a formal argument that is a {\em legal argument mapping}.

\paragraph{Valid Mapping}

A function call is mapped to a function according to the following
steps:
\begin{itemize}
\item
Each actual argument that is passed by name is matched to the formal
argument with that name.  If there is no formal argument with that
name, there is no valid mapping.
\item
The remaining actual arguments are mapped in order to the remaining
formal arguments in order.  If there are more actual arguments then
formal arguments, there is no valid mapping.  If any formal argument
that is not mapped to by an actual argument does not have a default
value, there is no valid mapping.
\item
The valid mapping is the mapping of actual arguments to formal
arguments plus default values to formal arguments that are not mapped
to by actual arguments.
\end{itemize}

\paragraph{Legal Argument Mapping}

An actual argument of type $T_A$ can be mapped to a formal argument of
type $T_F$ if any of the following conditions hold:
\begin{itemize}
\item $T_A$ and $T_F$ are the same type.
\item There is an implicit coercion from $T_A$ to $T_F$.
\item $T_A$ is derived from $T_F$.
\item $T_A$ is scalar promotable to $T_F$.
\end{itemize}

\subsubsection{Determining More Specific Functions}
\label{Determining_More_Specific_Functions}
\index{functions!most specific}

Given two functions $F_1$ and $F_2$, $F_1$ is determined to be more
specific than $F_2$ by the following steps:
\begin{itemize}
\item
If at least one of the legal argument mappings to $F_1$ is a {\em more
specific argument mapping} than the corresponding legal argument
mapping to $F_2$ and none of the legal argument mappings to $F_2$ is a
more specific argument mapping than the corresponding legal argument
mapping to $F_1$, then $F_1$ is more specific.
\item If $F_1$ does not require promotion and $F_2$ does require promotion, then $F_1$ is more specific.
\item If $F_1$ shadows $F_2$, then $F_1$ is more specific.
\item Otherwise, $F_1$ is not more specific than $F_2$.
\end{itemize}

Given an argument mapping, $M_1$, from an actual argument, $A$, of
type $T_A$ to a formal argument, $F1$, of type $T_{F1}$ and an
argument mapping, $M_2$, from the same actual argument to a formal
argument, $F2$, of type $T_{F2}$, the more specific argument mapping
is determined by the following steps:
\begin{itemize}
\item
 If $T_{F1}$ and $T_{F2}$ are the same type and $F1$ is an
 instantiated parameter, $M_1$ is more specific.
\item
 If $T_{F1}$ and $T_{F2}$ are the same type and $F2$ is an
 instantiated parameter, $M_2$ is more specific.
\item
 If $M_1$ requires scalar promotion and $M_2$ does not require scalar
 promotion, $M_2$ is more specific.
\item
 If $M_2$ requires scalar promotion and $M_1$ does not require scalar
 promotion, $M_1$ is more specific.
\item
 If $F1$ is generic over all types and $F2$ is not generic over all
 types, $M_2$ is more specific.
\item
 If $F2$ is generic over all types and $F1$ is not generic over all
 types, $M_1$ is more specific.
\item
 If $T_{F1}$ and $T_{F2}$ are the same type, neither mapping is more
 specific.
\item
 If $T_A$ and $T_{F1}$ are the same type, $M_1$ is more specific.
\item
 If $T_A$ and $T_{F2}$ are the same type, $M_2$ is more specific.
\item
 If $T_{F1}$ is derived from $T_{F2}$, then $M_1$ is more specific.
\item
 If $T_{F2}$ is derived from $T_{F1}$, then $M_2$ is more specific.
\item
 If there is an implicit coercion from $T_{F1}$ to $T_{F2}$, then
 $M_1$ is more specific.
\item
 If there is an implicit coercion from $T_{F2}$ to $T_{F1}$, then
 $M_2$ is more specific.
\item
 If $T_{F1}$ is any \chpl{int} type and $T_{F2}$ is any \chpl{uint}
 type, $M_1$ is more specific.
\item
 If $T_{F2}$ is any \chpl{int} type and $T_{F1}$ is any \chpl{uint}
 type, $M_2$ is more specific.
\item
 Otherwise neither mapping is more specific.
\end{itemize}

\subsection{Functions without Parentheses}
\label{Functions_without_Parentheses}
\index{functions!functions without parentheses}

Functions do not require parentheses if they have empty argument
lists.  Functions declared without parentheses around empty argument
lists must be called without parentheses.

\begin{example}
Given the definitions
\begin{chapel}
def foo { }
def bar() { }
\end{chapel}
the function \chpl{foo} can be called by writing \chpl{foo} and the
function \chpl{bar} can be called by writing \chpl{bar()}.  It is an
error to apply parentheses to \chpl{foo} or omit them from \chpl{bar}.
\end{example}

\subsection{Nested Functions}
\label{Nested_Functions}
\index{functions!nested}

A function defined in another function is called a nested function.
Nesting of functions may be done to arbitrary degrees, i.e., a
function can be nested in a nested function.

Nested functions are only visible to function calls within the scope
in which they are defined.  An exception is to a function that has an
argument that is a class type.  Such functions are globally visible.

\subsubsection{Accessing Outer Variables}
\label{Accessing_Outer_Variables}

Nested functions may refer to variables defined in the function in
which they are nested.  If the function has class arguments, and is
thus globally visible, it is a compiler error to refer to a variable
in the outer function.

\begin{rationale}
It may be too strict to make this a compiler error.  Are there
advantages to making this a runtime error?
\end{rationale}

\subsection{Variable Length Argument Lists}
\label{Variable_Length_Argument_Lists}
\index{functions!variable number of arguments}

Functions can be defined to take a variable number of arguments.  This
allows the call site to pass a different number of actual arguments.

If the variable argument expression is an identifier prepended by a
question mark, the number of arguments is variable.  Alternatively,
the variable expression can evaluate to an integer parameter value
requiring the call site to pass that number of arguments to the
function.

In the function, the formal argument is a tuple of the actual
arguments.

\begin{example}
The code
\begin{chapel}
def mywriteln(x: int ...?k) {
  for param i in 1..k do
    writeln(x(i));
}
\end{chapel}
defines a function called \chpl{mywriteln} that takes a variable
number of arguments and then writes them out on separate lines.  The
parameter for-loop~(\rsec{Parameter_For_Loops}) is unrolled by the
compiler so that \chpl{i} is a parameter, rather than a variable.
This function can be made generic~(\rsec{Generics}) to take arguments
of different types by eliding the type.
\end{example}

A tuple of variables arguments can be passed to a function that takes
variable arguments by destructuring the tuple in a tuple destructuring
expression.  The syntax of this expression is given by
\begin{syntax}
tuple-destructuring-expression:
  ( ... expression )
\end{syntax}
In this expression, the tuple defined by \sntx{expression} is expanded
in place to represent its components.  This allows for the forwarding
of variable arguments as variable arguments.

\cleardoublepage
\sekshun{Tuples}
\label{Tuples}
\index{tuples}

A tuple is an ordered set of components that allows for the
specification of a light-weight record with anonymous fields.

\subsection{Tuple Expressions}
\label{Tuple_Expressions}

A tuple expression is a comma-separated list of expressions that is
enclosed in parentheses.  The number of expressions is the size of the
tuple and the types of the expressions determine the component types
of the tuple.

The syntax of a tuple expression is given by:
\begin{syntax}
tuple-expression:
  ( expression-list )

expression-list:
  expression
  expression , expression-list
\end{syntax}

\begin{example}
The statement
\begin{chapel}
var x = (1, 2);
\end{chapel}
defines a variable \chpl{x} that is a 2-tuple containing the values
\chpl{1} and \chpl{2}.
\end{example}

\subsection{Tuple Type Definitions}
\label{Tuple_Type_Definitions}
\index{tuples!types}

A tuple type is a comma-separated list of types.  The number of types
in the list defines the size of the tuple, which is part of the
tuple's type.  The syntax of a tuple type is given by:
\begin{syntax}
tuple-type:
  ( type-list )

type-list:
  type
  type , type-list
\end{syntax}

\begin{example}
Given a tuple expression \chpl{(1, 2)}, the type of the tuple value is
\chpl{(int, int)}, referred to as a 2-tuple of integers.
\end{example}

\subsection{Tuple Assignment}
\label{Tuple_Assignment}
\index{assignment!tuples}
\index{tuples!assignment}

In tuple assignment, the components of the tuple on the left-hand
side of the assignment operator are each assigned the components of
the tuple on the right-hand side of the assignment.  The assignments
are simultaneous so that each component expression on the right-hand
side is fully evaluated before being assigned to the left-hand side.

\subsection{Tuple Operators}
\label{Tuple_Operators}
\index{tuples!operators}

The arithmetic ~(\rsec{Arithmetic_Operators}), bitwise
~(\rsec{Bitwise_Operators}), shift ~(\rsec{Shift_Operators}), and
relational ~(\rsec{Relational_Operators}) operators are also defined
over tuples.

With the exception of relational operators, operations applied to two
tuples result in element-by-element application of the operation.

Relational operators over tuples apply in an "alphabetical" manner.
Each component is compared to the corresponding component or to the
scalar value until the relation is found to be true or false.

\begin{example}
In the code:
\begin{chapel}
var x = ("c", "h", "p", "l") > ("c", "h", "a", "t"); 
\end{chapel}
The value of \chpl{x} is \chpl{true}. After comparing \chpl{"c"} to
\chpl{"c"}, and \chpl{"h"} to \chpl{"h"}, \chpl{"p"} is found to be
greater than \chpl{"a"}, so the expression is \chpl{true}. 
\end{example}

\subsection{Tuple Destructuring}
\label{Tuple_Destructuring}
\index{tuples!destructuring}

When a tuple expression appears on the left-hand side of an assignment
statement, the expression on the right-hand side is said to be {\em
destructured}.  The components of the tuple on the right-hand side are
assigned to each of the component expressions on the left-hand side.
This assignment is simultaneous in that the right-hand side is
evaluated before the assignments are made.
\begin{example}
Given two variables of the same type, x and y, they can be swapped by
the following single assignment statement:
\begin{chapel}
(x, y) = (y, x);
\end{chapel}
\end{example}

\subsubsection{Variable Declarations in a Tuple}
\label{Variable_Declarations_in_a_Tuple}
\index{tuples!variable declarations}

Variables can be defined in a tuple to facilitate capturing the values
from a function that returns a tuple.  The extension to the syntax of
variable declarations is as follows:
\begin{syntax}
tuple-variable-declaration-statement:
  `config'[OPT] variable-kind tuple-variable-declaration ;

tuple-variable-declaration:
  ( tuple-identifier-list ) type-part[OPT] initialization-part
  ( tuple-identifier-list ) type-part

tuple-identifier-list:
  tuple-identifier
  tuple-identifier , tuple-identifier-list

tuple-identifier:
  identifier
  ( tuple-identifier-list )
\end{syntax}
The identifiers defined within the \sntx{tuple-identifier-list} are declared
to be new variables in the scope of the statement.  The
\sntx{type-part} and/or \sntx{initialization-part} defines a tuple
that is destructured when assigned to the new variables. The shape of the
\sntx{tuple-identifier-list} must match the shape of any specified
\sntx{type-part} or \sntx{initialization-part}.

\subsubsection{Ignoring Values with Underscore}
\label{Ignoring_Values_with_Underscore}
\index{_@\chpl{_}}

If an underscore appears as a component in a tuple expression in a
destructuring context, the expression on the right-hand side is
ignored, though it is still evaluated.

\subsection{Homogeneous Tuples}
\label{Homogeneous_Tuples}
\index{tuples!homogeneous}

A homogeneous tuple is a special-case of a general tuple where the
types of the components are identical.  Homogeneous tuples have fewer
restrictions for how they can be indexed~(\rsec{Tuple_Indexing}).

\subsubsection{Declaring Homogeneous Tuples}
\label{Declaring_Homogeneous_Tuples}

\index{* (tuples)@\chpl{*} tuples}

A homogeneous tuple type may be specified with the following syntax if
it appears as a top-level type in a variable declaration, formal
argument declaration, return type specification, or type alias
declaration:
\begin{syntax}
homogeneous-tuple-type:
  integer-parameter-expression * type

integer-parameter-expression:
  expression
\end{syntax}
The homogeneous tuple type specification is syntactic sugar for the
type explicitly replicated a number of times equal to the
\sntx{integer-parameter-expression}.
\begin{example}
The following types are equivalent:
\begin{center}
\chpl{3*int} \hspace{2pc} \chpl{(int, int, int)}
\end{center}
\end{example}

\subsection{Tuple Indexing}
\label{Tuple_Indexing}
\index{tuples!indexing}

A tuple may be indexed into by an integer.  Indexing a tuple is given
by the following syntax:
\begin{syntax}
tuple-indexing-expression:
  expression ( integer-expression )
\end{syntax}

The result of indexing a tuple by integer $k$ is the value of the
$k$th component.  If the tuple is not homogeneous, the tuple can only
be indexed by an integer parameter.  This ensures that the type of the
indexing expression is known at compile-time.

\subsection{Formal Arguments of Tuple Type}
\label{Formal_Arguments_of_Tuple_Type}

\index{formal arguments!tuples}

\begin{status}
Formal arguments of tuple type are treated as if they were records.
Conversions are not applied to the components.
\end{status}

\subsubsection{Formal Argument Declarations in a Tuple}
\label{Formal_Argument_Declarations_in_a_Tuple}
\index{formal arguments!tuples}

Formal argument declarations can be grouped into a tuple similarly to
variable declarations to facilitate passing the result of a function that
returns a tuple directly to another function.

\begin{status}Formal arguments grouped in a tuple cannot be explicitly
typed. A function with formal arguments grouped in a tuple is
therefore generic.
\end{status}

\cleardoublepage
\sekshun{Classes}
\label{Classes}

Classes are an abstraction of a data structure where the storage
location is allocated independent of the scope of the variable of
class type.  Each call to the constructor creates a new data object
and returns a reference to the object.  Storage is reclaimed
automatically as described in~\rsec{Automatic_Memory_Management}.

\subsection{Class Declarations}
\label{Class_Declarations}

A class is defined with the following syntax:
\begin{syntax}
class-declaration-statement:
  `class' identifier class-inherit-type-list[OPT] {
    class-statement-list }

class-inherit-expression-list:
  class-type
  class-type , inherit-expression-list

class-statement-list:
  class-statement
  class-statement class-statement-list

class-statement:
  type-declaration-statement
  function-declaration-statement
  variable-declaration-statement
\end{syntax}
A \sntx{class-declaration-statement} defines a new type symbol
specified by the identifier.  Classes inherit data and functionality
from other classes if the \sntx{inherit-type-list} is specified.
Inheritance is described in~\rsec{Inheritance}.

The body of a class declaration consists of a sequence of statements
where each of the statements either defines a variable, called a
field, a function, called a method, or a type.

If a class contains a type alias or a parameter, the class is generic.
Generic classes are described in~\rsec{Generics}.

\subsection{Class Assignment}
\label{Class_Assignment}

Classes are assigned by reference.  After an assignment from one
variable of class type to another, the variables reference the same
storage location.

\subsection{Class Fields}
\label{Class_Fields}

Variables and constants declared within class declarations define
fields within that class.  (Parameters make a class generic.)  Fields
define the storage associated with a class.

\begin{example}
The code
\begin{chapel}
class Actor {
  var name: string;
  var age: uint;
}
\end{chapel}
defines a new class type called \chpl{Actor} that has two fields: the
string field \chpl{name} and the unsigned integer field \chpl{age}.
\end{example}

\subsubsection{Class Field Accesses}
\label{Class_Field_Accesses}

The field in a class is accessed via a member access expression as
described in~\rsec{Member_Access_Expressions}.  Fields in a class can
be modified via an assignment statement where the left-hand side of
the assignment is a member access.
\begin{example}
Given a variable \chpl{anActor} of type \chpl{Actor}, defined above,
the code
\begin{chapel}
var s: string = anActor.name;
anActor.age = 27;
\end{chapel}
reads the field \chpl{name} and assigns the value to the variable
\chpl{s}, and assigns the storage location in the object
\chpl{anActor} associated with the field \chpl{age} the value
\chpl{27}.
\end{example}

\subsection{Class Methods}
\label{Class_Methods}

A method is a function that is bound to a class.  A method is called
by passing an instance of the class to the method via a special
syntax that is similar to a field access.

\subsubsection{Class Method Declarations}
\label{Class_Method_Declarations}

Methods are declared with the following syntax:
\begin{syntax}
method-declaration-statement:
  `def' type-binding function-name argument-list[OPT] var-clause[OPT]
    return-type[OPT] where-clause[OPT] block-level-statement

type-binding:
  identifier .
\end{syntax}
If a method is declared within the lexical scope of a class, record,
or union, the type binding can be omitted and is taken to be the
innermost class, record, or union that the method is defined in.

\subsubsection{Class Method Calls}
\label{Class_Method_Calls}

A method is called by using the member access syntax as described
in~\rsec{Member_Access_Expressions} where the accessed expression is
the name of the method.

\begin{example}
A method to output information about an instance of the \chpl{Actor}
class can be defined as follows:
\begin{chapel}
def Actor.print() {
  writeln("Actor ", name, " is ", age, " years old");
}
\end{chapel}
This method can be called on an instance of the \chpl{Actor}
class, \chpl{anActor}, by writing \chpl{anActor.print()}.
\end{example}

\subsubsection{The {\em this} Reference}
\label{The_em_this_Reference}

The instance of a class is passed to a method using special syntax.
It does not appear in the argument list to the method.  The
reference \chpl{this} is an alias to the instance of the class on
which the method is called.

\begin{example}
Let class \chpl{C}, method \chpl{foo}, and function \chpl{bar} be
defined as
\begin{chapel}
class C {
  def foo() {
    bar(this);
  }
}
def bar(c: C) { }
\end{chapel}
Then given an instance of \chpl{C} called \chpl{c}, the method
call \chpl{c.foo()} results in a call to \chpl{bar} where the argument
is \chpl{c}.
\end{example}

\subsubsection{Class Methods without Parentheses}
\label{Class_Methods_without_Parentheses}

Methods do not require parentheses if they have empty argument lists.
Methods declared without parentheses around empty argument lists must
be called without parentheses.

\begin{example}
Given the definitions
\begin{chapel}
class C {
  def foo { }
  def bar() { }
}
\end{chapel}
and an instance of \chpl{C} called \chpl{c}, then the
method \chpl{foo} can be called by writing \chpl{c.foo} and the
method \chpl{bar} can be called by writing \chpl{c.bar()}.  It is an
error to apply parentheses to \chpl{foo} or omit them from \chpl{bar}.
\end{example}

\subsubsection{The {\em this} Method}
\label{The_em_this_Method}

A method declared with the name \chpl{this} allows a class to be
``indexed'' similarly to how a tuple, sequence, or array is indexed.
Indexing into a class has the semantics of calling a method on the
class named \chpl{this}.  There is no other way to call a method
called \chpl{this}.  The \chpl{this} method must be declared with
parentheses even if the argument list is empty.

\begin{example}
In the following code, the \chpl{this} method is used to create a
class that acts like a simple array that contains three integers
indexed by one, two, and three.
\begin{chapel}
class ThreeArray {
  var x1, x2, x3: int;
  def this(i: int) var {
    select i {
      when 1 do return x1;
      when 2 do return x2;
      when 3 do return x3;
    }
    halt("ThreeArray index out of bounds: ", i);
  }
}
\end{chapel}
\end{example}

\subsection{Class Constructors}
\label{Class_Constructors}

A class constructor is defined by declaring a method with the same
name as the class.  The constructor is used to create instances of the
class.  When the constructor is called, memory is allocated to store a
class instance.

\subsubsection{The Default Constructor}
\label{The_Default_Constructor}

A default constructor is automatically created for every class in the
Chapel program.  This constructor is defined such that it has one
argument for every field in the class.  Each of the arguments has a
default value.

The default constructor is very useful but its generality in terms of
having one argument for each field all of which have default values
makes it slightly difficult for the user to create their own
constructor.  It is expected that in many simple cases, the default
constructor will be all that is necessary.

\begin{example}
Given the class
\begin{chapel}
class C {
  def x: int;
  def y: real = 3.14;
  def z: string = "Hello, World!";
}
\end{chapel}
then instances of the class can be created by calling the default
constructor as follows:
\begin{itemize}
\item The call \chpl{C()} is equivalent to \chpl{C(0,3.14,"Hello, World!")}.
\item The call \chpl{C(2)} is equivalent to \chpl{C(2,3.14,"Hello, World!")}.
\item The call \chpl{C(z="")} is equivalent to \chpl{C(0,3.14,"")}.
\item The call \chpl{C(0,0.0,"")} is equivalent to \chpl{C(0,0.0,"")}.
\end{itemize}
\end{example}

\subsection{Getters and Setters}
\label{Getters_and_Setters}

All field accesses are resolved via getter and setter methods that are
defined in the class with the same name as the field.  A setter is
defined as an explicit setter
function~(\rsec{Explicit_Setter_Functions}).  Default getters and
setters are defined that simply access or set the field if the user
does not define their own.

\begin{example}
In the code
\begin{chapel}
class C {
  var x: int;
  def =x(value: int) {
    if value < 0 then
      halt("x assigned negative value");
    x = value;
  }
}
\end{chapel}
a setter is defined for field \chpl{x} that ensures that \chpl{x} is
never assigned a negative value.
\end{example}

\subsection{Inheritance}
\label{Inheritance}

A ``derived'' class can inherit from one or more other classes by
specifying those classes, the base classes, following the name of the
derived class in the declaration of the derived class.  When
inheriting from multiple base classes, only one of the base classes
may contain fields.  The other classes can only define methods.  Note
that a class can still be derived from a class that contains fields
which is itself derived from a class that contains fields.

\subsubsection{Accessing Base Class Fields}
\label{Accessing_Base_Class_Fields}

A derived class contains data associated with the fields in its base
classes.  The fields can be accessed in the same way that they are
accessed in their base class unless the getter or setter methods is
overridden in the derived class, as discussed
in~\rsec{Overriding_Base_Class_Methods}.

\subsubsection{Derived Class Constructors}
\label{Derived_Class_Constructors}

Derived class constructors automatically call the default constructor
of the base class.  There is an expectation that a more standard way
of chaining constructor calls will be supported.

\subsubsection{Shadowing Base Class Fields}
\label{Shadowing_Base_Class_Fields}

A field in the derived class can be declared with the same name as a
field in the base class.  Such a field shadows the field in the base
class in that it is always referenced when it is accessed in the
context of the derived class.  There is an expectation that there will
be a way to reference the field in the base class but this is not
defined at this time.

\subsubsection{Overriding Base Class Methods}
\label{Overriding_Base_Class_Methods}

If a method in a derived class is declared with the identical
signature as a method in a base class, then it is said to override the
base class's method.  Such a method is a candidate for dynamic
dispatch in the event that a variable that has the base class type
references a variable that has the derived class type.

The identical signature requires that the names, types, and order of
the formal arguments be identical.

\subsubsection{Inheriting from Multiple Classes}
\label{Inheriting_from_Multiple_Classes}

\begin{implementation}
Multiple inheritance is not yet supported.
\end{implementation}

A class can be derived from multiple base classes provided that only
one of the base classes contains fields either directly or from base
classes that it is derived from.  The methods defined by the other
base classes can be overridden.

\subsection{Class Promotion of Scalar Functions}
\label{Scalar Promotion}

A class can be defined to promote scalar functions by defining an
iterator in the class named \chpl{this} and specifying a return type.
The return type indicates the type that the class promotes.  The body
of the \chpl{this} iterator is ignored.  The class must also implement
the iterator interface as described in~\rsec{Iterator_Interface}.

There is an expectation that class promotion will be implemented in a
different way in the future.

\subsection{Nested Classes}
\label{Nested_Classes}

\begin{implementation}
Nested classes are not yet supported.
\end{implementation}

A class defined within another class is a nested class.

\subsection{Automatic Memory Management}
\label{Automatic_Memory_Management}

\begin{implementation}
Memory allocated to store class objects is not yet reclaimed.
\end{implementation}

Memory associated with class instances is reclaimed automatically when
there is no way for the current program to reference this memory.  The
programmer does not need to free the memory associated with class
instances.

\cleardoublepage
\sekshun{Records}
\label{Records}
\index{records}

A record is a data structure that is similar to a class except it has
value semantics, similar to primitive types.  Value semantics mean that
assignment, argument passing and function return values are by default
all done by copying.  Value semantics also imply that a variable of
record type is associated with only one piece of storage and has only one
type throughout its lifetime.  Storage is allocated for a variable of
record type when the variable declaration is executed, and the record
variable is also initialized at that time. When the record variable goes
out of scope, or at the end of the program if it is a global, it is
deinitialized and its storage is deallocated.

A record declaration statement creates a record
type~\rsec{Record_Declarations}.  A variable of record type contains all
and only the fields defined by that type (\rsec{Record_Types}).  Value
semantics imply that the type of a record variable is known at compile
time (i.e. it is statically typed).

A record can be created using the \chpl{new} operator, which allocates
storage, initializes it via a call to a record constructor, and returns
it.  A record is also created upon a variable declaration of a record
type.

A record type is generic if it contains generic fields.  Generic record types
are discussed in detail in~\rsec{Generic_Types}.

\section{Record Declarations}
\label{Record_Declarations}
\index{records!declarations}
\index{declarations!records}
\index{record@\chpl{record}}

A record type is defined with the following syntax:
\begin{syntax}
record-declaration-statement:
  simple-record-declaration-statement
  external-record-declaration-statement

simple-record-declaration-statement:
  `record' identifier { record-statement-list }

record-statement-list:
  record-statement
  record-statement record-statement-list

record-statement:
  variable-declaration-statement
  method-declaration-statement
  type-declaration-statement
  empty-statement
\end{syntax}

A \sntx{record-declaration-statement} defines a new type symbol specified
by the identifier. As in a class declaration, the body of a record declaration
can contain variable, method, and type declarations.

If a record declaration contains a type alias or parameter field, or it
contains a variable or constant field without a specified type and
without an initialization expression, then it declares a generic record
type.  Generic record types are described in~\rsec{Generic_Types}.

If the \chpl{extern} keyword appears before the \chpl{record} keyword, then an
external record type is declared. An external record is used within Chapel
for type and field resolution, but no corresponding backend definition is
generated.  It is presumed that the definition of an external record is supplied
by a library or the execution environment.  See the chapter on interoperability
(\rsec{Interoperability}) for more information on external records.

\begin{future}
Privacy controls for classes and records are currently not specified,
as discussion is needed regarding its impact on inheritance, for
instance.
\end{future}

\subsection{Record Types}
\label{Record_Types}
\index{records!record types}
\index{records!types}
\index{types!records}

A record type specifier simply names a record type, using
the following syntax:
\begin{syntax}
record-type:
  identifier
  identifier ( named-expression-list )
\end{syntax}
A record type specifier may appear anywhere a type specifier is permitted.

For non-generic records, the record name by itself is sufficient to specify the
type.  Generic records must be instantiated to serve as a fully-specified
type, for example to declare a variable.  This is done with
type constructors, which are defined in Section~\ref{Type_Constructors}.

\subsection{Record Fields}
\label{Record_Fields}
\index{records!fields}
\index{fields!records}

Variable declarations within a record type declaration define fields within that
record type.  The presence of at least one parameter field causes the record
type to become generic.  Variable fields define the storage associated with a
record.

\begin{chapelexample}{defineActorRecord.chpl}
The code
\begin{chapel}
record ActorRecord {
  var name: string;
  var age: uint;
}
\end{chapel}
\begin{chapeloutput}
\end{chapeloutput}
defines a new record type called \chpl{ActorRecord} that has two fields: the
string field \chpl{name} and the unsigned integer field \chpl{age}.  The data
contained by a record of this type is exactly the same as that contained by
an instance of the \chpl{Actor} class defined in the preceding
chapter~\rsec{Class_Fields}.
\end{chapelexample}

\subsection{Record Methods}
\label{Record_Methods}
\index{records!methods}
\index{methods!records}

A record method is a function or iterator that is bound to a record.
See the methods section~\rsec{Methods} for more information about
methods.

Note that the receiver of a record method is passed by \chpl{ref} or
\chpl{const ref} intent by default, depending on whether or not
\chpl{this} is modified in the body of the method.

\subsection{Nested Record Types}
\label{Nested_Record_Types}
\index{nested records}
\index{records!nested}

Record type declarations may be nested within other class, record and union
declarations.  Methods defined in a nested record type may access fields
declared in the containing aggregate type either implicitly, or explicitly by
means of an \chpl{outer} reference.

\section{Record Variable Declarations}
\label{Record_Variable_Declarations}
\index{records!variable declarations}
\index{variables!records}

A record variable declaration is a variable declaration using a record type.
When a variable of record type is declared, storage is allocated sufficient to
store all of the fields defined in that record type.

In the context of a class or record or union declaration, the fields are
allocated within the object as if they had been declared individually.  In this
sense, records provide a way to group related fields within a containing class
or record type.

In the context of a function body, a record variable declaration
causes storage to be allocated sufficient to store all of the fields in that
record type.  The record variable is initialized through a call to its
default initializer.  The default initializer for a record is defined in the
same way as the default initializer for a class (\rsec{Default_Initialization}).

\subsection{Storage Allocation}
\label{Record_Storage}
\index{records!allocation}

Storage for a record variable directly contains the data associated
with the fields in the record, in the same manner as variables
of primitive types directly contain the primitive values.
Record storage is reclaimed when the record variable goes out of scope.
No additional storage for a record is allocated or reclaimed.
Field data of one variable's record is not shared with data
of another variable's record.

\subsection{Record Initialization}
\label{Record_Initialization}
\index{records!initialization}
\index{initialization!record}

A variable of a record type declared without an initialization expression
is initialized through a call to the record's default initializer,
passing no arguments.  The default initializer for a record is defined in
the same way as the default initializer for a class
(\rsec{Default_Initialization}).

To construct a record as an expression,
i.e. without binding it to a variable, the \chpl{new} operator is
required.  In this case, storage is allocated and reclaimed as for a record
variable declaration (\rsec{Record_Storage}), except that the temporary record
goes out of scope at the end of the enclosing statement.
The constructors for a record are
defined in the same way as those for a class (\rsec{Class_Constructors}).

\begin{rationale}

The \chpl{new} keyword disambiguates types from values. This is needed
because of the syntactic similarity between constructors and type
specifiers for classes and records.

\end{rationale}

\begin{chapelexample}{recordCreation.chpl}
The program
\begin{chapel}
record TimeStamp {
  var time: string = "1/1/1011";
}

var timestampDefault: TimeStamp;                  // use the default for 'time'
var timestampCustom = new TimeStamp("2/2/2022");  // ... or a different one
writeln(timestampDefault);
writeln(timestampCustom);

var idCounter = 0;
record UniqueID {
  var id: int;
  proc UniqueID() { idCounter += 1; id = idCounter; }
}

writeln(new UniqueID());  // create and use a record value without a variable
writeln(new UniqueID());
\end{chapel}
\begin{chapelcompopts}
--no-warn-constructors
\end{chapelcompopts}
produces the output
\begin{chapelprintoutput}{}
(time = 1/1/1011)
(time = 2/2/2022)
(id = 1)
(id = 2)
\end{chapelprintoutput}
The variable \chpl{timestampDefault} is initialized with \chpl{TimeStamp}'s
default initializer. The expression \chpl{new TimeStamp} creates a record that
is assigned to \chpl{timestampCustom}.  It effectively
initializes \chpl{timestampCustom} via a call to the constructor with desired
arguments. The records created with \chpl{new UniqueID()} are discarded after
they are used.
\end{chapelexample}

As with classes, the user can provide his own constructors
(\rsec{User_Defined_Constructors}).  If any user-defined constructors are
supplied, the default initializer cannot be called directly.

\subsection{Record Deinitializer}
\label{Record_Deinitializer}
\index{records!deinitializer}
\index{deinitializer!records}

A record author may specify additional actions to be performed before record storage is
reclaimed by defining a record deinitializer.  A record deinitializer is a method named
\chpl{deinit()}.  A record deinitializer takes no arguments
(aside from the implicit \chpl{this} argument).  If defined, the deinitializer is called
on a record object after it goes out of scope and before its memory is reclaimed.

% TODO: The above ambiguous language is intended to allow optimizations involving extending
% the lifetime of an object.  However, we leave unspecified the means by which a user may
% demand avid running of the deinitializer and reclamation of memory (as in C\#).  We need to
% specify this so the above language can be tightened up for that case.

% For now, the actual lifetime of a record object is under the control of the compiler.  For
% example, as an optimization, ownership of an object may be transferred between variables with
% non-overlapping lifetimes.  When this happens, there will be no observable deinitialization of
% one of those variables.  The compiler may also choose to insert temporary copies e.g. of
% record formals or of a record return value.

% The compiler guarantees that every record constructor call will have exactly one
% corresponding record deinitializer call.  However, the exact number of constructor-deinitializer
% pairs is determined by the compiler, and may also be influenced by various compiler
% options.

\begin{chapelexample}{recordDeinitializer.chpl}
\begin{chapel}
class C { var x: int; } // A class with nonzero size.
// If the class were empty, whether or not its memory was reclaimed
// would not be observable.

// Defines a record implementing simple memory management.
record R {
  var c: unmanaged C;
  proc init() {
    c = new unmanaged C(0);
  }
  proc deinit() {
    delete c; c = nil;
  }
}

proc foo()
{
  var r: R; // Initialized using default constructor.
  writeln(r);
  // r will go out of scope here.
  // Its deinitializer will be called to free the C object it contains.
}

foo();
\end{chapel}
\begin{chapeloutput}
(c = {x = 0})

====================
Leaked Memory Report
==============================================================
Number of leaked allocations
           Total leaked memory (bytes)
                      Description of allocation
==============================================================
==============================================================
\end{chapeloutput}
\begin{chapelexecopts}
--memLeaksByType
\end{chapelexecopts}
\end{chapelexample}


\section{Record Arguments}
\label{Record_Arguments}
\index{records!arguments}
\index{arguments!records}

When records are copied into or out of a function's formal argument,
the copy is performed consistently with the semantics described for
record assignment (\rsec{Record_Assignment}).

\begin{chapelexample}{paramPassing.chpl}
The program
\begin{chapel}
record MyColor {
  var color: int;
}
proc printMyColor(in mc: MyColor) {
  writeln("my color is ", mc.color);
  mc.color = 6;   // does not affect the caller's record
}
var mc1: MyColor;        // 'color' defaults to 0
var mc2: MyColor = mc1;  // mc1's value is copied into mc2
mc1.color = 3;           // mc1's value is modified
printMyColor(mc2);       // mc2 is not affected by assignment to mc1
printMyColor(mc2);       // ... or by assignment in printMyColor()

proc modifyMyColor(inout mc: MyColor, newcolor: int) {
  mc.color = newcolor;
}
modifyMyColor(mc2, 7);   // mc2 is affected because of the 'inout' intent
printMyColor(mc2);
\end{chapel}
produces
\begin{chapelprintoutput}{}
my color is 0
my color is 0
my color is 7
\end{chapelprintoutput}
The assignment to \chpl{mc1.color} affects only the record stored
in \chpl{mc1}. The record in \chpl{mc2} is not affected by
the assignment to \chpl{mc1} or by the assignment in \chpl{printMyColor}.
\chpl{mc2} is affected by the assignment in \chpl{modifyMyColor}
because the intent \chpl{inout} is used.
\end{chapelexample}

\section{Record Field Access}
\label{Record_Field_Access}
\index{records!field access}
\index{field access}

A record field is accessed the same way as a class field
(\rsec{Class_Field_Accesses}).  When a field access is used as an
rvalue, the value of that field is returned.  When it is used as
an lvalue, the value of the record field is updated.

Accessing a parameter or type field returns a parameter or type,
respectively. Also, parameter and type fields can be accessed from
an instantiated record type in addition to from a record value.


\subsection{Field Getter Methods}
\label{Field_Getter_Methods}
\index{records!getters}

As in classes, field accesses are performed via getter methods
(\rsec{Getter_Methods}).  By default, these methods simply return a reference to
the specified field (so they can be written as well as read).  The user may
redefine these as needed.

\section{Record Method Calls}
\label{Record_Method_Access}
\index{records!method calls}
\index{method calls}

Record method calls are written the same way as other method calls
(\rsec{Method_Calls}). Unlike class methods, record methods are
always resolved at compile time.

\section{Common Operations}

\subsection{Record Assignment}
\label{Record_Assignment}
\index{records!assignment}

A variable of record type may be updated by assignment.  The compiler
provides a default assignment operator for each record type \chpl{R}
having the signature:

\begin{chapel}
proc =(ref lhs:R, rhs) : void ;
\end{chapel}

In it, the value of each field of the record on the right-hand side is assigned
to the corresponding field of the record on the left-hand side. It is
a type error if the left-hand side and the right-hand side do not have
the same set of field names. It is also a type error if two fields with
the same name do not have assignable types.

The compiler-provided assignment operator may be overridden as described
in \ref{Assignment_Statements}.

The following example demonstrates record assignment.
\begin{chapelexample}{assignment.chpl}
\begin{chapel}
record R {
  var i: int;
  var x: real;
  proc print() { writeln("i = ", this.i, ", x = ", this.x); }
}
var A: R;
A.i = 3;
A.print();	// "i = 3, x = 0.0"

var C: R;
A = C;
A.print();	// "i = 0, x = 0.0"

C.x = 3.14;
A.print();	// "i = 0, x = 0.0"
\end{chapel}
\begin{chapeloutput}
i = 3, x = 0.0
i = 0, x = 0.0
i = 0, x = 0.0
\end{chapeloutput}
Prior to the first call to \chpl{R.print}, the record \chpl{A} is created and
initialized to all zeroes.  Then, its \chpl{i} field is set to \chpl{3}.
For the second call to \chpl{R.print}, the record \chpl{C} is created assigned
to \chpl{A}.  Since \chpl{C} is default-initialized to all zeroes, those zero
values overwrite both values in \chpl{A}.

The next clause demonstrates that \chpl{A} and \chpl{C} are distinct entities,
rather than two references to the same object.  Assigning \chpl{3.14}
to \chpl{C.x} does not affect the \chpl{x} field in \chpl{A}.
\end{chapelexample}

\subsection{Default Comparison Operators}
\label{Record_Comparison_Operators}
\index{records!equality}
\index{records!inequality}
\index{records!==@\chpl{==}}
\index{records!"!=@\chpl{"\"!=}}
\index{== (record)@\chpl{==} (record)}
\index{"!= (record)@\chpl{"\"!=} (record)}
Default functions to overload \chpl{==} and \chpl{\!=} are defined for
records if none are explicitly defined.
The default implementation of \chpl{==} applies \chpl{==} to each
field of the two argument records and reduces the result with
the \chpl{&&} operator.  The default implementation of \chpl{\!=}
applies \chpl{\!=} to each field of the two argument records and
reduces the result with the \chpl{||} operator.

\section{Differences between Classes and Records}
\label{Class_and_Record_Differences}
\index{records!differences with classes}

The key differences between records and classes are listed below.

\subsection{Declarations}
\label{Declaration_Differences}
\index{records!declarations!differences with classes}

Syntactically, class and record type declarations are identical, except that
they begin with the \chpl{class} and \chpl{record} keywords, respectively.
In contrast to classes, records do not support inheritance.

\subsection{Storage Allocation}
\label{Storage_Allocation_Differences}
\index{classes!allocation}
\index{records!allocation}

For a variable of record type, storage necessary to contain the data fields
has a lifetime equivalent to the scope in which it is declared.  No two record
variables share the same data.  It is not necessary to call \chpl{new} to create
a record.

By contrast, a class variable contains only a reference to a
class instance.  A class instance is created through a call to its \chpl{new}
operator.  Storage for a class instance, including storage for
the data associated with the fields in the class, is allocated and reclaimed
separately from variables referencing that instance.  The same class instance
can be referenced by multiple class variables.

\subsection{Assignment}
\label{Assignment_Differences}
\index{classes!assignment}
\index{records!assignment}

Assignment to a class variable is performed by reference, whereas assignment to
a record is performed by value.  When a variable of class type is assigned to
another variable of class type, they both become names for the same object.  In
contrast, when a record variable is assigned to another record variable, then
contents of the source record are copied into the target record field-by-field.

When a variable of class type is assigned to a record, matching fields (matched
by name) are copied from the class instance into the corresponding record
fields.  Subsequent changes to the fields in the target record have no effect
upon the class instance.

Assignment of a record to a class variable is not permitted.

\subsection{Arguments}
\label{Argument_Differences}
\index{classes!arguments}
\index{records!arguments}

Record arguments use the \chpl{const ref} intent by default - in contrast
to class arguments which pass by \chpl{const in} intent by default.

Similarly, the \chpl{this} receiver argument is passed by \chpl{const in} by
default for class methods. In contrast, it is passed by \chpl{ref} or
\chpl{const ref} by default for record methods.

\subsection{No {\em nil} Value}
\index{nil@\chpl{nil}!not provided for records}

Records do not provide a counterpart of the \chpl{nil} value.  A variable of
record type is associated with storage throughout its lifetime, so \chpl{nil}
has no meaning with respect to records.

\subsection{The {\em delete} operator}
\label{Record_Delete_Illegal}
\index{records!delete illegal}
\index{delete!illegal for records}

Calling \chpl{delete} on a record is illegal.

%REVIEW: we could discuss this:
%An explicit call to \chpl{delete} with a record argument has no effect.  The
%compiler may treat this as a hint that the record should not be accessed later
%within its scope and diagnose that case.

\subsection{Default Comparison Operators}
\label{Comparison_Operator_Differences}
\index{classes!comparison}
\index{records!comparison}

For records, the compiler will supply default comparison operators if
they are not supplied by the user.  In contrast, the user cannot redefine
\chpl{==} and \chpl{!=} for classes.  The default comparison operators
for a record examine the arguments' fields, while the comparison
operators for classes check whether the l.h.s. and r.h.s. refer to the
same class instance or are both \chpl{nil}.

\cleardoublepage
This is a stub.  This portion of the document does not exist.

\cleardoublepage
\sekshun{Ranges}
\label{Ranges}
\index{ranges}

Ranges represent a sequence of integral values.  Ranges are
either \emph{bounded} or \emph{unbounded}.

Bounded ranges are characterized by a low bound~$l$, a high bound~$h$,
and a stride~$s$.  If the stride is positive, the values described by
the range are $l, l+s, l+2s, l+3s, ...$ such that all of the values in
the sequence are less than or equal to $h$.  If the stride is negative,
the values described by the range are $h, h+s, h+2s, h+3s, ...$ such
that all of the values in the sequence are greater than or equal to
$l$.  If $l > h$, the range is considered degenerate and represents an
empty sequence. Ranges support iteration over the values they represent
as described in ~\rsec{The_For_Loop}.

Unbounded ranges are those in which the low and/or high bounds are
omitted.  Unbounded ranges conceptually represent a countably infinite
number of values.

\subsection{Range Types}
\label{Range_Types}
\index{ranges!types}

The type of a range is characterized by three things:
(1)~the type of the values being represented, (2)~the boundedness of
the range, and (3)~whether or not the range is \emph{stridable}.

The type of the range's values is represented using a type parameter
named \emph{idxType}.  This must be one of the \chpl{int} or
\chpl{uint} types.  The default type is \chpl{int}.

\begin{openissue}
It has been hypothesized that ranges of other types, such as floating
point values, might also be of interest to represent a range of legal
tolerances, for example.  If you believe such support would be of
interest to you, please let us know.
\end{openissue}

The boundedness of the range is represented using an enumerated
parameter named \emph{boundedType} of type \chpl{BoundedRangeType}.
Legal values are \chpl{bounded}, \chpl{boundedLow},
\chpl{boundedHigh}, and \chpl{boundedNone}.  The first value specifies
a bounded range while the other three values specify a range in which
the high bound is omitted, the low bound is omitted, or both bounds
are omitted, respectively.  The default value is \chpl{bounded}.

The stridability of a range is represented by a boolean parameter
named \emph{stridable}.  If this parameter is set to true, the range's
stride can take on any signed integer value other than 0 of the same
bit-width as \chpl{idxType}.  If set to false, the range's stride is
fixed to 1.  The default value is \chpl{false}.

\begin{rationale}
The \emph{boundedType} and \emph{stridable} values of a range are used
to optimize the generated code for common cases of ranges, as well as
to optimize the implementation of domains and arrays defined using
ranges.
\end{rationale}

The syntax of a range type is summarized as follows:
\begin{syntax}
range-type:
  `range' ( named-expression-list )
\end{syntax}

\begin{example}
The following declaration declares a variable \chpl{r}
of range type that can represent ranges of 64-bit integers, with both
high and low bounds specified, and the ability to have a stride other
than 1.
\begin{chapelpre}
% test_rangeVariable.chpl
\end{chapelpre}
\begin{chapel}
var r: range(int(64), BoundedRangeType.bounded, stridable=true);
\end{chapel}
\begin{chapelpost}
writeln(r);
var i64: int(64) = 3;
r = i64..13 by 3;
writeln(r);
\end{chapelpost}
\begin{chapeloutput}
1..0
3..12 by 3
\end{chapeloutput}
\end{example}

The default value for a range is \chpl{1..0}.

\subsection{Literal Range Values}
\label{Range_Literals}
\index{ranges!literals} 

Range literals are specified as follows:
\begin{syntax}
range-literal:
  bounded-range-literal
  unbounded-range-literal
\end{syntax}

\subsubsection{Bounded Range Literals}
\label{Bounded_Ranges}
\index{ranges!bounded}

A bounded range is specified by the syntax
\begin{syntax}
bounded-range-literal:
  expression .. expression
\end{syntax}
The first expression is taken to be the lower bound $l$ and the second
expression is taken to be the upper bound $h$.  The stride of the
range is 1 and can be modified with the \chpl{by} operator as described
in~\rsec{By_Operator_For_Ranges}.

\index{ranges!integral element type}
The element type of the range type is determined by the type of the
low and high bound.  It is either \chpl{int}, \chpl{uint},
\chpl{int(64)}, or \chpl{uint(64)}.  The type is determined by
conceptually adding the low and high bounds together.  The boundedness
of such a range is \chpl{BoundedRangeType.bounded}.  The stridability of
the range is \chpl{false}.

\subsubsection{Unbounded Range Literals}
\label{Unbounded_Ranges}
\index{ranges!unbounded}

An unbounded range is specified by the syntax
\begin{syntax}
unbounded-range-literal:
  expression ..
  .. expression
  ..
\end{syntax}

The first form results in a \chpl{BoundedRangeType.boundedLow} range, the
second in a \chpl{BoundedRangeType.boundedHigh} range, and the third in
a \chpl{BoundedRangeType.boundedNone} range.

Unbounded ranges can be iterated over with zipper iteration
(~\rsec{Zipper_Iteration}) and their shape conforms to the shape of the
other iterators they are being iterated over with.
\begin{example}
The code
\begin{chapelpre}
% test_zipWithUnbounded.chpl
\end{chapelpre}
\begin{chapel}
for i in (1..5, 3..) do
  write(i, "; ");
\end{chapel}
\begin{chapelpost}
writeln();
\end{chapelpost}
\begin{chapeloutput}
(1, 3); (2, 4); (3, 5); (4, 6); (5, 7); 
\end{chapeloutput}
produces the output ``(1, 3); (2, 4); (3, 5); (4, 6); (5, 7); ''.
\end{example}

It is an error to iterate over a \chpl{BoundedRangeType.boundedNone} range,
a \chpl{BoundedRangeType.boundedLow} range with negative stride or a
\chpl{BoundedRangeType.boundedHigh} range with positive stride.

Unbounded ranges can also be used to index into ranges, domains,
arrays, and strings.  In these cases, elided bounds are inherited
from the bounds of the expression being indexed.


\subsection{Range Assignment}
\label{Range_Assignment}
\index{ranges!assignment}

Assigning one range to another results in its low, high, and stride
values being copied from the source range to the destination range.

In order for range assignment to be legal, the element type of the
source range must be implicitly coercible to the element type of the
destination range.  The two range types must have the same boundedness
parameter.  It is legal to assign a non-stridable range to a stridable
range, but illegal to assign a stridable range to a non-stridable
range unless the stridable range has a stride value of 1.


\subsection{Range Operators}
\label{Range_Operators}
\index{ranges!operators}

\subsubsection{By Operator}
\label{By_Operator_For_Ranges}
\index{ranges!strided}
\index{ranges!by operator}
\index{by@\chpl{by}}

The \chpl{by} operator can be applied to any range to create a strided
range.

The \chpl{by} operator takes a range and an integer value to yield a
new range that is strided by the integer.  Striding a strided range
results in a stride whose value is the product of the two strides.
The stride argument can either be of type \chpl{idxType} or some other
integer value that can coerce to a signed integer value of the same
bit-width as \chpl{idxType}.

\begin{example}
In the following declarations, range \chpl{r1} represents the odd integers
between 1 and 20. Range \chpl{r2} strides \chpl{r1} by two and represents
every other odd integer between 1 and 20: 1, 5, 9, ...
\begin{chapelpre}
% test_rangeByOperator.chpl
\end{chapelpre}
\begin{chapel}
var r1 = 1..20 by 2;
var r2 = r1 by 2;
\end{chapel}
\begin{chapelpost}
writeln(r1);
writeln(r2);
\end{chapelpost}
\begin{chapeloutput}
1..19 by 2
1..17 by 4
\end{chapeloutput}
\end{example}

\begin{rationale}
{\it Why isn't the high bound specified first if the stride is
negative?}  The reason for this choice is that the \chpl{by} operator
is binary, not ternary.  Given a range \chpl{R} initialized
to \chpl{1..3}, we want \chpl{R by -1} to contain the ordered sequence
$3,2,1$.  But then \chpl{R by -1} would be different than \chpl{3..1
by -1} even though it should be identical by substituting the value in
R into the expression.
\end{rationale}

\subsubsection{Count Operator}
\label{Count_Operator}
\index{ranges!count operator}

The \chpl{#} operator can be applied to a range that has a high bound,
a low bound, or both.

The \chpl{#} operator takes a range and an integral count and creates
a new range with \emph{count} elements. The stride of the resulting range is
the same as that of the initial range. It is an error for the count to
be negative.  The \emph{idxType} of the resulting range is the same
type that would be obtained by adding the integral count value to a value
with the range's \emph{idxType}.

When applied to a \chpl{BoundedRangeType.bounded} range with a positive
stride, \emph{count} elements are taken starting from the low
bound. When the stride is negative, \emph{count} elements are taken
starting from the high bound. It is an error for \emph{count} to be larger
than the length of the range.

When applied to a \chpl{BoundedRangeType.boundedLow} range, the low bound
is fixed and and the high bound is set based on the count and the
absolute value of the stride.

When applied to a \chpl{BoundedRangeType.boundedHigh} range, the high
bound is fixed and the low bound is set based on the count and the
absolute value of the stride.

It is an error to apply the count operator to a
\chpl{BoundedRangeType.boundedNone} range.

\begin{example}
The following declarations result in equivalent ranges.
\begin{chapelpre}
% test_rangeCountOperator.chpl
\end{chapelpre}
\begin{chapel}
var r1 = 2.. by -2 # 3;
var r2 = ..6 by -2 # 3;
var r3 = 0..6 by -2 # 3;
var r4 = 1..#6 by -2;
\end{chapel}
\begin{chapelpost}
writeln(r1 == r2 \&\& r2 == r3 \&\& r3 == r4);
writeln((r1, r2, r3, r4));
\end{chapelpost}
\begin{chapeloutput}
true
(2..6 by -2, 2..6 by -2, 2..6 by -2, 2..6 by -2)
\end{chapeloutput}
Each of these ranges represents the ordered set of three values: 6, 4, 2.
\end{example}

\subsubsection{Arithmetic Operators}
\label{Range_Arithmetic}
\index{ranges!arithmetic operators}

The following arithmetic operators are defined on ranges and integral
types:

\begin{chapel}
def +(r: range, s: integral): range
def +(s: integral, r: range): range
def -(r: range, s: integral): range
\end{chapel}

The \chpl{+} and \chpl{-} operators apply the scalar via the operator
to the range's low and high bounds, producing a shifted version of the
range.  The element type of the resulting range is the type of the value
that would result from an addition between the scalar value and a value
with the range's element type.  The bounded and stridable parameters for
the result range are the same as for the input range.

\begin{example}
The following code creates a bounded, non-stridable range \chpl{r}
which has an element type of \chpl{int} representing the values ${0,
  1, 2, 3}$.  It then uses the \chpl{+} operator to
create a second range \chpl{r2} representing the values ${1, 2, 3,
  4}$.  The \chpl{r2} range is bounded, non-stridable, and represents
values of type \chpl{int}.
\begin{chapelpre}
% test_rangeAdd.chpl
\end{chapelpre}
\begin{chapel}
var r = 0..3;
var r2 = r + 1;
\end{chapel}
\begin{chapelpost}
writeln((r, r2));
\end{chapelpost}
\begin{chapeloutput}
(0..3, 1..4)
\end{chapeloutput}
\end{example}


\subsubsection{Range Slicing}
\label{Range_Slicing}
\index{ranges!slicing}

Ranges can be \emph{sliced} using other ranges to create new
sub-ranges.  The resulting range represents the intersection between
the two ranges.  Range slicing is defined by using the range as a
function in a call expression where the argument is another range.
If the slicing range is unbounded in one or both directions, it
inherits its missing bounds from the range being sliced.

\begin{example}
In the following example, \chpl{r} represents the integers from 1 to
20 inclusive.  Ranges \chpl{r2} and \chpl{r3} are defined using range
slices and represent the indices from 3 to 20 and the odd integers
between 1 and 20 respectively. Range \chpl{r4} represents the odd
integers between 1 and 20 that are also divisible by 3.
\begin{chapelpre}
% test_rangeSlicing.chpl
\end{chapelpre}
\begin{chapel}
var r = 1..20;
var r2 = r[3..];
var r3 = r[1.. by 2];
var r4 = r3[0.. by 3];
\end{chapel}
\begin{chapelpost}
writeln((r, r2, r3, r4));
\end{chapelpost}
\begin{chapeloutput}
(1..20, 3..20, 1..19 by 2, 3..15 by 6)
\end{chapeloutput}
\end{example}

\subsection{Predefined Functions and Methods on Ranges}
\index{ranges!predefined functions}
\begin{protohead}
def $range$.low : idxType
\end{protohead}
\begin{protobody}
Returns the low bound of the range.
\end{protobody}

\begin{protohead}
def $range$.high : idxType
\end{protohead}
\begin{protobody}
Returns the high bound of the range.
\end{protobody}

\begin{protohead}
def $range$.stride : int(numBits(idxType))
\end{protohead}
\begin{protobody}
Returns the stride of the range.
\end{protobody}

\begin{protohead}
def $range$.length : idxType
\end{protohead}
\begin{protobody}
Returns the number of elements in the range.
\end{protobody}

\begin{protohead}
def $range$.member(i: idxType): bool
\end{protohead}
\begin{protobody}
Returns whether or not \chpl{i} is in the range.
\end{protobody}

\begin{protohead}
def $range$.member(other: range): bool
\end{protohead}
\begin{protobody}
Returns whether or not every element in other is also in this.
\end{protobody}

\begin{protohead}
def $range$.order(i: idxType): idxType
\end{protohead}
\begin{protobody}
If \chpl{i} is a member of the range, returns an integer value giving
the ordinal value of \chpl{i} within the range using 0-based indexing.
Otherwise, it returns \chpl{(-1):idxType}.
\end{protobody}

\begin{example}
The following calls show the order of index 4 in each of the given
ranges:
\begin{chapel}
(0..10).order(4) == 4
(1..10).order(4) == 3
(3..5).order(4) == 1
(0..10 by 2).order(4) == 2
(3..5 by 2).order(4) == -1
\end{chapel}
\end{example}

\cleardoublepage
\sekshun{Domains}
\label{Domains}
\index{domains}

A \emph{domain} is a first-class representation of an index set.
Domains are used to specify iteration spaces, to define the size and
shape of arrays (\rsec{Arrays}), and to specify aggregate operations
like slicing.
A domain can specify a single- or multi-dimensional
rectangular iteration space or represent a set of indices of
a given type.  Domains can also represent a subset of another domain's index set,
using either a dense or sparse representation.
A domain's
indices may potentially be distributed across multiple locales as
described in~\rsec{Domain_Maps}, thus supporting global-view data
structures.

In the next subsection, we introduce the key characteristics of
domains.  In~\rsec{Base_Domain_Types_and_Values}, we discuss the types
and values that can be associated with a base domain.
In~\rsec{Simple_Subdomain_Types_and_Values}, we discuss the types and
values of simple subdomains that can be created from those base
domains.  In~\rsec{Sparse_Subdomain_Types_and_Values}, we discuss the
types and values of sparse subdomains.  The remaining sections
describe the important manipulations that can be performed with
domains, as well as the predefined operators and functions defined for
domains.

\section{Domain Overview}
\index{domains!kinds}

There are three \emph{kinds} of domain, distinguished by their subset
dependencies: \emph{base domains}, \emph{subdomains} and \emph{sparse
subdomains}.  A base domain describes an index set spanning one or more
dimensions.  A subdomain creates an index set that is a subset of the indices in
a base domain or another subdomain.  Sparse subdomains are subdomains which can
represent sparse index subsets efficiently.  Simple subdomains are subdomains
that are not sparse.  These relationships can be represented as follows:

\begin{syntax}
\begin{verbatim}
domain-type:
  base-domain-type
  simple-subdomain-type
  sparse-subdomain-type
\end{verbatim}
\end{syntax}

Domains can be further classified according to whether they are \emph{regular}
or \emph{irregular}.  A regular domain represents a rectangular iteration
space and can have a compact representation whose size is independent
of the number of indices. Rectangular domains, with the exception of
sparse subdomains, are regular.

An irregular domain can store an arbitrary set of indices of an arbitrary but
homogeneous index type.  Irregular domains typically require space proportional
to the number of indices being represented.  All \emph{associative} domain types
and their subdomains (including sparse subdomains) are irregular.  Sparse
subdomains of regular domains are also irregular.

An index set can be either \emph{ordered} or \emph{unordered} depending on
whether its members have a well-defined order relationship.  All regular
domains are ordered.  All associative domains are
unordered.

The type of a domain describes how a domain is represented and the operations
that can be performed upon it, while its value is the set of indices it represents.
In addition to storing a value, each domain variable has an identity that
distinguishes it from other domains that may have the same type and
value.  This identity is used to define the domain's relationship
with subdomains, index types~(\rsec{Index_Types}),
and arrays~(\rsec{Association_of_Arrays_to_Domains}).

The runtime representation of a domain is controlled by its domain map.
Domain maps are presented in \rsec{Domain_Maps}.


\section{Base Domain Types and Values}
\label{Base_Domain_Types_and_Values}
\index{domains!types and values}

Base domain types can be classified as regular or irregular.  Dense and
strided rectangular domains are regular domains.
Irregular base domain types include all of the associative domain types.

\begin{syntax}
\begin{verbatim}
base-domain-type:
  rectangular-domain-type
  associative-domain-type
\end{verbatim}
\end{syntax}

These base domain types are discussed in turn in the following
subsections.

\subsection{Rectangular Domains}
\index{rectangular domains (see also domains, rectangular)}
\index{domains!rectangular}

Rectangular domains describe multidimensional rectangular index sets.  They are
characterized by a tensor product of ranges and represent indices that are
tuples of an integral type.  Because their index sets can be represented using
ranges, regular domain values typically require only $O(1)$ space.

\subsubsection{Rectangular Domain Types}
\index{domains!rectangular!types}
\index{types!rectangular domains}

Rectangular domain types are parameterized by three things:
\begin{itemize}
\item \chpl{rank} a positive \chpl{int} value indicating the number
of dimensions that the domain represents;
\item \chpl{idxType} a type member representing the index type for
each dimension; and
% BLC: we should potentially rename idxType to idxDimType to make it
% more consistent with the irregular case
\item \chpl{stridable} a \chpl{bool} parameter indicating whether
any of the domain's dimensions will be characterized by a strided
range.
\end{itemize}
If \chpl{rank} is $1$, the index type represented by a rectangular
domain is \chpl{idxType}.  Otherwise, the index type is the homogeneous
tuple type \chpl{rank*idxType}.
If unspecified, \chpl{idxType} defaults
to \chpl{int} and \chpl{stridable} defaults to \chpl{false}.

\begin{openissue}
We may represent a rectangular domain's index type as rank*idxType even if rank is 1.  This
would eliminate a lot of code currently used to support the special (rank == 1) case.
\end{openissue}

The syntax of a rectangular domain type is summarized as follows:
\begin{syntax}
\begin{verbatim}
rectangular-domain-type:
  `domain' ( named-expression-list )
\end{verbatim}
\end{syntax}

\noindent where \sntx{named-expression-list} permits the values of
\chpl{rank}, \chpl{idxType}, and \chpl{stridable} to be specified
using standard type signature.

\begin{chapelexample}{typeFunctionDomain.chpl}
The following declarations both create an uninitialized rectangular domain with three dimensions, with \chpl{int} indices:
\begin{chapel}
\begin{verbatim}
var D1 : domain(rank=3, idxType=int, stridable=false);
var D2 : domain(3);
\end{verbatim}
\end{chapel}
\begin{chapelpost}
\begin{verbatim}
writeln(D1);
writeln(D2);
\end{verbatim}
\end{chapelpost}
\begin{chapeloutput}
\begin{verbatim}
{1..0, 1..0, 1..0}
{1..0, 1..0, 1..0}
\end{verbatim}
\end{chapeloutput}
\end{chapelexample}

\subsubsection{Rectangular Domain Values}
\label{Rectangular_Domain_Values}
\index{domains!values!rectangular}
\index{domains!rectangular!values}

Each dimension of a rectangular domain is a range of
type \chpl{range(idxType, BoundedRangeType.bounded, stridable)}.
The index set for a rank~1 domain is the set of indices
described by its singleton range.  The index set for a rank~$n$
domain is the set of all \chpl{n*idxType} tuples described by the
tensor product of its ranges.  When expanded (as by an iterator), rectangular domain indices are ordered
according to the lexicographic order of their values.  That is, the index with
the highest rank is listed first and changes most slowly.\footnote{This is also
known as row-major ordering.}

%REVIEW: vass: we have not settled on the lexicographic order of the values.
% That needs to be reflected here.

%REVIEW: vass: rephrase the futures below to (a) be more formal
% and (b) motivate why we are considering them (or add some contents)?

\begin{future}
Domains defined using unbounded ranges may be supported.
\end{future}

\index{domains!rectangular!literals}

Literal rectangular domain values are represented by a comma-separated
list of range expressions of matching \chpl{idxType} enclosed in
curly braces:

%% \begin{future}
%% Will we support domains with heterogeneous index types?
%% \end{future}

\begin{syntax}
\begin{verbatim}
rectangular-domain-literal:
  { range-expression-list }

range-expression-list:
  range-expression
  range-expression, range-expression-list
\end{verbatim}
\end{syntax}

\noindent The type of a rectangular domain literal is defined as follows:

\begin{itemize}
\item \chpl{rank} = the number of range expressions in the literal;
\item \chpl{idxType} = the type of the range expressions;
\item \chpl{stridable} = \chpl{true} if any of the range expressions
are stridable, otherwise \chpl{false}.
\end{itemize}
\noindent If the index types in the ranges differ and all of them can be
promoted to the same type, then that type is used as the \chpl{idxType}.
Otherwise, the domain literal is invalid.

\begin{example}
The expression \chpl{\{1..5, 1..5\}} defines a rectangular domain with
type \chpl{domain(rank=2,} \chpl{ idxType=int,} \chpl{ stridable=false)}.
It is a $5 \times 5$ domain with the indices:
\begin{equation}
(1, 1), (1, 2), \ldots, (1, 5), (2, 1), \ldots (5, 5).
\end{equation}
\end{example}

A domain expression may contain bounds which are evaluated at runtime.
\begin{example}
In the code
\begin{chapel}
\begin{verbatim}
var D: domain(2) = {1..n, 1..n};
\end{verbatim}
\end{chapel}

\chpl{D} is defined as a two-dimensional, nonstridable rectangular
domain with an index type of \chpl{2*int} and is initialized to
contain the set of indices $(i,j)$ for all $i$ and $j$ such that
$i \in {1, 2, \ldots, n}$ and $j \in {1, 2, \ldots, n}$.
\end{example}

\index{domains!rectangular!default value}

The default value of a domain type is the \chpl{rank} default range
values for type:
\begin{quote}
\chpl{range(idxType, BoundedRangeType.bounded, stridable)}
\end{quote}

\begin{chapelexample}{rectangularDomain.chpl}
The following creates a two-dimensional rectangular domain and then uses this to
declare an array.  The array indices are iterated over using the domain's
\chpl{dim()} method, and each element is filled with
some value.  Then the array is printed out.

Thus, the code
\begin{chapel}
\begin{verbatim}
var D : domain(2) = {1..2, 1..7};
var A : [D] int;
for i in D.dim(0) do
  for j in D.dim(1) do
    A[i,j] = 7 * i**2 + j;
writeln(A);
\end{verbatim}
\end{chapel}
produces
\begin{chapelprintoutput}
\begin{verbatim}
8 9 10 11 12 13 14
29 30 31 32 33 34 35
\end{verbatim}
\end{chapelprintoutput}
\end{chapelexample}

\subsection{Associative Domains}
\index{associative domains (see also domains, associative)}

Associative domains represent an arbitrary set of indices
of a given type and can be used to describe sets or to create
dictionary-style arrays (hash tables).
The type of indices of an associative domain, or its \chpl{idxType},
can be any primitive type except \chpl{void} or any class type.

\subsubsection{Associative Domain Types}

\label{Associative_Domain_Types}
\index{types!associative domains}
\index{domains!associative}

An associative domain type is parameterized by \chpl{idxType}, the
type of the indices that it stores.  The syntax is as follows:

\begin{syntax}
\begin{verbatim}
associative-domain-type:
  `domain' ( associative-index-type )

associative-index-type:
  type-expression
\end{verbatim}
\end{syntax}

The \sntx{associative-index-type} determines the \chpl{idxType}
of the associative domain type.

When an associative domain is used as the index set of an array, the relation
between the indices and the array elements can be thought of as a map between
the values of the index set and the elements stored in the array.

\subsubsection{Associative Domain Values}
\label{Associative_Domain_Values}
\index{domains!values!associative}
\index{domains!associative!values}

An associative domain's value is simply the set of all index values
that the domain describes.  The iteration order over the indices of
an associative domain is undefined.

\index{domains!associative!literals}
\index{domains!associative!initialization}

Specification of an associative domain literal value follows a similar syntax as
rectangular domain literal values.  What differentiates the two are the types 
of expressions specified in the comma separated list.  Use of values of a 
type other than ranges will result in the construction of an associative domain.  

\begin{syntax}
\begin{verbatim}
associative-domain-literal:
   { associative-expression-list }

associative-expression-list:
   non-range-expression
   non-range-expression, associative-expression-list

non-range-expression:
   expression
\end{verbatim}
\end{syntax}

It is required that the types of the values used in constructing an associative
domain literal value be of the same type.  If the types of the indices does not
match a compiler error will be issued.

\begin{future}
Due to implementation of == over arrays it is currently not possible to use
arrays as indices within an associative domain. 
\end{future}

\begin{openissue}
Assignment of an associative domain literal results in a warning message
being printed alerting the user that whole-domain assignment has been
serialized. This results from the resize operation over associative arrays not
being parsafe. 
\end{openissue}

\begin{chapelexample}{associativeDomain.chpl}
The following example illustrates construction of an associative domain
containing string indices "bar" and "foo".  Note that due to internal hashing 
of indices the order in which the values of the associative domain are iterated
is not the same as their specification order.

This code
\begin{chapel}
\begin{verbatim}
var D : domain(string) = {"bar", "foo"};
writeln(D);
\end{verbatim}
\end{chapel}
\begin{chapelcompopts}
\begin{verbatim}
--no-warnings
\end{verbatim}
\end{chapelcompopts}
produces the output
\begin{chapelprintoutput}
\begin{verbatim}
{foo, bar}
\end{verbatim}
\end{chapelprintoutput}
\end{chapelexample}

\index{domains!associative!default values}

If uninitialized, the default value of an associative domain is the
empty index set.

Indices can be added to or removed from an associative domain
as described in \rsec{Adding_and_Removing_Domain_Indices}.


\section{Simple Subdomain Types and Values}
\label{Simple_Subdomain_Types_and_Values}
\index{subdomains}
\index{subdomains!simple}

A subdomain is a domain whose indices are guaranteed to be a subset of
those described by another domain known as its \emph{parent domain}.
A subdomain has the same type as its parent domain, and by default
it inherits the domain map of its parent domain.  All domain types
support subdomains.

Simple subdomains are subdomains which are not sparse.  Sparse
subdomains are discussed in the following section
(\rsec{Sparse_Subdomain_Types_and_Values}).  A simple subdomain
inherits its representation (regular or irregular) from its base
domain (or base subdomain).  A sparse subdomain is always irregular,
even if its base domain is regular.

In all other respects, the two kinds of subdomain behave identically.  In this
specification, ``subdomain'' refers to both simple and sparse subdomains, unless
it is specifically distinguished as one or the other.

\begin{rationale}
Subdomains are provided in Chapel for a number of reasons: to
facilitate the ability of the compiler or a reader to reason about the
inter-relationship of distinct domain variables; to support the
author's ability to omit redundant domain mapping specifications; to
support the compiler's ability to reason about the relative alignment
of multiple domains; and to improve the compiler's ability to prove
away bounds checks for array accesses.
\end{rationale}

\subsection{Simple Subdomain Types}
\label{Simple_Subdomain_Types}
\index{subdomains!simple!types}
\index{subdomains!types!simple}
\index{types!subdomains!simple}

A simple subdomain type is specified using the following syntax:
\begin{syntax}
\begin{verbatim}
simple-subdomain-type:
  `subdomain' ( domain-expression )
\end{verbatim}
\end{syntax}

This declares that \sntx{domain-expression} is the parent domain of
this subdomain type.  A simple subdomain specifies a subdomain
with the same underlying representation as its base domain.  

\begin{openissue}

An open semantic issue for subdomains is when a subdomain's subset
property should be re-verified once its parent domain is reassigned
and whether this should be done aggressively or lazily.

\end{openissue}

\subsection{Simple Subdomain Values}
\index{subdomains!simple!values}
\index{values!subdomains!simple}

The value of a simple subdomain is the set of all index values
that the subdomain describes.

\index{subdomains!simple!default values}

The default value of a simple subdomain type is the same as the default value
of its parent's type
(\rsec{Rectangular_Domain_Values}, \rsec{Associative_Domain_Values}).

A simple subdomain variable can be initialized or assigned to
with a tuple of values of the parent's \chpl{idxType}.
Indices can also be added to or removed from a simple subdomain
as described in \rsec{Adding_and_Removing_Domain_Indices}.
It is an error to attempt to add an index to a subdomain that is not also
a member of the parent domain.


\section{Sparse Subdomain Types and Values}
\label{Sparse_Subdomain_Types_and_Values}
\index{domains!sparse}
\index{subdomains!sparse}

\begin{syntax}
\begin{verbatim}
sparse-subdomain-type:
  `sparse' `subdomain'[OPT] ( domain-expression )
\end{verbatim}
\end{syntax}

This declaration creates a sparse subdomain.
 \emph{Sparse subdomains} are irregular domains that describe an
arbitrary subset of a domain, even if the parent domain is a regular
domain.  Sparse subdomains are useful in Chapel for
defining \emph{sparse arrays} in which a single element value (usually ``zero'')
 occurs
frequently enough that it is worthwhile to avoid storing it
redundantly.  The set difference between a sparse subdomain's index set
and that of parent domain is the set of indices for which the
sparse array will store this replicated value.
%%NB: This is a nice mathematical definition, but do we really want to torque
%%the reader's brain with the notion of redundant values?
%REVIEW:hilde -- I would suggest "uninteresting values" or values that can be omitted.
See~\rsec{Sparse_Arrays} for details about sparse arrays.

\subsection{Sparse Subdomain Types}
\index{domains!sparse!types}
\index{subdomains!sparse!types}
\index{types!domains!sparse}
\index{types!subdomains!sparse}

Each root domain type has a unique corresponding sparse subdomain
type.  Sparse subdomains whose parent domains are also sparse
subdomains share the same type.

\subsection{Sparse Subdomain Values}
\label{Sparse_Domain_Values}
\index{domains!sparse!values}
\index{subdomains!sparse!values}
\index{values!domains!sparse}
\index{values!subdomains!sparse}

A sparse subdomain's value is simply the set of all index values that
the domain describes.  If the parent domain defines an iteration order
over its indices, the sparse subdomain inherits that order.

\index{sparse domains!literals!lack thereof}
\index{sparse domains!initialization}
\index{domains!sparse!initialization}
\index{initialization!sparse domains}
There is no literal syntax for a sparse subdomain.  However, a variable of a
sparse subdomain type can be initialized using a tuple of values
of the parent domain's index type.

\index{sparse domains!default value}
\index{domains!sparse!default value}
The default value for a sparse subdomain value is the empty set.

\begin{example}
The following code declares a two-dimensional dense domain \chpl{D},
followed by a two dimensional sparse subdomain of \chpl{D}
named \chpl{SpsD}.  Since \chpl{SpsD} is uninitialized, it will
initially describe an empty set of indices from \chpl{D}.
\begin{chapel}
\begin{verbatim}
const D: domain(2) = {1..n, 1..n};
var SpsD: sparse subdomain(D);
\end{verbatim}
\end{chapel}
\end{example}

\section{Domain Index Types}
\label{Index_Types}
\index{domains!index types}

Each domain value has a corresponding compiler-provided \emph{index
type} which can be used to represent values belonging to that domain's
index set.  Index types are described using the following syntax:

\begin{syntax}
\begin{verbatim}
index-type:
  `index' ( domain-expression )
\end{verbatim}
\end{syntax}

A variable with a given index type is constrained to take on only values
available within the domain on which it is defined.  This restriction allows the
compiler to prove away the bound checking that code safety considerations might
otherwise require.  Due to the subset relationship between a base domain and its
subdomains, a variable of an index type defined with respect to a subdomain is
also necessarily a valid index into the base domain.

Since an index types are known to be legal for a given domain, it may
also afford the opportunity to represent that index using an optimized
format that doesn't simply store the index variable's value.  This fact could be
used to support accelerated access to arrays declared over that domain.  For
example, iteration over an index type could be implemented using memory pointers
and strides, rather than explicitly calculating the offset of each index
within the domain.

These potential optimizations may make it less expensive to
index into arrays using index type variables of their domains or
subdomains.

In addition, since an index type is associated with a specific domain or subdomain, it
carries more semantic weight than a generic index.  For example, one could
iterate over a rectangular domain with integer bounds using an \chpl{int(n)} as
the index variable.  However, it would be more precise to use a variable of the
domain's index type.

\begin{openissue}

An open issue for index types is what the semantics should be for an
index type value that is live across a modification to its domain's
index set---particularly one that shrinks the index set.  Our
hypothesis is that most stored indices will either have short
lifespans or belong to constant or monotonically growing domains.  But
these semantics need to be defined nevertheless.

\end{openissue}

\section{Iteration Over Domains}
\label{Iteration_over_Domains}
\index{domains!iteration}
\index{iteration!domain}

All domains support iteration via standard \chpl{for}, \chpl{forall}, and \chpl{coforall}
loops.  These loops iterate over all of the indices that the domain
describes.  If the domain defines an iteration order of its indices,
then the indices are visited in that order.  

The type of the iterator variable for an iteration over a
domain named \chpl{D} is that domain's index type, \chpl{index(D)}.


\section{Domains as Arguments}
\label{Domain_Arguments}
\index{domains!as arguments}
\index{argument passing!domains}

This section describes the semantics of passing domains as arguments
to functions.

\subsection{Formal Arguments of Domain Type}

When a domain value is passed to a formal argument of compatible
domain type by default intent, it is passed by reference in order to
preserve the domain's identity.

\subsection{Domain Promotion of Scalar Functions}
\label{Domain_Promotion_of_Scalar_Functions}
\index{domains!promotion}
\index{promotion!domain}

Domain values may be passed to a scalar function argument whose type
matches the domain's index type.  This results in a promotion of the
scalar function as defined in~\rsec{Promotion}.

\begin{example}
Given a function \chpl{foo()} that accepts real floating point values
and an associative domain \chpl{D} of
type \chpl{domain(real)}, \chpl{foo} can be called with \chpl{D} as
its actual argument which will result in the function being invoked
for each value in the index set of \chpl{D}.
\end{example}

\begin{example}
Given an array \chpl{A} with element type \chpl{int} declared over a
one-dimensional domain \chpl{D} with \chpl{idxType} \chpl{int}, the
array elements can be assigned their corresponding index values by
writing:
\begin{chapel}
\begin{verbatim}
A = D;
\end{verbatim}
\end{chapel}
This is equivalent to:
\begin{chapel}
\begin{verbatim}
forall (a,i) in zip(A,D) do
  a = i;
\end{verbatim}
\end{chapel}
\end{example}


\section{Domain Operations}

Chapel supplies predefined operators and functions that can be used to manipulate
domains.  Unless otherwise noted, these operations are applicable to a domain of
any type, whether a base domain or a subdomain.

\subsection{Domain Assignment}
\label{Domain_Assignment}
\index{domains!assignment}
\index{assignment!domain}

All domain types support domain assignment.  

\begin{syntax}
\begin{verbatim}
domain-expression:
  domain-literal
  domain-name
  domain-assignment-expression
  domain-striding-expression
  domain-alignment-expression
  domain-slice-expression

domain-literal:
  rectangular-domain-literal
  associative-domain-literal

domain-assignment-expression:
  domain-name = domain-expression

domain-name:
  identifier
\end{verbatim}
\end{syntax}

Domain assignment is by
value and causes the target domain variable to take on the index set
of the right-hand side expression.  In practice, the right-hand side
expression is often another domain value; a tuple of ranges (for
regular domains); or a tuple of indices or a loop that enumerates
indices (for irregular domains).  If the domain variable being
assigned was used to declare arrays, these arrays are reallocated as
discussed in~\rsec{Association_of_Arrays_to_Domains}.

It is an error to assign a stridable domain to an unstridable domain
without an explicit conversion.

\begin{example}
The following three assignments show ways of assigning indices to a
sparse domain, \chpl{SpsD}.  The first assigns the domain two index
values, \chpl{(1,1)} and \chpl{(n,n)}.  The second assigns the domain
all of the indices along the diagonal from
\chpl{(1,1)}$\ldots$\chpl{(n,n)}.  The third invokes an iterator that
is written to \chpl{yield} indices read from a file named
``inds.dat''.  Each of these assignments has the effect of replacing
the previous index set with a completely new set of values.
\begin{chapel}
\begin{verbatim}
SpsD = ((1,1), (n,n));
SpsD = [i in 1..n] (i,i);
SpsD = readIndicesFromFile("inds.dat");
\end{verbatim}
\end{chapel}
\end{example}

\subsection{Domain Striding}
\label{Domain_Striding}
\index{domains!striding}
\index{by@\chpl{by}!on rectangular domains}
\index{operators!by (domain)@\chpl{by} (domain)}

The \chpl{by} operator can be applied to a rectangular domain value in
order to create a strided rectangular domain value.  The right-hand
operand to the \chpl{by} operator can either be an integral value or
an integral tuple whose size matches the domain's rank.

\begin{syntax}
\begin{verbatim}
domain-striding-expression:
  domain-expression `by' expression
\end{verbatim}
\end{syntax}

The type of the resulting domain is the same as the original domain
but with \chpl{stridable} set to true.  In the case of an integer
stride value, the value of the resulting domain is computed by
applying the integer value to each range in the value using the
\chpl{by} operator.  In the case of a tuple stride value, the resulting
domain's value is computed by applying each tuple component to the
corresponding range using the \chpl{by} operator.


\subsection{Domain Alignment}
\label{Domain_Alignment}
\index{domains!alignment}
\index{align@\chpl{align}!on rectangular domains}
\index{operators!align (domain)@\chpl{align} (domain)}

The \chpl{align} operator can be applied to a rectangular domain value in
order to change the alignment of a rectangular domain value.  The right-hand
operand to the \chpl{align} operator can either be an integral value or
an integral tuple whose size matches the domain's rank.

\begin{syntax}
\begin{verbatim}
domain-alignment-expression:
  domain-expression `align' expression
\end{verbatim}
\end{syntax}

The type of the resulting domain is the same as the original domain
but with \chpl{stridable} set to true.
In the case of an integer alignment value, the value of the resulting
domain is computed by applying the integer value to each range in the
value using the \chpl{align} operator.  In the case of a tuple
alignment value, the resulting domain's value is computed by applying
each tuple component to the corresponding range using the \chpl{align}
operator.


\subsection{Domain Slicing}
\label{Domain_Slicing}
\index{slicing!domains}
\index{domains!slicing}

Slicing is the application of an index set to a domain.
It can be written using either parentheses or square brackets.
The index set can be defined with either a domain or a list of ranges.

\begin{syntax}
\begin{verbatim}
domain-slice-expression:
  domain-expression [ slicing-index-set ]
  domain-expression ( slicing-index-set )

slicing-index-set:
  domain-expression
  range-expression-list
\end{verbatim}
\end{syntax}

The result of slicing, or a \emph{slice}, is a new domain value
that represents the intersection of
the index set of the domain being sliced and
the index set being applied.
The type and domain map of the slice match the domain being sliced.

Slicing can also be performed on an array,
resulting in aliasing a subset of the array's elements
(\rsec{Array_Slicing}).

\subsubsection{Domain-based Slicing}
\index{domain-based slicing}
\index{slicing!domain-based}

If the brackets or parentheses contain a domain value,
its index set is applied for slicing.

\begin{openissue}
Can we say that it is an alias in the case of sparse/associative?
% If so, need to reconcile getting an "alias" with getting
% a "new domain value", as claimed earlier.
\end{openissue}

\subsubsection{Range-based Slicing}
\label{Range_Based_Slicing}
\index{slicing!range-based}
\index{range-based slicing}
When slicing rectangular domains or arrays, the brackets or parentheses
can contain a list of \chpl{rank} ranges.  These ranges can either be bounded
or unbounded.
%
%REVIEW: vass: no, they don't inherit. We should either drop
% the following sentence or rephrase it to make it correct.
When unbounded, they inherit their bounds from the
domain or array being sliced.
%
The Cartesian product of the ranges' index sets is applied for slicing.

\begin{example}
The following code declares a two dimensional rectangular
domain \chpl{D}, and then a number of subdomains of \chpl{D} by
slicing into \chpl{D} using bounded and unbounded ranges.
The \chpl{InnerD} domain describes the inner indices of
D, \chpl{Col2OfD} describes the 2nd column of
\chpl{D}, and \chpl{AllButLastRow} describes all of \chpl{D} except
for the last row.

\begin{chapel}
\begin{verbatim}
const D: domain(2) = {1..n, 1..n},
      InnerD = D[2..n-1, 2..n-1],
      Col2OfD = D[.., 2..2],
      AllButLastRow = D[..n-1, ..];
\end{verbatim}
\end{chapel}
\end{example}

\subsubsection{Rank-Change Slicing}
\label{Rank_Change_Slicing}
\index{slicing!rank-change}
\index{rank-change slicing}

For multidimensional rectangular domains and arrays, substituting
integral values for one or more of the ranges in a range-based slice
will result in a domain or array of lower rank.

The result of a rank-change slice on an array is an alias to a subset
of the array's elements as described
in~\rsec{Rectangular_Array_Slicing}.

The result of rank-change slice on a domain is a subdomain of the
domain being sliced.  The resulting
subdomain's type will be the same as the original domain, but with
a \chpl{rank} equal to the number of dimensions that were sliced by
ranges rather than integers.


\subsection{Count Operator}
\label{Count_Operator_Domains}
\index{domains!count operator}
\index{domains!#@\chpl{\#}}
\index{# (domain)@\chpl{\#} (domain)}
\index{operators!# (domain)@\chpl{\#} (domain)}
The \chpl{\#} operator can be applied to dense rectangular domains with
a tuple argument whose size matches the rank of the domain (or
optionally an integer in the case of a 1D domain).  The operator is
equivalent to applying the \chpl{\#} operator to the component ranges
of the domain and then using them to slice the domain as in
Section~\ref{Range_Based_Slicing}.


\subsection{Adding and Removing Domain Indices}
\label{Adding_and_Removing_Domain_Indices}
\index{domains!adding indices}
\index{domains!removing indices}

All irregular domain types support the ability to incrementally add
and remove indices from their index sets.  This can either be done
using \chpl{add(i:idxType)} and \chpl{remove(i:idxType)} methods on a
domain variable or by using the \chpl{+=} and \chpl{-=} assignment
operators.  It is legal to add the same index to an irregular domain's
index set twice, but illegal to remove an index that does not belong
to the domain's index set.

\begin{openissue}
These remove semantics seem dangerous in a parallel context; maybe
add flags to both the method versions of the call that say whether
they should balk or not?  Or add exceptions...
\end{openissue}

As with normal domain assignments, arrays declared in terms of a
domain being modified in this way will be reallocated as discussed
in~\rsec{Association_of_Arrays_to_Domains}.


\section{Predefined Methods on Domains}
\index{domains!predefined functions}

This section gives a brief description of the library functions provided for
Domains.  These are categorized by the type of domain to which they apply: all,
regular or irregular.  Within each subsection, entries are listed in
alphabetical order.

\subsection{Methods on All Domain Types}
\index{domains!methods!common}
\index{domains!common methods}

The methods in this subsection can be applied to any domain.

\index{domains!clear@\chpl{clear}}
\index{predefined functions!clear@\chpl{clear}}
\begin{protohead}
\begin{verbatim}
proc domain.clear()
\end{verbatim}
\end{protohead}
\begin{protobody}
Resets this domain's index set to the empty set.
\end{protobody}

\begin{chapelexample}{clearAssociativeDomain}
This function provides a way to produce an empty associative domain.

When run, the code
\begin{chapel}
\begin{verbatim}
enum Counter { one, two, three };
var D : domain ( Counter ) = {Counter.one, Counter.two};
writeln("D has ", D.numIndices, " indices.");
D.clear();
writeln("D has ", D.numIndices, " indices.");
\end{verbatim}
\end{chapel}
prints out
\begin{chapelprintoutput}
\begin{verbatim}
D has 2 indices.
D has 0 indices.
\end{verbatim}
\end{chapelprintoutput}
\end{chapelexample}

\index{domains!dist@\chpl{dist}}
\index{predefined functions!dist@\chpl{dist}}
\begin{protohead}
\begin{verbatim}
proc domain.dist : dmap
\end{verbatim}
\end{protohead}
\begin{protobody}
Returns the domain map that implements this domain
\end{protobody}

\begin{chapelexample}{getDomainMap}
In the code
\begin{chapel}
\begin{verbatim}
use BlockDist;
proc foo(d : domain) where isSubtype(d.dist.type, Block) {
  writeln("Block-distributed domain");
}
proc foo(d : domain) {
  writeln("Unknown distribution");
}
var D = {1..10} dmapped Block({1..10});
foo(D);
\end{verbatim}
\end{chapel}
\chpl{dist} is used in a where-clause to determine the type of the argument's
distribution. The output is:
\begin{chapelprintoutput}
\begin{verbatim}
Block-distributed domain
\end{verbatim}
\end{chapelprintoutput}
\end{chapelexample}

\index{domains!idxType@\chpl{idxType}}
\index{predefined functions!idxType@\chpl{idxType}}
\begin{protohead}
\begin{verbatim}
proc domain.idxType type
\end{verbatim}
\end{protohead}
\begin{protobody}
Returns the domain type's \chpl{idxType}.
\end{protobody}

\begin{protohead}
\begin{verbatim}
proc domain.indexOrder(i: index(domain)): idxType
\end{verbatim}
\end{protohead}
\begin{protobody}
If \chpl{i} is a member of the domain, returns the ordinal value of
\chpl{i} using a total ordering of the domain's indices using 0-based
indexing.  Otherwise, it returns \chpl{(-1):idxType}.  For rectangular
domains, this ordering will be based on a row-major ordering of the
indices; for other domains, the ordering may be
implementation-defined and unstable as indices are added and
removed from the domain.
\end{protobody}

\index{domains!isIrregularDom@\chpl{isIrregularDom}}
\index{predefined functions!isIrregularDom@\chpl{isIrregularDom}}
\begin{protohead}
\begin{verbatim}
proc isIrregularDom(d: domain) param
\end{verbatim}
\end{protohead}
\begin{protobody}
Returns a param \chpl{true} if the given domain is irregular, false otherwise.
\end{protobody}

\index{domains!isRectangularDom@\chpl{isRectangularDom}}
\index{predefined functions!isRectangularDom@\chpl{isRectangularDom}}
\begin{protohead}
\begin{verbatim}
proc isRectangularDom(d: domain) param
\end{verbatim}
\end{protohead}
\begin{protobody}
Returns a param \chpl{true} if the given domain is rectangular, false otherwise.
\end{protobody}

\index{domains!isSparseDom@\chpl{isSparseDom}}
\index{predefined functions!isSparseDom@\chpl{isSparseDom}}
\begin{protohead}
\begin{verbatim}
proc isSparseDom(d: domain) param
\end{verbatim}
\end{protohead}
\begin{protobody}
Returns a param \chpl{true} if the given domain is sparse, false otherwise.
\end{protobody}

\index{domains!member@\chpl{member}}
\index{predefined functions!member (domain)@\chpl{member} (domain)}
\begin{protohead}
\begin{verbatim}
proc domain.member(i)
\end{verbatim}
\end{protohead}
\begin{protobody}
Returns true if the given index \chpl{i} is a member of this domain's index set,
and false otherwise.
\end{protobody}

\begin{openissue}
We would like to call the type of i above idxType, but it's not true
for rectangular domains.  That observation provides some motivation to normalize
the behavior.
\end{openissue}

%REVIEW: vass: need to define 'capType' or replace with something that is defined
\index{domains!numIndices@\chpl{numIndices}}
\index{predefined functions!numIndices (domain)@\chpl{numIndices} (domain)}
\begin{protohead}
\begin{verbatim}
proc domain.numIndices: capType
\end{verbatim}
\end{protohead}
\begin{protobody}
Returns the number of indices in the domain as a value of the capacity type.
\end{protobody}

\subsection{Methods on Regular Domains}
\index{domains!methods!regular}
\index{regular domains!methods}

The methods described in this subsection can be applied to regular domains only.

\index{domains!dim@\chpl{dim}}
\index{predefined functions!dim (domain)@\chpl{dim} (domain)}
\begin{protohead}
\begin{verbatim}
proc domain.dim(d: int): range
\end{verbatim}
\end{protohead}
\begin{protobody}
Returns the range of indices described by dimension \chpl{d} of the
domain.
\end{protobody}

\begin{example}
In the code
\begin{chapel}
\begin{verbatim}
for i in D.dim(1) do
  for j in D.dim(2) do
    writeln(A(i,j));
\end{verbatim}
\end{chapel}
domain \chpl{D} is iterated over by two nested loops.  The first
dimension of \chpl{D} is iterated over in the outer loop.  The second
dimension is iterated over in the inner loop.
\end{example}

\index{domains!dims@\chpl{dims}}
\index{predefined functions!dims (domain)@\chpl{dims} (domain)}
\begin{protohead}
\begin{verbatim}
proc domain.dims(): rank*range
\end{verbatim}
\end{protohead}
\begin{protobody}
Returns a tuple of ranges describing the dimensions of the domain.
\end{protobody}

% BLC: ``integral'' isn't really correct in the two 1D cases below,
% however, we don't really seem to have a user-level name for the
% per-dimension index type in the language that I can see.

\index{domains!expand@\chpl{expand}}
\index{predefined functions!expand (domain)@\chpl{expand} (domain)}
\begin{protohead}
\begin{verbatim}
proc domain.expand(off: integral): domain
proc domain.expand(off: rank*integral): domain
\end{verbatim}
\end{protohead}
\begin{protobody}
Returns a new domain that is the current domain expanded in
dimension \chpl{d} if \chpl{off} or \chpl{off(d)} is positive or
contracted in dimension \chpl{d} if \chpl{off} or \chpl{off(d)} is
negative.
\end{protobody}

\index{domains!exterior@\chpl{exterior}}
\index{predefined functions!exterior (domain)@\chpl{exterior} (domain)}
\begin{protohead}
\begin{verbatim}
proc domain.exterior(off: integral): domain
proc domain.exterior(off: rank*integral): domain
\end{verbatim}
\end{protohead}
\begin{protobody}
Returns a new domain that is the exterior portion of the current
domain with \chpl{off} or \chpl{off(d)} indices for each
dimension \chpl{d}.  If \chpl{off} or \chpl{off(d)} is negative,
compute the exterior from the low bound of the dimension; if positive,
compute the exterior from the high bound.
\end{protobody}

\index{domains!high@\chpl{high}}
\index{predefined functions!high (domain)@\chpl{high} (domain)}
\begin{protohead}
\begin{verbatim}
proc domain.high: index(domain)
\end{verbatim}
\end{protohead}
\begin{protobody}
Returns the high index of the domain as a value of the domain's index
type.
\end{protobody}

\index{domains!interior@\chpl{interior}}
\index{predefined functions!interior (domain)@\chpl{interior} (domain)}
\begin{protohead}
\begin{verbatim}
proc domain.interior(off: integral): domain
proc domain.interior(off: rank*integral): domain
\end{verbatim}
\end{protohead}
\begin{protobody}
Returns a new domain that is the interior portion of the current
domain with \chpl{off} or \chpl{off(d)} indices for each
dimension \chpl{d}.  If \chpl{off} or \chpl{off(d)} is negative,
compute the interior from the low bound of the dimension; if positive,
compute the interior from the high bound.
\end{protobody}

\index{domains!low@\chpl{low}}
\index{predefined functions!low (domain)@\chpl{low} (domain)}
\begin{protohead}
\begin{verbatim}
proc domain.low: index(domain)
\end{verbatim}
\end{protohead}
\begin{protobody}
Returns the low index of the domain as a value of the domain's index
type.
\end{protobody}

\index{domains!rank@\chpl{rank}}
\index{predefined functions!rank (domain)@\chpl{rank} (domain)}
\begin{protohead}
\begin{verbatim}
proc domain.rank param : int
\end{verbatim}
\end{protohead}
\begin{protobody}
Returns the rank of the domain.
\end{protobody}

\index{domains!size@\chpl{size}}
\index{predefined functions!size (domain)@\chpl{size} (domain)}
\begin{protohead}
\begin{verbatim}
proc domain.size: capType
\end{verbatim}
\end{protohead}
\begin{protobody}
Same as $Domain$.numIndices.
\end{protobody}

\index{domains!stridable@\chpl{stridable}}
\index{predefined functions!stridable (domain)@\chpl{stridable} (domain)}
\begin{protohead}
\begin{verbatim}
proc domain.stridable param : bool
\end{verbatim}
\end{protohead}
\begin{protobody}
Returns whether or not the domain is stridable.
\end{protobody}

\index{domains!stride@\chpl{stride}}
\index{predefined functions!stride (domain)@\chpl{stride} (domain)}
\begin{protohead}
\begin{verbatim}
proc domain.stride: int(numBits(idxType)) where rank == 1
proc domain.stride: rank*int(numBits(idxType))
\end{verbatim}
\end{protohead}
\begin{protobody}
Returns the stride of the domain as the domain's stride type (for 1D
domains) or a tuple of the domain's stride type (for multidimensional
domains).
\end{protobody}

\index{domains!translate@\chpl{translate}}
\index{predefined functions!translate (domain)@\chpl{translate} (domain)}
\begin{protohead}
\begin{verbatim}
proc domain.translate(off: integral): domain
proc domain.translate(off: rank*integral): domain
\end{verbatim}
\end{protohead}
\begin{protobody}
Returns a new domain that is the current domain translated
by \chpl{off} or \chpl{off(d)} for each dimension \chpl{d}.
\end{protobody}

%% \begin{protohead} **/
%% proc $Domain$.position(i: index($Domain$)): rank*idxType **/
%% \end{protohead} **/
%% \begin{protobody} **/
%% Returns a tuple holding the order of index i in each range defining **/
%% the domain. **/
%% \end{protobody} **/

\subsection{Methods on Irregular Domains}
\index{domains!methods!irregular}
\index{irregular domains!methods}

The following methods are available only on irregular domain types.

\index{domains!+@\chpl{+}}
\index{predefined functions!+ (domain)@\chpl{+} (domain)}
\begin{protohead}
\begin{verbatim}
proc +(d: domain, i: index(d))
proc +(i, d: domain) where i: index(d)
\end{verbatim}
\end{protohead}
\begin{protobody}
Adds the given index to the given domain.  If the given index is already a
member of that domain, it is ignored.
\end{protobody}

\index{domains!+@\chpl{+}}
\index{predefined functions!+ (domain)@\chpl{+} (domain)}
\begin{protohead}
\begin{verbatim}
proc +(d1: domain, d2: domain)
\end{verbatim}
\end{protohead}
\begin{protobody}
Merges the index sets of the two domain arguments.
\end{protobody}

\index{domains!-@\chpl{-}}
\index{predefined functions!- (domain)@\chpl{-} (domain)}
\begin{protohead}
\begin{verbatim}
proc -(d: domain, i: index(d))
\end{verbatim}
\end{protohead}
\begin{protobody}
Removes the given index from the given domain.  It is an error if the domain
does not contain the given index.
\end{protobody}

\index{domains!-@\chpl{-}}
\index{predefined functions!- (domain)@\chpl{-} (domain)}
\begin{protohead}
\begin{verbatim}
proc -(d1: domain, d2: domain)
\end{verbatim}
\end{protohead}
\begin{protobody}
Removes the indices in domain \chpl{d2} from those in \chpl{d1}.  It is an error
if \chpl{d2} contains indices which are not also in \chpl{d1}.
\end{protobody}

\index{domains!requestCapacity@\chpl{requestCapacity}}
\index{predefined functions!requestCapacity@\chpl{requestCapacity}}
\begin{protohead}
\begin{verbatim}
proc requestCapacity(s: int)
\end{verbatim}
\end{protohead}
\begin{protobody}
Resizes the domain internal storage to hold at least \chpl{s} indices.
\end{protobody}

\cleardoublepage
\sekshun{Arrays}
\label{Arrays}
\index{arrays}

An \emph{array} is a map from a domain's indices to a collection of
variables of homogeneous type.  Since Chapel domains support a rich
variety of index sets, Chapel arrays are also richer than the
traditional linear or rectilinear array types in conventional
languages.  Like domains, arrays may be distributed across multiple
locales without explicitly partitioning them using Chapel's Domain
Maps~(\rsec{Domain_Maps}).


\section{Array Types}
\label{Array_Types}
\index{arrays!types}

An array type is specified by the identity of the domain that it is
declared over and the element type of the array.  Array types are
given by the following syntax:

\begin{syntax}
array-type:
  [ domain-expression ] type-specifier
\end{syntax}
The \sntx{domain-expression} must specify a domain that the array can
be declared over.  If the \sntx{domain-expression} is a domain
literal, the curly braces around the literal may be omitted.

\begin{chapelexample}{decls.chpl}
In the code
\begin{chapel}
const D: domain(2) = {1..10, 1..10};
var A: [D] real;
\end{chapel}
\begin{chapelpost}
writeln(D);
writeln(A);
\end{chapelpost}
\begin{chapeloutput}
{1..10, 1..10}
0.0 0.0 0.0 0.0 0.0 0.0 0.0 0.0 0.0 0.0
0.0 0.0 0.0 0.0 0.0 0.0 0.0 0.0 0.0 0.0
0.0 0.0 0.0 0.0 0.0 0.0 0.0 0.0 0.0 0.0
0.0 0.0 0.0 0.0 0.0 0.0 0.0 0.0 0.0 0.0
0.0 0.0 0.0 0.0 0.0 0.0 0.0 0.0 0.0 0.0
0.0 0.0 0.0 0.0 0.0 0.0 0.0 0.0 0.0 0.0
0.0 0.0 0.0 0.0 0.0 0.0 0.0 0.0 0.0 0.0
0.0 0.0 0.0 0.0 0.0 0.0 0.0 0.0 0.0 0.0
0.0 0.0 0.0 0.0 0.0 0.0 0.0 0.0 0.0 0.0
0.0 0.0 0.0 0.0 0.0 0.0 0.0 0.0 0.0 0.0
\end{chapeloutput}
\chpl{A} is declared to be an arithmetic array over rectangular
domain \chpl{D} with elements of type \chpl{real}.  As a result, it
represents a 2-dimensional $10 \times 10$ real floating point
variables indexed using the indices $(1, 1), (1, 2), \ldots, (1, 10),
(2, 1), \ldots, (10, 10)$.
\end{chapelexample}

%
% should the following be moved elsewhere?  Should handle these
% param/type queries consistently between this chapter and domains
% (and ranges?)
%
\index{arrays!element type}
An array's element type can be referred to using the member symbol
\chpl{eltType}.

\begin{chapelexample}{eltType.chpl}
In the following example, \chpl{x} is declared to be of type
\chpl{real} since that is the element type of array \chpl{A}.
\begin{chapelpre}
const D: domain(2) = {1..10, 1..10};
\end{chapelpre}
\begin{chapel}
var A: [D] real;
var x: A.eltType;
\end{chapel}
\begin{chapelpost}
writeln(x.type:string);
writeln(A.eltType:string);
\end{chapelpost}
\begin{chapeloutput}
real(64)
real(64)
\end{chapeloutput}
\end{chapelexample}

\section{Array Values}
\label{Array_Values}
\index{arrays!values}
\index{arrays!initialization}
\index{initialization!arrays}

An array's value is the collection of its elements' values.
Assignments between array variables are performed by value as
described in~\rsec{Array_Assignment}.  Chapel semantics are defined so
that the compiler will never need to insert temporary arrays of the
same size as a user array variable.

\index{arrays!literals}

Array literal values can be either rectangular or associative, corresponding to
the underlying domain which defines its indices. 

\begin{syntax}
array-literal:
  rectangular-array-literal
  associative-array-literal
\end{syntax}

\subsection{Rectangular Array Literals}
\label{Rectangular_Array_Literals}
\index{rectangular array literals}
\index{arrays!rectangular!literals}

Rectangular array literals are specified by enclosing a comma separated list of 
expressions representing values in square brackets. A 1-based domain will 
automatically be generated for the given array literal.  The type of the array's 
values will be the type of the first element listed. A trailing comma is
allowed.

\begin{syntax}
rectangular-array-literal:
  [ expression-list ]
  [ expression-list , ]
\end{syntax}

\begin{chapelexample}{adecl-literal.chpl}
The following example declares a 5 element rectangular array literal 
containing strings, then subsequently prints each string element to the console.
\begin{chapel}
var A = ["1", "2", "3", "4", "5"];

for i in 1..5 do
  writeln(A[i]);
\end{chapel}
\begin{chapeloutput}
1
2
3
4
5
\end{chapeloutput}
\end{chapelexample}

\begin{future}
Provide syntax which allows users to specify the domain for a rectangular 
array literal.
\end{future}

\begin{future}
Determine the type of a rectangular array literal based on the most promoted 
type, rather than the first element's type.
\end{future}

\begin{chapelexample}{decl-with-anon-domain.chpl}
The following example declares a 2-element array \chpl{A} containing 3-element
arrays of real numbers.  \chpl{A} is initialized using array literals.
\begin{chapel}
var A: [1..2] [1..3] real = [[1.1, 1.2, 1.3], [2.1, 2.2, 2.3]];
\end{chapel}
\begin{chapelpost}
writeln(A.domain);
\end{chapelpost}
\begin{chapeloutput}
{1..2}
\end{chapeloutput}
\end{chapelexample}

\begin{openissue}
We would like to differentiate syntactically between array literals for an array
of arrays and a multi-dimensional array. 
\end{openissue}

\index{arrays!rectangular!default values}
An rectangular array's default value is for each array element to be initialized to
the default value of the element type.

\subsection{Associative Array Literals}
\label{Associative_Array_Literals}
\index{associative array literals}
\index{arrays!associative!literals}

Associative array values are specified by enclosing a comma separated list of
index-to-value bindings within square brackets. It is expected that the indices 
in the listing match in type and, likewise, the types of values in the listing 
also match. A trailing comma is allowed.

\begin{syntax}
associative-array-literal:
  [ associative-expr-list ]
  [ associative-expr-list , ]

associative-expr-list:
  index-expr => value-expr
  index-expr => value-expr, associative-expr-list

index-expr:
  expression

value-expr:
  expression
\end{syntax}

\begin{openissue}
Currently it is not possible to use other associative domains as values within
an associative array literal.
\end{openissue}

\begin{chapelexample}{adecl-assocLiteral.chpl}
The following example declares a 5 element associative array literal which maps
integers to their corresponding string representation. The indices and their
corresponding values are then printed. 
\begin{chapel}
var A = [1 => "one", 10 => "ten", 3 => "three", 16 => "sixteen"];

for da in zip (A.domain, A) do
  writeln(da);
\end{chapel}
\begin{chapelprediff}
\#!/usr/bin/env sh
testname=$1
outfile=$2
sort $outfile > $outfile.2
mv $outfile.2 $outfile
\end{chapelprediff}
\begin{chapeloutput}
(1, one)
(10, ten)
(16, sixteen)
(3, three)
\end{chapeloutput}
\end{chapelexample}

\subsection{Runtime Representation of Array Values}
\label{Array_Runtime_Representation}
\index{arrays!runtime representation}
\index{arrays!domain maps}

The runtime representation of an array in memory is controlled by its
domain's domain map.  Through this mechanism, users can reason about
and control the runtime representation of an array's elements.  See
~\rsec{Domain_Maps} for more details.


\section{Array Indexing}
\label{Array_Indexing}
\index{arrays!indexing}
\index{indexing!arrays}

Arrays can be indexed using index values from the domain over which
they are declared.  Array indexing is expressed using either
parentheses or square brackets.  This results in a reference to the
element that corresponds to the index value.

% NEED SYNTAX DIAGRAM HERE?

\begin{chapelexample}{array-indexing.chpl}
Given:
\begin{chapel}
var A: [1..10] real;
\end{chapel}
the first two elements of A can be assigned the value 1.2 and 3.4
respectively using the assignment:
\begin{chapel}
A(1) = 1.2;
A[2] = 3.4;
\end{chapel}
\begin{chapelpost}
writeln(A.domain);
writeln(A);
\end{chapelpost}
\begin{chapeloutput}
{1..10}
1.2 3.4 0.0 0.0 0.0 0.0 0.0 0.0 0.0 0.0
\end{chapeloutput}
\end{chapelexample}

Except for associative arrays, if an array is indexed using an index that
is not part of its domain's index set, the reference is considered
out-of-bounds and a runtime error will occur, halting the program.

\subsection{Rectangular Array Indexing}
\label{Rectangular_Array_Indexing}
\index{indexing!rectangular arrays}
\index{rectangular arrays!indexing}

Since the indices for multidimensional rectangular domains are tuples,
for convenience, rectangular arrays can be indexed using the list of
integer values that make up the tuple index.  This is semantically
equivalent to creating a tuple value out of the integer values and
using that tuple value to index the array.  For symmetry,
1-dimensional rectangular arrays can be accessed using 1-tuple indices
even though their index type is an integral value.  This is
semantically equivalent to de-tupling the integral value from the
1-tuple and using it to index the array.

\begin{chapelexample}{array-indexing-2.chpl}
Given:
\begin{chapel}
var A: [1..5, 1..5] real;
var ij: 2*int = (1, 1);
\end{chapel}
the elements of array A can be indexed using any of the following
idioms:
\begin{chapel}
A(ij) = 1.1;
A((1, 2)) = 1.2;
A(1, 3) = 1.3;
A[ij] = -1.1;
A[(1, 4)] = 1.4;
A[1, 5] = 1.5;
\end{chapel}
\begin{chapelpost}
writeln(ij);
writeln(A);
\end{chapelpost}
\begin{chapeloutput}
(1, 1)
-1.1 1.2 1.3 1.4 1.5
0.0 0.0 0.0 0.0 0.0
0.0 0.0 0.0 0.0 0.0
0.0 0.0 0.0 0.0 0.0
0.0 0.0 0.0 0.0 0.0
\end{chapeloutput}
\end{chapelexample}

\begin{chapelexample}{index-using-var-arg-tuple.chpl}
The code
\begin{chapel}
proc f(A: [], is...)
  return A(is);
\end{chapel}
\begin{chapelpost}
var B: [1..5] int;
[i in 1..5] B(i) = i;
var C: [1..5,1..5] int;
[(i,j) in {1..5,1..5}] C(i,j) = i+i*j;
writeln(f(B, 3));
writeln(f(C, 3, 3));
\end{chapelpost}
\begin{chapeloutput}
3
12
\end{chapeloutput}
defines a function that takes an array as the first argument and a
variable-length argument list.  It then indexes into the array using
the tuple that captures the actual arguments.  This function works
even for one-dimensional arrays because one-dimensional arrays can be
indexed into by 1-tuples.
\end{chapelexample}

\subsection{Associative Array Indexing}
\label{Associative_Array_Indexing}
\index{indexing!associative arrays}
\index{associative arrays!indexing}

Indices can be added to associative arrays in two different ways.

The first way is through the array's domain.
\begin{chapelexample}{assoc-add-index.chpl}
Given:
\begin{chapel}
var D : domain(string);
var A : [D] int;
\end{chapel}

the array A initially contains no elements. We can change that by adding
indices to the domain D:
\begin{chapel}
D.add("a");
D.add("b");
\end{chapel}

The array A can now be indexed with indices "a" and "b":

\begin{chapel}
A["a"] = 1;
A["b"] = 2;
var x = A["a"];
\end{chapel}
\end{chapelexample}

The second way is more concise, and has the same effect as the first method:
\begin{chapelexample}{assoc-add-index-2.chpl}
\begin{chapel}
var D : domain(string);
var A : [D] int;
\end{chapel}
For other array types, assigning to an index not in the array's domain
would incur an out-of-bounds error. For associative arrays such assignment will
add the index to the array's domain, and the array can be indexed with the
newly added indices:
\begin{chapel}
A["a"] = 1;
A["b"] = 2;
var x = A["a"];
\end{chapel}
Here, the indices "a" and "b" are implicitly added to the domain D. Reading from
an index not in the array is still an out-of-bounds error.

\begin{chapel}
// writeln(A["c"]); // halts if "c" is not in A's domain
\end{chapel}
An important restriction for this method is that A may not share its domain
with another array. This restriction exists because it may be surprising to
seemingly modify one array, and to then see a change in another array. This
restriction is checked at runtime.

Note that extending an associative array via indexing in this way is
not parallel-safe.  Thus, if multiple tasks are adding elements to a
single array via indexing simultaneously, the responsibility of
synchronizing between the tasks to avoid races is the user's.
\end{chapelexample}


\section{Iteration over Arrays}
\label{Iteration_over_Arrays}
\index{arrays!iteration}
\index{iteration!array}

% FYI: Similar to text regarding tuple iteration.  Slightly less
% similar for domain iteration.
All arrays support iteration via standard \chpl{for}, \chpl{forall}
and \chpl{coforall} loops.  These loops iterate over all of the array
elements as described by its domain.  A loop of the form:

% This is difficult to capture in a test program
\begin{chapel}
[for|forall|coforall] a in A do
  ...a...
\end{chapel}

is semantically equivalent to:

% This is difficult to capture in a test program
\begin{chapel}
[for|forall|coforall] i in A.domain do
  ...A[i]...
\end{chapel}

The iterator variable for an array iteration is a reference to the
array element type.


\pagebreak
\section{Array Assignment}
\label{Array_Assignment}
\index{arrays!assignment}
\index{assignment!array}

Array assignment is by value.  Arrays can be assigned arrays, ranges,
domains, iterators, or tuples.

\begin{chapelexample}{assign.chpl}
If \chpl{A} is an lvalue of array type and \chpl{B} is an expression
of either array, range, or domain type, or an iterator, then the
assignment
\begin{chapelpre}
var A: [1..3] int;
var B: [1..3] int;
A = -1;
B = 1;
\end{chapelpre}
\begin{chapelnoprint}
writeln(A);
writeln(B);
\end{chapelnoprint}
\begin{chapel}
A = B;
\end{chapel}
\begin{chapelnoprint}
writeln(A);
writeln(B);
A = -2;
B = 2;
writeln(A);
writeln(B);
\end{chapelnoprint}
is equivalent to
\begin{chapel}
forall (a,b) in zip(A,B) do
  a = b;
\end{chapel}
\begin{chapelpost}
writeln(A);
writeln(B);
\end{chapelpost}
\begin{chapeloutput}
-1 -1 -1
1 1 1
1 1 1
1 1 1
-2 -2 -2
2 2 2
2 2 2
2 2 2
\end{chapeloutput}
If the zipper iteration is illegal, then the assignment is illegal.
Notice that the assignment is implemented with the semantics of
a \chpl{forall} loop.
\end{chapelexample}

Arrays can be assigned tuples of values of their element type if the
tuple contains the same number of elements as the array.  For
multidimensional arrays, the tuple must be a nested tuple such that
the nesting depth is equal to the rank of the array and the shape of
this nested tuple must match the shape of the array.  The values are
assigned element-wise.

% Is the above true for unordered array types?  Should it be?

Arrays can also be assigned single values of their element type.  In
this case, each element in the array is assigned this value.
\begin{chapelexample}{assign-2.chpl}
If \chpl{e} is an expression of the element type of the array or a
type that can be implicitly converted to the element type of the
array, then the assignment
\begin{chapelpre}
var A: [1..4] uint;
writeln(A);
var e: uint = 77;
\end{chapelpre}
\begin{chapel}
A = e;
\end{chapel}
\begin{chapelnoprint}
writeln(A);
e = 33;
\end{chapelnoprint}
is equivalent to
\begin{chapel}
forall a in A do
  a = e;
\end{chapel}
\begin{chapelpost}
writeln(A);
\end{chapelpost}
\begin{chapeloutput}
0 0 0 0
77 77 77 77
33 33 33 33
\end{chapeloutput}
\end{chapelexample}

\section{Array Slicing}
\label{Array_Slicing}
\index{arrays!slicing}
\index{slicing!array}

An array can be sliced using a domain that has the same type as the
domain over which it was declared.  The result of an array slice is an
alias to the subset of the array elements from the original array
corresponding to the slicing domain's index set.
 
\begin{chapelexample}{slicing.chpl}
Given the definitions
\begin{chapelpre}
config const n = 2;
\end{chapelpre}
\begin{chapel}
var OuterD: domain(2) = {0..n+1, 0..n+1};
var InnerD: domain(2) = {1..n, 1..n};
var A, B: [OuterD] real;
\end{chapel}
\begin{chapelnoprint}
writeln(OuterD);
writeln(InnerD);
B = 1;
\end{chapelnoprint}
the assignment given by
\begin{chapel}
A[InnerD] = B[InnerD];
\end{chapel}
\begin{chapelpost}
writeln(A);
writeln(B);
\end{chapelpost}
\begin{chapeloutput}
{0..3, 0..3}
{1..2, 1..2}
0.0 0.0 0.0 0.0
0.0 1.0 1.0 0.0
0.0 1.0 1.0 0.0
0.0 0.0 0.0 0.0
1.0 1.0 1.0 1.0
1.0 1.0 1.0 1.0
1.0 1.0 1.0 1.0
1.0 1.0 1.0 1.0
\end{chapeloutput}
assigns the elements in the interior of \chpl{B} to the elements in
the interior of \chpl{A}.
\end{chapelexample}

\subsection{Rectangular Array Slicing}
\label{Rectangular_Array_Slicing}
\index{arrays!slicing!rectangular}
\index{slicing!arrays!rectangular}

A rectangular array can be sliced by any rectangular domain that is a
subdomain of the array's defining domain.  If the subdomain
relationship is not met, an out-of-bounds error will occur.  The
result is a subarray whose indices are those of the slicing domain and
whose elements are an alias of the original array's.

Rectangular arrays also support slicing by ranges directly.  If each
dimension is indexed by a range, this is equivalent to slicing the
array by the rectangular domain defined by those ranges.  These
range-based slices may also be expressed using partially unbounded or
completely unbounded ranges.  This is equivalent to slicing the
array's defining domain by the specified ranges to create a subdomain
as described in~\rsec{Array_Slicing} and then using that subdomain to slice
the array.

\subsection{Rectangular Array Slicing with a Rank Change}
\label{Rectangular_Array_Slicing_With_Rank_Change}
\index{arrays!slicing!rectangular!rank change}

For multidimensional rectangular arrays, slicing with a rank change is
supported by substituting integral values within a dimension's range
for an actual range.  The resulting array will have a rank less than
the rectangular array's rank and equal to the number of ranges that are
passed in to take the slice.

\begin{chapelexample}{array-decl.chpl}
Given an array
\begin{chapelpre}
config const n = 4;
\end{chapelpre}
\begin{chapel}
var A: [1..n, 1..n] int;
\end{chapel}
\begin{chapelpost}
writeln(A);
\end{chapelpost}
\begin{chapeloutput}
0 0 0 0
0 0 0 0
0 0 0 0
0 0 0 0
\end{chapeloutput}
the slice \chpl{A[1..n, 1]} is a one-dimensional array whose elements
are the first column of \chpl{A}.
\end{chapelexample}


\section{Count Operator}
\label{Count_Operator_Arrays}
\index{arrays!count operator}
\index{operators!# (on arrays)@\chpl{#} (on arrays)}
The \chpl{#} operator can be applied to dense rectangular arrays with
a tuple argument whose size matches the rank of the array (or
optionally an integer in the case of a 1D array).  The operator is
equivalent to applying the \chpl{#} operator to the array's domain and
using the result to slice the array as described in
Section~\ref{Rectangular_Array_Slicing}.


\section{Array Arguments to Functions}
\label{Array_Arguments_To_Functions}
\index{arrays!actual arguments}
\index{arguments!array}

By default, arrays are passed to function by \chpl{ref} or \chpl{const
ref} depending on whether or not the formal argument is modified. The
\chpl{in}, \chpl{inout}, and \chpl{out} intent can create copies of
arrays.

When a formal argument has array type, the element type of the array
can be omitted and/or the domain of the array can be queried or
omitted.  In such cases, the argument is generic and is discussed
in~\rsec{Formal_Arguments_of_Generic_Array_Types}.

If a formal array argument specifies a domain as part of its type
signature, the domain of the actual argument must represent the same
index set.  If the formal array's domain was declared using an
explicit domain map, the actual array's domain must use an equivalent
domain map.

\section{Returning Arrays from Functions}
\label{Returning_Arrays_from_Functions}
\index{arrays!returning}
\index{returning!array}

Arrays return by value by default. The \chpl{ref} and \chpl{const ref}
return intents can be used to return a reference to an array.

\subsection{Array Promotion of Scalar Functions}
\label{Array_Promotion_of_Scalar_Functions}
\index{arrays!promotion}
\index{promotion!arrays}

Array promotion of a scalar function is defined over the array type
and the element type of the array.  The domain of the returned array,
if an array is captured by the promotion, is the domain of the array
that promoted the function.  In the event of zipper promotion over
multiple arrays, the promoted function returns an array with a domain
that is equal to the domain of the first argument to the function that
enables promotion.  If the first argument is an iterator or a range,
the result is a one-based one-dimensional array.

See also~\rsec{Promotion}.

\begin{chapelexample}{whole-array-ops.chpl}
Whole array operations is a special case of array promotion of scalar
functions.  In the code
\begin{chapelpre}
var A, B, C: [1..3] real;
A = -1;
B = 2;
C = 3;
\end{chapelpre}
\begin{chapel}
A = B + C;
\end{chapel}
\begin{chapelpost}
writeln(A);
\end{chapelpost}
\begin{chapeloutput}
5.0 5.0 5.0
\end{chapeloutput}
if \chpl{A}, \chpl{B}, and \chpl{C} are arrays, this code assigns each
element in \chpl{A} the element-wise sum of the elements in \chpl{B}
and \chpl{C}.
\end{chapelexample}

%
% TODO: should have an example of promoting an actual function here
%


\section{Sparse Arrays}
\label{Sparse_Arrays}
\index{arrays!sparse}

Sparse arrays in Chapel are those whose domain is sparse.  A
sparse array differs from other array types in that it stores a single
value corresponding to multiple indices.  This value is commonly
referred to as the \emph{zero value}, but we refer to it as the
\emph{implicitly replicated value} or \emph{IRV} since it can take
on any value of the array's element type in practice including
non-zero numeric values, a class reference, a record or tuple value,
etc.

An array declared over a sparse domain can be indexed using any of the
indices in the sparse domain's parent domain.  If it is read using an
index that is not part of the sparse domain's index set, the IRV value
is returned.  Otherwise, the array element corresponding to the index
is returned.

Sparse arrays can only be written at locations corresponding to
indices in their domain's index set.  In general, writing to other
locations corresponding to the IRV value will result in a runtime
error.

By default a sparse array's IRV is defined as the default value for
the array's element type.  The IRV can be set to any value of the
array's element type by assigning to a pseudo-field named \chpl{IRV}
in the array.

\begin{chapelexample}{sparse-error.chpl}
The following code example declares a sparse array, \chpl{SpsA} using
the sparse domain \chpl{SpsD} (For this example, assume that
\chpl{n}$>$1).  Line~2 assigns two indices to \chpl{SpsD}'s index set
and then lines 3--4 store the values 1.1 and 9.9 to the corresponding
values of \chpl{SpsA}.  The IRV of \chpl{SpsA} will initially be 0.0
since its element type is \chpl{real}.  However, the fifth line sets
the IRV to be the value 5.5, causing \chpl{SpsA} to represent the
value 1.1 in its low corner, 9.9 in its high corner, and 5.5
everywhere else.  The final statement is an error since it attempts to
assign to \chpl{SpsA} at an index not described by its domain,
\chpl{SpsD}.

\begin{chapelpre}
config const n = 5;
const D = {1..n, 1..n};
\end{chapelpre}
\begin{chapel}
var SpsD: sparse subdomain(D);
var SpsA: [SpsD] real;
SpsD = ((1,1), (n,n));
SpsA(1,1) = 1.1;
SpsA(n,n) = 9.9;
SpsA.IRV = 5.5;
SpsA(1,n) = 0.0;  // ERROR!
\end{chapel}
\begin{chapeloutput}
sparse-error.chpl:9: error: halt reached - attempting to assign a 'zero' value in a sparse array: (1, 5)
\end{chapeloutput}
\end{chapelexample}



\section{Association of Arrays to Domains}
\label{Association_of_Arrays_to_Domains}
\index{domains!association with arrays}
\index{arrays!association with domains}

%
% Be sure to talk about resetting array values & assigning IRVs
%

When an array is declared, it is linked during execution to the domain
identity over which it was declared.  This linkage is invariant for
the array's lifetime and cannot be changed.

When indices are added or removed from a domain, the change impacts
the arrays declared over this particular domain.  In the case of
adding an index, an element is added to the array and initialized to
the IRV for sparse arrays, and to the default value for the element
type for dense arrays.  In the case of removing an index, the element
in the array is removed.

When a domain is reassigned a new value, its arrays are also impacted.
Values that correspond to indices in the intersection of the old and
new domain are preserved in the arrays.  Values that could only be
indexed by the old domain are lost.  Values that can only be indexed
by the new domain have elements added to the new array, initialized to
the IRV for sparse arrays, and to the element type's default value for
other array types.

For performance reasons, there is an expectation that a method will be
added to domains to allow non-preserving assignment, \emph{i.e.}, all
values in the arrays associated with the assigned domain will be lost.
Today this can be achieved by assigning the array's domain an empty
index set (causing all array elements to be deallocated) and then
re-assigning the new index set to the domain.

An array's domain can only be modified directly, via the domain's name
or an alias created by passing it to a function via default intent.  In
particular, the domain may not be modified via the array's
\chpl{.domain} method, nor by using the domain query syntax on a
function's formal array
argument~(\rsec{Formal_Arguments_of_Generic_Array_Types}).

\begin{rationale}
When multiple arrays are declared using a single domain, modifying the
domain affects all of the arrays.  Allowing an array's domain to be
queried and then modified suggests that the change should only affect
that array.  By requiring the domain to be modified directly, the user
is encouraged to think in terms of the domain distinctly from a
particular array.

In addition, this choice has the beneficial effect that arrays
declared via an anonymous domain have a constant domain.  Constant
domains are considered a common case and have potential compilation
benefits such as eliminating bounds checks.  Therefore making this
convenient syntax support a common, optimizable case seems prudent.
\end{rationale}


\section{Predefined Functions and Methods on Arrays}
\label{Predefined_Functions_and_Methods_on_Arrays}
\index{arrays!predefined functions}
\index{predefined functions!arrays}
\index{functions!arrays!predefined}

There is an expectation that this list of predefined methods will grow.

\index{arrays!eltType@\chpl{eltType}}
\index{predefined functions!eltType (array)@\chpl{eltType} (array)}
\begin{protohead}
proc $Array$.eltType type
\end{protohead}
\begin{protobody}
Returns the element type of the array.
\end{protobody}

\index{arrays!rank@\chpl{rank}}
\index{predefined functions!rank (array)@\chpl{rank} (array)}
\begin{protohead}
proc $Array$.rank param
\end{protohead}
\begin{protobody}
Returns the rank of the array.
\end{protobody}

\index{arrays!domain@\chpl{domain}}
\index{predefined functions!domain (array)@\chpl{domain} (array)}
\begin{protohead}
proc $Array$.domain: this.domain
\end{protohead}
\begin{protobody}
Returns the domain of the given array.  This domain is constant,
implying that the domain cannot be resized by assigning to its domain
field, only by modifying the domain directly.
\end{protobody}

\index{arrays!numElements@\chpl{numElements}}
\index{predefined functions!numElements (array)@\chpl{numElements} (array)}
\begin{protohead}
proc $Array$.numElements: this.domain.dim_type
\end{protohead}
\begin{protobody}
Returns the number of elements in the array.
\end{protobody}

\index{arrays!reshape@\chpl{reshape}}
\index{predefined functions!reshape (array)@\chpl{reshape} (array)}
\begin{protohead}
proc reshape(A: $Array$, D: $Domain$): $Array$
\end{protohead}
\begin{protobody}
Returns a copy of the array containing the same values but in the
shape of the new domain.  The number of indices in the domain must
equal the number of elements in the array.  The elements of the array
are copied into the new array using the default iteration orders over
both arrays.
\end{protobody}

\index{arrays!size@\chpl{size}}
\index{predefined functions!size (array)@\chpl{size} (array)}
\begin{protohead}
proc $Array$.size: this.domain.dim_type
\end{protohead}
\begin{protobody}
Same as $Array$.numElements.
\end{protobody}

\cleardoublepage
\sekshun{Iterators}
\label{Iterators}
\index{iterators}

An iterator is a function that conceptually returns multiple values
rather than simply a single value.

\begin{openissue}
The parallel iterator story is under development.  It is expected that
the specification will be expanded regarding parallel iterators soon.
\end{openissue}

\subsection{Iterator Function Definitions}
\label{Iterator_Function_Definitions}
\index{iterator function definitions}

The syntax to declare an iterator function (or simply, ``iterator''), is given
by:
\begin{syntax}
iterator-declaration-statement:
  `iter' iterator-name argument-list[OPT] var-param-type-clause[OPT] where-clause[OPT]
  iterator-body

iterator-name:
  identifier

iterator-body:
  block-statement
  yield-statement
\end{syntax}

The syntax of an iterator declaration is similar to a function declaration, with
some key differences:
\begin{itemize}
\item The keyword \chpl{iter} is used instead of the keyword \chpl{proc}.
\item The name of the iterator cannot overload any operator.
\item \chpl{yield} statements may appear in the body of an iterator, but not in
a regular function.
\end{itemize}

\subsection{The Yield Statement}
\label{The_Yield_Statement}
\index{yield@\chpl{yield}}

The yield statement can only appear in iterators.  The syntax of the
yield statement is given by
\begin{syntax}
yield-statement:
  `yield' expression ;
\end{syntax}

When an iterator is executed and a \chpl{yield} is encountered, the value of the yield
expression is returned.  However, the state of execution of the iterator is
saved.  On its next invocation, execution resumes from the point immediately
following that \chpl{yield} statement.

When a \chpl{return} is encountered, the iterator finishes without yielding another
index value.  The \chpl{return} statements appearing in an iterator are not
permitted to have a return value.
An iterator also completes after the last
statement in the iterator function is executed.
An iterator need not contain any yield statements.

\subsection{Iterator Calls}
\label{Iterator_Calls}

The syntax used to call an interator is given by:
\begin{syntax}
iterator-call-expression:
  call-expression
\end{syntax}
This is identical to the function-call syntax.
%REVIEW: hilde
% Can iterator definitions and uses have field syntax?

All details of the \sntx{iterator-call-expression} semantics --- including
resolution, the use of parentheses versus brackets to delimit the parameter
list, calling the iterator without an argument list and named arguments ---
are identical with the corresponding semantics for function calls.
See~\rsec{Function_Calls}.

However, the result of an iterator call depends upon its context, as described below.

\subsubsection{Iterators in For and Forall Loops}
\label{Iterators_in_For_and_Forall_Loops}

When an iterator is accessed via a for or forall loop, the iterator is
evaluated alongside the loop body in an interleaved manner.  For each
iteration, the iterator yields a value and the body is executed.

\subsubsection{Iterators as Arrays}
\label{Iterators_as_Arrays}
\index{iterators!and arrays}

If an iterator function is captured into a new variable declaration or
assigned to an array, the iterator is iterated over in total and the
expression evaluates to a one-dimensional arithmetic array that
contains the values returned by the iterator on each iteration.
\begin{example}
Given an iterator
\begin{chapel}
iter squares(n: int): int {
  for i in 1..n do
    yield i * i;
}
\end{chapel}
\begin{chapelpost}
writeln(squares(5));
\end{chapelpost}
\begin{chapeloutput}
1 4 9 16 25
\end{chapeloutput}
the expression \chpl{squares(5)} evaluates to the array \chpl{1, 4, 9, 16, 25}.
\end{example}

\subsubsection{Iterators and Generics}
\label{Iterators_and_Generics}
\index{iterators!and generics}

An iterator call expression can be passed to a generic function argument that
has neither a declared type nor default value
(\rsec{Formal_Arguments_without_Types}).
In this case the iterator is passed without being evaluated.
Within the generic function the corresponding formal argument
can be used as an iterator, e.g. in for loops.
The arguments to the iterator call expression, if any, are evaluated
at the call site, i.e. prior to passing the iterator to the generic function.

\subsubsection{Recursive Iterators}
\label{Recursive_Iterators}
\index{iterators!recursive}

Recursive iterators are allowed. A recursive iterator invocation is
typically made by iterating over it in a loop.


\begin{example}
A post-order traversal of a tree data structure could be written like this:
\begin{chapel}
iter postorder(tree: Tree): string {
  if tree != nil {
    for child in postorder(tree.left) do
      yield child;
    for child in postorder(tree.right) do
      yield child;
    yield tree.data;
  }
}
\end{chapel}
By contrast, using calls \chpl{postorder(tree.left)}
and \chpl{postorder(tree.right)} as stand-alone statements would
result in generating temporary arrays containing the outcomes of these
recursive calls, which would then be discarded.
\end{example}

\subsection{Parallel Iterators}
\label{Parallel_Iterators}

Iterators used in explicit forall-statements or -expressions must be
parallel iterators.  Reductions, scans and promotion over serial
iterators will be serialized.

The definition of parallel iterators is forthcoming.  Parallel
iterators are defined over standard constructs in Chapel such as
ranges, domains, and arrays (including Block- and Cyclic-distributed
domains and arrays), thereby allowing these constructs to be used with
forall-statements and -expressions.

\cleardoublepage
\sekshun{Generics}
\label{Generics}

Chapel supports generic functions and types that are parameterizable
over both types and parameters.  The generic functions and types look
similar to non-generic functions and types already discussed.

\section{Generic Functions}
\label{Generic_Functions}
\index{functions!generic}
\index{generics!functions}

A function is generic if any of the following conditions hold:
\begin{itemize}
\item
Some formal argument is specified with an intent of \chpl{type} or
\chpl{param}.
\item
Some formal argument has no specified type and no default value.
\item
Some formal argument is specified with a queried type.
\item
The type of some formal argument is a generic type, e.g., \chpl{List}.
Queries may be inlined in generic types, e.g., \chpl{List(?eltType)}.
\item
The type of some formal argument is an array type where either the
element type is queried or omitted or the domain is queried or
omitted.
\end{itemize}
These conditions are discussed in the next sections.

\subsection{Formal Type Arguments}
\label{Formal_Type_Arguments}
\index{intents!type@\chpl{type}}

If a formal argument is specified with intent \chpl{type}, then a type
must be passed to the function at the call site.  A copy of the
function is instantiated for each unique type that is passed to this
function at a call site.  The formal argument has the semantics of a
type alias.
\begin{example}
The following code defines a function that takes two types at the call
site and returns a 2-tuple where the types of the components of the
tuple are defined by the two type arguments and the values are
specified by the types default values.
\begin{chapelpre}
% build2tuple.chpl
\end{chapelpre}
\begin{chapel}
proc build2Tuple(type t, type tt) {
  var x1: t;
  var x2: tt;
  return (x1, x2);
}
\end{chapel}
This function is instantiated with ``normal'' function call syntax
where the arguments are types:
\begin{chapel}
var t2 = build2Tuple(int, string);
t2 = (1, "hello");
\end{chapel}
\begin{chapelpost}
writeln(t2);
\end{chapelpost}
\begin{chapeloutput}
(1, hello)
\end{chapeloutput}
\end{example}

\subsection{Formal Parameter Arguments}
\label{Formal_Parameter_Arguments}
\index{intents!param@\chpl{param}}

If a formal argument is specified with intent \chpl{param}, then a
parameter must be passed to the function at the call site.  A copy of
the function is instantiated for each unique parameter that is passed
to this function at a call site.  The formal argument is a parameter.
\begin{example}
The following code defines a function that takes an integer parameter
\chpl{p} at the call site as well as a regular actual argument of
integer type \chpl{x}.  The function returns a homogeneous tuple of
size \chpl{p} where each component in the tuple has the value of
\chpl{x}.
\begin{chapelpre}
% fillTuple.chpl
\end{chapelpre}
\begin{chapel}
proc fillTuple(param p: int, x: int) {
  var result: p*int;
  for param i in 1..p do
    result(i) = x;
  return result;
}
\end{chapel}
\begin{chapelpost}
writeln(fillTuple(3,3));
\end{chapelpost}
\begin{chapeloutput}
(3, 3, 3)
\end{chapeloutput}
The function call \chpl{fillTuple(3, 3)} returns a 3-tuple where each
component contains the value \chpl{3}.
\end{example}

\subsection{Formal Arguments without Types}
\label{Formal_Arguments_without_Types}
\index{formal arguments!without types}

If the type of a formal argument is omitted, the type of the formal
argument is taken to be the type of the actual argument passed to the
function at the call site.  A copy of the function is instantiated for
each unique actual type.
\begin{example}
The example from the previous section can be extended to be generic on
a parameter as well as the actual argument that is passed to it by
omitting the type of the formal argument \chpl{x}.  The following code
defines a function that returns a homogeneous tuple of size \chpl{p}
where each component in the tuple is initialized to \chpl{x}:
\begin{chapelpre}
% fillTuple2.chpl
\end{chapelpre}
\begin{chapel}
proc fillTuple(param p: int, x) {
  var result: p*x.type;
  for param i in 1..p do
    result(i) = x;
  return result;
}
\end{chapel}
\begin{chapelpost}
var x = fillTuple(3, 3.14);
writeln(x);
writeln(typeToString(x.type));
\end{chapelpost}
\begin{chapeloutput}
(3.14, 3.14, 3.14)
3*real
\end{chapeloutput}
In this function, the type of the tuple is taken to be the type of the
actual argument.  The call \chpl{fillTuple(3, 3.14)} returns a 3-tuple
of real values \chpl{(3.14, 3.14, 3.14)}.  The return type is
\chpl{(real, real, real)}.
\end{example}

\subsection{Formal Arguments with Queried Types}
\label{Formal_Arguments_with_Queried_Types}
\index{formal arguments!queried types}

If the type of a formal argument is specified as a queried type, the
type of the formal argument is taken to be the type of the actual
argument passed to the function at the call site.  A copy of the
function is instantiated for each unique actual type.  The queried
type has the semantics of a type alias.
\begin{example}
The example from the previous section can be rewritten to use a
queried type for clarity:
\begin{chapelpre}
% fillTuple3.chpl
\end{chapelpre}
\begin{chapel}
proc fillTuple(param p: int, x: ?t) {
  var result: p*t;
  for param i in 1..p do
    result(i) = x;
  return result;
}
\end{chapel}
\begin{chapelpost}
var x = fillTuple(3, 3.14);
writeln(x);
writeln(typeToString(x.type));
\end{chapelpost}
\begin{chapeloutput}
(3.14, 3.14, 3.14)
3*real
\end{chapeloutput}
\end{example}

\subsection{Formal Arguments of Generic Type}
\label{Formal_Arguments_of_Generic_Type}
\index{formal arguments!generic types}

If the type of a formal argument is a generic type, the type of the
formal argument is taken to be the type of the actual argument passed
to the function at the call site with the constraint that the type of
the actual argument is an instantiation of the generic type.  A copy
of the function is instantiated for each unique actual type.
\begin{example}
The following code defines a function \chpl{writeTop} that takes an
actual argument that is a generic stack
(see~\rsec{Example_Generic_Stack}) and outputs the top element of the
stack.  The function is generic on the type of its argument.
\begin{chapel}
proc writeTop(s: Stack) {
  write(s.top.item);
}
\end{chapel}
\end{example}

Types and parameters may be queried from the top-level types of formal
arguments as well.  In the example above, the formal argument's type
could also be specified as \chpl{Stack(?type)} in which case the
symbol \chpl{type} is equivalent to \chpl{s.itemType}.

Note that generic types which have default values for all of their
generic fields, \emph{e.g. range}, are not generic when simply
specified and require a query to mark the argument as generic.  For
simplicity, the identifier may be omitted.
\begin{example}
The following code defines a class with a type field that has a
default value.  Function \chpl{f} is defined to take an argument of
this class type where the type field is instantiated to the default.
Function \chpl{g}, on the other hand, is generic on its argument
because of the use of the question mark.
\begin{chapel}
class C {
  type t = int;
}
proc f(c: C) {
  // c.type is always int
}
proc g(c: C(?)) {
  // c.type may not be int
}
\end{chapel}
\end{example}

\index{where@\chpl{where}!implicit}
The generic type may be specified with some queries and some exact
values.  Thesse exact values result in \emph{implicit where clauses}
for the purpose of function resolution.
\begin{example}
Given the class definition
\begin{chapel}
class C {
  type t;
  type tt;
}
\end{chapel}
then the function definition
\begin{chapel}
proc f(c: C(?t,real)) {
  // body
}
\end{chapel}
is equivalent to
\begin{chapel}
proc f(c: C(?t,?tt)) where tt == real {
  // body
}
\end{chapel}
\end{example}
For tuples with query arguments, an implicit where clause is always
created to ensure that the size of the actual tuple matches the
implicitly specified size of the formal tuple.
\begin{example}
The function definition
\begin{chapel}
proc f(tuple: (?t,real)) {
  // body
}
\end{chapel}
is equivalent to
\begin{chapel}
proc f(tuple: (?t,?tt)) where tuple.size == 2 && tt == real {
  // body
}
\end{chapel}
\end{example}

\index{integral@\chpl{integral}}
\index{numeric@\chpl{numeric}}
\index{enumerated@\chpl{enumerated}}
The generic types \chpl{integral}, \chpl{numeric} and \chpl{enumerated}
are generic types that can only be instantiated with, respectively, the
signed and unsigned integral types, all of the numeric types, and
enumerated types.

\subsection{Formal Arguments of Generic Array Types}
\label{Formal_Arguments_of_Generic_Array_Types}
\index{formal arguments!array types}

If the type of a formal argument is an array where either the domain
or the element type is queried or omitted, the type of the formal
argument is taken to be the type of the actual argument passed to the
function at the call site.  If the domain is omitted, the domain of
the formal argument is taken to be the domain of the actual argument.

A queried domain may not be modified via the name to which it is bound
(see~\rsec{Association_of_Arrays_to_Domains} for rationale).

\section{Function Visibility in Generic Functions}
\label{Function_Visibility_in_Generic_Functions}
\index{generics!function visibility}

Function visibility in generic functions is altered depending on the
instantiation.  When resolving function calls made within generic
functions, the visible functions are taken from any call site at which
the generic function is instantiated for each particular
instantiation.  The specific call site chosen is arbitrary and it is
referred to as the \emph{point of instantiation}.

For function calls that specify the module
explicitly~(\rsec{Explicit_Naming}), an implicit use of the specified
module exists at the call site.

\begin{example}
Consider the following code which defines a generic
function \chpl{bar}:
\begin{chapelpre}
% point_of_instantiation.chpl
\end{chapelpre}
\begin{chapel}
module M1 {
  record R {
    var x: int;
    proc foo() { }
  }
}

module M2 {
  proc bar(x) {
    x.foo();
  }
}

module M3 {
  use M1, M2;
  proc main() {
    var r: R;
    bar(r);
  }
}
\end{chapel}
\begin{chapeloutput}
\end{chapeloutput}
In the function \chpl{main}, the variable \chpl{r} is declared to be
of type \chpl{R} defined in module \chpl{M1} and a call is made to the
generic function \chpl{bar} which is defined in module \chpl{M2}.
This is the only place where \chpl{bar} is called in this program and
so it becomes the point of instantiation for \chpl{bar} when the
argument \chpl{x} is of type \chpl{R}.  Therefore, the call to
the \chpl{foo} method in \chpl{bar} is resolved by looking for visible
functions from within \chpl{main} and going through the use of
module \chpl{M1}.
\end{example}

If the generic function is only called indirectly through dynamic
dispatch, the point of instantiation is defined as the point at which
the derived type (the type of the implicit \chpl{this} argument) is
defined or instantiated (if the derived type is generic).

\begin{rationale}
Visible function lookup in Chapel's generic functions is handled
differently than in C++'s template functions in that there is no split
between dependent and independent types.

Also, dynamic dispatch and instantiation is handled differently.
Chapel supports dynamic dispatch over methods that are generic in some
of its formal arguments.

Note that the Chapel lookup mechanism is still under development and
discussion.  Comments or questions are appreciated.
\end{rationale}

\section{Generic Types}
\label{Generic_Types}
\index{generics!types}
\index{types!generic}
\index{generics!classes}
\index{classes!generic}
\index{generics!records}
\index{records!generic}

Generic types are generic classes and generic records.
A class or record is generic if it contains one or more
\index{generics!fields}
\index{fields!generic}
generic fields. A generic field is one of:
\begin{itemize}
\item a specified or unspecified type alias,
\item a parameter field, or
\item a \chpl{var} or \chpl{const} field that has no type and no initialization
expression.
\end{itemize}

\mbox{} % push the following line to the next page

For each generic field, the class or record is parameterized over:
\begin{itemize}
\item the type bound to the type alias,
\item the value of the parameter field, or
\item the type of the \chpl{var} or \chpl{const} field, respectively.
\end{itemize}
Correspondingly, the class or record is instantiated with a set
of types and parameter values, one type or value per generic field.

% Here are the aspects to be defined for each kind of generic field:
% - what it makes the class/record generic over
% - the type constructor arg that gets created
% - the default constructor arg that gets created
% - the requirements on the corresponding user-defined constructor arg
% - for each of the above args:
%    - what kind of actual it accepts (type, param, value)
%    - what is the semantics;
%      i.e. how it corresponds to the class/record's genericity
%    - what is the arg's default, if any
% 
% In the presentation below, some of these aspects are discussed
% in the field-kind-specific subsections, some in the constructor-specific
% subsections, some in both.  I.e. there is an overlap between
% field-kind and constructor subsections; that should be OK but feel free
% to clean up.
% 
% It would be cool to summarize that in a table
% (one dimension: field kinds; the other dimension: aspects).

\subsection{Type Aliases in Generic Types}
\label{Type_Aliases_in_Generic_Types}
\index{type aliases!in classes or records}
\index{fields!type alias}

If a class or record defines a type alias, the class or record
is generic over the type that is bound to that alias.
% Type aliases defined in a class or a record can be unspecified type
% aliases; type aliases that are not bound to a type.  If a class or
% record contains an unspecified type alias, the aliased type must be
% specified whenever the type is used.
Such a type alias is accessed as if it were a field;
similar to a parameter field, it cannot be assigned
except in its declaration.

The type alias becomes an argument with intent \chpl{type} to
the compiler-generated constructor (\rsec{Generic_Compiler_Generated_Constructors})
for its class or record. This makes the compiler-generated constructor generic.
The type alias also becomes an argument with intent \chpl{type} to
the type constructor (\rsec{Type_Constructors}).
If the type alias declaration binds it to a type, that type
becomes the default for these arguments, otherwise they have no defaults.

The class or record is instantiated by binding the type alias
to the actual type passed to the corresponding argument of
a user-defined (\rsec{Generic_User_Constructors})
or compiler-generated constructor or type constructor.
If that argument has a default, the actual type can be omitted, in
which case the default will be used instead.

\begin{example}
The following code defines a class called \chpl{Node} that implements
a linked list data structure.  It is generic over the type of the
element contained in the linked list.
\begin{chapelpre}
% NodeClass.chpl
\end{chapelpre}
\begin{chapel}
class Node {
  type eltType;
  var data: eltType;
  var next: Node(eltType);
}
\end{chapel}
\begin{chapelpost}
var n: Node(real) = new Node(real, 3.14);
writeln(n.data);
writeln(n.next);
writeln(typeToString(n.next.type));
\end{chapelpost}
\begin{chapeloutput}
3.14
nil
Node(real)
\end{chapeloutput}
The call \chpl{new Node(real, 3.14)} creates a node in the linked list
that contains the value \chpl{3.14}.  The \chpl{next} field is set to
nil.  The type specifier \chpl{Node} is a generic type and cannot be
used to define a variable.  The type specifier \chpl{Node(real)}
denotes the type of the \chpl{Node} class instantiated over
\chpl{real}.  Note that the type of the \chpl{next} field is specified
as \chpl{Node(eltType)}; the type of \chpl{next} is the same type as
the type of the object that it is a field of.
\end{example}

\subsection{Parameters in Generic Types}
\label{Parameters_in_Generic_Types}
\index{parameters!in classes or records}
\index{fields!parameter}

If a class or record defines a parameter field, the class or record
is generic over the value that is bound to that field.
% A parameter defined in a class or record is accessed as if it were a
% field.  This access returns a parameter.  
The parameter becomes an argument with intent \chpl{param} to the
compiler-generated constructor (\rsec{Generic_Compiler_Generated_Constructors})
for that class or record.  This makes the compiler-generated
constructor generic.  The parameter also becomes an argument
with intent \chpl{param} to the type  constructor (\rsec{Type_Constructors}).
If the parameter declaration has an initialization expression, that expression
becomes the default for these arguments, otherwise they have no defaults.

The class or record is instantiated by binding the parameter
to the actual value passed to the corresponding argument of
a user-defined (\rsec{Generic_User_Constructors})
or compiler-generated constructor or type constructor.
If that argument has a default, the actual value can be omitted, in
which case the default will be used instead.

\begin{example}
The following code defines a class called \chpl{IntegerTuple} that is
generic over an integer parameter which defines the number of
components in the class.
\begin{chapelpre}
% IntegerTuple.chpl
\end{chapelpre}
\begin{chapel}
class IntegerTuple {
  param size: int;
  var data: size*int;
}
\end{chapel}
\begin{chapelpost}
var x = new IntegerTuple(3);
writeln(x.data);
\end{chapelpost}
\begin{chapeloutput}
(0, 0, 0)
\end{chapeloutput}
The call \chpl{new IntegerTuple(3)} creates an instance of the
\chpl{IntegerTuple} class that is instantiated over parameter
\chpl{3}.  The field \chpl{data} becomes a 3-tuple of integers.  The
type of this class instance is \chpl{IntegerTuple(3)}.  The type
specified by \chpl{IntegerTuple} is a generic type.
\end{example}

\subsection{Fields without Types}
\label{Fields_without_Types}
\index{fields!variable and constant, without types}
\index{variables!in classes or records}
\index{constants!in classes or records}

If a \chpl{var} or \chpl{const} field in a class or record has no specified type or
initialization expression, the class or record is generic over the
type of that field.  The field becomes an argument with blank intent to
the compiler-generated constructor (\rsec{Generic_Compiler_Generated_Constructors}).
That argument has no specified type and no default
value. This makes the compiler-generated constructor generic.
The field also becomes an argument with \chpl{type} intent and no default
to the type constructor (\rsec{Type_Constructors}).
Correspondingly, an actual value must always be passed to the default
constructor argument and an actual type to the type constructor argument.

The class or record is instantiated by binding the type of the field
to the type of the value passed to the corresonding argument
of a user-defined (\rsec{Generic_User_Constructors}) or compiler-generated constructor (\rsec{Generic_Compiler_Generated_Constructors}).
When the type constructor is invoked, the class or record is instantiated
by binding the type of the field to the actual type passed to
the corresponding argument.

\begin{example}
The following code defines another class called \chpl{Node} that
implements a linked list data structure.  It is generic over the type
of the element contained in the linked list.  This code does not
specify the element type directly in the class as a type alias but
rather omits the type from the \chpl{data} field.
\begin{chapelpre}
% fieldWithoutType.chpl
\end{chapelpre}
\begin{chapel}
class Node {
  var data;
  var next: Node(data.type) = nil;
}
\end{chapel}
A node with integer element type can be defined in the call to the
constructor.  The call \chpl{new Node(1)} defines a node with the
value \chpl{1}.  The code
\begin{chapel}
var list = new Node(1);
list.next = new Node(2);
\end{chapel}
\begin{chapelpost}
writeln(list.data);
writeln(list.next.data);
\end{chapelpost}
\begin{chapeloutput}
1
2
\end{chapeloutput}
defines a two-element list with nodes containing the values \chpl{1}
and \chpl{2}.  The type of each object could be specified
as \chpl{Node(int)}.
\end{example}

\subsection{The Type Constructor}
\label{Type_Constructors}
\index{generics!type constructor}
\index{constructors!type constructors}

A type constructor is automatically created for each class or record.
A type constructor is a type function (\rsec{Type_Functions}) that has
the same name as the class or record.  It takes one argument per the
class's or record's generic field, including fields inherited from the
superclasses, if any.
The formal argument has intent \chpl{type} for a type alias field and is a
parameter for a parameter field. It accepts the type to be bound
to the type alias and the value to be bound to the parameter, respectively.
For a generic \chpl{var} or \chpl{const} field, the corresponding
formal argument also has intent \chpl{type}. It accepts the type
of the field, as opposed to a value as is the case for a parameter field.
The formal arguments occur in the same order as the fields are
declared and have the same names as the corresponding fields.
Unlike the compiler-generated constructor, the type constructor has only
those arguments that correspond to generic fields.

A call to a type constructor accepts actual types and parameter values
and returns the type of the class or record that is instantiated
appropriately for each field
(\rsec{Type_Aliases_in_Generic_Types}, \rsec{Parameters_in_Generic_Types},
\rsec{Fields_without_Types}).
\index{generics!instantiated type}
Such an instantiated type must be used as the type of a variable,
array element, non-generic formal argument, and in other cases
where uninstantiated generic class or record types are not allowed.

When a generic field declaration has an initialization expression
or a type alias is specified, that initializer becomes the default value
for the corresponding type constructor argument.  Uninitialized
fields, including all generic \chpl{var} and \chpl{const} fields,
and unspecified type aliases result in arguments with no defaults;
actual types or values for these arguments must always be provided
when invoking the type constructor.

\subsection{Generic Methods}
\label{Generic_Methods}
\index{generics!methods}

All methods bound to generic classes or records, including
constructors, are generic over the implicit \chpl{this} argument.
This is in addition to being generic over any other argument that is generic.

\subsection{The Compiler-Generated Constructor}
\label{Generic_Compiler_Generated_Constructors}
\index{generics!constructors!compiler-generated}
\index{constructors!compiler-generated!for generic classes or records}

If no user-defined constructors are supplied for a given generic class, the
compiler generates one following in a manner similar to that for concrete
classes (\rsec{The_Compiler_Generated_Constructor}).
However, the compiler-generated constructor for a generic class or record
(\rsec{The_Compiler_Generated_Constructor}) is generic over each argument that
corresponds to a generic field, as specified above.
The argument has intent \chpl{type} for a type alias field and is a
parameter for a parameter field. It accepts the type to be bound
to the type alias and the value to be bound to the parameter, respectively.
This is the same as for the type constructor.
For a generic \chpl{var} or \chpl{const} field, the corresponding
formal argument has the blank intent and accepts the value
for the field to be initialized with. The type of the field
is inferred automatically to be the type of the initialization value.

The default values for the generic arguments of the compiler-generated constructor
are the same as for the type constructor (\rsec{Type_Constructors}).
For example, the arguments corresponding to the generic \chpl{var}
and \chpl{const} fields, if any, never have defaults, so the corresponding
actual values must always be provided.

\subsection{User-Defined Constructors}
\label{Generic_User_Constructors}
\index{generics!constructors!user-defined}
\index{constructors!user-defined!for generic classes or records}

If a generic field of a class does not have an initialization expression
or a type alias is unspecified, each user-defined constructor for that
class must provide a formal argument whose name
matches the name of the field.

If the name of a formal argument in a user-defined constructor matches the name
of a generic field that does not have an initialization
expression, is a type alias, or is a parameter field, the field is
automatically initialized at the beginning of the constructor invocation
to the actual value of that argument.
This is done by passing that formal argument to the implicit invocation
of the compiler-generated constructor during default-initialization (\rsec{Default_Initialization}).

%%  The following story is nicer but it's not how it is implemented:
%If the name of a formal argument in a class constructor
%matches the name of a generic field, the field is automatically initialized
%to the actual value for that argument upon the constructor invocation.
%If the generic field does not have an initialization expression,
%such a matching formal argument must be provided in each constructor
%for that class.

\begin{example}
In the following code:
\begin{chapelpre}
% constructorsForGenericFields.chpl
\end{chapelpre}
\begin{chapel}
class MyGenericClass {
  type t1;
  param p1;
  const c1;
  var v1;
  var x1: t1; // this field is not generic

  type t5 = real;
  param p5 = "a string";
  const c5 = 5.5;
  var v5 = 555;
  var x5: t5; // this field is not generic

  proc MyGenericClass(c1, v1, type t1, param p1) { }
  proc MyGenericClass(type t5, param p5, c5, v5, x5,
                     type t1, param p1, c1, v1, x1) { }
}  // class MyGenericClass

var g1 = new MyGenericClass(11, 111, int, 1);
var g2 = new MyGenericClass(int, "this is g2", 3.3, 333, 3333,
                            real, 2, 222, 222.2, 22);
\end{chapel}
\begin{chapelpost}
writeln(g1);
writeln(g2);
\end{chapelpost}
\begin{chapeloutput}
{p1 = 1, c1 = 11, v1 = 111, x1 = 0, p5 = a string, c5 = 5.5, v5 = 555, x5 = 0.0}
{p1 = 2, c1 = 222, v1 = 222.2, x1 = 0.0, p5 = this is g2, c5 = 5.5, v5 = 555, x5 = 0}
\end{chapeloutput}
The arguments \chpl{t1}, \chpl{p1}, \chpl{c1}, and \chpl{v1} are
required in all constructors for \chpl{MyGenericClass}. They can appear
in any order. Both \chpl{MyGenericClass} constructors initialize the
corresponding fields implicitly because these fields do not have initialization
expressions. The second constructor also initializes implicitly
the fields \chpl{t5} and \chpl{p5} because they are a type field
and a parameter field. It does not initialize the fields \chpl{c5}
and \chpl{v5} because they have initialization expressions, or
the fields \chpl{x1} and \chpl{x5} because they are not generic fields
(even though they are of generic types).
\end{example}

\begin{openissue}
The design of constructors, especially for generic classes, is
under development, so the above specification may change.
\end{openissue}

\section{Where Expressions}
\label{Where_Expressions}
\index{where@\chpl{where}}
\index{generics!where@\chpl{where}}

The instantiation of a generic function can be constrained by {\em
where clauses}.  A where clause is specified in the definition of a
function~(\rsec{Function_Definitions}).  When a function is
instantiated, the expression in the where clause must be a parameter
expression and must evaluate to either \chpl{true} or \chpl{false}.
If it evaluates to \chpl{false}, the instantiation is rejected and the
function is not a possible candidate for function resolution.
Otherwise, the function is instantiated.
\begin{example}
Given two overloaded function definitions
\begin{chapelpre}
% whereClause.chpl
\end{chapelpre}
\begin{chapel}
proc foo(x) where x.type == int { writeln("int"); }
proc foo(x) where x.type == real { writeln("real"); }
\end{chapel}
\begin{chapelpost}
foo(3);
foo(3.14);
\end{chapelpost}
\begin{chapeloutput}
int
real
\end{chapeloutput}
the call foo(3) resolves to the first definition because when the
second function is instantiated the where clause evaluates to false.
\end{example}

\section{User-Defined Compiler Diagnostics}
\label{User_Defined_Compiler_Errors}
\index{compiler diagnostics!user-defined}
\index{compiler errors!user-defined}
\index{compiler warnings!user-defined}
\index{compilerError}
\index{compilerWarning}

The special compiler diagnostic function calls \chpl{compilerError}
and \chpl{compilerWarning} generate compiler diagnostic of the
indicated severity if the function containing these calls may be
called when the program is executed and the function call is not
eliminated by parameter folding.

The compiler diagnostic is defined by the actual arguments which must
be string parameters.  The diagnostic points to the spot in the Chapel
program from which the function containing the call is called.
Compilation halts if a \chpl{compilerError} is encountered whereas it
will continue after encountering a \chpl{compilerWarning}.

Note that when a variable function is called in a context where the
implicit \chpl{setter} argument is true or false, both versions of the
variable function are resolved by the compiler.  Consequently,
the \chpl{setter} argument cannot be effectively used to guard a
compiler diagnostic statements.

\begin{example}
The following code shows an example of using user-defined compiler
diagnostics to generate warnings and errors:
\begin{chapelpre}
% compilerDiagnostics.chpl
\end{chapelpre}
\begin{chapel}
proc foo(x, y) {
  if (x.type != y.type) then
    compilerError("foo() called with non-matching types: ", 
                  typeToString(x.type), " != ", typeToString(y.type));
  writeln("In 2-argument foo...");
}

proc foo(x) {
  compilerWarning("1-argument version of foo called");
  writeln("In generic foo!");
}
\end{chapel}
\begin{chapelpost}
foo(3.4);
foo("hi");
foo(1, 2);
foo(1.2, 3.4);
foo("hi", "bye");
\end{chapelpost}
\begin{chapeloutput}
compilerDiagnostics.chpl:12: warning: 1-argument version of foo called
compilerDiagnostics.chpl:13: warning: 1-argument version of foo called
In generic foo!
In generic foo!
In 2-argument foo...
In 2-argument foo...
In 2-argument foo...
\end{chapeloutput}

The first routine generates a compiler error whenever the compiler
encounters a call to it where the two arguments have different types.
It prints out an error message indicating the types of the arguments.
The second routine generates a compiler warning whenver the compiler
encounters a call to it.

Thus, if the program foo.chpl contained the following calls:

\begin{numberedchapel}
foo(3.4);
foo("hi");
foo(1, 2);
foo(1.2, 3.4);
foo("hi", "bye");
foo(1, 2.3);
foo("hi", 2.3);
\end{numberedchapel}

\noindent compiling the program would generate output like:

\begin{commandline}
foo.chpl:1: warning: 1-argument version of foo called with type: real
foo.chpl:2: warning: 1-argument version of foo called with type: string
foo.chpl:6: error: foo() called with non-matching types: int != real
\end{commandline}

\end{example}

\section{Example: A Generic Stack}
\label{Example_Generic_Stack}
\begin{chapelpre}
% genericStack.chpl
\end{chapelpre}
\begin{chapel}
class MyNode {
  type itemType;              // type of item
  var item: itemType;         // item in node
  var next: MyNode(itemType); // reference to next node (same type)
}

record Stack {
  type itemType;             // type of items
  var top: MyNode(itemType); // top node on stack linked list

  proc push(item: itemType) {
    top = new MyNode(itemType, item, top);
  }

  proc pop() {
    if isEmpty then
      halt("attempt to pop an item off an empty stack");
    var oldTop = top;
    top = top.next;
    return oldTop.item;
  }

  proc isEmpty return top == nil;
}
\end{chapel}
\begin{chapelpost}
var s: Stack(int);
s.push(1);
s.push(2);
s.push(3);
while !s.isEmpty do
  writeln(s.pop());
\end{chapelpost}
\begin{chapeloutput}
3
2
1
\end{chapeloutput}

\cleardoublepage
\sekshun{Input and Output}
\label{Input_and_Output}

Chapel provides a built-in \chpl{file} class to handle input and
output to files using functions and methods
called \chpl{read}, \chpl{readln},
\chpl{write}, and \chpl{writeln}.

\section{The {\em file} type}
\index{file type}

The file class contains the following fields:
\begin{itemize}
\item
The \chpl{filename} field is a \chpl{string} that contains the name of
the file.
\item
The \chpl{mode} field is a \chpl{FileAccessMode} enum value that indicates
whether the file is being read or written.
\item
The \chpl{path} field is a \chpl{string} that contains the path of the
file.
\item
The \chpl{style} field can be set to \chpl{text} or \chpl{binary} to
specify that reading from or writing to the file should be done with
text or binary formats.
\end{itemize}
These fields can be modified any time that the file is closed.

The \chpl{mode} field supports the following \chpl{FileAccessMode} values:
\begin{itemize}
\item
\chpl{FileAccessMode.read} The file can be read.
\item
\chpl{FileAccessMode.write} The file can be written.
\end{itemize}

The file type supports the following methods:
\index{file type!methods}
\begin{itemize}
\item
The \chpl{open()} method opens the file for reading and/or writing.
\item
The \chpl{close()} method closes the file for reading and/or writing.
\item
The \chpl{isOpen} method returns true if the file is open for reading
and/or writing, and otherwise returns false.
\item
The \chpl{eof} method returns true if the file is at its end-of-file
position and returns false otherwise.
\item
The \chpl{flush()} method flushes the file, finishing outstanding
reading and writing.
\end{itemize}

Additionally, the file type supports the
methods \chpl{read}, \chpl{readln}, \chpl{write}, and \chpl{writeln} for 
input and output as discussed in~\rsec{filewrite} and~\rsec{fileread}.

\section{Standard files {\em stdout}, {\em stdin}, and {\em stderr}}
\index{file type!standard files stdin, stdout, stderr}
\index{stdin@\chpl{stdin}}
\index{stdout@\chpl{stdout}}
\index{stderr@\chpl{stderr}}

The files \chpl{stdout}, \chpl{stdin}, and \chpl{stderr} are
predefined and map to standard output, standard input, and standard
error as implemented in a platform dependent fashion.

\section{The {\em write}, {\em writeln}, {\em read}, and {\em readln} 
functions}
\index{writeln@\chpl{writeln}}
\index{write@\chpl{write}}
\index{read@\chpl{read}}
\index{readln@\chpl{readln}}
\index{read}
\index{write}

The built-in function \chpl{write} takes an arbitrary number of
arguments and prints each out in turn to \chpl{stdout}.  The built-in
function \chpl{writeln} is identical to \chpl{write} except that it
outputs an additional {\em end-of-line} character after writing out
the argument expressions.  Both of these functions will generate their
output atomically with respect to other calls to these functions from
other tasks.

The built-in function \chpl{read} takes an arbitrary number of
variable expressions and reads into each in turn from \chpl{stdin}.
Any whitespace is skipped over and is used only to separate one
argument from the next.  The built-in function \chpl{readln} is
identical except that upon reading all of its arguments it scans ahead
in the input stream until just after the next {\em end-of-line}
character.

The \chpl{read} and \chpl{readln} functions are also defined to take
an arbitrary number of types as arguments.  In this case, the
functions read an expression of each argument type.  In the event that
a single type is specified, the return value is the value that was
read; if multiple types are specified, a tuple of the values is
returned.

These functions are wrappers for the methods on files described next.

\begin{example}
The \chpl{writeln} wrapper function allows for a simple implementation
of the {\em Hello-World} program:
\begin{chapel}
writeln("Hello, World!");
\end{chapel}
\end{example}

\begin{example}
The following code shows three ways to read values into a pair of
variables \chpl{x} and \chpl{y}:
\begin{chapel}
var x: int;
var y: real;

/* reading into variable expressions */
read(x, y);

/* reading via a single type argument */
x = read(int);
y = read(real);

/* reading via multiple type arguments */
(x, y) = read(int, real);
\end{chapel}
\end{example}

\section{User-Defined {\em writeThis} methods}

To define the output for a given type, the user must define a method
called \chpl{writeThis} on that type that takes a single argument of
\chpl{Writer} type.  If such a method does not exist, a default method is
created.

\section{The {\em write} and {\em writeln} method on files}
\label{filewrite}
\index{write!on files}

The \chpl{file} type supports methods \chpl{write} and \chpl{writeln}
for output.  These methods are defined to take an arbitrary number of
arguments.  Each argument is written in turn by calling
the \chpl{writeThis} method on that argument.
Default \chpl{writeThis} methods are bound to any type that the user
does not explicitly create one for.

A lock is used to ensure that output is serialized across multiple
tasks.

\subsection{The {\em write} and {\em writeln} method on strings}
\label{stringwrite}
\index{write!on strings}

The \chpl{write} and \chpl{writeln} methods can also be called on
strings to write the output to a string instead of a file.

\subsection{Generalized {\em write} and {\em writeln}}
\label{writer}
\index{Writer@\chpl{Writer}}

The \chpl{Writer} class contains no arguments and serves as a base
class to allow user-defined classes to be written to.  If a class is
defined to be a subclass of Writer, it must override
the \chpl{writeIt} method that takes a \chpl{string} as an argument.

\begin{example}
The following code defines a subclass of \chpl{Writer} that overrides
the \chpl{writeIt} method to allow it to be written to.  It also
overrides the \chpl{writeThis} method to override the default way that
it is written.
\begin{chapel}
class C: Writer {
  var data: string;
  proc writeIt(s: string) {
    data += s.substring(1);
  }
  proc writeThis(x: Writer) {
    x.write(data);
  }
}

var c = new C();
c.write(41, 32, 23, 14);
writeln(c);
\end{chapel}
The \chpl{C} class filters the arguments sent to it, printing out only
the first letter.  The output to the above is thus \chpl{4321}.
\end{example}


\section{The {\em read} and {\em readln} methods on files}
\label{fileread}
\index{read!on files}

The \chpl{file} type supports \chpl{read} and \chpl{readln} methods.
The \chpl{read} method takes an arbitrary number of arguments, reading
in each argument from file.  The \chpl{readln} method also
takes an arbitrary number of arguments, reading in each argument
from a single line or multiple lines in the file and 
advancing the file pointer to the next line after the last argument 
is read.

The \chpl{file} type also supports overloaded methods \chpl{read}
and \chpl{readln} that take an arbitrary number of types as arguments.
These methods read values of the specified types from the file and
return them in a tuple.  If only one type is read, the value is not
returned in a tuple, but is returned directly.

\begin{example}
The following line of code reads a value of type \chpl{int} from
\chpl{stdin} and uses it to initialize variable \chpl{x} (causing
\chpl{x} to have an inferred type of \chpl{int}):
\begin{chapel}
var x = stdin.read(int);
\end{chapel}
\end{example}


\section{Default {\em read} and {\em write} methods}
\index{write!default methods}
\index{read!default methods}

Default \chpl{write} methods are created for all types for which a user
\chpl{write} method is not defined.  They have the following semantics:
\begin{itemize}
\item
{\bf arrays} Outputs the elements of the array in row-major order
where rows are separated by line-feeds and blank lines are used to
separate other dimensions.
\item
{\bf domains} Outputs the dimensions of the domain enclosed
by \chpl{[} and \chpl{]}.
\item
{\bf ranges} Outputs the lower bound of the range followed
by \chpl{..} followed by the upper bound of the range.  If the stride
of the range is not one, the output is additionally followed by the
word \chpl{by} followed by the stride of the range.
\item
{\bf tuples} Outputs the components of the tuple in order delimited
by \chpl{(} and \chpl{)}, and separated by commas.
\item
{\bf classes} Outputs the values within the fields of the class
prefixed by the name of the field and the character \chpl{=}.  Each
field is separated by a comma.  The output is delimited by \chpl{\{}
and \chpl{\}}.
\item
{\bf records} Outputs the values within the fields of the class
prefixed by the name of the field and the character \chpl{=}.  Each
field is separated by a comma.  The output is delimited by \chpl{(}
and \chpl{)}.
\end{itemize}

Default \chpl{read} methods are created for all types for which a user
\chpl{read} method is not defined.  The default \chpl{read} methods are
defined to read in the output of the default \chpl{write} method.

\cleardoublepage
\sekshun{Task Parallelism and Synchronization}
\label{Task_Parallelism_and_Synchronization}
\index{synchronization}

Chapel supports both task parallelism and data parallelism.  This
chapter details task parallelism as follows:
\begin{itemize}
\item \rsec{Task_parallelism} introduces tasks and task parallelism.
\item \rsec{Begin} describes the begin statement, an unstructured way
to introduce concurrency into a program.
\item \rsec{Synchronization_Variables} describes synchronization
variables, an unstructured mechanism for synchronizing tasks.
\item \rsec{Atomic_Variables} describes atomic variables, a mechanism
for supporting atomic operations.
\item \rsec{Cobegin} describes the cobegin statement, a structured way to
introduce concurrency into a program.
\item \rsec{Coforall} describes the coforall loop, another structured way to
introduce concurrency into a program.
\item \rsec{Task_Intents} specifies how variables from outer scopes
are handled within \chpl{begin}, \chpl{cobegin} and \chpl{coforall}
statements.
\item \rsec{Sync_Statement} describes the sync statement, a structured
way to control parallelism.
\item \rsec{Serial} describes the serial statement, a structured way to suppress
parallelism.
\item \rsec{Atomic_Statement} describes the atomic statement, a construct to
support atomic transactions.
\end{itemize}

\section{Tasks and Task Parallelism}
\label{Task_parallelism}
\index{task parallelism}
\index{parallelism!task}

A Chapel \emph{task} is a distinct context of execution that may be
running concurrently with other tasks.  Chapel provides a simple
construct, the \chpl{begin} statement, to create tasks, introducing
concurrency into a program in an unstructured way.  In addition,
Chapel introduces two type qualifiers, \chpl{sync} and \chpl{single},
for synchronization between tasks.

Chapel provides two constructs, the \chpl{cobegin} and \chpl{coforall} statements,
to introduce concurrency in a more structured way.  These constructs
create multiple tasks but do not continue until these tasks have
completed.  In addition, Chapel provides two constructs, the \chpl{sync} and
\chpl{serial} statements, to insert synchronization and suppress parallelism.
All four of these constructs can be implemented through judicious uses
of the unstructured task-parallel constructs described in the previous
paragraph.

\index{task parallelism!task creation}
\index{task creation}
Tasks are considered to be created when execution reaches the start
of a \chpl{begin}, \chpl{cobegin}, or \chpl{coforall} statement.
When the tasks are actually executed depends on the Chapel
implementation and run-time execution state.

\index{task function}
\index{task parallelism!task function}
A task is represented as a call to a \emph{task function}, whose body
contains the Chapel code for the task. Variables defined in outer
scopes are considered to be passed into a task function by default intent,
unless a different \emph{task intent} is specified explicitly
by a \sntx{task-intent-clause}.

% Should this be placed more prominently right before this section?
Accesses to the same variable from different tasks are subject
to the Memory Consistency Model (\rsec{Memory_Consistency_Model}).
Such accesses can result from aliasing due to \chpl{ref} argument intents
or task intents, among others.

\section{The Begin Statement}
\label{Begin}
\index{begin@\chpl{begin}}
\index{statements!begin@\chpl{begin}}

The begin statement creates a task to execute a statement.  The syntax
for the begin statement is given by
\begin{syntax}
begin-statement:
  `begin' task-intent-clause[OPT] statement
\end{syntax}
Control continues concurrently with the statement following the
begin statement.

\begin{chapelexample}{beginUnordered.chpl}
The code
\begin{chapel}
begin writeln("output from spawned task");
writeln("output from main task");
\end{chapel}
\begin{chapelprediff}
\#!/usr/bin/env sh
testname=$1
outfile=$2
sort $outfile > $outfile.2
mv $outfile.2 $outfile
\end{chapelprediff}
\begin{chapeloutput}
output from main task
output from spawned task
\end{chapeloutput}
executes two \chpl{writeln} statements that output the strings to the
terminal, but the ordering is purposely unspecified.  There is no
guarantee as to which statement will execute first.  When the
begin statement is executed, a new task is created that will execute
the \chpl{writeln} statement within it.  However, execution will
continue immediately after task creation with the next statement.
\end{chapelexample}

A begin statement creates a single task function,
whose body is the body of the begin statement.
The handling of the outer variables within the task function and
the role of \sntx{task-intent-clause} are defined in \rsec{Task_Intents}.

Yield and return statements are not allowed in begin blocks.  Break
and continue statements may not be used to exit a begin block.

%
% TODO: Future environment about task teams.
%

\section{Synchronization Variables}
\label{Synchronization_Variables}
\index{synchronization variables!sync@\chpl{sync}}
\index{synchronization variables!sync@\chpl{single}}
\index{sync@\chpl{sync}}
\index{single@\chpl{single}}

Synchronization variables have a logical state associated with the
value.  The state of the variable is either {\em full} or {\em empty}.
Normal reads of a synchronization variable cannot proceed until the
variable's state is full.  Normal writes of a synchronization variable
cannot proceed until the variable's state is empty.

Chapel supports two types of synchronization variables: sync and
single.  Both types behave similarly, except that a single variable
may only be written once.  Consequently, when a sync variable is read,
its state transitions to empty, whereas when a single variable is
read, its state does not change.  When either type of synchronization
variable is written, its state transitions to full.

\chpl{sync} and \chpl{single} are type qualifiers and precede
the type of the variable's value in the declaration.  Sync and single
are supported for all Chapel primitive types (~\rsec{Primitive_Types})
except complex.  They are also supported for enumerated types
(~\rsec{Enumerated_Types}) and variables of class type
(~\rsec{Class_Types}).  For sync variables of class type, the
full/empty state applies to the reference to the class object, not to
its member fields.

\begin{rationale}
It is only well-formed to apply full-empty semantics to types that
have no more than a single logical value.  Booleans, integers, real
and imaginary numbers, enums, and class references all meet this
criteria.  Since it is possible to read/write the individual elements
of a complex value, it's not obvious how the full-empty semantics
would interact with such operations.  While one could argue that
record types with a single field could also be included, the user can
more directly express such cases by declaring the field itself to be
of sync type.
\end{rationale}

If a task attempts to read or write a synchronization variable that is
not in the correct state, the task is suspended.  When the variable
transitions to the correct state, the task is resumed.  If there are
multiple tasks blocked waiting for the state transition, one is
non-deterministically selected to proceed and the others continue to
wait if it is a sync variable; all tasks are selected to proceed
if it is a single variable.

%
% TODO: The following should really be a 'type-expression' right?
% i.e., if there was a user type alias 'eltType', one should be
% able to do 'sync eltType'...
%

A synchronization variable is specified with a sync or single type
given by the following syntax:
\begin{syntax}
sync-type:
  `sync' type-expression

single-type:
  `single' type-expression
\end{syntax}

If a synchronization variable declaration has an initialization
expression, then the variable is initially full, otherwise it is
initially empty.

\begin{chapelexample}{beginWithSyncVar.chpl}
The code
\begin{chapel}
class Tree {
  var isLeaf: bool;
  var left, right: unmanaged Tree?;
  var value: int;

  proc sum():int {
    if (isLeaf) then
       return value;

    var x(*\texttt{\$}*): sync int;
    begin x(*\texttt{\$}*) = left.sum();
    var y = right.sum();
    return x(*\texttt{\$}*) + y;
  }
}
\end{chapel}
\begin{chapelpost}
var tree: unmanaged Tree = new unmanaged Tree(false, new unmanaged Tree(false, new unmanaged Tree(true, nil, nil, 1),
                                                 new unmanaged Tree(true, nil, nil, 1), 1),
                                 new unmanaged Tree(false, new unmanaged Tree(true, nil, nil, 1),
                                                 new unmanaged Tree(true, nil, nil, 1), 1), 1);
writeln(tree.sum());
proc Tree.deinit() {
  if isLeaf then return;
  delete left;
  delete right;
}
delete tree;
\end{chapelpost}
\begin{chapeloutput}
4
\end{chapeloutput}
the sync variable \chpl{x$\mbox{\texttt{\$}}$} is assigned by an
asynchronous task created with the begin statement.  The task
returning the sum waits on the reading of \chpl{x$\mbox{\texttt{\$}}$}
until it has been assigned.  By convention, synchronization variables
end in \texttt{\$} to provide a visual cue to the programmer
indicating that the task may block.
\end{chapelexample}

\begin{chapelexample}{syncCounter.chpl}
Sync variables are useful for tallying data from multiple tasks as
well.  If all updates to an initialized sync variable are via compound
assignment operators (or equivalently, traditional assignments that
read and write the variable once), the full/empty state of the sync
variable guarantees that the reads and writes will be interleaved
in a manner that makes the updates atomic.  For example, the code:
\begin{chapel}
var count(*\texttt{\$}*): sync int = 0;
cobegin {
  count(*\texttt{\$}*) += 1;
  count(*\texttt{\$}*) += 1;
  count(*\texttt{\$}*) += 1;
}
\end{chapel}
\begin{chapelpost}
writeln("count is: ", count(*\texttt{\$}*).readFF());
\end{chapelpost}
\begin{chapeloutput}
count is: 3
\end{chapeloutput}
creates three tasks that increment \chpl{count$\mbox{\texttt{\$}}$}.
If \chpl{count$\mbox{\texttt{\$}}$} were not a sync variable, this code
would be unsafe because two tasks could then read the same value
before either had written its updated value, causing one of the
increments to be lost.
\end{chapelexample}

\pagebreak
\begin{chapelexample}{singleVar.chpl}
The following code implements a simple split-phase barrier using a
single variable.
\begin{chapelpre}
config const n = 44;
proc work(i) {
  // do nothing
}
\end{chapelpre}
\begin{chapel}
var count(*\texttt{\$}*): sync int = n;  // counter which also serves as a lock
var release(*\texttt{\$}*): single bool; // barrier release

forall t in 1..n do begin {
  work(t);
  var myc = count(*\texttt{\$}*);  // read the count, set state to empty
  if myc!=1 {
    write(".");
    count(*\texttt{\$}*) = myc-1;  // update the count, set state to full
    // we could also do some work here before blocking
    release(*\texttt{\$}*);
  } else {
    release(*\texttt{\$}*) = true;  // last one here, release everyone
    writeln("done");
  }
}
\end{chapel}
\begin{chapeloutput}
...........................................done
\end{chapeloutput}
In each iteration of the forall loop after the work is completed, the
task reads the \chpl{count$\mbox{\texttt{\$}}$} variable, which is
used to tally the number of tasks that have arrived.  All tasks except
the last task to arrive will block while trying to read the
variable \chpl{release$\mbox{\texttt{\$}}$}.  The last task to arrive
will write to \chpl{release$\mbox{\texttt{\$}}$}, setting its state to
full at which time all the other tasks can be unblocked and run.
\end{chapelexample}

\index{synchronization types!formal arguments}
If a formal argument with a default intent either has a synchronization
type or the formal is generic (\rsec{Formal_Arguments_of_Generic_Type})
and the actual has a synchronization type, the actual must be an
lvalue and is passed by reference. In these cases the formal itself
is an lvalue, too. The actual argument is not read or written during
argument passing; its state is not changed or waited on. The qualifier
\chpl{sync} or \chpl{single} without the value type can be used to
specify a generic formal argument that requires a \chpl{sync}
or \chpl{single} actual.

\index{synchronization types!actual arguments}
When the actual argument is a \chpl{sync} or \chpl{single} and the
corresponding formal has the actual's base type or is implicitly
converted from that type, a normal read of the actual is performed
when the call is made, and the read value is passed to the formal.


\subsection{Predefined Single and Sync Methods}
\label{Functions_on_Synchronization_Variables}
\index{synchronization variables!predefined methods on}

The following methods are defined for variables of sync and single
type.

\index{readFE (sync var)@\chpl{readFE} (sync var)}
\index{predefined functions!readFE (sync var)@\chpl{readFE} (sync var)}
\begin{protohead}
proc (sync t).readFE(): t
\end{protohead}
\begin{protobody}
Returns the value of the sync variable.  This method blocks until the
sync variable is full.  The state of the sync variable is set to empty
when this method completes.
This method implements the normal read of a \chpl{sync} variable.
\end{protobody}

\index{readFF (sync var)@\chpl{readFF} (sync var)}
\index{predefined functions!readFF (sync var)@\chpl{readFF} (sync var)}
\begin{protohead}
proc (sync t).readFF(): t
proc (single t).readFF(): t
\end{protohead}
\begin{protobody}
Returns the value of the sync or single variable.  This method blocks
until the sync or single variable is full.  The state of the sync or
single variable remains full when this method completes.
This method implements the normal read of a \chpl{single} variable.
\end{protobody}

\pagebreak
\index{readXX (sync var)@\chpl{readXX} (sync var)}
\index{predefined functions!readXX (sync var)@\chpl{readXX} (sync var)}
\begin{protohead}
proc (sync t).readXX(): t
proc (single t).readXX(): t
\end{protohead}
\begin{protobody}
Returns the value of the sync or single variable.  This method is non-blocking
and the state of the sync or single variable is unchanged when this method
completes.
\end{protobody}

\index{writeEF (sync var)@\chpl{writeEF} (sync var)}
\index{predefined functions!writeEF (sync var)@\chpl{writeEF} (sync var)}
\begin{protohead}
proc (sync t).writeEF(v: t)
proc (single t).writeEF(v: t)
\end{protohead}
\begin{protobody}
Assigns \chpl{v} to the value of the sync or single variable.  This
method blocks until the sync or single variable is empty.  The state
of the sync or single variable is set to full when this method
completes.
This method implements the normal write of a \chpl{sync} or \chpl{single}
variable.
\end{protobody}

\index{writeFF (sync var)@\chpl{writeFF} (sync var)}
\index{predefined functions!writeFF (sync var)@\chpl{writeFF} (sync var)}
\begin{protohead}
proc (sync t).writeFF(v: t)
\end{protohead}
\begin{protobody}
Assigns \chpl{v} to the value of the sync variable.  This method
blocks until the sync variable is full.  The state of the sync
variable remains full when this method completes.
\end{protobody}

\index{writeXF (sync var)@\chpl{writeXF} (sync var)}
\index{predefined functions!writeXF (sync var)@\chpl{writeXF} (sync var)}
\begin{protohead}
proc (sync t).writeXF(v: t)
\end{protohead}
\begin{protobody}
Assigns \chpl{v} to the value of the sync variable.  This method is
non-blocking and the state of the sync variable is set to full when
this method completes.
\end{protobody}

\index{reset (sync var)@\chpl{reset} (sync var)}
\index{predefined functions!reset (sync var)@\chpl{reset} (sync var)}
\begin{protohead}
proc (sync t).reset()
\end{protohead}
\begin{protobody}
Assigns the default value of type \chpl{t} to the value of the sync
variable.  This method is non-blocking and the state of the sync
variable is set to empty when this method completes.
\end{protobody}

\index{isFull (sync var)@\chpl{isFull} (sync var)}
\index{predefined functions!isFull (sync var)@\chpl{isFull} (sync var)}
\begin{protohead}
proc (sync t).isFull: bool
proc (single t).isFull: bool
\end{protohead}
\begin{protobody}
Returns \chpl{true} if the sync or single variable is full and \chpl{false}
otherwise.  This method is non-blocking and the state of the sync or single
variable is unchanged when this method completes.
\end{protobody}

Note that \chpl{writeEF} and \chpl{readFE}/\chpl{readFF} methods
(for \chpl{sync} and \chpl{single} variables, respectively) are
implicitly invoked for normal writes and reads of synchronization variables.


\begin{chapelexample}{syncMethods.chpl}
Given the following declarations
\begin{chapelpre}
{ // }
\end{chapelpre}
\begin{chapel}
var x(*\texttt{\$}*): sync int;
var y(*\texttt{\$}*): single int;
var z: int;
\end{chapel}
the code
\begin{chapel}
x(*\texttt{\$}*) = 5;
y(*\texttt{\$}*) = 6;
z = x(*\texttt{\$}*) + y(*\texttt{\$}*);
\end{chapel}
\begin{chapelnoprint}
writeln((x(*\texttt{\$}*).readXX(), y(*\texttt{\$}*).readFF(), z));
// {
}
{ // }
var x(*\texttt{\$}*): sync int;
var y(*\texttt{\$}*): single int;
var z: int;
\end{chapelnoprint}
is equivalent to
\begin{chapel}
x(*\texttt{\$}*).writeEF(5);
y(*\texttt{\$}*).writeEF(6);
z = x(*\texttt{\$}*).readFE() + y(*\texttt{\$}*).readFF();
\end{chapel}
\begin{chapelpost}
writeln((x(*\texttt{\$}*).readXX(), y(*\texttt{\$}*).readFF(), z));
// {
}
\end{chapelpost}
\begin{chapeloutput}
(5, 6, 11)
(5, 6, 11)
\end{chapeloutput}
\end{chapelexample}



\section{Atomic Variables}
\label{Atomic_Variables}
\index{atomic variables!atomic@\chpl{atomic}}
\index{atomic@\chpl{atomic}}

Atomic variables are variables that support atomic operations. Chapel
currently supports atomic operations for bools, all supported sizes of
signed and unsigned integers, as well as all supported sizes of reals.

\begin{rationale}
The choice of supported atomic variable types as well as the atomic
operations was strongly influenced by the C11 standard.
\end{rationale}

Atomic is a type qualifier that precedes the variable's type in
the declaration. Atomic operations are supported for bools, and all
sizes of ints, uints, and reals.

An atomic variable is specified with an atomic type given by the
following syntax:

\begin{syntax}
atomic-type:
  `atomic' type-expression
\end{syntax}

\subsection{Predefined Atomic Methods}
\label{Functions_on_Atomic_Variables}
\index{atomic variables!predefined methods on}

The following methods are defined for variables of atomic type. Note
that not all operations are supported for all atomic types. The
supported types are listed for each method.

\index{atomic types!memory order}
Most of the predefined atomic methods accept an optional argument
named \chpl{order} of type memoryOrder. The \chpl{order} argument is
used to specify the ordering constraints of atomic operations. The
supported memoryOrder values are:
\begin{itemize}
\item{memoryOrder.relaxed}
\item{memoryOrder.acquire}
\item{memoryOrder.release}
\item{memoryOrder.acqRel}
\item{memoryOrder.seqCst}
\end{itemize}

Unless specified, the default for the memoryOrder parameter is
memoryOrder.seqCst.

\begin{note}
Not all architectures or implementations may support all memoryOrder
values.  In these cases, the implementation should default to a more
conservative ordering than specified.
\end{note}

\index{read (atomic var)@\chpl{read} (atomic var)}
\index{predefined functions!read (atomic var)@\chpl{read} (atomic var)}
\begin{protohead}
proc (atomic T).read(order:memoryOrder = memoryOrder.seqCst): T
\end{protohead}
\begin{protobody}
Reads and returns the stored value. Defined for all atomic types.
\end{protobody}

\index{write (atomic var)@\chpl{write} (atomic var)}
\index{predefined functions!write (atomic var)@\chpl{write} (atomic var)}
\begin{protohead}
proc (atomic T).write(v: T, order:memoryOrder = memoryOrder.seqCst)
\end{protohead}
\begin{protobody}
Stores \chpl{v} as the new value. Defined for all atomic types.
\end{protobody}

\index{exchange (atomic var)@\chpl{exchange} (atomic var)}
\index{predefined functions!exchange (atomic var)@\chpl{exchange} (atomic var)}
\begin{protohead}
proc (atomic T).exchange(v: T, order:memoryOrder = memoryOrder.seqCst): T
\end{protohead}
\begin{protobody}
Stores \chpl{v} as the new value and returns the original
value. Defined for all atomic types.
\end{protobody}

\index{compareExchangWeak (atomic var)@\chpl{compareExchangeWeak} (atomic var)}
\index{predefined functions!compareExchangeWeak (atomic var)@\chpl{compareExchangeWeak} (atomic var)}
\index{compareExchangStrong (atomic var)@\chpl{compareExchangeStrong} (atomic var)}
\index{predefined functions!compareExchangeStrong (atomic var)@\chpl{compareExchangeStrong} (atomic var)}
\index{compareExchange (atomic var)@\chpl{compareExchange} (atomic var)}
\index{predefined functions!compareExchange (atomic var)@\chpl{compareExchange} (atomic var)}
\begin{protohead}
proc (atomic T).compareExchangeWeak(e: t, v: T, order:memoryOrder = memoryOrder.seqCst): bool
proc (atomic T).compareExchangeStrong(e: t, v: T, order:memoryOrder = memoryOrder.seqCst): bool
proc (atomic T).compareExchange(e: t, v: T, order:memoryOrder = memoryOrder.seqCst): bool
\end{protohead}
\begin{protobody}
Stores \chpl{v} as the new value, if and only if the original value is
equal to \chpl{e}. Returns \chpl{true} if \chpl{v} was
stored, \chpl{false} otherwise. The 'weak' variation may
return \chpl{false} even if the original value was equal to \chpl{e},
if, for example, the value could not be updated
atomically. \chpl{compareExchange} is equivalent to
\chpl{compareExchangeStrong}.  Defined for all atomic types.
\end{protobody}

\index{add (atomic var)@\chpl{add} (atomic var)}
\index{predefined functions!add (atomic var)@\chpl{add} (atomic var)}
\index{sub (atomic var)@\chpl{sub} (atomic var)}
\index{predefined functions!sub (atomic var)@\chpl{sub} (atomic var)}
\index{or (atomic var)@\chpl{or} (atomic var)}
\index{predefined functions!or (atomic var)@\chpl{or} (atomic var)}
\index{and (atomic var)@\chpl{and} (atomic var)}
\index{predefined functions!and (atomic var)@\chpl{and} (atomic var)}
\index{xor (atomic var)@\chpl{xor} (atomic var)}
\index{predefined functions!xor (atomic var)@\chpl{xor} (atomic var)}
\begin{protohead}
proc (atomic T).add(v: T, order:memoryOrder = memoryOrder.seqCst)
proc (atomic T).sub(v: T, order:memoryOrder = memoryOrder.seqCst)
proc (atomic T).or(v: T, order:memoryOrder = memoryOrder.seqCst)
proc (atomic T).and(v: T, order:memoryOrder = memoryOrder.seqCst)
proc (atomic T).xor(v: T, order:memoryOrder = memoryOrder.seqCst)
\end{protohead}
\begin{protobody}
Applies the appropriate operator (\verb@+@, \verb@-@, \verb@|@,
\verb@&@, \verb@^@) to the original value and \chpl{v} and stores the result.
All of the methods are defined for integral atomic types. Only add and
sub (\verb@+@, \verb@-@) are defined for \chpl{real} atomic types.
None of the methods are defined for the \chpl{bool} atomic type.
\end{protobody}

\begin{future}
In the future we may overload certain operations such as \verb@+=@ to
call the above methods automatically for atomic variables.
\end{future}

\index{fetchAdd (atomic var)@\chpl{fetchAdd} (atomic var)}
\index{predefined functions!fetchAdd (atomic var)@\chpl{fetchAdd} (atomic var)}
\index{fetchSub (atomic var)@\chpl{fetchSub} (atomic var)}
\index{predefined functions!fetchSub (atomic var)@\chpl{fetchSub} (atomic var)}
\index{fetchOr (atomic var)@\chpl{fetchOr} (atomic var)}
\index{predefined functions!fetchOr (atomic var)@\chpl{fetchOr} (atomic var)}
\index{fetchAnd (atomic var)@\chpl{fetchAnd} (atomic var)}
\index{predefined functions!fetchAnd (atomic var)@\chpl{fetchAnd} (atomic var)}
\index{fetchXor (atomic var)@\chpl{fetchXor} (atomic var)}
\index{predefined functions!fetchXor (atomic var)@\chpl{fetchXor} (atomic var)}
\begin{protohead}
proc (atomic T).fetchAdd(v: T, order:memoryOrder = memoryOrder.seqCst): T
proc (atomic T).fetchSub(v: T, order:memoryOrder = memoryOrder.seqCst): T
proc (atomic T).fetchOr(v: T, order:memoryOrder = memoryOrder.seqCst): T
proc (atomic T).fetchAnd(v: T, order:memoryOrder = memoryOrder.seqCst): T
proc (atomic T).fetchXor(v: T, order:memoryOrder = memoryOrder.seqCst): T
\end{protohead}
\begin{protobody}
Applies the appropriate operator (\verb@+@, \verb@-@, \verb@|@,
\verb@&@, \verb@^@) to the original value and \chpl{v}, stores the result, and
returns the original value. All of the methods are defined for
integral atomic types. Only add and sub (\verb@+@, \verb@-@) are
defined for \chpl{real} atomic types.  None of the methods are defined
for the \chpl{bool} atomic type.
\end{protobody}


\index{testAndSet (atomic bool)@\chpl{testAndSet} (atomic bool)}
\index{predefined functions!testAndSet (atomic bool)@\chpl{testAndSet}
(atomic bool)}
\begin{protohead}
proc (atomic bool).testAndSet(order:memoryOrder = memoryOrder.seqCst): bool
\end{protohead}
\begin{protobody}
Stores \chpl{true} as the new value and returns the old
value. Equivalent to \chpl{exchange(true)}. Only defined for
the \chpl{bool} atomic type.
\end{protobody}

\index{clear (atomic bool)@\chpl{clear} (atomic bool)}
\index{predefined functions!clear (atomic bool)@\chpl{clear} (atomic bool)}
\begin{protohead}
proc (atomic bool).clear(order:memoryOrder = memoryOrder.seqCst)
\end{protohead}
\begin{protobody}
Stores \chpl{false} as the new value. Equivalent
to \chpl{write(false)}. Only defined for the \chpl{bool} atomic type.
\end{protobody}

\index{waitFor (atomic var)@\chpl{waitFor} (atomic var)}
\index{predefined functions!waitFor (atomic var)@\chpl{waitFor} (atomic var)}
\begin{protohead}
proc (atomic T).waitFor(v: T)
\end{protohead}
\begin{protobody}
Waits until the stored value is equal to \chpl{v}. The implementation
may yield the running task while waiting.  Defined for all atomic types.
\end{protobody}



\section{The Cobegin Statement}
\label{Cobegin}
\index{cobegin@\chpl{cobegin}}
\index{statements!cobegin@\chpl{cobegin}}

The cobegin statement is used to introduce concurrency within a
block.  The \chpl{cobegin} statement syntax is
\begin{syntax}
cobegin-statement:
  `cobegin' task-intent-clause[OPT] block-statement
\end{syntax}

A new task and a corresponding task function are created for each statement
in the \sntx{block-statement}.  Control
continues when all of the tasks have finished.
The handling of the outer variables within each task function and
the role of \sntx{task-intent-clause} are defined in \rsec{Task_Intents}.

Return statements are not allowed in cobegin blocks.  Yield statement
may only be lexically enclosed in cobegin blocks in parallel
iterators~(\rsec{Parallel_Iterators}).  Break and continue statements
may not be used to exit a cobegin block.


\begin{chapelexample}{cobeginAndEquivalent.chpl}
The cobegin statement
\begin{chapelpre}
var s1, s2: sync int;
proc stmt1() { s1; }
proc stmt2() { s2; s1 = 1; }
proc stmt3() { s2 = 1; }
\end{chapelpre}
\begin{chapel}
cobegin {
  stmt1();
  stmt2();
  stmt3();
}
\end{chapel}
is equivalent to the following code that uses only begin statements
and single variables to introduce concurrency and synchronize:
\begin{chapel}
var s1(*\texttt{\$}*), s2(*\texttt{\$}*), s3(*\texttt{\$}*): single bool;
begin { stmt1(); s1(*\texttt{\$}*) = true; }
begin { stmt2(); s2(*\texttt{\$}*) = true; }
begin { stmt3(); s3(*\texttt{\$}*) = true; }
s1(*\texttt{\$}*); s2(*\texttt{\$}*); s3(*\texttt{\$}*);
\end{chapel}
\begin{chapeloutput}
\end{chapeloutput}
Each begin statement is executed concurrently but control does not
continue past the final line above until each of the single variables
is written, thereby ensuring that each of the functions has finished.
\end{chapelexample}

\section{The Coforall Loop}
\label{Coforall}
\index{coforall@\chpl{coforall}}
\index{statements!coforall@\chpl{coforall}}

The coforall loop is a variant of the cobegin statement in loop form.
The syntax for the coforall loop is given by
\begin{syntax}
coforall-statement:
  `coforall' index-var-declaration `in' iteratable-expression task-intent-clause[OPT] `do' statement
  `coforall' index-var-declaration `in' iteratable-expression task-intent-clause[OPT] block-statement
  `coforall' iteratable-expression task-intent-clause[OPT] `do' statement
  `coforall' iteratable-expression task-intent-clause[OPT] block-statement
\end{syntax}

The \chpl{coforall} loop creates a separate task for each iteration of
the loop.  Control continues with the statement following
the \chpl{coforall} loop after all tasks corresponding to the
iterations of the loop have completed.

The single task function created for a \chpl{coforall} and invoked by
each task contains the loop body.
The handling of the outer variables within the task function and
the role of \sntx{task-intent-clause} are defined in \rsec{Task_Intents}.

Return statements are not allowed in coforall blocks.  Yield statement
may only be lexically enclosed in coforall blocks in parallel
iterators~(\rsec{Parallel_Iterators}).  Break and continue statements
may not be used to exit a coforall block.

\begin{chapelexample}{coforallAndEquivalent.chpl}
The coforall statement
\begin{chapelpre}
iter iterator() { for i in 1..3 do yield i; }
proc body() { }
\end{chapelpre}
\begin{chapel}
coforall i in iterator() {
  body();
}
\end{chapel}
is equivalent to the following code that uses only begin statements
and sync and single variables to introduce concurrency and
synchronize:
\begin{chapel}
var runningCount(*\texttt{\$}*): sync int = 1;
var finished(*\texttt{\$}*): single bool;
for i in iterator() {
  runningCount(*\texttt{\$}*) += 1;
  begin {
    body();
    var tmp = runningCount(*\texttt{\$}*);
    runningCount(*\texttt{\$}*) = tmp-1;
    if tmp == 1 then finished(*\texttt{\$}*) = true;
  }
}
var tmp = runningCount(*\texttt{\$}*);
runningCount(*\texttt{\$}*) = tmp-1;
if tmp == 1 then finished(*\texttt{\$}*) = true;
finished(*\texttt{\$}*);
\end{chapel}
\begin{chapeloutput}
\end{chapeloutput}
Each call to \chpl{body()} executes concurrently because it is in a
begin statement.  The sync
variable \chpl{runningCount$\mbox{\texttt{\$}}$} is used to keep track
of the number of executing tasks plus one for the main task.  When
this variable reaches zero, the single
variable \chpl{finished$\mbox{\texttt{\$}}$} is used to signal that
all of the tasks have completed.  Thus control does not continue past
the last line until all of the tasks have completed.
\end{chapelexample}


\section{Task Intents}
\label{Task_Intents}
\index{task intents}
\index{task parallelism!task functions}
\index{task parallelism!task intents}

% Would be nice to give this arrangement a name. Could say
% "the task intent rule", although that sounds a bit like
% a colloquialism.
If a variable is referenced within the lexical scope of a
\chpl{begin}, \chpl{cobegin}, or \chpl{coforall} statement
and is declared outside that statement, it is considered
to be passed as an actual argument to the corresponding task function
at task creation time. All references to the variable
within the task function implicitly refer to the task function's
corresponding formal argument.

Each formal argument of a task function has the default intent by default.
For variables of primitive and class types, this has the effect
of capturing the value of the variable at task creation time
and referencing that value instead of the original variable
within the lexical scope of the task construct.

A formal can be given another intent explicitly by listing it
with that intent in the optional \sntx{task-intent-clause}.
For example, for variables of most types, the \chpl{ref} intent allows
the task construct to modify the corresponding original variable
or to read its updated value after concurrent modifications.

The syntax of the task intent clause is:

\begin{syntax}
task-intent-clause:
  `with' ( task-intent-list )

task-intent-list:
  task-intent-item
  task-intent-item, task-intent-list

task-intent-item:
  formal-intent identifier
  task-private-var-decl
\end{syntax}

where the following intents can be used as a \sntx{formal-intent}:
\chpl{ref}, \chpl{in}, \chpl{const}, \chpl{const in}, \chpl{const ref}.
\sntx{task-private-var-decl} is defined in \rsec{Task_Private_Variables}.
In addition, \sntx{task-intent-item} may define a \chpl{reduce} intent.
Reduce intents are described in the \emph{Reduce Intents} technical note
in the online documentation:
\\ %formatting
\mbox{$$ $$ $$} %indent
\url{https://chapel-lang.org/docs/technotes/reduceIntents.html}

% TODO for task intents:
% * Introduce a 'task-formal-intent' syntactic rule
%   that expands to the legal intents - far preferable than
%   qualifying which intents are legal in the text of this paragraph.
% * That's assuming we do not support 'out' and 'inout',
%   which is still up for debate; otherwise leave as-is.
% * Do we want to allow default intents? Current implementation allows them.

The implicit treatment of outer scope variables as the task function's
formal arguments applies to both module level and local variables.
It applies to variable references within the lexical scope
of a task construct, but does not extend to its dynamic scope, i.e.,
to the functions called from the task(s) but declared outside of
the lexical scope.
The loop index variables of a \chpl{coforall} statement are not
subject to such treatment within that statement; however, they are
subject to such treatment within nested task constructs, if any.


\begin{rationale}
The primary motivation for task intents is to avoid some races on
scalar/record variables, which are possible when one task modifies a
variable and another task reads it. Without task intents,
for example, it would be easy to introduce and overlook a bug
illustrated by this simplified example:

  \begin{chapel}
  {
    var i = 0;
    while i < 10 {
      begin {
        f(i);
      }
      i += 1;
    }
  }
  \end{chapel}

If all the tasks created by the \chpl{begin} statement start executing
only after the \chpl{while} loop completes, and \chpl{i} within the
\chpl{begin} is treated as a reference to the original \chpl{i},
there will be ten tasks executing \chpl{f(10)}. However, the user most
likely intended to generate ten tasks executing
\chpl{f(0)}, \chpl{f(1)}, ..., \chpl{f(9)}.
Task intents ensure that, regardless of the timing of task execution.

Another motivation for task intents is that referring to a captured
copy in a task is often more efficient than referring to the original
variable. That's because the copy is a local constant, e.g. it could
be placed in a register when it fits.  Without task intents,
references to the original variable would need to be implemented using
a pointer dereference. This is less efficient and can hinder optimizations
in the surrounding code, for example loop-invariant code motion.

Furthermore, in the above example the scope where \chpl{i} is declared
may exit before all the ten tasks complete.  Without task intents,
the user would have to protect \chpl{i} to make sure its lexical scope
doesn't exit before the tasks referencing it complete.

We decided to treat \chpl{cobegin} and \chpl{coforall} statements the
same way as \chpl{begin}. This is for consistency and to make the
race-avoidance benefit available to more code.

We decided to apply task intents to module level variables, in addition
to local variables. Again, this is for consistency. One could view module
level variables differently than local variables (e.g. a module level
variable is ``always available''), but we favored consistency over such
an approach.

We decided not to apply task intents to ``closure'' variables, i.e.,
the variables in the dynamic scope of a task construct. This is to
keep this feature manageable, so that all variables subject to task
intents can be obtained by examining just the lexical scope of the
task construct. In general, the set of closure variables can be hard
to determine, unwieldy to implement and reason about, it is unclear
what to do with extern functions, etc.

We do not provide \chpl{inout} or \chpl{out} as task intents because they
will necessarily create a data race in a \chpl{cobegin} or \chpl{coforall}.
% that does not necessarily apply to a 'begin'
\chpl{type} and \chpl{param} intents are not available either
as they do not seem useful as task intents.
\end{rationale}

\begin{future}
For a given intent, we would also like to provide a blanket clause,
which would apply the intent to all variables.
An example of syntax for a blanket \chpl{ref} intent would be \chpl{ref *}.
\end{future}


\section{The Sync Statement}
\label{Sync_Statement}
\index{sync@\chpl{sync}}
\index{statements!sync@\chpl{sync}}

The sync statement acts as a join of all dynamically encountered
begins from within a statement.  The syntax for the sync statement is
given by
\begin{syntax}
sync-statement:
  `sync' statement
  `sync' block-statement
\end{syntax}

Return statements are not allowed in sync statement blocks.  Yield
statement may only be lexically enclosed in sync statement blocks in
parallel iterators~(\rsec{Parallel_Iterators}).  Break and continue
statements may not be used to exit a sync statement block.

\begin{chapelexample}{syncStmt1.chpl}
The sync statement can be used to wait for many dynamically created
tasks.
\begin{chapelpre}
config const n = 9;
proc work() {
  write(".");
}
\end{chapelpre}
\begin{chapel}
sync for i in 1..n do begin work();
\end{chapel}
\begin{chapelpost}
writeln("done");
\end{chapelpost}
\begin{chapeloutput}
.........done
\end{chapeloutput}
The for loop is within a sync statement and thus the tasks created
in each iteration of the loop must complete before the continuing past
the sync statement.
\end{chapelexample}

\begin{chapelexample}{syncStmt2.chpl}
The sync statement
\begin{chapelpre}
proc stmt1() { }
proc stmt2() { }
\end{chapelpre}
\begin{chapel}
sync {
  begin stmt1();
  begin stmt2();
}
\end{chapel}
is similar to the following cobegin statement
\begin{chapel}
cobegin {
  stmt1();
  stmt2();
}
\end{chapel}
\begin{chapeloutput}
\end{chapeloutput}
except that if begin statements are dynamically encountered
when \chpl{stmt1()} or \chpl{stmt2()} are executed, then the former
code will wait for these begin statements to complete whereas the
latter code will not.
\end{chapelexample}

\section{The Serial Statement}
\label{Serial}
\index{serial@\chpl{serial}}
\index{statements!serial@\chpl{serial}}

The \chpl{serial} statement can be used to dynamically disable
parallelism.  The syntax is:
\begin{syntax}
serial-statement:
  `serial' expression[OPT] `do' statement
  `serial' expression[OPT] block-statement
\end{syntax}
where the optional \sntx{expression} evaluates to a boolean value.  If
the expression is omitted, it is as though 'true' were specified.
Whatever the expression's value, the statement following it is
evaluated. If the expression is true, any dynamically encountered code
that would normally create new tasks within the statement is instead
executed by the original task without creating any new ones.  In
effect, execution is serialized.  If the expression is false, code
within the statement will generates task according to normal Chapel
rules.

\begin{chapelexample}{serialStmt1.chpl}
In the code
\begin{chapelpre}
config const lo = 9;
config const hi = 23;
proc work(i) {
  if \_\_primitive("task\_get\_serial") then
    writeln("serial ", i);
}
\end{chapelpre}
\begin{chapel}
proc f(i) {
  serial i<13 {
    cobegin {
      work(i);
      work(i);
    }
  }
}

for i in lo..hi {
  f(i);
}
\end{chapel}
\begin{chapelpost}
\end{chapelpost}
\begin{chapeloutput}
serial 9
serial 9
serial 10
serial 10
serial 11
serial 11
serial 12
serial 12
\end{chapeloutput}
the serial statement in procedure f() inhibits concurrent execution of
work() if the variable i is less than 13.
\end{chapelexample}

\begin{chapelexample}{serialStmt2.chpl}
The code
\begin{chapelpre}
proc stmt1() { write(1); }
proc stmt2() { write(2); }
proc stmt3() { write(3); }
proc stmt4() { write(4); }
var n = 3;
\end{chapelpre}
\begin{chapel}
serial {
  begin stmt1();
  cobegin {
    stmt2();
    stmt3();
  }
  coforall i in 1..n do stmt4();
}
\end{chapel}
is equivalent to
\begin{chapel}
stmt1();
{
  stmt2();
  stmt3();
}
for i in 1..n do stmt4();
\end{chapel}
\begin{chapelpost}
writeln();
\end{chapelpost}
\begin{chapeloutput}
123444123444
\end{chapeloutput}
because the expression evaluated to determine whether to serialize
always evaluates to true.
\end{chapelexample}

\section{Atomic Statements}
\label{Atomic_Statement}
\index{atomic transactions}
\index{atomic statement}
\index{atomic@\chpl{atomic}}
\index{statements!atomic@\chpl{atomic}}

\begin{openissue}
  This section describes a feature that is a work-in-progress.  We seek feedback
  and collaboration in this area from the broader community.
\end{openissue}

The \emph{atomic statement} is used to specify that a statement should appear
to execute atomically from other tasks' point of view.
In particular, no task will see memory in a state that would reflect that
the atomic statement had begun executing but had not yet completed.

\begin{openissue}
  This definition of the atomic statement provides a notion of {\em
    strong atomicity} since the action will appear atomic to any task
  at any point in its execution.  For performance reasons, it could be
  more practical to support {\em weak atomicity} in which the
  statement's atomicity is only guaranteed with respect to other
  atomic statements.  We may also pursue using atomic type qualifiers
  as a means of marking data that should be accessed atomically inside
  or outside an atomic section.
\end{openissue}

The syntax for the atomic statement is given by:
\begin{syntax}
atomic-statement:
  `atomic' statement
\end{syntax}

%\begin{chapelexample}{atomicStmt}
\begin{example}
The following code illustrates the use of an atomic statement
to perform an insertion into a doubly-linked list:

\begin{chapelpre}
class Node {
  var data: int;
  var next: Node;
  var prev: Node;
}
var head = new Node(1);
head.insertAfter(new Node(4));
head.insertAfter(new Node(2));

var obj = new Node(3);
head.next.insertAfter(obj);
\end{chapelpre}
\begin{chapel}
proc Node.insertAfter(newNode: Node) {
  atomic {
    newNode.prev = this;
    newNode.next = this.next;
    if this.next then this.next.prev = newNode;
    this.next = newNode;
  }
}
\end{chapel}
\begin{chapelpost}
writeln(head.data, head.next.data, head.next.next.data, head.next.next.next.data);
proc Node.remove() {
  if this.prev then this.prev = this.next;
  if this.next then this.next = this.prev;
  return this;
}
while (head) {
  next = head.next;
  delete head;
  head = next;
}
\end{chapelpost}
\begin{chapeloutput}
atomic.chpl:13: warning: atomic keyword is ignored (not implemented)
1234
\end{chapeloutput}
The use of the atomic statement in this routine prevents other tasks
from viewing the list in a partially-updated state in which the
pointers might not be self-consistent.
\end{example}
%\end{chapelexample}

\cleardoublepage
\sekshun{Data Parallelism}
\label{Data_Parallelism}
\index{data parallelism}
\index{parallelism!data}

Chapel provides two explicit data-parallel constructs (the
forall-statement and the forall-expression) and several idioms that
support data parallelism implicitly (whole-array assignment, function
and operator promotion, reductions, and scans).

This chapter details data parallelism as follows:
\begin{itemize}
\item \rsec{Forall} describes the forall statement.
\item \rsec{Forall_Expressions} describes forall expressions
\item \rsec{Forall_Intents} specifies how variables from outer scopes
are handled within forall statements and expressions.
\item \rsec{Promotion} describes promotion.
\item \rsec{Reductions_and_Scans} describes reductions and scans.
\item \rsec{data_parallel_knobs} describes the configuration constants for
controlling default data parallelism.
\end{itemize}       

Data-parallel constructs may result in accesses to the same variable
from different tasks, possibly due to aliasing using
\chpl{ref} argument intents or forall intents, among others.
Such accesses are subject to the Memory Consistency Model
(\rsec{Memory_Consistency_Model}).

\section{The Forall Statement}
\label{Forall}
\index{forall@\chpl{forall} (see also statements, forall)}
\index{loops!forall (see also statements, forall)}
\index{data parallelism!forall}
\index{statements!forall@\chpl{forall}}

The forall statement is a concurrent variant of the for statement
described in~\rsec{The_For_Loop}.

\subsection{Syntax}
\label{forall_syntax}
\index{statements!forall@\chpl{forall}!syntax}

The syntax of the forall statement is given by
\begin{syntax}
\begin{verbatim}
forall-statement:
  `forall' index-var-declaration `in' iteratable-expression task-intent-clause[OPT] `do' statement
  `forall' index-var-declaration `in' iteratable-expression task-intent-clause[OPT] block-statement
  `forall' iteratable-expression task-intent-clause[OPT] `do' statement
  `forall' iteratable-expression task-intent-clause[OPT] block-statement
  [ index-var-declaration `in' iteratable-expression task-intent-clause[OPT] ] statement
  [ iteratable-expression task-intent-clause[OPT] ] statement
\end{verbatim}
\end{syntax}
As with the for statement, the indices may be omitted if they are
unnecessary and the \chpl{do} keyword may be omitted before a block
statement.

The square bracketed form will resort to serial iteration
when \sntx{iteratable-expression} does not support parallel iteration.
The \chpl{forall} form will result in an error when parallel iteration
is not available.

The handling of the outer variables within the forall statement and
the role of \sntx{task-intent-clause} are defined in \rsec{Forall_Intents}.

\subsection{Execution and Serializability}
\label{forall_semantics}
\index{statements!forall@\chpl{forall}!semantics}

The forall statement evaluates the loop body once for each element
yielded by the \sntx{iteratable-expression}.  Each instance of the
forall loop's body may be executed concurrently with the others, but
this is not guaranteed.  In particular, the loop must be serializable.
Details regarding concurrency and iterator implementation are
described in~\ref{Parallel_Iterators}.

This differs from the semantics of the \chpl{coforall} loop, discussed
in~\rsec{Coforall}, where each iteration is guaranteed to run using a
distinct task.  The \chpl{coforall} loop thus has potentially higher
overhead than a forall loop with the same number of iterations, but in
cases where concurrency is required for correctness, it is essential.

\index{leading the execution of a loop}
\index{data parallelism!leader iterator}
In practice, the number of tasks that will be used to evaluate
a \chpl{forall} loop is determined by the object or iterator that
is \emph{leading} the execution of the loop, as is the mapping of
iterations to tasks.

This concept will be formalized in future drafts of the Chapel
specification. For now, the primer on parallel iterators
in the online documentation provides a brief introduction:
\\ %formatting
\mbox{$$ $$ $$ $$ $$} %indent
\url{https://chapel-lang.org/docs/primers/parIters.html}
\\
Please also refer to \emph{User-Defined Parallel Zippered Iterators in
Chapel}, published in the PGAS 2011 workshop.

Control continues with the statement following the forall loop only
after every iteration has been completely evaluated.  At this point,
all data accesses within the body of the forall loop will be
guaranteed to be completed.

A \chpl{return} statement may not be lexically enclosed in a forall
statement. A \chpl{yield} statement may only be lexically enclosed in
a forall statement that is within a parallel iterator
\rsec{Parallel_Iterators}.
A \chpl{break} statement may not be used to exit a forall statement.
A \chpl{continue} statement skips the rest of the current iteration
of the forall loop.

\begin{chapelexample}{forallStmt.chpl}
In the code
\begin{chapelpre}
\begin{verbatim}
config const N = 5;
var a: [1..N] int;
var b = [i in 1..N] i;
\end{verbatim}
\end{chapelpre}
\begin{chapel}
\begin{verbatim}
forall i in 1..N do
  a(i) = b(i);
\end{verbatim}
\end{chapel}
the user has stated that the element-wise assignments can execute
concurrently.  This loop may be executed serially with a single task,
or by using a distinct task for every iteration, or by using a number
of tasks where each task executes a number of iterations.  This loop
can also be written as
\begin{chapel}
\begin{verbatim}
[i in 1..N] a(i) = b(i);
\end{verbatim}
\end{chapel}
\begin{chapelpost}
\begin{verbatim}
writeln(a);
\end{verbatim}
\end{chapelpost}
\begin{chapeloutput}
\begin{verbatim}
1 2 3 4 5
\end{verbatim}
\end{chapeloutput}
\end{chapelexample}

\subsection{Zipper Iteration}
\label{forall_zipper}
\index{statements!forall@\chpl{forall}!zipper iteration}

Zipper iteration has the same semantics as described
in~\rsec{Zipper_Iteration} and~\rsec{Parallel_Iterators} for parallel
iteration.

\pagebreak
\section{The Forall Expression}
\label{Forall_Expressions}
\index{data parallelism!forall expressions}
\index{forall expressions (see also expressions, forall)}
\index{expressions!forall}

The forall expression is a concurrent variant of the for expression
described in~\rsec{For_Expressions}.

\subsection{Syntax}
\label{forall_expr_syntax}
\index{expressions!forall!syntax}

The syntax of a forall expression is given by
\begin{syntax}
\begin{verbatim}
forall-expression:
  `forall' index-var-declaration `in' iteratable-expression task-intent-clause[OPT] `do' expression
  `forall' iteratable-expression task-intent-clause[OPT] `do' expression
  [ index-var-declaration `in' iteratable-expression task-intent-clause[OPT] ] expression
  [ iteratable-expression task-intent-clause[OPT] ] expression
\end{verbatim}
\end{syntax}
As with the for expression, the indices may be omitted if they are
unnecessary.  The \chpl{do} keyword is always required in the
keyword-based notation.

As with the forall statement, the square bracketed
form will resort to serial iteration when \sntx{iteratable-expression}
does not support parallel iteration.  The \chpl{forall} form will
result in an error when parallel iteration is not available.

The handling of the outer variables within the forall expression and
the role of \sntx{task-intent-clause} are defined in \rsec{Forall_Intents}.

\subsection{Execution}
\label{Forall_Expression_Execution}
\index{expressions!forall!semantics}

A forall expression is an iterator that executes a forall loop (\rsec{Forall}),
evaluates the body expression on each iteration of the loop,
and yields each resulting value.

When a forall expression is used to initialize a variable, such as
\begin{chapel}
\begin{verbatim}
var X = forall iterableExpression() do computeValue();
\end{verbatim}
\end{chapel}
the variable will be inferred to have an array type.
The array's domain is defined by the \sntx{iterable-expression}
following the same rules as for promotion, both in the regular
case \rsec{Promotion} and in the zipper case \rsec{Zipper_Promotion}.

\begin{chapelexample}{forallExpr.chpl}
The code
\begin{chapel}
\begin{verbatim}
writeln(+ reduce [i in 1..10] i**2);
\end{verbatim}
\end{chapel}
\begin{chapeloutput}
\begin{verbatim}
385
\end{verbatim}
\end{chapeloutput}
applies a reduction to a forall-expression that evaluates the square
of the indices in the range \chpl{1..10}.
\end{chapelexample}

The forall expression follows the semantics of the forall statement as
described in~\ref{forall_semantics}.

\subsection{Zipper Iteration}
\index{expressions!forall!zipper iteration}
Forall expression also support zippered iteration semantics as
described in~\rsec{Zipper_Iteration} and~\rsec{Parallel_Iterators} for
parallel iteration.

\subsection{Filtering Predicates in Forall Expressions}
\label{Filtering_Predicates_Forall}
\index{expressions!forall!and conditional expressions}
\index{expressions!forall!filtering}

A filtering predicate is an if expression that is immediately enclosed
by a forall expression and does not have an
else clause.  Such an if expression filters the iterations of the
forall expression.  The iterations for which the condition does not
hold are not reflected in the result of the forall expression.

When a forall expression with a filtering predicate is captured into
a variable, the resulting array has a 1-based one-dimensional domain.

\begin{chapelexample}{forallFilter.chpl}
The following expression returns every other element starting with the
first:
\begin{chapelpre}
\begin{verbatim}
var s: [1..10] int = [i in 1..10] i;
var result =
\end{verbatim}
\end{chapelpre}
\begin{chapel}
\begin{verbatim}
[i in 1..s.numElements] if i % 2 == 1 then s(i)
\end{verbatim}
\end{chapel}
\begin{chapelpost}
\begin{verbatim}
;
writeln(result);
\end{verbatim}
\end{chapelpost}
\begin{chapeloutput}
\begin{verbatim}
1 3 5 7 9
\end{verbatim}
\end{chapeloutput}
\end{chapelexample}


\section{Forall Intents}
\label{Forall_Intents}
\index{forall intents}
\index{shadow variables}
\index{data parallelism!forall intents}
\index{data parallelism!shadow variables}
\index{statements!forall@\chpl{forall}!forall intents}
\index{statements!forall@\chpl{forall}!shadow variables}

If a variable is referenced within the lexical scope of a
forall statement or expression and is declared outside
that forall construct, it is subject to \emph{forall intents},
analogously to task intents (\rsec{Task_Intents})
for task-parallel constructs. That is, the variable is considered
to be passed as an actual argument to
each task function created by the object or iterator leading
the execution of the loop. If no tasks are created,
it is considered to be an actual argument to the leader or standalone
iterator itself. All references to the variable within the forall construct
implicitly refer to a \emph{shadow variable}, i.e.
the corresponding formal argument of the task function
or the leader/standalone iterator.

When the forall construct is inside a method on a record and accesses
a field of \chpl{this}, the field is treated as a regular variable.
That is, it is subject to forall intents and all references to this field
within the forall construct implicitly refer to the corresponding
shadow variable.

Each formal argument of a task function or iterator has the default
intent by default.  For variables of primitive, enum, and class
types, this has the effect of capturing the value of the
variable at task creation time.  Within the lexical scope of the
forall construct, the variable name references the
captured value instead of the original value.

A formal can be given another intent explicitly by listing it
with that intent in the optional \sntx{task-intent-clause}.
For example, for variables of most types, the \chpl{ref} intent allows
the body of the forall loop to modify the corresponding original
variable or to read its updated value after concurrent modifications.
The \chpl{in} intent is an alternative way to obtain task-private
variables (\rsec{Task_Private_Variables}).
A \chpl{reduce} intent can be used to reduce values across iterations
of a forall or coforall loop.
Reduce intents are described in the \emph{Reduce Intents} technical note
in the online documentation:
\\ %formatting
\mbox{$$ $$ $$} %indent
\url{https://chapel-lang.org/docs/technotes/reduceIntents.html}

\begin{rationale}
A forall statement or expression may create tasks in its implementation.
Forall intents affect those tasks in the same way that task intents
\rsec{Task_Intents}
affect the behavior of a task construct such as a \chpl{coforall} loop.
\end{rationale}


\section{Task-Private Variables}
\label{Task_Private_Variables}
\index{task-private variables}
\index{shadow variables}
\index{data parallelism!task-private variables}
\index{data parallelism!shadow variables}
\index{statements!forall@\chpl{forall}!task-private variables}
\index{statements!forall@\chpl{forall}!shadow variables}

A \emph{task-private variable} declared in a forall loop results
in a separate shadow variable in each task created by the forall
loop's parallel iterator, as well as a "top-level" shadow variable
created at the top level of the parallel iterator itself.
In contrast to regular forall intents \rsec{Forall_Intents},
these shadow variables are unrelated to outer variables
of the same name, if any.

A given shadow variable is created at the start and destroyed
at the end of its task.
Within the lexical scope of the body of the forall statement or expression,
the variable name refers to the shadow variable created in the task
that executed the current yield statement.

The "top-level" shadow variable is created at the start and destroyed
at the end of the parallel iterator. It is referenced in those iterations
of the forall loop that are due to "top-level" yields, i.e. yields
that are outside any of the task constructs that the iterator may have.

The syntax of a task-private variable declaration in a forall statement's
with-clause is:

\begin{syntax}
\begin{verbatim}
task-private-var-decl:
  task-private-var-kind identifier type-part[OPT] initialization-part[OPT]

task-private-var-kind:
  `const'
  `var'
  `ref'
\end{verbatim}
\end{syntax}

The declaration of a \chpl{const} or \chpl{var} task-private variable must
have at least one of \sntx{type-part} and \sntx{initialization-part}.
A \chpl{ref} task-private variable must have \sntx{initialization-part}
and cannot have \sntx{type-part}. A \chpl{ref} shadow variable
is a reference to the \sntx{initialization-part} as calculated at
the start of the corresponding task or the iterator.
\chpl{ref} shadow variables are never destroyed.

\begin{craychapel}
Currently task-private variables are not available for task constructs.
A regular variable declared at the start of the begin/cobegin/coforall
block can be used instead.
\end{craychapel}

\begin{chapelexample}{task-private-variable.chpl}
In the following example, the \chpl{writeln()} statement will observe
the first shadow variable 4 times: twice each for the yields
{\tt "before coforall"} and {\tt "after coforall"}.
An additional shadow variable will be created and observed twice
for each of the three \chpl{coforall} tasks.
\begin{chapel}
\begin{verbatim}
var cnt: atomic int;                     // count our shadow variables
record R { var id = cnt.fetchAdd(1); }

iter myIter() { yield ""; }              // serial iterator, unused

iter myIter(param tag) where tag == iterKind.standalone {
  for 1..2 do
    yield "before coforall";             // shadow var 0 ("top-level")
  coforall 1..3 do
    for 1..2 do
      yield "inside coforall";           // shadow vars 1..3
  for 1..2 do
    yield "after coforall";              // shadow var 0, again
}

forall str in myIter()
  with (var tpv: R)                      // declare a task-private variable
do
  writeln("shadow var: ", tpv.id, "  yield: ", str);
\end{verbatim}
\end{chapel}
\begin{chapelprediff}
\begin{verbatim}
\#!/usr/bin/env sh
testname=$1
outfile=$2
sort $outfile > $outfile.2
mv $outfile.2 $outfile
\end{verbatim}
\end{chapelprediff}
\begin{chapeloutput}
\begin{verbatim}
shadow var: 0  yield: after coforall
shadow var: 0  yield: after coforall
shadow var: 0  yield: before coforall
shadow var: 0  yield: before coforall
shadow var: 1  yield: inside coforall
shadow var: 1  yield: inside coforall
shadow var: 2  yield: inside coforall
shadow var: 2  yield: inside coforall
shadow var: 3  yield: inside coforall
shadow var: 3  yield: inside coforall
\end{verbatim}
\end{chapeloutput}
\end{chapelexample}


\section{Promotion}
\label{Promotion}
\index{promotion}
\index{data parallelism!promotion}

A function that expects one or more scalar arguments but is called
with one or more arrays, domains, ranges, or iterators is promoted if
the element types of the arrays, the index types of the domains and/or
ranges, or the yielded types of the iterators can be resolved to the
type of the argument.  The rules of when an overloaded function can be
promoted are discussed in~\rsec{Function_Resolution}.

Functions that can be promoted include procedures, operators, casts,
and methods. Also note that since class and record field access
is performed with getter methods~(\rsec{Getter_Methods}), field
access can also be promoted.

If the original function returns a value or a reference, the
corresponding promoted expression is an iterator yielding each
computed value or reference.

When a promoted expression is used to initialize a variable,
such as \chpl{var X = A.x;} in the above example,
the variable's type will be inferred to be an array.
The array's domain is defined by the expression that causes promotion:

\begin{center}
\begin{tabular}[c]{|l|l|}
\hline
input expression & resulting array's domain \\
\hline
array    &  that array's domain \\
domain   &  that domain \\
range    &  one-dimensional domain built from that range \\
iterator &  1-based one-dimensional domain \\
\hline
\end{tabular}
\end{center}

\begin{future}
We would like to allow the iterator author to specify the shape
of the iterator, i.e. the domain of the array that would capture
the result of the corresponding promoted expression, such as
\begin{chapel}
\begin{verbatim}
var myArray = myScalarFunction(myIterator());
\end{verbatim}
\end{chapel}
This will be helpful, for example, when the iterator yields
one value per an array or domain element that it iterates over
internally.
\end{future}

\begin{chapelexample}{promotion.chpl}
Given the array
\begin{chapel}
\begin{verbatim}
var A: [1..5] int = [i in 1..5] i;
\end{verbatim}
\end{chapel}
and the function
\begin{chapel}
\begin{verbatim}
proc square(x: int) return x**2;
\end{verbatim}
\end{chapel}
then the call \chpl{square(A)} results in the promotion of
the \chpl{square} function over the values in the array \chpl{A}.  The
result is an iterator that returns the
values \chpl{1}, \chpl{4}, \chpl{9}, \chpl{16}, and \chpl{25}.
\begin{chapelnoprint}
\begin{verbatim}
for s in square(A) do writeln(s);
\end{verbatim}
\end{chapelnoprint}
\begin{chapeloutput}
\begin{verbatim}
1
4
9
16
25
\end{verbatim}
\end{chapeloutput}
\end{chapelexample}

\begin{chapelexample}{field-promotion.chpl}
Given an array of points, such as \chpl{A} defined below:
\begin{chapel}
\begin{verbatim}
record Point {
  var x: real;
  var y: real;
}
var A: [1..5] Point = [i in 1..5] new Point(x=i, y=i);
\end{verbatim}
\end{chapel}
the following statement will create a new array consisting of
the \chpl{x} field value for each value in A:
\begin{chapel}
\begin{verbatim}
var X = A.x;
\end{verbatim}
\end{chapel}
and the following call will set the \chpl{y} field values for each
element in A to 1.0:
\begin{chapel}
\begin{verbatim}
A.y = 1.0;
\end{verbatim}
\end{chapel}

\begin{chapelnoprint}
\begin{verbatim}
writeln(X);
writeln(A);
\end{verbatim}
\end{chapelnoprint}
\begin{chapeloutput}
\begin{verbatim}
1.0 2.0 3.0 4.0 5.0
(x = 1.0, y = 1.0) (x = 2.0, y = 1.0) (x = 3.0, y = 1.0) (x = 4.0, y = 1.0) (x = 5.0, y = 1.0)
\end{verbatim}
\end{chapeloutput}
\end{chapelexample}


\subsection{Default Arguments}
\label{Promotion_Default_Arguments}
\index{promotion!default arguments}

When a call is promoted and that call relied upon default
arguments~(\rsec{Default_Values}), the default argument expression can
be evaluated many times. For example:

\begin{chapelexample}{promotes-default.chpl}
\begin{chapel}
\begin{verbatim}
  var counter: atomic int;

  proc nextCounterValue():int {
    var i = counter.fetchAdd(1);
    return i;
  }

  proc assignCounter(ref x:int, counter=nextCounterValue()) {
    x = counter;
  }
\end{verbatim}
\end{chapel}

Here the function assignCounter has a default argument
providing the next value from an atomic counter as the value to set.

\begin{chapel}
\begin{verbatim}
  var A: [1..5] int;
  assignCounter(A);
\end{verbatim}
\end{chapel}

The assignCounter call uses both the default argument for counter as well
as promotion. When these features are combined, the default argument
will be evaluated once per promoted element. As a result, after this
command, A will contain the elements 0 1 2 3 4 in some order.

\begin{chapelnoprint}
\begin{verbatim}
writeln(A.sorted());
\end{verbatim}
\end{chapelnoprint}
\begin{chapeloutput}
\begin{verbatim}
0 1 2 3 4
\end{verbatim}
\end{chapeloutput}
\end{chapelexample}


\subsection{Zipper Promotion}
\label{Zipper_Promotion}
\index{promotion!zipper iteration}

Promotion also supports zippered iteration semantics as described
in~\rsec{Zipper_Iteration} and~\rsec{Parallel_Iterators} for parallel
iteration.

Consider a function \chpl{f} with formal
arguments \chpl{s1}, \chpl{s2},~... that are promoted and formal
arguments \chpl{a1}, \chpl{a2},~... that are not promoted.  The call
\begin{chapel}
\begin{verbatim}
f(s1, s2, ..., a1, a2, ...)
\end{verbatim}
\end{chapel}
is equivalent to
\begin{chapel}
\begin{verbatim}
[(e1, e2, ...) in zip(s1, s2, ...)] f(e1, e2, ..., a1, a2, ...)
\end{verbatim}
\end{chapel}
The usual constraints of zipper iteration apply to zipper promotion so
the promoted actuals must have the same shape.

A zipper promotion can be captured in a variable, such as
\chpl{var X = f(s1, s2, ..., a1, a2, ...);} using the above example.
If so, the domain of the resulting array is defined by the first argument
that causes promotion. The rules are the same as in the non-zipper case.


\begin{chapelexample}{zipper-promotion.chpl}
Given a function defined as
\begin{chapel}
\begin{verbatim}
proc foo(i: int, j: int) {
  return (i,j);
}
\end{verbatim}
\end{chapel}
and a call to this function written
\begin{chapel}
\begin{verbatim}
writeln(foo(1..3, 4..6));
\end{verbatim}
\end{chapel}
then the output is
\begin{chapelprintoutput}
\begin{verbatim}
(1, 4) (2, 5) (3, 6)
\end{verbatim}
\end{chapelprintoutput}
\end{chapelexample}


\subsection{Whole Array Operations and Evaluation Order}
\label{Whole_Array_Operations}
\index{whole array assignment}
\index{whole array operations}
\index{arrays!assignment}
\index{assignment!whole array}
\index{data parallelism!evaluation order}

Whole array operations are a form of promotion as applied to operators
rather than functions.

Whole array assignment is one example. It is is implicitly parallel.
The array assignment statement:
\begin{chapel}
\begin{verbatim}
LHS = RHS;
\end{verbatim}
\end{chapel}
is equivalent to
\begin{chapel}
\begin{verbatim}
forall (e1,e2) in zip(LHS,RHS) do
  e1 = e2;
\end{verbatim}
\end{chapel}

The semantics of whole array assignment and promotion are different
from most array programming languages.  Specifically, the compiler
does not insert array temporaries for such operations if any of the
right-hand side array expressions alias the left-hand side expression.

%
% sungeun 4/8/2011
% Did not convert this one due to non-deterministic output
%
\begin{example}
If \chpl{A} is an array declared over the indices \chpl{1..5}, then
the following codes are not equivalent:
\begin{chapel}
\begin{verbatim}
A[2..4] = A[1..3] + A[3..5];
\end{verbatim}
\end{chapel}
and
\begin{chapel}
\begin{verbatim}
var T = A[1..3] + A[3..5];
A[2..4] = T;
\end{verbatim}
\end{chapel}
This follows because, in the former code, some of the new values that
are assigned to \chpl{A} may be read to compute the sum depending on
the number of tasks used to implement the data parallel statement.
\end{example}



\section{Reductions and Scans}
\label{Reductions_and_Scans}
\index{reductions}
\index{scans}
\index{data parallelism!reductions}
\index{data parallelism!scans}

Chapel provides reduction and scan expressions that apply operators to
aggregate expressions in stylized ways.  Reduction expressions
collapse the aggregate's values down to a summary value.  Scan
expressions compute an aggregate of results where each result value
stores the result of a reduction applied to all of the elements in the
aggregate up to that expression.  Chapel provides a number of predefined
reduction and scan operators, and also supports a mechanism for the
user to define additional reductions and
scans (Chapter~\ref{User_Defined_Reductions_and_Scans}).

\subsection{Reduction Expressions}
\label{reduce}
\index{reduction expressions}
\index{expressions!reduction}

A reduction expression applies a reduction operator to an aggregate
expression, collapsing the aggregate's dimensions down into a result
value (typically a scalar or summary expression that is independent of
the input aggregate's size).  For example, a sum reduction computes
the sum of all the elements in the input aggregate expression.

The syntax for a reduction expression is given by:
\begin{syntax}
\begin{verbatim}
reduce-expression:
  reduce-scan-operator `reduce' iteratable-expression
  class-type `reduce' iteratable-expression

reduce-scan-operator: one of
  + * && || & | ^ `min' `max' `minloc' `maxloc'
\end{verbatim}
\end{syntax}

Chapel's predefined reduction operators are defined
by \sntx{reduce-scan-operator} above.  In order, they are: sum,
product, logical-and, logical-or, bitwise-and, bitwise-or,
bitwise-exclusive-or, minimum, maximum, minimum-with-location, and
maximum-with-location.  The minimum reduction returns the minimum
value as defined by the \verb@<@ operator.  The maximum reduction
returns the maximum value as defined by the \verb@>@ operator.  The
minimum-with-location reduction returns the lowest index position with
the minimum value (as defined by the \verb@<@ operator).  The
maximum-with-location reduction returns the lowest index position with
the maximum value (as defined by the \verb@>@ operator).
When a minimum, maximum, minimum-with-location, or maximum-with-location
reduction encounters a NaN, the result is a NaN.

The expression on the right-hand side of the \chpl{reduce} keyword
can be of any type that can be iterated over, provided
the reduction operator can be applied to the values yielded
by the iteration. For example, the bitwise-and
operator can be applied to arrays of boolean or integral types to
compute the bitwise-and of all the values in the array.

For the minimum-with-location and maximum-with-location reductions,
the argument on the right-hand side of the \chpl{reduce} keyword
must be a 2-tuple. Its first component is the collection
of values for which the minimum/maximum value is to be computed.  The
second argument component is a collection of indices with the same size and
shape that provides names for the locations of the values in the first
component.  The reduction returns a tuple containing the
minimum/maximum value in the first argument component and the value
at the corresponding location in the second argument component.

\begin{chapelexample}{reduce-loc.chpl}
The first line below computes the smallest element in an array
\chpl{A} as well as its index, storing the results in \chpl{minA} and
\chpl{minALoc}, respectively.  It then computes the largest element in
a forall expression making calls to a function \chpl{foo()}, storing
the value and its number in \chpl{maxVal} and \chpl{maxValNum}.
\begin{chapelnoprint}
\begin{verbatim}
config const n = 10;
const D = {1..n};
var A: [D] int = [i in D] i % 7;
proc foo(x) return x % 7;
\end{verbatim}
\end{chapelnoprint}
\begin{chapel}
\begin{verbatim}
var (minA, minALoc) = minloc reduce zip(A, A.domain); 
var (maxVal, maxValNum) = maxloc reduce zip([i in 1..n] foo(i), 1..n);
\end{verbatim}
\end{chapel}
\begin{chapelnoprint}
\begin{verbatim}
writeln((minA, minALoc));
writeln((maxVal, maxValNum));
\end{verbatim}
\end{chapelnoprint}
\begin{chapeloutput}
\begin{verbatim}
(0, 7)
(6, 6)
\end{verbatim}
\end{chapeloutput}
\end{chapelexample}

User-defined reductions are specified by preceding the
keyword \chpl{reduce} by the class type that implements the reduction
interface as described in~\rsec{User_Defined_Reductions_and_Scans}.

\subsection{Scan Expressions}
\label{scan}
\index{scan expressions}
\index{expressions!scan}

A scan expression applies a scan operator to an aggregate expression,
resulting in an aggregate expression of the same size and shape.  The
output values represent the result of the operator applied to all
elements up to and including the corresponding element in the input.

The syntax for a scan expression is given by:
\begin{syntax}
\begin{verbatim}
scan-expression:
  reduce-scan-operator `scan' iteratable-expression
  class-type `scan' iteratable-expression
\end{verbatim}
\end{syntax}

The predefined scans are defined by \sntx{reduce-scan-operator}.  These
are identical to the predefined reductions and are described
in~\rsec{reduce}.

The expression on the right-hand side of the scan can be of any type
that can be iterated over and to which the operator can be applied.

%
% sungeun: 4/8/2011
% Did not convert this one yet due to warning about serializing scans
%
\begin{example}
Given an array
\begin{chapel}
\begin{verbatim}
var A: [1..3] int = 1;
\end{verbatim}
\end{chapel}
that is initialized such that each element contains one, then the code
\begin{chapel}
\begin{verbatim}
writeln(+ scan A);
\end{verbatim}
\end{chapel}
outputs the results of scanning the array with the sum operator.  The
output is
\begin{chapelprintoutput}
\begin{verbatim}
1 2 3
\end{verbatim}
\end{chapelprintoutput}
\end{example}

User-defined scans are specified by preceding the keyword \chpl{scan}
by the class type that implements the scan interface as described
in Chapter~\ref{User_Defined_Reductions_and_Scans}.

\section{Configuration Constants for Default Data Parallelism}
\label{data_parallel_knobs}
\index{data parallelism!knobs for default data parallelism}
\index{data parallelism!configuration constants}
\index{dataParTasksPerLocale@\chpl{dataParTasksPerLocale}}
\index{dataParIgnoreRunningTasks@\chpl{dataParIgnoreRunningTasks}}
\index{dataParMinGranularity@\chpl{dataParMinGranularity}}

The following configuration constants are provided to control the
degree of data parallelism over ranges, default domains, and default
arrays:

\begin{center}
\begin{tabular}{|l|l|l|}
\hline
{\bf Config Const} & {\bf Type} & {\bf Default} \\
\hline
\chpl{dataParTasksPerLocale} & \chpl{int} &
top level \chpl{.maxTaskPar}~(see~\rsec{Locale_Methods}) \\
\chpl{dataParIgnoreRunningTasks} & \chpl{bool} & \chpl{true} \\
\chpl{dataParMinGranularity} & \chpl{int} & \chpl{1} \\
\hline
\end{tabular}
\end{center}

The configuration constant \chpl{dataParTasksPerLocale} specifies the
number of tasks to use when executing a forall loop over a range,
default domain, or default array.  The actual number of tasks may be
fewer depending on the other two configuration constants.  A value of
zero results in using the default value.

The configuration constant \chpl{dataParIgnoreRunningTasks}, when
true, has no effect on the number of tasks to use to execute the
forall loop.  When false, the number of tasks per locale is decreased
by the number of tasks that are already running on the locale, with a
minimum value of one.

The configuration constant \chpl{dataParMinGranularity} specifies the
minimum number of iterations per task created.  The number of tasks is
decreased so that the number of iterations per task is never less than
the specified value.

For distributed domains and arrays that have these same configuration
constants (\eg, Block and Cyclic distributions), these same
module level configuration constants are used to specify their
default behavior within each locale.

\cleardoublepage
This is a stub.  This portion of the document does not exist.

\cleardoublepage
\sekshun{Domain Maps}
\label{Domain_Maps}
\index{domain maps}

A mapping from domain index values to locales is called a \emph{domain
map}.

\subsection{Domain Map Types}
\label{Domain_Map_Types}

Distribution types are defined by the type of a distribution class
that derives from BaseDist and is typically generic.  They are
distinct from the distribution class type.  Typically, the
distribution class type is only used on its own in defining the
distribution itself~\rsec{User_Defined_Domain_Maps}

Defining a distribution type involves specifying a distribution class
type and wrapping this in a distribution type \chpl{dist}.
\begin{example}
The code
\begin{chapel}
use BlockDist;
var MyBlockDist: dist(Block(rank=2));
\end{chapel}
defines a Block distribution called \chpl{MyBlockDist} with rank 2 and
a default index type that can be used to distribute 2-dimensional
arithmetic domains.  The Block distribution is described in more
detail in~\rsec{Block_Dist}.
\end{example}

\subsection{Domain Map Values}
\label{Domain_Map_Values}

Constructing a distribution involves calling the constructor of a
distribution class and defining a new distribution type \chpl{dist}.
\begin{example}
The code
\begin{chapel}
use BlockDist;
var MyBlockDist = new dist(new Block(rank=2, bbox=[1..n,1..n]));
\end{chapel}
constructs a Block distribution that partitions the index space
specified by \chpl{[1..n, 1..n]} over all of the locales.  The Block
distribution is described in more detail in~\rsec{Block_Dist}.
\end{example}

\subsection{Mapped Domains and Arrays}
\label{Mapped_Domains_and_Arrays}

\index{domains!distributed}
A domain for which a distribution is specified is referred to as a
{\em distributed domain}.

\begin{rationale}
Should this be? A domain supports a method, \chpl{locale}, that maps
index values in the domain to locales that correspond to the domain's
distribution.
\end{rationale}

The syntax to create a distributed domain type is the same as the
syntax to create a distributed domain value:
\begin{syntax}
distributed-domain-type:
  domain-type `distributed' distribution-expression

distributed-domain-expression:
  domain-expression `distributed' distribution-expression

distribution-expression:
  expression
\end{syntax}

\begin{example}
The code
\begin{chapel}
use BlockDist;
var MyBlockDist = new dist(new Block(rank=2, bbox=[1..n,1..n]));
var Dom: domain(2) distributed MyBlockDist =
           [1..n, 1..n] distributed MyBlockDist;
\end{chapel}
defines a new domain that is distributed by \chpl{MyBlockDist}.  Note
that, as usual, the type does not need to be specified if the variable
is initialized.
\end{example}

When defining a new distribution inline with the \chpl{distributed}
keyword, a syntactic sugar is supported in which the ``new dist new''
characters may be omitted.
\begin{example}
The code
\begin{chapel}
use BlockDist;
var D = [1..n, 1..n] distributed new dist(new Block(rank=2, bbox=[1..n,1..n]));
\end{chapel}
is equivalent to
\begin{chapel}
use BlockDist;
var D = [1..n, 1..n] distributed Block(rank=2, bbox=[1..n,1..n]);
\end{chapel}
\end{example}

Iteration over a distributed domain implicitly executes the controlled
task in the domain of the associated locale.  Similarly, when
iterating over the elements of an array defined over a distributed
domain, the controlled tasks are determined by the distribution of the
domain.  If there are conflicting distributions in product iterations,
the locale of a task is taken to be the first component in the
product.

\begin{example}
If \chpl{D} is a distributed domain, then in the code
\begin{chapel}
forall d in D {
  // body
}
\end{chapel}
the body of the loop is executed in the locale where the
index \chpl{d} maps to by the distribution of \chpl{D}.
\end{example}

\subsection{Undistributed Domains and Arrays}
\label{Undistributed_Domains_and_Arrays}

If a domain or an array does not have a distributed part, the domain
or array is not distributed and exists only on the locale on which it
is defined.

\cleardoublepage
\sekshun{User-Defined Reductions and Scans}
\label{User_Defined_Reductions_and_Scans}

User-defined reductions and scans are supported via class definitions
where the class implements a structural interface.  The definition of
this structural interface is forthcoming.  The following paper
sketched out such an interface:
\begin{quote}
S.~J.~Deitz, D.~Callahan, B.~L.~Chamberlain, and L.~Snyder.  {\bf
Global-view abstractions for user-defined reductions and scans}.  In
{\it Proceedings of the Eleventh ACM SIGPLAN Symposium on Principles
and Practice of Parallel Programming}, 2006.
\end{quote}

\cleardoublepage
This is a stub.  This portion of the document does not exist.

\cleardoublepage
\sekshun{Interoperability}
\label{Interoperability}
\index{interoperability}

Chapel's interoperability features support cooperation between Chapel
and other languages.  They provide the ability to create software
systems that incorporate both Chapel and non-Chapel components.
Thus, they support the reuse of existing software components while
leveraging the unique features of the Chapel language.

Interoperability can be broken down in terms of the exchange of types, variables
and procedures, and whether these are imported or exported.  An overview of
procedure importing and exporting is provided in~\rsec{Interop_Overview}.
Details on sharing types, variables and procedures are supplied
in \rsec{Shared_Language_Elements}.
%  The creation and use of Chapel libraries is
%treated in~\rsec{Interop_Libraries}.

\begin{future}

At present, the backend language for Chapel is C, which makes it relatively
easy to call C libraries from Chapel and vice versa.  To support a variety of
platforms without requiring recompilation, it may be desirable to move
to an intermediate-language model.

In that case, each supported platform must minimally support that virtual
machine.  However, in addition to increased portability, a virtual machine
model may expose elements of the underlying machine's programming model
(hardware task queues, automated garbage collection, etc.) that are not easily
rendered in C.  In addition, the virtual machine model can support run-time task
migration.

\end{future}

The remainder of this chapter documents Chapel support of interoperability through
the existing C-language backend.

\section{Interoperability Overview}
\label{Interop_Overview}
\index{interoperability!overview}

The following two subsections provide an overview of calling externally-defined
(C) routines in Chapel, and setting up Chapel routines so they can be called
from external (C) code.

\subsection{Calling External Functions}
\label{Calling_External_Functions}
\index{interoperability!external functions!calling}

To use an external function in a Chapel program, it is necessary to inform the
Chapel compiler of that routine's signature through an external function
declaration.  This permits Chapel to bind calls to that function signature
during function resolution.  The user must also supply a definition for the
referenced function by naming a C source file, an object file or an object
library on the \chpl{chpl} command line. 

An external procedure declaration has the following syntax:
\begin{syntax}
external-procedure-declaration-statement:
  `extern' external-name[OPT] `proc' function-name argument-list return-intent[OPT] return-type[OPT]
\end{syntax}

Chapel code will call the external function using the parameter types supplied
in the \chpl{extern} declaration. Therefore, in general, the type of each
argument in the supplied \sntx{argument-list} must be the Chapel
equivalent of the corresponding external type.

The return value of the function can be used by Chapel only if its type is
declared using the optional \sntx{return-type} specifier.  If it is omitted,
Chapel assumes that no value is returned, or equivalently that the function
returns \chpl{void}.

It is possible to use the \sntx{external-name} syntax to create an
\chpl{extern} function that presents a different name to Chapel code than
the name of the function actually used when linking. The
\sntx{external-name} expression must evaluate to a \chpl{param}
\chpl{string}. For example, the code below declares a function callable
in Chapel as \chpl{c_atoi} but that will actually link with the C
\chpl{atoi} function.

\begin{chapel}
  extern "atoi" proc c_atoi(arg:c_string):c_int;
\end{chapel}

At present, external iterators are not supported.  

\begin{future}
The overloading of function names is
also not supported directly in the compiler.  However, one can use
the \sntx{external-name} syntax to supply a name to be used by the linker.  In
this way, function overloading can be implemented ``by hand''.  This syntax also
supports calling external C++ routines: The \sntx{external-name} to use is the
mangled function name generated by the external compilation
environment\footnote{In UNIX-like programming environments, \chpl{nm} and \chpl{grep}
can be used to find the mangled name of a given function within an object file
or object library.}.
\end{future}

\begin{future}
Dynamic dispatch (polymorphism) is also unsupported in this version.  But this
is not ruled out in future versions.  Since Chapel already supports type-based
procedure declaration and resolution, it is a small step to translate a
type-relative extern method declaration into a virtual method table entry.  The
mangled name of the correct external function must be supplied for each
polymorphic type available.  However, most likely the generation of \chpl{.chpl}
header files from C and C++ libraries can be fully automated.
\end{future}

There are three ways to supply to the Chapel compiler the definition of an
external function: as a C source file (\chpl{.c} or \chpl{.h}), as an object
file and as an object library.  It is platform-dependent whether static
libraries (archives), dynamic libraries or both are supported.  See
the \chpl{chpl} man page for more information on how these file types are handled.

\subsection{Calling Chapel Functions}
\label{Calling_Chapel_Functions}
\index{interoperability!Chapel functions!calling}

To call a Chapel procedure from external code, it is necessary to expose the
corresponding function symbol to the linker.  This is done by adding
the \chpl{export} linkage specifier to the function definition.
The \chpl{export} specifier ensures that the corresponding procedure will be
resolved, even if it is not called within the Chapel program or library being
compiled.

An exported procedure declaration has the following syntax:
\begin{syntax}
exported-procedure-declaration-statement:
  `export' external-name[OPT] `proc' function-name argument-list return-intent[OPT] return-type[OPT]
    function-body

external-name:
  expression
\end{syntax}

The rest of the procedure declaration is the same as for a non-exported
function.  An exported procedure can be called from within Chapel as
well. Currently, iterators cannot be exported.

As with the \sntx{extern-name} for \chpl{extern} \chpl{proc}, if this
syntax element is provided, then it is a \chpl{param} \chpl{string}
determining the name of the function to use when linking. For example,
the code below declares a function callable in C as \chpl{chapel_addone}
but it is callable from Chapel code as \chpl{addone}:

\begin{chapel}
  export "chapel_addone" proc addone(arg:c_int):c_int {
    return arg+1;
  }
\end{chapel}

\begin{future}
Currently, exported functions cannot have generic, \chpl{param} or type arguments.
This is because such functions actually represent a family of functions,
specific versions of which are instantiated as need during function resolution.

Instantiating all possible versions of a template function is not
practical in general.  However, if explicit instantiation were supported in
Chapel, an explicit instantiation with the export linkage specifier would
clearly indicate that the matching template function was to be instantiated with
the given \chpl{param} values and argument types.
\end{future}

\section{Shared Language Elements}
\label{Shared_Language_Elements}
\index{interoperability!sharing}

This section provides details on how to share Chapel types, variables and
procedures with external code.  It is written assuming that the intermediate
language is C.

\subsection{Shared Types}

This subsection discusses how specific types are shared between Chapel and
external code.

\subsubsection{Referring to Standard C Types}
\label{Referring_to_Standard_C_Types}
\index{interoperability!standard C types}
\index{interoperability!C types!standard}

In Chapel code, all standard C types must be expressed in terms of their Chapel
equivalents.  This is true, whether the entity is exported, imported or private.
Standard C types and their corresponding Chapel types are shown in the following
table.

\begin{tabular}{rlrlrl}
C Type & Chapel Type & C Type & Chapel Type & C Type & Chapel Type \\
\hline
\tt int8\_t  & \tt int(8)  & \tt uint8\_t  & \tt uint(8)  & \tt \_real32 & \tt real(32) \\
\tt int16\_t & \tt int(16) & \tt uint16\_t & \tt uint(16) & \tt \_real64 & \tt real(64) \\
\tt int32\_t & \tt int(32) & \tt uint32\_t & \tt uint(32) & \tt \_imag32 & \tt imag(32) \\
\tt int64\_t & \tt int(64) & \tt uint64\_t & \tt uint(64) & \tt \_imag64 & \tt imag(64) \\
\tt chpl\_bool & \tt bool & \tt  const char* & \tt c\_string \\
\tt \_complex64 & \tt complex(64) & \tt \_complex128 & \tt complex(128) \\
\end{tabular}

Standard C types are built-in.  Their Chapel equivalents do not have to be
declared using the \chpl{extern} keyword.

In C, the ``colloquial'' integer type names \chpl{char}, \chpl{signed
char}, \chpl{unsigned char}, (\chpl{signed}) \chpl{short}
(\chpl{int}), \chpl{unsigned short} (\chpl{int}),
(\chpl{signed}) \chpl{int}, \chpl{unsigned int},
(\chpl{signed}) \chpl{long} (\chpl{int}), \chpl{unsigned long} (\chpl{int}), (\chpl{signed}) \chpl{long
long} (\chpl{int}) and \chpl{unsigned long long} (\chpl{int}) may have an
implementation-defined width.\footnote{However, most implementations have settled
on using 8, 16, 32, and 64 bits (respectively) to
represent \chpl{char}, \chpl{short}, \chpl{int} and \chpl{long}, and \chpl{long
long} types}.  When referring to C types in a Chapel program, the burden of
making sure the type sizes agree is on the user.  A Chapel implementation must
ensure that all of the C equivalents in the above table are defined and have the
correct representation with respect to the corresponding Chapel type.

\subsubsection{Referring to External C Types}
\label{Referring_to_External_C_Types}
\index{interoperability!external C types}
\index{interoperability!C types!external}

An externally-defined type can be referenced using a external type declaration
with the following syntax.
\begin{syntax}
external-type-alias-declaration-statement:
  `extern' `type' type-alias-declaration-list ;
\end{syntax}

In each \sntx{type-alias-declaration}, if the \sntx{type-specifier} part is
supplied, then Chapel uses the supplied type specifier internally.  Otherwise,
it treats the named type as an opaque type.  The definition for an external type
must be supplied by a C header file named on the \chpl{chpl} command line.

Fixed-size C array types can be described within Chapel using its
homogeneous tuple type.  For example, the C typedef
\begin{chapel}
typedef double vec[3];
\end{chapel}
can be described in Chapel using
\begin{chapel}
extern type vec = 3*real(64);
\end{chapel}


\subsubsection{Referring to External C Structs}
\label{Referring_to_External_C_Structs}
\index{interoperability!C structs!external}

External C struct types can be referred to within Chapel by prefixing
a Chapel \chpl{record} definition with the \chpl{extern} keyword.
\begin{syntax}
external-record-declaration-statement:
  `extern' external-name[OPT] simple-record-declaration-statement
\end{syntax}

For example, consider an external C structure defined in \chpl{foo.h} called \chpl{fltdbl}.
\begin{chapel}
    typedef struct _fltdbl {
      float x;
      double y;
    } fltdbl;
\end{chapel}
This type could be referred to within a Chapel program using
\begin{chapel}
   extern record fltdbl {
     var x: real(32);
     var y: real(64);
   }
\end{chapel}
\noindent
and defined by supplying \chpl{foo.h} on the \chpl{chpl} command line.

Within the Chapel declaration, some or all of the fields from the C
structure may be omitted.  The order of these fields need not match
the order they were specified within the C code.  Any fields that are
not specified (or that cannot be specified because there is no
equivalent Chapel type) cannot be referenced within the Chapel code.  Some
effort is made to preserve the values of the undefined fields when copying
these structs but Chapel cannot guarantee the contents or memory story of
fields of which it has no knowledge.

If the optional \sntx{external-name} is supplied, then it is used verbatim as
the exported struct symbol.

A C header file containing the struct's definition in C must be specified on the
chpl compiler command line.  Note that only typdef'd C structures are supported
by default.  That is, in the C header file, the \chpl{struct} must be supplied
with a type name through a \chpl{typedef} declaration. If this is not true, you
can use the \sntx{external-name} part to apply the \chpl{struct} specifier.
As an example of this, given a C declaration of:

% Not a spec test, we would need to be able to provide .c and .h files
\begin{chapel}
  struct Vec3 {
    double x, y, z;
  };
\end{chapel}

in Chapel you would refer to this \chpl{struct} via

\begin{chapel}
  extern "struct Vec3" record Vec3 {
    var x, y, z: real(64);
  }
\end{chapel}


\subsubsection{Referring to External Structs Through Pointers}
\label{Referring_to_External_Structs_Through_Pointers}
\index{interoperability!C Structs!external!pointers to}

An external type which is a pointer to a \chpl{struct} can be referred to from
Chapel using an external \chpl{class} declaration.  External class declarations
have the following syntax.
\begin{syntax}
external-class-declaration-statement:
  `extern' external-name[OPT] simple-class-declaration-statement
\end{syntax}
External class declarations are similar to external record declarations as
discussed above, but place additional requirements on the C code.

For example, given the declaration
\begin{chapel}
  extern class D {
    var x: real;
  }
\end{chapel}
\noindent
the requirements on the corresponding C code are:
\begin{enumerate}
\item There must be a struct type that is typedef'd to have the name \chpl{\_D}.
\item A pointer-to-\chpl{\_D} type must be typedef'd to have the name \chpl{D}.
\item The \chpl{\_D} struct type must contain a field named \chpl{x} of type \chpl{double}.
\end{enumerate}
\noindent
Like external records/structs, it may also contain other fields
that will simply be ignored by the Chapel compiler.

The following C typedef would fulfill the external Chapel class
declaration shown above:
\begin{chapel}
   typedef struct __D {
     double x;
     int y;
   } _D, *D;
\end{chapel}
where the Chapel compiler would not know about the 'y' field and
therefore could not refer to it or manipulate it.

If the optional \sntx{external-name} is supplied, then it is used verbatim as
the exported class symbol.


\subsubsection{Opaque Types}
\label{Opaque_Types}
\index{interoperability!opaque types}

It is possible refer to external pointer-based C types that cannot be
described in Chapel by using the "opaque" keyword.  As the name implies,
these types are opaque as far as Chapel is concerned and cannot be
used for operations other than argument passing and assignment.

For example, Chapel could be used to call an external C function that
returns a pointer to a structure (that can't or won't be described as
an external class) as follows:
\begin{chapel}
    extern proc returnStructPtr(): opaque;

    var structPtr: opaque = returnStructPtr();
\end{chapel}

However, because the type of \chpl{structPtr} is opaque, it can be used only in
assignments and the arguments of functions expecting the same underlying type.
\begin{chapel}
    var copyOfStructPtr = structPtr;

    extern proc operateOnStructPtr(ptr: opaque);
    operateOnStructPtr(structPtr);
\end{chapel}
\noindent
Like a \chpl{void*} in C, Chapel's \chpl{opaque} carries no information
regarding the underlying type.  It therefore subverts type safety, and should be
used with caution.

\subsection{Shared Data}
\label{Shared_Data}
\index{interoperability!shared data}

This subsection discusses how to access external variables and constants.

A C variable or constant can be referred to within Chapel by prefixing
its declaration with the extern keyword.  For example:
\begin{chapel}
    extern var bar: foo;
\end{chapel}
\noindent
would tell the Chapel compiler about an external C variable named
\chpl{bar} of type \chpl{foo}.  Similarly,
\begin{chapel}
   extern const baz: int(32);
\end{chapel}
would refer to an external 32-bit integer constant named \chpl{baz} in the
C code.  In practice, external consts can be used to provide Chapel
definitions for \#defines and enum symbols in addition to traditional C
constants.

\begin{craychapel}
Note that since params must be known to Chapel at compile-time and
the Chapel compiler does not necessarily parse C code,
external params are not supported.
\end{craychapel}

\subsection{Shared Procedures}
\label{Shared_Procedures}
\index{interoperability!shared procedures}

This subsection provides additional detail and examples for calling external
procedures from Chapel and for exporting Chapel functions for external use.

\subsubsection{Calling External C Functions}
\label{Calling_External_C_Functions}
\index{interoperability!external functions!calling}

To call an external C function, a prototype of
the routine must appear in the Chapel code.  This is accomplished by providing
the Chapel signature of the function preceded by the \chpl{extern} keyword.  For
example, for a C function foo() that takes no arguments and returns
nothing, the prototype would be:
\begin{chapel}
       extern proc foo();
\end{chapel}

To refer to the return value of a C function, its type must be supplied through
a \sntx{return-type} clause in the prototype.\footnote{The return type cannot be
inferred, since an \chpl{extern} procedure declaration has no body.}

If the above function returns a C \chpl{double}, it would be declared as:
\begin{chapel}
       extern proc foo(): real;
\end{chapel}
Similarly, for external functions that expect arguments, the types of those
arguments types may be declared in Chapel using explicit argument type specifiers.

The types of function arguments may be omitted from the external procedure
declaration, in which case they are inferred based on the Chapel callsite.
For example, the Chapel code
\begin{chapel}
       extern proc foo(x: int, y): real;
       var a, b: int;
       foo(a, b);
\end{chapel}
\noindent
would imply that the external function foo takes two 64-bit integer values
and returns a 64-bit real.  External function declarations with omitted type
arguments can also be used call external C macros.

External function arguments can be declared using the \sntx{default-expression}
syntax.  In this case, the default argument will be supplied by the Chapel
compiler if the corresponding actual argument is omitted at the callsite.  For example:
\begin{chapel}
       extern proc foo(x: int, y = 1.2): real;
       foo(0);
\end{chapel}
Would cause external function foo() to be invoked with the arguments 0
and 1.2.

C varargs functions can be declared using
Chapel's \sntx{variable-argument-expression} syntax (\chpl{...}).  For example,
the C \chpl{printf} function can be declared in Chapel as
\begin{chapel}
       extern proc printf(fmt: c_string, vals...?numvals): int;
\end{chapel}

External C functions or macros that accept type arguments can also be
prototyped in Chapel by declaring the argument as a type.  For
example:
\begin{chapel}
       extern foo(type t);
\end{chapel}
Calling such a routine with a Chapel type will cause the type
identifier (e.g., 'int') to be passed to the routine.\footnote{In practice,
this will typically only be useful if the external function is a macro
or built-in that can handle type identifiers.}

\subsection{Calling Chapel Procedures Externally}
\label{Calling_Chapel_Procedures_Externally}
\index{interoperability!Chapel procedures!calling}

To call a Chapel procedure from external code, the procedure name must be
exported using the \chpl{export} keyword.  An exported procedure taking no
arguments and returning void can be declared as:
\begin{chapel}
export proc foo();
\end{chapel}
If the procedure body is omitted, the procedure declaration is a prototype; the
body of the procedure must be supplied elsewhere.  In a prototype, the return
type must be declared; otherwise, it is assumed to be \chpl{void}.  If the body
is supplied, the return type of the exported procedure is inferred from the
type of its return expression(s).

If the optional \sntx{external-name} is supplied, that is the name used in
linking with external code.  For example, if we declare
\begin{chapel}
export "myModule_foo" proc foo();
\end{chapel}
\noindent
then the name \chpl{foo} is used to refer to the procedure within chapel code,
whereas a call to the same function from C code would appear
as \chpl{myModule_foo();}.  If the external name is omitted, then its internal
name is also used externally.

When a procedure is exported, all of the types and functions on which it depends
are also exported.  Iterators cannot be explicitly exported.  However, they are
inlined in Chapel code which uses them, so they are exported in effect.

\subsection{Argument Passing}
\label{Interop_Argument_Passing}
\index{interoperability!argument passing}

The manner in which arguments are passed to an external function can be
controlled using argument intents.  The following table shows the correspondence
between Chapel intents and C argument type declarations.  These correspondences
pertain to both imported and exported function signatures.

\begin{tabular}{rl}
Chapel & C \\
\hline
\tt T & \tt const T \\
\tt in T & \tt T \\
\tt inout T & \tt T* \\
\tt out T & \tt T* \\
\tt ref T & \tt T* \\
\tt param & \tt \\
\tt type & \tt char*\\
\end{tabular}

Currently, \chpl{param} arguments are not allowed in an extern function
declaration, and \chpl{type} args are passed as a string containing the name of
the actual type being passed.  Note that the level of indirection is changed
when passing arguments to a C function using \chpl{inout}, \chpl{out},
or \chpl{ref} intent.  The C code implementing that function must dereference
the argument to extract its value.

\cleardoublepage
\appendix
%%
%% Do not modify this file.  This file is automatically
%% generated by collect_syntax.pl.
%%

\sekshun{Collected Lexical and Syntax Productions}
\label{Syntax}

This appendix collects the syntax productions listed throughout the specification.  There are no new syntax productions in this appendix.  The productions are listed both alphabetically and in depth-first order for convenience.

\section{Alphabetical Lexical Productions}

\begin{syntax}
\end{syntax}

\begin{syntax}
binary-digit: one of
  `0' `1'
\end{syntax}

\begin{syntax}
binary-digits:
  binary-digit
  binary-digit binary-digits
\end{syntax}

\begin{syntax}
bool-literal: one of
  `true' $ $ $ $ `false'
\end{syntax}

\begin{syntax}
digit: one of
  `0' `1' `2' `3' `4' `5' `6' `7' `8' `9'
\end{syntax}

\begin{syntax}
digits:
  digit
  digit digits
\end{syntax}

\begin{syntax}
double-quote-delimited-characters:
  string-character double-quote-delimited-characters[OPT]
  ' double-quote-delimited-characters[OPT]
\end{syntax}

\begin{syntax}
exponent-part:
  `e' sign[OPT] digits
  `E' sign[OPT] digits
\end{syntax}

\begin{syntax}
hexadecimal-digit: one of
  `0' `1' `2' `3' `4' `5' `6' `7' `8' `9' `A' `B' `C' `D' `E' `F' `a' `b' `c' `d' `e' `f'
\end{syntax}

\begin{syntax}
hexadecimal-digits:
  hexadecimal-digit
  hexadecimal-digit hexadecimal-digits
\end{syntax}

\begin{syntax}
hexadecimal-escape-character:
  `$\backslash$x' hexadecimal-digits
\end{syntax}

\begin{syntax}
identifier:
  letter-or-underscore legal-identifier-chars[OPT]
\end{syntax}

\begin{syntax}
imaginary-literal:
  real-literal `i'
  integer-literal `i'
\end{syntax}

\begin{syntax}
integer-literal:
  digits
  `0x' hexadecimal-digits
  `0X' hexadecimal-digits
  `0o' octal-digits
  `0O' octal-digits
  `0b' binary-digits
  `0B' binary-digits
\end{syntax}

\begin{syntax}
legal-identifier-char:
  letter-or-underscore
  digit
  `(*\texttt{\$}*)'
\end{syntax}

\begin{syntax}
legal-identifier-chars:
  legal-identifier-char legal-identifier-chars[OPT]
\end{syntax}

\begin{syntax}
letter-or-underscore:
  letter
  `_'
\end{syntax}

\begin{syntax}
letter: one of
  `A' `B' `C' `D' `E' `F' `G' `H' `I' `J' `K' `L' `M' `N' `O' `P' `Q' `R' `S' `T' `U' `V' `W' `X' `Y' `Z'
  `a' `b' `c' `d' `e' `f' `g' `h' `i' `j' `k' `l' `m' `n' `o' `p' `q' `r' `s' `t' `u' `v' `w' `x' `y' `z'
\end{syntax}

\begin{syntax}
octal-digit: one of
  `0' `1' `2' `3' `4' `5' `6' `7'
\end{syntax}

\begin{syntax}
octal-digits:
  octal-digit
  octal-digit octal-digits
\end{syntax}

\begin{syntax}
p-exponent-part:
  `p' sign[OPT] digits
  `P' sign[OPT] digits
\end{syntax}

\begin{syntax}
real-literal:
  digits[OPT] . digits exponent-part[OPT]
  digits .[OPT] exponent-part
  `0x' hexadecimal-digits[OPT] . hexadecimal-digits p-exponent-part[OPT]
  `0X' hexadecimal-digits[OPT] . hexadecimal-digits p-exponent-part[OPT]
  `0x' hexadecimal-digits .[OPT] p-exponent-part
  `0X' hexadecimal-digits .[OPT] p-exponent-part
\end{syntax}

\begin{syntax}
sign: one of
  + $ $ $ $ -
\end{syntax}

\begin{syntax}
simple-escape-character: one of
  `$\backslash\mbox{\bf '}\hspace{5pt}$' `$\backslash$"$\hspace{5pt}$' `$\backslash$?$\hspace{5pt}$' `$\backslash$$\backslash$$\hspace{5pt}$' `$\backslash$a$\hspace{5pt}$' `$\backslash$b$\hspace{5pt}$' `$\backslash$f$\hspace{5pt}$' `$\backslash$n$\hspace{5pt}$' `$\backslash$r$\hspace{5pt}$' `$\backslash$t$\hspace{5pt}$' `$\backslash$v$\hspace{5pt}$'
\end{syntax}

\begin{syntax}
single-quote-delimited-characters:
  string-character single-quote-delimited-characters[OPT]
  " single-quote-delimited-characters[OPT]
\end{syntax}

\begin{syntax}
string-character:
  `any character except the double quote, single quote, or new line'
  simple-escape-character
  hexadecimal-escape-character
\end{syntax}

\begin{syntax}
string-literal:
  " double-quote-delimited-characters[OPT] "
  ' single-quote-delimited-characters[OPT] '
\end{syntax}

\section{Alphabetical Syntax Productions}

\begin{syntax}
\end{syntax}

\begin{syntax}
aligned-range-expression:
  range-expression `align' expression
\end{syntax}

\begin{syntax}
argument-list:
  ( formals[OPT] )
\end{syntax}

\begin{syntax}
array-alias-declaration:
  identifier reindexing-expression[OPT] => array-expression ;
\end{syntax}

\begin{syntax}
array-expression:
  expression
\end{syntax}

\begin{syntax}
array-literal:
  rectangular-array-literal
  associative-array-literal
\end{syntax}

\begin{syntax}
array-type:
  [ domain-expression ] type-specifier
\end{syntax}

\begin{syntax}
assignment-operator: one of
   = $ $ $ $ += $ $ $ $ -= $ $ $ $ *= $ $ $ $ /= $ $ $ $ %= $ $ $ $ **= $ $ $ $ &= $ $ $ $ |= $ $ $ $ ^= $ $ $ $ &&= $ $ $ $ ||= $ $ $ $ <<= $ $ $ $ >>=
\end{syntax}

\begin{syntax}
assignment-statement:
  lvalue-expression assignment-operator expression
\end{syntax}

\begin{syntax}
associative-array-literal:
  [ associative-expr-list ]
  [ associative-expr-list , ]
\end{syntax}

\begin{syntax}
associative-domain-literal:
   { associative-expression-list }
\end{syntax}

\begin{syntax}
associative-domain-type:
  `domain' ( associative-index-type )
  `domain' ( enum-type )
  `domain' ( `opaque' )
\end{syntax}

\begin{syntax}
associative-expr-list:
  index-expr => value-expr
  index-expr => value-expr, associative-expr-list
\end{syntax}

\begin{syntax}
associative-expression-list:
   non-range-expression
   non-range-expression, associative-expression-list
\end{syntax}

\begin{syntax}
associative-index-type:
  type-specifier
\end{syntax}

\begin{syntax}
atomic-statement:
  `atomic' statement
\end{syntax}

\begin{syntax}
atomic-type:
  `atomic' type-specifier
\end{syntax}

\begin{syntax}
base-domain-type:
  rectangular-domain-type
  associative-domain-type
\end{syntax}

\begin{syntax}
begin-statement:
  `begin' task-intent-clause[OPT] statement
\end{syntax}

\begin{syntax}
binary-expression:
  expression binary-operator expression
\end{syntax}

\begin{syntax}
binary-operator: one of
  + $ $ $ $ - $ $ $ $ * $ $ $ $ / $ $ $ $ % $ $ $ $ ** $ $ $ $ & $ $ $ $ | $ $ $ $ ^ $ $ $ $ << $ $ $ $ >> $ $ $ $ && $ $ $ $ || $ $ $ $ == $ $ $ $ != $ $ $ $ <= $ $ $ $ >= $ $ $ $ < $ $ $ $ > $ $ $ $ `by' $ $ $ $ #
\end{syntax}

\begin{syntax}
block-statement:
  { statements[OPT] }
\end{syntax}

\begin{syntax}
break-statement:
  `break' identifier[OPT] ;
\end{syntax}

\begin{syntax}
call-expression:
  lvalue-expression ( named-expression-list )
  lvalue-expression [ named-expression-list ]
  parenthesesless-function-identifier
\end{syntax}

\begin{syntax}
cast-expression:
  expression : type-specifier
\end{syntax}

\begin{syntax}
class-declaration-statement:
  simple-class-declaration-statement
  external-class-declaration-statement
\end{syntax}

\begin{syntax}
class-inherit-list:
  : class-type-list
\end{syntax}

\begin{syntax}
class-name:
  identifier
\end{syntax}

\begin{syntax}
class-statement-list:
  class-statement
  class-statement class-statement-list
\end{syntax}

\begin{syntax}
class-statement:
  variable-declaration-statement
  method-declaration-statement
  type-declaration-statement
  empty-statement
\end{syntax}

\begin{syntax}
class-type-list:
  class-type
  class-type , class-type-list
\end{syntax}

\begin{syntax}
class-type:
  identifier
  identifier ( named-expression-list )
\end{syntax}

\begin{syntax}
cobegin-statement:
  `cobegin' task-intent-clause[OPT] block-statement
\end{syntax}

\begin{syntax}
coforall-statement:
  `coforall' index-var-declaration `in' iteratable-expression task-intent-clause[OPT] `do' statement
  `coforall' index-var-declaration `in' iteratable-expression task-intent-clause[OPT] block-statement
  `coforall' iteratable-expression task-intent-clause[OPT] `do' statement
  `coforall' iteratable-expression task-intent-clause[OPT] block-statement
\end{syntax}

\begin{syntax}
conditional-statement:
  `if' expression `then' statement else-part[OPT]
  `if' expression block-statement else-part[OPT]
\end{syntax}

\begin{syntax}
config-or-extern: one of
  `config' $ $ $ $ `extern'
\end{syntax}

\begin{syntax}
constructor-call-expression:
  `new' class-name ( argument-list )
\end{syntax}

\begin{syntax}
continue-statement:
  `continue' identifier[OPT] ;
\end{syntax}

\begin{syntax}
counted-range-expression:
  range-expression # expression
\end{syntax}

\begin{syntax}
dataparallel-type:
  range-type
  domain-type
  mapped-domain-type
  array-type
  index-type
\end{syntax}

\begin{syntax}
default-expression:
  = expression
\end{syntax}

\begin{syntax}
delete-statement:
  `delete' expression ;
\end{syntax}

\begin{syntax}
dmap-value:
  expression
\end{syntax}

\begin{syntax}
do-while-statement:
  `do' statement `while' expression ;
\end{syntax}

\begin{syntax}
domain-alignment-expression:
  domain-expression `align' expression
\end{syntax}

\begin{syntax}
domain-assignment-expression:
  domain-name = domain-expression
\end{syntax}

\begin{syntax}
domain-expression:
  domain-literal
  domain-name
  domain-assignment-expression
  domain-striding-expression
  domain-alignment-expression
  domain-slice-expression
\end{syntax}

\begin{syntax}
domain-literal:
  rectangular-domain-literal
  associative-domain-literal
\end{syntax}

\begin{syntax}
domain-name:
  identifier
\end{syntax}

\begin{syntax}
domain-slice-expression:
  domain-expression [ slicing-index-set ]
  domain-expression ( slicing-index-set )
\end{syntax}

\begin{syntax}
domain-striding-expression:
  domain-expression `by' expression
\end{syntax}

\begin{syntax}
domain-type:
  base-domain-type
  simple-subdomain-type
  sparse-subdomain-type
\end{syntax}

\begin{syntax}
else-part:
  `else' statement
\end{syntax}

\begin{syntax}
empty-statement:
  ;
\end{syntax}

\begin{syntax}
enum-constant-expression:
  enum-type . identifier
\end{syntax}

\begin{syntax}
enum-constant-list:
  enum-constant
  enum-constant , enum-constant-list[OPT]
\end{syntax}

\begin{syntax}
enum-constant:
  identifier init-part[OPT]
\end{syntax}

\begin{syntax}
enum-declaration-statement:
  `enum' identifier { enum-constant-list }
\end{syntax}

\begin{syntax}
enum-type:
  identifier
\end{syntax}

\begin{syntax}
exclude-list:
  identifier-list
  $ * $
\end{syntax}

\begin{syntax}
exported-procedure-declaration-statement:
  `export' external-name[OPT] `proc' function-name argument-list return-intent[OPT] return-type[OPT]
    function-body
\end{syntax}

\begin{syntax}
expression-list:
  expression
  expression , expression-list
\end{syntax}

\begin{syntax}
expression-statement:
  variable-expression ;
  member-access-expression ;
  call-expression ;
  constructor-call-expression ;
  let-expression ; 
\end{syntax}

\begin{syntax}
expression:
  literal-expression
  nil-expression
  variable-expression
  enum-constant-expression
  call-expression
  iteratable-call-expression
  member-access-expression
  constructor-call-expression
  query-expression
  cast-expression
  lvalue-expression
  parenthesized-expression
  unary-expression
  binary-expression
  let-expression
  if-expression
  for-expression
  forall-expression
  reduce-expression
  scan-expression
  module-access-expression
  tuple-expression
  tuple-expand-expression
  locale-access-expression
  mapped-domain-expression
\end{syntax}

\begin{syntax}
external-class-declaration-statement:
  `extern' external-name[OPT] simple-class-declaration-statement
\end{syntax}

\begin{syntax}
external-name:
  identifier
  string-literal
\end{syntax}

\begin{syntax}
external-procedure-declaration-statement:
  `extern' external-name[OPT] `proc' function-name argument-list return-intent[OPT] return-type[OPT]
\end{syntax}

\begin{syntax}
external-record-declaration-statement:
  `extern' external-name[OPT] simple-record-declaration-statement
\end{syntax}

\begin{syntax}
external-type-alias-declaration-statement:
  `extern' `type' type-alias-declaration-list ;
\end{syntax}

\begin{syntax}
field-access-expression:
  receiver-clause[OPT] identifier
\end{syntax}

\begin{syntax}
for-expression:
  `for' index-var-declaration `in' iteratable-expression `do' expression
  `for' iteratable-expression `do' expression
\end{syntax}

\begin{syntax}
for-statement:
  `for' index-var-declaration `in' iteratable-expression `do' statement
  `for' index-var-declaration `in' iteratable-expression block-statement
  `for' iteratable-expression `do' statement
  `for' iteratable-expression block-statement
\end{syntax}

\begin{syntax}
forall-expression:
  `forall' index-var-declaration `in' iteratable-expression task-intent-clause[OPT] `do' expression
  `forall' iteratable-expression task-intent-clause[OPT] `do' expression
  [ index-var-declaration `in' iteratable-expression task-intent-clause[OPT] ] expression
  [ iteratable-expression task-intent-clause[OPT] ] expression
\end{syntax}

\begin{syntax}
forall-statement:
  `forall' index-var-declaration `in' iteratable-expression task-intent-clause[OPT] `do' statement
  `forall' index-var-declaration `in' iteratable-expression task-intent-clause[OPT] block-statement
  `forall' iteratable-expression task-intent-clause[OPT] `do' statement
  `forall' iteratable-expression task-intent-clause[OPT] block-statement
  [ index-var-declaration `in' iteratable-expression task-intent-clause[OPT] ] statement
  [ iteratable-expression task-intent-clause[OPT] ] statement
\end{syntax}

\begin{syntax}
formal-intent:
  `const'
  `const in'
  `const ref'
  `in'
  `out'
  `inout'
  `ref'
  `param'
  `type'
\end{syntax}

\begin{syntax}
formal-type:
  : type-specifier
  : ? identifier[OPT]
\end{syntax}

\begin{syntax}
formal:
  formal-intent[OPT] identifier formal-type[OPT] default-expression[OPT]
  formal-intent[OPT] identifier formal-type[OPT] variable-argument-expression
  formal-intent[OPT] tuple-grouped-identifier-list formal-type[OPT] default-expression[OPT]
  formal-intent[OPT] tuple-grouped-identifier-list formal-type[OPT] variable-argument-expression
\end{syntax}

\begin{syntax}
formals:
  formal
  formal , formals
\end{syntax}

\begin{syntax}
function-body:
  block-statement
  return-statement
\end{syntax}

\begin{syntax}
function-name:
  identifier
  operator-name
\end{syntax}

\begin{syntax}
identifier-list:
  identifier
  identifier , identifier-list
  tuple-grouped-identifier-list
  tuple-grouped-identifier-list , identifier-list
\end{syntax}

\begin{syntax}
if-expression:
  `if' expression `then' expression `else' expression
  `if' expression `then' expression
\end{syntax}

\begin{syntax}
index-expr:
  expression
\end{syntax}

\begin{syntax}
index-type:
  `index' ( domain-expression )
\end{syntax}

\begin{syntax}
index-var-declaration:
  identifier
  tuple-grouped-identifier-list
\end{syntax}

\begin{syntax}
init-part:
  = expression
\end{syntax}

\begin{syntax}
initialization-part:
  = expression
\end{syntax}

\begin{syntax}
integer-parameter-expression:
  expression
\end{syntax}

\begin{syntax}
io-expression:
  expression
  io-expression io-operator expression
\end{syntax}

\begin{syntax}
io-operator:
  <`(*$\sim$*)'>
\end{syntax}

\begin{syntax}
io-statement:
  io-expression io-operator expression
\end{syntax}

\begin{syntax}
iteratable-call-expression:
  call-expression
\end{syntax}

\begin{syntax}
iteratable-expression:
  expression
  `zip' ( expression-list )
\end{syntax}

\begin{syntax}
iterator-body:
  block-statement
  yield-statement
\end{syntax}

\begin{syntax}
iterator-declaration-statement:
  privacy-specifier[OPT] `iter' iterator-name argument-list[OPT] return-intent[OPT] return-type[OPT] where-clause[OPT]
  iterator-body
\end{syntax}

\begin{syntax}
iterator-name:
  identifier
\end{syntax}

\begin{syntax}
label-statement:
  `label' identifier statement
\end{syntax}

\begin{syntax}
let-expression:
  `let' variable-declaration-list `in' expression
\end{syntax}

\begin{syntax}
limitation-clause:
  `except' exclude-list
  `only' rename-list[OPT]
\end{syntax}

\begin{syntax}
linkage-specifier:
  `inline'
\end{syntax}

\begin{syntax}
literal-expression:
  bool-literal
  integer-literal
  real-literal
  imaginary-literal
  string-literal
  range-literal
  domain-literal
  array-literal
\end{syntax}

\begin{syntax}
locale-access-expression:
  expression . `locale'
\end{syntax}

\begin{syntax}
lvalue-expression:
  variable-expression
  member-access-expression
  call-expression
  parenthesized-expression
\end{syntax}

\begin{syntax}
mapped-domain-expression:
  domain-expression `dmapped' dmap-value
\end{syntax}

\begin{syntax}
mapped-domain-type:
  domain-type `dmapped' dmap-value
\end{syntax}

\begin{syntax}
member-access-expression:
  field-access-expression
  method-call-expression
\end{syntax}

\begin{syntax}
method-call-expression:
  receiver-clause[OPT] expression ( named-expression-list )
  receiver-clause[OPT] expression [ named-expression-list ]
  receiver-clause[OPT] parenthesesless-function-identifier
\end{syntax}

\begin{syntax}
method-declaration-statement:
  linkage-specifier[OPT] proc-or-iter this-intent[OPT] type-binding[OPT] function-name argument-list[OPT] 
    return-intent[OPT] return-type[OPT] where-clause[OPT] function-body
\end{syntax}

\begin{syntax}
module-access-expression:
  module-identifier-list . identifier
\end{syntax}

\begin{syntax}
module-declaration-statement:
  privacy-specifier[OPT] `module' module-identifier block-statement
\end{syntax}

\begin{syntax}
module-identifier-list:
  module-identifier
  module-identifier . module-identifier-list
\end{syntax}

\begin{syntax}
module-identifier:
  identifier
\end{syntax}

\begin{syntax}
module-or-enum-name-list:
  module-or-enum-name limitation-clause[OPT]
  module-or-enum-name , module-or-enum-name-list
\end{syntax}

\begin{syntax}
module-or-enum-name:
  identifier
  identifier . module-or-enum-name
\end{syntax}

\begin{syntax}
named-expression-list:
  named-expression
  named-expression , named-expression-list
\end{syntax}

\begin{syntax}
named-expression:
  expression
  identifier = expression
\end{syntax}

\begin{syntax}
nil-expression:
  `nil'
\end{syntax}

\begin{syntax}
no-initialization-part:
  = `noinit'
\end{syntax}

\begin{syntax}
non-range-expression:
   expression
\end{syntax}

\begin{syntax}
on-statement:
  `on' expression `do' statement
  `on' expression block-statement
\end{syntax}

\begin{syntax}
operator-name: one of
  + $ $ $ $ - $ $ $ $ * $ $ $ $ / $ $ $ $ % $ $ $ $ ** $ $ $ $ ! $ $ $ $ == $ $ $ $ != $ $ $ $ <= $ $ $ $ >= $ $ $ $ < $ $ $ $ > $ $ $ $ << $ $ $ $ >> $ $ $ $ & $ $ $ $ | $ $ $ $ ^ $ $ $ $ ~
  += $ $ $ $ -= $ $ $ $ *= $ $ $ $ /= $ $ $ $ %= $ $ $ $ **= $ $ $ $ &= $ $ $ $ |= $ $ $ $ ^= $ $ $ $ <<= $ $ $ $ >>= $ $ $ $ <=> $ $ $ $ <~>
\end{syntax}

\begin{syntax}
param-for-statement:
  `for' `param' identifier `in' param-iteratable-expression `do' statement
  `for' `param' identifier `in' param-iteratable-expression block-statement
\end{syntax}

\begin{syntax}
param-iteratable-expression:
  range-literal
  range-literal `by' integer-literal
\end{syntax}

\begin{syntax}
parenthesesless-function-identifier:
  identifier
\end{syntax}

\begin{syntax}
parenthesized-expression:
  ( expression )
\end{syntax}

\begin{syntax}
primitive-type-parameter-part:
  ( integer-parameter-expression )
\end{syntax}

\begin{syntax}
primitive-type:
  `void'
  `bool' primitive-type-parameter-part[OPT]
  `int' primitive-type-parameter-part[OPT]
  `uint' primitive-type-parameter-part[OPT]
  `real' primitive-type-parameter-part[OPT]
  `imag' primitive-type-parameter-part[OPT]
  `complex' primitive-type-parameter-part[OPT]
  `string'
\end{syntax}

\begin{syntax}
privacy-specifier:
  `private'
  `public'
\end{syntax}

\begin{syntax}
proc-or-iter:
  `proc'
  `iter'
\end{syntax}

\begin{syntax}
procedure-declaration-statement:
  privacy-specifier[OPT] linkage-specifier[OPT] `proc' function-name argument-list[OPT] return-intent[OPT] return-type[OPT] where-clause[OPT]
    function-body
\end{syntax}

\begin{syntax}
query-expression:
  ? identifier[OPT]
\end{syntax}

\begin{syntax}
range-expression-list:
  range-expression
  range-expression, range-expression-list
\end{syntax}

\begin{syntax}
range-expression:
  expression
  strided-range-expression
  counted-range-expression
  aligned-range-expression
  sliced-range-expression
\end{syntax}

\begin{syntax}
range-literal:
  expression .. expression
  expression ..
  .. expression
  ..
\end{syntax}

\begin{syntax}
range-type:
  `range' ( named-expression-list )
\end{syntax}

\begin{syntax}
receiver-clause:
  expression .
\end{syntax}

\begin{syntax}
record-declaration-statement:
  simple-record-declaration-statement
  external-record-declaration-statement
\end{syntax}

\begin{syntax}
record-inherit-list:
  : record-type-list
\end{syntax}

\begin{syntax}
record-statement-list:
  record-statement
  record-statement record-statement-list
\end{syntax}

\begin{syntax}
record-statement:
  variable-declaration-statement
  method-declaration-statement
  type-declaration-statement
  empty-statement
\end{syntax}

\begin{syntax}
record-type-list:
  record-type
  record-type , record-type-list
\end{syntax}

\begin{syntax}
record-type:
  identifier
  identifier ( named-expression-list )
\end{syntax}

\begin{syntax}
rectangular-array-literal:
  [ expression-list ]
  [ expression-list , ]
\end{syntax}

\begin{syntax}
rectangular-domain-literal:
  { range-expression-list }
\end{syntax}

\begin{syntax}
rectangular-domain-type:
  `domain' ( named-expression-list )
\end{syntax}

\begin{syntax}
reduce-expression:
  reduce-scan-operator `reduce' iteratable-expression
  class-type `reduce' iteratable-expression
\end{syntax}

\begin{syntax}
reduce-scan-operator: one of
  + $ $ $ $ * $ $ $ $ && $ $ $ $ || $ $ $ $ & $ $ $ $ | $ $ $ $ ^ $ $ $ $ `min' $ $ $ $ `max' $ $ $ $ `minloc' $ $ $ $ `maxloc'
\end{syntax}

\begin{syntax}
reindexing-expression:
  : [ domain-expression ]
\end{syntax}

\begin{syntax}
remote-variable-declaration-statement:
  `on' expression variable-declaration-statement
\end{syntax}

\begin{syntax}
rename-base:
  identifier `as' identifier
  identifier
\end{syntax}

\begin{syntax}
rename-list:
  rename-base
  rename-base , rename-list
\end{syntax}

\begin{syntax}
return-intent:
  `const'
  `const ref'
  `ref'
  `param'
  `type'
\end{syntax}

\begin{syntax}
return-statement:
  `return' expression[OPT] ;
\end{syntax}

\begin{syntax}
return-type:
  : type-specifier
\end{syntax}

\begin{syntax}
scan-expression:
  reduce-scan-operator `scan' iteratable-expression
  class-type `scan' iteratable-expression
\end{syntax}

\begin{syntax}
select-statement:
  `select' expression { when-statements }
\end{syntax}

\begin{syntax}
serial-statement:
  `serial' expression[OPT] `do' statement
  `serial' expression[OPT] block-statement
\end{syntax}

\begin{syntax}
simple-class-declaration-statement:
  `class' identifier class-inherit-list[OPT] { class-statement-list[OPT] }
\end{syntax}

\begin{syntax}
simple-record-declaration-statement:
  `record' identifier record-inherit-list[OPT] { record-statement-list }
\end{syntax}

\begin{syntax}
simple-subdomain-type:
  `subdomain' ( domain-expression )
\end{syntax}

\begin{syntax}
single-type:
  `single' type-specifier
\end{syntax}

\begin{syntax}
sliced-range-expression:
  range-expression ( range-expression )
  range-expression [ range-expression ]
\end{syntax}

\begin{syntax}
slicing-index-set:
  domain-expression
  range-expression-list
\end{syntax}

\begin{syntax}
sparse-subdomain-type:
  `sparse' `subdomain'[OPT] ( domain-expression )
\end{syntax}

\begin{syntax}
statement:
  block-statement
  expression-statement
  assignment-statement
  swap-statement
  io-statement
  conditional-statement
  select-statement
  while-do-statement
  do-while-statement
  for-statement
  label-statement
  break-statement
  continue-statement
  param-for-statement
  use-statement
  empty-statement
  return-statement
  yield-statement
  module-declaration-statement
  procedure-declaration-statement
  external-procedure-declaration-statement
  exported-procedure-declaration-statement
  iterator-declaration-statement
  method-declaration-statement
  type-declaration-statement
  variable-declaration-statement
  remote-variable-declaration-statement
  on-statement
  cobegin-statement
  coforall-statement
  begin-statement
  sync-statement
  serial-statement
  atomic-statement
  forall-statement
  delete-statement
\end{syntax}

\begin{syntax}
statements:
  statement
  statement statements
\end{syntax}

\begin{syntax}
step-expression:
  expression
\end{syntax}

\begin{syntax}
strided-range-expression:
  range-expression `by' step-expression
\end{syntax}

\begin{syntax}
structured-type:
  class-type
  record-type
  union-type
  tuple-type
\end{syntax}

\begin{syntax}
swap-operator:
  <=>
\end{syntax}

\begin{syntax}
swap-statement:
  lvalue-expression swap-operator lvalue-expression
\end{syntax}

\begin{syntax}
sync-statement:
  `sync' statement
  `sync' block-statement
\end{syntax}

\begin{syntax}
sync-type:
  `sync' type-specifier
\end{syntax}

\begin{syntax}
synchronization-type:
  sync-type
  single-type
  atomic-type
\end{syntax}

\begin{syntax}
task-intent-clause:
  `with' ( task-intent-list )
\end{syntax}

\begin{syntax}
task-intent-list:
  formal-intent identifier
  formal-intent identifier, task-intent-list
\end{syntax}

\begin{syntax}
this-intent:
  `param'
  `ref'
  `type'
\end{syntax}

\begin{syntax}
tuple-component-list:
  tuple-component
  tuple-component , tuple-component-list
\end{syntax}

\begin{syntax}
tuple-component:
  expression
  `_'
\end{syntax}

\begin{syntax}
tuple-expand-expression:
  ( ... expression )
\end{syntax}

\begin{syntax}
tuple-expression:
  ( tuple-component , )
  ( tuple-component , tuple-component-list )
  ( tuple-component , tuple-component-list , )
\end{syntax}

\begin{syntax}
tuple-grouped-identifier-list:
  ( identifier-list )
\end{syntax}

\begin{syntax}
tuple-type:
  ( type-specifier , type-list )
\end{syntax}

\begin{syntax}
type-alias-declaration-list:
  type-alias-declaration
  type-alias-declaration , type-alias-declaration-list
\end{syntax}

\begin{syntax}
type-alias-declaration-statement:
  privacy-specifier[OPT] `config'[OPT] `type' type-alias-declaration-list ;
  external-type-alias-declaration-statement
\end{syntax}

\begin{syntax}
type-alias-declaration:
  identifier = type-specifier
  identifier
\end{syntax}

\begin{syntax}
type-binding:
  identifier .
\end{syntax}

\begin{syntax}
type-declaration-statement:
  enum-declaration-statement
  class-declaration-statement
  record-declaration-statement
  union-declaration-statement
  type-alias-declaration-statement
\end{syntax}

\begin{syntax}
type-list:
  type-specifier
  type-specifier , type-list
\end{syntax}

\begin{syntax}
type-part:
  : type-specifier
\end{syntax}

\begin{syntax}
type-specifier:
  primitive-type
  enum-type
  structured-type
  dataparallel-type
  synchronization-type
\end{syntax}

\begin{syntax}
unary-expression:
  unary-operator expression
\end{syntax}

\begin{syntax}
unary-operator: one of
  + $ $ $ $ - $ $ $ $ ~ $ $ $ $ !
\end{syntax}

\begin{syntax}
union-declaration-statement:
  `extern'[OPT] `union' identifier { union-statement-list }
\end{syntax}

\begin{syntax}
union-statement-list:
  union-statement
  union-statement union-statement-list
\end{syntax}

\begin{syntax}
union-statement:
  type-declaration-statement
  procedure-declaration-statement
  iterator-declaration-statement
  variable-declaration-statement
  empty-statement
\end{syntax}

\begin{syntax}
union-type:
  identifier
\end{syntax}

\begin{syntax}
use-statement:
  `use' module-or-enum-name-list ;
\end{syntax}

\begin{syntax}
value-expr:
  expression
\end{syntax}

\begin{syntax}
variable-argument-expression:
  ... expression
  ... ? identifier[OPT]
  ...
\end{syntax}

\begin{syntax}
variable-declaration-list:
  variable-declaration
  variable-declaration , variable-declaration-list
\end{syntax}

\begin{syntax}
variable-declaration-statement:
  privacy-specifier[OPT] config-or-extern[OPT] variable-kind variable-declaration-list ;
\end{syntax}

\begin{syntax}
variable-declaration:
  identifier-list type-part[OPT] initialization-part
  identifier-list type-part no-initialization-part[OPT]
  array-alias-declaration
\end{syntax}

\begin{syntax}
variable-expression:
  identifier
\end{syntax}

\begin{syntax}
variable-kind:
  `param'
  `const'
  `var'
  `ref'
  `const ref'
\end{syntax}

\begin{syntax}
when-statement:
  `when' expression-list `do' statement
  `when' expression-list block-statement
  `otherwise' statement
  `otherwise' `do' statement
\end{syntax}

\begin{syntax}
when-statements:
  when-statement
  when-statement when-statements
\end{syntax}

\begin{syntax}
where-clause:
  `where' expression
\end{syntax}

\begin{syntax}
while-do-statement:
  `while' expression `do' statement
  `while' expression block-statement
\end{syntax}

\begin{syntax}
yield-statement:
  `yield' expression ;
\end{syntax}

\section{Depth-First Lexical Productions}

\begin{syntax}
bool-literal: one of
  `true' $ $ $ $ `false'
\end{syntax}

\begin{syntax}
identifier:
  letter-or-underscore legal-identifier-chars[OPT]
\end{syntax}

\begin{syntax}
letter-or-underscore:
  letter
  `_'
\end{syntax}

\begin{syntax}
letter: one of
  `A' `B' `C' `D' `E' `F' `G' `H' `I' `J' `K' `L' `M' `N' `O' `P' `Q' `R' `S' `T' `U' `V' `W' `X' `Y' `Z'
  `a' `b' `c' `d' `e' `f' `g' `h' `i' `j' `k' `l' `m' `n' `o' `p' `q' `r' `s' `t' `u' `v' `w' `x' `y' `z'
\end{syntax}

\begin{syntax}
legal-identifier-chars:
  legal-identifier-char legal-identifier-chars[OPT]
\end{syntax}

\begin{syntax}
legal-identifier-char:
  letter-or-underscore
  digit
  `(*\texttt{\$}*)'
\end{syntax}

\begin{syntax}
digit: one of
  `0' `1' `2' `3' `4' `5' `6' `7' `8' `9'
\end{syntax}

\begin{syntax}
imaginary-literal:
  real-literal `i'
  integer-literal `i'
\end{syntax}

\begin{syntax}
real-literal:
  digits[OPT] . digits exponent-part[OPT]
  digits .[OPT] exponent-part
  `0x' hexadecimal-digits[OPT] . hexadecimal-digits p-exponent-part[OPT]
  `0X' hexadecimal-digits[OPT] . hexadecimal-digits p-exponent-part[OPT]
  `0x' hexadecimal-digits .[OPT] p-exponent-part
  `0X' hexadecimal-digits .[OPT] p-exponent-part
\end{syntax}

\begin{syntax}
digits:
  digit
  digit digits
\end{syntax}

\begin{syntax}
exponent-part:
  `e' sign[OPT] digits
  `E' sign[OPT] digits
\end{syntax}

\begin{syntax}
sign: one of
  + $ $ $ $ -
\end{syntax}

\begin{syntax}
hexadecimal-digits:
  hexadecimal-digit
  hexadecimal-digit hexadecimal-digits
\end{syntax}

\begin{syntax}
hexadecimal-digit: one of
  `0' `1' `2' `3' `4' `5' `6' `7' `8' `9' `A' `B' `C' `D' `E' `F' `a' `b' `c' `d' `e' `f'
\end{syntax}

\begin{syntax}
p-exponent-part:
  `p' sign[OPT] digits
  `P' sign[OPT] digits
\end{syntax}

\begin{syntax}
integer-literal:
  digits
  `0x' hexadecimal-digits
  `0X' hexadecimal-digits
  `0o' octal-digits
  `0O' octal-digits
  `0b' binary-digits
  `0B' binary-digits
\end{syntax}

\begin{syntax}
octal-digits:
  octal-digit
  octal-digit octal-digits
\end{syntax}

\begin{syntax}
octal-digit: one of
  `0' `1' `2' `3' `4' `5' `6' `7'
\end{syntax}

\begin{syntax}
binary-digits:
  binary-digit
  binary-digit binary-digits
\end{syntax}

\begin{syntax}
binary-digit: one of
  `0' `1'
\end{syntax}

\begin{syntax}
string-literal:
  " double-quote-delimited-characters[OPT] "
  ' single-quote-delimited-characters[OPT] '
\end{syntax}

\begin{syntax}
double-quote-delimited-characters:
  string-character double-quote-delimited-characters[OPT]
  ' double-quote-delimited-characters[OPT]
\end{syntax}

\begin{syntax}
string-character:
  `any character except the double quote, single quote, or new line'
  simple-escape-character
  hexadecimal-escape-character
\end{syntax}

\begin{syntax}
simple-escape-character: one of
  `$\backslash\mbox{\bf '}\hspace{5pt}$' `$\backslash$"$\hspace{5pt}$' `$\backslash$?$\hspace{5pt}$' `$\backslash$$\backslash$$\hspace{5pt}$' `$\backslash$a$\hspace{5pt}$' `$\backslash$b$\hspace{5pt}$' `$\backslash$f$\hspace{5pt}$' `$\backslash$n$\hspace{5pt}$' `$\backslash$r$\hspace{5pt}$' `$\backslash$t$\hspace{5pt}$' `$\backslash$v$\hspace{5pt}$'
\end{syntax}

\begin{syntax}
hexadecimal-escape-character:
  `$\backslash$x' hexadecimal-digits
\end{syntax}

\begin{syntax}
single-quote-delimited-characters:
  string-character single-quote-delimited-characters[OPT]
  " single-quote-delimited-characters[OPT]
\end{syntax}

\section{Depth-First Syntax Productions}

\begin{syntax}
module-declaration-statement:
  privacy-specifier[OPT] `module' module-identifier block-statement
\end{syntax}

\begin{syntax}
privacy-specifier:
  `private'
  `public'
\end{syntax}

\begin{syntax}
module-identifier:
  identifier
\end{syntax}

\begin{syntax}
block-statement:
  { statements[OPT] }
\end{syntax}

\begin{syntax}
statements:
  statement
  statement statements
\end{syntax}

\begin{syntax}
statement:
  block-statement
  expression-statement
  assignment-statement
  swap-statement
  io-statement
  conditional-statement
  select-statement
  while-do-statement
  do-while-statement
  for-statement
  label-statement
  break-statement
  continue-statement
  param-for-statement
  use-statement
  empty-statement
  return-statement
  yield-statement
  module-declaration-statement
  procedure-declaration-statement
  external-procedure-declaration-statement
  exported-procedure-declaration-statement
  iterator-declaration-statement
  method-declaration-statement
  type-declaration-statement
  variable-declaration-statement
  remote-variable-declaration-statement
  on-statement
  cobegin-statement
  coforall-statement
  begin-statement
  sync-statement
  serial-statement
  atomic-statement
  forall-statement
  delete-statement
\end{syntax}

\begin{syntax}
expression-statement:
  variable-expression ;
  member-access-expression ;
  call-expression ;
  constructor-call-expression ;
  let-expression ; 
\end{syntax}

\begin{syntax}
variable-expression:
  identifier
\end{syntax}

\begin{syntax}
member-access-expression:
  field-access-expression
  method-call-expression
\end{syntax}

\begin{syntax}
field-access-expression:
  receiver-clause[OPT] identifier
\end{syntax}

\begin{syntax}
receiver-clause:
  expression .
\end{syntax}

\begin{syntax}
expression:
  literal-expression
  nil-expression
  variable-expression
  enum-constant-expression
  call-expression
  iteratable-call-expression
  member-access-expression
  constructor-call-expression
  query-expression
  cast-expression
  lvalue-expression
  parenthesized-expression
  unary-expression
  binary-expression
  let-expression
  if-expression
  for-expression
  forall-expression
  reduce-expression
  scan-expression
  module-access-expression
  tuple-expression
  tuple-expand-expression
  locale-access-expression
  mapped-domain-expression
\end{syntax}

\begin{syntax}
literal-expression:
  bool-literal
  integer-literal
  real-literal
  imaginary-literal
  string-literal
  range-literal
  domain-literal
  array-literal
\end{syntax}

\begin{syntax}
range-literal:
  expression .. expression
  expression ..
  .. expression
  ..
\end{syntax}

\begin{syntax}
domain-literal:
  rectangular-domain-literal
  associative-domain-literal
\end{syntax}

\begin{syntax}
rectangular-domain-literal:
  { range-expression-list }
\end{syntax}

\begin{syntax}
range-expression-list:
  range-expression
  range-expression, range-expression-list
\end{syntax}

\begin{syntax}
range-expression:
  expression
  strided-range-expression
  counted-range-expression
  aligned-range-expression
  sliced-range-expression
\end{syntax}

\begin{syntax}
strided-range-expression:
  range-expression `by' step-expression
\end{syntax}

\begin{syntax}
step-expression:
  expression
\end{syntax}

\begin{syntax}
counted-range-expression:
  range-expression # expression
\end{syntax}

\begin{syntax}
aligned-range-expression:
  range-expression `align' expression
\end{syntax}

\begin{syntax}
sliced-range-expression:
  range-expression ( range-expression )
  range-expression [ range-expression ]
\end{syntax}

\begin{syntax}
associative-domain-literal:
   { associative-expression-list }
\end{syntax}

\begin{syntax}
associative-expression-list:
   non-range-expression
   non-range-expression, associative-expression-list
\end{syntax}

\begin{syntax}
non-range-expression:
   expression
\end{syntax}

\begin{syntax}
array-literal:
  rectangular-array-literal
  associative-array-literal
\end{syntax}

\begin{syntax}
rectangular-array-literal:
  [ expression-list ]
  [ expression-list , ]
\end{syntax}

\begin{syntax}
expression-list:
  expression
  expression , expression-list
\end{syntax}

\begin{syntax}
associative-array-literal:
  [ associative-expr-list ]
  [ associative-expr-list , ]
\end{syntax}

\begin{syntax}
associative-expr-list:
  index-expr => value-expr
  index-expr => value-expr, associative-expr-list
\end{syntax}

\begin{syntax}
index-expr:
  expression
\end{syntax}

\begin{syntax}
value-expr:
  expression
\end{syntax}

\begin{syntax}
nil-expression:
  `nil'
\end{syntax}

\begin{syntax}
enum-constant-expression:
  enum-type . identifier
\end{syntax}

\begin{syntax}
enum-type:
  identifier
\end{syntax}

\begin{syntax}
iteratable-call-expression:
  call-expression
\end{syntax}

\begin{syntax}
query-expression:
  ? identifier[OPT]
\end{syntax}

\begin{syntax}
cast-expression:
  expression : type-specifier
\end{syntax}

\begin{syntax}
type-specifier:
  primitive-type
  enum-type
  structured-type
  dataparallel-type
  synchronization-type
\end{syntax}

\begin{syntax}
primitive-type:
  `void'
  `bool' primitive-type-parameter-part[OPT]
  `int' primitive-type-parameter-part[OPT]
  `uint' primitive-type-parameter-part[OPT]
  `real' primitive-type-parameter-part[OPT]
  `imag' primitive-type-parameter-part[OPT]
  `complex' primitive-type-parameter-part[OPT]
  `string'
\end{syntax}

\begin{syntax}
primitive-type-parameter-part:
  ( integer-parameter-expression )
\end{syntax}

\begin{syntax}
integer-parameter-expression:
  expression
\end{syntax}

\begin{syntax}
structured-type:
  class-type
  record-type
  union-type
  tuple-type
\end{syntax}

\begin{syntax}
class-type:
  identifier
  identifier ( named-expression-list )
\end{syntax}

\begin{syntax}
named-expression-list:
  named-expression
  named-expression , named-expression-list
\end{syntax}

\begin{syntax}
named-expression:
  expression
  identifier = expression
\end{syntax}

\begin{syntax}
record-type:
  identifier
  identifier ( named-expression-list )
\end{syntax}

\begin{syntax}
union-type:
  identifier
\end{syntax}

\begin{syntax}
tuple-type:
  ( type-specifier , type-list )
\end{syntax}

\begin{syntax}
type-list:
  type-specifier
  type-specifier , type-list
\end{syntax}

\begin{syntax}
dataparallel-type:
  range-type
  domain-type
  mapped-domain-type
  array-type
  index-type
\end{syntax}

\begin{syntax}
range-type:
  `range' ( named-expression-list )
\end{syntax}

\begin{syntax}
domain-type:
  base-domain-type
  simple-subdomain-type
  sparse-subdomain-type
\end{syntax}

\begin{syntax}
base-domain-type:
  rectangular-domain-type
  associative-domain-type
\end{syntax}

\begin{syntax}
rectangular-domain-type:
  `domain' ( named-expression-list )
\end{syntax}

\begin{syntax}
associative-domain-type:
  `domain' ( associative-index-type )
  `domain' ( enum-type )
  `domain' ( `opaque' )
\end{syntax}

\begin{syntax}
associative-index-type:
  type-specifier
\end{syntax}

\begin{syntax}
simple-subdomain-type:
  `subdomain' ( domain-expression )
\end{syntax}

\begin{syntax}
domain-expression:
  domain-literal
  domain-name
  domain-assignment-expression
  domain-striding-expression
  domain-alignment-expression
  domain-slice-expression
\end{syntax}

\begin{syntax}
domain-name:
  identifier
\end{syntax}

\begin{syntax}
domain-assignment-expression:
  domain-name = domain-expression
\end{syntax}

\begin{syntax}
domain-striding-expression:
  domain-expression `by' expression
\end{syntax}

\begin{syntax}
domain-alignment-expression:
  domain-expression `align' expression
\end{syntax}

\begin{syntax}
domain-slice-expression:
  domain-expression [ slicing-index-set ]
  domain-expression ( slicing-index-set )
\end{syntax}

\begin{syntax}
slicing-index-set:
  domain-expression
  range-expression-list
\end{syntax}

\begin{syntax}
sparse-subdomain-type:
  `sparse' `subdomain'[OPT] ( domain-expression )
\end{syntax}

\begin{syntax}
mapped-domain-type:
  domain-type `dmapped' dmap-value
\end{syntax}

\begin{syntax}
dmap-value:
  expression
\end{syntax}

\begin{syntax}
array-type:
  [ domain-expression ] type-specifier
\end{syntax}

\begin{syntax}
index-type:
  `index' ( domain-expression )
\end{syntax}

\begin{syntax}
synchronization-type:
  sync-type
  single-type
  atomic-type
\end{syntax}

\begin{syntax}
sync-type:
  `sync' type-specifier
\end{syntax}

\begin{syntax}
single-type:
  `single' type-specifier
\end{syntax}

\begin{syntax}
atomic-type:
  `atomic' type-specifier
\end{syntax}

\begin{syntax}
lvalue-expression:
  variable-expression
  member-access-expression
  call-expression
  parenthesized-expression
\end{syntax}

\begin{syntax}
parenthesized-expression:
  ( expression )
\end{syntax}

\begin{syntax}
unary-expression:
  unary-operator expression
\end{syntax}

\begin{syntax}
unary-operator: one of
  + $ $ $ $ - $ $ $ $ ~ $ $ $ $ !
\end{syntax}

\begin{syntax}
binary-expression:
  expression binary-operator expression
\end{syntax}

\begin{syntax}
binary-operator: one of
  + $ $ $ $ - $ $ $ $ * $ $ $ $ / $ $ $ $ % $ $ $ $ ** $ $ $ $ & $ $ $ $ | $ $ $ $ ^ $ $ $ $ << $ $ $ $ >> $ $ $ $ && $ $ $ $ || $ $ $ $ == $ $ $ $ != $ $ $ $ <= $ $ $ $ >= $ $ $ $ < $ $ $ $ > $ $ $ $ `by' $ $ $ $ #
\end{syntax}

\begin{syntax}
if-expression:
  `if' expression `then' expression `else' expression
  `if' expression `then' expression
\end{syntax}

\begin{syntax}
for-expression:
  `for' index-var-declaration `in' iteratable-expression `do' expression
  `for' iteratable-expression `do' expression
\end{syntax}

\begin{syntax}
forall-expression:
  `forall' index-var-declaration `in' iteratable-expression task-intent-clause[OPT] `do' expression
  `forall' iteratable-expression task-intent-clause[OPT] `do' expression
  [ index-var-declaration `in' iteratable-expression task-intent-clause[OPT] ] expression
  [ iteratable-expression task-intent-clause[OPT] ] expression
\end{syntax}

\begin{syntax}
index-var-declaration:
  identifier
  tuple-grouped-identifier-list
\end{syntax}

\begin{syntax}
tuple-grouped-identifier-list:
  ( identifier-list )
\end{syntax}

\begin{syntax}
identifier-list:
  identifier
  identifier , identifier-list
  tuple-grouped-identifier-list
  tuple-grouped-identifier-list , identifier-list
\end{syntax}

\begin{syntax}
iteratable-expression:
  expression
  `zip' ( expression-list )
\end{syntax}

\begin{syntax}
task-intent-clause:
  `with' ( task-intent-list )
\end{syntax}

\begin{syntax}
task-intent-list:
  formal-intent identifier
  formal-intent identifier, task-intent-list
\end{syntax}

\begin{syntax}
formal-intent:
  `const'
  `const in'
  `const ref'
  `in'
  `out'
  `inout'
  `ref'
  `param'
  `type'
\end{syntax}

\begin{syntax}
reduce-expression:
  reduce-scan-operator `reduce' iteratable-expression
  class-type `reduce' iteratable-expression
\end{syntax}

\begin{syntax}
reduce-scan-operator: one of
  + $ $ $ $ * $ $ $ $ && $ $ $ $ || $ $ $ $ & $ $ $ $ | $ $ $ $ ^ $ $ $ $ `min' $ $ $ $ `max' $ $ $ $ `minloc' $ $ $ $ `maxloc'
\end{syntax}

\begin{syntax}
scan-expression:
  reduce-scan-operator `scan' iteratable-expression
  class-type `scan' iteratable-expression
\end{syntax}

\begin{syntax}
module-access-expression:
  module-identifier-list . identifier
\end{syntax}

\begin{syntax}
module-identifier-list:
  module-identifier
  module-identifier . module-identifier-list
\end{syntax}

\begin{syntax}
tuple-expression:
  ( tuple-component , )
  ( tuple-component , tuple-component-list )
  ( tuple-component , tuple-component-list , )
\end{syntax}

\begin{syntax}
tuple-component:
  expression
  `_'
\end{syntax}

\begin{syntax}
tuple-component-list:
  tuple-component
  tuple-component , tuple-component-list
\end{syntax}

\begin{syntax}
tuple-expand-expression:
  ( ... expression )
\end{syntax}

\begin{syntax}
locale-access-expression:
  expression . `locale'
\end{syntax}

\begin{syntax}
mapped-domain-expression:
  domain-expression `dmapped' dmap-value
\end{syntax}

\begin{syntax}
method-call-expression:
  receiver-clause[OPT] expression ( named-expression-list )
  receiver-clause[OPT] expression [ named-expression-list ]
  receiver-clause[OPT] parenthesesless-function-identifier
\end{syntax}

\begin{syntax}
parenthesesless-function-identifier:
  identifier
\end{syntax}

\begin{syntax}
call-expression:
  lvalue-expression ( named-expression-list )
  lvalue-expression [ named-expression-list ]
  parenthesesless-function-identifier
\end{syntax}

\begin{syntax}
constructor-call-expression:
  `new' class-name ( argument-list )
\end{syntax}

\begin{syntax}
class-name:
  identifier
\end{syntax}

\begin{syntax}
argument-list:
  ( formals[OPT] )
\end{syntax}

\begin{syntax}
formals:
  formal
  formal , formals
\end{syntax}

\begin{syntax}
formal:
  formal-intent[OPT] identifier formal-type[OPT] default-expression[OPT]
  formal-intent[OPT] identifier formal-type[OPT] variable-argument-expression
  formal-intent[OPT] tuple-grouped-identifier-list formal-type[OPT] default-expression[OPT]
  formal-intent[OPT] tuple-grouped-identifier-list formal-type[OPT] variable-argument-expression
\end{syntax}

\begin{syntax}
default-expression:
  = expression
\end{syntax}

\begin{syntax}
formal-type:
  : type-specifier
  : ? identifier[OPT]
\end{syntax}

\begin{syntax}
variable-argument-expression:
  ... expression
  ... ? identifier[OPT]
  ...
\end{syntax}

\begin{syntax}
let-expression:
  `let' variable-declaration-list `in' expression
\end{syntax}

\begin{syntax}
assignment-statement:
  lvalue-expression assignment-operator expression
\end{syntax}

\begin{syntax}
assignment-operator: one of
   = $ $ $ $ += $ $ $ $ -= $ $ $ $ *= $ $ $ $ /= $ $ $ $ %= $ $ $ $ **= $ $ $ $ &= $ $ $ $ |= $ $ $ $ ^= $ $ $ $ &&= $ $ $ $ ||= $ $ $ $ <<= $ $ $ $ >>=
\end{syntax}

\begin{syntax}
swap-statement:
  lvalue-expression swap-operator lvalue-expression
\end{syntax}

\begin{syntax}
swap-operator:
  <=>
\end{syntax}

\begin{syntax}
io-statement:
  io-expression io-operator expression
\end{syntax}

\begin{syntax}
io-expression:
  expression
  io-expression io-operator expression
\end{syntax}

\begin{syntax}
io-operator:
  <`(*$\sim$*)'>
\end{syntax}

\begin{syntax}
conditional-statement:
  `if' expression `then' statement else-part[OPT]
  `if' expression block-statement else-part[OPT]
\end{syntax}

\begin{syntax}
else-part:
  `else' statement
\end{syntax}

\begin{syntax}
select-statement:
  `select' expression { when-statements }
\end{syntax}

\begin{syntax}
when-statements:
  when-statement
  when-statement when-statements
\end{syntax}

\begin{syntax}
when-statement:
  `when' expression-list `do' statement
  `when' expression-list block-statement
  `otherwise' statement
  `otherwise' `do' statement
\end{syntax}

\begin{syntax}
while-do-statement:
  `while' expression `do' statement
  `while' expression block-statement
\end{syntax}

\begin{syntax}
do-while-statement:
  `do' statement `while' expression ;
\end{syntax}

\begin{syntax}
for-statement:
  `for' index-var-declaration `in' iteratable-expression `do' statement
  `for' index-var-declaration `in' iteratable-expression block-statement
  `for' iteratable-expression `do' statement
  `for' iteratable-expression block-statement
\end{syntax}

\begin{syntax}
label-statement:
  `label' identifier statement
\end{syntax}

\begin{syntax}
break-statement:
  `break' identifier[OPT] ;
\end{syntax}

\begin{syntax}
continue-statement:
  `continue' identifier[OPT] ;
\end{syntax}

\begin{syntax}
param-for-statement:
  `for' `param' identifier `in' param-iteratable-expression `do' statement
  `for' `param' identifier `in' param-iteratable-expression block-statement
\end{syntax}

\begin{syntax}
param-iteratable-expression:
  range-literal
  range-literal `by' integer-literal
\end{syntax}

\begin{syntax}
use-statement:
  `use' module-or-enum-name-list ;
\end{syntax}

\begin{syntax}
module-or-enum-name-list:
  module-or-enum-name limitation-clause[OPT]
  module-or-enum-name , module-or-enum-name-list
\end{syntax}

\begin{syntax}
limitation-clause:
  `except' exclude-list
  `only' rename-list[OPT]
\end{syntax}

\begin{syntax}
exclude-list:
  identifier-list
  $ * $
\end{syntax}

\begin{syntax}
rename-list:
  rename-base
  rename-base , rename-list
\end{syntax}

\begin{syntax}
rename-base:
  identifier `as' identifier
  identifier
\end{syntax}

\begin{syntax}
module-or-enum-name:
  identifier
  identifier . module-or-enum-name
\end{syntax}

\begin{syntax}
empty-statement:
  ;
\end{syntax}

\begin{syntax}
return-statement:
  `return' expression[OPT] ;
\end{syntax}

\begin{syntax}
yield-statement:
  `yield' expression ;
\end{syntax}

\begin{syntax}
module-declaration-statement:
  privacy-specifier[OPT] `module' module-identifier block-statement
\end{syntax}

\begin{syntax}
procedure-declaration-statement:
  privacy-specifier[OPT] linkage-specifier[OPT] `proc' function-name argument-list[OPT] return-intent[OPT] return-type[OPT] where-clause[OPT]
    function-body
\end{syntax}

\begin{syntax}
linkage-specifier:
  `inline'
\end{syntax}

\begin{syntax}
function-name:
  identifier
  operator-name
\end{syntax}

\begin{syntax}
operator-name: one of
  + $ $ $ $ - $ $ $ $ * $ $ $ $ / $ $ $ $ % $ $ $ $ ** $ $ $ $ ! $ $ $ $ == $ $ $ $ != $ $ $ $ <= $ $ $ $ >= $ $ $ $ < $ $ $ $ > $ $ $ $ << $ $ $ $ >> $ $ $ $ & $ $ $ $ | $ $ $ $ ^ $ $ $ $ ~
  += $ $ $ $ -= $ $ $ $ *= $ $ $ $ /= $ $ $ $ %= $ $ $ $ **= $ $ $ $ &= $ $ $ $ |= $ $ $ $ ^= $ $ $ $ <<= $ $ $ $ >>= $ $ $ $ <=> $ $ $ $ <~>
\end{syntax}

\begin{syntax}
return-intent:
  `const'
  `const ref'
  `ref'
  `param'
  `type'
\end{syntax}

\begin{syntax}
return-type:
  : type-specifier
\end{syntax}

\begin{syntax}
where-clause:
  `where' expression
\end{syntax}

\begin{syntax}
function-body:
  block-statement
  return-statement
\end{syntax}

\begin{syntax}
external-procedure-declaration-statement:
  `extern' external-name[OPT] `proc' function-name argument-list return-intent[OPT] return-type[OPT]
\end{syntax}

\begin{syntax}
exported-procedure-declaration-statement:
  `export' external-name[OPT] `proc' function-name argument-list return-intent[OPT] return-type[OPT]
    function-body
\end{syntax}

\begin{syntax}
iterator-declaration-statement:
  privacy-specifier[OPT] `iter' iterator-name argument-list[OPT] return-intent[OPT] return-type[OPT] where-clause[OPT]
  iterator-body
\end{syntax}

\begin{syntax}
iterator-name:
  identifier
\end{syntax}

\begin{syntax}
iterator-body:
  block-statement
  yield-statement
\end{syntax}

\begin{syntax}
method-declaration-statement:
  linkage-specifier[OPT] proc-or-iter this-intent[OPT] type-binding[OPT] function-name argument-list[OPT] 
    return-intent[OPT] return-type[OPT] where-clause[OPT] function-body
\end{syntax}

\begin{syntax}
proc-or-iter:
  `proc'
  `iter'
\end{syntax}

\begin{syntax}
this-intent:
  `param'
  `ref'
  `type'
\end{syntax}

\begin{syntax}
type-binding:
  identifier .
\end{syntax}

\begin{syntax}
type-declaration-statement:
  enum-declaration-statement
  class-declaration-statement
  record-declaration-statement
  union-declaration-statement
  type-alias-declaration-statement
\end{syntax}

\begin{syntax}
enum-declaration-statement:
  `enum' identifier { enum-constant-list }
\end{syntax}

\begin{syntax}
enum-constant-list:
  enum-constant
  enum-constant , enum-constant-list[OPT]
\end{syntax}

\begin{syntax}
enum-constant:
  identifier init-part[OPT]
\end{syntax}

\begin{syntax}
init-part:
  = expression
\end{syntax}

\begin{syntax}
class-declaration-statement:
  simple-class-declaration-statement
  external-class-declaration-statement
\end{syntax}

\begin{syntax}
simple-class-declaration-statement:
  `class' identifier class-inherit-list[OPT] { class-statement-list[OPT] }
\end{syntax}

\begin{syntax}
class-inherit-list:
  : class-type-list
\end{syntax}

\begin{syntax}
class-type-list:
  class-type
  class-type , class-type-list
\end{syntax}

\begin{syntax}
class-statement-list:
  class-statement
  class-statement class-statement-list
\end{syntax}

\begin{syntax}
class-statement:
  variable-declaration-statement
  method-declaration-statement
  type-declaration-statement
  empty-statement
\end{syntax}

\begin{syntax}
external-class-declaration-statement:
  `extern' external-name[OPT] simple-class-declaration-statement
\end{syntax}

\begin{syntax}
external-name:
  identifier
  string-literal
\end{syntax}

\begin{syntax}
record-declaration-statement:
  simple-record-declaration-statement
  external-record-declaration-statement
\end{syntax}

\begin{syntax}
simple-record-declaration-statement:
  `record' identifier record-inherit-list[OPT] { record-statement-list }
\end{syntax}

\begin{syntax}
record-inherit-list:
  : record-type-list
\end{syntax}

\begin{syntax}
record-type-list:
  record-type
  record-type , record-type-list
\end{syntax}

\begin{syntax}
record-statement-list:
  record-statement
  record-statement record-statement-list
\end{syntax}

\begin{syntax}
record-statement:
  variable-declaration-statement
  method-declaration-statement
  type-declaration-statement
  empty-statement
\end{syntax}

\begin{syntax}
external-record-declaration-statement:
  `extern' external-name[OPT] simple-record-declaration-statement
\end{syntax}

\begin{syntax}
union-declaration-statement:
  `extern'[OPT] `union' identifier { union-statement-list }
\end{syntax}

\begin{syntax}
union-statement-list:
  union-statement
  union-statement union-statement-list
\end{syntax}

\begin{syntax}
union-statement:
  type-declaration-statement
  procedure-declaration-statement
  iterator-declaration-statement
  variable-declaration-statement
  empty-statement
\end{syntax}

\begin{syntax}
type-alias-declaration-statement:
  privacy-specifier[OPT] `config'[OPT] `type' type-alias-declaration-list ;
  external-type-alias-declaration-statement
\end{syntax}

\begin{syntax}
type-alias-declaration-list:
  type-alias-declaration
  type-alias-declaration , type-alias-declaration-list
\end{syntax}

\begin{syntax}
type-alias-declaration:
  identifier = type-specifier
  identifier
\end{syntax}

\begin{syntax}
external-type-alias-declaration-statement:
  `extern' `type' type-alias-declaration-list ;
\end{syntax}

\begin{syntax}
variable-declaration-statement:
  privacy-specifier[OPT] config-or-extern[OPT] variable-kind variable-declaration-list ;
\end{syntax}

\begin{syntax}
config-or-extern: one of
  `config' $ $ $ $ `extern'
\end{syntax}

\begin{syntax}
variable-kind:
  `param'
  `const'
  `var'
  `ref'
  `const ref'
\end{syntax}

\begin{syntax}
variable-declaration-list:
  variable-declaration
  variable-declaration , variable-declaration-list
\end{syntax}

\begin{syntax}
variable-declaration:
  identifier-list type-part[OPT] initialization-part
  identifier-list type-part no-initialization-part[OPT]
  array-alias-declaration
\end{syntax}

\begin{syntax}
initialization-part:
  = expression
\end{syntax}

\begin{syntax}
type-part:
  : type-specifier
\end{syntax}

\begin{syntax}
no-initialization-part:
  = `noinit'
\end{syntax}

\begin{syntax}
array-alias-declaration:
  identifier reindexing-expression[OPT] => array-expression ;
\end{syntax}

\begin{syntax}
reindexing-expression:
  : [ domain-expression ]
\end{syntax}

\begin{syntax}
array-expression:
  expression
\end{syntax}

\begin{syntax}
remote-variable-declaration-statement:
  `on' expression variable-declaration-statement
\end{syntax}

\begin{syntax}
on-statement:
  `on' expression `do' statement
  `on' expression block-statement
\end{syntax}

\begin{syntax}
cobegin-statement:
  `cobegin' task-intent-clause[OPT] block-statement
\end{syntax}

\begin{syntax}
coforall-statement:
  `coforall' index-var-declaration `in' iteratable-expression task-intent-clause[OPT] `do' statement
  `coforall' index-var-declaration `in' iteratable-expression task-intent-clause[OPT] block-statement
  `coforall' iteratable-expression task-intent-clause[OPT] `do' statement
  `coforall' iteratable-expression task-intent-clause[OPT] block-statement
\end{syntax}

\begin{syntax}
begin-statement:
  `begin' task-intent-clause[OPT] statement
\end{syntax}

\begin{syntax}
sync-statement:
  `sync' statement
  `sync' block-statement
\end{syntax}

\begin{syntax}
serial-statement:
  `serial' expression[OPT] `do' statement
  `serial' expression[OPT] block-statement
\end{syntax}

\begin{syntax}
atomic-statement:
  `atomic' statement
\end{syntax}

\begin{syntax}
forall-statement:
  `forall' index-var-declaration `in' iteratable-expression task-intent-clause[OPT] `do' statement
  `forall' index-var-declaration `in' iteratable-expression task-intent-clause[OPT] block-statement
  `forall' iteratable-expression task-intent-clause[OPT] `do' statement
  `forall' iteratable-expression task-intent-clause[OPT] block-statement
  [ index-var-declaration `in' iteratable-expression task-intent-clause[OPT] ] statement
  [ iteratable-expression task-intent-clause[OPT] ] statement
\end{syntax}

\begin{syntax}
delete-statement:
  `delete' expression ;
\end{syntax}


\cleardoublepage
\markboth{Chapel Language Specification}{Index}
\documentclass[10pt,twoside,titlepage]{article}
\usepackage{color}
\usepackage{times}
\usepackage{fullpage}
\usepackage{graphicx}
\usepackage{listings}
\usepackage{longtable}
\usepackage[nottoc]{tocbibind}
\lstdefinelanguage{chapel}
  {
    morekeywords={
      and, array, atomic,
      begin, bool, break,
      call, class, cobegin, complex, config, const, constructor, continue,
      def, distribute, do, domain,
      else, enum, except,
      for, forall,
      goto,
      if, imag, implements, in, int, inout, _invariant, iterator,
      let, like,
      module,
      nil, not,
      on, or, ordered, otherwise, out,
      param, _private, private, public,
      real, record, _release, repeat, return,
      select, serial, single, subtype, sync
      then, to, type, typeselect,
      uint, union, until, _unordered,
      var, _view,
      when, where, while, with,
      yield
    },
    sensitive=false,
    mathescape=false,
    morecomment=[l]{//},
    morecomment=[s]{/*}{*/},
    morestring=[b]",
}

\lstset{
    basicstyle=\footnotesize\tt,
    keywordstyle=\bf,
    commentstyle=\em,
    showstringspaces=false,
    flexiblecolumns=false,
    numbers=left,
    numbersep=5pt,
    numberstyle=\tiny,
    numberblanklines=false,
    stepnumber=0
  }

\newcommand{\chpl}[1]{\lstinline[language=chapel,basicstyle=\normalsize\tt,keywordstyle=]!#1!}

\lstnewenvironment{chapel}{\lstset{language=chapel,xleftmargin=2pc}}{}

\lstdefinelanguage{syntax}
  {
    sensitive=false,
    mathescape=true,
    basewidth = 0.50em,
    fontadjust=true,
    columns=fullflexible,
    basicstyle=\small,
    keywordstyle=\footnotesize\ttfamily,
    commentstyle=\footnotesize\ttfamily,
    identifierstyle=\small\slshape,
    moredelim=[is][\small\bf]{`}{'},
    literate={-}{{\ttfamily -}}{1}
             {[OPT]}{{{\scriptsize $_{opt}$}}}{2}
  }

\lstnewenvironment{syntax}{\lstset{language=syntax,xleftmargin=2pc}}{}

\newcommand{\sntx}[1]{\lstinline[language=syntax]!#1!}


%% High section numbers require different number widths
\usepackage[titles]{tocloft}
\usepackage{ifpdf}
\ifpdf
\usepackage[pdftex,
            bookmarks,
            plainpages=false,
            breaklinks,
            pdftitle={Chapel Language Specification},
            pdfauthor={Cray Inc, 901 Fifth Avenue Suite 1000, Seattle, WA 98164},
            pdfsubject={Chapel High Productivity Language}
           ]{hyperref}
\else
\usepackage[ps2pdf]{hyperref}
\fi
\setlength{\cftsecnumwidth}{1.7em}
\setlength{\cftsubsecnumwidth}{2.6em}
\setlength{\cftsubsubsecnumwidth}{3.4em}
\setlength{\cftsubsecindent}{1.7em}
\setlength{\cftsubsubsecindent}{4.3em}

\newcommand{\ie}{\emph{i.e.}}
\newcommand{\eg}{\emph{e.g.}}

\newenvironment{TODO} {
\begin{quote}
{\it TODO:}
}{
\end{quote}
}

\newenvironment{example}{
\begin{quote}
{\it Example}.
}{
\end{quote}
}

\newenvironment{note}{
\begin{quote}
{\it Implementors' note}.
}{
\end{quote}
}

\newenvironment{rationale}{
\begin{quote}
{\it Rationale}.
}{
\end{quote}
}

\newenvironment{openissue}{
\begin{quote}
{\it Open issue}.
}{
\end{quote}
}

\newenvironment{craychapel}{
\begin{quote}
{\it Cray's Chapel Implementation}.
}{
\end{quote}
}

\newenvironment{suggestionbox}{
\begin{quote}
{\it Suggestions?}
}{
\end{quote}
}

\newcommand{\rsec}[1]
           {\S\ref{#1}}

% courtesy: http://www.iam.ubc.ca/~newbury/tex/page-set-up.html
\newcommand{\sekshun}[1]
           {
             \section{#1}
             \markboth{Chapel Language Specification}{#1}
           }

\oddsidemargin 0.0in
\evensidemargin 0.5in
\textwidth 6in
\headheight 0.2in
\topmargin 0in
\headsep 0.3in
\textheight 8.5in

\makeindex
\title{Chapel Language Specification 0.782}

\author{Cray Inc\\
901 Fifth Avenue, Suite 1000\\
Seattle, WA 98164}

\date{}

\setcounter{tocdepth}{3}

\begin{document}

\pagestyle{empty}
\pagenumbering{alph}

\ifpdf
\pdfbookmark[1]{Title}{titlepage}
\fi
\maketitle

\cleardoublepage
\include{tm}
\cleardoublepage

\pagestyle{myheadings}
\markboth{Chapel Language Specification}{Chapel Language Specification}
\pagenumbering{roman}

\ifpdf
\pdfbookmark[1]{Table of Contents}{tablecontents}
\fi
\tableofcontents

\cleardoublepage

\pagestyle{myheadings}
\pagenumbering{arabic}

\setlength{\parindent}{0in}
\setlength{\parskip}{4mm plus2mm minus1mm}

\sekshun{Scope}
\label{Scope}

Chapel is a new parallel programming language that is under
development at Cray Inc. in the context of the DARPA High Productivity
Language Systems initiative and the DARPA High Productivity Computing
Systems initiative.

This document specifies the Chapel language.  It is a work in progress
and is not definitive.  In particular, it is not a standard.

\cleardoublepage
\sekshun{Notation}
\label{Notation}

Special notations are used in this specification to denote Chapel code
and to denote Chapel syntax.

Chapel code is represented with a fixed-width font where keywords are
bold and comments are italicized.
\begin{example}
\begin{chapel}
for i in D do   // iterate over domain D
  writeln(i);   // output indices in D
\end{chapel}
\end{example}

Chapel syntax is represented with standard syntax notation in which
productions define the syntax of the language.  A production is
defined in terms of non-terminal ({\it italicized}) and terminal
(non-italicized) symbols.  The complete syntax defines all of the
non-terminal symbols in terms of one another and terminal symbols.

A definition of a non-terminal symbol is a multi-line construct.  The
first line shows the name of the non-terminal that is being defined
followed by a colon.  The next lines before an empty line define the
alternative productions to define the non-terminal.
\begin{example}
The production
\begin{syntax_donotcollect}
bool-literal:
  `true'
  `false'
\end{syntax_donotcollect}
defines \sntx{bool-literal} to be either the symbol \sntx{`true'} or
\sntx{`false'}.
\end{example}
In the event that a single line of a definition needs to break across
multiple lines of text, more indentation is used to indicate that it
is a continuation of the same alternative production.

As a short-hand for cases where there are many alternatives that
define one symbol, the first line of the definition of the
non-terminal may be followed by ``one of'' to indicate that the single
line in the production defines alternatives for each symbol.
\begin{example}
The production
\begin{syntax_donotcollect}
unary-operator: one of
  + - ~ !
\end{syntax_donotcollect}
is equivalent to
\begin{syntax_donotcollect}
unary-operator:
  +
  -
  ~
  !
\end{syntax_donotcollect}
\end{example}

As a short-hand to indicate an optional symbol in the definition of a
production, the subscript ``opt'' is suffixed to the symbol.
\begin{example}
The production
\begin{syntax_donotcollect}
formal:
  formal-tag identifier formal-type[OPT] default-expression[OPT]
\end{syntax_donotcollect}
is equivalent to
\begin{syntax_donotcollect}
formal:
  formal-tag identifier formal-type default-expression
  formal-tag identifier formal-type
  formal-tag identifier default-expression
  formal-tag identifier
\end{syntax_donotcollect}
\end{example}

\cleardoublepage
\sekshun{Organization}
\label{Organization}

This specification is organized as follows:
\begin{itemize}

\item
Chapter~\ref{Scope}, Scope, describes the scope of this specification.

\item
Chapter~\ref{Notation}, Notation, introduces the notation that is used
throughout this specification.

\item
Chapter~\ref{Organization}, Organization, describes the contents of
each of the chapters within this specification.

\item
Chapter~\ref{Acknowledgments}, Acknowledgments, offers a note of
thanks to people and projects.

\item
Chapter~\ref{Language_Overview}, Language Overview, describes Chapel
at a high level.

\item
Chapter~\ref{Lexical_Structure}, Lexical Structure, describes the
lexical components of Chapel.

\item
Chapter~\ref{Types}, Types, describes the types in Chapel and defines
the primitive and enumerated types.

\item
Chapter~\ref{Variables}, Variables, describes variables and constants
in Chapel.

\item
Chapter~\ref{Conversions}, Conversions, describes the legal implicit
and explicit conversions allowed between values of different types.
Chapel does not allow for user-defined conversions.

\item
Chapter~\ref{Expressions}, Expressions, describes the non-parallel
expressions in Chapel.

\item
Chapter~\ref{Statements}, Statements, describes the non-parallel
statements in Chapel.

\item
Chapter~\ref{Modules}, Modules, describes modules in Chapel., Chapel
modules allow for name space management.

\item
Chapter~\ref{Functions}, Functions, describes functions and function
resolution in Chapel.

\item
Chapter~\ref{Tuples}, Tuples, describes tuples in Chapel.

\item
Chapter~\ref{Classes}, Classes, describes reference classes in Chapel.

\item
Chapter~\ref{Records}, Records, describes records or value classes in
Chapel.

\item
Chapter~\ref{Unions}, Unions, describes unions in Chapel.

\item
Chapter~\ref{Ranges}, Ranges, describes ranges in Chapel.

\item
Chapter~\ref{Domains}, Domains, describes domains in Chapel.  Chapel
domains are first-class index sets that support the description of
iteration spaces, array sizes and shapes, and sets of indices.

\item
Chapter~\ref{Arrays}, Arrays, describes arrays in Chapel.  Chapel arrays are
more general than in most languages including support for
multidimensional, sparse, associative, and unstructured arrays.

\item
Chapter~\ref{Iterators}, Iterators, describes iterator functions.

\item
Chapter~\ref{Generics}, Generics, describes Chapel's support for
generic functions and types.

\item
Chapter~\ref{Input_and_Output}, Input and Output, describes support
for input and output in Chapel, including file input and output..

\item
Chapter~\ref{Task_Parallelism_and_Synchronization}, Task Parallelism
and Synchronization, describes task-parallel expressions and
statements in Chapel as well as synchronization constructs and the atomic
statement.

\item
Chapter~\ref{Data_Parallelism}, Data Parallelism, describes
data-parallel expressions and statements in Chapel including
reductions and scans, whole array assignment, and promotion.

\item
Chapter~\ref{Locales}, Locales, describes constructs for managing
locality and executing Chapel programs on distributed-memory systems.

\item
Chapter~\ref{Domain_Maps}, Domain Maps, describes
Chapel's \emph{domain map} construct for defining the layout of
domains and arrays within a single locale and/or the distribution of
domains and arrays across multiple locales.

\item
Chapter~\ref{User_Defined_Reductions_and_Scans}, User-Defined
Reductions and Scans, describes how Chapel programmers can define
their own reduction and scan operators.

\item
Chapter~\ref{User_Defined_Domain_Maps}, User-Defined Domain Maps,
describes how Chapel programmers can define their own domain maps to
implement domains and arrays.

\item
  Chapter~\ref{Memory_Consistency_Model}, Memory Consistency Model,
  describes Chapel's rules for ordering the reads and writes performed
  by a program's tasks.

\item
Chapter~\ref{Interoperability} describes Chapel's interoperability
features for combining Chapel programs with code written in different
languages.

\item
Chapter~\ref{Standard_Modules}, Standard Modules, describes the
standard modules that are provided with the Chapel language.

\item
Chapter~\ref{Standard_Distributions}, Standard Distributions,
describes the standard distributions (multi-locale domain maps) that
are provided with the Chapel language.

\item
Chapter~\ref{Standard_Layouts}, Standard Layouts, describes the
standard layouts (single locale domain maps) that are provided with
the Chapel language.

\item
Appendix~\ref{Syntax}, Collected Lexical and Syntax Productions,
contains the syntax productions listed throughout this specification
in both alphabetical and depth-first order.

\end{itemize}

\cleardoublepage
This is a stub.  This portion of the document does not exist.

\cleardoublepage
\sekshun{Language Overview}
\label{Language_Overview}

In HPC applications, the current dominant parallel programming paradigm 
is characterized by a localized
view of the computation combined with explicit control
over message passing, as exemplified by a combination
of Fortran or C/C++ with MPI. Such a fragmented memory
model provides the programmer with full control over data
distribution and communication, at the expense of productivity,
conciseness, and clarity.

Chapel is a new parallel programming language that 
strives to improve the programmability of parallel computer systems.
It provides a higher level of expression 
than current parallel languages do and it improves the separation between 
algorithmic expression and data structure implementation details. 

Chapel supports a global-view parallel programming model at a high level by 
supporting abstractions for data parallelism, task parallelism, and nested parallelism. 
It supports optimization for the locality of data and computation in the program 
via abstractions for data distribution and data-driven placement of subcomputations. 
It supports code reuse and generality via object-oriented concepts and generic 
programming features. While Chapel borrows concepts from many preceding languages, 
its parallel concepts are most closely based on ideas from High-Performance Fortran 
(HPF), ZPL, and the Cray MTA's extensions to Fortran/C. 

The key features of the Chapel language for productive parallel programming are as 
follows: 
\begin{itemize}
\item {\bf Locale type} - an opaque type used for organizing and referring to 
units of machine locality.
\item {\bf Domains} - first-class index sets that can potentially be distributed 
between multiple locales.  Domains are Chapel's primary vehicle for global-view 
data parallelism.
\item {\bf Arrays} - generalized support for distributed data aggregates, including 
dynamic multidimensional rectilinear arrays, potentially strided and/or sparse 
in each dimension; associative arrays; set- and graph-based arrays.
\item {\bf User-defined distributions} - a capability for users to specify the 
low-level distributed implementation of domains and arrays orthogonally 
to the computations that operate on these concepts.
\item {\bf \chpl{forall} loops and iterators} - concepts for specifying parallel 
iteration in a manner that separates algorithm and implementation.
\item {\bf Index types} - types representing domain indices to support code 
clarity and bounds-checking optimizations.    
\item {\bf User-defined reductions and scans} - a framework for expressing parallel 
prefix operations over data aggregates cleanly and efficiently.
\item {\bf \chpl{cobegin} and \chpl{begin} statements} - statement types for 
supporting task-parallel computations.
\item {\bf Sync and single-assignment variables} - variable types that support 
synchronization between parallel tasks.
\item {\bf Atomic sections} - compound statements that support atomic execution 
from the perspective of other threads.
\item {\bf \chpl{On} clauses} - specifications that support explicit placement of 
data values and computation on the machine's locales.
\item {\bf Value and reference classes} - object-oriented software containers 
that support encapsulation of state and the separation of interfaces from 
implementations.
\item {\bf Function and operator overloading, multiple dispatch, pass-by-argument name, 
default argument values} - concepts that support modern and productive function call 
capabilities.
\item {\bf Type variables and latent types} - capabilities for writing algorithms 
independently of types to support code reuse, exploratory programming, and 
generic functions and data structures.
\item {\bf Modules} - software containers for namespace management and programming 
in-the-large.
\item {\bf Other features for productive programming} - tuples, type-safe unions, 
sequences, etc.
\end{itemize}

\subsection{Motivating Principles}
\label{Motivating_Principles}

Chapel pushes the state-of-the-art in parallel programming
by focusing on productivity and not just performance. In particular
Chapel combines the goal of highest possible
object code performance with that of programmability
by supporting a high level interface resulting in
shorter time-to-solution and reduced application development
cost. The design of Chapel is guided by four key
areas of programming language technology: global-view programming,
locality-awareness, object-orientation, and generic
programming.

\subsubsection{Global View Programming Model}
\label{Global_View_Programming_Model}

Parallel programming models can be divided into two types of models:
{\em fragmented} and {\em global-view}.  Fragmented programming models
require programmers to express algorithms on a task-by-task basis so that
the tasks can execute in parallel.  Global-view programming models
allow programmers to express a parallel algorithm as a whole, similar to
a serial algorithm.  The compiler and runtime libraries identify and assign
tasks to run in parallel across processors.  Chapel provides a global-view 
programming model.

The global-view programming model is is able:
\begin{itemize}
\item to operate on distributed data structures monolithically as 
though they were local to the executing thread's memory, and 
\item to express parallelism within a single source text without requiring 
multiple executables to be run simultaneously. 
\end{itemize}
While the global-view programming model can be implemented on any distributed memory 
machine, specific architectures provide an ideal target for this model.  
Global-view models map particularly well to architectures that 
support a global address space, DGAS and PGAS memory segments, a high performance 
network, lightweight synchronization, and latency-tolerant processors.  This 
synergy results in improved performance as compared to implementations on less-
productive architectures.

We believe that the dominance of the fragmented programming model is the primary
inhibitor of parallel programmability today, and therefore recommend that new
productivity-oriented languages focus on supporting a global view of parallel
programming.  





\subsubsection{Locality Aware Programming}
\label{Locality_Aware_Programming}

Locality-aware programming, in the style of HPF and
ZPL, provides distribution of shared data structures without
requiring a fragmentation of control structure. The programmer
reasons about load-balance and locality by specifying
the placement of data objects and threads.

\subsubsection{Object-Oriented Programming}
\label{Object-Oriented_Programming}

Object-oriented programming helps in managing complexity
by separating common function from specific implementation
to facilitate reuse.

\subsubsection{Generic Programming}
\label{Generic_Programming}

Generic programming and type-inference simplify the
type systems presented to users. High-performance computing
requires type systems to provide data structure details
that allow for efficient implementation. Generic programming
avoids the need for explicit specification of such
details when they can be inferred from the source or from
specialization of program templates.

\subsection{Basic Language Features}
\label{Basic_Language_Features}

Chapel is an imperative programming language.  The basic concepts of the
language should be familiar to users of C, Fortran, Java, Modula and Ada.
However, the syntax of the Chapel language does not directly build upon any of 
these existing languages.   Programmers should start afresh when programming in 
Chapel and not be limited to the constructs of existing languages.

\subsubsection{Getting Started}

Consider the classic first program.  The program file,
\chpl{helloworld.chpl} contains:
\begin{chapel}
def main() {
  writeln("hello, world");
}
\end{chapel}

The syntax of this simple program somewhat resembles C.  There are a
few items to note.
The program contains one module, \chpl{helloworld}, which is implicitly
named from the name of the file.  The keyword \chpl{def} indicates that the
definition of a function follows.

To compile and run this program, execute the following
commands at the system prompt:
\begin{verbatim}
> chpl helloworld.chpl
> ./a.out
\end{verbatim}
The following output is shown:
\begin{verbatim}
> hello, world
\end{verbatim}

\subsubsection{Programs and Modules}
\label{Programs_and_Modules}

All Chapel code is organized using \emph{modules} which serve as code
containers to help manage code complexity as programs grow in size.
One module may ``\chpl{use}'' another, giving it access to that
module's public global symbols.  In the following example, the
standard Chapel module \chpl{Types} is used.  This module contains
the \chpl{numBits} function that returns the size of a Chapel numeric
type in bits. 
%% test:  modulesexample.chpl
\begin{chapel}
use Types;

writeln("The default size of Chapel integers is ",numBits(int)," bits.");
\end{chapel}
The output of this example is:
\begin{verbatim}
The default size of Chapel integers is 32 bits.
\end{verbatim}

For convenience in exploratory programming,
explicit module declarations are not required.  If code is specified 
without a module declaration, the code's
filename is used as the module name for the code that it contains.

All Chapel programs must define a single subroutine named
\fnname{main()} that specifies the entry point for the program.  
This entry point is executed by a single logical thread.

Chapel provides standard modules for bit level operations, computing
random numbers and quering the system time.  See~\rsec{Standard_Modules}
for more details about these modules and how to use them.

\subsubsection{Data Types and Variables}
\label{Data_Types_and_Variables}

The following example demonstrates some variable declarations in Chapel.
\begin{chapel}
config const n = 10;

var x = 1.0,
    y = n:real,
    z: real;
\end{chapel}

The constant \chpl{n} can be set at runtime, as indicated by \chpl{config}, 
or it is set to its default value of \chpl{10}.  It is inferred to be of 
type \chpl{int} from this integer default value.  Similarly, \chpl{x} and 
\chpl{y} are inferred to be of type \chpl{real}.  The variable \chpl{x}
is initialized to \chpl{1.0} and the variable \chpl{y} is initialized to
the value of \chpl{n}, converted from an integer value to a real value.
The variable \chpl{z} has an explicit type declaration.  
Because \chpl{z} is not initialized, it has a default intial value of \chpl{0.0}.

Variable declarations in Chapel include the kind of variable, the variable's
name, type and initial value.  A variable's initialization may be omitted, 
in which case it will be initialized to an value dependent on its definition for 
safety (\eg, ``zero'' for numerical types).  Alternatively, a variable's type 
may be omitted, in which case it will be inferred from its
initializer.

There are three kinds of variables in Chapel specified by the following 
keywords:  \chpl{var}, \chpl{const},
and \chpl{param}.  The optional keyword \chpl{config} may precede
any of these variable keywords.  The \chpl{var} keyword indicates that a
variable is truly ``variable'' and may be modified throughout its
lifetime.  The \chpl{const} keyword indicates that a variable is a
constant, meaning that it \emph{must} be initialized and that its
value cannot change during its lifetime.  Unlike many languages,
Chapel's constant initializers need not be known at compile-time.  The
\chpl{param} keyword is used to define a \emph{parameter}, which is a
compile-time constant.  Parameter values are
required in certain language contexts, such as when specifying a
scalar type's bit-width or an array's rank.  In other contexts,
parameter values can be used to assert to the compiler that a
variable's value is known and unchanging.

Labeling a variable declaration with the optional \chpl{config}
keyword allows its value to be specified on the command line of the
compiler-generated executable (for \chpl{config const} and
\chpl{config var} declarations), or on the command-line of the
compiler itself (for \chpl{config param} declarations).

Variable declarations may also be specified in a variety of
comma-separated ways which allow multiple variables to share the same
variable-kind, type definition or initializer.  

Chapel has support for boolean, integer and floating point primitive types,
including the support for unsigned integers and complex types.  There is
also support for strings as primitive types.  The following table
lists the set of primitive types.  For each type, the default size and
all possible sizes are given.

\begin{center}
\begin{tabular}{|l|l|l|}
\hline
{\bf Type} & {\bf Default Size} & {\bf Supported Sizes}  \\
\hline
\begin{chapel}
int
\end{chapel}
& 32 bits & 
\begin{chapel}
int(8)
int(16) 
int(32) 
int(64) 
\end{chapel} \\
\hline
\begin{chapel}
uint
\end{chapel}
& 32 bits & 
\begin{chapel}
uint(8)
uint(16) 
uint(32) 
uint(64) 
\end{chapel} \\
\hline
\begin{chapel}
real
\end{chapel}
& 64 bits & 
\begin{chapel}
real(32)
real(64)
real(128)
\end{chapel} \\
\hline
\begin{chapel}
imag
\end{chapel} 
& 64 bits & 
\begin{chapel}
imag(32) 
imag(64)
imag(128)
\end{chapel} \\
\hline
\begin{chapel}
complex
\end{chapel}
& 128 bits & 
\begin{chapel}
complex(64)
complex(128)
complex(256)
\end{chapel} \\
\hline
\begin{chapel}
bool
\end{chapel} 
& 1 bit & 
\begin{chapel}
bool
\end{chapel}  \\
\hline
\begin{chapel}
string
\end{chapel}
& unbounded & 
\begin{chapel}
string 
\end{chapel} \\
\hline
\end{tabular}
\end{center}

%% Should locale be listed in the above table?

Chapel is a type-safe language.  When assigning from one type to another, explicit 
casts are often required by the compiler.  Details of implicit and explicit
conversions are discussed in~\rsec{Conversions}.

%% Not sure how much to say about coercions and conversions here.

Beyond these primitive types, there is support for enumerated types, tuples and
unions.  Additionally, arrays, domains, sequences, classes and records are
used in variable declarations as type definitions.

Chapel supports the ability to created named type definitions using
the \chpl{type} keyword.  Like parameter variables, type definitions
must be known at compile-time.  The example below demonstrates a use of a type
definition.

\begin{chapel}
type elemType = real(32);
var alpha: elemType;
\end{chapel}

The first line of the example code defines the identifier 
\typename{elemType} to be
an alias for a \chpl{real(32)}---Chapel's 32-bit floating point type.
The identifier \typename{elemType} may be used to specify a variable's
definition or anywhere else that a type is allowed.  

\subsubsection{Statements and Expressions}
\label{Statements_and_Expressions}

Examples of Chapel statements are given in the following table.

\begin{center}
\begin{tabular}{|l|l|}
\hline
{\bf Statement} & {\bf Example} \\
\hline
Block Statement &
\begin{chapel} % test:  block.chpl
var tau, s, c: real;
const a = 2.0, b = 5.5;
const b = 5.5;
{
tau = -a/b;
s = 1/sqrt(1 + tau*tau);
c = s*tau;
}
writeln("Givens rotation = ", s, " ", c);
\end{chapel} \\
\hline
Expression Statement & 
\begin{chapel} % test:  expstmt.chpl
var denom = 1.0;
var x: real;

testForZero(denom);
testForZero(x);
testForZero(0.0);

def testForZero(x: real) {
  if (x == 0.0) then halt("Value is zero.");
  else writeln("Non-zero value.  Continuing.");   
}
\end{chapel} \\
\hline
Assignment Statement & 
\begin{chapel} % test: assign.chpl
var i: int;

i = 0;
i = i + 1;
i += 1;
writeln(i);
\end{chapel} \\
\hline
Conditional Statement &
\begin{chapel} % test:  cond.chpl
const D = [1..5];
var x, y: [D] real;
var alpha = 2.0;

[i in D] y(i) = 3.0*i;
scale(x, y, alpha);
writeln(x);

def scale(x, y, alpha: real) {
  if (x.numElements != y.numElements) {
    writeln("Error:  Input vectors are not the same length.");
    return;
  }
  if (alpha == 0.0) {
    x = 0.0;
  } else if (alpha == 1.0) {
    x = y;
  } else {
    x = alpha*y;
  }
}
\end{chapel} \\
\hline
Select Statement &
\begin{chapel} % test:  select.chpl
const D = [1..5];
var A: [D] real;

[i in D] A(i) = i;

writeln(getvalue("first",A));
writeln(getvalue("last",A));
writeln(getvalue("middle",A));

def getvalue(pos:string,y) {
  var x = 0.0;
  select pos {
    when "first" do x = y(1);
    when "last" do x = y(y.numElements);
    when "middle" do x = y((y.numElements/2):int + y.numElements%2);
    otherwise writeln("Unrecognized element position");
  }
  return x;
}
\end{chapel} \\
\hline
\end{tabular}

\begin{tabular}{|l|l|}
\hline
{\bf Statement} & {\bf Example} \\
\hline
While and Do While Loops &
\begin{chapel} % test: while.chpl
var t = 11;

writeln("Scope of do while loop:");
do {
  t += 1;
  writeln(t);
} while (t <= 10);

t = 11;
writeln("Scope of while loop:");
while (t <= 10) {
  t += 1;
  writeln(t);
}
\end{chapel} \\
\hline
For Loop &
\begin{chapel} % test: for.chpl
const D = [1..5];
var A: [D] real;

[i in D] A(i) = -i*i;
writeln(norm1(A));

def norm1(x) {
  var norm = 0.0;
  for i in x.domain {
    norm += abs(x(i));
  }
  return norm;
}
\end{chapel} \\
\hline
Use Statement &
\begin{chapel} % test:  use.chpl
use Time;
var programTimer:Timer;

programTimer.start();
writeln("Write one line.");
programTimer.stop();
writeln(programTimer.accumulated);
\end{chapel} \\
\hline
Type Select Statement &
\begin{chapel} % test:  typeselect.chpl
var x = 32, y = 15.5;
var z: int(8);
var coord = (0.0,0.0);
var yes: bool;

writetype(x);
writetype(y);
writetype(z);
writetype(coord);
writetype(yes);
writetype("no");

def writetype(x) {
  type select x {
    when int do writeln("Integer type");
    when uint do writeln("Unsigned integer type");
    when real do writeln("Real type");
    when complex do writeln("Complex type");
    when string do writeln("String type");
    when bool do writeln("Boolean type");
    otherwise writeln("Non-primitive type");
  }
}
\end{chapel} \\
\hline
Empty Statement &
\begin{chapel}
;
\end{chapel} \\
\hline
\end{tabular}
\end{center}

\begin{center}
\begin{tabular}{|l|l|}
\hline
{\bf Expression} & {\bf Example} \\
\hline
Query Expression &
\begin{chapel} % query.chpl
writeln(sumOfThree(1,2,3));
writeln(sumOfThree(4.0,5.0,3.0));

def sumOfThree(x: ?t, y:t, z:t):t {
   var sum: t;

   sum = x + y + z;
   return sum;
}
\end{chapel} \\
\hline
Casts &
\begin{chapel} % casts.chpl
var x, y: complex;
x = 2.56 + 9.0i;
y = (3.12, 8.7): complex;
var z = (4.2, 6.1);

writeln(x);
writeln(y);
writeln(z);

var m = 2: int(64);
var n = 2;
var i = 1;
var j = 1;

while (n > 0) do {
  n *= 2;
  i += 1;
}
while (m > 0) do {
  m *= 2;
  j += 1;
}

writeln("For 32-bit integers, 2 ** (",i,") overflows.");
writeln("For 64-bit integers, 2 ** (",j,") overflows.");
\end{chapel} \\
\hline
Let Expression &
\begin{chapel} % let.chpl
quadsol(3.0,8.0,5.0);
quadsol(3.0,4.0,5.0);

def quadsol(a:real, b:real, c:real) {
  writeln("The solution of ",a,"x^2 + ",b,"x + ",c," = 0 is:");
  if (b*b > 4.0*a*c) {
    var x:  (real, real);

    x = let temp1 = sqrt(b*b - 4.0*a*c), temp2 = 2.0*a in
        ((-b + temp1)/temp2, (-b - temp1)/temp2);

    writeln(x);
  } else {
    var x: (complex, complex);

    x = let temp1 = sqrt(4.0*a*c - b*b)/(2.0*a), temp2 = -b/(2.0*a) in
        ((temp2,temp1):complex,(temp2,-temp1):complex);

    writeln(x);
  }
}
\end{chapel} \\
\hline
Conditional Expression &
\begin{chapel} % condexp.chpl
writehalf(8);
writehalf(21);
writehalf(1000);

def writehalf(i: int) {
  var half = if (i % 2) then i/2 +1 else i/2;
  writeln("Half of ",i," is ",half);
}
\end{chapel} \\
\hline
\end{tabular}
\end{center}

\subsubsection{Structured Data Types}
\label{Structured_Data_Types}

\subsubsection{Functions and Methods}
\label{Functions_and_Methods}

\subsubsection{Sequences and Iterators}
\label{Sequences_and_Iterators}


\subsubsection{Arrays and Domains}
\label{Arrays_and_Domains}

In Chapel, arrays are reference types that are declared using domains.
A domain is a first-class representation of an index space, potentially 
defined to be distributed across multiple locales.   All arrays
declared with a particular domain are indexed and distributed according 
to that domain's specifications.  

The following example shows three arrays that are declared to be
vectors of length \chpl{m} and then used to compute and store a 
scaled addition.
\begin{chapel}
const VectorD: domain(1) = [1..m];
var A, B, C: [VectorD] real;

A = B + alpha * C;
\end{chapel}
The first line declares a constant named \chpl{VectorD} that
is defined to be a \chpl{domain} that is  
1-dimensional, describing indices $\{ 1, 2, \ldots, m \}$.
The next line uses the \chpl{VectorD} domain to declare three
arrays \chpl{A}, \chpl{B}, and \chpl{C} of type \chpl{real}.  The
domain's index set defines the size and shape of these arrays. 

The final line uses whole-array syntax to specify the elementwise 
multiplications, additions, and assignments.  In this case since
all three arrays are declared with the same domain, the compiler
knows that the arrays are the right shape and size to successfully
compute the array addition and can generate the appropriate elementwise
additions.

Whole-array operations like this one are implicitly parallel, if the \chpl{VectorD}
domain were distributed across a set of processors.  For example, 
a block distribution of \chpl{VectorD} would be specified as follows.    
\begin{chapel}
const VectorD: domain(1) distributed(Block) = [1..m];
\end{chapel}
Each processor would perform the operations for the array elements that it owns, 
as defined by \chpl{VectorD}'s distribution since that was the domain
used to define all three arrays.

Arrays may be multi-dimensional if declared with multi-dimensional domains,
and they may be of any type.  
%% Need to qualify previous statement.
Since arrays are reference types, they are passed by reference to functions
where they may be modified and remain modified upon return.  However, assigning
from one array to another merely copies the values from one to the other.  The
two arrays each continue to reference individual arrays. 
%% Is previous statement correct?

The above example uses arithmetic domains and arrays.  Domains may
also be sparse, indefinite, enumerated or opaque.  Subdomains may be
defined to specify a subset of the domain's indices, as in the case of
inner, non-boundary points of a grid.  See~\rsec{Domains_and_Arrays} for
a complete description of domains and arrays.
 
\subsection{Parallel Features}
\label{Parallel_Features}


\subsubsection{Data Parallel Constructs}
\label{Data_Parallel_Constructs}


\subsubsection{Task Parallel Constructs}
\label{Task_Parallel_Constructs}


\subsubsection{Exploiting Data Locality}
\label{Exploiting_Data_Locality}


\subsubsection{Synchronizing and Serializing Tasks}
\label{Synchronizing_and_Serializing_Tasks}


\subsection{Data Distributions}
\label{Data_Distributions}


\cleardoublepage
\sekshun{Lexical Structure}
\label{Lexical_Structure}

This is a stub.  This portion of the document does not exist.

\subsection{Programs}
\label{Programs}

This is a stub.  This portion of the document does not exist.

\subsection{Comments}
\label{Comments}

This is a stub.  This portion of the document does not exist.

\subsection{White Space}
\label{White_Space}

This is a stub.  This portion of the document does not exist.

\subsection{Case Sensitivity}
\label{Case_Sensitivity}

This is a stub.  This portion of the document does not exist.

\subsection{Tokens}
\label{Tokens}

This is a stub.  This portion of the document does not exist.

\subsubsection{Identifiers}
\label{Identifiers}

This is a stub.  This portion of the document does not exist.

\subsubsection{Keywords}
\label{Keywords}

This is a stub.  This portion of the document does not exist.

\subsubsection{Literals}
\label{Literals}

This is a stub.  This portion of the document does not exist.

\subsubsection{Operators and Punctuation}
\label{Operators_and_Punctuation}

This is a stub.  This portion of the document does not exist.

\subsubsection{Grouping Tokens}
\label{Grouping_Tokens}

This is a stub.  This portion of the document does not exist.

\subsection{Compile-Time Conditionals}
\label{Compile-Time_Conditionals}

This is a stub.  This portion of the document does not exist.

\subsection{User-Defined Compiler Errors}
\label{User-Defined_Compiler_Errors}

This is a stub.  This portion of the document does not exist.

\cleardoublepage
\sekshun{Types}
\label{Types}

Chapel is a statically typed language with a rich set of types.  These
include a set of predefined primitive types, enumerated types,
locale types, structured types (classes, records, unions, tuples),
data parallel types (ranges, domains, arrays), and synchronization
types (sync, single).

% This section defines the primitive
% types, enumerated types, and type aliases.  

The syntax of a type is as follows:

\begin{syntax}
type-specifier:
  primitive-type
  enum-type
  locale-type
  structured-type
  dataparallel-type
  synchronization-type
\end{syntax}

Programmers can define their own enumerated types, classes, records,
unions, and type aliases using type declaration statements:

\begin{syntax}
type-declaration-statement:
  enum-declaration-statement
  class-declaration-statement
  record-declaration-statement
  union-declaration-statement
  type-alias-declaration-statement
\end{syntax}

These statements are defined in Sections \rsec{Enumerated_Types},
\rsec{Class_Declarations}, \rsec{Record_Declarations},
\rsec{Union_Declarations}, and \rsec{Type_Aliases}, respectively.

\section{Primitive Types}
\label{Primitive_Types}
\index{types!primitive}

The primitive types include the following types: \chpl{void}, chpl{bool},
\chpl{int}, \chpl{uint}, \chpl{real}, \chpl{imag}, \chpl{complex},
\chpl{string}, and \chpl{locale}.  These primitive types are defined
in this section.

The primitive types are summarized by the following syntax:
\begin{syntax}
primitive-type:
  `void'
  `bool' primitive-type-parameter-part[OPT]
  `int' primitive-type-parameter-part[OPT]
  `uint' primitive-type-parameter-part[OPT]
  `real' primitive-type-parameter-part[OPT]
  `imag' primitive-type-parameter-part[OPT]
  `complex' primitive-type-parameter-part[OPT]
  `string'

primitive-type-parameter-part:
  ( integer-parameter-expression )

integer-parameter-expression:
  expression
\end{syntax}

If present, the parenthesized \sntx{integer-parameter-expression} must
evaluate to a compile-time constant of integer type.  See~\rsec{Compile-Time_Constants}

\begin{openissue}
There is an expectation of future support for larger bit width
primitive types depending on a platform's native support for those
types.
\end{openissue}

\subsection{The Void Type}
\label{The_Void_Type}
\index{void@\chpl{void}}

The \chpl{void} type is used to represent the lack of a value, for
example when a function has no arguments and/or no return type.  

There may be storage associated with a value of type \chpl{void}, in which
case its lifetime obeys the same rules as a value of type \chpl{int}.

\subsection{The Bool Type}
\label{The_Bool_Type}
\index{bool@\chpl{bool}}

Chapel defines a logical data type designated by the symbol
\chpl{bool} with the two predefined values \chpl{true} and
\chpl{false}.  This default boolean type is stored using an
implementation-defined number of bits.  A particular number of bits
can be specified using a parameter value following the \chpl{bool}
keyword, such as \chpl{bool(8)} to request an 8-bit boolean value.
Legal sizes are 8, 16, 32, and 64 bits.

%% The relational operators return values of \chpl{bool} type and the
%% logical operators operate on values of \chpl{bool} type.

Some statements require expressions of \chpl{bool} type and Chapel
supports a special conversion of values to \chpl{bool} type when used
in this context~(\rsec{Implicit_Statement_Bool_Conversions}).

\subsection{Signed and Unsigned Integral Types}
\label{Signed_and_Unsigned_Integral_Types}
\index{uint@\chpl{uint}}
\index{int@\chpl{int}}

The integral types can be parameterized by the number of bits used to
represent them.  Valid bit-sizes are 8, 16, 32, and 64.  
The default signed integral type, \chpl{int}, and the
default unsigned integral type, \chpl{uint}, are 32 bits.

The integral types and their ranges are given in the following table:

\begin{center}
\begin{tabular}{|l|r|r|}
\hline
{\bf Type} & {\bf Minimum Value} & {\bf Maximum Value} \\
\hline
{\tt int(8)} & -128 & 127 \\
{\tt uint(8)} & 0 & 255 \\
{\tt int(16)} & -32768 & 32767 \\
{\tt uint(16)} & 0 & 65535 \\
{\tt int(32)}, {\tt int} & -2147483648 & 2147483647 \\
{\tt uint(32)}, {\tt uint} & 0 & 4294967295 \\
{\tt int(64)} & -9223372036854775808 & 9223372036854775807 \\
{\tt uint(64)} & 0 & 18446744073709551615 \\
\hline
\end{tabular}
\end{center}

The unary and binary operators that are pre-defined over the integral
types operate with 32- and 64-bit precision.  Using these operators on
integral types represented with fewer bits results in a coercion
according to the rules defined in~\rsec{Implicit_Conversions}.

\begin{openissue}
There is on going discussion on whether the default size of the
integral types should be changed to 64 bits.
\end{openissue}


\subsection{Real Types}
\label{Real_Types}
\index{real@\chpl{real}}

Like the integral types, the real types can be parameterized by the
number of bits used to represent them.  The default real
type, \chpl{real}, is 64 bits.  The real types that are supported are
machine-dependent, but usually include \chpl{real(32)} (single
precision) and \chpl{real(64)} (double precision) following the IEEE
754 standard.  

\subsection{Imaginary Types}
\label{Imaginary_Types}
\index{imaginary@\chpl{imaginary}}

The imaginary types can be parameterized by the number of bits used to
represent them.  The default imaginary type, \chpl{imag}, is 64 bits.
The imaginary types that are supported are machine-dependent, but
usually include \chpl{imag(32)} and \chpl{imag(64)}.

\begin{rationale}
The imaginary type is included to avoid numeric instabilities and
under-optimized code stemming from always coercing real values to
complex values with a zero imaginary part.
\end{rationale}

\subsection{Complex Types}
\label{Complex_Types}
\index{complex@\chpl{complex}}

Like the integral and real types, the complex types can be
parameterized by the number of bits used to represent them.  A complex
number is composed of two real numbers so the number of bits used to
represent a complex is twice the number of bits used to represent the
real numbers.  The default complex type, \chpl{complex}, is 128 bits;
it consists of two 64-bit real numbers.  The complex types that are
supported are machine-dependent, but usually
include \chpl{complex(64)} and \chpl{complex(128)}.

The real and imaginary components can be accessed via the methods
\chpl{re} and \chpl{im}.  The type of these components is real.
See~\rsec{Math} for math routines for complex types.

\begin{example}
Given a complex number \chpl{c} with the value \chpl{3.14+2.72i}, the
expressions \chpl{c.re} and \chpl{c.im} refer to \chpl{3.14}
and \chpl{2.72} respectively.
\end{example}

\subsection{The String Type}
\label{The_String_Type}
\index{string@\chpl{string}}

Strings are a primitive type designated by the symbol \chpl{string}
comprised of ASCII characters.  Their length is unbounded.
See~\rsec{Standard} for routines for manipulating strings.


\begin{openissue}
There is an expectation of future support for fixed-length strings.
\end{openissue}

\begin{openissue}
There is an expectation of future support for different character
sets, possibly including internationalization.
\end{openissue}

\section{Enumerated Types}
\label{Enumerated_Types}
\index{enumerated types}

Enumerated types are declared with the following syntax:

\begin{syntax}
enum-declaration-statement:
  `enum' identifier { enum-constant-list }

enum-constant-list:
  enum-constant
  enum-constant , enum-constant-list[OPT]

enum-constant:
  identifier init-part[OPT]

init-part:
  = expression
\end{syntax}

The enumerated type can then be referenced by its name, as summarized
by the following syntax:

\begin{syntax}
enum-type:
  identifier
\end{syntax}

An enumerated type defines a set of named constants that can be
referred to via a member access on the enumerated type.
These constants are treated as parameters of integral type.  Each
enumerated type is a distinct type. If the \sntx{init-part} is
omitted, the \sntx{enum-constant} has an integral value one higher
than the previous \sntx{enum-constant} in the enum, with the first
having the value \chpl{1}.

\begin{chapelexample}{enum.chpl}
The code
\begin{chapel}
enum statesman { Aristotle, Roosevelt, Churchill, Kissinger }
\end{chapel}
defines an enumerated type with four constants.  The function
\begin{chapel}
proc quote(s: statesman) {
  select s {
    when statesman.Aristotle do
       writeln("All paid jobs absorb and degrade the mind.");
    when statesman.Roosevelt do
       writeln("Every reform movement has a lunatic fringe.");
    when statesman.Churchill do
       writeln("A joke is a very serious thing.");
    when statesman.Kissinger do
       { write("No one will ever win the battle of the sexes; ");
         writeln("there's too much fraternizing with the enemy."); }
  }
}
\end{chapel}
\begin{chapelnoprint}
for s in statesman.Aristotle..statesman.Kissinger do
  quote(s:statesman);
\end{chapelnoprint}
\begin{chapeloutput}
All paid jobs absorb and degrade the mind.
Every reform movement has a lunatic fringe.
A joke is a very serious thing.
No one will ever win the battle of the sexes; there's too much fraternizing with the enemy.
\end{chapeloutput}
outputs a quote from the given statesman.  Note that enumerated
constants must be prefixed by the enumerated type and a dot.
\end{chapelexample}


\section{Locale Types}
\label{Locale_Types}
\index{types!locale}

Locale types are summarized by the following syntax:

\begin{syntax}
locale-type:
  `locale'
\end{syntax}

The \chpl{locale} type is defined in~\rsec{The_Locale_Type}.

\begin{openissue}
We expect to support \emph{realms} as another locale type.
\end{openissue}

\section{Structured Types}
\label{Structured_Types}
\index{types!structured}

The structured types are summarized by the following syntax:

\begin{syntax}
structured-type:
  class-type
  record-type
  union-type
  tuple-type
\end{syntax}
% in README.firstClassFns: function-type

Classes are discussed in \rsec{Classes}.  Records are discussed
in \rsec{Records}.  Unions are discussed in \rsec{Unions}.  Tuples are
discussed in \rsec{Tuples}.

\subsection{Class Types}

The class type defines a type that contains variables and constants,
called fields, and functions, called methods.  Classes are defined
in~\rsec{Classes}.  The class type can also contain type aliases and
parameters.  Such a class is generic and is defined
in~\rsec{Generic_Types}.

\subsection{Record Types}

The record type is similar to a class type; the primary difference is
that a record is a value rather than a reference.  Records are defined
in~\rsec{Records}.

\subsection{Union Types}

The union type defines a type that contains one of a set of variables.
Like classes and records, unions may also define methods.  Unions are
defined in~\rsec{Unions}.

\subsection{Tuple Types}

A tuple is a light-weight record that consists of one or more
anonymous fields.  If all the fields are of the same type, the tuple
is homogeneous.  Tuples are defined in~\rsec{Tuples}.

\section{Data Parallel Types}
\label{Data_Parallel_Types}
\index{types!dataparallel}

The data parallel types are summarized by the following syntax:

\begin{syntax}
dataparallel-type:
  range-type
  domain-type
  mapped-domain-type
  array-type
  index-type
\end{syntax}

Ranges and their index types are discussed in \rsec{Ranges}.
Domains and their index types are discussed in \rsec{Domains}.
Arrays are discussed in \rsec{Arrays}.

\subsection{Range Types}

A range defines an integral sequence of some integral type.  Ranges
are defined in \rsec{Ranges}.

\subsection{Domain, Array, and Index Types}
\label{Domain_and_Array_Types}

A domain defines a set of indices. An array defines a set of
elements that correspond to the indices in its domain.
A domain's indicies can be of any type.
Domains, arrays, and their index
types are defined in \rsec{Domains} and \rsec{Arrays}.

\section{Synchronisation Types}
\label{Synchronisation_Types}
\index{types!synchronisation}

The synchronization types are summarized by the following syntax:

\begin{syntax}
synchronization-type:
  sync-type
  single-type
\end{syntax}

Sync and single types are discussed in \rsec{Sync_Variables}
and \rsec{Single_Variables}.

\section{Type Aliases}
\label{Type_Aliases}
\index{type aliases}

Type aliases are declared with the following syntax:
\begin{syntax}
type-alias-declaration-statement:
  `config'[OPT] `type' type-alias-declaration-list ;

type-alias-declaration-list:
  type-alias-declaration
  type-alias-declaration , type-alias-declaration-list

type-alias-declaration:
  identifier = type-specifier
  identifier
\end{syntax}
A type alias is a symbol that aliases the type specified in the
\sntx{type-part}.  A use of a type alias has the same meaning as using
the type specified by \sntx{type-part} directly.

If the keyword \chpl{config} precedes the keyword \chpl{type}, the
type alias is called a configuration type alias.  Configuration type
aliases can be set at compilation time via compilation flags or other
implementation-defined means.  The \chpl{type-specifier} in the
program is ignored if the type-alias is alternatively set.

The \sntx{type-part} is optional in the definition of a class or
record.  Such a type alias is called an unspecified type
alias. Classes and records that contain type aliases, specified or
unspecified, are generic~(\rsec{Type_Aliases_in_Generic_Types}).

\begin{openissue}
There is on going discussion on whether a type alias is a new
type or simply an alias.  The former should enable redefinition of
default values, identity elements, etc.
%hilde
% Would inheritance work?
\end{openissue}

\cleardoublepage
\sekshun{Variables}
\label{Variables}

A variable is a symbol that represents memory.  Chapel is a
statically-typed, type-safe language so every variable has a type that
is known at compile-time and the compiler enforces that values
assigned to the variable can be stored in that variable as specified
by its type.

\subsection{Variable Declarations}
\label{Variable_Declarations}
\index{variables!declarations}

Variables are declared with the following syntax:
\begin{syntax}
variable-declaration-statement:
  `config'[OPT] variable-kind variable-declaration ;

variable-kind: one of
  `param' `const' `var'

variable-declaration-list:
  variable-declaration
  variable-declaration , variable-declaration-list

variable-declaration:
  identifier-list type-part[OPT] initialization-part
  identifier-list type-part

identifier-list:
  identifier
  identifier , identifier-list

type-part:
  : type
  : synchronization-type type

initialization-part:
  = expression
\end{syntax}
A \sntx{variable-declaration-statement} is used to define one or more
variables.  If the statement is a top-level module statement, the
variables are global; otherwise they are local.  Global variables are
discussed in~\rsec{Global_Variables}.  Local variables are discussed
in~\rsec{Local_Variables}.

The optional keyword \chpl{config} specifies that the variables are
configuration variables, described in
Section~\rsec{Configuration_Variables}.

The \sntx{variable-kind} specifies whether the variables are
parameters (\chpl{param}), constants (\chpl{const}), or regular
variables (\chpl{var}).  Parameters are compile-time constants whereas
constants are runtime constants.  Both levels of constants are
discussed in~\rsec{Constants}.

Multiple variables can be defined in the same variable declaration
list.  All variables defined in the same \sntx{identifier-list} are
defined to have the same type and initialization expression.

The \sntx{type-part} of a variable declaration specifies the type of
the variable.  It is optional if the \sntx{initialization-part} is
specified.  If the \sntx{type-part} is omitted, the type of the
variable is inferred using local type inference described
in~\rsec{Local_Type_Inference}.

The \sntx{initialization-part} of a variable declaration specifies an
initial expression to assign to the variable.  If
the \sntx{initialization-part} is omitted, the variable is initialized
to a default value described in~\rsec{Default_Initialization}.

\subsubsection{Default Initialization}
\label{Default_Initialization}
\index{variables!default initialization}

If a variable declaration has no initialization expression, a variable
is initialized to the default value of its type.  The default values
are as follows:
\begin{center}
\begin{tabular}{|l|l|}
\hline
{\bf Type} & {\bf Default Value} \\
\hline
{\tt bool} & {\tt false} \\
{\tt int(*)} & {\tt 0} \\
{\tt uint(*)} & {\tt 0} \\
{\tt real(*)} & {\tt 0.0} \\
{\tt imag(*)} & {\tt 0.0i} \\
{\tt complex(*)} & {\tt 0.0 + 0.0i} \\
{\tt string} & {\tt ""} \\
enums & first enum constant \\
classes & {\tt nil} \\
records & default constructed record \\
sequences & empty sequence \\
arrays & elements are default values \\
tuples & components are default values \\
\hline
\end{tabular}
\end{center}

\subsubsection{Local Type Inference}
\label{Local_Type_Inference}
\index{type inference}

If the type is omitted from a variable declaration, the type of the
variable becomes the type of the initialization expression.

\subsection{Global Variables}
\label{Global_Variables}
\index{variables!global}

Variables declared in statements that are in a module but not in a
function or block within that module are global variables.  Global
variables can be accessed anywhere within that module after the
declaration of that variable.  They can also be accessed in other
modules that use that module.

\subsection{Local Variables}
\label{Local_Variables}
\index{variables!local}

Local variables are variables that are not global.  Local variables
are declared within block statements.  They can only be accessed
within the scope of that block statement (including all inner nested
block statements and functions).

A local variable only exists during the execution of code that lies
within that block statement.  This time is called the lifetime of the
variable.  When execution has finished within that block statement,
the local variable and the storage it represents is removed.
Variables of class type are the sole exception.  Constructors of class
types create storage that is not associated with any scope.  Such
storage is managed automatically as discussed
in~\rsec{Automatic_Memory_Management}.

\subsection{Constants}
\label{Constants}

Constants are divided into two categories: parameters, specified with
the keyword \chpl{param}, are compile-time constants and constants,
specified with the keyword \chpl{const}, are runtime constants.

\subsubsection{Compile-Time Constants}
\label{Compile-Time_Constants}
\index{constants!compile-time}
\index{param@\chpl{param}}
\index{parameters}

A compile-time constant or parameter must have a single value that is
known statically by the compiler.  Parameters are restricted to
primitive and enumerated types.

Parameters can be assigned expressions that are parameter expressions.
Parameter expressions are restricted to the following constructs:
\begin{itemize}
\item
 Literals of primitive type.
\item
 Parenthesized parameter expressions.
\item
 Casts of parameter expressions to primitive or enumerated types.
\item
 Applications of the unary operators \verb@+@, \verb@-@, \verb@!@,
 and \verb@~@ on operands that are bool or integral parameter
 expressions.
\item
 Applications of the binary operators \verb@+@, \verb@-@, \verb@*@, \verb@/@, \verb@%@, \verb@**@, \verb@&&@, \verb@||@, \verb@!@, \verb@&@, \verb@|@, \verb@^@, \verb@~@, \verb@<<@, \verb@>>@, \verb@==@, \verb@!=@, \verb@<=@, \verb@>=@, \verb@<@, and \verb@>@ on operands that are bool or integral parameter expressions.
\item
 The conditional expression where the condition is a parameter and the
 then- and else-expressions are parameters.
\end{itemize}

There is an expectation that parameters will be expanded to more types
and more operations, and that functions that return parameters will be
introduced, in the future.

\subsubsection{Runtime Constants}
\label{Runtime_Constants}
\index{constants!runtime}
\index{const@\chpl{const}}

Constants, as opposed to parameters, do not have the restrictions that
are associated with parameters.  Constants can be any type.  They
require an initialization expression and contain the value of that
expression throughout their lifetime.

Variables of class type that are constants are constant references.
The fields of the class can be modified, but the variable always
points to the object that it was initialized to reference.

\subsection{Configuration Variables}
\label{Configuration_Variables}
\index{variables!configuration}
\index{config@\chpl{config}}

If the keyword \chpl{config} precedes the
keyword \chpl{var}, \chpl{const}, or \chpl{param}, the variable,
constant, or parameter is called a configuration variable,
configuration constant, or configuration parameter respectively.  Such
variables, constants, and parameters must be global.

The initialization of these variables can be set via implementation
dependent means, such as command-line switches or environment
variables.  The initialization expression in the program is ignored if
the initialization is alternatively set.

\index{parameters!configuration}
Configuration parameters are set during compilation time via
compilation flags or other implementation dependent means.
\begin{example}
A configuration parameter is set via a compiler flag.  It may be used
to control the target that is being compiled.  For example, the code
\begin{chapel}
config param target: string = "XT3";
\end{chapel}
sets a string parameter \chpl{target} to \chpl{"XT3"}.  This can be
checked to compile different code for this target.
\end{example}

\cleardoublepage
\sekshun{Conversions}
\label{Conversions}
\index{conversions}

A \emph{conversion} converts an expression of one type to another type,
possibly changing its value.
\index{conversions!source type}
\index{conversions!target type}
We refer to these two types the \emph{source} and \emph{target} types.
Conversions can be either
implicit~(\rsec{Implicit_Conversions}) or
explicit~(\rsec{Explicit_Conversions}).


\section{Implicit Conversions}
\label{Implicit_Conversions}
\index{conversions!implicit}

An \emph{implicit conversion} is a conversion that occurs implicitly,
that is, not due to an explicit specification in the program.
Implicit conversions occur at the locations in the program listed below.
Each location determines the target type.
The source and target types of an implicit conversion must be allowed.
They determine whether and how the expression's value changes.

Implicit conversions are not applied when initializing \chpl{ref} or
\chpl{type} values or for actual arguments passed to \chpl{ref} or
\chpl{type} formal arguments.

\index{conversions!implicit!occurs at}
An implicit conversion occurs at each of the following program locations:

\begin{itemize}
\item In an assignment, the expression on the right-hand side of
      the assignment is converted to the type of the variable
      or another lvalue on the left-hand side of the assignment.

\item The actual argument of a function call or an operator is converted
      to the type of the corresponding formal argument, if the formal's
      intent is \chpl{param}, \chpl{in}, \chpl{const in}, or an abstract intent
      (\rsec{Abstract_Intents}) with the semantics of
      \chpl{in} or \chpl{const in}.

% MPF: This rule doesn't seem to be implemented right now,
%      but rather reflects ideal language design.
\item If the formal argument's intent is \chpl{out}, the formal argument
      is converted to the type of the corresponding actual argument
      upon function return.

\item The return or yield expression within a function without a \chpl{ref}
      return intent is converted to the return type of that function.

\item The condition of a conditional expression,
      conditional statement, while-do or do-while loop statement
      is converted to the boolean type~(\rsec{Implicit_Statement_Bool_Conversions}).
      A special rule defines the allowed source types and
      how the expression's value changes in this case.
\end{itemize}

\index{conversions!implicit!allowed types}
Implicit conversions \emph{are allowed} between
the following source and target types,
as defined in the referenced subsections:

\begin{itemize}
\item numeric and boolean types~(\rsec{Implicit_NumBool_Conversions}),
\item class types~(\rsec{Implicit_Class_Conversions}),
\item integral types in the special case when the expression's value
      is a compile-time constant~(\rsec{Implicit_Compile_Time_Constant_Conversions}), and
\item from an integral or class type to \chpl{bool}
      in certain cases~(\rsec{Implicit_Statement_Bool_Conversions}).
\end{itemize}

In addition,
an implicit conversion from a type to the same type is allowed for any type.
Such conversion does not change the value of the expression.

% TODO: If an implicit conversion is not allowed, it is an error.

Implicit conversion is not transitive. That is, if an implicit conversion
is allowed from type \chpl{T1} to \chpl{T2} and from \chpl{T2} to \chpl{T3},
that by itself does not allow an implicit conversion
from \chpl{T1} to \chpl{T3}.

\subsection{Implicit Numeric and Bool Conversions}
\label{Implicit_NumBool_Conversions}

\index{conversions!numeric}
\index{conversions!implicit!numeric}
Implicit conversions among numeric types are allowed when
all values representable in the source type can also be represented
in the target type, retaining their full precision.
%
%REVIEW: vass: I did not understand the point of the following,
% so I am commenting it out for now.
%When the implicit conversion is from an integral to a real type, source
%types are converted to type \chpl{int} before determining if the
%conversion is valid.
%
In addition, implicit conversions from
types \chpl{int(64)} and \chpl{uint(64)} to types \chpl{real(64)}
and \chpl{complex(128)} are allowed, even though they may result in a loss of
precision.

%REVIEW: hilde
% Unless we are supporting some legacy behavior, I would recommend removing this
% provision.  A loss of precision is a loss of precision, so I would favor
% consistent behavior that does not lead to surprising results.  EVERY ``if''
% costs money: which is to say that if a behavior can be described simply, it can
% be implemented simply.

\begin{rationale}
We allow these additional conversions because they are an important
convenience for application programmers. Therefore we are willing to
lose precision in these cases. The largest real and complex types
are chosen to retain precision as often as as possible.
\end{rationale}

\index{conversions!boolean}
\index{conversions!implicit!boolean}
Any boolean type can be implicitly converted to any other boolean type,
retaining the boolean value.
Any boolean type can be implicitly converted to any integral type
by representing \chpl{false} as 0 and \chpl{true} as 1,
except (if applicable)
a boolean cannot be converted to \chpl{int(1)}.
% Rationale: because 1 cannot be represented by \chpl{int(1)}.

\begin{rationale}
We disallow implicit conversion of a boolean to
a real, imaginary, or complex type because of the following.
We expect that the cases where such a conversion is needed
will more likely be unintended by the programmer.
Marking those cases as errors will draw the programmer's attention.
If such a conversion is actually desired, a cast \rsec{Explicit_Conversions}
can be inserted.
\end{rationale}

Legal implicit conversions with numeric and boolean types
may thus be tabulated as follows:

\begin{center}
\begin{tabular}{l|llllll}
& \multicolumn{6}{c}{Target Type} \\ [4pt]

Source Type  & bool($t$) & uint($t$) & int($t$) & real($t$) & imag($t$) & complex($t$) \\  [3pt]

\cline{1-7} \\

bool($s$)    & all $s,t$ & all $s,t$   & all $s$; $2 \le t$ & & & \\ [7pt]

uint($s$)    & & $s \le t$ & $s < t$   & $s \le mant(t)$   & & $s \le mant(t/2)$   \\ [7pt]

uint(64)     & &           &           & real(64)          & & complex(128)        \\ [7pt]

int($s$)     & &           & $s \le t$ & $s \le mant(t)+1$ & & $s \le mant(t/2)+1$ \\ [7pt]

int(64)      & &           &           & real(64)          & & complex(128)        \\ [7pt]

real($s$)    & & & & $s \le t$ &           & $s \le t/2$ \\ [7pt]

imag($s$)    & & & &           & $s \le t$ & $s \le t/2$ \\ [7pt]

complex($s$) & & & &           &           & $s \le t$   \\ [5pt]

\end{tabular}
\end{center}
Here, $mant(i)$ is the number of bits in the (unsigned) mantissa of
the $i$-bit floating-point type.\footnote{For the IEEE 754 format,
$mant(32)=24$ and $mant(64)=53$.}
%
Conversions for the default integral and real types (\chpl{uint},
\chpl{complex}, etc.) are the same as for their
explicitly-sized counterparts.

\subsection{Implicit Compile-Time Constant Conversions}
\label{Implicit_Compile_Time_Constant_Conversions}
\index{conversions!numeric!parameter}
\index{conversions!implicit!parameter}

The following implicit conversion of a parameter is allowed:
\begin{itemize}
\item A parameter of type \chpl{int(64)} can be implicitly converted
to \chpl{int(8)}, \chpl{int(16)}, \chpl{int(32)}, or any unsigned integral type if the
value of the parameter is within the range of the target type.
\end{itemize}

\subsection{Implicit Statement Bool Conversions}
\label{Implicit_Statement_Bool_Conversions}
\index{conversions!boolean!in a statement}
\index{conversions!implicit!boolean}

In the condition of an if-statement, while-loop, and do-while-loop,
the following implicit conversions to \chpl{bool} are supported:
\begin{itemize}
\item An expression of integral type is taken to be false if it is zero and is true otherwise.
\item An expression of a class type is taken to be false if it is nil and is true otherwise.
\end{itemize}

\section{Explicit Conversions}
\label{Explicit_Conversions}
\index{conversions!explicit}

Explicit conversions require a cast in the code.  Casts are defined
in~\rsec{Casts}.  Explicit conversions are supported between more
types than implicit conversions, but explicit conversions are not
supported between all types.

The explicit conversions are a superset of the implicit conversions.
In addition to the following definitions,
an explicit conversion from a type to the same type is allowed for any type.
Such conversion does not change the value of the expression.

\subsection{Explicit Numeric Conversions}
\label{Explicit_Numeric_Conversions}
\index{conversions!numeric}
\index{conversions!explicit!numeric}

Explicit conversions are allowed from any numeric type or boolean to bytes or
string, and vice-versa.

% A valid \chpl{bool} value behaves like a single unsigned bit.  
When a \chpl{bool} is converted to a \chpl{bool}, \chpl{int}
or \chpl{uint} of equal or larger size, its value is zero-extended to fit the
new representation.  When a \chpl{bool} is converted to a
smaller \chpl{bool}, \chpl{int} or \chpl{uint}, its most significant
bits are truncated (as appropriate) to fit the new representation.

When a \chpl{int}, \chpl{uint}, or \chpl{real} is converted to a \chpl{bool}, the result is \chpl{false} if the number was equal to 0 and \chpl{true} otherwise.
% This has the odd effect that a bool stored in a signed one-bit bitfield would
% change sign without generating a conversion error.  But its subsequent
% conversion back to a bool would yield the original value.
% In regard to supporting bitfields: Be careful what you wish for.

% The source type determines whether a value is zero- or sign-extended.
When an \chpl{int} is converted to a larger \chpl{int} or \chpl{uint}, its value is
sign-extended to fit the new representation.  
When a \chpl{uint} is converted to a larger \chpl{int} or \chpl{uint}, its value
is zero-extended.
When an \chpl{int} or \chpl{uint} is converted to an \chpl{int} or \chpl{uint}
of the same size, its binary representation is unchanged.
When an \chpl{int} or \chpl{uint} is converted to a smaller \chpl{int}
or \chpl{uint}, its value is truncated to fit the new representation.

\begin{future}
There are several kinds of integer conversion which can result in a loss of
precision.  Currently, the conversions are performed as specified, and no error
is reported.  In the future, we intend to improve type checking, so the user can
be informed of potential precision loss at compile time, and actual precision
loss at run time.  Such cases include:
%
% An exception is thrown if the source value cannot be represented in the target type.
When an \chpl{int} is converted to a \chpl{uint} and the original value is
negative;
When a \chpl{uint} is converted to an \chpl{int} and the sign bit of the result
is true;
When an \chpl{int} is converted to a smaller \chpl{int} or \chpl{uint} and any
of the truncated bits differs from the original sign bit;
%
When a \chpl{uint} is converted to a smaller \chpl{int} or \chpl{uint} and any
of the truncated bits is true;
\end{future}

\begin{rationale}
For integer conversions, the default behavior of a program should be to produce
a run-time error if there is a loss of precision.  Thus, cast expressions not only
give rise to a value conversion at run time, but amount to an assertion
that the required precision is preserved.  Explicit conversion procedures would be
available in the run-time library so that one can perform explicit conversions
that result in a loss of precision but do not generate a run-time diagnostic.
\end{rationale}

When converting from a \chpl{real} type to a larger \chpl{real} type, the
represented value is preserved.  When converting from a \chpl{real} type to a
smaller \chpl{real} type, the closest representation in the target type is
chosen.\footnote{When converting to a smaller real type, a loss of precision is \emph{expected}.
Therefore, there is no reason to produce a run-time diagnostic.}

When converting to a \chpl{real} type from an integer type, integer types
smaller than \chpl{int} are first converted to \chpl{int}.  Then, the closest
representation of the converted value in the target type is chosen.  The exact
behavior of this conversion is implementation-defined.

When converting from \chpl{real($k$)} to \chpl{complex($2k$)}, the original
value is copied into the real part of the result, and the imaginary part of the
result is set to zero.  When converting from a \chpl{real($k$)} to
a \chpl{complex($\ell$)} such that $\ell > 2k$, the conversion is performed as
if the original value is first converted to \chpl{real($\ell/2$)} and then
to \chpl{$\ell$}.

The rules for converting from \chpl{imag} to \chpl{complex} are the same as for
converting from real, except that the imaginary part of the result is set using
the input value, and the real part of the result is set to zero.

\subsection{Explicit Tuple to Complex Conversion}
\label{Explicit_Tuple_to_Complex_Conversion}
\index{conversions!tuple to complex}
\index{conversions!explicit!tuple to complex}

A two-tuple of numerical values may be converted to a \chpl{complex} value.  If
the destination type is \chpl{complex(128)}, each member of the two-tuple must
be convertible to \chpl{real(64)}.  If the destination type
is \chpl{complex(64)}, each member of the two-tuple must be convertible
to \chpl{real(32)}.  The first member of the tuple becomes the real part of the
resulting complex value; the second member of the tuple becomes the imaginary
part of the resulting complex value.

\subsection{Explicit Enumeration Conversions}
\label{Explicit_Enumeration_Conversions}
\index{conversions!enumeration}
\index{conversions!explicit!enumeration}

Explicit conversions are allowed from any enumerated type to any
\chpl{bytes} or \chpl{string} and vice-versa, and include \chpl{param} conversions.
For enumerated types that are either concrete or semi-concrete
(\rsec{Enumerated_Types}), conversions are supported between the
enum's constants and any numeric type or \chpl{bool},
including \chpl{param} conversions.  For a semi-concrete enumerated
type, if a conversion is attempted involving a constant with no
underlying integer value, it will generate a compile-time error for
a \chpl{param} conversion or an execution-time error otherwise.

When the target type is an integer type, the value is first converted to its
underlying integer type and then to the target type, following the rules above
for converting between integers.

When the target type is a real, imaginary, or complex type, the value
is first converted to its underlying integer type and then to the
target type.

When the target type is \chpl{bool}, the value is first converted to its
underlying integer type.  If the result is zero, the value of the \chpl{bool}
is \chpl{false}; otherwise, it is \chpl{true}.

When the target type is \chpl{bytes} or \chpl{string}, the value becomes the
name of the enumerator.

When the source type is \chpl{bool}, enumerators corresponding to the values 0
and 1 in the underlying integer type are selected, corresponding to input values
of \chpl{false} and \chpl{true}, respectively.

%REVIEW: hilde
% As with default values for variables of enumerated types, I am pushing for the
% simplest implementation -- in which the conversion does not actually change
% the stored value.  This means that it may be possible for an enumerated variable
% to assume a value that does not correspond to any of its enumerators.  Further
% encouragement to always supply a default clause in your switch statements!

When the source type is a real or integer type, the value is converted to the
target type's underlying integer type.  

The conversion from \chpl{complex} and \chpl{imag} types to an enumerated type is not
permitted.

When the source type is \chpl{bytes} or \chpl{string}, the enumerator whose name
matches value of the input is selected.  If no such enumerator exists, an
\chpl{IllegalArgumentError} is thrown.

\subsection{Explicit Class Conversions}
\label{Explicit_Class_Conversions}
\index{conversions!class}
\index{conversions!explicit!class}

An expression of static class type \chpl{C} can be explicitly
converted to a class type \chpl{D} provided that \chpl{C} is derived
from \chpl{D} or \chpl{D} is derived from \chpl{C}.

When at run time the source expression refers to an instance of \chpl{D}
or it subclass, its value is not changed.  Otherwise, the cast fails and
the result depends on whether or not the destination type is nilable. If
the cast fails and the destination type is not nilable, the cast
expression will throw a \chpl{classCastError}. If the cast fails and the
destination type is nilable, as with \chpl{D?}, then the result will be
\chpl{nil}.

In some cases a new variant of a class type needs to be computed that has
different nilability or memory management strategy. Supposing that
\chpl{T} represents a class type, then these casts may compute a new type:

\begin{itemize}
\item
\chpl{T:owned} - new management is \chpl{owned}, nilability from \chpl{T}

\item
\chpl{T:shared} - new management \chpl{shared}, nilability from \chpl{T}

\item
\chpl{T:borrowed} - new management \chpl{borrowed}, nilability from \chpl{T}

\item
\chpl{T:unmanaged} - new management \chpl{unmanaged}, nilability from \chpl{T}

\item
\chpl{T:class} - non-nilable type with specific concrete or generic management from \chpl{T}

\item
\chpl{T:class?} - nilable type with specific concrete or generic management from \chpl{T}

\item
\chpl{T:owned class} - non-nilable type with \chpl{owned} management
\item
\chpl{T:owned class?} - nilable type with \chpl{owned} management

\item
\chpl{T:shared class} - non-nilable type with \chpl{shared} management
\item
\chpl{T:shared class?} - nilable type with \chpl{shared} management

\item
\chpl{T:borrowed class} - non-nilable type with \chpl{borrowed} management
\item
\chpl{T:borrowed class?} - nilable type with \chpl{borrowed} management

\item
\chpl{T:unmanaged class} - non-nilable type with \chpl{unmanaged} management
\item
\chpl{T:unmanaged class?} - nilable type with \chpl{unmanaged} management

\end{itemize}

\subsection{Explicit Range Conversions}
\label{Explicit_Range_Conversions}
\index{conversions!range}
\index{conversions!explicit!range}

An expression of stridable range type can be explicitly converted
to an unstridable range type, changing the stride to 1 in the process.

\subsection{Explicit Domain Conversions}
\label{Explicit_Domain_Conversions}
\index{conversions!domain}
\index{conversions!explicit!domain}

An expression of stridable domain type can be explicitly converted
to an unstridable domain type, changing all strides to 1 in the process.

\subsection{Explicit String to Bytes Conversions}
\label{Explicit_String_to_Bytes_Conversions}
\index{conversions!string to bytes}
\index{conversions!explicit!string to bytes}

An expression of \chpl{string} type can be explicitly converted to a
\chpl{bytes}. However, the reverse is not possible as a \chpl{bytes} can contain
arbitrary bytes. Instead, \chpl{bytes.decode()} method should be used to produce
a \chpl{string} from a \chpl{bytes}.

\subsection{Explicit Type to String Conversions}
\label{Explicit_Type_to_String_Conversions}
\index{conversions!type to string}
\index{conversions!explicit!type to string}

A type expression can be explicitly converted to a \chpl{string}. The resultant
\chpl{string} is the name of the type.

\begin{chapelexample}{explicit-type-to-string.chpl}
For example:
\begin{chapel}
var x: real(64) = 10.0;
writeln(x.type:string);
\end{chapel}
\begin{chapeloutput}
real(64)
\end{chapeloutput}
This program will print out the string \chpl{"real(64)"}.
\end{chapelexample}

\cleardoublepage
\sekshun{Expressions}
\label{Expressions}
\index{expressions}

Chapel provides the following expressions:

\begin{syntax}
expression:
  literal-expression
  nil-expression
  variable-expression
  enum-constant-expression
  call-expression
  iteratable-call-expression
  member-access-expression
  constructor-call-expression
  query-expression
  cast-expression
  lvalue-expression
  parenthesized-expression
  unary-expression
  binary-expression
  let-expression
  if-expression
  for-expression
  forall-expression
  reduce-expression
  scan-expression
  module-access-expression
  tuple-expression
  tuple-expand-expression
  locale-access-expression
  mapped-domain-expression
\end{syntax}
% in README.firstClassFns: lambda-declaration-expression

Individual expressions are defined in the remainder of this chapter
and additionally as follows:

\begin{itemize}
\item forall, reduce, and scan \rsec{Data_Parallelism}
\item module access \rsec{Explicit_Naming}
\item tuple and tuple expand \rsec{Tuples}
\item locale access \rsec{Querying_the_Locale_of_a_Variable}
\item mapped domain \rsec{Domain_Maps}
\item constructor calls \rsec{Class_New}
\item \chpl{nil} \rsec{Class_nil_value}
\end{itemize}

\section{Literal Expressions}
\label{Literal_Expressions}
\index{literal expressions}
\index{expressions!literal}

A literal value for any of the predefined
types~(\rsec{Primitive_Type_Literals}) is a literal expression.
Literal expressions are given by the following syntax:
\begin{syntax}
literal-expression:
  bool-literal
  integer-literal
  real-literal
  imaginary-literal
  string-literal
  range-literal
  domain-literal
  array-literal
\end{syntax}

\section{Variable Expressions}
\label{Variable_Expressions}
\index{expressions!variable}

A use of a variable, constant, parameter, or formal argument, is
itself an expression.  The syntax of a variable expression is given
by:
\begin{syntax}
variable-expression:
  identifier
\end{syntax}

\section{Enumeration Constant Expression}
\label{Enumeration_Constant_Expression}
\index{expressions!enumeration constant}

A use of an enumeration constant is itself an expression.  Such a
constant must be preceded by the enumeration type name.  The syntax of
an enumeration constant expression is given by:
\begin{syntax}
enum-constant-expression:
  enum-type . identifier
\end{syntax}

For an example of using enumeration constants,
see~\rsec{Enumerated_Types}.

\section{Parenthesized Expressions}
\label{Parenthesized_Expressions}
\index{expressions!parenthesized}

A \sntx{parenthesized-expression} is an expression that is delimited
by parentheses as given by:
\begin{syntax}
parenthesized-expression:
  ( expression )
\end{syntax}
Such an expression evaluates to the expression.  The parentheses are
ignored and have only a syntactical effect.

\section{Call Expressions}
\label{Call_Expressions}
\index{function calls}
\index{expressions!call}

Functions and function calls are defined in~\rsec{Functions}.

\section{Indexing Expressions}
\label{Indexing_Expressions}
\index{indexing}
\index{expressions!indexing}

Indexing, for example into arrays, tuples, and domains,
has the same syntax as a call expression.
 
Indexing is performed by an implicit invocation of the \chpl{this} method
on the value being indexed,
passing the indices as the actual arguments.

\section{Member Access Expressions}
\label{Member_Access_Expressions}
\index{member access}
\index{expressions!member access}

Member access expressions provide access to a field or invoke a method
of an instance of a class, record, or union.
They are defined in \rsec{Class_Field_Accesses} and
\rsec{Class_Method_Calls}, respectively.

\begin{syntax}
member-access-expression:
  field-access-expression
  method-call-expression
\end{syntax}

\section{The Query Expression}
\label{The_Query_Expression}
\index{expressions!type query}
\index{? (type query)@\chpl{?} (type query)}
\index{operators!? (type query)@\chpl{?} (type query)}

A query expression is used to query a type or value within a formal
argument type expression.  The syntax of a query expression is given
by:
\begin{syntax}
query-expression:
  ? identifier[OPT]
\end{syntax}
Querying is restricted to querying the type of a formal argument, the
element type of a formal argument that is an array, the domain of a
formal argument that is an array, the size of a primitive type, or a
type or parameter field of a formal argument type.

The identifier can be omitted.  This is useful for ensuring the
genericity of a generic type that defines default values for all of
its generic fields when specifying a formal argument as discussed
in~\rsec{Formal_Arguments_of_Generic_Type}.

\begin{chapelexample}{query.chpl}
The following code defines a generic function where the type of the
first argument is queried and stored in the type alias \chpl{t} and
the domain of the second argument is queried and stored in the
variable \chpl{D}:
\begin{chapelnoprint}
{ // }
\end{chapelnoprint}
\begin{chapel}
proc foo(x: ?t, y: [?D] t) {
  for i in D do
    y[i] = x;
}
\end{chapel}
\begin{chapelnoprint}
// {
var x = 1.5;
var y: [1..4] x.type;
foo(x, y);
writeln(y);
}
\end{chapelnoprint}
This allows a generic specification of assigning a
particular value to all elements of an array.  The value and the
elements of the array are constrained to be the same type.  This
function can be rewritten without query expression as follows:
\begin{chapelnoprint}
{ // }
\end{chapelnoprint}
\begin{chapel}
proc foo(x, y: [] x.type) {
  for i in y.domain do
    y[i] = x;
}
\end{chapel}
\begin{chapelnoprint}
// {
var x = 1.5;
var y: [1..4] x.type;
foo(x, y);
writeln(y);
}
\end{chapelnoprint}
\begin{chapeloutput}
1.5 1.5 1.5 1.5
1.5 1.5 1.5 1.5
\end{chapeloutput}
\end{chapelexample}

There is an expectation that query expressions will be allowed in more
places in the future.

\section{Casts}
\label{Casts}
\index{casts}
\index{expressions!cast}
\index{: (cast)@\chpl{:} (cast)}
\index{operators!: (cast)@\chpl{:} (cast)}

A cast is specified with the following syntax:
\begin{syntax}
cast-expression:
  expression : type-specifier
\end{syntax}
The expression is converted to the specified type.  A cast expression invokes
the corresponding explicit conversion~(\rsec{Explicit_Conversions}).  A
resolution error occurs if no such conversion exists.

\section{LValue Expressions}
\label{LValue_Expressions}
\index{lvalues}
\index{expressions!lvalue}

An {\em lvalue} is an expression that can be used on the left-hand
side of an assignment statement or on either side of a swap statement,
that can be passed to a formal argument of a function that
has \chpl{out}, \chpl{inout} or \chpl{ref} intent, or that can be returned by a
variable function.  Valid lvalue expressions include the following:
\begin{itemize}
\item
 Variable expressions.
\item
 Member access expressions.
\item
 Call expressions of variable functions.
\item
 Indexing expressions.
\end{itemize}

LValue expressions are given by the following syntax:
\begin{syntax}
lvalue-expression:
  variable-expression
  member-access-expression
  call-expression
  parenthesized-expression
\end{syntax}
The syntax is less restrictive than the definition above.  For
example, not all \sntx{call-expression}s are lvalues.

\section{Precedence and Associativity}
\label{Operator_Precedence_and_Associativity}
\index{operators!precedence}
\index{operators!associativity}
\index{expressions!precedence}
\index{expressions!associativity}

The following table summarizes operator and expression precedence and
associativity.  Operators and expressions listed earlier have higher
precedence than those listed later.
\begin{center}
\begin{tabular}{|l|l|l|}
\hline
{\bf Operator} & {\bf Associativity} & {\bf Use} \\
\hline
\verb@.@ & \multirow{3}{*}{left} & member access \\
\verb@()@ & & function call or access \\
\verb@[]@ & & function call or access \\
\hline
\verb@new@ & right & constructor call \\
\hline
\verb@:@ & left & cast \\
\hline
\verb@**@ & right & exponentiation \\
\hline
\verb@reduce@ & \multirow{3}{*}{left} & reduction \\
\verb@scan@ & & scan \\
\verb@dmapped@ & & domain map application \\
\hline
\verb@!@ & \multirow{2}{*}{right} & logical negation \\
\verb@~@ & & bitwise negation \\
\hline
\verb@*@ & \multirow{3}{*}{left} & multiplication \\
\verb@/@ & & division \\
\verb@%@ & & modulus \\
\hline
unary \verb@+@ & \multirow{2}{*}{right} & positive identity \\
unary \verb@-@ & & negation \\
\hline
\verb@+@ & \multirow{2}{*}{left} & addition \\
\verb@-@ & & subtraction \\
\hline
\verb@<<@ & \multirow{2}{*}{left} & left shift \\
\verb@>>@ & & right shift \\
\hline
\verb@<=@ & \multirow{4}{*}{left} & less-than-or-equal-to comparison \\
\verb@>=@ & & greater-than-or-equal-to comparison \\
\verb@<@ & & less-than comparison \\
\verb@>@ & & greater-than comparison \\
\hline
\verb@==@ & \multirow{2}{*}{left} & equal-to comparison \\
\verb@!=@ & & not-equal-to comparison \\
\hline
\verb@&@ & left & bitwise/logical and \\
\hline
\verb@^@ & left & bitwise/logical xor \\
\hline
\verb@|@ & left & bitwise/logical or \\
\hline
\verb@&&@ & left & short-circuiting logical and \\
\hline
\verb@||@ & left & short-circuiting logical or \\
\hline
\verb@..@ & left & range construction \\
\hline
\verb@in@ & left & forall expression \\
\hline
\verb@by@ & \multirow{2}{*}{left} & range/domain stride application \\
\verb@#@ & & range count application \\
\hline
\verb@if then else@ & \multirow{5}{*}{left} & conditional expression \\
\verb@forall do@ & & forall expression \\
\verb@[ ]@ & & forall expression \\
\verb@for do@ & & for expression \\
\verb@sync single@ & & sync and single type \\
\hline
\verb@,@ & left & comma separated expressions \\
\hline
\end{tabular}
\end{center}

\begin{rationale}
In general, our operator precedence is based on that of the C family
of languages including C++, Java, Perl, and C\#.  We comment on a few
of the differences and unique factors here.

We find that there is tension between the relative precedence of
exponentiation, unary minus/plus, and casts.  The following three
expressions show our intuition for how these expressions should be
parenthesized.

\begin{center}
\begin{tabular}{lcl}
\chpl{-2**4} & wants & \chpl{-(2**4)} \\
\chpl{-2:uint} & wants & \chpl{(-2):uint} \\
\chpl{2:uint**4:uint} & wants & \chpl{(2:uint)**(4:uint)} \\
\end{tabular}
\end{center}

Trying to support all three of these cases results in a
circularity---exponentiation wants precedence over unary minus, unary
minus wants precedence over casts, and casts want precedence over
exponentiation.  We chose to break the circularity by making unary
minus have a lower precedence.  This means that for the second case
above:

\begin{center}
\begin{tabular}{lcl}
\chpl{-2:uint} & requires & \chpl{(-2):uint} \\
\end{tabular}
\end{center}

We also chose to depart from the C family of languages by making unary
plus/minus have lower precedence than binary multiplication, division,
and modulus as in Fortran.  We have found very few cases that
distinguish between these cases.  An interesting one is:

\begin{center}
\begin{tabular}{l}
\chpl{const minint = min(int(32));}\\
\chpl{...-minint/2...}
\end{tabular}
\end{center}

Intuitively, this should result in a positive value, yet C's
precedence rules results in a negative value due to asymmetry in
modern integer representations.  If we learn of cases that argue in
favor of the C approach, we would likely reverse this decision in
order to more closely match C.

We were tempted to diverge from the C precedence rules for the binary
bitwise operators to make them bind less tightly than comparisons.
This would allow us to interpret:

\begin{center}
\begin{tabular}{lcl}
\chpl{a | b == 0} & as & \chpl{(a | b) == 0} \\
\end{tabular}
\end{center}

However, given that no other popular modern language has made this
change, we felt it unwise to stray from the pack.  The typical
rationale for the C ordering is to allow these operators to be used as
non-short-circuiting logical operations.

One final area of note is the precedence of reductions.  Two common
cases tend to argue for making reductions very low or very high in the
precedence table:

\begin{center}
\begin{tabular}{lcl}
\chpl{max reduce A - min reduce A} & wants & \chpl{(max reduce A) - (min reduce A)} \\
\chpl{max reduce A * B} & wants & \chpl{max reduce (A * B)} \\
\end{tabular}
\end{center}

The first statement would require reductions to have a higher
precedence than the arithmetic operators while the second would
require them to be lower.  We opted to make reductions have high
precedence due to the argument that they tend to resemble unary
operators.  Thus, to support our intuition:

\begin{center}
\begin{tabular}{lcl}
\chpl{max reduce A * B} & requires & \chpl{max reduce (A * B)} \\
\end{tabular}
\end{center}

This choice also has the (arguably positive) effect of making the
unparenthesized version of this statement result in an aggregate value
if A and B are both aggregates---the reduction of A results in a
scalar which promotes when being multiplied by B, resulting in an
aggregate.  Our intuition is that users who forget the parenthesis
will learn of their error at compilation time because the resulting
expression is not a scalar as expected.

\end{rationale}

\section{Operator Expressions}
\label{Binary_Expressions}
\label{Unary_Expressions}
\index{expressions!operator}

\index{operators!unary}
\index{expressions!unary operator}
The application of operators to expressions is itself an expression.
The syntax of a unary expression is given by:
\begin{syntax}
unary-expression:
  unary-operator expression

unary-operator: one of
  + - ~ !
\end{syntax}

\index{operators!binary}
\index{expressions!binary operator}
The syntax of a binary expression is given by:
\begin{syntax}
binary-expression:
  expression binary-operator expression

binary-operator: one of
  + - * / % ** & | ^ << >> && || == != <= >= < > `by' #
\end{syntax}

The operators are defined in subsequent sections.

\section{Arithmetic Operators}
\label{Arithmetic_Operators}
\index{operators!arithmetic}

This section describes the predefined arithmetic operators.  These
operators can be redefined over different types using operator
overloading~(\rsec{Function_Overloading}).

For each operator, implicit conversions are applied to the operands of
an operator such that they are compatible with one of the function
forms listed, those listed earlier in the list being given
preference.  If no compatible implicit conversions exist, then a
compile-time error occurs.  In these cases, an explicit cast is required.

All integral arithmetic operators are implemented over integral types
of size 32 and 64 bits only.  For example, adding two 8-bit integers
is done by first converting them to 32-bit integers and then adding
the 32-bit integers.  The result is a 32-bit integer.

\subsection{Unary Plus Operators}
\label{Unary_Plus_Operators}
\index{+ (unary)@\chpl{+} (unary)}
\index{operators!+ (unary)@\chpl{+} (unary)}

The unary plus operators are predefined as follows:
\begin{chapel}
proc +(a: int(8)): int(8)
proc +(a: int(16)): int(16)
proc +(a: int(32)): int(32)
proc +(a: int(64)): int(64)

proc +(a: uint(8)): uint(8)
proc +(a: uint(16)): uint(16)
proc +(a: uint(32)): uint(32)
proc +(a: uint(64)): uint(64)

proc +(a: real(32)): real(32)
proc +(a: real(64)): real(64)

proc +(a: imag(32)): imag(32)
proc +(a: imag(64)): imag(64)

proc +(a: complex(64)): complex(64)
proc +(a: complex(128)): complex(128)
\end{chapel}
For each of these definitions, the result is the value of the operand.

\subsection{Unary Minus Operators}
\label{Unary_Minus_Operators}
\index{operators!negation}
\index{- (unary)@\chpl{-} (unary)}
\index{operators!- (unary)@\chpl{-} (unary)}

The unary minus operators are predefined as follows:
\begin{chapel}
proc -(a: int(8)): int(8)
proc -(a: int(16)): int(16)
proc -(a: int(32)): int(32)
proc -(a: int(64)): int(64)

proc -(a: real(32)): real(32)
proc -(a: real(64)): real(64)

proc -(a: imag(32)): imag(32)
proc -(a: imag(64)): imag(64)

proc -(a: complex(64)): complex(64)
proc -(a: complex(128)): complex(128)
\end{chapel}
For each of these definitions that return a value, the result is the
negation of the value of the operand.  For integral types, this
corresponds to subtracting the value from zero.  For real and
imaginary types, this corresponds to inverting the sign.  For complex
types, this corresponds to inverting the signs of both the real and
imaginary parts.

It is an error to try to negate a value of type \chpl{uint(64)}.  Note
that negating a value of type \chpl{uint(32)} first converts the type
to \chpl{int(64)} using an implicit conversion.

\subsection{Addition Operators}
\label{Addition_Operators}
\index{operators!addition}
\index{+@\chpl{+}}
\index{operators!+@\chpl{+}}

The addition operators are predefined as follows:
\begin{chapel}
proc +(a: int(8), b: int(8)): int(8)
proc +(a: int(16), b: int(16)): int(16)
proc +(a: int(32), b: int(32)): int(32)
proc +(a: int(64), b: int(64)): int(64)

proc +(a: uint(8), b: uint(8)): uint(8)
proc +(a: uint(16), b: uint(16)): uint(16)
proc +(a: uint(32), b: uint(32)): uint(32)
proc +(a: uint(64), b: uint(64)): uint(64)

proc +(a: real(32), b: real(32)): real(32)
proc +(a: real(64), b: real(64)): real(64)

proc +(a: imag(32), b: imag(32)): imag(32)
proc +(a: imag(64), b: imag(64)): imag(64)

proc +(a: complex(64), b: complex(64)): complex(64)
proc +(a: complex(128), b: complex(128)): complex(128)

proc +(a: real(32), b: imag(32)): complex(64)
proc +(a: imag(32), b: real(32)): complex(64)
proc +(a: real(64), b: imag(64)): complex(128)
proc +(a: imag(64), b: real(64)): complex(128)

proc +(a: real(32), b: complex(64)): complex(64)
proc +(a: complex(64), b: real(32)): complex(64)
proc +(a: real(64), b: complex(128)): complex(128)
proc +(a: complex(128), b: real(64)): complex(128)

proc +(a: imag(32), b: complex(64)): complex(64)
proc +(a: complex(64), b: imag(32)): complex(64)
proc +(a: imag(64), b: complex(128)): complex(128)
proc +(a: complex(128), b: imag(64)): complex(128)
\end{chapel}
For each of these definitions that return a value, the result is the
sum of the two operands.

It is a compile-time error to add a value of type \chpl{uint(64)} and
a value of type \chpl{int(64)}.

Addition over a value of real type and a value of imaginary type
produces a value of complex type.  Addition of values of complex type
and either real or imaginary types also produces a value of complex
type.

\subsection{Subtraction Operators}
\label{Subtraction_Operators}
\index{operators!subtraction}
\index{-@\chpl{-}}
\index{operators!-@\chpl{-}}

The subtraction operators are predefined as follows:
\begin{chapel}
proc -(a: int(8), b: int(8)): int(8)
proc -(a: int(16), b: int(16)): int(16)
proc -(a: int(32), b: int(32)): int(32)
proc -(a: int(64), b: int(64)): int(64)

proc -(a: uint(8), b: uint(8)): uint(8)
proc -(a: uint(16), b: uint(16)): uint(16)
proc -(a: uint(32), b: uint(32)): uint(32)
proc -(a: uint(64), b: uint(64)): uint(64)

proc -(a: real(32), b: real(32)): real(32)
proc -(a: real(64), b: real(64)): real(64)

proc -(a: imag(32), b: imag(32)): imag(32)
proc -(a: imag(64), b: imag(64)): imag(64)

proc -(a: complex(64), b: complex(64)): complex(64)
proc -(a: complex(128), b: complex(128)): complex(128)

proc -(a: real(32), b: imag(32)): complex(64)
proc -(a: imag(32), b: real(32)): complex(64)
proc -(a: real(64), b: imag(64)): complex(128)
proc -(a: imag(64), b: real(64)): complex(128)

proc -(a: real(32), b: complex(64)): complex(64)
proc -(a: complex(64), b: real(32)): complex(64)
proc -(a: real(64), b: complex(128)): complex(128)
proc -(a: complex(128), b: real(64)): complex(128)

proc -(a: imag(32), b: complex(64)): complex(64)
proc -(a: complex(64), b: imag(32)): complex(64)
proc -(a: imag(64), b: complex(128)): complex(128)
proc -(a: complex(128), b: imag(64)): complex(128)
\end{chapel}
For each of these definitions that return a value, the result is the
value obtained by subtracting the second operand from the first
operand.

It is a compile-time error to subtract a value of type \chpl{uint(64)}
from a value of type \chpl{int(64)}, and vice versa.

Subtraction of a value of real type from a value of imaginary type,
and vice versa, produces a value of complex type.  Subtraction of
values of complex type from either real or imaginary types, and vice
versa, also produces a value of complex type.

\subsection{Multiplication Operators}
\label{Multiplication_Operators}
\index{operators!multiplication}
\index{operators!*@\chpl{*}}
\index{*@\chpl{*}}

The multiplication operators are predefined as follows:
\begin{chapel}
proc *(a: int(8), b: int(8)): int(8)
proc *(a: int(16), b: int(16)): int(16)
proc *(a: int(32), b: int(32)): int(32)
proc *(a: int(64), b: int(64)): int(64)

proc *(a: uint(8), b: uint(8)): uint(8)
proc *(a: uint(16), b: uint(16)): uint(16)
proc *(a: uint(32), b: uint(32)): uint(32)
proc *(a: uint(64), b: uint(64)): uint(64)

proc *(a: real(32), b: real(32)): real(32)
proc *(a: real(64), b: real(64)): real(64)

proc *(a: imag(32), b: imag(32)): real(32)
proc *(a: imag(64), b: imag(64)): real(64)

proc *(a: complex(64), b: complex(64)): complex(64)
proc *(a: complex(128), b: complex(128)): complex(128)

proc *(a: real(32), b: imag(32)): imag(32)
proc *(a: imag(32), b: real(32)): imag(32)
proc *(a: real(64), b: imag(64)): imag(64)
proc *(a: imag(64), b: real(64)): imag(64)

proc *(a: real(32), b: complex(64)): complex(64)
proc *(a: complex(64), b: real(32)): complex(64)
proc *(a: real(64), b: complex(128)): complex(128)
proc *(a: complex(128), b: real(64)): complex(128)

proc *(a: imag(32), b: complex(64)): complex(64)
proc *(a: complex(64), b: imag(32)): complex(64)
proc *(a: imag(64), b: complex(128)): complex(128)
proc *(a: complex(128), b: imag(64)): complex(128)
\end{chapel}
For each of these definitions that return a value, the result is the
product of the two operands.

It is a compile-time error to multiply a value of type \chpl{uint(64)} and
a value of type \chpl{int(64)}.

Multiplication of values of imaginary type produces a value of real
type.  Multiplication over a value of real type and a value of
imaginary type produces a value of imaginary type.  Multiplication of
values of complex type and either real or imaginary types produces a
value of complex type.

\subsection{Division Operators}
\label{Division_Operators}
\index{operators!division}
\index{/@\chpl{/}}
\index{operators!/@\chpl{/}}

The division operators are predefined as follows:
\begin{chapel}
proc /(a: int(8), b: int(8)): int(8)
proc /(a: int(16), b: int(16)): int(16)
proc /(a: int(32), b: int(32)): int(32)
proc /(a: int(64), b: int(64)): int(64)

proc /(a: uint(8), b: uint(8)): uint(8)
proc /(a: uint(16), b: uint(16)): uint(16)
proc /(a: uint(32), b: uint(32)): uint(32)
proc /(a: uint(64), b: uint(64)): uint(64)

proc /(a: real(32), b: real(32)): real(32)
proc /(a: real(64), b: real(64)): real(64)

proc /(a: imag(32), b: imag(32)): real(32)
proc /(a: imag(64), b: imag(64)): real(64)

proc /(a: complex(64), b: complex(64)): complex(64)
proc /(a: complex(128), b: complex(128)): complex(128)

proc /(a: real(32), b: imag(32)): imag(32)
proc /(a: imag(32), b: real(32)): imag(32)
proc /(a: real(64), b: imag(64)): imag(64)
proc /(a: imag(64), b: real(64)): imag(64)

proc /(a: real(32), b: complex(64)): complex(64)
proc /(a: complex(64), b: real(32)): complex(64)
proc /(a: real(64), b: complex(128)): complex(128)
proc /(a: complex(128), b: real(64)): complex(128)

proc /(a: imag(32), b: complex(64)): complex(64)
proc /(a: complex(64), b: imag(32)): complex(64)
proc /(a: imag(64), b: complex(128)): complex(128)
proc /(a: complex(128), b: imag(64)): complex(128)
\end{chapel}
For each of these definitions that return a value, the result is the
quotient of the two operands.

It is a compile-time error to divide a value of type \chpl{uint(64)} by
a value of type \chpl{int(64)}, and vice versa.

Division of values of imaginary type produces a value of real type.
Division over a value of real type and a value of imaginary type
produces a value of imaginary type.  Division of values of complex
type and either real or imaginary types produces a value of complex
type.

When the operands are integers, the result (quotient) is also an integer.  If \chpl{b}
does not divide \chpl{a} exactly, then there are two candidate quotients $q1$ and $q2$
such that $b * q1$ and $b * q2$ are the two multiples of \chpl{b} closest to \chpl{a}.
The integer result $q$ is the candidate quotient which lies closest to zero.

\subsection{Modulus Operators}
\label{Modulus_Operators}
\index{operators!modulus}
\index{\%@\chpl{\%}}
\index{operators!\%@\chpl{\%}}

The modulus operators are predefined as follows:
\begin{chapel}
proc %(a: int(8), b: int(8)): int(8)
proc %(a: int(16), b: int(16)): int(16)
proc %(a: int(32), b: int(32)): int(32)
proc %(a: int(64), b: int(64)): int(64)

proc %(a: uint(8), b: uint(8)): uint(8)
proc %(a: uint(16), b: uint(16)): uint(16)
proc %(a: uint(32), b: uint(32)): uint(32)
proc %(a: uint(64), b: uint(64)): uint(64)
\end{chapel}
For each of these definitions that return a value, the result is the
remainder when the first operand is divided by the second operand.

The sign of the result is the same as the sign of the dividend \chpl{a}, and the
magnitude of the result is always smaller than that of the divisor \chpl{b}.
For integer operands, the \chpl{\%} and \chpl{/} operators are related by the
following identity:
\begin{chapel}
var q = a / b;
var r = a % b;
writeln(q * b + r == a);    // true
\end{chapel}

It is a compile-time error to take the remainder of a value of
type \chpl{uint(64)} and a value of type \chpl{int(64)}, and vice
versa.

There is an expectation that the predefined modulus operators will be
extended to handle real, imaginary, and complex types in the future.

\subsection{Exponentiation Operators}
\label{Exponentiation_Operators}
\index{operators!exponentiation}
\index{**@\chpl{**}}
\index{operators!**@\chpl{**}}

The exponentiation operators are predefined as follows:
\begin{chapel}
proc **(a: int(8), b: int(8)): int(8)
proc **(a: int(16), b: int(16)): int(16)
proc **(a: int(32), b: int(32)): int(32)
proc **(a: int(64), b: int(64)): int(64)

proc **(a: uint(8), b: uint(8)): uint(8)
proc **(a: uint(16), b: uint(16)): uint(16)
proc **(a: uint(32), b: uint(32)): uint(32)
proc **(a: uint(64), b: uint(64)): uint(64)

proc **(a: real(32), b: real(32)): real(32)
proc **(a: real(64), b: real(64)): real(64)
\end{chapel}
For each of these definitions that return a value, the result is the
value of the first operand raised to the power of the second operand.

It is a compile-time error to take the exponent of a value of
type \chpl{uint(64)} by a value of type \chpl{int(64)}, and vice
versa.

There is an expectation that the predefined exponentiation operators
will be extended to handle imaginary and complex types in the future.

\section{Bitwise Operators}
\label{Bitwise_Operators}
\index{operators!bitwise}

This section describes the predefined bitwise operators.  These
operators can be redefined over different types using operator
overloading~(\rsec{Function_Overloading}).

\subsection{Bitwise Complement Operators}
\label{Bitwise_Complement_Operators}
\index{operators!bitwise!complement}
\index{\~@\chpl{\~}}
\index{operators!\~@\chpl{\~}}

The bitwise complement operators are predefined as follows:
\begin{chapel}
proc ~(a: bool): bool

proc ~(a: int(8)): int(8)
proc ~(a: int(16)): int(16)
proc ~(a: int(32)): int(32)
proc ~(a: int(64)): int(64)

proc ~(a: uint(8)): uint(8)
proc ~(a: uint(16)): uint(16)
proc ~(a: uint(32)): uint(32)
proc ~(a: uint(64)): uint(64)
\end{chapel}
For each of these definitions, the result is the bitwise complement of
the operand.

\subsection{Bitwise And Operators}
\label{Bitwise_And_Operators}
\index{operators!bitwise!and}
\index{&@\chpl{&}}
\index{operators!&@\chpl{&}}

The bitwise and operators are predefined as follows:
\begin{chapel}
proc &(a: bool, b: bool): bool

proc &(a: int(?w), b: int(w)): int(w)
proc &(a: uint(?w), b: uint(w)): uint(w)

proc &(a: int(?w), b: uint(w)): uint(w)
proc &(a: uint(?w), b: int(w)): uint(w)
\end{chapel}
For each of these definitions, the result is
computed by applying the logical and operation to the bits of the
operands.

Chapel allows mixing signed and unsigned integers of the same size
when passing them as arguments to bitwise and.
In the mixed case the result is of the same size as the arguments
and is unsigned.
No run-time error is issued, even if the apparent sign changes as the
required conversions are performed.

\begin{rationale}
The mathematical meaning of integer arguments
is discarded when they are passed to bitwise operators.
Instead the arguments are treated simply as bit vectors.
The bit-vector meaning is preserved when converting
between signed and unsigned of the same size.
The choice of unsigned over signed as the result type in the mixed case
reflects the semantics of standard C.
\end{rationale}

\subsection{Bitwise Or Operators}
\label{Bitwise_Or_Operators}
\index{operators!bitwise!or}
\index{|@\chpl{|}}
\index{operators!|@\chpl{|}}

The bitwise or operators are predefined as follows:
\begin{chapel}
proc |(a: bool, b: bool): bool

proc |(a: int(?w), b: int(w)): int(w)
proc |(a: uint(?w), b: uint(w)): uint(w)

proc |(a: int(?w), b: uint(w)): uint(w)
proc |(a: uint(?w), b: int(w)): uint(w)
\end{chapel}

For each of these definitions, the result is
computed by applying the logical or operation to the bits of the
operands.
Chapel allows mixing signed and unsigned integers of the same size
when passing them as arguments to bitwise or.
No run-time error is issued, even if the apparent sign changes as the
required conversions are performed.

\begin{rationale}
The same as for bitwise and (\rsec{Bitwise_And_Operators}).
\end{rationale}

\subsection{Bitwise Xor Operators}
\label{Bitwise_Xor_Operators}
\index{operators!bitwise!exclusive or}
\index{^@\chpl{^}}
\index{operators!^@\chpl{^}}

The bitwise xor operators are predefined as follows:
\begin{chapel}
proc ^(a: bool, b: bool): bool

proc ^(a: int(?w), b: int(w)): int(w)
proc ^(a: uint(?w), b: uint(w)): uint(w)

proc ^(a: int(?w), b: uint(w)): uint(w)
proc ^(a: uint(?w), b: int(w)): uint(w)
\end{chapel}

For each of these definitions, the result is
computed by applying the XOR operation to the bits of the operands.
Chapel allows mixing signed and unsigned integers of the same size
when passing them as arguments to bitwise xor.
No run-time error is issued, even if the apparent sign changes as the required
conversions are performed.

\begin{rationale}
The same as for bitwise and (\rsec{Bitwise_And_Operators}).
\end{rationale}

\section{Shift Operators}
\label{Shift_Operators}
\index{operators!shift}
\index{<<@\chpl{<<}}
\index{operators!<<@\chpl{<<}}
\index{>>@\chpl{>>}}
\index{operators!>>@\chpl{>>}}

This section describes the predefined shift operators.  These
operators can be redefined over different types using operator
overloading~(\rsec{Function_Overloading}).

The shift operators are predefined as follows:
\begin{chapel}
proc <<(a: int(8), b): int(8)
proc <<(a: int(16), b): int(16)
proc <<(a: int(32), b): int(32)
proc <<(a: int(64), b): int(64)

proc <<(a: uint(8), b): uint(8)
proc <<(a: uint(16), b): uint(16)
proc <<(a: uint(32), b): uint(32)
proc <<(a: uint(64), b): uint(64)

proc >>(a: int(8), b): int(8)
proc >>(a: int(16), b): int(16)
proc >>(a: int(32), b): int(32)
proc >>(a: int(64), b): int(64)

proc >>(a: uint(8), b): uint(8)
proc >>(a: uint(16), b): uint(16)
proc >>(a: uint(32), b): uint(32)
proc >>(a: uint(64), b): uint(64)
\end{chapel}
The type of the second actual argument must be any integral type.

The \chpl{<<} operator shifts the bits of \chpl{a} left by the
integer \chpl{b}.  The new low-order bits are set to zero.

The \chpl{>>} operator shifts the bits of \chpl{a} right by the
integer \chpl{b}.  When \chpl{a} is negative, the new high-order bits
are set to one; otherwise the new high-order bits are set to zero.

The value of \chpl{b} must be non-negative.

\section{Logical Operators}
\label{Logical_Operators}
\index{operators!logical}

This section describes the predefined logical operators.  These
operators can be redefined over different types using operator
overloading~(\rsec{Function_Overloading}).

\subsection{The Logical Negation Operator}
\label{Logical_Negation_Operators}
\index{operators!logical!not}
\index{\!@\chpl{!}}
\index{operators!\!@\chpl{!}}

The logical negation operator is predefined as follows:
\begin{chapel}
proc !(a: bool): bool
\end{chapel}
The result is the logical negation of the operand.

\subsection{The Logical And Operator}
\label{Logical_And_Operators}
\index{operators!logical!and}
\index{&&@\chpl{&&}}
\index{operators!&&@\chpl{&&}}

The logical and operator is predefined over bool type.  It returns
true if both operands evaluate to true; otherwise it returns false.
If the first operand evaluates to false, the second operand is not
evaluated and the result is false.
%% hilde sez: In the interest of supporting parallel execution, we should leave
%% unspecified whether the right operand is evaluated.
%% Where sufficient processing resources are available, it is faster on average
%% to evaluate both the left and right operands and perform the conjunction or
%% disjunction than to block until the value of the left operand is known and
%% only then commence to evaluate the right operand.

The logical and operator over expressions \chpl{a} and \chpl{b} given
by
\begin{chapel}
a && b
\end{chapel}
is evaluated as the expression
\begin{chapel}
if isTrue(a) then isTrue(b) else false
\end{chapel}

The function \chpl{isTrue} is predefined over bool type as follows:
\begin{chapel}
proc isTrue(a:bool) return a;
\end{chapel}
Overloading the logical and operator over other types is accomplished
by overloading the \chpl{isTrue} function over other types.

\subsection{The Logical Or Operator}
\label{Logical_Or_Operators}
\index{operators!logical!or}
\index{||@\chpl{||}}
\index{operators!||@\chpl{||}}


The logical or operator is predefined over bool type.  It returns
true if either operand evaluate to true; otherwise it returns false.
If the first operand evaluates to true, the second operand is not
evaluated and the result is true.

The logical or operator over expressions \chpl{a} and \chpl{b} given
by
\begin{chapel}
a || b
\end{chapel}
is evaluated as the expression
\begin{chapel}
if isTrue(a) then true else isTrue(b)
\end{chapel}

The function \chpl{isTrue} is predefined over bool type as described
in~\rsec{Logical_And_Operators}.  Overloading the logical or operator
over other types is accomplished by overloading the \chpl{isTrue}
function over other types.

\section{Relational Operators}
\label{Relational_Operators}
\index{operators!relational}

This section describes the predefined relational operators.  These
operators can be redefined over different types using operator
overloading~(\rsec{Function_Overloading}).

\subsection{Ordered Comparison Operators}
\label{Ordered_Comparison_Operators}

\index{operators!less than}
\index{<@\chpl{<}}
\index{operators!<@\chpl{<}}
The ``less than'' comparison operators are predefined over numeric
types as follows:
\begin{chapel}
proc <(a: int(8), b: int(8)): bool
proc <(a: int(16), b: int(16)): bool
proc <(a: int(32), b: int(32)): bool
proc <(a: int(64), b: int(64)): bool

proc <(a: uint(8), b: uint(8)): bool
proc <(a: uint(16), b: uint(16)): bool
proc <(a: uint(32), b: uint(32)): bool
proc <(a: uint(64), b: uint(64)): bool

proc <(a: real(32), b: real(32)): bool
proc <(a: real(64), b: real(64)): bool

proc <(a: imag(32), b: imag(32)): bool
proc <(a: imag(64), b: imag(64)): bool
\end{chapel}
The result of \chpl{a < b} is true if \chpl{a} is less than \chpl{b};
otherwise the result is false.

\index{operators!greater than}
\index{>@\chpl{>}}
\index{operators!>@\chpl{>}}
The ``greater than'' comparison operators are predefined over numeric
types as follows:
\begin{chapel}
proc >(a: int(8), b: int(8)): bool
proc >(a: int(16), b: int(16)): bool
proc >(a: int(32), b: int(32)): bool
proc >(a: int(64), b: int(64)): bool

proc >(a: uint(8), b: uint(8)): bool
proc >(a: uint(16), b: uint(16)): bool
proc >(a: uint(32), b: uint(32)): bool
proc >(a: uint(64), b: uint(64)): bool

proc >(a: real(32), b: real(32)): bool
proc >(a: real(64), b: real(64)): bool

proc >(a: imag(32), b: imag(32)): bool
proc >(a: imag(64), b: imag(64)): bool
\end{chapel}
The result of \chpl{a > b} is true if \chpl{a} is greater
than \chpl{b}; otherwise the result is false.

\index{operators!less than or equal}
\index{<=@\chpl{<=}}
\index{operators!<=@\chpl{<=}}
The ``less than or equal to'' comparison operators are predefined over
numeric types as follows:
\begin{chapel}
proc <=(a: int(8), b: int(8)): bool
proc <=(a: int(16), b: int(16)): bool
proc <=(a: int(32), b: int(32)): bool
proc <=(a: int(64), b: int(64)): bool

proc <=(a: uint(8), b: uint(8)): bool
proc <=(a: uint(16), b: uint(16)): bool
proc <=(a: uint(32), b: uint(32)): bool
proc <=(a: uint(64), b: uint(64)): bool

proc <=(a: real(32), b: real(32)): bool
proc <=(a: real(64), b: real(64)): bool

proc <=(a: imag(32), b: imag(32)): bool
proc <=(a: imag(64), b: imag(64)): bool
\end{chapel}
The result of \chpl{a <= b} is true if \chpl{a} is less than or equal
to \chpl{b}; otherwise the result is false.

\index{operators!greater than or equal}
\index{>=@\chpl{>=}}
\index{operators!>=@\chpl{>=}}
The ``greater than or equal to'' comparison operators are predefined
over numeric types as follows:
\begin{chapel}
proc >=(a: int(8), b: int(8)): bool
proc >=(a: int(16), b: int(16)): bool
proc >=(a: int(32), b: int(32)): bool
proc >=(a: int(64), b: int(64)): bool

proc >=(a: uint(8), b: uint(8)): bool
proc >=(a: uint(16), b: uint(16)): bool
proc >=(a: uint(32), b: uint(32)): bool
proc >=(a: uint(64), b: uint(64)): bool

proc >=(a: real(32), b: real(32)): bool
proc >=(a: real(64), b: real(64)): bool

proc >=(a: imag(32), b: imag(32)): bool
proc >=(a: imag(64), b: imag(64)): bool
\end{chapel}
The result of \chpl{a >= b} is true if \chpl{a} is greater than or
equal to \chpl{b}; otherwise the result is false.

The ordered comparison operators are predefined over strings as follows:
\begin{chapel}
proc <(a: string, b: string): bool
proc >(a: string, b: string): bool
proc <=(a: string, b: string): bool
proc >=(a: string, b: string): bool
\end{chapel}
Comparisons between strings are defined based on the ordering of the
character set used to represent the string, which is applied
elementwise to the string's characters in order.


\subsection{Equality Comparison Operators}
\label{Equality_Comparison_Operators}
\index{operators!equality}
\index{==@\chpl{==}}
\index{operators!==@\chpl{==}}
\index{"!=@\chpl{"\"!=}}
\index{operators!"!=@\chpl{"\"!=}}

The equality comparison operators \chpl{==} and \chpl{\!=} are predefined over bool and the
numeric types as follows:
\begin{chapel}
proc ==(a: int(8), b: int(8)): bool
proc ==(a: int(16), b: int(16)): bool
proc ==(a: int(32), b: int(32)): bool
proc ==(a: int(64), b: int(64)): bool

proc ==(a: uint(8), b: uint(8)): bool
proc ==(a: uint(16), b: uint(16)): bool
proc ==(a: uint(32), b: uint(32)): bool
proc ==(a: uint(64), b: uint(64)): bool

proc ==(a: real(32), b: real(32)): bool
proc ==(a: real(64), b: real(64)): bool

proc ==(a: imag(32), b: imag(32)): bool
proc ==(a: imag(64), b: imag(64)): bool

proc ==(a: complex(64), b: complex(64)): bool
proc ==(a: complex(128), b: complex(128)): bool

proc !=(a: int(8), b: int(8)): bool
proc !=(a: int(16), b: int(16)): bool
proc !=(a: int(32), b: int(32)): bool
proc !=(a: int(64), b: int(64)): bool

proc !=(a: uint(8), b: uint(8)): bool
proc !=(a: uint(16), b: uint(16)): bool
proc !=(a: uint(32), b: uint(32)): bool
proc !=(a: uint(64), b: uint(64)): bool

proc !=(a: real(32), b: real(32)): bool
proc !=(a: real(64), b: real(64)): bool

proc !=(a: imag(32), b: imag(32)): bool
proc !=(a: imag(64), b: imag(64)): bool

proc !=(a: complex(64), b: complex(64)): bool
proc !=(a: complex(128), b: complex(128)): bool
\end{chapel}
The result of \chpl{a == b} is true if \chpl{a} and \chpl{b} contain
the same value; otherwise the result is false.  The result of \chpl{a
\!= b} is equivalent to \chpl{\!(a == b)}.

The equality comparison operators are predefined over classes as
follows:
\begin{chapel}
proc ==(a: object, b: object): bool
proc !=(a: object, b: object): bool
\end{chapel}
The result of \chpl{a == b} is true if \chpl{a} and \chpl{b} reference
the same storage location; otherwise the result is false.  The result
of \chpl{a \!= b} is equivalent to \chpl{\!(a == b)}.

Default equality comparison operators are generated for records if the
user does not define them.  These operators are described
in~\rsec{Record_Comparison_Operators}.

\index{== (string)@\chpl{==} (string)}
\index{operators!== (string)@\chpl{==} (string)}
\index{"!= (string)@\chpl{"\"!=} (string)}
\index{operators!"!= (string)@\chpl{"\"!=} (string)}
The equality comparison operators are predefined over strings as
follows:
\begin{chapel}
proc ==(a: string, b: string): bool
proc !=(a: string, b: string): bool
\end{chapel}
The result of \chpl{a == b} is true if the sequence of characters
in \chpl{a} matches exactly the sequence of characters in \chpl{b};
otherwise the result is false.  The result of \chpl{a \!= b} is
equivalent to \chpl{\!(a == b)}.

\section{Miscellaneous Operators}
\label{Miscellaneous_Operators}

This section describes several miscellaneous operators.  These
operators can be redefined over different types using operator
overloading~(\rsec{Function_Overloading}).

\subsection{The String Concatenation Operator}
\label{The_String_Concatenation_Operator}
\index{operators!string concatenation}
\index{operators!concatenation!string}
\index{operators!+ (string)@\chpl{+} (string)}

The string concatenation operator \chpl{+} is predefined over numeric, boolean,
and enumerated types with strings. It casts its operands to string type and
concatenates them together.

\begin{chapelexample}{string-concat.chpl}
The code
\begin{chapelnoprint}
var i:int = 3;
writeln(
\end{chapelnoprint}
\begin{chapel}
"result: "+i
\end{chapel}
\begin{chapelnoprint}
);
\end{chapelnoprint}
\begin{chapeloutput}
result: 3
\end{chapeloutput}
where \chpl{i} is an integer appends the string representation of \chpl{i} to the
string literal \chpl{"result: "}.  If \chpl{i} is \chpl{3}, then the resulting string
would be \chpl{"result: 3"}.
\begin{chapelnoprint}
\end{chapelnoprint}
\end{chapelexample}

\subsection{The By Operator}
\label{The_By_Operator}
\index{by@\chpl{by}}
\index{operators!by@\chpl{by}}

The operator \chpl{by} is predefined on ranges and rectangular domains.
It is described in~\rsec{By_Operator_For_Ranges} for ranges
and~\rsec{Domain_Striding} for domains.

\subsection{The Range Count Operator}
\label{The_Range_Count_Operator}
\index{operators!range!count}
\index{#@\chpl{#}}
\index{operators!#@\chpl{#}}

The operator \chpl{#} is predefined on ranges. It is described
in ~\rsec{Count_Operator}.

\section{Let Expressions}
\label{Let_Expressions}
\index{let@\chpl{let}}
\index{operators!let@\chpl{let}}

A let expression allows variables to be declared at the expression
level and used within that expression.  The syntax of a let expression
is given by:
\begin{syntax}
let-expression:
  `let' variable-declaration-list `in' expression
\end{syntax}
The scope of the variables is the let-expression.
\begin{chapelexample}{let.chpl}
Let expressions are useful for defining variables in the context of
an expression.  In the code
\begin{chapelnoprint}
  var a = 4;
  var b = 5;
  var l =
\end{chapelnoprint}
\begin{chapel}
  let x: real = a*b, y = x*x in 1/y
\end{chapel}
the value determined by \chpl{a*b} is computed and converted to type
real if it is not already a real.  The square of the real is then
stored in \chpl{y} and the result of the expression is the reciprocal
of that value.
\begin{chapelnoprint}
  ;
  writeln(l);
\end{chapelnoprint}
\begin{chapeloutput}
0.0025
\end{chapeloutput}
\end{chapelexample}

\section{Conditional Expressions}
\label{Conditional_Expressions}
\index{conditional expressions}
\index{expressions!conditional}
\index{expressions!if-then-else}
\index{if@\chpl{if}}
\index{then@\chpl{then}}
\index{else@\chpl{else}}

A conditional expression is given by the following syntax:
\begin{syntax}
if-expression:
  `if' expression `then' expression `else' expression
  `if' expression `then' expression
\end{syntax}
The conditional expression is evaluated in two steps.  First, the
expression following the \chpl{if} keyword is evaluated.  Then, if the
expression evaluated to true, the expression following the \chpl{then}
keyword is evaluated and taken to be the value of this expression.
Otherwise, the expression following the \chpl{else} keyword is
evaluated and taken to be the value of this expression.  In both
cases, the unselected expression is not evaluated.

The `else' clause can be omitted only when the conditional expression
is nested immediately inside a for or forall expression.  Such an expression
is used to filter predicates as described
in~\rsec{Filtering_Predicates_For} and~\rsec{Filtering_Predicates_Forall},
respectively.

\begin{chapelexample}{condexp.chpl}
This example shows how if-then-else can be used in a context in which an
expression is expected.
\begin{chapel}
writehalf(8);
writehalf(21);
writehalf(1000);

proc writehalf(i: int) {
  var half = if (i % 2) then i/2 +1 else i/2;
  writeln("Half of ",i," is ",half); 
}
\end{chapel}
\begin{chapelprintoutput}
Half of 8 is 4\\
Half of 21 is 11\\
Half of 1000 is 500\\
\end{chapelprintoutput}
\end{chapelexample}

\section{For Expressions}
\label{For_Expressions}
\index{for@\chpl{for}}
\index{expressions!for@\chpl{for}}

A for expression is given by the following syntax:
\begin{syntax}
for-expression:
  `for' index-var-declaration `in' iteratable-expression `do' expression
  `for' iteratable-expression `do' expression
\end{syntax}
The for expression executes a for loop (\rsec{The_For_Loop}),
evaluates the body expression on each iteration of the loop,
and returns the resulting values as a collection.
The size and shape of that collection
are determined by the iteratable-expression.

\subsection{Filtering Predicates in For Expressions}
\label{Filtering_Predicates_For}
\index{for@\chpl{for}!filtering predicates}
\index{expressions!for@\chpl{for}!filtering predicates}

A conditional expression that is immediately enclosed in a for
expression and does not require an else clause filters the iterations of the for expression.
The iterations for which the condition does not hold
are not reflected in the result of the for expression.

\begin{chapelexample}{yieldPredicates.chpl}
The code
\begin{chapel}
var A = for i in 1..10 do if i % 3 != 0 then i;
\end{chapel}
\begin{chapelpost}
writeln(A);
\end{chapelpost}
\begin{chapeloutput}
1 2 4 5 7 8 10
\end{chapeloutput}
declares an array A that is initialized to the integers between
1 and 10 that are not divisible by 3.
\end{chapelexample}

\cleardoublepage
\sekshun{Statements}
\label{Statements}

\index{statement}

Chapel is an imperative language with statements that may have side
effects.  Statements allow for the sequencing of program execution.
They are as follows:
\begin{syntax}
statement:
  block-statement
  expression-statement
  assignment-statement
  swap-statement
  conditional-statement
  select-statement
  while-do-statement
  do-while-statement
  for-statement
  label-statement
  break-statement
  continue-statement
  param-for-statement
  return-statement
  yield-statement
  module-declaration-statement
  function-declaration-statement
  method-declaration-statement
  type-declaration-statement
  variable-declaration-statement
  remote-variable-declaration-statement
  use-statement
  type-select-statement
  empty-statement
  parallel-statement
  on-statement
\end{syntax}

The declaration statements are discussed in the sections that define
what they declare.  Module declaration statements are defined
in~\rsec{Modules}.  Function declaration statements are defined
in~\rsec{Functions}.  Method declaration statements are defined
in~\rsec{Class_Methods}.  Type declaration statements are defined
in~\rsec{Types}.  Variable declaration statements are defined
in~\rsec{Variables}.  Remote variable declaration statements are
defined in~\rsec{remote_variable_declarations}.  Tuple variable
declaration statements are defined
in~\rsec{Variable_Declarations_in_a_Tuple}.  Return statements are
defined in~\rsec{The_Return_Statement}.  Yield statements are defined
in~\rsec{The_Yield_Statement}.  The \sntx{parallel-statement} consists
of statements that create or limit parallelism.  These statements are
described in~\rsec{Task_Parallelism_and_Synchronization}
and~\rsec{Data_Parallelism}.  The \sntx{on-statement} is defined
in~\rsec{On}.  The compiler error statement is defined
in~\rsec{User_Defined_Compiler_Errors}.

\subsection{Blocks}
\label{Blocks}

\index{block}

A block is a statement or a possibly empty list of statements that
form their own scope.  A block is given by
\begin{syntax}
block-statement:
  { statements[OPT] }

statements:
  statement
  statement statements
\end{syntax}

Variables defined within a block are local
variables~(\rsec{Local_Variables}).

The statements within a block are executed serially unless the block
is in a cobegin statement~(\rsec{Cobegin}).

\subsection{Expression Statements}
\label{Expression_Statements}

\index{expression statement}
\index{expression!as a statement}
The expression statement evaluates an expression solely for side
effects. The syntax for an expression statement is given by
\begin{syntax}
expression-statement:
  expression ;
\end{syntax}

\subsection{Assignment Statements}
\label{Assignment_Statements}
\index{assignment}

An assignment statement assigns the value of an expression to another
expression that can appear on the left-hand side of the operator, for
example, a variable.  Assignment statements are given by

\index{=@\chpl{=}}
\index{+=@\chpl{+=}}
\index{-=@\chpl{-=}}
\index{*=@\chpl{*=}}
\index{/=@\chpl{/=}}
\index{\%=@\chpl{\%=}}
\index{**=@\chpl{**=}}
\index{&=@\chpl{&=}}
\index{|=@\chpl{|=}}
\index{^=@\chpl{^=}}
\index{||=@\chpl{||=}}
\index{&&=@\chpl{&&=}}
\index{<<=@\chpl{<<=}}
\index{>>=@\chpl{>>=}}
\begin{syntax}
assignment-statement:
  lvalue-expression assignment-operator expression

assignment-operator: one of
   = += -= *= /= %= **= &= |= ^= &&= ||= <<= >>=
\end{syntax}

The expression on the left-hand side of the assignment operator must
be a valid lvalue~(\rsec{lvalue}).  It is evaluated before the
expression on the right-hand side of the assignment operator, which
can be any expression.

The assignment operators that contain a binary operator as a prefix is
a short-hand for applying the binary operator to the left and
right-hand side expressions and then assigning the value of that
application to the already evaluated left-hand side.  Thus, for
example, \chpl{x += y} is equivalent to \chpl{x = x + y} where the
expression \chpl{x} is evaluated once.

In a compound assignment, a cast to the type on the left-hand side is
inserted before the simple assignment if the operator is a shift or
both the right-hand side expression can be assigned to the left-hand
side expression and the type of the left-hand side is a primitive
type.

\begin{rationale}
This cast is necessary to handle \chpl{+=} where the type of the
left-hand side is, for example, \chpl{int(8)} because the \chpl{+}
operator is defined on \chpl{int(32)}, not \chpl{int(8)}.
\end{rationale}

Values of one primitive or enumerated type can be assigned to another
primitive or enumerated type if an implicit coercion exists between
those types~(\rsec{Implicit_Conversions}).

The validity and semantics of assigning between
classes~(\rsec{Class_Assignment}), records~(\rsec{Record_Assignment}),
unions~(\rsec{Union_Assignment}), tuples~(\rsec{Tuple_Assignment}),
ranges~(\rsec{Range_Assignment}),
domains~(\rsec{Domain_Assignment}), and arrays~(\rsec{Array_Assignment})
is discussed in these later sections.

\subsection{The Swap Statement}
\label{The_Swap_Statement}
\index{swap!statement}
\index{swap!operator}
The swap statement indicates to swap the values in the expressions
on either side of the swap operator.  Since both expressions are assigned
to, each must be a valid lvalue expression~(\rsec{lvalue}).
\begin{syntax}
swap-statement:
  lvalue-expression swap-operator lvalue-expression

swap-operator:
  <=>
\end{syntax}

To implement the swap operation, the compiler uses temporary variables
as necessary.

\begin{example}
The following swap statement
\begin{chapel}
var a, b: real;

a <=> b;
\end{chapel}
is semantically equivalent to:
\begin{chapel}
const t = b;
b = a;
a = t;
\end{chapel}
\end{example}

\subsection{The Conditional Statement}
\label{The_Conditional_Statement}

\index{if@\chpl{if}}
\index{then@\chpl{then}}
\index{else@\chpl{else}}
\index{conditional!statement}
The conditional statement allows execution to choose between two
statements based on the evaluation of an expression of \chpl{bool}
type. The syntax for a conditional statement is given by
\begin{syntax}
conditional-statement:
  `if' expression `then' statement else-part[OPT]
  `if' expression block-statement else-part[OPT]

else-part:
  `else' statement
\end{syntax}

A conditional statement evaluates an expression of bool type. If the
expression evaluates to true, the first statement in the conditional
statement is executed.  If the expression evaluates to false and the
optional else-clause exists, the statement following the
\chpl{else} keyword is executed.

If the expression is a parameter, the conditional statement is folded
by the compiler. If the expression evaluates to true, the first
statement replaces the conditional statement. If the expression
evaluates to false, the second statement, if it exists, replaces the
conditional statement; if the second statement does not exist, the
conditional statement is removed.

\index{conditional statement!dangling else}
If the statement that immediately follows the optional \chpl{then}
keyword is a conditional statement and it is not in a block, the
else-clause is bound to the nearest preceding conditional statement
without an else-clause.

Each statement embedded in the {\em conditional-statement} has its own
scope whether or not an explicit block surrounds it.

\subsection{The Select Statement}
\label{The_Select_Statement}

\index{select@\chpl{select}}
\index{when@\chpl{when}}

The select statement is a multi-way variant of the conditional
statement.  The syntax is given by:
\begin{syntax}
select-statement:
  `select' expression { when-statements }

when-statements:
  when-statement
  when-statement when-statements

when-statement:
  `when' expression-list `do' statement
  `when' expression-list block-statement
  `otherwise' statement

expression-list:
  expression
  expression , expression-list
\end{syntax}
The expression that follows the keyword \chpl{select}, the select
expression, is compared with the list of expressions following the
keyword \chpl{when}, the case expressions, using the equality
operator \chpl{==}.  If the expressions cannot be compared with the
equality operator, a compile-time error is generated.  The first case
expression that contains an expression where that comparison
is \chpl{true} will be selected and control transferred to the
associated statement.  If the comparison is always \chpl{false}, the
statement associated with the keyword \chpl{otherwise}, if it exists,
will be selected and control transferred to it.  There may be at most
one \chpl{otherwise} statement and its location within the select
statement does not matter.

Each statement embedded in the {\em when-statement} has its own scope
whether or not an explicit block surrounds it.

\subsection{The While and Do While Loops}
\label{The_While_and_Do_While_Loops}

\index{while loops}
\index{while@\chpl{while}}

There are two variants of the while loop in Chapel.  The syntax of the
while-do loop is given by:
\begin{syntax}
while-do-statement:
  `while' expression `do' statement
  `while' expression block-statement
\end{syntax}
The syntax of the do-while loop is given by:
\begin{syntax}
do-while-statement:
  `do' statement `while' expression ;
\end{syntax}
In both variants, the expression evaluates to a value of type \chpl{bool}
which determines when the loop terminates and control continues with
the statement following the loop.

The while-do loop is executed as follows:
\begin{enumerate}
\item The expression is evaluated.
\item If the expression evaluates to \chpl{false}, the statement is
  not executed and control continues to the statement following the
  loop.
\item If the expression evaluates to \chpl{true}, the statement is
  executed and control continues to step 1, evaluating the expression
  again.
\end{enumerate}

The do-while loop is executed as follows:
\begin{enumerate}
\item The statement is executed.
\item The expression is evaluated.
\item If the expression evaluates to \chpl{false}, control continues
  to the statement following the loop.
\item If the expression evaluates to \chpl{true}, control continues to
  step 1 and the the statement is executed again.
\end{enumerate}
In this second form of the loop, note that the statement is executed
unconditionally the first time.

\subsection{The For Loop}
\label{The_For_Loop}

\index{for@\chpl{for}}
\index{for loops}

The for loop iterates over ranges, domains, arrays, iterators, or any
class that implements an iterator named \chpl{these}.  The syntax of
the for loop is given by:
\begin{syntax}
for-statement:
  `for' loop-control-part loop-body-part

loop-control-part:
  index-expression `in' iterator-expression
  iterator-expression

loop-body-part:
  `do' statement
  block-statement

index-expression:
  expression

iterator-expression:
  expression
\end{syntax}

The index-expression declares new variables for the scope of the loop.
It may specify a new identifier.  Alternatively, the index-expression
may specify multiple identifiers grouped using a tuple notation in
order to destructure the values returned by the iterator expression,
as described in~\rsec{Indices_in_a_Tuple}.

The index-expression is optional and may be omitted if the indices do
not need to be referenced in the loop.

If the iterator-expression is a tuple delimited by parentheses, the
components of the tuple must support iteration, e.g., a tuple of
arrays, and those components are iterated over using a zipper
iteration defined in~\rsec{Zipper_Iteration}.  If the
iterator-expression is a tuple delimited by brackets, the components
of the tuple must support iteration and these components are iterated
over using a tensor product iteration defined
in~\rsec{Tensor_Product_Iteration}.

\subsubsection{Zipper Iteration}
\label{Zipper_Iteration}
\index{zipper iteration}

When multiple iterators are iterated over in a zipper context, on each
iteration, each expression is iterated over, the values are returned
by the iterators in a tuple and assigned to the index, and the
statement is executed.

The shape of each iterator, the rank and the extents in each
dimension, must be identical.

\begin{example}
The output of
\begin{chapel}
for (i, j) in (1..3, 4..6) do
  write(i, " ", j, " ");
\end{chapel}
is ``1 4 2 5 3 6 ''.
\end{example}

\subsubsection{Tensor Product Iteration}
\label{Tensor_Product_Iteration}
\index{tensor product iterator}
When multiple iterators are iterated over in a tensor product context,
they are iterated over as if they were nested in distinct for loops.
There is no constraint on the iterators as there is in the zipper
context.

\begin{example}
The output of
\begin{chapel}
for (i, j) in [1..3, 4..6] do
  write(i, " ", j, " ");
\end{chapel}
is ``1 4 1 5 1 6 2 4 2 5 2 6 3 4 3 5 3 6 ''. The statement is
equivalent to
\begin{chapel}
for i in 1..3 do
  for j in 4..6 do
    write(i, " ", j, " ");
\end{chapel}
\end{example}

\subsubsection{Parameter For Loops}
\label{Parameter_For_Loops}

\index{for loops!parameters}
\index{for@\chpl{for}}
\index{param@\chpl{param}}

Parameter for loops are unrolled by the compiler so that the index
variable is a parameter rather than a variable.  The syntax for a
parameter for loop statement is given by:
\begin{syntax}
param-iterator-expression:
  range-literal
  range-literal `by' integer-literal

param-for-statement:
  `for' `param' identifier `in' param-iterator-expression `do' statement
  `for' `param' identifier `in' param-iterator-expression block-statement
\end{syntax}
Parameter for loops are restricted to iteration over range literals
with an optional by expression where the bounds and stride must be
parameters.  The loop is then unrolled for each iteration.

\subsection{The Label, Break, and Continue Statements}
\label{Label_Break_Continue}
\index{label@\chpl{label}}
\index{break@\chpl{break}}
\index{continue@\chpl{continue}}

The label-statement is used to name a specific loop which can then
be the target of a break- or continue-statement.  If a break-
or continue-statement has no label, the target is the lexically
inner-most loop. Labels can only be given to for-, while-do- and
do-while-statements.

The syntax for label, break, and continue statements is given by:
\begin{syntax}
label-statement:
  `label' identifier statement

break-statement:
  `break' identifier[OPT] ;

continue-statement:
  `continue' identifier[OPT] ;
\end{syntax}

If a break-statement is encountered, control will be transferred to
after the associated loop.  If a continue-statement is encountered,
control will be transferred to the end of the associated loop, but
still inside the loop.  Break-statements cannot be used to break out of
parallel loops.  Neither break- nor continue-statements can
cross out of cobegin-, coforall-, begin-, or sync-statements.

\begin{example}
In the following code, the index of the first element in each row of
\chpl{A} that is equal to \chpl{findVal} is printed.  Once a match is
found, the continue statement is executed causing the outer loop to
move to the next row.
\begin{chapel}
label outer for i in 1..n {
  for j in 1..n {
    if A[i, j] == findVal {
      writeln("index: ", (i, j), " matches.");
      continue outer;
    }
  }
}
\end{chapel}
\end{example}

\subsection{The Use Statement}
\label{The_Use_Statement}
\index{use@\chpl{use}}
\index{modules!using}

The use statement makes symbols in modules available without accessing
them via the module name.  The syntax of the use statement is given
by:
\begin{syntax}
use-statement:
  `use' module-name-list ;

module-name-list:
  module-name
  module-name , module-name-list

module-name:
  identifier
  module-name . module-name
\end{syntax}
The use statement makes symbols in each listed module's scope available
from the scope where the use statement occurs.

Symbols injected by a use statement are at an outer scope from those
defined directly in the scope where the use statement occurs, but at a
nearer scope than symbols defined in the scope containing the scope where
the use statement occurs.

If used modules themselves use other modules, symbols are scoped according
the depth of use statements followed to find them. It is an error for two
variables, types, or modules to be defined at the same depth.

\begin{openissue}
There is an expectation that this statement will be extended to allow
the programmer to restrict which symbols are 'used' as well as to
rename symbols that are used.
\end{openissue}

\subsection{The Type Select Statement}
\label{The_Type_Select_Statement}

\index{type select statements}

A type select statement has two uses.  It can be used to determine the
type of a union, as discussed
in~\rsec{The_Type_Select_Statement_and_Unions}.  In its more general
form, it can be used to determine the types of one or more values
using the same mechanisms used to disambiguate function definitions.
It syntax is given by:
\begin{syntax}
type-select-statement:
  `type' `select' expression-list { type-when-statements }

type-when-statements:
  type-when-statement
  type-when-statement type-when-statements

type-when-statement:
  `when' type-list `do' statement
  `when' type-list block-statement
  `otherwise' statement

expression-list:
  expression
  expression , expression-list

type-list:
  type-specifier
  type-specifier , type-list
\end{syntax}

Call the expressions following the keyword \chpl{select}, the select
expressions.  The number of select expressions must be equal to the
number of types following each of the \chpl{when} keywords.  Like the
select statement, one of the statements associated with a \chpl{when}
will be executed.  In this case, that statement is chosen by the
function resolution mechanism.  The select expressions are the actual
arguments, the types following the \chpl{when} keywords are the types
of the formal arguments for different anonymous functions.  The
function that would be selected by function resolution determines the
statement that is executed.  If none of the functions are chosen, the
the statement associated with the keyword \chpl{otherwise}, if it
exists, will be selected.

As with function resolution, this can result in an ambiguous
situation.  Unlike with function resolution, in the event of an
ambiguity, the first statement in the list of when statements is
chosen.

\subsection{The Empty Statement}
\label{The_Empty_Statement}

An empty statement has no effect.  The syntax of an empty statement is
given by
\begin{syntax}
empty-statement:
  ;
\end{syntax}

\cleardoublepage
\sekshun{Modules}
\label{Modules}
\index{modules}

Chapel supports modules to manage name spaces.  A program consists of
one or more modules.  Every symbol, including variables, functions,
and types, is associated with some module.

Module definitions are described in~\rsec{Module_Definitions}.  The
relation between files and modules is described
in~\rsec{Implicit_Modules}.  Nested modules are described
in~\rsec{Nested_Modules}.  The visibility of a module's symbols by
users of the module is described in~\rsec{Visibility_Of_Symbols}.  The execution
of a program and module initialization is described
in~\rsec{Program_Execution}.

\section{Module Definitions}
\label{Module_Definitions}
\index{module@\chpl{module}}
\index{modules!definitions}

A module is declared with the following syntax:
\begin{syntax}
module-declaration-statement:
  privacy-specifier[OPT] `module' module-identifier block-statement

privacy-specifier:
  `private'
  `public'

module-identifier:
  identifier
\end{syntax}

A module's name is specified after the \chpl{module} keyword.
The \sntx{block-statement} opens the module's scope.  Symbols defined
in this block statement are defined in the module's scope and are
called \emph{top-level module symbols}.  The visibility of a module is
defined by its \sntx{privacy-specifier}~(\rsec{Visibility_Of_A_Module}).

Module declaration statements must be top-level statements within a
module.  A module that is declared within another module is called a
nested module~(\rsec{Nested_Modules}).

\section{Files and Implicit Modules}
\label{Implicit_Modules}
\index{modules!and files}

Multiple modules can be defined in the same file and need not bear any
relation to the file in terms of their names.

\begin{chapelexample}{two-modules.chpl}
The following file contains two explicitly named modules
(\rsec{Explicit_Naming}), MX and MY.
\begin{chapel}
module MX {
  var x: string = "Module MX";
  proc printX() {
    writeln(x);
  }
}

module MY {
  var y: string = "Module MY";
  proc printY() {
    writeln(y);
  }
}
\end{chapel}
\begin{chapelpost}
MX.printX();
MY.printY();
\end{chapelpost}
\begin{chapeloutput}
Module MX
Module MY
\end{chapeloutput}
Module MX defines top-level module symbols x and printX, while MY
defines top-level module symbols y and printY.
\end{chapelexample}

For any file that contains top-level statements other than module
declarations, the file itself is treated as the module declaration.
In this case,
\index{implicit modules}
\index{modules!implicit}
the module is implicit and takes its name from the base filename.  In
particular, the module name is defined as the remaining string after
removing the \chpl{.chpl} suffix and any path specification from the
specified filename.  If the resulting name is not a legal Chapel
identifier, it cannot be referenced in a use statement.

\begin{chapelexample}{implicit.chpl}
The following file, named implicit.chpl, defines an implicitly named
module called implicit.
\begin{chapel}
var x: int = 0;
var y: int = 1;

proc printX() {
  writeln(x);
}
proc printY() {
  writeln(y);
}
\end{chapel}
\begin{chapelpost}
printX();
printY();
\end{chapelpost}
\begin{chapeloutput}
0
1
\end{chapeloutput}
Module implicit defines the top-level module symbols x, y, printX, and
printY.
\end{chapelexample}


\section{Nested Modules}
\label{Nested_Modules}
\index{modules!nested}

A nested module is a module that is defined within another module, the
outer module.  Nested modules automatically have access to all of the
symbols in the outer module.  However, the outer module needs to
explicitly use a nested module to have access to its symbols.

A nested module can be used without using the outer module by
explicitly naming the outer module in the use statement.
\begin{chapelexample}{nested-use.chpl}
The code
\begin{chapelpre}
module libsci {
  writeln("Initializing libsci");
  module blas {
    writeln("\\tInitializing blas");
  }
}
module testmain { // used to avoid warnings
}
\end{chapelpre}
\begin{chapel}
use libsci.blas;
\end{chapel}
\begin{chapeloutput}
Initializing libsci
	Initializing blas
\end{chapeloutput}
uses a module named \chpl{blas} that is nested inside a module
named \chpl{libsci}.
\end{chapelexample}

Files with both module declarations and top-level statements result in
nested modules.

\begin{chapelexample}{nested.chpl}
The following file, named nested.chpl, defines an
implicitly named module called nested, with nested modules
MX and MY.
\begin{chapel}
module MX {
  var x: int = 0;
}

module MY {
  var y: int = 0;
}

use MX, MY;

proc printX() {
  writeln(x);
}

proc printY() {
  writeln(y);
}
\end{chapel}
\begin{chapelpost}
printX();
printY();
\end{chapelpost}
\begin{chapeloutput}
0
0
\end{chapeloutput}
\end{chapelexample}


\section{Access of Module Contents}
\label{Access_Of_Module_Contents}
\index{modules!access}

A module's contents can be accessed by code outside of that module
depending on the visibility of the module
itself~(\rsec{Visibility_Of_A_Module}) and the visibility of each
individual symbol~(\rsec{Visibility_Of_Symbols}).  This can be done
via explicit naming~(\rsec{Explicit_Naming}) or the use
statement~(\rsec{Using_Modules}).

\subsection{Visibility Of A Module}
\label{Visibility_Of_A_Module}
\index{modules!access}

A module defined at file scope is visible anywhere. The visibility of a nested
module is subject to the rules of~\rsec{Visibility_Of_Symbols}. There,
the nested module is considered a "symbol defined at the top level
scope" of its outer module.

\subsection{Visibility Of A Module's Symbols}
\label{Visibility_Of_Symbols}
\index{modules!access}

A symbol defined at the top level scope of a module is \emph{visible}
from outside the module when the \sntx{privacy-specifier} of its
definition is \chpl{public} or is omitted (i.e. by default). When a
symbol defined at the top level scope of a module is declared
\chpl{private}, it is not visible outside of that module. A
symbol's visibility inside its module is controlled by normal lexical
scoping and is not affected by its \sntx{privacy-specifier}.  A
module's visible symbols are accessible via explicit
naming~(\rsec{Explicit_Naming}) or the use
statement~(\rsec{Using_Modules}) only where the module's symbol is
visible~(\rsec{Visibility_Of_A_Module}).

\subsection{Explicit Naming}
\label{Explicit_Naming}
\index{modules!explicitly named}

All publicly visible top-level module symbols can be named explicitly
with the following syntax:
\begin{syntax}
module-access-expression:
  module-identifier-list . identifier

module-identifier-list:
  module-identifier
  module-identifier . module-identifier-list

\end{syntax}
This allows two variables that have the same name to be distinguished
based on the name of their module.  Using explicit module naming in a
function call restricts the set of candidate functions to those in the
specified module.

If code refers to symbols that are defined by multiple modules, the
compiler will issue an error.  Explicit naming can be used to
disambiguate the symbols in this case.

\begin{openissue}
It is currently unspecified whether the
first-named module is always at the outermost module level scope, or whether a
scope-search mechanism is used starting at the scope containing the
usage.
\end{openissue}

\begin{chapelexample}{ambiguity.chpl}
In the following example,
\begin{chapel}
module M1 {
  var x: int = 1;
  var y: int = -1;
  proc printX() {
    writeln("M1's x is: ", x);
  }
  proc printY() {
    writeln("M1's y is: ", y);
  }
}
 
module M2 {
  use M3;
  use M1;

  var x: int = 2;

  proc printX() {
    writeln("M2's x is: ", x);
  }

  proc main() {
    M1.x = 4;
    M1.printX();
    writeln(x);
    printX(); // This is not ambiguous
    printY(); // ERROR: This is ambiguous
  }
}

module M3 {
  var x: int = 3;
  var y: int = -3;
  proc printY() {
    writeln("M3's y is: ", y);
  }
}
\end{chapel}
\begin{chapeloutput}
ambiguity.chpl:22: In function 'main':
ambiguity.chpl:27: error: ambiguous call 'printY()'
ambiguity.chpl:34: note: candidates are: printY()
ambiguity.chpl:7: note:                 printY()
\end{chapeloutput}
The call to printX() is not ambiguous because M2's definition shadows
that of M1.  On the other hand, the call to printY() is ambiguous
because it is defined in both M1 and M3.  This will result in a
compiler error.
\end{chapelexample}

\subsection{Using Modules}
\label{Using_Modules}
\index{modules!using}

If a module is visible to the scope in which accessing its symbols is desirable,
then a use statement on that module may be employed.  Use statements
make a module's visible symbols available without requiring them to be
prefixed by the module's name.  For information about use statements in general,
see~\rsec{The_Use_Statement}.

If a type is specified in the \sntx{limitation-clause}, then the type's fields
and methods are treated similarly to the type name.  These fields and methods
cannot be specified in a \sntx{limitation-clause} on their own.

% We need to figure out what to do about functions that return types which due
% to the limitation-clause are not visible without prefix.


\subsection{Module Initialization}
\label{Module_Initialization}
\index{modules!initialization}

Module initialization occurs at program start-up.  All top-level
statements in a module other than function and type declarations are
executed during module initialization.

\begin{chapelexample}{init.chpl}
In the code,
\begin{chapelpre}
proc foo() {
    return 1;
}
\end{chapelpre}
\begin{chapel}
var x = foo();       // executed at module initialization
writeln("Hi!");      // executed at module initialization
proc sayGoodbye {
  writeln("Bye!");   // not executed at module initialization
}
\end{chapel}
\begin{chapeloutput}
Hi!
\end{chapeloutput}
The function foo() will be invoked and its result assigned to x.  Then
``Hi!'' will be printed.
\end{chapelexample}

Module initialization order is discussed
in~\rsec{Module_Initialization_Order}.


\section{Program Execution}
\label{Program_Execution}
\index{program execution}
\index{program initialization}

Chapel programs start by initializing all modules and then executing
the main function~(\rsec{The_main_Function}).

\subsection{The {\em main} Function}
\label{The_main_Function}

\index{main@\chpl{main}}
\index{functions!main@\chpl{main}}
The main function must be called \chpl{main} and must have zero
arguments.  It can be specified with or without parentheses.  In any
Chapel program, there is a single main function that defines the
program's entry point.  If a program defines multiple potential entry
points, the implementation may provide a compiler flag that
disambiguates between main functions in multiple modules.

\begin{craychapel}
In the Cray Chapel compiler implementation, the \emph{--
--main-module} flag can be used to specify the module from which the
main function definition will be used.
\end{craychapel}

\begin{chapelexample}{main-module.chpl}
Because it defines two \chpl{main} functions, the following code will yield an
error unless a main module is specified on the command line.
\begin{chapel}
module M1 {
  const x = 1;
  proc main() {
    writeln("M", x, "'s main");
  }
}
 
module M2 {
  use M1;

  const x = 2;
  proc main() {
    M1.main();
    writeln("M", x, "'s main");
  }
}
\end{chapel}
\begin{chapelcompopts}
--main-module M1 \# main\_module.M1.good
--main-module M2 \# main\_module.M2.good
\end{chapelcompopts}
If M1 is specified as the main module, the program will output:
\begin{chapelprintoutput}{main_module.M1.good}
M1's main
\end{chapelprintoutput}
If M2 is specified as the main module the program will output:
\begin{chapelprintoutput}{main_module.M2.good}
M1's main
M2's main
\end{chapelprintoutput}
Notice that main is treated like just another function if it is not in
the main module and can be called as such.
\end{chapelexample}

\index{exploratory programming}

%subsubsection{Programs with a Single Module}
%% \label{Programs_with_a_Single_Module}

To aid in exploratory programming, a default main function is
created if the program does not contain a user-defined main function.  The
default main function is equivalent to
\begin{chapel}
proc main() {}
\end{chapel}

\begin{chapelexample}{no-main.chpl}
The code
\begin{chapel}
writeln("hello, world");
\end{chapel}
\begin{chapeloutput}
hello, world
\end{chapeloutput}
is a legal and complete Chapel program.  The startup code for a Chapel program
first calls the module initialization code for the main module and then
calls \chpl{main()}.  This program's initialization function is the top-level
writeln() statement.  The module declaration is taken to be the entire file,
as described in~\rsec{Implicit_Modules}.
\end{chapelexample}


\subsection{Module Initialization Order}
\label{Module_Initialization_Order}
\index{modules!initialization order}

Module initialization is performed using the following algorithm.

Starting from the module that defines the main function, the modules named in
its use statements are visited depth-first and initialized in post-order.  If a
use statement names a module that has already been visited, it is not visited a
second time.  Thus, infinite recursion is avoided.

Modules used by a given module are visited in the order in which
they appear in the program text.  For nested modules, the
parent module and its uses are initialized before the nested module and its uses.

\begin{chapelexample}{init-order.chpl}
The code
\begin{chapel}
module M1 {
  use M2.M3;
  use M2;
  writeln("In M1's initializer");
  proc main() {
    writeln("In main");
  }
}

module M2 {
  use M4;
  writeln("In M2's initializer");
  module M3 {
    writeln("In M3's initializer");
  }
}

module M4 {
  writeln("In M4's initializer");
}
\end{chapel}
prints the following
\begin{chapelprintoutput}{}
In M4's initializer
In M2's initializer
In M3's initializer
In M1's initializer
In main
\end{chapelprintoutput}
M1, the main module, uses M2.M3 and then M2, thus M2.M3 must be
initialized.  Because M2.M3 is a nested module, M4 (which is used by
M2) must be initialized first.  M2 itself is initialized, followed by
M2.M3.  Finally M1 is initialized, and the main function is run.
\end{chapelexample}

\cleardoublepage
\sekshun{Functions}
\label{Functions}
\index{functions}

This section defines functions.  Methods and iterators are functions
and most of this section applies to them as well.  They are defined
separately in~\rsec{Iterators} and~\rsec{Class_Methods}.

\subsection{Function Definitions}
\label{Function_Definitions}
\index{functions!syntax}

\index{def@\chpl{def}}
Functions are declared with the following syntax:
\begin{syntax}
function-declaration-statement:
  `def' function-name argument-list[OPT] var-param-clause[OPT]
    return-type[OPT] where-clause[OPT] block-level-statement

function-name:
  identifier
  operator-name

operator-name: one of
  + - * / % ** ! == <= >= < > << >> & | ^ ~

argument-list:
  ( formals[OPT] )

formals:
  formal
  formal , formals

formal:
  formal-tag identifier formal-type[OPT] default-expression[OPT]
  formal-tag identifier formal-type[OPT] variable-argument-expression

formal-type:
  : type
  : TQUESTION identifier

default-expression:
  = expression

variable-argument-expression:
  ... expression
  ... TQUESTION identifier

formal-tag: one of
  in out inout param type

var-param-clause:
  `var'
  `const'
  `param'

where-clause:
  `where' expression
\end{syntax}

Operator overloading is supported in Chapel on the operators listed
above under operator name.  Operator and function overloading is
discussed in~\rsec{Function_Overloading}.

The intents \chpl{in}, \chpl{out}, and \chpl{inout} are discussed
in~\rsec{Intents}.  The formal tags \chpl{param} and \chpl{type} make
a function generic and are discussed in~\rsec{Generics}.  If the
formal argument's type is elided, generic, or prefixed with a question
mark, the function is also generic and is discussed
in~\rsec{Generics}.

Default expressions allow for the omission of actual arguments at the
call site, resulting in the implicit passing of a default value.
Default values are discussed in~\rsec{Default_Values}.

Functions do not require parentheses if they have no arguments.  Such
functions are described in~\rsec{Functions_without_Parentheses}.

Return types are optional and are discussed in~\rsec{Return_Types}.

Functions can take a variable number of arguments.  Such functions are
discussed in~\rsec{Variable_Length_Argument_Lists}.

The optional \sntx{var-param-clause} defines a variable function,
discussed in~\rsec{Variable_Functions}, or a parameter function,
discussed in~\rsec{Parameter_Functions}.  By default, a function call
cannot be treated as an lvalue and is constant.  This may be
explicitly specified via the keyword~\chpl{const}.

The optional where clause is only applicable if the function is
generic.  It is discussed in~\rsec{Where_Expressions}.

\subsection{The Return Statement}
\label{The_Return_Statement}
\index{return@\chpl{return}}

The return statement can only appear in a function.  It exits that
function, returning control to the point at which that function was
called.  It can optionally return a value.  The syntax of the return
statement is given by
\begin{syntax}
return-statement:
  `return' expression[OPT] ;
\end{syntax}

\begin{example}
The following code defines a function that returns the sum of three
integers:
\begin{chapel}
def sum(i1: int, i2: int, i3: int)
  return i1 + i2 + i3;
\end{chapel}
\end{example}

\subsection{Function Calls}
\label{Function_Calls}
\index{function calls}

Functions are called in call expressions described
in~\rsec{Call_Expressions}.  The function that is called is resolved
according to the algorithm described in~\rsec{Function_Resolution}.

\subsection{Formal Arguments}
\label{Formal_Arguments}
\index{formal arguments}

Chapel supports an intuitive formal argument passing mechanism.  An
argument is passed by value unless it is a class, array, or domain in
which case it is passed by reference.

Intents~(\rsec{Intents}) result in potential assignments to temporary
variables during a function call.  For example, passing an array by
intent \chpl{in}, a temporary array will be created.

\subsubsection{Named Arguments}
\label{Named_Arguments}
\index{named arguments}
\index{formal arguments!naming}

A formal argument can be named at the call site to explicitly map an
actual argument to a formal argument.

\begin{example}
In the code
\begin{chapel}
def foo(x: int, y: int) { ... }

foo(x=2, y=3);
foo(y=3, x=2);
\end{chapel}
named argument passing is used to map the actual arguments to the
formal arguments.  The two function calls are equivalent.
\end{example}

Named arguments are sometimes necessary to disambiguate calls or
ignore arguments with default values.  For a function that has many
arguments, it is sometimes good practice to name the arguments at the
call-site for compiler-checked documentation.

\subsubsection{Default Values}
\label{Default_Values}
\index{default values}
\index{formal arguments!defaults}

Default values can be specified for a formal argument by appending the
assignment operator and a default expression the declaration of the
formal argument.  If the actual argument is omitted from the function
call, the default expression is evaluated when the function call is
made and the evaluated result is passed to the formal argument as if
it were passed from the call site.

\begin{example}
In the code
\begin{chapel}
def foo(x: int = 5, y: int = 7) { ... }

foo();
foo(7);
foo(y=5);
\end{chapel}
default values are specified for the formal arguments \chpl{x}
and \chpl{y}.  The three calls to \chpl{foo} are equivalent to the
following three calls where the actual arguments are
explicit: \chpl{foo(5, 7)}, \chpl{foo(7, 7)}, and \chpl{foo(5, 5)}.
Note that named arguments are necessary to pass actual arguments to
formal arguments but use default values for arguments that are
specified earlier in the formal argument list.
\end{example}

\subsection{Intents}
\label{Intents}
\index{intents}

Intents allow the actual arguments to be copied to a formal argument
and also to be copied back.

\subsubsection{The Blank Intent}
\label{The_Blank_Intent}

If the intent is omitted, it is called a blank intent.  In such a
case, the value is copied in using the assignment operator.  Thus
classes are passed by reference and records are passed by value.
Arrays and domains are an exception because assignment does not apply
from the actual to the formal.  Instead, arrays and domains are passed
by reference.

With the exception of arrays, any argument that has blank intent
cannot be assigned within the function.

\subsubsection{The In Intent}
\label{The_In_Intent}
\index{in@\chpl{in}}
\index{intents!in@\chpl{in}}

If \chpl{in} is specified as the intent, the actual argument is copied
to the formal argument as usual, but it may also be assigned to within
the function.  This assignment is not reflected back at the call site.

If an array is passed to a formal argument that has \chpl{in} intent,
a copy of the array is made via assignment.  Changes to the elements
within the array are thus not reflected back at the call site.
Domains cannot be passed to a function via the \chpl{in} intent.

\subsubsection{The Out Intent}
\label{The_Out_Intent}
\index{out@\chpl{out}}
\index{intents!out@\chpl{out}}

If \chpl{out} is specified as the intent, the actual argument is
ignored when the call is made, but after the call, the formal argument
is assigned to the actual argument at the call site.  The actual
argument must be a valid lvalue.  The formal argument can be assigned
to and read from within the function.

The formal argument cannot not be generic and is treated as a variable
declaration.  Domains cannot be passed to a function via
the \chpl{out} intent.

\subsubsection{The Inout Intent}
\label{The_Inout_Intent}
\index{inout@\chpl{inout}}
\index{intents!inout@\chpl{inout}}

If \chpl{inout} is specified as the intent, the actual argument is
both passed to the formal argument as if the \chpl{in} intent applied
and then copied back as if the \chpl{out} intent applied.  The formal
argument can be generic and takes its type from the actual argument.
Domains cannot be passed to a function via the \chpl{inout} intent.
The formal argument can be assigned to and read from within the
function.

\subsection{Return Types}
\label{Return_Types}
\index{return@\chpl{return}!types}

A function can optionally return a value.  If the function does not
return a value, then no return type can be specified.  If the function
does return a value, the return type is optional.

\subsubsection{Explicit Return Types}
\label{Explicit_Return_Types}

If a return type is specified, the values that the function returns
via return statements must be assignable to a value of the return
type.  For variable functions~(\rsec{Variable_Functions}), the return
type must match the type returned in all of the return statements
exactly.

\subsubsection{Implicit Return Types}
\label{Implicit_Return_Types}
\index{type inference!of return types}

If a return type is not specified, it will be inferred from the return
statements.  Given the types that are returned by the different
statements, if exactly one of those types can be a target, via
implicit conversions, of every other type, then that is the inferred
return type.  Otherwise, it is an error.  For variable
functions~(\rsec{Variable_Functions}), every return statement must
return the same exact type and it becomes the inferred type.

\subsection{Variable Functions}
\label{Variable_Functions}
\index{functions!as lvalues}

A variable function is a function that can be assigned a value.  Note
that a variable function does not return a reference.  That is, the
reference cannot be captured.

A variable function is specified by following the argument list with
the \chpl{var} keyword.  A variable function must return an lvalue.

When a variable function is called on the left-hand side of an
assignment statement or in the context of a call to a formal argument
by out or inout intent, the lvalue that is returned by the function is
assigned a value.

Variable functions support an implicit argument \chpl{setter} of type
bool.  If the variable function is called in a context such that the
returned lvalue is assigned a value, the argument \chpl{setter}
is \chpl{true}; otherwise it is \chpl{false}.  This argument is useful
for controlling different behavior depending on the call site.

\begin{example}
The following code creates a function that can be interpreted as a
simple two-element array where the elements are actually global
variables:
\begin{chapel}
var x, y = 0;

def A(i: int) var {
  if i < 0 || i > 1 then
    halt("array access out of bounds");
  if i == 0 then
    return x;
  else
    return y;
}
\end{chapel}
This function can be assigned to in order to write to the ``elements''
of the array as in
\begin{chapel}
A(0) = 1;
A(1) = 2;
\end{chapel}
It can be called as an expression to access the ``elements'' as in
\begin{chapel}
writeln(A(0) + A(1));
\end{chapel}
This code outputs the number \chpl{3}.

The implicit \chpl{setter} argument can be used to ensure, for
example, that the second element in the pseudo-array is only assigned
a value if the first argument is positive.  To do this, the line
\begin{chapel}
if setter && i == 1 && x <= 0 then
  halt("cannot assign value to A(1) if A(0) <= 0");
\end{chapel}
\end{example}

\subsection{Parameter Functions}
\label{Parameter_Functions}
\index{functions!as parameters}

A parameter function is a function that returns a parameter
expression.  It is specified by following the function's argument list
by the keyword \chpl{param}.  It is often, but not necessarily,
generic.

It is a compile-time error if a parameter function does not return a
parameter expression.  The result of a parameter function is computed
during compilation and the result is inlined at the call site.

\begin{example}
In the code
\begin{chapel}
def sumOfSquares(param a: int, param b: int) param
  return a**2 + b**2;

var x: sumOfSquares(2, 3)*int;
\end{chapel}
the function \chpl{sumOfSquares} is a parameter function that takes
two parameters as arguments.  Calls to this function can be used in
places where a parameter expression is required.  In this example, the
call is used in the declaration of a homogeneous and so is required to
be a parameter.
\end{example}.

\subsection{Function Overloading}
\label{Function_Overloading}
\index{functions!overloading}
\index{operators!overloading}

Functions that have the same name but different argument lists are
called overloaded functions.  Function calls to overloaded functions
are resolved according to the algorithm in~\rsec{Function_Resolution}.

Operator overloading is achieved by defining a function with a name
specified by that operator.  The operators that may be overloaded are
listed in the following table:

\begin{center}
\begin{tabular}{|l|l|}
\hline
{\bf arity} & {\bf operators} \\
\hline
unary & \verb@+ - ! ~@ \\
binary & \verb@+ - * / % ** ! == <= >= < > << >> & | ^ @ \\
\hline
\end{tabular}
\end{center}

The arity and precedence of the operator must be maintained when it is
overloaded.  Operator resolution follows the same algorithm as
function resolution.

\subsection{Function Resolution}
\label{Function_Resolution}

Given a function call, the function that the call resolves to is
determined according to the following algorithm:
\begin{itemize}
\item
Identify the set of visible functions.  A visible function is any
function with the same name that satisfies the criteria
in~\rsec{Identifying_Visible_Functions}.
\item
From the set of visible functions, determine the set of candidate
functions.  A function is a candidate if the function is a valid
application of the function call's actual arguments as determined
in~\rsec{Determining_Candidate_Functions}.  A compiler error occurs if
there are no candidate functions.
\item
From the set of candidate functions, the most specific function is
determined.  The most specific function is a candidate function that
is more specific than every other candidate function.  If there is no
function that is more specific than every other candidate function,
the function call is ambiguous and a compiler error occurs.  The term
{\em more specific function} is defined
in~\rsec{Determining_More_Specific_Functions}.
\end{itemize}.

\subsubsection{Identifying Visible Functions}
\label{Identifying_Visible_Functions}
\index{functions!visible}

A function is a visible function to a function call if the name of the
function is the same as the name of the function call and the function
is defined or used in a lexical outer scope.

\index{functions!with class arguments}
Additionally, functions that have arguments of class type are
considered globally visible and so are always visible regardless of
the location of their definition.

\subsubsection{Determining Candidate Functions}
\label{Determining_Candidate_Functions}
\index{functions!candidates}

A function is a candidate function if there is a {\em valid mapping}
from the function call to the function and each actual argument is
mapped to a formal argument that is a {\em legal argument mapping}.

\paragraph{Valid Mapping}

A function call is mapped to a function according to the following
steps:
\begin{itemize}
\item
Each actual argument that is passed by name is matched to the formal
argument with that name.  If there is no formal argument with that
name, there is no valid mapping.
\item
The remaining actual arguments are mapped in order to the remaining
formal arguments in order.  If there are more actual arguments then
formal arguments, there is no valid mapping.  If any formal argument
that is not mapped to by an actual argument does not have a default
value, there is no valid mapping.
\item
The valid mapping is the mapping of actual arguments to formal
arguments plus default values to formal arguments that are not mapped
to by actual arguments.
\end{itemize}

\paragraph{Legal Argument Mapping}

An actual argument of type $T_A$ can be mapped to a formal argument of
type $T_F$ if any of the following conditions hold:
\begin{itemize}
\item $T_A$ and $T_F$ are the same type.
\item There is an implicit coercion from $T_A$ to $T_F$.
\item $T_A$ is derived from $T_F$.
\item $T_A$ is scalar promotable to $T_F$.
\end{itemize}

\subsubsection{Determining More Specific Functions}
\label{Determining_More_Specific_Functions}
\index{functions!most specific}

Given two functions $F_1$ and $F_2$, $F_1$ is determined to be more
specific than $F_2$ by the following steps:
\begin{itemize}
\item
If at least one of the legal argument mappings to $F_1$ is a {\em more
specific argument mapping} than the corresponding legal argument
mapping to $F_2$ and none of the legal argument mappings to $F_2$ is a
more specific argument mapping than the corresponding legal argument
mapping to $F_1$, then $F_1$ is more specific.
\item If $F_1$ does not require promotion and $F_2$ does require promotion, then $F_1$ is more specific.
\item If $F_1$ shadows $F_2$, then $F_1$ is more specific.
\item Otherwise, $F_1$ is not more specific than $F_2$.
\end{itemize}

Given an argument mapping, $M_1$, from an actual argument, $A$, of
type $T_A$ to a formal argument, $F1$, of type $T_{F1}$ and an
argument mapping, $M_2$, from the same actual argument to a formal
argument, $F2$, of type $T_{F2}$, the more specific argument mapping
is determined by the following steps:
\begin{itemize}
\item
 If $T_{F1}$ and $T_{F2}$ are the same type and $F1$ is an
 instantiated parameter, $M_1$ is more specific.
\item
 If $T_{F1}$ and $T_{F2}$ are the same type and $F2$ is an
 instantiated parameter, $M_2$ is more specific.
\item
 If $M_1$ requires scalar promotion and $M_2$ does not require scalar
 promotion, $M_2$ is more specific.
\item
 If $M_2$ requires scalar promotion and $M_1$ does not require scalar
 promotion, $M_1$ is more specific.
\item
 If $F1$ is generic over all types and $F2$ is not generic over all
 types, $M_2$ is more specific.
\item
 If $F2$ is generic over all types and $F1$ is not generic over all
 types, $M_1$ is more specific.
\item
 If $T_{F1}$ and $T_{F2}$ are the same type, neither mapping is more
 specific.
\item
 If $T_A$ and $T_{F1}$ are the same type, $M_1$ is more specific.
\item
 If $T_A$ and $T_{F2}$ are the same type, $M_2$ is more specific.
\item
 If $T_{F1}$ is derived from $T_{F2}$, then $M_1$ is more specific.
\item
 If $T_{F2}$ is derived from $T_{F1}$, then $M_2$ is more specific.
\item
 If there is an implicit coercion from $T_{F1}$ to $T_{F2}$, then
 $M_1$ is more specific.
\item
 If there is an implicit coercion from $T_{F2}$ to $T_{F1}$, then
 $M_2$ is more specific.
\item
 If $T_{F1}$ is any \chpl{int} type and $T_{F2}$ is any \chpl{uint}
 type, $M_1$ is more specific.
\item
 If $T_{F2}$ is any \chpl{int} type and $T_{F1}$ is any \chpl{uint}
 type, $M_2$ is more specific.
\item
 Otherwise neither mapping is more specific.
\end{itemize}

\subsection{Functions without Parentheses}
\label{Functions_without_Parentheses}
\index{functions!functions without parentheses}

Functions do not require parentheses if they have empty argument
lists.  Functions declared without parentheses around empty argument
lists must be called without parentheses.

\begin{example}
Given the definitions
\begin{chapel}
def foo { }
def bar() { }
\end{chapel}
the function \chpl{foo} can be called by writing \chpl{foo} and the
function \chpl{bar} can be called by writing \chpl{bar()}.  It is an
error to apply parentheses to \chpl{foo} or omit them from \chpl{bar}.
\end{example}

\subsection{Nested Functions}
\label{Nested_Functions}
\index{functions!nested}

A function defined in another function is called a nested function.
Nesting of functions may be done to arbitrary degrees, i.e., a
function can be nested in a nested function.

Nested functions are only visible to function calls within the scope
in which they are defined.  An exception is to a function that has an
argument that is a class type.  Such functions are globally visible.

\subsubsection{Accessing Outer Variables}
\label{Accessing_Outer_Variables}

Nested functions may refer to variables defined in the function in
which they are nested.  If the function has class arguments, and is
thus globally visible, it is a compiler error to refer to a variable
in the outer function.

\begin{rationale}
It may be too strict to make this a compiler error.  Are there
advantages to making this a runtime error?
\end{rationale}

\subsection{Variable Length Argument Lists}
\label{Variable_Length_Argument_Lists}
\index{functions!variable number of arguments}

Functions can be defined to take a variable number of arguments.  This
allows the call site to pass a different number of actual arguments.

If the variable argument expression is an identifier prepended by a
question mark, the number of arguments is variable.  Alternatively,
the variable expression can evaluate to an integer parameter value
requiring the call site to pass that number of arguments to the
function.

In the function, the formal argument is a tuple of the actual
arguments.

\begin{example}
The code
\begin{chapel}
def mywriteln(x: int ...?k) {
  for param i in 1..k do
    writeln(x(i));
}
\end{chapel}
defines a function called \chpl{mywriteln} that takes a variable
number of arguments and then writes them out on separate lines.  The
parameter for-loop~(\rsec{Parameter_For_Loops}) is unrolled by the
compiler so that \chpl{i} is a parameter, rather than a variable.
This function can be made generic~(\rsec{Generics}) to take arguments
of different types by eliding the type.
\end{example}

A tuple of variables arguments can be passed to a function that takes
variable arguments by destructuring the tuple in a tuple destructuring
expression.  The syntax of this expression is given by
\begin{syntax}
tuple-destructuring-expression:
  ( ... expression )
\end{syntax}
In this expression, the tuple defined by \sntx{expression} is expanded
in place to represent its components.  This allows for the forwarding
of variable arguments as variable arguments.

\cleardoublepage
\sekshun{Classes}
\label{Classes}

Classes are an abstraction of a data structure where the storage
location is allocated independent of the scope of the variable of
class type.  Each call to the constructor creates a new data object
and returns a reference to the object.  Storage is reclaimed
automatically as described in~\rsec{Automatic_Memory_Management}.

\subsection{Class Declarations}
\label{Class_Declarations}

A class is defined with the following syntax:
\begin{syntax}
class-declaration-statement:
  `class' identifier class-inherit-type-list[OPT] {
    class-statement-list }

class-inherit-expression-list:
  class-type
  class-type , inherit-expression-list

class-statement-list:
  class-statement
  class-statement class-statement-list

class-statement:
  type-declaration-statement
  function-declaration-statement
  variable-declaration-statement
\end{syntax}
A \sntx{class-declaration-statement} defines a new type symbol
specified by the identifier.  Classes inherit data and functionality
from other classes if the \sntx{inherit-type-list} is specified.
Inheritance is described in~\rsec{Inheritance}.

The body of a class declaration consists of a sequence of statements
where each of the statements either defines a variable, called a
field, a function, called a method, or a type.

If a class contains a type alias or a parameter, the class is generic.
Generic classes are described in~\rsec{Generics}.

\subsection{Class Assignment}
\label{Class_Assignment}

Classes are assigned by reference.  After an assignment from one
variable of class type to another, the variables reference the same
storage location.

\subsection{Class Fields}
\label{Class_Fields}

Variables and constants declared within class declarations define
fields within that class.  (Parameters make a class generic.)  Fields
define the storage associated with a class.

\begin{example}
The code
\begin{chapel}
class Actor {
  var name: string;
  var age: uint;
}
\end{chapel}
defines a new class type called \chpl{Actor} that has two fields: the
string field \chpl{name} and the unsigned integer field \chpl{age}.
\end{example}

\subsubsection{Class Field Accesses}
\label{Class_Field_Accesses}

The field in a class is accessed via a member access expression as
described in~\rsec{Member_Access_Expressions}.  Fields in a class can
be modified via an assignment statement where the left-hand side of
the assignment is a member access.
\begin{example}
Given a variable \chpl{anActor} of type \chpl{Actor}, defined above,
the code
\begin{chapel}
var s: string = anActor.name;
anActor.age = 27;
\end{chapel}
reads the field \chpl{name} and assigns the value to the variable
\chpl{s}, and assigns the storage location in the object
\chpl{anActor} associated with the field \chpl{age} the value
\chpl{27}.
\end{example}

\subsection{Class Methods}
\label{Class_Methods}

A method is a function that is bound to a class.  A method is called
by passing an instance of the class to the method via a special
syntax that is similar to a field access.

\subsubsection{Class Method Declarations}
\label{Class_Method_Declarations}

Methods are declared with the following syntax:
\begin{syntax}
method-declaration-statement:
  `def' type-binding function-name argument-list[OPT] var-clause[OPT]
    return-type[OPT] where-clause[OPT] block-level-statement

type-binding:
  identifier .
\end{syntax}
If a method is declared within the lexical scope of a class, record,
or union, the type binding can be omitted and is taken to be the
innermost class, record, or union that the method is defined in.

\subsubsection{Class Method Calls}
\label{Class_Method_Calls}

A method is called by using the member access syntax as described
in~\rsec{Member_Access_Expressions} where the accessed expression is
the name of the method.

\begin{example}
A method to output information about an instance of the \chpl{Actor}
class can be defined as follows:
\begin{chapel}
def Actor.print() {
  writeln("Actor ", name, " is ", age, " years old");
}
\end{chapel}
This method can be called on an instance of the \chpl{Actor}
class, \chpl{anActor}, by writing \chpl{anActor.print()}.
\end{example}

\subsubsection{The {\em this} Reference}
\label{The_em_this_Reference}

The instance of a class is passed to a method using special syntax.
It does not appear in the argument list to the method.  The
reference \chpl{this} is an alias to the instance of the class on
which the method is called.

\begin{example}
Let class \chpl{C}, method \chpl{foo}, and function \chpl{bar} be
defined as
\begin{chapel}
class C {
  def foo() {
    bar(this);
  }
}
def bar(c: C) { }
\end{chapel}
Then given an instance of \chpl{C} called \chpl{c}, the method
call \chpl{c.foo()} results in a call to \chpl{bar} where the argument
is \chpl{c}.
\end{example}

\subsubsection{Class Methods without Parentheses}
\label{Class_Methods_without_Parentheses}

Methods do not require parentheses if they have empty argument lists.
Methods declared without parentheses around empty argument lists must
be called without parentheses.

\begin{example}
Given the definitions
\begin{chapel}
class C {
  def foo { }
  def bar() { }
}
\end{chapel}
and an instance of \chpl{C} called \chpl{c}, then the
method \chpl{foo} can be called by writing \chpl{c.foo} and the
method \chpl{bar} can be called by writing \chpl{c.bar()}.  It is an
error to apply parentheses to \chpl{foo} or omit them from \chpl{bar}.
\end{example}

\subsubsection{The {\em this} Method}
\label{The_em_this_Method}

A method declared with the name \chpl{this} allows a class to be
``indexed'' similarly to how a tuple, sequence, or array is indexed.
Indexing into a class has the semantics of calling a method on the
class named \chpl{this}.  There is no other way to call a method
called \chpl{this}.  The \chpl{this} method must be declared with
parentheses even if the argument list is empty.

\begin{example}
In the following code, the \chpl{this} method is used to create a
class that acts like a simple array that contains three integers
indexed by one, two, and three.
\begin{chapel}
class ThreeArray {
  var x1, x2, x3: int;
  def this(i: int) var {
    select i {
      when 1 do return x1;
      when 2 do return x2;
      when 3 do return x3;
    }
    halt("ThreeArray index out of bounds: ", i);
  }
}
\end{chapel}
\end{example}

\subsection{Class Constructors}
\label{Class_Constructors}

A class constructor is defined by declaring a method with the same
name as the class.  The constructor is used to create instances of the
class.  When the constructor is called, memory is allocated to store a
class instance.

\subsubsection{The Default Constructor}
\label{The_Default_Constructor}

A default constructor is automatically created for every class in the
Chapel program.  This constructor is defined such that it has one
argument for every field in the class.  Each of the arguments has a
default value.

The default constructor is very useful but its generality in terms of
having one argument for each field all of which have default values
makes it slightly difficult for the user to create their own
constructor.  It is expected that in many simple cases, the default
constructor will be all that is necessary.

\begin{example}
Given the class
\begin{chapel}
class C {
  def x: int;
  def y: real = 3.14;
  def z: string = "Hello, World!";
}
\end{chapel}
then instances of the class can be created by calling the default
constructor as follows:
\begin{itemize}
\item The call \chpl{C()} is equivalent to \chpl{C(0,3.14,"Hello, World!")}.
\item The call \chpl{C(2)} is equivalent to \chpl{C(2,3.14,"Hello, World!")}.
\item The call \chpl{C(z="")} is equivalent to \chpl{C(0,3.14,"")}.
\item The call \chpl{C(0,0.0,"")} is equivalent to \chpl{C(0,0.0,"")}.
\end{itemize}
\end{example}

\subsection{Getters and Setters}
\label{Getters_and_Setters}

All field accesses are resolved via getter and setter methods that are
defined in the class with the same name as the field.  A setter is
defined as an explicit setter
function~(\rsec{Explicit_Setter_Functions}).  Default getters and
setters are defined that simply access or set the field if the user
does not define their own.

\begin{example}
In the code
\begin{chapel}
class C {
  var x: int;
  def =x(value: int) {
    if value < 0 then
      halt("x assigned negative value");
    x = value;
  }
}
\end{chapel}
a setter is defined for field \chpl{x} that ensures that \chpl{x} is
never assigned a negative value.
\end{example}

\subsection{Inheritance}
\label{Inheritance}

A ``derived'' class can inherit from one or more other classes by
specifying those classes, the base classes, following the name of the
derived class in the declaration of the derived class.  When
inheriting from multiple base classes, only one of the base classes
may contain fields.  The other classes can only define methods.  Note
that a class can still be derived from a class that contains fields
which is itself derived from a class that contains fields.

\subsubsection{Accessing Base Class Fields}
\label{Accessing_Base_Class_Fields}

A derived class contains data associated with the fields in its base
classes.  The fields can be accessed in the same way that they are
accessed in their base class unless the getter or setter methods is
overridden in the derived class, as discussed
in~\rsec{Overriding_Base_Class_Methods}.

\subsubsection{Derived Class Constructors}
\label{Derived_Class_Constructors}

Derived class constructors automatically call the default constructor
of the base class.  There is an expectation that a more standard way
of chaining constructor calls will be supported.

\subsubsection{Shadowing Base Class Fields}
\label{Shadowing_Base_Class_Fields}

A field in the derived class can be declared with the same name as a
field in the base class.  Such a field shadows the field in the base
class in that it is always referenced when it is accessed in the
context of the derived class.  There is an expectation that there will
be a way to reference the field in the base class but this is not
defined at this time.

\subsubsection{Overriding Base Class Methods}
\label{Overriding_Base_Class_Methods}

If a method in a derived class is declared with the identical
signature as a method in a base class, then it is said to override the
base class's method.  Such a method is a candidate for dynamic
dispatch in the event that a variable that has the base class type
references a variable that has the derived class type.

The identical signature requires that the names, types, and order of
the formal arguments be identical.

\subsubsection{Inheriting from Multiple Classes}
\label{Inheriting_from_Multiple_Classes}

\begin{implementation}
Multiple inheritance is not yet supported.
\end{implementation}

A class can be derived from multiple base classes provided that only
one of the base classes contains fields either directly or from base
classes that it is derived from.  The methods defined by the other
base classes can be overridden.

\subsection{Class Promotion of Scalar Functions}
\label{Scalar Promotion}

A class can be defined to promote scalar functions by defining an
iterator in the class named \chpl{this} and specifying a return type.
The return type indicates the type that the class promotes.  The body
of the \chpl{this} iterator is ignored.  The class must also implement
the iterator interface as described in~\rsec{Iterator_Interface}.

There is an expectation that class promotion will be implemented in a
different way in the future.

\subsection{Nested Classes}
\label{Nested_Classes}

\begin{implementation}
Nested classes are not yet supported.
\end{implementation}

A class defined within another class is a nested class.

\subsection{Automatic Memory Management}
\label{Automatic_Memory_Management}

\begin{implementation}
Memory allocated to store class objects is not yet reclaimed.
\end{implementation}

Memory associated with class instances is reclaimed automatically when
there is no way for the current program to reference this memory.  The
programmer does not need to free the memory associated with class
instances.

\cleardoublepage
\sekshun{Records}
\label{Records}
\index{records}

A record is a data structure that is similar to a class except it has
value semantics, similar to primitive types.  Value semantics mean that
assignment, argument passing and function return values are by default
all done by copying.  Value semantics also imply that a variable of
record type is associated with only one piece of storage and has only one
type throughout its lifetime.  Storage is allocated for a variable of
record type when the variable declaration is executed, and the record
variable is also initialized at that time. When the record variable goes
out of scope, or at the end of the program if it is a global, it is
deinitialized and its storage is deallocated.

A record declaration statement creates a record
type~\rsec{Record_Declarations}.  A variable of record type contains all
and only the fields defined by that type (\rsec{Record_Types}).  Value
semantics imply that the type of a record variable is known at compile
time (i.e. it is statically typed).

A record can be created using the \chpl{new} operator, which allocates
storage, initializes it via a call to a record constructor, and returns
it.  A record is also created upon a variable declaration of a record
type.

A record type is generic if it contains generic fields.  Generic record types
are discussed in detail in~\rsec{Generic_Types}.

\section{Record Declarations}
\label{Record_Declarations}
\index{records!declarations}
\index{declarations!records}
\index{record@\chpl{record}}

A record type is defined with the following syntax:
\begin{syntax}
record-declaration-statement:
  simple-record-declaration-statement
  external-record-declaration-statement

simple-record-declaration-statement:
  `record' identifier { record-statement-list }

record-statement-list:
  record-statement
  record-statement record-statement-list

record-statement:
  variable-declaration-statement
  method-declaration-statement
  type-declaration-statement
  empty-statement
\end{syntax}

A \sntx{record-declaration-statement} defines a new type symbol specified
by the identifier. As in a class declaration, the body of a record declaration
can contain variable, method, and type declarations.

If a record declaration contains a type alias or parameter field, or it
contains a variable or constant field without a specified type and
without an initialization expression, then it declares a generic record
type.  Generic record types are described in~\rsec{Generic_Types}.

If the \chpl{extern} keyword appears before the \chpl{record} keyword, then an
external record type is declared. An external record is used within Chapel
for type and field resolution, but no corresponding backend definition is
generated.  It is presumed that the definition of an external record is supplied
by a library or the execution environment.  See the chapter on interoperability
(\rsec{Interoperability}) for more information on external records.

\begin{future}
Privacy controls for classes and records are currently not specified,
as discussion is needed regarding its impact on inheritance, for
instance.
\end{future}

\subsection{Record Types}
\label{Record_Types}
\index{records!record types}
\index{records!types}
\index{types!records}

A record type specifier simply names a record type, using
the following syntax:
\begin{syntax}
record-type:
  identifier
  identifier ( named-expression-list )
\end{syntax}
A record type specifier may appear anywhere a type specifier is permitted.

For non-generic records, the record name by itself is sufficient to specify the
type.  Generic records must be instantiated to serve as a fully-specified
type, for example to declare a variable.  This is done with
type constructors, which are defined in Section~\ref{Type_Constructors}.

\subsection{Record Fields}
\label{Record_Fields}
\index{records!fields}
\index{fields!records}

Variable declarations within a record type declaration define fields within that
record type.  The presence of at least one parameter field causes the record
type to become generic.  Variable fields define the storage associated with a
record.

\begin{chapelexample}{defineActorRecord.chpl}
The code
\begin{chapel}
record ActorRecord {
  var name: string;
  var age: uint;
}
\end{chapel}
\begin{chapeloutput}
\end{chapeloutput}
defines a new record type called \chpl{ActorRecord} that has two fields: the
string field \chpl{name} and the unsigned integer field \chpl{age}.  The data
contained by a record of this type is exactly the same as that contained by
an instance of the \chpl{Actor} class defined in the preceding
chapter~\rsec{Class_Fields}.
\end{chapelexample}

\subsection{Record Methods}
\label{Record_Methods}
\index{records!methods}
\index{methods!records}

A record method is a function or iterator that is bound to a record.
See the methods section~\rsec{Methods} for more information about
methods.

Note that the receiver of a record method is passed by \chpl{ref} or
\chpl{const ref} intent by default, depending on whether or not
\chpl{this} is modified in the body of the method.

\subsection{Nested Record Types}
\label{Nested_Record_Types}
\index{nested records}
\index{records!nested}

Record type declarations may be nested within other class, record and union
declarations.  Methods defined in a nested record type may access fields
declared in the containing aggregate type either implicitly, or explicitly by
means of an \chpl{outer} reference.

\section{Record Variable Declarations}
\label{Record_Variable_Declarations}
\index{records!variable declarations}
\index{variables!records}

A record variable declaration is a variable declaration using a record type.
When a variable of record type is declared, storage is allocated sufficient to
store all of the fields defined in that record type.

In the context of a class or record or union declaration, the fields are
allocated within the object as if they had been declared individually.  In this
sense, records provide a way to group related fields within a containing class
or record type.

In the context of a function body, a record variable declaration
causes storage to be allocated sufficient to store all of the fields in that
record type.  The record variable is initialized through a call to its
default initializer.  The default initializer for a record is defined in the
same way as the default initializer for a class (\rsec{Default_Initialization}).

\subsection{Storage Allocation}
\label{Record_Storage}
\index{records!allocation}

Storage for a record variable directly contains the data associated
with the fields in the record, in the same manner as variables
of primitive types directly contain the primitive values.
Record storage is reclaimed when the record variable goes out of scope.
No additional storage for a record is allocated or reclaimed.
Field data of one variable's record is not shared with data
of another variable's record.

\subsection{Record Initialization}
\label{Record_Initialization}
\index{records!initialization}
\index{initialization!record}

A variable of a record type declared without an initialization expression
is initialized through a call to the record's default initializer,
passing no arguments.  The default initializer for a record is defined in
the same way as the default initializer for a class
(\rsec{Default_Initialization}).

To construct a record as an expression,
i.e. without binding it to a variable, the \chpl{new} operator is
required.  In this case, storage is allocated and reclaimed as for a record
variable declaration (\rsec{Record_Storage}), except that the temporary record
goes out of scope at the end of the enclosing statement.
The constructors for a record are
defined in the same way as those for a class (\rsec{Class_Constructors}).

\begin{rationale}

The \chpl{new} keyword disambiguates types from values. This is needed
because of the syntactic similarity between constructors and type
specifiers for classes and records.

\end{rationale}

\begin{chapelexample}{recordCreation.chpl}
The program
\begin{chapel}
record TimeStamp {
  var time: string = "1/1/1011";
}

var timestampDefault: TimeStamp;                  // use the default for 'time'
var timestampCustom = new TimeStamp("2/2/2022");  // ... or a different one
writeln(timestampDefault);
writeln(timestampCustom);

var idCounter = 0;
record UniqueID {
  var id: int;
  proc UniqueID() { idCounter += 1; id = idCounter; }
}

writeln(new UniqueID());  // create and use a record value without a variable
writeln(new UniqueID());
\end{chapel}
\begin{chapelcompopts}
--no-warn-constructors
\end{chapelcompopts}
produces the output
\begin{chapelprintoutput}{}
(time = 1/1/1011)
(time = 2/2/2022)
(id = 1)
(id = 2)
\end{chapelprintoutput}
The variable \chpl{timestampDefault} is initialized with \chpl{TimeStamp}'s
default initializer. The expression \chpl{new TimeStamp} creates a record that
is assigned to \chpl{timestampCustom}.  It effectively
initializes \chpl{timestampCustom} via a call to the constructor with desired
arguments. The records created with \chpl{new UniqueID()} are discarded after
they are used.
\end{chapelexample}

As with classes, the user can provide his own constructors
(\rsec{User_Defined_Constructors}).  If any user-defined constructors are
supplied, the default initializer cannot be called directly.

\subsection{Record Deinitializer}
\label{Record_Deinitializer}
\index{records!deinitializer}
\index{deinitializer!records}

A record author may specify additional actions to be performed before record storage is
reclaimed by defining a record deinitializer.  A record deinitializer is a method named
\chpl{deinit()}.  A record deinitializer takes no arguments
(aside from the implicit \chpl{this} argument).  If defined, the deinitializer is called
on a record object after it goes out of scope and before its memory is reclaimed.

% TODO: The above ambiguous language is intended to allow optimizations involving extending
% the lifetime of an object.  However, we leave unspecified the means by which a user may
% demand avid running of the deinitializer and reclamation of memory (as in C\#).  We need to
% specify this so the above language can be tightened up for that case.

% For now, the actual lifetime of a record object is under the control of the compiler.  For
% example, as an optimization, ownership of an object may be transferred between variables with
% non-overlapping lifetimes.  When this happens, there will be no observable deinitialization of
% one of those variables.  The compiler may also choose to insert temporary copies e.g. of
% record formals or of a record return value.

% The compiler guarantees that every record constructor call will have exactly one
% corresponding record deinitializer call.  However, the exact number of constructor-deinitializer
% pairs is determined by the compiler, and may also be influenced by various compiler
% options.

\begin{chapelexample}{recordDeinitializer.chpl}
\begin{chapel}
class C { var x: int; } // A class with nonzero size.
// If the class were empty, whether or not its memory was reclaimed
// would not be observable.

// Defines a record implementing simple memory management.
record R {
  var c: unmanaged C;
  proc init() {
    c = new unmanaged C(0);
  }
  proc deinit() {
    delete c; c = nil;
  }
}

proc foo()
{
  var r: R; // Initialized using default constructor.
  writeln(r);
  // r will go out of scope here.
  // Its deinitializer will be called to free the C object it contains.
}

foo();
\end{chapel}
\begin{chapeloutput}
(c = {x = 0})

====================
Leaked Memory Report
==============================================================
Number of leaked allocations
           Total leaked memory (bytes)
                      Description of allocation
==============================================================
==============================================================
\end{chapeloutput}
\begin{chapelexecopts}
--memLeaksByType
\end{chapelexecopts}
\end{chapelexample}


\section{Record Arguments}
\label{Record_Arguments}
\index{records!arguments}
\index{arguments!records}

When records are copied into or out of a function's formal argument,
the copy is performed consistently with the semantics described for
record assignment (\rsec{Record_Assignment}).

\begin{chapelexample}{paramPassing.chpl}
The program
\begin{chapel}
record MyColor {
  var color: int;
}
proc printMyColor(in mc: MyColor) {
  writeln("my color is ", mc.color);
  mc.color = 6;   // does not affect the caller's record
}
var mc1: MyColor;        // 'color' defaults to 0
var mc2: MyColor = mc1;  // mc1's value is copied into mc2
mc1.color = 3;           // mc1's value is modified
printMyColor(mc2);       // mc2 is not affected by assignment to mc1
printMyColor(mc2);       // ... or by assignment in printMyColor()

proc modifyMyColor(inout mc: MyColor, newcolor: int) {
  mc.color = newcolor;
}
modifyMyColor(mc2, 7);   // mc2 is affected because of the 'inout' intent
printMyColor(mc2);
\end{chapel}
produces
\begin{chapelprintoutput}{}
my color is 0
my color is 0
my color is 7
\end{chapelprintoutput}
The assignment to \chpl{mc1.color} affects only the record stored
in \chpl{mc1}. The record in \chpl{mc2} is not affected by
the assignment to \chpl{mc1} or by the assignment in \chpl{printMyColor}.
\chpl{mc2} is affected by the assignment in \chpl{modifyMyColor}
because the intent \chpl{inout} is used.
\end{chapelexample}

\section{Record Field Access}
\label{Record_Field_Access}
\index{records!field access}
\index{field access}

A record field is accessed the same way as a class field
(\rsec{Class_Field_Accesses}).  When a field access is used as an
rvalue, the value of that field is returned.  When it is used as
an lvalue, the value of the record field is updated.

Accessing a parameter or type field returns a parameter or type,
respectively. Also, parameter and type fields can be accessed from
an instantiated record type in addition to from a record value.


\subsection{Field Getter Methods}
\label{Field_Getter_Methods}
\index{records!getters}

As in classes, field accesses are performed via getter methods
(\rsec{Getter_Methods}).  By default, these methods simply return a reference to
the specified field (so they can be written as well as read).  The user may
redefine these as needed.

\section{Record Method Calls}
\label{Record_Method_Access}
\index{records!method calls}
\index{method calls}

Record method calls are written the same way as other method calls
(\rsec{Method_Calls}). Unlike class methods, record methods are
always resolved at compile time.

\section{Common Operations}

\subsection{Record Assignment}
\label{Record_Assignment}
\index{records!assignment}

A variable of record type may be updated by assignment.  The compiler
provides a default assignment operator for each record type \chpl{R}
having the signature:

\begin{chapel}
proc =(ref lhs:R, rhs) : void ;
\end{chapel}

In it, the value of each field of the record on the right-hand side is assigned
to the corresponding field of the record on the left-hand side. It is
a type error if the left-hand side and the right-hand side do not have
the same set of field names. It is also a type error if two fields with
the same name do not have assignable types.

The compiler-provided assignment operator may be overridden as described
in \ref{Assignment_Statements}.

The following example demonstrates record assignment.
\begin{chapelexample}{assignment.chpl}
\begin{chapel}
record R {
  var i: int;
  var x: real;
  proc print() { writeln("i = ", this.i, ", x = ", this.x); }
}
var A: R;
A.i = 3;
A.print();	// "i = 3, x = 0.0"

var C: R;
A = C;
A.print();	// "i = 0, x = 0.0"

C.x = 3.14;
A.print();	// "i = 0, x = 0.0"
\end{chapel}
\begin{chapeloutput}
i = 3, x = 0.0
i = 0, x = 0.0
i = 0, x = 0.0
\end{chapeloutput}
Prior to the first call to \chpl{R.print}, the record \chpl{A} is created and
initialized to all zeroes.  Then, its \chpl{i} field is set to \chpl{3}.
For the second call to \chpl{R.print}, the record \chpl{C} is created assigned
to \chpl{A}.  Since \chpl{C} is default-initialized to all zeroes, those zero
values overwrite both values in \chpl{A}.

The next clause demonstrates that \chpl{A} and \chpl{C} are distinct entities,
rather than two references to the same object.  Assigning \chpl{3.14}
to \chpl{C.x} does not affect the \chpl{x} field in \chpl{A}.
\end{chapelexample}

\subsection{Default Comparison Operators}
\label{Record_Comparison_Operators}
\index{records!equality}
\index{records!inequality}
\index{records!==@\chpl{==}}
\index{records!"!=@\chpl{"\"!=}}
\index{== (record)@\chpl{==} (record)}
\index{"!= (record)@\chpl{"\"!=} (record)}
Default functions to overload \chpl{==} and \chpl{\!=} are defined for
records if none are explicitly defined.
The default implementation of \chpl{==} applies \chpl{==} to each
field of the two argument records and reduces the result with
the \chpl{&&} operator.  The default implementation of \chpl{\!=}
applies \chpl{\!=} to each field of the two argument records and
reduces the result with the \chpl{||} operator.

\section{Differences between Classes and Records}
\label{Class_and_Record_Differences}
\index{records!differences with classes}

The key differences between records and classes are listed below.

\subsection{Declarations}
\label{Declaration_Differences}
\index{records!declarations!differences with classes}

Syntactically, class and record type declarations are identical, except that
they begin with the \chpl{class} and \chpl{record} keywords, respectively.
In contrast to classes, records do not support inheritance.

\subsection{Storage Allocation}
\label{Storage_Allocation_Differences}
\index{classes!allocation}
\index{records!allocation}

For a variable of record type, storage necessary to contain the data fields
has a lifetime equivalent to the scope in which it is declared.  No two record
variables share the same data.  It is not necessary to call \chpl{new} to create
a record.

By contrast, a class variable contains only a reference to a
class instance.  A class instance is created through a call to its \chpl{new}
operator.  Storage for a class instance, including storage for
the data associated with the fields in the class, is allocated and reclaimed
separately from variables referencing that instance.  The same class instance
can be referenced by multiple class variables.

\subsection{Assignment}
\label{Assignment_Differences}
\index{classes!assignment}
\index{records!assignment}

Assignment to a class variable is performed by reference, whereas assignment to
a record is performed by value.  When a variable of class type is assigned to
another variable of class type, they both become names for the same object.  In
contrast, when a record variable is assigned to another record variable, then
contents of the source record are copied into the target record field-by-field.

When a variable of class type is assigned to a record, matching fields (matched
by name) are copied from the class instance into the corresponding record
fields.  Subsequent changes to the fields in the target record have no effect
upon the class instance.

Assignment of a record to a class variable is not permitted.

\subsection{Arguments}
\label{Argument_Differences}
\index{classes!arguments}
\index{records!arguments}

Record arguments use the \chpl{const ref} intent by default - in contrast
to class arguments which pass by \chpl{const in} intent by default.

Similarly, the \chpl{this} receiver argument is passed by \chpl{const in} by
default for class methods. In contrast, it is passed by \chpl{ref} or
\chpl{const ref} by default for record methods.

\subsection{No {\em nil} Value}
\index{nil@\chpl{nil}!not provided for records}

Records do not provide a counterpart of the \chpl{nil} value.  A variable of
record type is associated with storage throughout its lifetime, so \chpl{nil}
has no meaning with respect to records.

\subsection{The {\em delete} operator}
\label{Record_Delete_Illegal}
\index{records!delete illegal}
\index{delete!illegal for records}

Calling \chpl{delete} on a record is illegal.

%REVIEW: we could discuss this:
%An explicit call to \chpl{delete} with a record argument has no effect.  The
%compiler may treat this as a hint that the record should not be accessed later
%within its scope and diagnose that case.

\subsection{Default Comparison Operators}
\label{Comparison_Operator_Differences}
\index{classes!comparison}
\index{records!comparison}

For records, the compiler will supply default comparison operators if
they are not supplied by the user.  In contrast, the user cannot redefine
\chpl{==} and \chpl{!=} for classes.  The default comparison operators
for a record examine the arguments' fields, while the comparison
operators for classes check whether the l.h.s. and r.h.s. refer to the
same class instance or are both \chpl{nil}.

\cleardoublepage
This is a stub.  This portion of the document does not exist.

\cleardoublepage
\sekshun{Tuples}
\label{Tuples}
\index{tuples}

A tuple is an ordered set of components that allows for the
specification of a light-weight record with anonymous fields.

\subsection{Tuple Expressions}
\label{Tuple_Expressions}

A tuple expression is a comma-separated list of expressions that is
enclosed in parentheses.  The number of expressions is the size of the
tuple and the types of the expressions determine the component types
of the tuple.

The syntax of a tuple expression is given by:
\begin{syntax}
tuple-expression:
  ( expression-list )

expression-list:
  expression
  expression , expression-list
\end{syntax}

\begin{example}
The statement
\begin{chapel}
var x = (1, 2);
\end{chapel}
defines a variable \chpl{x} that is a 2-tuple containing the values
\chpl{1} and \chpl{2}.
\end{example}

\subsection{Tuple Type Definitions}
\label{Tuple_Type_Definitions}
\index{tuples!types}

A tuple type is a comma-separated list of types.  The number of types
in the list defines the size of the tuple, which is part of the
tuple's type.  The syntax of a tuple type is given by:
\begin{syntax}
tuple-type:
  ( type-list )

type-list:
  type
  type , type-list
\end{syntax}

\begin{example}
Given a tuple expression \chpl{(1, 2)}, the type of the tuple value is
\chpl{(int, int)}, referred to as a 2-tuple of integers.
\end{example}

\subsection{Tuple Assignment}
\label{Tuple_Assignment}
\index{assignment!tuples}
\index{tuples!assignment}

In tuple assignment, the components of the tuple on the left-hand
side of the assignment operator are each assigned the components of
the tuple on the right-hand side of the assignment.  The assignments
are simultaneous so that each component expression on the right-hand
side is fully evaluated before being assigned to the left-hand side.

\subsection{Tuple Operators}
\label{Tuple_Operators}
\index{tuples!operators}

The arithmetic ~(\rsec{Arithmetic_Operators}), bitwise
~(\rsec{Bitwise_Operators}), shift ~(\rsec{Shift_Operators}), and
relational ~(\rsec{Relational_Operators}) operators are also defined
over tuples.

With the exception of relational operators, operations applied to two
tuples result in element-by-element application of the operation.

Relational operators over tuples apply in an "alphabetical" manner.
Each component is compared to the corresponding component or to the
scalar value until the relation is found to be true or false.

\begin{example}
In the code:
\begin{chapel}
var x = ("c", "h", "p", "l") > ("c", "h", "a", "t"); 
\end{chapel}
The value of \chpl{x} is \chpl{true}. After comparing \chpl{"c"} to
\chpl{"c"}, and \chpl{"h"} to \chpl{"h"}, \chpl{"p"} is found to be
greater than \chpl{"a"}, so the expression is \chpl{true}. 
\end{example}

\subsection{Tuple Destructuring}
\label{Tuple_Destructuring}
\index{tuples!destructuring}

When a tuple expression appears on the left-hand side of an assignment
statement, the expression on the right-hand side is said to be {\em
destructured}.  The components of the tuple on the right-hand side are
assigned to each of the component expressions on the left-hand side.
This assignment is simultaneous in that the right-hand side is
evaluated before the assignments are made.
\begin{example}
Given two variables of the same type, x and y, they can be swapped by
the following single assignment statement:
\begin{chapel}
(x, y) = (y, x);
\end{chapel}
\end{example}

\subsubsection{Variable Declarations in a Tuple}
\label{Variable_Declarations_in_a_Tuple}
\index{tuples!variable declarations}

Variables can be defined in a tuple to facilitate capturing the values
from a function that returns a tuple.  The extension to the syntax of
variable declarations is as follows:
\begin{syntax}
tuple-variable-declaration-statement:
  `config'[OPT] variable-kind tuple-variable-declaration ;

tuple-variable-declaration:
  ( tuple-identifier-list ) type-part[OPT] initialization-part
  ( tuple-identifier-list ) type-part

tuple-identifier-list:
  tuple-identifier
  tuple-identifier , tuple-identifier-list

tuple-identifier:
  identifier
  ( tuple-identifier-list )
\end{syntax}
The identifiers defined within the \sntx{tuple-identifier-list} are declared
to be new variables in the scope of the statement.  The
\sntx{type-part} and/or \sntx{initialization-part} defines a tuple
that is destructured when assigned to the new variables. The shape of the
\sntx{tuple-identifier-list} must match the shape of any specified
\sntx{type-part} or \sntx{initialization-part}.

\subsubsection{Ignoring Values with Underscore}
\label{Ignoring_Values_with_Underscore}
\index{_@\chpl{_}}

If an underscore appears as a component in a tuple expression in a
destructuring context, the expression on the right-hand side is
ignored, though it is still evaluated.

\subsection{Homogeneous Tuples}
\label{Homogeneous_Tuples}
\index{tuples!homogeneous}

A homogeneous tuple is a special-case of a general tuple where the
types of the components are identical.  Homogeneous tuples have fewer
restrictions for how they can be indexed~(\rsec{Tuple_Indexing}).

\subsubsection{Declaring Homogeneous Tuples}
\label{Declaring_Homogeneous_Tuples}

\index{* (tuples)@\chpl{*} tuples}

A homogeneous tuple type may be specified with the following syntax if
it appears as a top-level type in a variable declaration, formal
argument declaration, return type specification, or type alias
declaration:
\begin{syntax}
homogeneous-tuple-type:
  integer-parameter-expression * type

integer-parameter-expression:
  expression
\end{syntax}
The homogeneous tuple type specification is syntactic sugar for the
type explicitly replicated a number of times equal to the
\sntx{integer-parameter-expression}.
\begin{example}
The following types are equivalent:
\begin{center}
\chpl{3*int} \hspace{2pc} \chpl{(int, int, int)}
\end{center}
\end{example}

\subsection{Tuple Indexing}
\label{Tuple_Indexing}
\index{tuples!indexing}

A tuple may be indexed into by an integer.  Indexing a tuple is given
by the following syntax:
\begin{syntax}
tuple-indexing-expression:
  expression ( integer-expression )
\end{syntax}

The result of indexing a tuple by integer $k$ is the value of the
$k$th component.  If the tuple is not homogeneous, the tuple can only
be indexed by an integer parameter.  This ensures that the type of the
indexing expression is known at compile-time.

\subsection{Formal Arguments of Tuple Type}
\label{Formal_Arguments_of_Tuple_Type}

\index{formal arguments!tuples}

\begin{status}
Formal arguments of tuple type are treated as if they were records.
Conversions are not applied to the components.
\end{status}

\subsubsection{Formal Argument Declarations in a Tuple}
\label{Formal_Argument_Declarations_in_a_Tuple}
\index{formal arguments!tuples}

Formal argument declarations can be grouped into a tuple similarly to
variable declarations to facilitate passing the result of a function that
returns a tuple directly to another function.

\begin{status}Formal arguments grouped in a tuple cannot be explicitly
typed. A function with formal arguments grouped in a tuple is
therefore generic.
\end{status}

\cleardoublepage
\sekshun{Ranges}
\label{Ranges}
\index{ranges}

Ranges represent a sequence of integral values.  Ranges are
either \emph{bounded} or \emph{unbounded}.

Bounded ranges are characterized by a low bound~$l$, a high bound~$h$,
and a stride~$s$.  If the stride is positive, the values described by
the range are $l, l+s, l+2s, l+3s, ...$ such that all of the values in
the sequence are less than or equal to $h$.  If the stride is negative,
the values described by the range are $h, h+s, h+2s, h+3s, ...$ such
that all of the values in the sequence are greater than or equal to
$l$.  If $l > h$, the range is considered degenerate and represents an
empty sequence. Ranges support iteration over the values they represent
as described in ~\rsec{The_For_Loop}.

Unbounded ranges are those in which the low and/or high bounds are
omitted.  Unbounded ranges conceptually represent a countably infinite
number of values.

\subsection{Range Types}
\label{Range_Types}
\index{ranges!types}

The type of a range is characterized by three things:
(1)~the type of the values being represented, (2)~the boundedness of
the range, and (3)~whether or not the range is \emph{stridable}.

The type of the range's values is represented using a type parameter
named \emph{idxType}.  This must be one of the \chpl{int} or
\chpl{uint} types.  The default type is \chpl{int}.

\begin{openissue}
It has been hypothesized that ranges of other types, such as floating
point values, might also be of interest to represent a range of legal
tolerances, for example.  If you believe such support would be of
interest to you, please let us know.
\end{openissue}

The boundedness of the range is represented using an enumerated
parameter named \emph{boundedType} of type \chpl{BoundedRangeType}.
Legal values are \chpl{bounded}, \chpl{boundedLow},
\chpl{boundedHigh}, and \chpl{boundedNone}.  The first value specifies
a bounded range while the other three values specify a range in which
the high bound is omitted, the low bound is omitted, or both bounds
are omitted, respectively.  The default value is \chpl{bounded}.

The stridability of a range is represented by a boolean parameter
named \emph{stridable}.  If this parameter is set to true, the range's
stride can take on any signed integer value other than 0 of the same
bit-width as \chpl{idxType}.  If set to false, the range's stride is
fixed to 1.  The default value is \chpl{false}.

\begin{rationale}
The \emph{boundedType} and \emph{stridable} values of a range are used
to optimize the generated code for common cases of ranges, as well as
to optimize the implementation of domains and arrays defined using
ranges.
\end{rationale}

The syntax of a range type is summarized as follows:
\begin{syntax}
range-type:
  `range' ( named-expression-list )
\end{syntax}

\begin{example}
The following declaration declares a variable \chpl{r}
of range type that can represent ranges of 64-bit integers, with both
high and low bounds specified, and the ability to have a stride other
than 1.
\begin{chapelpre}
% test_rangeVariable.chpl
\end{chapelpre}
\begin{chapel}
var r: range(int(64), BoundedRangeType.bounded, stridable=true);
\end{chapel}
\begin{chapelpost}
writeln(r);
var i64: int(64) = 3;
r = i64..13 by 3;
writeln(r);
\end{chapelpost}
\begin{chapeloutput}
1..0
3..12 by 3
\end{chapeloutput}
\end{example}

The default value for a range is \chpl{1..0}.

\subsection{Literal Range Values}
\label{Range_Literals}
\index{ranges!literals} 

Range literals are specified as follows:
\begin{syntax}
range-literal:
  bounded-range-literal
  unbounded-range-literal
\end{syntax}

\subsubsection{Bounded Range Literals}
\label{Bounded_Ranges}
\index{ranges!bounded}

A bounded range is specified by the syntax
\begin{syntax}
bounded-range-literal:
  expression .. expression
\end{syntax}
The first expression is taken to be the lower bound $l$ and the second
expression is taken to be the upper bound $h$.  The stride of the
range is 1 and can be modified with the \chpl{by} operator as described
in~\rsec{By_Operator_For_Ranges}.

\index{ranges!integral element type}
The element type of the range type is determined by the type of the
low and high bound.  It is either \chpl{int}, \chpl{uint},
\chpl{int(64)}, or \chpl{uint(64)}.  The type is determined by
conceptually adding the low and high bounds together.  The boundedness
of such a range is \chpl{BoundedRangeType.bounded}.  The stridability of
the range is \chpl{false}.

\subsubsection{Unbounded Range Literals}
\label{Unbounded_Ranges}
\index{ranges!unbounded}

An unbounded range is specified by the syntax
\begin{syntax}
unbounded-range-literal:
  expression ..
  .. expression
  ..
\end{syntax}

The first form results in a \chpl{BoundedRangeType.boundedLow} range, the
second in a \chpl{BoundedRangeType.boundedHigh} range, and the third in
a \chpl{BoundedRangeType.boundedNone} range.

Unbounded ranges can be iterated over with zipper iteration
(~\rsec{Zipper_Iteration}) and their shape conforms to the shape of the
other iterators they are being iterated over with.
\begin{example}
The code
\begin{chapelpre}
% test_zipWithUnbounded.chpl
\end{chapelpre}
\begin{chapel}
for i in (1..5, 3..) do
  write(i, "; ");
\end{chapel}
\begin{chapelpost}
writeln();
\end{chapelpost}
\begin{chapeloutput}
(1, 3); (2, 4); (3, 5); (4, 6); (5, 7); 
\end{chapeloutput}
produces the output ``(1, 3); (2, 4); (3, 5); (4, 6); (5, 7); ''.
\end{example}

It is an error to iterate over a \chpl{BoundedRangeType.boundedNone} range,
a \chpl{BoundedRangeType.boundedLow} range with negative stride or a
\chpl{BoundedRangeType.boundedHigh} range with positive stride.

Unbounded ranges can also be used to index into ranges, domains,
arrays, and strings.  In these cases, elided bounds are inherited
from the bounds of the expression being indexed.


\subsection{Range Assignment}
\label{Range_Assignment}
\index{ranges!assignment}

Assigning one range to another results in its low, high, and stride
values being copied from the source range to the destination range.

In order for range assignment to be legal, the element type of the
source range must be implicitly coercible to the element type of the
destination range.  The two range types must have the same boundedness
parameter.  It is legal to assign a non-stridable range to a stridable
range, but illegal to assign a stridable range to a non-stridable
range unless the stridable range has a stride value of 1.


\subsection{Range Operators}
\label{Range_Operators}
\index{ranges!operators}

\subsubsection{By Operator}
\label{By_Operator_For_Ranges}
\index{ranges!strided}
\index{ranges!by operator}
\index{by@\chpl{by}}

The \chpl{by} operator can be applied to any range to create a strided
range.

The \chpl{by} operator takes a range and an integer value to yield a
new range that is strided by the integer.  Striding a strided range
results in a stride whose value is the product of the two strides.
The stride argument can either be of type \chpl{idxType} or some other
integer value that can coerce to a signed integer value of the same
bit-width as \chpl{idxType}.

\begin{example}
In the following declarations, range \chpl{r1} represents the odd integers
between 1 and 20. Range \chpl{r2} strides \chpl{r1} by two and represents
every other odd integer between 1 and 20: 1, 5, 9, ...
\begin{chapelpre}
% test_rangeByOperator.chpl
\end{chapelpre}
\begin{chapel}
var r1 = 1..20 by 2;
var r2 = r1 by 2;
\end{chapel}
\begin{chapelpost}
writeln(r1);
writeln(r2);
\end{chapelpost}
\begin{chapeloutput}
1..19 by 2
1..17 by 4
\end{chapeloutput}
\end{example}

\begin{rationale}
{\it Why isn't the high bound specified first if the stride is
negative?}  The reason for this choice is that the \chpl{by} operator
is binary, not ternary.  Given a range \chpl{R} initialized
to \chpl{1..3}, we want \chpl{R by -1} to contain the ordered sequence
$3,2,1$.  But then \chpl{R by -1} would be different than \chpl{3..1
by -1} even though it should be identical by substituting the value in
R into the expression.
\end{rationale}

\subsubsection{Count Operator}
\label{Count_Operator}
\index{ranges!count operator}

The \chpl{#} operator can be applied to a range that has a high bound,
a low bound, or both.

The \chpl{#} operator takes a range and an integral count and creates
a new range with \emph{count} elements. The stride of the resulting range is
the same as that of the initial range. It is an error for the count to
be negative.  The \emph{idxType} of the resulting range is the same
type that would be obtained by adding the integral count value to a value
with the range's \emph{idxType}.

When applied to a \chpl{BoundedRangeType.bounded} range with a positive
stride, \emph{count} elements are taken starting from the low
bound. When the stride is negative, \emph{count} elements are taken
starting from the high bound. It is an error for \emph{count} to be larger
than the length of the range.

When applied to a \chpl{BoundedRangeType.boundedLow} range, the low bound
is fixed and and the high bound is set based on the count and the
absolute value of the stride.

When applied to a \chpl{BoundedRangeType.boundedHigh} range, the high
bound is fixed and the low bound is set based on the count and the
absolute value of the stride.

It is an error to apply the count operator to a
\chpl{BoundedRangeType.boundedNone} range.

\begin{example}
The following declarations result in equivalent ranges.
\begin{chapelpre}
% test_rangeCountOperator.chpl
\end{chapelpre}
\begin{chapel}
var r1 = 2.. by -2 # 3;
var r2 = ..6 by -2 # 3;
var r3 = 0..6 by -2 # 3;
var r4 = 1..#6 by -2;
\end{chapel}
\begin{chapelpost}
writeln(r1 == r2 \&\& r2 == r3 \&\& r3 == r4);
writeln((r1, r2, r3, r4));
\end{chapelpost}
\begin{chapeloutput}
true
(2..6 by -2, 2..6 by -2, 2..6 by -2, 2..6 by -2)
\end{chapeloutput}
Each of these ranges represents the ordered set of three values: 6, 4, 2.
\end{example}

\subsubsection{Arithmetic Operators}
\label{Range_Arithmetic}
\index{ranges!arithmetic operators}

The following arithmetic operators are defined on ranges and integral
types:

\begin{chapel}
def +(r: range, s: integral): range
def +(s: integral, r: range): range
def -(r: range, s: integral): range
\end{chapel}

The \chpl{+} and \chpl{-} operators apply the scalar via the operator
to the range's low and high bounds, producing a shifted version of the
range.  The element type of the resulting range is the type of the value
that would result from an addition between the scalar value and a value
with the range's element type.  The bounded and stridable parameters for
the result range are the same as for the input range.

\begin{example}
The following code creates a bounded, non-stridable range \chpl{r}
which has an element type of \chpl{int} representing the values ${0,
  1, 2, 3}$.  It then uses the \chpl{+} operator to
create a second range \chpl{r2} representing the values ${1, 2, 3,
  4}$.  The \chpl{r2} range is bounded, non-stridable, and represents
values of type \chpl{int}.
\begin{chapelpre}
% test_rangeAdd.chpl
\end{chapelpre}
\begin{chapel}
var r = 0..3;
var r2 = r + 1;
\end{chapel}
\begin{chapelpost}
writeln((r, r2));
\end{chapelpost}
\begin{chapeloutput}
(0..3, 1..4)
\end{chapeloutput}
\end{example}


\subsubsection{Range Slicing}
\label{Range_Slicing}
\index{ranges!slicing}

Ranges can be \emph{sliced} using other ranges to create new
sub-ranges.  The resulting range represents the intersection between
the two ranges.  Range slicing is defined by using the range as a
function in a call expression where the argument is another range.
If the slicing range is unbounded in one or both directions, it
inherits its missing bounds from the range being sliced.

\begin{example}
In the following example, \chpl{r} represents the integers from 1 to
20 inclusive.  Ranges \chpl{r2} and \chpl{r3} are defined using range
slices and represent the indices from 3 to 20 and the odd integers
between 1 and 20 respectively. Range \chpl{r4} represents the odd
integers between 1 and 20 that are also divisible by 3.
\begin{chapelpre}
% test_rangeSlicing.chpl
\end{chapelpre}
\begin{chapel}
var r = 1..20;
var r2 = r[3..];
var r3 = r[1.. by 2];
var r4 = r3[0.. by 3];
\end{chapel}
\begin{chapelpost}
writeln((r, r2, r3, r4));
\end{chapelpost}
\begin{chapeloutput}
(1..20, 3..20, 1..19 by 2, 3..15 by 6)
\end{chapeloutput}
\end{example}

\subsection{Predefined Functions and Methods on Ranges}
\index{ranges!predefined functions}
\begin{protohead}
def $range$.low : idxType
\end{protohead}
\begin{protobody}
Returns the low bound of the range.
\end{protobody}

\begin{protohead}
def $range$.high : idxType
\end{protohead}
\begin{protobody}
Returns the high bound of the range.
\end{protobody}

\begin{protohead}
def $range$.stride : int(numBits(idxType))
\end{protohead}
\begin{protobody}
Returns the stride of the range.
\end{protobody}

\begin{protohead}
def $range$.length : idxType
\end{protohead}
\begin{protobody}
Returns the number of elements in the range.
\end{protobody}

\begin{protohead}
def $range$.member(i: idxType): bool
\end{protohead}
\begin{protobody}
Returns whether or not \chpl{i} is in the range.
\end{protobody}

\begin{protohead}
def $range$.member(other: range): bool
\end{protohead}
\begin{protobody}
Returns whether or not every element in other is also in this.
\end{protobody}

\begin{protohead}
def $range$.order(i: idxType): idxType
\end{protohead}
\begin{protobody}
If \chpl{i} is a member of the range, returns an integer value giving
the ordinal value of \chpl{i} within the range using 0-based indexing.
Otherwise, it returns \chpl{(-1):idxType}.
\end{protobody}

\begin{example}
The following calls show the order of index 4 in each of the given
ranges:
\begin{chapel}
(0..10).order(4) == 4
(1..10).order(4) == 3
(3..5).order(4) == 1
(0..10 by 2).order(4) == 2
(3..5 by 2).order(4) == -1
\end{chapel}
\end{example}

\cleardoublepage
\sekshun{Domains and Arrays}
\label{Domains_and_Arrays}

A {\em domain} is a description of a collection of names for data.
These names are referred to as the {\em indices} of the domain.  All
indices for a particular domain are values with some common type.
Valid types for indices are primitive types and class references or
unions, tuples or records whose fields are valid types for indices.
This excludes sequences, domains, and arrays.  Like sequences, domains
have a rank and a total order on their elements.  An {\em array} is
generically a function that maps from a {\em domain} to a collection
of variables.  Chapel supports a variety of kinds of domains and
arrays defined over those domains as well as a mechanism to allow
application specific implementations of arrays.

Arrays abstract mappings from sets of values to variables.  This key
use of data structures coupled with the generic syntactic support for
array usage increases software reusability.  By separating the sets of
values into their own abstraction, domains, distributions can be
associated with sets rather than variables.  This enables the
orthogonality of data distributions.  Distributions are discussed
in~\rsec{Locality_and_Distribution}.

\subsection{Domains}
\label{Domains}

Domains are first-class ordered sets of indices.  There are five kinds
of domains:
\begin{itemize}
\item
Arithmetic domains are rectilinear sets of Cartesian indices that can
have an arbitrary rank.
\item
Sparse domains are subsets of indices in arithmetic domains.
\item
Indefinite domains are sets of indices where the type of the index is
some type that is not an array, domain, or sequence.  Indefinite
domains define dictionaries or associative arrays implemented via hash
tables.
\item
Opaque domains are sets of anonymous indices.  Opaque domains define
graphs and unspecified sets.
\item
Enumerated domains are sets of constants defined by some enumerated
type.
\end{itemize}

\subsubsection{Domain Types}
\label{Domain_Types}

Domain types vary based on the kind of the domain.  The type of an
arithmetic domain is parameterized by the rank of the domain and the
integral type of the indices.  The type of a sparse domain is
parameterized by the type of the arithmetic domain that defines the
superset of its indices.  The type of an indefinite domain is
parameterized by the type of the index.  The type of an opaque domain
is unique.  The type of an enumerated domain is parameterized by the
enumerated type.

\begin{example}
In the code
\begin{chapel}
var D: domain(2) = [1..n, 1..n];
\end{chapel}
\chpl{D} is defined as a two-dimensional arithmetic domain and is
initialized to contain the set of indices $(i,j)$ for all $i$ and $j$
such that $i \in {1, 2, \ldots, n}$ and $j \in {1, 2, \ldots, n}$.
\end{example}

\subsubsection{Index Types}
\label{Index_Types}

Each domain has a corresponding {\em index} type which is the type of
the domain's indices qualified by its status as an index.  Variables
of this type can be declared using the following syntax:
\begin{syntax}
index-type:
  `index' ( domain-expression )
\end{syntax}
If the type of the indices of the domain is \chpl{int}, then the index
type can be converted into this type.

A value with a type that is the same as the type of the indices in a
domain but is not the index type can be converted into the index type
using a special ``method'' called \chpl{index}.

\begin{example}
In the code
\begin{chapel}
var j = D.index(i);
\end{chapel}
the type of the variable \chpl{j} is the index type of
domain \chpl{D}.  The variable \chpl{i}, which must have the same type
as the underlying type of the indices of \chpl{D}, is verified to be
in domain \chpl{D} before it is assigned to \chpl{j}.
\end{example}

Values of index type are known to be valid and may have specialized
representations to facilitate accessing arrays defined for that
domain. It may therefore be less expensive to access arrays using
values of appropriate index type rather than values of the more
general type the domain is defined over.

\begin{implementation}
In the current implementation, the index type is not distinguished
from the underlying type of the indices.  The index method is not yet
implemented.
\end{implementation}

\subsubsection{Domain Assignment}
\label{Domain_Assignment}

Domain assignment is by value.  If arrays are declared over a domain,
domain assignment impacts these arrays as discussed
in~\rsec{Association_of_Arrays_to_Domains}, but the arrays remain
associated with the same domain regardless of the assignment.

\subsubsection{Formal Arguments of Domain Type}
\label{Formal_Arguments_of_Domain_Type}

Domains are passed to functions by reference.  Formal arguments that
receive domains are aliases of the actual arguments.  It is a
compile-time error to pass a domain to a formal argument that has a
non-blank intent.

\subsubsection{Iteration over Domains}
\label{Iteration_over_Domains}

All domains support iteration via forall and for loops over the
indices in the set that the domain defines.  The type of the indices
returned by iterating over a domain is the index type of the domain.

\subsubsection{Domain Promotion of Scalar Functions}
\label{Domain_Promotion_of_Scalar_Functions}
\index{domains!promotion}

Domain promotion of a scalar function is defined over the domain type
and the type of the indices of the domain (not the index type).
Domain promotion has the same semantics as sequence promotion where
the scalar type is the indices of the domain and the promotion type is
the domain type.

\begin{example}
Given an array \chpl{A} with element type \chpl{int} declared over a
one-dimensional domain \chpl{D} with integral type \chpl{int}, then
the array can be assigned the values given by the indices in the
domain by writing
\begin{chapel}
A = D;
\end{chapel}
\end{example}

\subsection{Arrays}
\label{Arrays}

Arrays associate variables or elements with the sets of indices in a
domain.  Arrays must be declared over domains and have a specified
element type.

\subsubsection{Array Types}
\label{Array_Types}

The type of an array is parameterized by the type of the domain that
it is declared over and the element type of the array.  Array types
are given by the following syntax:
\begin{syntax}
array-type:
  [ domain-expression ] type

domain-expression:
  expression
\end{syntax}
The \sntx{domain-expression} must specify a domain that the array can
be declared over.  This can be a domain literal.  If it is a domain
literal, the square brackets around the domain literal can be omitted.

\begin{example}
In the code
\begin{chapel}
var A: [D] real;
\end{chapel}
\chpl{A} is declared to be an array over domain \chpl{D} with
elements of type \chpl{real}.
\end{example}

\begin{implementation}
Arrays of arrays are not currently supported.
\end{implementation}

\subsubsection{Array Indexing}
\label{Array_Indexing}

Arrays can be indexed by indices in the domain they are declared over.
The indexing results in an access of the element that is mapped by
this index.

\begin{example}
If \chpl{A} is an array with element type \chpl{real} declared over a
one-dimensional arithmetic domain \chpl{[1..n]}, then the first
element in \chpl{A} can be accessed via the expression \chpl{A(1)} and
set to zero via the assignment \chpl{A(1) = 0.0}.
\end{example}

Indexing into an array with a domain is call array slicing and is
discussed in the next section.

Arithmetic arrays also support indexing over the components of their
indices for multidimensional arithmetic domains (where the indices are
tuples), as described in~\rsec{Arithmetic_Array_Indexing}.

\subsubsection{Array Slicing}
\label{Array_Slicing}

An array can be indexed by a domain that has the same type as the
domain which the array was declared over.  Indexing in this manner has
the effect of array slicing.  The result is a new array declared over
the indexing domain where the elements in the array alias the elements
in the array being indexed.

\begin{example}
Given the definitions
\begin{chapel}
var OuterD: domain(2) = [0..n+1, 0..n+1];
var InnerD: domain(2) = [1..n, 1..n];
var A, B: [OuterD] real;
\end{chapel}
the assignment given by
\begin{chapel}
A(InnerD) = B(InnerD);
\end{chapel}
assigns the elements in the interior of \chpl{B} to the elements in
the interior of \chpl{A}.
\end{example}

Arithmetic arrays also support slicing by indexing into them with
arithmetic sequences or tuples of arithmetic sequences as described
in~\rsec{Arithmetic_Array_Slicing}.

\subsubsection{Array Assignment}
\label{Array_Assignment}

Array assignment is by value.  Arrays can be assigned arrays,
sequences, or domains.  If \chpl{A} is an lvalue of array type
and \chpl{B} is an expression of either array, sequence, or domain
type, then the assignment
\begin{chapel}
A = B;
\end{chapel}
is equivalent to
\begin{chapel}
forall (i,e) in (A.domain,B) do
  A(i) = e;
\end{chapel}
If the zipper iteration is illegal, then the assignment is illegal.
Notice that the assignment is implemented with the semantics of
a \chpl{forall} loop.

Arrays can also be assigned single values of their element type.  In
this case, each element in the array is assigned this value.
If \chpl{e} is an expression of the element type of the array or a
type that can be implicitly converted to the element type of the
array, then the assignment
\begin{chapel}
A = e;
\end{chapel}
is equivalent to
\begin{chapel}
forall i in A.domain do
  A(i) = e;
\end{chapel}

\subsubsection{Formal Arguments of Array Type}
\label{Formal_Arguments_of_Array_Type}

Arrays are passed to functions by reference.  Formal arguments that
receive arrays are aliases of the actual arguments.  The ordinary rule
that disallows assignment to formal arguments of blank intent does not
apply to arrays.

\subsubsection{Iteration over Arrays}
\label{Iteration_over_Arrays}

All arrays support iteration via forall and for loops over the
elements mapped to by the indices in the array's domain.

\subsubsection{Array Promotion of Scalar Functions}
\label{Array_Promotion_of_Scalar_Functions}

Array promotion of a scalar function is defined over the array type
and the element type of the array.  Array promotion has the same
semantics as sequence promotion where the scalar type is the element
type of the array and the promotion type is the array type.  The only
difference between sequence promotion and array promotion is that if a
function returns a value, the promoted function returns an array of
those values rather than a sequence of those values.  The array is
defined over the same domain as the array that was passed to the
function.  In the event of zipper promotion over multiple arrays or
both arrays and sequences, the promoted function returns a value based
on the first argument to the function that enables promotion.

\begin{implementation}
In the current implementation, promotion always returns sequences.
\end{implementation}

\begin{example}
Whole array operations is a special case of array promotion of scalar
functions.  In the code
\begin{chapel}
A = B + C;
\end{chapel}
if \chpl{A}, \chpl{B}, and \chpl{C} are arrays, this code assigns each
element in \chpl{A} the element-wise sum of the elements in \chpl{B}
and \chpl{C}.
\end{example}

\subsubsection{Array Initialization}
\label{Array_Initialization}

By default, the elements in an array are initialized to the default
values associated with the element type of the array.  There is an
expectation that this default initialization can be overridden for
performance reasons by explicitly marking the array type or variable.

The initialization expression in the declaration of an array can be
based on the indices in the domain using special array declaration
syntax that replaces both the type and initialization specifications
of the declaration:
\begin{syntax}
special-array-declaration:
  identifier-list indexed-array-type-part initialization-part

indexed-array-type-part:
  : array-type-forall-expression type

array-type-forall-expression:
  [ identifier `in' domain-expression ]

initialization-part:
  = expression
\end{syntax}
In this code, the \sntx{array-type-forall-expression} is syntactic
sugar for surrounding the \sntx{initialization-part} with this basic
forall-expression.

Given a domain expression \chpl{D}, an element type \chpl{t}, an
expression \chpl{e} that is of type \chpl{t} or that can be implicitly
converted to type \chpl{t}, then the declaration of array \chpl{A}
given by
\begin{chapel}
var A: [i in D] t = e;
\end{chapel}
is equivalent to
\begin{chapel}
var A: [D] t = [i in D] e;
\end{chapel}
The scope of the forall expression is as in the rewritten part so the
expression \chpl{e} can include references to index \chpl{i}.

\subsection{Arithmetic Domains and Arrays}
\label{Arithmetic_Domains_and_Arrays}

An arithmetic domain is a rectilinear set of Cartesian indices.
Arithmetic domains are specified as a tuple of arithmetic sequences
enclosed in square brackets.

\subsubsection{Arithmetic Domain Literals}
\label{Arithmetic_Domain_Literals}

An arithmetic domain literal is specified by the following syntax:
\begin{syntax}
arithmetic-domain-literal:
  [ arithmetic-sequence-expression-list ]

arithmetic-sequence-expression-list:
  arithmetic-sequence-expression
  arithmetic-sequence-expression , arithmetic-sequence-expression-list

arithmetic-sequence-expression:
  expression
\end{syntax}

\begin{example}
The expression \chpl{[1..5, 1..5]} defines a $5 \times 5$ arithmetic
domain with the indices $(1, 1), (1, 2), \ldots, (5, 5)$.
\end{example}

\subsubsection{Arithmetic Domain Types}
\label{Arithmetic_Domain_Types}

The type of an arithmetic domain is determined from the rank of the
arithmetic domain (the number of arithmetic sequences that define it)
and by an underlying integeral type called the {\em dimensional index
type} which must be identical to each of the integral element types of
the arithmetic sequences that define the arithmetic domain.  By
default, the dimensional index type of an arithmetic domain
is \chpl{int}.

The arithmetic domain type is specified by the syntax of a function
call to the keyword \chpl{domain} that takes at least an argument
called \chpl{rank} that is a parameter of type \chpl{int} and
optionally an integral type named \chpl{dim_type}.

\begin{example}
The expression \chpl{[1..5, 1..5]} defines an arithmetic domain with
type \chpl{domain(2,int)}.
\end{example}

\subsubsection{Arithmetic Domain Indexing}
\label{Arithmetic_Domain_Indexing}

Arithmetic domains support indexing by a value of type \chpl{int} that
is at least one and no more than the rank of the array.  Indexing into
an arithmetic domain returns the arithmetic sequence associated with
that dimension.

\begin{example}
In the code
\begin{chapel}
for i in D(1) do
  for j in D(2) do
    writeln(A(i,j));
\end{chapel}
domain \chpl{D} is iterated over by two nested loops.  The first
dimension of \chpl{D} is iterated over in the outer loop.  The second
dimension is iterated over in the inner loop.
\end{example}

\subsubsection{Arithmetic Array Indexing}
\label{Arithmetic_Array_Indexing}

In addition to being indexed by indices defined by their arithmetic
domains, arithmetic arrays can be indexed directly by values of the
dimensional index type where the number of values is equal to the rank
of the array.  This has the semantics of first moving the values into
a tuple and then indexing into the array.

\begin{example}
Given the definition
\begin{chapel}
  var ij = (i,j);
\end{chapel}
the indexing expressions \chpl{A(ij)} and \chpl{A(i,j)} are
equivalent.
\end{example}

\subsubsection{Arithmetic Array Slicing}
\label{Arithmetic_Array_Slicing}

This is a stub.  This portion of the document does not exist.

\subsubsection{Formal Arguments of Arithmetic Array Type}
\label{Formal_Arguments_of_Arithmetic_Array_Type}

This is a stub.  This portion of the document does not exist.

\subsection{Sparse Domains and Arrays}
\label{Sparse_Domains_and_Arrays}

This is a stub.  This portion of the document does not exist.

\subsubsection{Adding Indices to Sparse Domains}
\label{Adding_Indices_to_Sparse_Domains}

This is a stub.  This portion of the document does not exist.

\subsubsection{Removing Indices from Sparse Domains}
\label{Removing_Indices_from_Sparse_Domains}

This is a stub.  This portion of the document does not exist.

\subsection{Indefinite Domains and Arrays}
\label{Indefinite_Domains_and_Arrays}

This is a stub.  This portion of the document does not exist.

\subsubsection{Indefinite Domain and Array Types}
\label{Indefinite_Domain_and_Array_Types}

This is a stub.  This portion of the document does not exist.

\subsubsection{Indefinite Domain Index Types}
\label{Indefinite_Domain_Index_Types}

This is a stub.  This portion of the document does not exist.

\subsubsection{Adding Indices to Indefinite Domains}
\label{Adding_Indices_to_Indefinite_Domains}

This is a stub.  This portion of the document does not exist.

\subsubsection{Removing Indices from Indefinite Domains}
\label{Removing_Indices_from_Indefinite_Domains}

This is a stub.  This portion of the document does not exist.

\subsection{Opaque Domains and Arrays}
\label{Opaque_Domains_and_Arrays}

This is a stub.  This portion of the document does not exist.

\subsubsection{Opaque Domain and Array Types}
\label{Opaque_Domain_and_Array_Types}

This is a stub.  This portion of the document does not exist.

\subsubsection{Opaque Domain Index Types}
\label{Opaque_Domain_Index_Types}

This is a stub.  This portion of the document does not exist.

\subsubsection{Adding Indices to Opaque Domains}
\label{Adding_Indices_to_Opaque_Domains}

This is a stub.  This portion of the document does not exist.

\subsubsection{Removing Indices from Opaque Domains}
\label{Removing_Indices_from_Opaque_Domains}

This is a stub.  This portion of the document does not exist.

\subsection{Enumerated Domains and Arrays}
\label{Enumerated_Domains_and_Arrays}

This is a stub.  This portion of the document does not exist.

\subsubsection{Enumerated Domain and Array Types}
\label{Enumerated_Domain_and_Array_Types}

This is a stub.  This portion of the document does not exist.

\subsubsection{Enumerated Domain Index Types}
\label{Enumerated_Domain_Index_Types}

This is a stub.  This portion of the document does not exist.

\subsection{Association of Arrays to Domains}
\label{Association_of_Arrays_to_Domains}

This is a stub.  This portion of the document does not exist.

\subsubsection{Preservative Reallocation of Arrays}
\label{Preservative_Reallocation_of_Arrays}

This is a stub.  This portion of the document does not exist.

\subsubsection{Destructive Reallocation of Arrays}
\label{Destructive_Reallocation_of_Arrays}

This is a stub.  This portion of the document does not exist.

\subsection{Subdomains}
\label{Subdomains}

This is a stub.  This portion of the document does not exist.

\subsubsection{Subdomain Definition}
\label{Subdomain_Definition}

This is a stub.  This portion of the document does not exist.

\subsubsection{Association of Subdomains to Domains}
\label{Association_of_Subdomains_to_Domains}

This is a stub.  This portion of the document does not exist.

\subsection{Predefined Functions and Methods on Domains and Arrays}

{\bf Functions and Methods on Domains}

\begin{chapel}
def Domain.numIndices: dim_type
\end{chapel}
Returns the number of indices in the domain.

{\bf Functions and Methods on Arrays}

\begin{chapel}
def Array.numElements: this.domain.dim_type
\end{chapel}
Returns the number of elements in the array.

\cleardoublepage
\sekshun{Iterators}
\label{Iterators}
\index{iterators}

An iterator is a function that conceptually returns multiple values
rather than simply a single value.

\begin{openissue}
The parallel iterator story is under development.  It is expected that
the specification will be expanded regarding parallel iterators soon.
\end{openissue}

\subsection{Iterator Function Definitions}
\label{Iterator_Function_Definitions}
\index{iterator function definitions}

The syntax to declare an iterator function (or simply, ``iterator''), is given
by:
\begin{syntax}
iterator-declaration-statement:
  `iter' iterator-name argument-list[OPT] var-param-type-clause[OPT] where-clause[OPT]
  iterator-body

iterator-name:
  identifier

iterator-body:
  block-statement
  yield-statement
\end{syntax}

The syntax of an iterator declaration is similar to a function declaration, with
some key differences:
\begin{itemize}
\item The keyword \chpl{iter} is used instead of the keyword \chpl{proc}.
\item The name of the iterator cannot overload any operator.
\item \chpl{yield} statements may appear in the body of an iterator, but not in
a regular function.
\end{itemize}

\subsection{The Yield Statement}
\label{The_Yield_Statement}
\index{yield@\chpl{yield}}

The yield statement can only appear in iterators.  The syntax of the
yield statement is given by
\begin{syntax}
yield-statement:
  `yield' expression ;
\end{syntax}

When an iterator is executed and a \chpl{yield} is encountered, the value of the yield
expression is returned.  However, the state of execution of the iterator is
saved.  On its next invocation, execution resumes from the point immediately
following that \chpl{yield} statement.

When a \chpl{return} is encountered, the iterator finishes without yielding another
index value.  The \chpl{return} statements appearing in an iterator are not
permitted to have a return value.
An iterator also completes after the last
statement in the iterator function is executed.
An iterator need not contain any yield statements.

\subsection{Iterator Calls}
\label{Iterator_Calls}

The syntax used to call an interator is given by:
\begin{syntax}
iterator-call-expression:
  call-expression
\end{syntax}
This is identical to the function-call syntax.
%REVIEW: hilde
% Can iterator definitions and uses have field syntax?

All details of the \sntx{iterator-call-expression} semantics --- including
resolution, the use of parentheses versus brackets to delimit the parameter
list, calling the iterator without an argument list and named arguments ---
are identical with the corresponding semantics for function calls.
See~\rsec{Function_Calls}.

However, the result of an iterator call depends upon its context, as described below.

\subsubsection{Iterators in For and Forall Loops}
\label{Iterators_in_For_and_Forall_Loops}

When an iterator is accessed via a for or forall loop, the iterator is
evaluated alongside the loop body in an interleaved manner.  For each
iteration, the iterator yields a value and the body is executed.

\subsubsection{Iterators as Arrays}
\label{Iterators_as_Arrays}
\index{iterators!and arrays}

If an iterator function is captured into a new variable declaration or
assigned to an array, the iterator is iterated over in total and the
expression evaluates to a one-dimensional arithmetic array that
contains the values returned by the iterator on each iteration.
\begin{example}
Given an iterator
\begin{chapel}
iter squares(n: int): int {
  for i in 1..n do
    yield i * i;
}
\end{chapel}
\begin{chapelpost}
writeln(squares(5));
\end{chapelpost}
\begin{chapeloutput}
1 4 9 16 25
\end{chapeloutput}
the expression \chpl{squares(5)} evaluates to the array \chpl{1, 4, 9, 16, 25}.
\end{example}

\subsubsection{Iterators and Generics}
\label{Iterators_and_Generics}
\index{iterators!and generics}

An iterator call expression can be passed to a generic function argument that
has neither a declared type nor default value
(\rsec{Formal_Arguments_without_Types}).
In this case the iterator is passed without being evaluated.
Within the generic function the corresponding formal argument
can be used as an iterator, e.g. in for loops.
The arguments to the iterator call expression, if any, are evaluated
at the call site, i.e. prior to passing the iterator to the generic function.

\subsubsection{Recursive Iterators}
\label{Recursive_Iterators}
\index{iterators!recursive}

Recursive iterators are allowed. A recursive iterator invocation is
typically made by iterating over it in a loop.


\begin{example}
A post-order traversal of a tree data structure could be written like this:
\begin{chapel}
iter postorder(tree: Tree): string {
  if tree != nil {
    for child in postorder(tree.left) do
      yield child;
    for child in postorder(tree.right) do
      yield child;
    yield tree.data;
  }
}
\end{chapel}
By contrast, using calls \chpl{postorder(tree.left)}
and \chpl{postorder(tree.right)} as stand-alone statements would
result in generating temporary arrays containing the outcomes of these
recursive calls, which would then be discarded.
\end{example}

\subsection{Parallel Iterators}
\label{Parallel_Iterators}

Iterators used in explicit forall-statements or -expressions must be
parallel iterators.  Reductions, scans and promotion over serial
iterators will be serialized.

The definition of parallel iterators is forthcoming.  Parallel
iterators are defined over standard constructs in Chapel such as
ranges, domains, and arrays (including Block- and Cyclic-distributed
domains and arrays), thereby allowing these constructs to be used with
forall-statements and -expressions.

\cleardoublepage
\sekshun{Generics}
\label{Generics}

Chapel supports generic functions and types that are parameterizable
over both types and parameters.  The generic functions and types look
similar to non-generic functions and types already discussed.

\section{Generic Functions}
\label{Generic_Functions}
\index{functions!generic}
\index{generics!functions}

A function is generic if any of the following conditions hold:
\begin{itemize}
\item
Some formal argument is specified with an intent of \chpl{type} or
\chpl{param}.
\item
Some formal argument has no specified type and no default value.
\item
Some formal argument is specified with a queried type.
\item
The type of some formal argument is a generic type, e.g., \chpl{List}.
Queries may be inlined in generic types, e.g., \chpl{List(?eltType)}.
\item
The type of some formal argument is an array type where either the
element type is queried or omitted or the domain is queried or
omitted.
\end{itemize}
These conditions are discussed in the next sections.

\subsection{Formal Type Arguments}
\label{Formal_Type_Arguments}
\index{intents!type@\chpl{type}}

If a formal argument is specified with intent \chpl{type}, then a type
must be passed to the function at the call site.  A copy of the
function is instantiated for each unique type that is passed to this
function at a call site.  The formal argument has the semantics of a
type alias.
\begin{example}
The following code defines a function that takes two types at the call
site and returns a 2-tuple where the types of the components of the
tuple are defined by the two type arguments and the values are
specified by the types default values.
\begin{chapelpre}
% build2tuple.chpl
\end{chapelpre}
\begin{chapel}
proc build2Tuple(type t, type tt) {
  var x1: t;
  var x2: tt;
  return (x1, x2);
}
\end{chapel}
This function is instantiated with ``normal'' function call syntax
where the arguments are types:
\begin{chapel}
var t2 = build2Tuple(int, string);
t2 = (1, "hello");
\end{chapel}
\begin{chapelpost}
writeln(t2);
\end{chapelpost}
\begin{chapeloutput}
(1, hello)
\end{chapeloutput}
\end{example}

\subsection{Formal Parameter Arguments}
\label{Formal_Parameter_Arguments}
\index{intents!param@\chpl{param}}

If a formal argument is specified with intent \chpl{param}, then a
parameter must be passed to the function at the call site.  A copy of
the function is instantiated for each unique parameter that is passed
to this function at a call site.  The formal argument is a parameter.
\begin{example}
The following code defines a function that takes an integer parameter
\chpl{p} at the call site as well as a regular actual argument of
integer type \chpl{x}.  The function returns a homogeneous tuple of
size \chpl{p} where each component in the tuple has the value of
\chpl{x}.
\begin{chapelpre}
% fillTuple.chpl
\end{chapelpre}
\begin{chapel}
proc fillTuple(param p: int, x: int) {
  var result: p*int;
  for param i in 1..p do
    result(i) = x;
  return result;
}
\end{chapel}
\begin{chapelpost}
writeln(fillTuple(3,3));
\end{chapelpost}
\begin{chapeloutput}
(3, 3, 3)
\end{chapeloutput}
The function call \chpl{fillTuple(3, 3)} returns a 3-tuple where each
component contains the value \chpl{3}.
\end{example}

\subsection{Formal Arguments without Types}
\label{Formal_Arguments_without_Types}
\index{formal arguments!without types}

If the type of a formal argument is omitted, the type of the formal
argument is taken to be the type of the actual argument passed to the
function at the call site.  A copy of the function is instantiated for
each unique actual type.
\begin{example}
The example from the previous section can be extended to be generic on
a parameter as well as the actual argument that is passed to it by
omitting the type of the formal argument \chpl{x}.  The following code
defines a function that returns a homogeneous tuple of size \chpl{p}
where each component in the tuple is initialized to \chpl{x}:
\begin{chapelpre}
% fillTuple2.chpl
\end{chapelpre}
\begin{chapel}
proc fillTuple(param p: int, x) {
  var result: p*x.type;
  for param i in 1..p do
    result(i) = x;
  return result;
}
\end{chapel}
\begin{chapelpost}
var x = fillTuple(3, 3.14);
writeln(x);
writeln(typeToString(x.type));
\end{chapelpost}
\begin{chapeloutput}
(3.14, 3.14, 3.14)
3*real
\end{chapeloutput}
In this function, the type of the tuple is taken to be the type of the
actual argument.  The call \chpl{fillTuple(3, 3.14)} returns a 3-tuple
of real values \chpl{(3.14, 3.14, 3.14)}.  The return type is
\chpl{(real, real, real)}.
\end{example}

\subsection{Formal Arguments with Queried Types}
\label{Formal_Arguments_with_Queried_Types}
\index{formal arguments!queried types}

If the type of a formal argument is specified as a queried type, the
type of the formal argument is taken to be the type of the actual
argument passed to the function at the call site.  A copy of the
function is instantiated for each unique actual type.  The queried
type has the semantics of a type alias.
\begin{example}
The example from the previous section can be rewritten to use a
queried type for clarity:
\begin{chapelpre}
% fillTuple3.chpl
\end{chapelpre}
\begin{chapel}
proc fillTuple(param p: int, x: ?t) {
  var result: p*t;
  for param i in 1..p do
    result(i) = x;
  return result;
}
\end{chapel}
\begin{chapelpost}
var x = fillTuple(3, 3.14);
writeln(x);
writeln(typeToString(x.type));
\end{chapelpost}
\begin{chapeloutput}
(3.14, 3.14, 3.14)
3*real
\end{chapeloutput}
\end{example}

\subsection{Formal Arguments of Generic Type}
\label{Formal_Arguments_of_Generic_Type}
\index{formal arguments!generic types}

If the type of a formal argument is a generic type, the type of the
formal argument is taken to be the type of the actual argument passed
to the function at the call site with the constraint that the type of
the actual argument is an instantiation of the generic type.  A copy
of the function is instantiated for each unique actual type.
\begin{example}
The following code defines a function \chpl{writeTop} that takes an
actual argument that is a generic stack
(see~\rsec{Example_Generic_Stack}) and outputs the top element of the
stack.  The function is generic on the type of its argument.
\begin{chapel}
proc writeTop(s: Stack) {
  write(s.top.item);
}
\end{chapel}
\end{example}

Types and parameters may be queried from the top-level types of formal
arguments as well.  In the example above, the formal argument's type
could also be specified as \chpl{Stack(?type)} in which case the
symbol \chpl{type} is equivalent to \chpl{s.itemType}.

Note that generic types which have default values for all of their
generic fields, \emph{e.g. range}, are not generic when simply
specified and require a query to mark the argument as generic.  For
simplicity, the identifier may be omitted.
\begin{example}
The following code defines a class with a type field that has a
default value.  Function \chpl{f} is defined to take an argument of
this class type where the type field is instantiated to the default.
Function \chpl{g}, on the other hand, is generic on its argument
because of the use of the question mark.
\begin{chapel}
class C {
  type t = int;
}
proc f(c: C) {
  // c.type is always int
}
proc g(c: C(?)) {
  // c.type may not be int
}
\end{chapel}
\end{example}

\index{where@\chpl{where}!implicit}
The generic type may be specified with some queries and some exact
values.  Thesse exact values result in \emph{implicit where clauses}
for the purpose of function resolution.
\begin{example}
Given the class definition
\begin{chapel}
class C {
  type t;
  type tt;
}
\end{chapel}
then the function definition
\begin{chapel}
proc f(c: C(?t,real)) {
  // body
}
\end{chapel}
is equivalent to
\begin{chapel}
proc f(c: C(?t,?tt)) where tt == real {
  // body
}
\end{chapel}
\end{example}
For tuples with query arguments, an implicit where clause is always
created to ensure that the size of the actual tuple matches the
implicitly specified size of the formal tuple.
\begin{example}
The function definition
\begin{chapel}
proc f(tuple: (?t,real)) {
  // body
}
\end{chapel}
is equivalent to
\begin{chapel}
proc f(tuple: (?t,?tt)) where tuple.size == 2 && tt == real {
  // body
}
\end{chapel}
\end{example}

\index{integral@\chpl{integral}}
\index{numeric@\chpl{numeric}}
\index{enumerated@\chpl{enumerated}}
The generic types \chpl{integral}, \chpl{numeric} and \chpl{enumerated}
are generic types that can only be instantiated with, respectively, the
signed and unsigned integral types, all of the numeric types, and
enumerated types.

\subsection{Formal Arguments of Generic Array Types}
\label{Formal_Arguments_of_Generic_Array_Types}
\index{formal arguments!array types}

If the type of a formal argument is an array where either the domain
or the element type is queried or omitted, the type of the formal
argument is taken to be the type of the actual argument passed to the
function at the call site.  If the domain is omitted, the domain of
the formal argument is taken to be the domain of the actual argument.

A queried domain may not be modified via the name to which it is bound
(see~\rsec{Association_of_Arrays_to_Domains} for rationale).

\section{Function Visibility in Generic Functions}
\label{Function_Visibility_in_Generic_Functions}
\index{generics!function visibility}

Function visibility in generic functions is altered depending on the
instantiation.  When resolving function calls made within generic
functions, the visible functions are taken from any call site at which
the generic function is instantiated for each particular
instantiation.  The specific call site chosen is arbitrary and it is
referred to as the \emph{point of instantiation}.

For function calls that specify the module
explicitly~(\rsec{Explicit_Naming}), an implicit use of the specified
module exists at the call site.

\begin{example}
Consider the following code which defines a generic
function \chpl{bar}:
\begin{chapelpre}
% point_of_instantiation.chpl
\end{chapelpre}
\begin{chapel}
module M1 {
  record R {
    var x: int;
    proc foo() { }
  }
}

module M2 {
  proc bar(x) {
    x.foo();
  }
}

module M3 {
  use M1, M2;
  proc main() {
    var r: R;
    bar(r);
  }
}
\end{chapel}
\begin{chapeloutput}
\end{chapeloutput}
In the function \chpl{main}, the variable \chpl{r} is declared to be
of type \chpl{R} defined in module \chpl{M1} and a call is made to the
generic function \chpl{bar} which is defined in module \chpl{M2}.
This is the only place where \chpl{bar} is called in this program and
so it becomes the point of instantiation for \chpl{bar} when the
argument \chpl{x} is of type \chpl{R}.  Therefore, the call to
the \chpl{foo} method in \chpl{bar} is resolved by looking for visible
functions from within \chpl{main} and going through the use of
module \chpl{M1}.
\end{example}

If the generic function is only called indirectly through dynamic
dispatch, the point of instantiation is defined as the point at which
the derived type (the type of the implicit \chpl{this} argument) is
defined or instantiated (if the derived type is generic).

\begin{rationale}
Visible function lookup in Chapel's generic functions is handled
differently than in C++'s template functions in that there is no split
between dependent and independent types.

Also, dynamic dispatch and instantiation is handled differently.
Chapel supports dynamic dispatch over methods that are generic in some
of its formal arguments.

Note that the Chapel lookup mechanism is still under development and
discussion.  Comments or questions are appreciated.
\end{rationale}

\section{Generic Types}
\label{Generic_Types}
\index{generics!types}
\index{types!generic}
\index{generics!classes}
\index{classes!generic}
\index{generics!records}
\index{records!generic}

Generic types are generic classes and generic records.
A class or record is generic if it contains one or more
\index{generics!fields}
\index{fields!generic}
generic fields. A generic field is one of:
\begin{itemize}
\item a specified or unspecified type alias,
\item a parameter field, or
\item a \chpl{var} or \chpl{const} field that has no type and no initialization
expression.
\end{itemize}

\mbox{} % push the following line to the next page

For each generic field, the class or record is parameterized over:
\begin{itemize}
\item the type bound to the type alias,
\item the value of the parameter field, or
\item the type of the \chpl{var} or \chpl{const} field, respectively.
\end{itemize}
Correspondingly, the class or record is instantiated with a set
of types and parameter values, one type or value per generic field.

% Here are the aspects to be defined for each kind of generic field:
% - what it makes the class/record generic over
% - the type constructor arg that gets created
% - the default constructor arg that gets created
% - the requirements on the corresponding user-defined constructor arg
% - for each of the above args:
%    - what kind of actual it accepts (type, param, value)
%    - what is the semantics;
%      i.e. how it corresponds to the class/record's genericity
%    - what is the arg's default, if any
% 
% In the presentation below, some of these aspects are discussed
% in the field-kind-specific subsections, some in the constructor-specific
% subsections, some in both.  I.e. there is an overlap between
% field-kind and constructor subsections; that should be OK but feel free
% to clean up.
% 
% It would be cool to summarize that in a table
% (one dimension: field kinds; the other dimension: aspects).

\subsection{Type Aliases in Generic Types}
\label{Type_Aliases_in_Generic_Types}
\index{type aliases!in classes or records}
\index{fields!type alias}

If a class or record defines a type alias, the class or record
is generic over the type that is bound to that alias.
% Type aliases defined in a class or a record can be unspecified type
% aliases; type aliases that are not bound to a type.  If a class or
% record contains an unspecified type alias, the aliased type must be
% specified whenever the type is used.
Such a type alias is accessed as if it were a field;
similar to a parameter field, it cannot be assigned
except in its declaration.

The type alias becomes an argument with intent \chpl{type} to
the compiler-generated constructor (\rsec{Generic_Compiler_Generated_Constructors})
for its class or record. This makes the compiler-generated constructor generic.
The type alias also becomes an argument with intent \chpl{type} to
the type constructor (\rsec{Type_Constructors}).
If the type alias declaration binds it to a type, that type
becomes the default for these arguments, otherwise they have no defaults.

The class or record is instantiated by binding the type alias
to the actual type passed to the corresponding argument of
a user-defined (\rsec{Generic_User_Constructors})
or compiler-generated constructor or type constructor.
If that argument has a default, the actual type can be omitted, in
which case the default will be used instead.

\begin{example}
The following code defines a class called \chpl{Node} that implements
a linked list data structure.  It is generic over the type of the
element contained in the linked list.
\begin{chapelpre}
% NodeClass.chpl
\end{chapelpre}
\begin{chapel}
class Node {
  type eltType;
  var data: eltType;
  var next: Node(eltType);
}
\end{chapel}
\begin{chapelpost}
var n: Node(real) = new Node(real, 3.14);
writeln(n.data);
writeln(n.next);
writeln(typeToString(n.next.type));
\end{chapelpost}
\begin{chapeloutput}
3.14
nil
Node(real)
\end{chapeloutput}
The call \chpl{new Node(real, 3.14)} creates a node in the linked list
that contains the value \chpl{3.14}.  The \chpl{next} field is set to
nil.  The type specifier \chpl{Node} is a generic type and cannot be
used to define a variable.  The type specifier \chpl{Node(real)}
denotes the type of the \chpl{Node} class instantiated over
\chpl{real}.  Note that the type of the \chpl{next} field is specified
as \chpl{Node(eltType)}; the type of \chpl{next} is the same type as
the type of the object that it is a field of.
\end{example}

\subsection{Parameters in Generic Types}
\label{Parameters_in_Generic_Types}
\index{parameters!in classes or records}
\index{fields!parameter}

If a class or record defines a parameter field, the class or record
is generic over the value that is bound to that field.
% A parameter defined in a class or record is accessed as if it were a
% field.  This access returns a parameter.  
The parameter becomes an argument with intent \chpl{param} to the
compiler-generated constructor (\rsec{Generic_Compiler_Generated_Constructors})
for that class or record.  This makes the compiler-generated
constructor generic.  The parameter also becomes an argument
with intent \chpl{param} to the type  constructor (\rsec{Type_Constructors}).
If the parameter declaration has an initialization expression, that expression
becomes the default for these arguments, otherwise they have no defaults.

The class or record is instantiated by binding the parameter
to the actual value passed to the corresponding argument of
a user-defined (\rsec{Generic_User_Constructors})
or compiler-generated constructor or type constructor.
If that argument has a default, the actual value can be omitted, in
which case the default will be used instead.

\begin{example}
The following code defines a class called \chpl{IntegerTuple} that is
generic over an integer parameter which defines the number of
components in the class.
\begin{chapelpre}
% IntegerTuple.chpl
\end{chapelpre}
\begin{chapel}
class IntegerTuple {
  param size: int;
  var data: size*int;
}
\end{chapel}
\begin{chapelpost}
var x = new IntegerTuple(3);
writeln(x.data);
\end{chapelpost}
\begin{chapeloutput}
(0, 0, 0)
\end{chapeloutput}
The call \chpl{new IntegerTuple(3)} creates an instance of the
\chpl{IntegerTuple} class that is instantiated over parameter
\chpl{3}.  The field \chpl{data} becomes a 3-tuple of integers.  The
type of this class instance is \chpl{IntegerTuple(3)}.  The type
specified by \chpl{IntegerTuple} is a generic type.
\end{example}

\subsection{Fields without Types}
\label{Fields_without_Types}
\index{fields!variable and constant, without types}
\index{variables!in classes or records}
\index{constants!in classes or records}

If a \chpl{var} or \chpl{const} field in a class or record has no specified type or
initialization expression, the class or record is generic over the
type of that field.  The field becomes an argument with blank intent to
the compiler-generated constructor (\rsec{Generic_Compiler_Generated_Constructors}).
That argument has no specified type and no default
value. This makes the compiler-generated constructor generic.
The field also becomes an argument with \chpl{type} intent and no default
to the type constructor (\rsec{Type_Constructors}).
Correspondingly, an actual value must always be passed to the default
constructor argument and an actual type to the type constructor argument.

The class or record is instantiated by binding the type of the field
to the type of the value passed to the corresonding argument
of a user-defined (\rsec{Generic_User_Constructors}) or compiler-generated constructor (\rsec{Generic_Compiler_Generated_Constructors}).
When the type constructor is invoked, the class or record is instantiated
by binding the type of the field to the actual type passed to
the corresponding argument.

\begin{example}
The following code defines another class called \chpl{Node} that
implements a linked list data structure.  It is generic over the type
of the element contained in the linked list.  This code does not
specify the element type directly in the class as a type alias but
rather omits the type from the \chpl{data} field.
\begin{chapelpre}
% fieldWithoutType.chpl
\end{chapelpre}
\begin{chapel}
class Node {
  var data;
  var next: Node(data.type) = nil;
}
\end{chapel}
A node with integer element type can be defined in the call to the
constructor.  The call \chpl{new Node(1)} defines a node with the
value \chpl{1}.  The code
\begin{chapel}
var list = new Node(1);
list.next = new Node(2);
\end{chapel}
\begin{chapelpost}
writeln(list.data);
writeln(list.next.data);
\end{chapelpost}
\begin{chapeloutput}
1
2
\end{chapeloutput}
defines a two-element list with nodes containing the values \chpl{1}
and \chpl{2}.  The type of each object could be specified
as \chpl{Node(int)}.
\end{example}

\subsection{The Type Constructor}
\label{Type_Constructors}
\index{generics!type constructor}
\index{constructors!type constructors}

A type constructor is automatically created for each class or record.
A type constructor is a type function (\rsec{Type_Functions}) that has
the same name as the class or record.  It takes one argument per the
class's or record's generic field, including fields inherited from the
superclasses, if any.
The formal argument has intent \chpl{type} for a type alias field and is a
parameter for a parameter field. It accepts the type to be bound
to the type alias and the value to be bound to the parameter, respectively.
For a generic \chpl{var} or \chpl{const} field, the corresponding
formal argument also has intent \chpl{type}. It accepts the type
of the field, as opposed to a value as is the case for a parameter field.
The formal arguments occur in the same order as the fields are
declared and have the same names as the corresponding fields.
Unlike the compiler-generated constructor, the type constructor has only
those arguments that correspond to generic fields.

A call to a type constructor accepts actual types and parameter values
and returns the type of the class or record that is instantiated
appropriately for each field
(\rsec{Type_Aliases_in_Generic_Types}, \rsec{Parameters_in_Generic_Types},
\rsec{Fields_without_Types}).
\index{generics!instantiated type}
Such an instantiated type must be used as the type of a variable,
array element, non-generic formal argument, and in other cases
where uninstantiated generic class or record types are not allowed.

When a generic field declaration has an initialization expression
or a type alias is specified, that initializer becomes the default value
for the corresponding type constructor argument.  Uninitialized
fields, including all generic \chpl{var} and \chpl{const} fields,
and unspecified type aliases result in arguments with no defaults;
actual types or values for these arguments must always be provided
when invoking the type constructor.

\subsection{Generic Methods}
\label{Generic_Methods}
\index{generics!methods}

All methods bound to generic classes or records, including
constructors, are generic over the implicit \chpl{this} argument.
This is in addition to being generic over any other argument that is generic.

\subsection{The Compiler-Generated Constructor}
\label{Generic_Compiler_Generated_Constructors}
\index{generics!constructors!compiler-generated}
\index{constructors!compiler-generated!for generic classes or records}

If no user-defined constructors are supplied for a given generic class, the
compiler generates one following in a manner similar to that for concrete
classes (\rsec{The_Compiler_Generated_Constructor}).
However, the compiler-generated constructor for a generic class or record
(\rsec{The_Compiler_Generated_Constructor}) is generic over each argument that
corresponds to a generic field, as specified above.
The argument has intent \chpl{type} for a type alias field and is a
parameter for a parameter field. It accepts the type to be bound
to the type alias and the value to be bound to the parameter, respectively.
This is the same as for the type constructor.
For a generic \chpl{var} or \chpl{const} field, the corresponding
formal argument has the blank intent and accepts the value
for the field to be initialized with. The type of the field
is inferred automatically to be the type of the initialization value.

The default values for the generic arguments of the compiler-generated constructor
are the same as for the type constructor (\rsec{Type_Constructors}).
For example, the arguments corresponding to the generic \chpl{var}
and \chpl{const} fields, if any, never have defaults, so the corresponding
actual values must always be provided.

\subsection{User-Defined Constructors}
\label{Generic_User_Constructors}
\index{generics!constructors!user-defined}
\index{constructors!user-defined!for generic classes or records}

If a generic field of a class does not have an initialization expression
or a type alias is unspecified, each user-defined constructor for that
class must provide a formal argument whose name
matches the name of the field.

If the name of a formal argument in a user-defined constructor matches the name
of a generic field that does not have an initialization
expression, is a type alias, or is a parameter field, the field is
automatically initialized at the beginning of the constructor invocation
to the actual value of that argument.
This is done by passing that formal argument to the implicit invocation
of the compiler-generated constructor during default-initialization (\rsec{Default_Initialization}).

%%  The following story is nicer but it's not how it is implemented:
%If the name of a formal argument in a class constructor
%matches the name of a generic field, the field is automatically initialized
%to the actual value for that argument upon the constructor invocation.
%If the generic field does not have an initialization expression,
%such a matching formal argument must be provided in each constructor
%for that class.

\begin{example}
In the following code:
\begin{chapelpre}
% constructorsForGenericFields.chpl
\end{chapelpre}
\begin{chapel}
class MyGenericClass {
  type t1;
  param p1;
  const c1;
  var v1;
  var x1: t1; // this field is not generic

  type t5 = real;
  param p5 = "a string";
  const c5 = 5.5;
  var v5 = 555;
  var x5: t5; // this field is not generic

  proc MyGenericClass(c1, v1, type t1, param p1) { }
  proc MyGenericClass(type t5, param p5, c5, v5, x5,
                     type t1, param p1, c1, v1, x1) { }
}  // class MyGenericClass

var g1 = new MyGenericClass(11, 111, int, 1);
var g2 = new MyGenericClass(int, "this is g2", 3.3, 333, 3333,
                            real, 2, 222, 222.2, 22);
\end{chapel}
\begin{chapelpost}
writeln(g1);
writeln(g2);
\end{chapelpost}
\begin{chapeloutput}
{p1 = 1, c1 = 11, v1 = 111, x1 = 0, p5 = a string, c5 = 5.5, v5 = 555, x5 = 0.0}
{p1 = 2, c1 = 222, v1 = 222.2, x1 = 0.0, p5 = this is g2, c5 = 5.5, v5 = 555, x5 = 0}
\end{chapeloutput}
The arguments \chpl{t1}, \chpl{p1}, \chpl{c1}, and \chpl{v1} are
required in all constructors for \chpl{MyGenericClass}. They can appear
in any order. Both \chpl{MyGenericClass} constructors initialize the
corresponding fields implicitly because these fields do not have initialization
expressions. The second constructor also initializes implicitly
the fields \chpl{t5} and \chpl{p5} because they are a type field
and a parameter field. It does not initialize the fields \chpl{c5}
and \chpl{v5} because they have initialization expressions, or
the fields \chpl{x1} and \chpl{x5} because they are not generic fields
(even though they are of generic types).
\end{example}

\begin{openissue}
The design of constructors, especially for generic classes, is
under development, so the above specification may change.
\end{openissue}

\section{Where Expressions}
\label{Where_Expressions}
\index{where@\chpl{where}}
\index{generics!where@\chpl{where}}

The instantiation of a generic function can be constrained by {\em
where clauses}.  A where clause is specified in the definition of a
function~(\rsec{Function_Definitions}).  When a function is
instantiated, the expression in the where clause must be a parameter
expression and must evaluate to either \chpl{true} or \chpl{false}.
If it evaluates to \chpl{false}, the instantiation is rejected and the
function is not a possible candidate for function resolution.
Otherwise, the function is instantiated.
\begin{example}
Given two overloaded function definitions
\begin{chapelpre}
% whereClause.chpl
\end{chapelpre}
\begin{chapel}
proc foo(x) where x.type == int { writeln("int"); }
proc foo(x) where x.type == real { writeln("real"); }
\end{chapel}
\begin{chapelpost}
foo(3);
foo(3.14);
\end{chapelpost}
\begin{chapeloutput}
int
real
\end{chapeloutput}
the call foo(3) resolves to the first definition because when the
second function is instantiated the where clause evaluates to false.
\end{example}

\section{User-Defined Compiler Diagnostics}
\label{User_Defined_Compiler_Errors}
\index{compiler diagnostics!user-defined}
\index{compiler errors!user-defined}
\index{compiler warnings!user-defined}
\index{compilerError}
\index{compilerWarning}

The special compiler diagnostic function calls \chpl{compilerError}
and \chpl{compilerWarning} generate compiler diagnostic of the
indicated severity if the function containing these calls may be
called when the program is executed and the function call is not
eliminated by parameter folding.

The compiler diagnostic is defined by the actual arguments which must
be string parameters.  The diagnostic points to the spot in the Chapel
program from which the function containing the call is called.
Compilation halts if a \chpl{compilerError} is encountered whereas it
will continue after encountering a \chpl{compilerWarning}.

Note that when a variable function is called in a context where the
implicit \chpl{setter} argument is true or false, both versions of the
variable function are resolved by the compiler.  Consequently,
the \chpl{setter} argument cannot be effectively used to guard a
compiler diagnostic statements.

\begin{example}
The following code shows an example of using user-defined compiler
diagnostics to generate warnings and errors:
\begin{chapelpre}
% compilerDiagnostics.chpl
\end{chapelpre}
\begin{chapel}
proc foo(x, y) {
  if (x.type != y.type) then
    compilerError("foo() called with non-matching types: ", 
                  typeToString(x.type), " != ", typeToString(y.type));
  writeln("In 2-argument foo...");
}

proc foo(x) {
  compilerWarning("1-argument version of foo called");
  writeln("In generic foo!");
}
\end{chapel}
\begin{chapelpost}
foo(3.4);
foo("hi");
foo(1, 2);
foo(1.2, 3.4);
foo("hi", "bye");
\end{chapelpost}
\begin{chapeloutput}
compilerDiagnostics.chpl:12: warning: 1-argument version of foo called
compilerDiagnostics.chpl:13: warning: 1-argument version of foo called
In generic foo!
In generic foo!
In 2-argument foo...
In 2-argument foo...
In 2-argument foo...
\end{chapeloutput}

The first routine generates a compiler error whenever the compiler
encounters a call to it where the two arguments have different types.
It prints out an error message indicating the types of the arguments.
The second routine generates a compiler warning whenver the compiler
encounters a call to it.

Thus, if the program foo.chpl contained the following calls:

\begin{numberedchapel}
foo(3.4);
foo("hi");
foo(1, 2);
foo(1.2, 3.4);
foo("hi", "bye");
foo(1, 2.3);
foo("hi", 2.3);
\end{numberedchapel}

\noindent compiling the program would generate output like:

\begin{commandline}
foo.chpl:1: warning: 1-argument version of foo called with type: real
foo.chpl:2: warning: 1-argument version of foo called with type: string
foo.chpl:6: error: foo() called with non-matching types: int != real
\end{commandline}

\end{example}

\section{Example: A Generic Stack}
\label{Example_Generic_Stack}
\begin{chapelpre}
% genericStack.chpl
\end{chapelpre}
\begin{chapel}
class MyNode {
  type itemType;              // type of item
  var item: itemType;         // item in node
  var next: MyNode(itemType); // reference to next node (same type)
}

record Stack {
  type itemType;             // type of items
  var top: MyNode(itemType); // top node on stack linked list

  proc push(item: itemType) {
    top = new MyNode(itemType, item, top);
  }

  proc pop() {
    if isEmpty then
      halt("attempt to pop an item off an empty stack");
    var oldTop = top;
    top = top.next;
    return oldTop.item;
  }

  proc isEmpty return top == nil;
}
\end{chapel}
\begin{chapelpost}
var s: Stack(int);
s.push(1);
s.push(2);
s.push(3);
while !s.isEmpty do
  writeln(s.pop());
\end{chapelpost}
\begin{chapeloutput}
3
2
1
\end{chapeloutput}

\cleardoublepage
\sekshun{Parallelism and Synchronization}
\label{Parallelism_and_Synchronization}

Chapel is an explicitly parallel programming language.  Parallelism is
introduced into a program by the user with the following three
constructs: \chpl{forall}, \chpl{cobegin}, and \chpl{begin}.  In
addition, some operations on arrays and domains, as well as
invocations of promotion, are executed in parallel.  Synchronization
is provided with \emph{synchronization variables} and \emph{atomic}
statements.  To avoid any unintended implications, the
terms \emph{computation} and \emph{sub-computation} will be used to
refer to distinct, concurrently executing portions of the program.

\subsection{The Forall Loop}
\label{Forall}
\index{forall@\chpl{forall}}
\index{forall loops}

The forall loop is a variant of the for loop that allows for the
concurrent execution of the loop body. The for loop is described
in~\rsec{The_For_Loop}. The syntax for the forall loop is given by
\begin{syntax}
forall-statement:
   `forall' loop-control-part loop-body-part
\end{syntax}

The forall loop evaluates the loop body once for each element in
the \sntx{iterator-expression}.  Each instance of the forall loop's
statement may be executed concurrently with each other, but this is
not guaranteed.  The compiler and runtime determine the actual
concurrency based on the specification of the iterator of the loop.
The keyword \chpl{ordered}, described in~\rsec{Ordered_Forall}, can be
used to constrain the parallelism to give a partial order on the
iterator.

Control continues with the statement following the forall loop only
after each iteration has been completely evaluated.  Control transfers
out of a loop body via \chpl{break}, \chpl{continue},
and \chpl{return} are not permitted.  Control can be transferred out
of the loop via a \chpl{yield} statement.

\begin{example}
In the code
\begin{chapel}
forall i in 1..N do
  a(i) = b(i);
\end{chapel}
the user has stated that the element-wise assignments can execute
concurrently.  This loop may be performed serially, with maximum
concurrency where each loop body iteration instance is executed in a
separate computation, or somewhere in between.
\end{example}

\begin{status}
The forall loop is currently executed serially.
\end{status}

\subsubsection{Alternative Forall Loop Syntax}
\label{Alternative_Forall_Loop_Syntax}
\index{forall loops!alternative syntax}

The forall loop may be alternatively specified with a more concise
syntax given by:
\begin{syntax}
alternative-forall-statement:
  [loop-control-part] statement
\end{syntax}
The semantics are unchanged.

\begin{example}
The previous \chpl{forall} example can be alternatively written as:
\begin{chapel}
[i in 1..N] a(i) = b(i);
\end{chapel}
\end{example}

\subsubsection{The Ordered Forall Loop}
\label{Ordered_Forall}
\index{forall loops!ordered}
\index{ordered@\chpl{ordered}}

By default a forall loop allows complete concurrent evaluation of the
iterator expression and among the loop instances. The
keyword \chpl{ordered} can be used to constrain the general
parallelism among instances of the loop to that expressed by an
iterator. This allows an iterator to both define an array of values
and to impose a partial order on that iterator.  This has the same
semantics as with the ordered expression which is explained
in \rsec{Ordered_Expressions}.  The syntax is:
\begin{syntax}
ordered-forall-statement:
   `ordered' `forall' loop-control-part loop-body-part
\end{syntax}

\begin{example}
In the code
\begin{chapel}
ordered forall i in walk(root) do
  work(i);

def walk(n: node) {
  yield n;
  forall c in 0..n.numOfChildren {
      yield n.child[c];
  }
}
\end{chapel}
there is a constraint on the parallel execution such that the
function \chpl{work} is evaluated on a node before any of its
immediate children nodes.  The work on sibling nodes can be executed
concurrently.
\end{example}

\begin{status}
The ordered forall loop is currently executed serially.
\end{status}

\subsection{The Forall Expression}
\label{Forall_Expressions}
\index{forall expressions}

A forall expression can be used to enable concurrent evaluation of
sub-expressions.  The sub-expressions are evaluated once for each
element in the iterator expression.  The syntax of a forall expression
is given by
\begin{syntax}
forall-expression:
   `forall' loop-control-part `do' expression
   [loop-control-part] expression
\end{syntax}

A forall expression is semantically equivalent to an iterator that
yields the expressions.

\begin{example}
The code
\begin{chapel}
[i in S] f(i);
\end{chapel}
is equivalent to
\begin{chapel}
def ff() {
  for i in S do
    yield f(i);
}
ff();
\end{chapel}
\end{example}

\begin{status}
Forall expressions are evaluated serially.
\end{status}

\subsubsection{Filtering Predicates in Forall Expressions}
\label{Filtering_Predicates_Forall}
\index{forall expressions!and conditional expressions}

An if expression that is immediately enclosed by a forall expression
does not require an else part.
\begin{example}
The following expression returns every other element starting with the
first:
\begin{chapel}
[i in 1..s.length] if i % 2 == 1 then s(i)
\end{chapel}
\end{example}

\subsection{The Cobegin Statement}
\label{Cobegin}
\index{cobegin@\chpl{cobegin}}

The cobegin statement is used to create parallelism among statements
within a block statement. The \chpl{cobegin} statement syntax is
\begin{syntax}
cobegin-statement:
  `cobegin' block-statement
\end{syntax}
Each statement within the block statement is executed concurrently and
is considered a separate computation.  Control continues after all of
the statements within the block statement have been evaluated.

As with the forall loop, control transfers are not permitted
either into or out of the cobegin's block statement. Similarly,
yield statements are allowed.

Variables declared in the cobegin statement are {\em single variables},
described in~\rsec{Single_Variables}.

\subsection{The Coforall Loop}
\label{Coforall}
\index{coforall@\chpl{coforall}}
\index{coforall loops}

The coforall loop is a variant of the cobegin statement and the forall
loop.  The syntax for the coforall loop is given by
\begin{syntax}
coforall-statement:
   `coforall' loop-control-part loop-body-part
\end{syntax}

The semantics of the \chpl{coforall} loop are identical to
the \chpl{forall} loop except that each iteration is guaranteed to run
concurrently.  It thus has potentially higher overhead than a forall
loop, but in cases where concurrency is required for correctness, it
is essential.

The semantics of the \chpl{coforall} loop are also identical to
a \chpl{cobegin} statement where each iteration of the \chpl{coforall}
loop is equivalent to a separate statement in a \chpl{cobegin} block.

Control continues with the statement following the \chpl{coforall}
loop only after each iteration has been completely evaluated.  Control
transfers out of a loop body via \chpl{break}, \chpl{continue},
and \chpl{return} are not permitted.  Control can be transferred out
of the loop via a \chpl{yield} statement.

\subsection{The Begin Statement}
\label{Begin}
\index{begin@\chpl{begin}}

The begin statement spawns a computation to execute a statement.
Control continues simultaneously with the statement following the
begin statement. The begin statement is an unstructured way to create
a new computation that is executed only for its side-effects. The
syntax for the begin statement is given by
\begin{syntax}
begin-statement:
  `begin' statement
\end{syntax}

The following statements cannot be contained in begin-statements:
break-statements, continue-statements, yield-statements, and
return-statements.

\subsection{The Ordered Expression}
\label{Ordered_Expressions}
\index{ordered@\chpl{ordered}}

\begin{status}
The ordered expression is not yet implemented.
\end{status}

The \chpl{ordered} keyword can be used as an unary operator to suppress
parallel execution among instances of an expression that can involve
side-effects to memory.  The \chpl{ordered} keyword does not inhibit
parallelism within the sub-expression.  The syntax is:
\begin{syntax}
ordered-expression:
   `ordered' expression
\end{syntax}

\begin{example}
In the code
\begin{chapel}
ordered [i in S] f(i) 
\end{chapel}
\chpl{f} is a function and \chpl{S} is an iterator expression. Each
instance of \chpl{f(i)} is executed once for each value in \chpl{S}
and in serial order. The \chpl{ordered} constraint does not propagate
to inhibit parallelism within \chpl{f}.
\end{example}

\subsection{The Serial Statement}
\label{Serial}
\index{serial@\chpl{serial}}

The \chpl{serial} statement can be used to dynamically control the
degree of parallelism.  The syntax is:
\begin{syntax}
serial-statement:
  `serial' expression block-level-statement
\end{syntax}
where the expression evaluates to a bool type.  Independent of that
value, the \sntx{block-level-statement} is evaluated. If the
expression is true, any dynamically encountered forall loop or cobegin
statement is executed serially within the current computation.  Any
dynamically encountered begin-statement is executed serially with the
current computation; no new computation is spawned.  Control continues
to the statement following the begin-statement after the
begin-statement finishes.

\begin{example}
In the code
\begin{chapel}
ordered forall i in walk(root) do
  work(i);

def walk(n: node) {
  yield n;
  serial n.depth > 4 forall c in 0..n.numOfChildren {
      yield n.child[c];
  }
}
\end{chapel}
the serial statement inhibits concurrent execution on the tree for
nodes that are deeper than four levels in the tree.
\end{example}

There is an expectation that functions that may be executed in a
serial context are cloned to avoid the overhead of testing and
suppressing parallelism.

\subsection{Synchronization Variables}
\label{Synchronization_Variables}
\index{synchronization variables}

{\em Synchronization variables} are used to coordinate computations
that share data.  The use of and assignment to these variables
implicitly controls the execution order of the computation.  There are
two kinds of synchronization variables, {\em single} and {\em sync}
variables.  A single variable can only be assigned once during its
lifetime.  A sync variable can be assigned multiple times during its
lifetime.

The normal use of and assignment to a synchronization variable is well
suited for producer-consumer data sharing.  Additional functions on
synchronization variable are provided such that other traditional
synchronization primitives, such as semaphores and mutexes, can be
constructed.

\subsubsection{Single Variables}
\label{Single_Variables}
\index{synchronization variables!single@\chpl{single}}
\index{single@\chpl{single}}

A single (assignment) variable can only be assigned once during its
lifetime.  A use of a single variable before it is assigned causes the
computation's execution to be suspended until the variable is
assigned. Otherwise, the use proceeds as with normal variables and the
computation continues.  After a single assignment variable is assigned,
all computations with pending uses resume in an unspecified order.  A
single variable is specified with a single type given by the following
syntax:
\begin{syntax}
 single-type:
   `single' type
\end{syntax}

\begin{example}
In the code
\begin{chapel}
class Tree {
  var is_leaf : bool;
  var left    : Tree;
  var right   : Tree;
  var value   : int;

  def sum() {
    if (is_leaf) then 
       return value;

    var x : single int;
    begin x = left.sum();
    var y = right.sum();
    return x+y;
  }
}
\end{chapel}
the single variable \chpl{x} is assigned by an asynchronous
computation created with the begin statement. The computation
returning the sum waits on the reading of \chpl{x} until it has been
assigned.

While a \chpl{cobegin} might be a more suitable formulation, this
fragment creates an asynchronous computation to compute the sum of the
left sub-tree while the main computation continues with the right
sub-tree. The final reference to variable x will be delayed until the
assignment to x completes and that value will be used as a summand.
\end{example}

When a single variable has an initializer, the evaluation of
that initializer is implicitly performed as an asynchronous computation. 
\begin{example}
The code
\begin{chapel}
var x: single int = left.sum;
\end{chapel}
is equivalent to
\begin{chapel}
var x: single int;
x = left.sum;
\end{chapel}
\end{example}

\index{synchronization variables!implicit in cobegin@implicit in \chpl{cobegin}}
Any variable declaration within a cobegin statement is implicitly
treated as a single variable for references in other statements of the
cobegin statement.
\begin{example}
In the code
\begin{chapel}
def sum() {
  if (is_leaf) then 
    return value;
  var z;
  cobegin {
    var x = left.sum();
    var y = right.sum();
    z = x+y;
  }
  return z;
}
\end{chapel}
the computation with assignment to \chpl{z} waits for the other
computations to assign to \chpl{x} and \chpl{y} before it
references \chpl{x} and \chpl{y} in order to assign to \chpl{z}.  The
variables \chpl{x} and \chpl{y} are implicitly single.
\end{example}

\subsubsection{Sync Variables}
\label{Sync_Variables}
\index{synchronization variables!sync@\chpl{sync}}
\index{sync@\chpl{sync}}

A sync variable generalizes the single assignment variable to permit
multiple assignments to the variable. A sync variable is logically
either {\em full} or {\em empty}. When it is empty, computations that
attempt to read that variable are suspended until the variable becomes
full by the next assignment to it, which atomically changes the state
to full. When the variable is full, a read of that variable consumes
the value and atomically transitions the state to empty. If there is
more than one computation waiting on a sync variable, one is
non-deterministically selected to use the variable and resume
execution.  The other computations continue to wait for the next
assignment.

If a computation attempts to assign to a sync variable that is full,
the computation is suspended and the assignment is delayed. When the
sync variable becomes empty, the computation is resumed and the
assignment proceeds, transitioning the state back to full. If there
are multiple computations attempting such an assignment, one is
non-deterministically selected to proceed and the other assignments
continue to wait until the sync variable is emptied again.

A sync variable is specified with a sync type given by the following
syntax:
\begin{syntax}
sync-type:
  `sync'
\end{syntax}

\subsubsection{Additional Synchronization Variable Functions}
\label{Functions_on_Synchronization_Variables}
\index{synchronization variables!built-in functions on}
\index{readFE@\chpl{readFE}}
\index{readFF@\chpl{readFF}}
\index{readXX@\chpl{readXX}}
\index{writeEF@\chpl{writeEF}}
\index{writeFF@\chpl{writeFF}}
\index{writeXF@\chpl{writeXF}}
\index{reset@\chpl{reset}}
\index{isFull@\chpl{isFull}}

Synchronization variables support additional methods that
can be used to bypass their semantics to provide new ones.
Let \chpl{sync_class} be a (compiler-generated) class
for a sync variable \chpl{s} of arbitrary type \chpl{t}.
The following methods are defined in this class:

\begin{protohead}
def $sync_class$.readFE(): t
\end{protohead}
\begin{protobody}
Wait for full, leave empty, and return \chpl{s}'s value.
\end{protobody}

\begin{protohead}
def $sync_class$.readFF(): t
\end{protohead}
\begin{protobody}
Wait for full, leave full, and return \chpl{s}'s value.
\end{protobody}

\begin{protohead}
def $sync_class$.readXX(): t
\end{protohead}
\begin{protobody}
No wait, leave F/E unchanged, and return \chpl{s}'s value.
\end{protobody}

\begin{protohead}
def $sync_class$.writeEF(v: t)
\end{protohead}
\begin{protobody}
Wait for empty, assign \chpl{s=v}, and leave full.
\end{protobody}

\begin{protohead}
def $sync_class$.writeFF(v: t)
\end{protohead}
\begin{protobody}
Wait for full, assign \chpl{s=v}, and leave full.
\end{protobody}

\begin{protohead}
def $sync_class$.writeXF(v: t)
\end{protohead}
\begin{protobody}
No wait, assign \chpl{s=v}, and leave full.
\end{protobody}

\begin{protohead}
def $sync_class$.reset()
\end{protohead}
\begin{protobody}
No wait, assign to \chpl{s} \chpl{t}'s default value, and leave empty.
\end{protobody}

\begin{protohead}
def $sync_class$.isFull: bool
\end{protohead}
\begin{protobody}
No wait, returns \chpl{true} if \chpl{s} is full, \chpl{false} otherwise.
\end{protobody}

For single variables \chpl{s},
only \chpl{readFF} and \chpl{writeEF} are defined.

\begin{example}
Given the following declarations
\begin{chapel}
var x: single int;
var y: int;
\end{chapel}
the code
\begin{chapel}
x = 5;
y = x;
\end{chapel}
is equivalent to
\begin{chapel}
x.writeEF(5);
y = x.readFF();
\end{chapel}
\end{example}

\begin{rationale}
Although the \chpl{readFE}, \chpl{readFF}, and \chpl{writeEF} methods
are implicitly called
when referencing or assigning to sync or single variables,
making these methods available supports programmers who wish to
make the semantics of these operations more explicit.
It might be desirable to have a compiler option
that disables the implicit use of these methods.
\end{rationale}

\subsubsection{Synchronization Variables of Record and Class Types}
\label{Synchronization_Variables_of_Record_Type}
\index{synchronization variables!of record type}
\index{synchronization variables!of class type}

A variable of record or class type can be a single or sync
variable. The semantics of single and sync variables are applied only
to the variable and not to accesses of individual fields.  A record or
class type may have synchronization variable fields to get
synchronization semantics on individual field accesses.

\subsubsection{Synchronization Formal Arguments}
\label{Synchronization_Formal_Arguments}
\index{synchronization variables!formal arguments}

If an argument is a sync or single type, the actual is passed by
reference and the argument itself is a valid lvalue.

The unqualified types \chpl{sync} and \chpl{single} can also be used
to specify a generic formal argument.  In this case, the actual must
be a synchronization variable and it is passed by reference.  For
generic formal arguments of any type, an actual that is \chpl{sync}
or \chpl{single} is ``read'' before being passed to the function and
the generic formal argument is not a \chpl{sync} or \chpl{single}
type.

\subsection{Memory Consistency Model}
\label{Memory_Consistency}
\index{memory consistency model}

This section is forthcoming.

\subsection{Atomic Statement}
\label{Atomic_Transactions}
\index{atomic transactions}
\index{atomic@\chpl{atomic}}

\begin{status}
Atomic statements are not yet implemented.
\end{status}

The atomic statement creates an atomic transaction of a statement. The
statement is executed with transaction semantics in that the statement
executes entirely, the statement appears to have completed in a single
order and serially with respect to other atomic statements, and no
variable assignment is visible until the statement has completely
executed.

This definition of an atomic statement is sometimes called {\em strong
atomicity} because the semantics are atomic to the entire program.
{\em Weak atomicity} is defined so that an atomic statement is atomic
only with respect to other atomic statements.  If the performance
implications of strong atomicity are not tolerable, the semantics of
atomic transactions may be revisited, and could become weaker.

The syntax for the atomic statement is given by:
\begin{syntax}
atomic-statement:
  `atomic' statement
\end{syntax}

\begin{example}
The following code illustrates one possible use of atomic statements:
\begin{chapel}
var found = false;
atomic {
  if head == obj {
    found = true;
    head = obj.next;
  } else  {
    var last = head;
    while last != null {
    if last.next == obj {
      found = true;
      last.next = object.next;
      break;
    }
    last = last.next;
  }
}
\end{chapel}
Inside the atomic statement is a sequential implementation of
removing a particular object denoted by \chpl{obj} from a singly
linked list.  This is an operation that is well-defined, assuming only
one computation is attempting it at a time. The atomic statement
ensures that, for example, the value of \chpl{head} does not change
after it is first in the first comparison and subsequently read to
initialize \chpl{last}. The variables eventually owned by this
computation are \chpl{found}, \chpl{head}, \chpl{obj}, and the various
\chpl{next} fields on examined objects.
\end{example}

The effect of an atomic statement is dynamic.

\begin{example}
If there is a method associated with a list that removes an object,
that method may not be parallel safe, but could be invoked safely inside an
atomic statement:
\begin{chapel}
atomic found = head.remove(obj);
\end{chapel}
\end{example}

\cleardoublepage
\sekshun{Locality and Distribution}
\label{Locality_and_Distribution}

\begin{implementation}
Programs can currently only run on a single locale.  The abstractions
described here are not yet implemented.
\end{implementation}

Chapel provides high-level abstractions that allow programmers to
exploit locality by defining the affinity of data and computation.
This is accomplished by associating both data objects and computations
with abstract {\em locales}. To provide a higher-level mechanism,
Chapel allows a mapping from domains to locales to be specified. This
mapping is called a {\em distribution} and it guides that placement of
variables associated with arrays and the placement of subcomputations
defined over the domain.

\index{local}
\index{remote}
Throughout this section, the term {\em local} refers to data that is
associated with the locale that a computation is running on and {\em
remote} refers to data that is not. We assume that there is some
overhead associated with accessing data that may be remote compared to
data known to be local.

\subsection{Locales}
\label{Locales}
\index{locales}

A locale abstracts a processor or node in a parallel computer system,
or the basic component in the computer system where local memory can
be accessed uniformly.

\subsubsection{The Locale Type}
\label{The_Locale_Type}
\index{locale@\chpl{locale}}

The identifier \chpl{locale} is a primitive type that abstracts a
locale as described above.  Both data and computations can be
associated with a value of locale type. The only operators defined
over locales are the equality and inequality comparison operators.

\subsubsection{Predefined Locales Array}
\label{Predefined_Locales_Array}
\index{Global@\chpl{Global}}
\index{Locales@\chpl{Locales}}
\index{numLocales@\chpl{numLocales}}
\index{execution environment}

A predefined configuration variable defines the {\em execution
environment} for a program.  This environment is defined by the
following definitions:
\begin{chapel}
config const numLocales: int;
const Locales: [1..numLocales] locale;
const Global: locale;
\end{chapel}
The environment consists of constants which are fixed when the program
begins execution.  The variable \chpl{Global} holds a special value
of \chpl{locale} type that can be distinct from the values stored
in \chpl{Locales}. This value is used to denote an object or
computation that has no defined affinity.

When a program starts, a single thread is executing.  It is running on
the locale given by \chpl{Locales(1)}.

\subsubsection{Querying the Locale of a Variable}
\label{Querying_the_Locale_of_a_Variable}
\index{locale@\chpl{locale}}

Every variable \chpl{v} is associated with some locale which can be
queried using the following syntax:
\begin{syntax}
locale-access:
  expression . `locale'
\end{syntax}
When the \sntx{expression} is a class type, the locale is where the
object is located rather than where the \sntx{expression} may be
located.

\subsection{Specifying Locales for Computation}
\label{Specifying_Locales_for_Computation}

When execution is proceeding on some locale, a computation can be
associated with a different locale in two ways: via distributions as
discussed in~\rsec{Distributions} or with an \sntx{on-statement} as
discussed below.

\subsubsection{On}
\label{On}
\index{on@\chpl{on}}

The on statement controls on which locale a computation or data should
be placed.  The syntax of the on statement is given by
\begin{syntax}
on-statement:
  `on' expression `do' statement
  `on' expression block-level-statement
\end{syntax}
If the \sntx{expression} is a value of \chpl{locale} type,
the \sntx{statement} or \sntx{block-level-statement} is executed on
the locale specified directly by the expression.  Otherwise, the
expression must be a variable and the locale is taken to be the locale
where the variable is located.  Execution continues after
the \chpl{on-statement} after execution of the \sntx{statement}
or \sntx{block-level-statement} completes.

\index{Global@\chpl{Global}}
If the locale that the \sntx{expression} refers to is equal
to \chpl{Global}, then the locale is unspecified and is determined by
the runtime and/or compiler.

\begin{example}
A common idiom is to use \chpl{on} in conjunction with \chpl{forall}
to access an array decomposed over multiple locales.  The code
\begin{chapel}
forall i in D do on A(i) {
  // some computation
}
\end{chapel}
executes each iteration of the forall loop on the locale where the
element of \chpl{A(i)} is located.
\end{example}

By default, when new variables and data objects are created, they are
created in the locale where the computation is running.  This locale
can be changed by using the \chpl{on} keyword.  Variables can be
defined within an \sntx{on-statement} to define them on a particular
locale.

\subsubsection{On and Iterators}
\label{On_and_Iterators}
\index{iterators!on@\chpl{on}}

When a loop iterates over a sequence specified by an iterator,
on-statements inside the iterator control where the corresponding loop
body is executed.

\begin{example}
An iterator over a distributed tree might include an iterator over the
nodes as defined in the following code:
\begin{chapel}
class Tree {
  var left, right: Tree;
  iterator nodes {
    on this yield this;
    if left then
      forall t in left.nodes do
        yield t;
    if right then
      forall t in right.nodes do
        yield t;
  }
}
\end{chapel}
Given this code and a binary tree of type \chpl{Tree} stored in
variable \chpl{tree}, then we can use the nodes iterator to iterate
over the tree with the following code:
\begin{chapel}
forall t in tree.nodes {
  // body executed on t as specified in nodes
}
\end{chapel}
Here, each instance of the body of the \chpl{forall} loop is executed
on the locale where the corresponding object \chpl{t} is located.
This is specified in the \chpl{nodes} iterator where the \chpl{on}
keyword is used.  In the case of zipper or tensor product iteration,
the location of execution is taken from the first iterator.  This can
be overridden by explictly using \chpl{on} in the body of the loop or
by reordering the product of iteration.
\end{example}

\subsection{Distributions}
\label{Distributions}
\index{distributions}

A mapping from domain index values to locales is called a {\em
distribution}.

\subsubsection{Distributed Domains}
\label{Distributed_Domains}

\index{domains!distributed}
A domain for which a distribution is specified is referred to as a
{\em distributed domain}.  A domain supports a method, \chpl{locale},
that maps index values in the domain to locales that correspond to the
domain's distribution.

Iteration over a distributed domain implicitly executes the control
computation in the domain of the associated locale.  Similarly, when
iterating over the elements of an array defined over a distributed
domain, the controlled computations are determined by the distribution
of the domain.  If there are conflicting distributions in product
iterations, the locale of the computation is taken to be the first
component in the product.

\begin{example}
If \chpl{D} is a distributed domain, then in the code
\begin{chapel}
forall d in D {
  // body
}
\end{chapel}
the body of the loop is executed in the locale where the
index \chpl{d} maps to by the distribution of \chpl{D}.
\end{example}

\subsubsection{Distributed Arrays}
\label{Distributed_Arrays}
\index{arrays!distributed}

Arrays defined over a distributed domain will have the element
variables stored on the locale determined by the distribution.  Thus,
if \chpl{d} is an index of distributed domain \chpl{D} and \chpl{A} is
an array defined over that domain, then \chpl{A(d).locale} is the
locale to which \chpl{d} maps to according to \chpl{D}.

\subsubsection{Undistributed Domains and Arrays}
\label{Undistributed_Domains_and_Arrays}

If a domain or an array does not have a distributed part, the domain
or array is not distributed and exists only on the locale on which it
is defined.

\subsection{Standard Distributions}
\label{Standard_Distributions}

Standard distributions include the following:
\begin{itemize}
\item The block distribution \chpl{Block}
\item The cyclic distribution \chpl{Cyclic}
\item The block-cyclic distribution \chpl{BlockCyclic}
\item The cut distribution \chpl{Cut}
\end{itemize}

A design goal is that all standard distributions are defined with the
same mechanisms that user-defined
distributions~(\rsec{User_Defined_Distributions}) are defined with.

\subsection{User-Defined Distributions}
\label{User_Defined_Distributions}

This section is forthcoming.

\cleardoublepage
This is a stub.  This portion of the document does not exist.

\cleardoublepage
\sekshun{Input and Output}
\label{Input_and_Output}

Chapel provides a built-in \chpl{file} class to handle input and
output to files using functions and methods
called \chpl{read}, \chpl{readln},
\chpl{write}, and \chpl{writeln}.

\section{The {\em file} type}
\index{file type}

The file class contains the following fields:
\begin{itemize}
\item
The \chpl{filename} field is a \chpl{string} that contains the name of
the file.
\item
The \chpl{mode} field is a \chpl{FileAccessMode} enum value that indicates
whether the file is being read or written.
\item
The \chpl{path} field is a \chpl{string} that contains the path of the
file.
\item
The \chpl{style} field can be set to \chpl{text} or \chpl{binary} to
specify that reading from or writing to the file should be done with
text or binary formats.
\end{itemize}
These fields can be modified any time that the file is closed.

The \chpl{mode} field supports the following \chpl{FileAccessMode} values:
\begin{itemize}
\item
\chpl{FileAccessMode.read} The file can be read.
\item
\chpl{FileAccessMode.write} The file can be written.
\end{itemize}

The file type supports the following methods:
\index{file type!methods}
\begin{itemize}
\item
The \chpl{open()} method opens the file for reading and/or writing.
\item
The \chpl{close()} method closes the file for reading and/or writing.
\item
The \chpl{isOpen} method returns true if the file is open for reading
and/or writing, and otherwise returns false.
\item
The \chpl{eof} method returns true if the file is at its end-of-file
position and returns false otherwise.
\item
The \chpl{flush()} method flushes the file, finishing outstanding
reading and writing.
\end{itemize}

Additionally, the file type supports the
methods \chpl{read}, \chpl{readln}, \chpl{write}, and \chpl{writeln} for 
input and output as discussed in~\rsec{filewrite} and~\rsec{fileread}.

\section{Standard files {\em stdout}, {\em stdin}, and {\em stderr}}
\index{file type!standard files stdin, stdout, stderr}
\index{stdin@\chpl{stdin}}
\index{stdout@\chpl{stdout}}
\index{stderr@\chpl{stderr}}

The files \chpl{stdout}, \chpl{stdin}, and \chpl{stderr} are
predefined and map to standard output, standard input, and standard
error as implemented in a platform dependent fashion.

\section{The {\em write}, {\em writeln}, {\em read}, and {\em readln} 
functions}
\index{writeln@\chpl{writeln}}
\index{write@\chpl{write}}
\index{read@\chpl{read}}
\index{readln@\chpl{readln}}
\index{read}
\index{write}

The built-in function \chpl{write} takes an arbitrary number of
arguments and prints each out in turn to \chpl{stdout}.  The built-in
function \chpl{writeln} is identical to \chpl{write} except that it
outputs an additional {\em end-of-line} character after writing out
the argument expressions.  Both of these functions will generate their
output atomically with respect to other calls to these functions from
other tasks.

The built-in function \chpl{read} takes an arbitrary number of
variable expressions and reads into each in turn from \chpl{stdin}.
Any whitespace is skipped over and is used only to separate one
argument from the next.  The built-in function \chpl{readln} is
identical except that upon reading all of its arguments it scans ahead
in the input stream until just after the next {\em end-of-line}
character.

The \chpl{read} and \chpl{readln} functions are also defined to take
an arbitrary number of types as arguments.  In this case, the
functions read an expression of each argument type.  In the event that
a single type is specified, the return value is the value that was
read; if multiple types are specified, a tuple of the values is
returned.

These functions are wrappers for the methods on files described next.

\begin{example}
The \chpl{writeln} wrapper function allows for a simple implementation
of the {\em Hello-World} program:
\begin{chapel}
writeln("Hello, World!");
\end{chapel}
\end{example}

\begin{example}
The following code shows three ways to read values into a pair of
variables \chpl{x} and \chpl{y}:
\begin{chapel}
var x: int;
var y: real;

/* reading into variable expressions */
read(x, y);

/* reading via a single type argument */
x = read(int);
y = read(real);

/* reading via multiple type arguments */
(x, y) = read(int, real);
\end{chapel}
\end{example}

\section{User-Defined {\em writeThis} methods}

To define the output for a given type, the user must define a method
called \chpl{writeThis} on that type that takes a single argument of
\chpl{Writer} type.  If such a method does not exist, a default method is
created.

\section{The {\em write} and {\em writeln} method on files}
\label{filewrite}
\index{write!on files}

The \chpl{file} type supports methods \chpl{write} and \chpl{writeln}
for output.  These methods are defined to take an arbitrary number of
arguments.  Each argument is written in turn by calling
the \chpl{writeThis} method on that argument.
Default \chpl{writeThis} methods are bound to any type that the user
does not explicitly create one for.

A lock is used to ensure that output is serialized across multiple
tasks.

\subsection{The {\em write} and {\em writeln} method on strings}
\label{stringwrite}
\index{write!on strings}

The \chpl{write} and \chpl{writeln} methods can also be called on
strings to write the output to a string instead of a file.

\subsection{Generalized {\em write} and {\em writeln}}
\label{writer}
\index{Writer@\chpl{Writer}}

The \chpl{Writer} class contains no arguments and serves as a base
class to allow user-defined classes to be written to.  If a class is
defined to be a subclass of Writer, it must override
the \chpl{writeIt} method that takes a \chpl{string} as an argument.

\begin{example}
The following code defines a subclass of \chpl{Writer} that overrides
the \chpl{writeIt} method to allow it to be written to.  It also
overrides the \chpl{writeThis} method to override the default way that
it is written.
\begin{chapel}
class C: Writer {
  var data: string;
  proc writeIt(s: string) {
    data += s.substring(1);
  }
  proc writeThis(x: Writer) {
    x.write(data);
  }
}

var c = new C();
c.write(41, 32, 23, 14);
writeln(c);
\end{chapel}
The \chpl{C} class filters the arguments sent to it, printing out only
the first letter.  The output to the above is thus \chpl{4321}.
\end{example}


\section{The {\em read} and {\em readln} methods on files}
\label{fileread}
\index{read!on files}

The \chpl{file} type supports \chpl{read} and \chpl{readln} methods.
The \chpl{read} method takes an arbitrary number of arguments, reading
in each argument from file.  The \chpl{readln} method also
takes an arbitrary number of arguments, reading in each argument
from a single line or multiple lines in the file and 
advancing the file pointer to the next line after the last argument 
is read.

The \chpl{file} type also supports overloaded methods \chpl{read}
and \chpl{readln} that take an arbitrary number of types as arguments.
These methods read values of the specified types from the file and
return them in a tuple.  If only one type is read, the value is not
returned in a tuple, but is returned directly.

\begin{example}
The following line of code reads a value of type \chpl{int} from
\chpl{stdin} and uses it to initialize variable \chpl{x} (causing
\chpl{x} to have an inferred type of \chpl{int}):
\begin{chapel}
var x = stdin.read(int);
\end{chapel}
\end{example}


\section{Default {\em read} and {\em write} methods}
\index{write!default methods}
\index{read!default methods}

Default \chpl{write} methods are created for all types for which a user
\chpl{write} method is not defined.  They have the following semantics:
\begin{itemize}
\item
{\bf arrays} Outputs the elements of the array in row-major order
where rows are separated by line-feeds and blank lines are used to
separate other dimensions.
\item
{\bf domains} Outputs the dimensions of the domain enclosed
by \chpl{[} and \chpl{]}.
\item
{\bf ranges} Outputs the lower bound of the range followed
by \chpl{..} followed by the upper bound of the range.  If the stride
of the range is not one, the output is additionally followed by the
word \chpl{by} followed by the stride of the range.
\item
{\bf tuples} Outputs the components of the tuple in order delimited
by \chpl{(} and \chpl{)}, and separated by commas.
\item
{\bf classes} Outputs the values within the fields of the class
prefixed by the name of the field and the character \chpl{=}.  Each
field is separated by a comma.  The output is delimited by \chpl{\{}
and \chpl{\}}.
\item
{\bf records} Outputs the values within the fields of the class
prefixed by the name of the field and the character \chpl{=}.  Each
field is separated by a comma.  The output is delimited by \chpl{(}
and \chpl{)}.
\end{itemize}

Default \chpl{read} methods are created for all types for which a user
\chpl{read} method is not defined.  The default \chpl{read} methods are
defined to read in the output of the default \chpl{write} method.

\cleardoublepage
\sekshun{Standard Modules}
\label{Standard_Modules}
\index{standard modules}

This section describes modules that are automatically used by every
Chapel program as well as a set of standard modules that can be used
manually, provididing standard library support.  The automatic modules
are as follows:

\begin{tabular}{lll}
\hspace{1pc} & \chpl{Math} & Math routines \\
             & \chpl{Standard} & Basic routines \\
             & \chpl{Types} & Routines related to primitive types \\
\end{tabular}

\noindent The standard modules include:

\begin{tabular}{lll}
\hspace{1pc} & \chpl{BitOps} & Bit manipulation routines \\
             & \chpl{Functions}   & Common higher-order functions \\
             & \chpl{Norm}   & Routines for computing vector and matrix norms \\
             & \chpl{Random} & Random number generation routines \\
             & \chpl{Search} & Generic searching routines \\
             & \chpl{Sort} & Generic sorting routines \\
             & \chpl{Time} & Types and routines related to time \\
\end{tabular}

There is an expectation that each of these modules will be extended
and that more standard modules will be defined over time.

\subsection{Automatic Modules}
\index{automatic modules}

Automatic modules are used by a Chapel program automatically.  There
is currently no way to avoid their use by a program though we
anticipate adding such a capability in the future.

\subsubsection{Functions}
\label{Functions}
\index{standard modules!Functions}

The module \chpl{Functions} defines functions (mostly higher-order functions) that 
commonly occur in functional programming.

\vspace{1pc}

\begin{protohead}
array(x...?n)
\end{protohead}
\begin{protobody}
A helper function for creating single dimension arrays by using the given values as the initial values.
\end{protobody}

\begin{protohead}
drop(arr, n)
\end{protohead}
\begin{protobody}
An iterator which returns elements of a sequence, having dropped the first n
\end{protobody}

\begin{protohead}
dropWhile(arr, op)
\end{protohead}
\begin{protobody}
An iterator which returns elements of a sequence, having dropped the first elements which meet the given conditional operator
\end{protobody}

\begin{protohead}
filter(arr, op)
\end{protohead}
\begin{protobody}
An iterator which returns only elements in the sequence which meet the given conditional operator
\end{protobody}

\begin{protohead}
foldLeft(arr, init, op)
\end{protohead}
\begin{protobody}
Reduces a sequence with an operator pair-wise from the left starting with an initial value.
\end{protobody}

\begin{protohead}
foldRight(arr, init, op)
\end{protohead}
\begin{protobody}
Reduces a sequence with an operator pair-wise from the right starting with an initial value.
\end{protobody}

\begin{protohead}
map(arr, op)
\end{protohead}
\begin{protobody}
An iterator which returns the result of an operator applied to each element of a sequence
\end{protobody}

\begin{protohead}
reverse(arr)
\end{protohead}
\begin{protobody}
An iterator which reverses the sequence given.
\end{protobody}

\begin{protohead}
splitAt(arr, n)
\end{protohead}
\begin{protobody}
Returns a tuple splitting a sequence at the position n.
\end{protobody}

\begin{protohead}
take(arr, n)
\end{protohead}
\begin{protobody}
An iterator which returns the first n elements of a sequence
\end{protobody}

\begin{protohead}
takeWhile(arr, op)
\end{protohead}
\begin{protobody}
An iterator which returns the first elements of a sequence which meet the given conditional operator
\end{protobody}

\subsubsection{Math}
\label{Math}
\index{automatic modules!Math}

The module \chpl{Math} defines routines for mathematical computations.
This module is used by default; there is no need to explicitly use
this module.  The Math module defines routines that are derived from
and implemented via the standard C routines defined in \chpl{math.h}.

\vspace{1pc}

\begin{protohead}
def abs(i: int(?w)): int(w)
def abs(i: uint(?w)): uint(w)
def abs(x: real): real
def abs(x: real(32)): real(32)
def abs(x: complex): real
\end{protohead}
\begin{protobody}
Returns the absolute value of the argument.
\end{protobody}

\begin{protohead}
def acos(x: real): real
def acos(x: real(32)): real(32)
\end{protohead}
\begin{protobody}
Returns the arc cosine of the argument.  It is an error if \chpl{x} is
less than $-1$ or greater than $1$.
\end{protobody}

\begin{protohead}
def acosh(x: real): real
def acosh(x: real(32)): real(32)
\end{protohead}
\begin{protobody}
Returns the inverse hyperbolic cosine of the argument.  It is an error
if \chpl{x} is less than $1$.
\end{protobody}

\begin{protohead}
def asin(x: real): real
def asin(x: real(32)): real(32)
\end{protohead}
\begin{protobody}
Returns the arc sine of the argument.  It is an error if \chpl{x} is
less than $-1$ or greater than $1$.
\end{protobody}

\begin{protohead}
def asinh(x: real): real
def asinh(x: real(32)): real(32)
\end{protohead}
\begin{protobody}
Returns the inverse hyperbolic sine of the argument.
\end{protobody}

\begin{protohead}
def atan(x: real): real
def atan(x: real(32)): real(32)
\end{protohead}
\begin{protobody}
Returns the arc tangent of the argument.
\end{protobody}

\begin{protohead}
def atan2(y: real, x: real): real
def atan2(y: real(32), x: real(32)): real(32)
\end{protohead}
\begin{protobody}
Returns the arc tangent of the two arguments.  This is equivalent to
the arc tangent of \chpl{y / x} except that the signs of \chpl{y}
and \chpl{x} are used to determine the quadrant of the result.
\end{protobody}

\begin{protohead}
def atanh(x: real): real
def atanh(x: real(32)): real(32)
\end{protohead}
\begin{protobody}
Returns the inverse hyperbolic tangent of the argument.  It is an error
if \chpl{x} is less than $-1$ or greater than $1$.
\end{protobody}

\begin{protohead}
def cbrt(x: real): real
def cbrt(x: real(32)): real(32)
\end{protohead}
\begin{protobody}
Returns the cube root of the argument.
\end{protobody}

\begin{protohead}
def ceil(x: real): real
def ceil(x: real(32)): real(32)
\end{protohead}
\begin{protobody}
Returns the value of the argument rounded up to the nearest integer.
\end{protobody}

\begin{protohead}
def conjg(a: complex(?w)): complex(w)
\end{protohead}
\begin{protobody}
Returns the conjugate of \chpl{a}.
\end{protobody}

\begin{protohead}
def cos(x: real): real
def cos(x: real(32)): real(32)
\end{protohead}
\begin{protobody}
Returns the cosine of the argument.
\end{protobody}

\begin{protohead}
def cosh(x: real): real
def cosh(x: real(32)): real(32)
\end{protohead}
\begin{protobody}
Returns the hyperbolic cosine of the argument.
\end{protobody}

\begin{protohead}
def erf(x: real): real
def erf(x: real(32)): real(32)
\end{protohead}
\begin{protobody}
Returns the error function of the argument defined as
$$\frac{2}{\sqrt{\pi}}\int^x_0e^{-t^2}dt$$
for the argument $x$.
\end{protobody}

\begin{protohead}
def erfc(x: real): real
def erfc(x: real(32)): real(32)
\end{protohead}
\begin{protobody}
Returns the complementary error function of the argument.  This is
equivalent to \chpl{1.0 - erf(x)}.
\end{protobody}

\begin{protohead}
def exp(x: real): real
def exp(x: real(32)): real(32)
\end{protohead}
\begin{protobody}
Returns the value of $e$ raised to the power of the argument.
\end{protobody}

\begin{protohead}
def exp2(x: real): real
def exp2(x: real(32)): real(32)
\end{protohead}
\begin{protobody}
Returns the value of $2$ raised to the power of the argument.
\end{protobody}

\begin{protohead}
def expm1(x: real): real
def expm1(x: real(32)): real(32)
\end{protohead}
\begin{protobody}
Returns one less than the value of $e$ raised to the power of the argument.
\end{protobody}

\begin{protohead}
def floor(x: real): real
def floor(x: real(32)): real(32)
\end{protohead}
\begin{protobody}
Returns the value of the argument rounded down to the nearest integer.
\end{protobody}

\begin{protohead}
def lgamma(x: real): real
def lgamma(x: real(32)): real(32)
\end{protohead}
\begin{protobody}
Returns the natural logarithm of the absolute value of the gamma
function of the argument.
\end{protobody}

\begin{protohead}
def log(x: real): real
def log(x: real(32)): real(32)
\end{protohead}
\begin{protobody}
Returns the natural logarithm of the argument.  It is an error if the
argument is less than or equal to zero.
\end{protobody}

\begin{protohead}
def log10(x: real): real
def log10(x: real(32)): real(32)
\end{protohead}
\begin{protobody}
Returns the base 10 logarithm of the argument.  It is an error if the
argument is less than or equal to zero.
\end{protobody}

\begin{protohead}
def log1p(x: real): real
def log1p(x: real(32)): real(32)
\end{protohead}
\begin{protobody}
Returns the natural logarithm of \chpl{x+1}.  It is an error
if \chpl{x} is less than or equal to $-1$.
\end{protobody}

\begin{protohead}
def log2(i: int(?w)): int(w)
def log2(i: uint(?w)): uint(w)
def log2(x: real): real
def log2(x: real(32)): real(32)
\end{protohead}
\begin{protobody}
Returns the base 2 logarithm of the argument.  It is an error if the
argument is less than or equal to zero.
\end{protobody}

\begin{protohead}
def nearbyint(x: real): real
def nearbyint(x: real(32)): real(32)
\end{protohead}
\begin{protobody}
Returns the rounded integral value of the argument determined by the
current rounding direction.
\end{protobody}

\begin{protohead}
def rint(x: real): real
def rint(x: real(32)): real(32)
\end{protohead}
\begin{protobody}
Returns the rounded integral value of the argument determined by the
current rounding direction.
\end{protobody}

\begin{protohead}
def round(x: real): real
def round(x: real(32)): real(32)
\end{protohead}
\begin{protobody}
Returns the rounded integral value of the argument.  Cases halfway
between two integral values are rounded towards zero.
\end{protobody}

\begin{protohead}
def sin(x: real): real
def sin(x: real(32)): real(32)
\end{protohead}
\begin{protobody}
Returns the sine of the argument.
\end{protobody}

\begin{protohead}
def sinh(x: real): real
def sinh(x: real(32)): real(32)
\end{protohead}
\begin{protobody}
Returns the hyperbolic sine of the argument.
\end{protobody}

\begin{protohead}
def sqrt(x: real): real
def sqrt(x: real(32)): real(32)
\end{protohead}
\begin{protobody}
Returns the square root of the argument.  It is an error if the
argument is less than zero.
\end{protobody}

\begin{protohead}
def tan(x: real): real
def tan(x: real(32)): real(32)
\end{protohead}
\begin{protobody}
Returns the tangent of the argument.
\end{protobody}

\begin{protohead}
def tanh(x: real): real
def tanh(x: real(32)): real(32)
\end{protohead}
\begin{protobody}
Returns the hyperbolic tangent of the argument.
\end{protobody}

\begin{protohead}
def tgamma(x: real): real
def tgamma(x: real(32)): real(32)
\end{protohead}
\begin{protobody}
Returns the gamma function of the argument defined as
$$\int_0^\infty t^{x-1} e^{-t} dt$$
for the argument $x$.
\end{protobody}

\begin{protohead}
def trunc(x: real): real
def trunc(x: real(32)): real(32)
\end{protohead}
\begin{protobody}
Returns the nearest integral value to the argument that is not larger
than the argument in absolute value.
\end{protobody}


\subsubsection{Standard}
\label{Standard}
\index{automatic modules!Standard}

\begin{protohead}
def ascii(s: string): int
\end{protohead}
\begin{protobody}
Returns the ASCII code number of the first letter in the
argument \chpl{s}.
\end{protobody}

\begin{protohead}
def assert(test: bool) {
\end{protohead}
\begin{protobody}
Exits the program if \chpl{test} is false and prints to standard error
the location in the Chapel code of the call to \chpl{assert}.
If \chpl{test} is true, no action is taken.
\end{protobody}

\begin{protohead}
def assert(test: bool, args ...?numArgs) {
\end{protohead}
\begin{protobody}
Exits the program if \chpl{test} is false and prints to standard error
the location in the Chapel code of the call to \chpl{assert} as well
as the rest of the arguments to the call.  If \chpl{test} is true, no
action is taken.
\end{protobody}

\begin{protohead}
def complex.re: real
\end{protohead}
\begin{protobody}
Returns the real component of the complex number.
\end{protobody}

\begin{protohead}
def complex.im: real
\end{protohead}
\begin{protobody}
Returns the imaginary component of the complex number.
\end{protobody}

\begin{protohead}
def complex.=re(f: real)
\end{protohead}
\begin{protobody}
Sets the real component of the complex number to \chpl{f}.
\end{protobody}

\begin{protohead}
def complex.=im(f: real)
\end{protohead}
\begin{protobody}
Sets the imaginary component of the complex number to \chpl{f}.
\end{protobody}

\begin{protohead}
def exit(status: int)
\end{protohead}
\begin{protobody}
Exits the program with code \chpl{status}.
\end{protobody}

\begin{protohead}
def halt()
\end{protohead}
\begin{protobody}
Exits the program and prints to standard error the location in the
Chapel code of the call to \chpl{halt}.
\end{protobody}

\begin{protohead}
def halt(args ...?numArgs)
\end{protohead}
\begin{protobody}
Exits the program and prints to standard error the location in the
Chapel code of the call to \chpl{halt} as well as the rest of the
arguments to the call.
\end{protobody}

\begin{protohead}
def length(s: string): int
\end{protohead}
\begin{protobody}
Returns the number of characters in the argument \chpl{s}.
\end{protobody}

\begin{protohead}
def max(x, y...?k)
\end{protohead}
\begin{protobody}
Returns the maximum of the arguments when compared using the
``greater-than'' operator.  The return type is inferred from the types
of the arguments as allowed by implicit coercions.
\end{protobody}

\begin{protohead}
def min(x, y...?k)
\end{protohead}
\begin{protobody}
Returns the minimum of the arguments when compared using the
``less-than'' operator.  The return type is inferred from the types of
the arguments as allowed by implicit coercions.
\end{protobody}

\begin{protohead}
def string.substring(x): string
\end{protohead}
\begin{protobody}
Returns a value of string type that is a substring of the base
expression.  If \chpl{x} is $i$, a value of type \chpl{int}, then the
result is the $i$th character.  If \chpl{x} is a range, the result is
the substring where the characters in the substring are given by the
values in the range.
\end{protobody}

\begin{protohead}
def typeToString(type t) param : string
\end{protohead}
\begin{protobody}
Returns a string parameter that represents the name of the
type \chpl{t}.
\end{protobody}

\subsubsection{Types}

\begin{protohead}
def numBits(type t) param : int
\end{protohead}
\begin{protobody}
Returns the number of bits used to store the values of type \chpl{t}.
This is implemented for all numeric types and fixed-width \chpl{bool} types.
It is not implemented for default-width \chpl{bool}.
\end{protobody}


\begin{protohead}
def numBytes(type t) param : int
\end{protohead}
\begin{protobody}
Returns the number of bytes used to store the values of type \chpl{t}.
This is implemented for all numeric types and fixed-width \chpl{bool} types.
It is not implemented for default-width \chpl{bool}.
\end{protobody}

\begin{protohead}
def max(type t): t
\end{protohead}
\begin{protobody}
Returns the maximum value that can be stored in type \chpl{t}.  This
is implemented for all numeric types.
\end{protobody}

\begin{protohead}
def min(type t): t
\end{protohead}
\begin{protobody}
Returns the minimum value that can be stored in type \chpl{t}.  This
is implemented for all numeric types.
\end{protobody}



\subsection{Standard Modules}

Standard modules can be used by a Chapel program via the \chpl{use}
keyword.

\subsubsection{BitOps}
\label{BitOps}
\index{standard modules!BitOps}

The module \chpl{BitOps} defines routines that manipulate the bits of
values of integral types.

\vspace{1pc}

\begin{protohead}
def bitPop(i: integral): int
\end{protohead}
\begin{protobody}
Returns the number of bits set to one in the integral
argument \chpl{i}.
\end{protobody}

\begin{protohead}
def bitMatMultOr(i: uint(64), j: uint(64)): uint(64)
\end{protohead}
\begin{protobody}
Returns the bitwise matrix multiplication of \chpl{i} and \chpl{j}
where the values of \chpl{uint(64)} type are treated as $8 \times 8$
bit matrices and the combinator function is bitwise or.
\end{protobody}

\begin{protohead}
def bitRotLeft(i: integral, shift: integral): i.type
\end{protohead}
\begin{protobody}
Returns the value of the integral argument \chpl{i} after rotating the
bits to the left \chpl{shift} number of times.
\end{protobody}

\begin{protohead}
def bitRotRight(i: integral, shift: integral): i.type
\end{protohead}
\begin{protobody}
Returns the value of the integral argument \chpl{i} after rotating the
bits to the right \chpl{shift} number of times.
\end{protobody}


\subsubsection{Norm}
\label{Norm}
\index{standard modules!Norm}

The module \chpl{Norm} supports the computation of standard vector and
matrix norms on Chapel arrays.  The current interface is minimal and
should be expected to grow and evolve over time.

\begin{protohead}
enum normType {norm1, norm2, normInf, normFrob};
\end{protohead}
\begin{protobody}
An enumerated type indicating the different types of norms supported
by this module: 1-norm, 2-norm, infinity norm and Frobenius norm,
respectively.
\end{protobody}

\begin{protohead}
def norm(x: [], p: normType) where x.rank == 1 || x.rank == 2
\end{protohead}
\begin{protobody}
Compute the norm indicated by \chpl{p} on the 1D or 2D array \chpl{x}.
\end{protobody}

\begin{protohead}
def norm(x: [])
\end{protohead}
\begin{protobody}
Compute the default norm on array \chpl{x}.  For a 1D array this is
the 2-norm, for a 2D array, this is the Frobenius norm.
\end{protobody}

\subsubsection{Random}
\label{Random}
\index{standard modules!Random}

The module \chpl{Random} supports the generation of pseudo-random
values and streams of values.  The current interface is minimal and
should be expected to grow and evolve over time.  In particular, we
expect to support other pseudo-random number generation algorithms,
more random value types (\eg, int), and both serial and parallel
iterators over the RandomStream class.

\begin{protohead}
class RandomStream
def RandomStream(seed: int(64), param parSafe: bool = true)
def RandomStream(seedGenerator: SeedGenerator = SeedGenerator.currentTime,
                 param parSafe: bool = true)
\end{protohead}
\begin{protobody}
Implements a pseudo-random stream of values based on a seed value.
The current implementation generates the values using a linear
congruential generator.  In future versions of this module, the
RandomStream class will offer a wider variety of algorithms for
generating pseudo-random values.

To construct a RandomStream class, the seed may be explicitly passed.
It must be an odd integer between $1$ and $2^{46}-1$.  Alternatively,
the RandomStream class can be constructed by passing a value of the
enumerated type SeedGenerator to choose an algorithm to use to set the
seed.  If neither a seed nor a SeedGenerator value is passed to the
RandomStream class, the seed will be initialized based on the current
time in microseconds (rounded via modular arithmetic to the nearest
odd integer between $1$ and $2^{46}-1$.

The parSafe parameter defaults to true and allows for safe use of this
class by concurrent tasks.  This can be overridden when calling
methods to make them safe when called by concurrent tasks.  This
mechanism allows for lower overhead calls when there is no threat of
concurrent calls, but correct calls when there is.
\end{protobody}

\begin{protohead}
enum SeedGenerator { currentTime };
\end{protohead}
\begin{protobody}
Values of this enumerated type may be used to choose a method for
initializing the seed in the RandomStream class.  The only value
supported at present is \chpl{currentTime} which can be used to
initialize the seed based on the current time in microseconds (rounded
via modular arithmetic to the nearest odd integer between $1$ and
$2^{46}-1$.
\end{protobody}

\begin{protohead}
def RandomStream.fillRandom(x:[?D], param parSafe = this.parSafe)
\end{protohead}
\begin{protobody}
Fill the argument array, \chpl{x}, with the next $|$\chpl{D}$|$ values
of the pseudo-random stream in row-major order.  The array must be an
array of real(64), imag(64), or complex(128) elements.  For complex
arrays, each complex element is initialized with two values from the
stream of random numbers.
\end{protobody}

\begin{protohead}
def RandomStream.skipToNth(in n: integral, param parSafe = this.parSafe)
\end{protohead}
\begin{protobody}
Skips ahead or back to the \chpl{n}-th value in the
random stream.  The value of n is assumed to be positive, such that
\chpl{n}~==~1 represents the initial value in the stream.
\end{protobody}

\begin{protohead}
def RandomStream.getNext(param parSafe = this.parSafe): real
\end{protohead}
\begin{protobody}
Returns the next value in the random stream as a real.
\end{protobody}

\begin{protohead}
def RandomStream.getNth(n: integral, param parSafe = this.parSafe): real
\end{protohead}
\begin{protobody}
Returns the \chpl{n}-th value in the random stream as a real.  Equivalent to
calling \chpl{skipToNth(n)} followed by \chpl{getNext()}.
\end{protobody}

\begin{protohead}
def fillRandom(x:[], initseed: int(64))
\end{protohead}
\begin{protobody}
A routine provided for convenience to support the functionality of the
fillRandom method (above) without explicitly constructing an instance
of the \chpl{RandomStream} class.  This is useful for filling a single
array or multiple arrays which require no coherence between them.
The \chpl{initseed} parameter corresponds to the \chpl{seed} member of
the \chpl{RandomStream} class.  If unspecified, the default for the
class will be used.
\end{protobody}

\subsubsection{Search}
\label{Search}
\index{standard modules!Search}

The \chpl{Search} module is designed to support standard search
routines.  The current interface is minimal and should be expected to
grow and evolve over time.

\begin{protohead}
def LinearSearch(Data: [?Dom], val): (bool, index(Dom))
\end{protohead}
\begin{protobody}
Searches through the pre-sorted array \chpl{Data} looking for the
value \chpl{val} using a sequential linear search.  Returns a tuple
indicating (1) whether or not the value was found and (2) the location
of the value if it was found, or the location where the value should
have been if it was not found.
\end{protobody}


\begin{protohead}
def BinarySearch(Data: [?Dom], val, in lo = Dom.low, in hi = Dom.high)
\end{protohead}
\begin{protobody}
Searches through the pre-sorted array \chpl{Data} looking for the
value \chpl{val} using a sequential binary search.  If provided, only
the indices \chpl{lo} through \chpl{hi} will be considered, otherwise
the whole array will be searched.  Returns a tuple indicating (1)
whether or not the value was found and (2) the location of the value
if it was found, or the location where the value should have been if
it was not found.
\end{protobody}


\subsubsection{Sort}
\label{Sort}
\index{standard modules!Sort}

The \chpl{Sort} module is designed to support standard sorting
routines.  The current interface is minimal and should be expected to
grow and evolve over time.

\begin{protohead}
def InsertionSort(Data: [?Dom]) where Dom.rank == 1
\end{protohead}
\begin{protobody}
Sorts the 1D array \chpl{Data} in-place using a sequential insertion
sort algorithm.
\end{protobody}

\begin{protohead}
def QuickSort(Data: [?Dom]) where Dom.rank == 1
\end{protohead}
\begin{protobody}
Sorts the 1D array \chpl{Data} in-place using a sequential
implementation of the QuickSort algorithm.
\end{protobody}

\subsubsection{Time}
\label{Time}
\index{standard modules!Time}

The module \chpl{Time} defines routines that query the system time and
a record \chpl{Timer} that is useful for timing portions of code.

\vspace{1pc}

\begin{protohead}
record Timer
\end{protohead}
\begin{protobody}
A timer is used to time portions of code.  Its semantics are similar
to a stopwatch.
\end{protobody}

\begin{protohead}
enum TimeUnits { microseconds, milliseconds, seconds, minutes, hours };
\end{protohead}
\begin{protobody}
The enumeration TimeUnits defines units of time.  These units can be
supplied to routines in this module to specify the desired time units.
\end{protobody}

\begin{protohead}
enum Day { sunday=0, monday, tuesday, wednesday, thursday, friday, saturday };
\end{protohead}
\begin{protobody}
The enumeration Day defines the days of the week, with Sunday defined to be 0.
\end{protobody}


\begin{protohead}
def getCurrentDate(): (int, int, int)
\end{protohead}
\begin{protobody}
Returns the year, month, and day of the month as integers.  The year
is the year since 0.  The month is in the range 1 to 12.  The day is
in the range 1 to 31.
\end{protobody}

\begin{protohead}
def getCurrentDayOfWeek(): Day
\end{protohead}
\begin{protobody}
Returns the current day of the week.
\end{protobody}


\begin{protohead}
def getCurrentTime(unit: TimeUnits = TimeUnits.seconds): real
\end{protohead}
\begin{protobody}
Returns the elapsed time since midnight in the units specified.
\end{protobody}

\begin{protohead}
def Timer.clear()
\end{protohead}
\begin{protobody}
Clears the elapsed time stored in the Timer.
\end{protobody}

\begin{protohead}
def Timer.elapsed(unit: TimeUnits = TimeUnits.seconds): real
\end{protohead}
\begin{protobody}
Returns the cumulative elapsed time, in the units specified, between
calls to \chpl{start} and \chpl{stop}.  If the timer is running, the
elapsed time since the last call to \chpl{start} is added to the
return value.
\end{protobody}

\begin{protohead}
def Timer.start()
\end{protohead}
\begin{protobody}
Starts the timer.  It is an error to start a timer that is already
running.
\end{protobody}

\begin{protohead}
def Timer.stop()
\end{protohead}
\begin{protobody}
Stops the timer.  It is an error to stop a timer that is not running.
\end{protobody}

\begin{protohead}
def sleep(t: uint)
\end{protohead}
\begin{protobody}
Delays a task for \chpl{t} seconds.
\end{protobody}


\cleardoublepage
\appendix
%%
%% Do not modify this file.  This file is automatically
%% generated by collect_syntax.pl.
%%

\sekshun{Collected Lexical and Syntax Productions}
\label{Syntax}

This appendix collects the syntax productions listed throughout the specification.  There are no new syntax productions in this appendix.  The productions are listed both alphabetically and in depth-first order for convenience.

\section{Alphabetical Lexical Productions}

\begin{syntax}
\end{syntax}

\begin{syntax}
binary-digit: one of
  `0' `1'
\end{syntax}

\begin{syntax}
binary-digits:
  binary-digit
  binary-digit binary-digits
\end{syntax}

\begin{syntax}
bool-literal: one of
  `true' $ $ $ $ `false'
\end{syntax}

\begin{syntax}
digit: one of
  `0' `1' `2' `3' `4' `5' `6' `7' `8' `9'
\end{syntax}

\begin{syntax}
digits:
  digit
  digit digits
\end{syntax}

\begin{syntax}
double-quote-delimited-characters:
  string-character double-quote-delimited-characters[OPT]
  ' double-quote-delimited-characters[OPT]
\end{syntax}

\begin{syntax}
exponent-part:
  `e' sign[OPT] digits
  `E' sign[OPT] digits
\end{syntax}

\begin{syntax}
hexadecimal-digit: one of
  `0' `1' `2' `3' `4' `5' `6' `7' `8' `9' `A' `B' `C' `D' `E' `F' `a' `b' `c' `d' `e' `f'
\end{syntax}

\begin{syntax}
hexadecimal-digits:
  hexadecimal-digit
  hexadecimal-digit hexadecimal-digits
\end{syntax}

\begin{syntax}
hexadecimal-escape-character:
  `$\backslash$x' hexadecimal-digits
\end{syntax}

\begin{syntax}
identifier:
  letter-or-underscore legal-identifier-chars[OPT]
\end{syntax}

\begin{syntax}
imaginary-literal:
  real-literal `i'
  integer-literal `i'
\end{syntax}

\begin{syntax}
integer-literal:
  digits
  `0x' hexadecimal-digits
  `0X' hexadecimal-digits
  `0o' octal-digits
  `0O' octal-digits
  `0b' binary-digits
  `0B' binary-digits
\end{syntax}

\begin{syntax}
legal-identifier-char:
  letter-or-underscore
  digit
  `(*\texttt{\$}*)'
\end{syntax}

\begin{syntax}
legal-identifier-chars:
  legal-identifier-char legal-identifier-chars[OPT]
\end{syntax}

\begin{syntax}
letter-or-underscore:
  letter
  `_'
\end{syntax}

\begin{syntax}
letter: one of
  `A' `B' `C' `D' `E' `F' `G' `H' `I' `J' `K' `L' `M' `N' `O' `P' `Q' `R' `S' `T' `U' `V' `W' `X' `Y' `Z'
  `a' `b' `c' `d' `e' `f' `g' `h' `i' `j' `k' `l' `m' `n' `o' `p' `q' `r' `s' `t' `u' `v' `w' `x' `y' `z'
\end{syntax}

\begin{syntax}
octal-digit: one of
  `0' `1' `2' `3' `4' `5' `6' `7'
\end{syntax}

\begin{syntax}
octal-digits:
  octal-digit
  octal-digit octal-digits
\end{syntax}

\begin{syntax}
p-exponent-part:
  `p' sign[OPT] digits
  `P' sign[OPT] digits
\end{syntax}

\begin{syntax}
real-literal:
  digits[OPT] . digits exponent-part[OPT]
  digits .[OPT] exponent-part
  `0x' hexadecimal-digits[OPT] . hexadecimal-digits p-exponent-part[OPT]
  `0X' hexadecimal-digits[OPT] . hexadecimal-digits p-exponent-part[OPT]
  `0x' hexadecimal-digits .[OPT] p-exponent-part
  `0X' hexadecimal-digits .[OPT] p-exponent-part
\end{syntax}

\begin{syntax}
sign: one of
  + $ $ $ $ -
\end{syntax}

\begin{syntax}
simple-escape-character: one of
  `$\backslash\mbox{\bf '}\hspace{5pt}$' `$\backslash$"$\hspace{5pt}$' `$\backslash$?$\hspace{5pt}$' `$\backslash$$\backslash$$\hspace{5pt}$' `$\backslash$a$\hspace{5pt}$' `$\backslash$b$\hspace{5pt}$' `$\backslash$f$\hspace{5pt}$' `$\backslash$n$\hspace{5pt}$' `$\backslash$r$\hspace{5pt}$' `$\backslash$t$\hspace{5pt}$' `$\backslash$v$\hspace{5pt}$'
\end{syntax}

\begin{syntax}
single-quote-delimited-characters:
  string-character single-quote-delimited-characters[OPT]
  " single-quote-delimited-characters[OPT]
\end{syntax}

\begin{syntax}
string-character:
  `any character except the double quote, single quote, or new line'
  simple-escape-character
  hexadecimal-escape-character
\end{syntax}

\begin{syntax}
string-literal:
  " double-quote-delimited-characters[OPT] "
  ' single-quote-delimited-characters[OPT] '
\end{syntax}

\section{Alphabetical Syntax Productions}

\begin{syntax}
\end{syntax}

\begin{syntax}
aligned-range-expression:
  range-expression `align' expression
\end{syntax}

\begin{syntax}
argument-list:
  ( formals[OPT] )
\end{syntax}

\begin{syntax}
array-alias-declaration:
  identifier reindexing-expression[OPT] => array-expression ;
\end{syntax}

\begin{syntax}
array-expression:
  expression
\end{syntax}

\begin{syntax}
array-literal:
  rectangular-array-literal
  associative-array-literal
\end{syntax}

\begin{syntax}
array-type:
  [ domain-expression ] type-specifier
\end{syntax}

\begin{syntax}
assignment-operator: one of
   = $ $ $ $ += $ $ $ $ -= $ $ $ $ *= $ $ $ $ /= $ $ $ $ %= $ $ $ $ **= $ $ $ $ &= $ $ $ $ |= $ $ $ $ ^= $ $ $ $ &&= $ $ $ $ ||= $ $ $ $ <<= $ $ $ $ >>=
\end{syntax}

\begin{syntax}
assignment-statement:
  lvalue-expression assignment-operator expression
\end{syntax}

\begin{syntax}
associative-array-literal:
  [ associative-expr-list ]
  [ associative-expr-list , ]
\end{syntax}

\begin{syntax}
associative-domain-literal:
   { associative-expression-list }
\end{syntax}

\begin{syntax}
associative-domain-type:
  `domain' ( associative-index-type )
  `domain' ( enum-type )
  `domain' ( `opaque' )
\end{syntax}

\begin{syntax}
associative-expr-list:
  index-expr => value-expr
  index-expr => value-expr, associative-expr-list
\end{syntax}

\begin{syntax}
associative-expression-list:
   non-range-expression
   non-range-expression, associative-expression-list
\end{syntax}

\begin{syntax}
associative-index-type:
  type-specifier
\end{syntax}

\begin{syntax}
atomic-statement:
  `atomic' statement
\end{syntax}

\begin{syntax}
atomic-type:
  `atomic' type-specifier
\end{syntax}

\begin{syntax}
base-domain-type:
  rectangular-domain-type
  associative-domain-type
\end{syntax}

\begin{syntax}
begin-statement:
  `begin' task-intent-clause[OPT] statement
\end{syntax}

\begin{syntax}
binary-expression:
  expression binary-operator expression
\end{syntax}

\begin{syntax}
binary-operator: one of
  + $ $ $ $ - $ $ $ $ * $ $ $ $ / $ $ $ $ % $ $ $ $ ** $ $ $ $ & $ $ $ $ | $ $ $ $ ^ $ $ $ $ << $ $ $ $ >> $ $ $ $ && $ $ $ $ || $ $ $ $ == $ $ $ $ != $ $ $ $ <= $ $ $ $ >= $ $ $ $ < $ $ $ $ > $ $ $ $ `by' $ $ $ $ #
\end{syntax}

\begin{syntax}
block-statement:
  { statements[OPT] }
\end{syntax}

\begin{syntax}
break-statement:
  `break' identifier[OPT] ;
\end{syntax}

\begin{syntax}
call-expression:
  lvalue-expression ( named-expression-list )
  lvalue-expression [ named-expression-list ]
  parenthesesless-function-identifier
\end{syntax}

\begin{syntax}
cast-expression:
  expression : type-specifier
\end{syntax}

\begin{syntax}
class-declaration-statement:
  simple-class-declaration-statement
  external-class-declaration-statement
\end{syntax}

\begin{syntax}
class-inherit-list:
  : class-type-list
\end{syntax}

\begin{syntax}
class-name:
  identifier
\end{syntax}

\begin{syntax}
class-statement-list:
  class-statement
  class-statement class-statement-list
\end{syntax}

\begin{syntax}
class-statement:
  variable-declaration-statement
  method-declaration-statement
  type-declaration-statement
  empty-statement
\end{syntax}

\begin{syntax}
class-type-list:
  class-type
  class-type , class-type-list
\end{syntax}

\begin{syntax}
class-type:
  identifier
  identifier ( named-expression-list )
\end{syntax}

\begin{syntax}
cobegin-statement:
  `cobegin' task-intent-clause[OPT] block-statement
\end{syntax}

\begin{syntax}
coforall-statement:
  `coforall' index-var-declaration `in' iteratable-expression task-intent-clause[OPT] `do' statement
  `coforall' index-var-declaration `in' iteratable-expression task-intent-clause[OPT] block-statement
  `coforall' iteratable-expression task-intent-clause[OPT] `do' statement
  `coforall' iteratable-expression task-intent-clause[OPT] block-statement
\end{syntax}

\begin{syntax}
conditional-statement:
  `if' expression `then' statement else-part[OPT]
  `if' expression block-statement else-part[OPT]
\end{syntax}

\begin{syntax}
config-or-extern: one of
  `config' $ $ $ $ `extern'
\end{syntax}

\begin{syntax}
constructor-call-expression:
  `new' class-name ( argument-list )
\end{syntax}

\begin{syntax}
continue-statement:
  `continue' identifier[OPT] ;
\end{syntax}

\begin{syntax}
counted-range-expression:
  range-expression # expression
\end{syntax}

\begin{syntax}
dataparallel-type:
  range-type
  domain-type
  mapped-domain-type
  array-type
  index-type
\end{syntax}

\begin{syntax}
default-expression:
  = expression
\end{syntax}

\begin{syntax}
delete-statement:
  `delete' expression ;
\end{syntax}

\begin{syntax}
dmap-value:
  expression
\end{syntax}

\begin{syntax}
do-while-statement:
  `do' statement `while' expression ;
\end{syntax}

\begin{syntax}
domain-alignment-expression:
  domain-expression `align' expression
\end{syntax}

\begin{syntax}
domain-assignment-expression:
  domain-name = domain-expression
\end{syntax}

\begin{syntax}
domain-expression:
  domain-literal
  domain-name
  domain-assignment-expression
  domain-striding-expression
  domain-alignment-expression
  domain-slice-expression
\end{syntax}

\begin{syntax}
domain-literal:
  rectangular-domain-literal
  associative-domain-literal
\end{syntax}

\begin{syntax}
domain-name:
  identifier
\end{syntax}

\begin{syntax}
domain-slice-expression:
  domain-expression [ slicing-index-set ]
  domain-expression ( slicing-index-set )
\end{syntax}

\begin{syntax}
domain-striding-expression:
  domain-expression `by' expression
\end{syntax}

\begin{syntax}
domain-type:
  base-domain-type
  simple-subdomain-type
  sparse-subdomain-type
\end{syntax}

\begin{syntax}
else-part:
  `else' statement
\end{syntax}

\begin{syntax}
empty-statement:
  ;
\end{syntax}

\begin{syntax}
enum-constant-expression:
  enum-type . identifier
\end{syntax}

\begin{syntax}
enum-constant-list:
  enum-constant
  enum-constant , enum-constant-list[OPT]
\end{syntax}

\begin{syntax}
enum-constant:
  identifier init-part[OPT]
\end{syntax}

\begin{syntax}
enum-declaration-statement:
  `enum' identifier { enum-constant-list }
\end{syntax}

\begin{syntax}
enum-type:
  identifier
\end{syntax}

\begin{syntax}
exclude-list:
  identifier-list
  $ * $
\end{syntax}

\begin{syntax}
exported-procedure-declaration-statement:
  `export' external-name[OPT] `proc' function-name argument-list return-intent[OPT] return-type[OPT]
    function-body
\end{syntax}

\begin{syntax}
expression-list:
  expression
  expression , expression-list
\end{syntax}

\begin{syntax}
expression-statement:
  variable-expression ;
  member-access-expression ;
  call-expression ;
  constructor-call-expression ;
  let-expression ; 
\end{syntax}

\begin{syntax}
expression:
  literal-expression
  nil-expression
  variable-expression
  enum-constant-expression
  call-expression
  iteratable-call-expression
  member-access-expression
  constructor-call-expression
  query-expression
  cast-expression
  lvalue-expression
  parenthesized-expression
  unary-expression
  binary-expression
  let-expression
  if-expression
  for-expression
  forall-expression
  reduce-expression
  scan-expression
  module-access-expression
  tuple-expression
  tuple-expand-expression
  locale-access-expression
  mapped-domain-expression
\end{syntax}

\begin{syntax}
external-class-declaration-statement:
  `extern' external-name[OPT] simple-class-declaration-statement
\end{syntax}

\begin{syntax}
external-name:
  identifier
  string-literal
\end{syntax}

\begin{syntax}
external-procedure-declaration-statement:
  `extern' external-name[OPT] `proc' function-name argument-list return-intent[OPT] return-type[OPT]
\end{syntax}

\begin{syntax}
external-record-declaration-statement:
  `extern' external-name[OPT] simple-record-declaration-statement
\end{syntax}

\begin{syntax}
external-type-alias-declaration-statement:
  `extern' `type' type-alias-declaration-list ;
\end{syntax}

\begin{syntax}
field-access-expression:
  receiver-clause[OPT] identifier
\end{syntax}

\begin{syntax}
for-expression:
  `for' index-var-declaration `in' iteratable-expression `do' expression
  `for' iteratable-expression `do' expression
\end{syntax}

\begin{syntax}
for-statement:
  `for' index-var-declaration `in' iteratable-expression `do' statement
  `for' index-var-declaration `in' iteratable-expression block-statement
  `for' iteratable-expression `do' statement
  `for' iteratable-expression block-statement
\end{syntax}

\begin{syntax}
forall-expression:
  `forall' index-var-declaration `in' iteratable-expression task-intent-clause[OPT] `do' expression
  `forall' iteratable-expression task-intent-clause[OPT] `do' expression
  [ index-var-declaration `in' iteratable-expression task-intent-clause[OPT] ] expression
  [ iteratable-expression task-intent-clause[OPT] ] expression
\end{syntax}

\begin{syntax}
forall-statement:
  `forall' index-var-declaration `in' iteratable-expression task-intent-clause[OPT] `do' statement
  `forall' index-var-declaration `in' iteratable-expression task-intent-clause[OPT] block-statement
  `forall' iteratable-expression task-intent-clause[OPT] `do' statement
  `forall' iteratable-expression task-intent-clause[OPT] block-statement
  [ index-var-declaration `in' iteratable-expression task-intent-clause[OPT] ] statement
  [ iteratable-expression task-intent-clause[OPT] ] statement
\end{syntax}

\begin{syntax}
formal-intent:
  `const'
  `const in'
  `const ref'
  `in'
  `out'
  `inout'
  `ref'
  `param'
  `type'
\end{syntax}

\begin{syntax}
formal-type:
  : type-specifier
  : ? identifier[OPT]
\end{syntax}

\begin{syntax}
formal:
  formal-intent[OPT] identifier formal-type[OPT] default-expression[OPT]
  formal-intent[OPT] identifier formal-type[OPT] variable-argument-expression
  formal-intent[OPT] tuple-grouped-identifier-list formal-type[OPT] default-expression[OPT]
  formal-intent[OPT] tuple-grouped-identifier-list formal-type[OPT] variable-argument-expression
\end{syntax}

\begin{syntax}
formals:
  formal
  formal , formals
\end{syntax}

\begin{syntax}
function-body:
  block-statement
  return-statement
\end{syntax}

\begin{syntax}
function-name:
  identifier
  operator-name
\end{syntax}

\begin{syntax}
identifier-list:
  identifier
  identifier , identifier-list
  tuple-grouped-identifier-list
  tuple-grouped-identifier-list , identifier-list
\end{syntax}

\begin{syntax}
if-expression:
  `if' expression `then' expression `else' expression
  `if' expression `then' expression
\end{syntax}

\begin{syntax}
index-expr:
  expression
\end{syntax}

\begin{syntax}
index-type:
  `index' ( domain-expression )
\end{syntax}

\begin{syntax}
index-var-declaration:
  identifier
  tuple-grouped-identifier-list
\end{syntax}

\begin{syntax}
init-part:
  = expression
\end{syntax}

\begin{syntax}
initialization-part:
  = expression
\end{syntax}

\begin{syntax}
integer-parameter-expression:
  expression
\end{syntax}

\begin{syntax}
io-expression:
  expression
  io-expression io-operator expression
\end{syntax}

\begin{syntax}
io-operator:
  <`(*$\sim$*)'>
\end{syntax}

\begin{syntax}
io-statement:
  io-expression io-operator expression
\end{syntax}

\begin{syntax}
iteratable-call-expression:
  call-expression
\end{syntax}

\begin{syntax}
iteratable-expression:
  expression
  `zip' ( expression-list )
\end{syntax}

\begin{syntax}
iterator-body:
  block-statement
  yield-statement
\end{syntax}

\begin{syntax}
iterator-declaration-statement:
  privacy-specifier[OPT] `iter' iterator-name argument-list[OPT] return-intent[OPT] return-type[OPT] where-clause[OPT]
  iterator-body
\end{syntax}

\begin{syntax}
iterator-name:
  identifier
\end{syntax}

\begin{syntax}
label-statement:
  `label' identifier statement
\end{syntax}

\begin{syntax}
let-expression:
  `let' variable-declaration-list `in' expression
\end{syntax}

\begin{syntax}
limitation-clause:
  `except' exclude-list
  `only' rename-list[OPT]
\end{syntax}

\begin{syntax}
linkage-specifier:
  `inline'
\end{syntax}

\begin{syntax}
literal-expression:
  bool-literal
  integer-literal
  real-literal
  imaginary-literal
  string-literal
  range-literal
  domain-literal
  array-literal
\end{syntax}

\begin{syntax}
locale-access-expression:
  expression . `locale'
\end{syntax}

\begin{syntax}
lvalue-expression:
  variable-expression
  member-access-expression
  call-expression
  parenthesized-expression
\end{syntax}

\begin{syntax}
mapped-domain-expression:
  domain-expression `dmapped' dmap-value
\end{syntax}

\begin{syntax}
mapped-domain-type:
  domain-type `dmapped' dmap-value
\end{syntax}

\begin{syntax}
member-access-expression:
  field-access-expression
  method-call-expression
\end{syntax}

\begin{syntax}
method-call-expression:
  receiver-clause[OPT] expression ( named-expression-list )
  receiver-clause[OPT] expression [ named-expression-list ]
  receiver-clause[OPT] parenthesesless-function-identifier
\end{syntax}

\begin{syntax}
method-declaration-statement:
  linkage-specifier[OPT] proc-or-iter this-intent[OPT] type-binding[OPT] function-name argument-list[OPT] 
    return-intent[OPT] return-type[OPT] where-clause[OPT] function-body
\end{syntax}

\begin{syntax}
module-access-expression:
  module-identifier-list . identifier
\end{syntax}

\begin{syntax}
module-declaration-statement:
  privacy-specifier[OPT] `module' module-identifier block-statement
\end{syntax}

\begin{syntax}
module-identifier-list:
  module-identifier
  module-identifier . module-identifier-list
\end{syntax}

\begin{syntax}
module-identifier:
  identifier
\end{syntax}

\begin{syntax}
module-or-enum-name-list:
  module-or-enum-name limitation-clause[OPT]
  module-or-enum-name , module-or-enum-name-list
\end{syntax}

\begin{syntax}
module-or-enum-name:
  identifier
  identifier . module-or-enum-name
\end{syntax}

\begin{syntax}
named-expression-list:
  named-expression
  named-expression , named-expression-list
\end{syntax}

\begin{syntax}
named-expression:
  expression
  identifier = expression
\end{syntax}

\begin{syntax}
nil-expression:
  `nil'
\end{syntax}

\begin{syntax}
no-initialization-part:
  = `noinit'
\end{syntax}

\begin{syntax}
non-range-expression:
   expression
\end{syntax}

\begin{syntax}
on-statement:
  `on' expression `do' statement
  `on' expression block-statement
\end{syntax}

\begin{syntax}
operator-name: one of
  + $ $ $ $ - $ $ $ $ * $ $ $ $ / $ $ $ $ % $ $ $ $ ** $ $ $ $ ! $ $ $ $ == $ $ $ $ != $ $ $ $ <= $ $ $ $ >= $ $ $ $ < $ $ $ $ > $ $ $ $ << $ $ $ $ >> $ $ $ $ & $ $ $ $ | $ $ $ $ ^ $ $ $ $ ~
  += $ $ $ $ -= $ $ $ $ *= $ $ $ $ /= $ $ $ $ %= $ $ $ $ **= $ $ $ $ &= $ $ $ $ |= $ $ $ $ ^= $ $ $ $ <<= $ $ $ $ >>= $ $ $ $ <=> $ $ $ $ <~>
\end{syntax}

\begin{syntax}
param-for-statement:
  `for' `param' identifier `in' param-iteratable-expression `do' statement
  `for' `param' identifier `in' param-iteratable-expression block-statement
\end{syntax}

\begin{syntax}
param-iteratable-expression:
  range-literal
  range-literal `by' integer-literal
\end{syntax}

\begin{syntax}
parenthesesless-function-identifier:
  identifier
\end{syntax}

\begin{syntax}
parenthesized-expression:
  ( expression )
\end{syntax}

\begin{syntax}
primitive-type-parameter-part:
  ( integer-parameter-expression )
\end{syntax}

\begin{syntax}
primitive-type:
  `void'
  `bool' primitive-type-parameter-part[OPT]
  `int' primitive-type-parameter-part[OPT]
  `uint' primitive-type-parameter-part[OPT]
  `real' primitive-type-parameter-part[OPT]
  `imag' primitive-type-parameter-part[OPT]
  `complex' primitive-type-parameter-part[OPT]
  `string'
\end{syntax}

\begin{syntax}
privacy-specifier:
  `private'
  `public'
\end{syntax}

\begin{syntax}
proc-or-iter:
  `proc'
  `iter'
\end{syntax}

\begin{syntax}
procedure-declaration-statement:
  privacy-specifier[OPT] linkage-specifier[OPT] `proc' function-name argument-list[OPT] return-intent[OPT] return-type[OPT] where-clause[OPT]
    function-body
\end{syntax}

\begin{syntax}
query-expression:
  ? identifier[OPT]
\end{syntax}

\begin{syntax}
range-expression-list:
  range-expression
  range-expression, range-expression-list
\end{syntax}

\begin{syntax}
range-expression:
  expression
  strided-range-expression
  counted-range-expression
  aligned-range-expression
  sliced-range-expression
\end{syntax}

\begin{syntax}
range-literal:
  expression .. expression
  expression ..
  .. expression
  ..
\end{syntax}

\begin{syntax}
range-type:
  `range' ( named-expression-list )
\end{syntax}

\begin{syntax}
receiver-clause:
  expression .
\end{syntax}

\begin{syntax}
record-declaration-statement:
  simple-record-declaration-statement
  external-record-declaration-statement
\end{syntax}

\begin{syntax}
record-inherit-list:
  : record-type-list
\end{syntax}

\begin{syntax}
record-statement-list:
  record-statement
  record-statement record-statement-list
\end{syntax}

\begin{syntax}
record-statement:
  variable-declaration-statement
  method-declaration-statement
  type-declaration-statement
  empty-statement
\end{syntax}

\begin{syntax}
record-type-list:
  record-type
  record-type , record-type-list
\end{syntax}

\begin{syntax}
record-type:
  identifier
  identifier ( named-expression-list )
\end{syntax}

\begin{syntax}
rectangular-array-literal:
  [ expression-list ]
  [ expression-list , ]
\end{syntax}

\begin{syntax}
rectangular-domain-literal:
  { range-expression-list }
\end{syntax}

\begin{syntax}
rectangular-domain-type:
  `domain' ( named-expression-list )
\end{syntax}

\begin{syntax}
reduce-expression:
  reduce-scan-operator `reduce' iteratable-expression
  class-type `reduce' iteratable-expression
\end{syntax}

\begin{syntax}
reduce-scan-operator: one of
  + $ $ $ $ * $ $ $ $ && $ $ $ $ || $ $ $ $ & $ $ $ $ | $ $ $ $ ^ $ $ $ $ `min' $ $ $ $ `max' $ $ $ $ `minloc' $ $ $ $ `maxloc'
\end{syntax}

\begin{syntax}
reindexing-expression:
  : [ domain-expression ]
\end{syntax}

\begin{syntax}
remote-variable-declaration-statement:
  `on' expression variable-declaration-statement
\end{syntax}

\begin{syntax}
rename-base:
  identifier `as' identifier
  identifier
\end{syntax}

\begin{syntax}
rename-list:
  rename-base
  rename-base , rename-list
\end{syntax}

\begin{syntax}
return-intent:
  `const'
  `const ref'
  `ref'
  `param'
  `type'
\end{syntax}

\begin{syntax}
return-statement:
  `return' expression[OPT] ;
\end{syntax}

\begin{syntax}
return-type:
  : type-specifier
\end{syntax}

\begin{syntax}
scan-expression:
  reduce-scan-operator `scan' iteratable-expression
  class-type `scan' iteratable-expression
\end{syntax}

\begin{syntax}
select-statement:
  `select' expression { when-statements }
\end{syntax}

\begin{syntax}
serial-statement:
  `serial' expression[OPT] `do' statement
  `serial' expression[OPT] block-statement
\end{syntax}

\begin{syntax}
simple-class-declaration-statement:
  `class' identifier class-inherit-list[OPT] { class-statement-list[OPT] }
\end{syntax}

\begin{syntax}
simple-record-declaration-statement:
  `record' identifier record-inherit-list[OPT] { record-statement-list }
\end{syntax}

\begin{syntax}
simple-subdomain-type:
  `subdomain' ( domain-expression )
\end{syntax}

\begin{syntax}
single-type:
  `single' type-specifier
\end{syntax}

\begin{syntax}
sliced-range-expression:
  range-expression ( range-expression )
  range-expression [ range-expression ]
\end{syntax}

\begin{syntax}
slicing-index-set:
  domain-expression
  range-expression-list
\end{syntax}

\begin{syntax}
sparse-subdomain-type:
  `sparse' `subdomain'[OPT] ( domain-expression )
\end{syntax}

\begin{syntax}
statement:
  block-statement
  expression-statement
  assignment-statement
  swap-statement
  io-statement
  conditional-statement
  select-statement
  while-do-statement
  do-while-statement
  for-statement
  label-statement
  break-statement
  continue-statement
  param-for-statement
  use-statement
  empty-statement
  return-statement
  yield-statement
  module-declaration-statement
  procedure-declaration-statement
  external-procedure-declaration-statement
  exported-procedure-declaration-statement
  iterator-declaration-statement
  method-declaration-statement
  type-declaration-statement
  variable-declaration-statement
  remote-variable-declaration-statement
  on-statement
  cobegin-statement
  coforall-statement
  begin-statement
  sync-statement
  serial-statement
  atomic-statement
  forall-statement
  delete-statement
\end{syntax}

\begin{syntax}
statements:
  statement
  statement statements
\end{syntax}

\begin{syntax}
step-expression:
  expression
\end{syntax}

\begin{syntax}
strided-range-expression:
  range-expression `by' step-expression
\end{syntax}

\begin{syntax}
structured-type:
  class-type
  record-type
  union-type
  tuple-type
\end{syntax}

\begin{syntax}
swap-operator:
  <=>
\end{syntax}

\begin{syntax}
swap-statement:
  lvalue-expression swap-operator lvalue-expression
\end{syntax}

\begin{syntax}
sync-statement:
  `sync' statement
  `sync' block-statement
\end{syntax}

\begin{syntax}
sync-type:
  `sync' type-specifier
\end{syntax}

\begin{syntax}
synchronization-type:
  sync-type
  single-type
  atomic-type
\end{syntax}

\begin{syntax}
task-intent-clause:
  `with' ( task-intent-list )
\end{syntax}

\begin{syntax}
task-intent-list:
  formal-intent identifier
  formal-intent identifier, task-intent-list
\end{syntax}

\begin{syntax}
this-intent:
  `param'
  `ref'
  `type'
\end{syntax}

\begin{syntax}
tuple-component-list:
  tuple-component
  tuple-component , tuple-component-list
\end{syntax}

\begin{syntax}
tuple-component:
  expression
  `_'
\end{syntax}

\begin{syntax}
tuple-expand-expression:
  ( ... expression )
\end{syntax}

\begin{syntax}
tuple-expression:
  ( tuple-component , )
  ( tuple-component , tuple-component-list )
  ( tuple-component , tuple-component-list , )
\end{syntax}

\begin{syntax}
tuple-grouped-identifier-list:
  ( identifier-list )
\end{syntax}

\begin{syntax}
tuple-type:
  ( type-specifier , type-list )
\end{syntax}

\begin{syntax}
type-alias-declaration-list:
  type-alias-declaration
  type-alias-declaration , type-alias-declaration-list
\end{syntax}

\begin{syntax}
type-alias-declaration-statement:
  privacy-specifier[OPT] `config'[OPT] `type' type-alias-declaration-list ;
  external-type-alias-declaration-statement
\end{syntax}

\begin{syntax}
type-alias-declaration:
  identifier = type-specifier
  identifier
\end{syntax}

\begin{syntax}
type-binding:
  identifier .
\end{syntax}

\begin{syntax}
type-declaration-statement:
  enum-declaration-statement
  class-declaration-statement
  record-declaration-statement
  union-declaration-statement
  type-alias-declaration-statement
\end{syntax}

\begin{syntax}
type-list:
  type-specifier
  type-specifier , type-list
\end{syntax}

\begin{syntax}
type-part:
  : type-specifier
\end{syntax}

\begin{syntax}
type-specifier:
  primitive-type
  enum-type
  structured-type
  dataparallel-type
  synchronization-type
\end{syntax}

\begin{syntax}
unary-expression:
  unary-operator expression
\end{syntax}

\begin{syntax}
unary-operator: one of
  + $ $ $ $ - $ $ $ $ ~ $ $ $ $ !
\end{syntax}

\begin{syntax}
union-declaration-statement:
  `extern'[OPT] `union' identifier { union-statement-list }
\end{syntax}

\begin{syntax}
union-statement-list:
  union-statement
  union-statement union-statement-list
\end{syntax}

\begin{syntax}
union-statement:
  type-declaration-statement
  procedure-declaration-statement
  iterator-declaration-statement
  variable-declaration-statement
  empty-statement
\end{syntax}

\begin{syntax}
union-type:
  identifier
\end{syntax}

\begin{syntax}
use-statement:
  `use' module-or-enum-name-list ;
\end{syntax}

\begin{syntax}
value-expr:
  expression
\end{syntax}

\begin{syntax}
variable-argument-expression:
  ... expression
  ... ? identifier[OPT]
  ...
\end{syntax}

\begin{syntax}
variable-declaration-list:
  variable-declaration
  variable-declaration , variable-declaration-list
\end{syntax}

\begin{syntax}
variable-declaration-statement:
  privacy-specifier[OPT] config-or-extern[OPT] variable-kind variable-declaration-list ;
\end{syntax}

\begin{syntax}
variable-declaration:
  identifier-list type-part[OPT] initialization-part
  identifier-list type-part no-initialization-part[OPT]
  array-alias-declaration
\end{syntax}

\begin{syntax}
variable-expression:
  identifier
\end{syntax}

\begin{syntax}
variable-kind:
  `param'
  `const'
  `var'
  `ref'
  `const ref'
\end{syntax}

\begin{syntax}
when-statement:
  `when' expression-list `do' statement
  `when' expression-list block-statement
  `otherwise' statement
  `otherwise' `do' statement
\end{syntax}

\begin{syntax}
when-statements:
  when-statement
  when-statement when-statements
\end{syntax}

\begin{syntax}
where-clause:
  `where' expression
\end{syntax}

\begin{syntax}
while-do-statement:
  `while' expression `do' statement
  `while' expression block-statement
\end{syntax}

\begin{syntax}
yield-statement:
  `yield' expression ;
\end{syntax}

\section{Depth-First Lexical Productions}

\begin{syntax}
bool-literal: one of
  `true' $ $ $ $ `false'
\end{syntax}

\begin{syntax}
identifier:
  letter-or-underscore legal-identifier-chars[OPT]
\end{syntax}

\begin{syntax}
letter-or-underscore:
  letter
  `_'
\end{syntax}

\begin{syntax}
letter: one of
  `A' `B' `C' `D' `E' `F' `G' `H' `I' `J' `K' `L' `M' `N' `O' `P' `Q' `R' `S' `T' `U' `V' `W' `X' `Y' `Z'
  `a' `b' `c' `d' `e' `f' `g' `h' `i' `j' `k' `l' `m' `n' `o' `p' `q' `r' `s' `t' `u' `v' `w' `x' `y' `z'
\end{syntax}

\begin{syntax}
legal-identifier-chars:
  legal-identifier-char legal-identifier-chars[OPT]
\end{syntax}

\begin{syntax}
legal-identifier-char:
  letter-or-underscore
  digit
  `(*\texttt{\$}*)'
\end{syntax}

\begin{syntax}
digit: one of
  `0' `1' `2' `3' `4' `5' `6' `7' `8' `9'
\end{syntax}

\begin{syntax}
imaginary-literal:
  real-literal `i'
  integer-literal `i'
\end{syntax}

\begin{syntax}
real-literal:
  digits[OPT] . digits exponent-part[OPT]
  digits .[OPT] exponent-part
  `0x' hexadecimal-digits[OPT] . hexadecimal-digits p-exponent-part[OPT]
  `0X' hexadecimal-digits[OPT] . hexadecimal-digits p-exponent-part[OPT]
  `0x' hexadecimal-digits .[OPT] p-exponent-part
  `0X' hexadecimal-digits .[OPT] p-exponent-part
\end{syntax}

\begin{syntax}
digits:
  digit
  digit digits
\end{syntax}

\begin{syntax}
exponent-part:
  `e' sign[OPT] digits
  `E' sign[OPT] digits
\end{syntax}

\begin{syntax}
sign: one of
  + $ $ $ $ -
\end{syntax}

\begin{syntax}
hexadecimal-digits:
  hexadecimal-digit
  hexadecimal-digit hexadecimal-digits
\end{syntax}

\begin{syntax}
hexadecimal-digit: one of
  `0' `1' `2' `3' `4' `5' `6' `7' `8' `9' `A' `B' `C' `D' `E' `F' `a' `b' `c' `d' `e' `f'
\end{syntax}

\begin{syntax}
p-exponent-part:
  `p' sign[OPT] digits
  `P' sign[OPT] digits
\end{syntax}

\begin{syntax}
integer-literal:
  digits
  `0x' hexadecimal-digits
  `0X' hexadecimal-digits
  `0o' octal-digits
  `0O' octal-digits
  `0b' binary-digits
  `0B' binary-digits
\end{syntax}

\begin{syntax}
octal-digits:
  octal-digit
  octal-digit octal-digits
\end{syntax}

\begin{syntax}
octal-digit: one of
  `0' `1' `2' `3' `4' `5' `6' `7'
\end{syntax}

\begin{syntax}
binary-digits:
  binary-digit
  binary-digit binary-digits
\end{syntax}

\begin{syntax}
binary-digit: one of
  `0' `1'
\end{syntax}

\begin{syntax}
string-literal:
  " double-quote-delimited-characters[OPT] "
  ' single-quote-delimited-characters[OPT] '
\end{syntax}

\begin{syntax}
double-quote-delimited-characters:
  string-character double-quote-delimited-characters[OPT]
  ' double-quote-delimited-characters[OPT]
\end{syntax}

\begin{syntax}
string-character:
  `any character except the double quote, single quote, or new line'
  simple-escape-character
  hexadecimal-escape-character
\end{syntax}

\begin{syntax}
simple-escape-character: one of
  `$\backslash\mbox{\bf '}\hspace{5pt}$' `$\backslash$"$\hspace{5pt}$' `$\backslash$?$\hspace{5pt}$' `$\backslash$$\backslash$$\hspace{5pt}$' `$\backslash$a$\hspace{5pt}$' `$\backslash$b$\hspace{5pt}$' `$\backslash$f$\hspace{5pt}$' `$\backslash$n$\hspace{5pt}$' `$\backslash$r$\hspace{5pt}$' `$\backslash$t$\hspace{5pt}$' `$\backslash$v$\hspace{5pt}$'
\end{syntax}

\begin{syntax}
hexadecimal-escape-character:
  `$\backslash$x' hexadecimal-digits
\end{syntax}

\begin{syntax}
single-quote-delimited-characters:
  string-character single-quote-delimited-characters[OPT]
  " single-quote-delimited-characters[OPT]
\end{syntax}

\section{Depth-First Syntax Productions}

\begin{syntax}
module-declaration-statement:
  privacy-specifier[OPT] `module' module-identifier block-statement
\end{syntax}

\begin{syntax}
privacy-specifier:
  `private'
  `public'
\end{syntax}

\begin{syntax}
module-identifier:
  identifier
\end{syntax}

\begin{syntax}
block-statement:
  { statements[OPT] }
\end{syntax}

\begin{syntax}
statements:
  statement
  statement statements
\end{syntax}

\begin{syntax}
statement:
  block-statement
  expression-statement
  assignment-statement
  swap-statement
  io-statement
  conditional-statement
  select-statement
  while-do-statement
  do-while-statement
  for-statement
  label-statement
  break-statement
  continue-statement
  param-for-statement
  use-statement
  empty-statement
  return-statement
  yield-statement
  module-declaration-statement
  procedure-declaration-statement
  external-procedure-declaration-statement
  exported-procedure-declaration-statement
  iterator-declaration-statement
  method-declaration-statement
  type-declaration-statement
  variable-declaration-statement
  remote-variable-declaration-statement
  on-statement
  cobegin-statement
  coforall-statement
  begin-statement
  sync-statement
  serial-statement
  atomic-statement
  forall-statement
  delete-statement
\end{syntax}

\begin{syntax}
expression-statement:
  variable-expression ;
  member-access-expression ;
  call-expression ;
  constructor-call-expression ;
  let-expression ; 
\end{syntax}

\begin{syntax}
variable-expression:
  identifier
\end{syntax}

\begin{syntax}
member-access-expression:
  field-access-expression
  method-call-expression
\end{syntax}

\begin{syntax}
field-access-expression:
  receiver-clause[OPT] identifier
\end{syntax}

\begin{syntax}
receiver-clause:
  expression .
\end{syntax}

\begin{syntax}
expression:
  literal-expression
  nil-expression
  variable-expression
  enum-constant-expression
  call-expression
  iteratable-call-expression
  member-access-expression
  constructor-call-expression
  query-expression
  cast-expression
  lvalue-expression
  parenthesized-expression
  unary-expression
  binary-expression
  let-expression
  if-expression
  for-expression
  forall-expression
  reduce-expression
  scan-expression
  module-access-expression
  tuple-expression
  tuple-expand-expression
  locale-access-expression
  mapped-domain-expression
\end{syntax}

\begin{syntax}
literal-expression:
  bool-literal
  integer-literal
  real-literal
  imaginary-literal
  string-literal
  range-literal
  domain-literal
  array-literal
\end{syntax}

\begin{syntax}
range-literal:
  expression .. expression
  expression ..
  .. expression
  ..
\end{syntax}

\begin{syntax}
domain-literal:
  rectangular-domain-literal
  associative-domain-literal
\end{syntax}

\begin{syntax}
rectangular-domain-literal:
  { range-expression-list }
\end{syntax}

\begin{syntax}
range-expression-list:
  range-expression
  range-expression, range-expression-list
\end{syntax}

\begin{syntax}
range-expression:
  expression
  strided-range-expression
  counted-range-expression
  aligned-range-expression
  sliced-range-expression
\end{syntax}

\begin{syntax}
strided-range-expression:
  range-expression `by' step-expression
\end{syntax}

\begin{syntax}
step-expression:
  expression
\end{syntax}

\begin{syntax}
counted-range-expression:
  range-expression # expression
\end{syntax}

\begin{syntax}
aligned-range-expression:
  range-expression `align' expression
\end{syntax}

\begin{syntax}
sliced-range-expression:
  range-expression ( range-expression )
  range-expression [ range-expression ]
\end{syntax}

\begin{syntax}
associative-domain-literal:
   { associative-expression-list }
\end{syntax}

\begin{syntax}
associative-expression-list:
   non-range-expression
   non-range-expression, associative-expression-list
\end{syntax}

\begin{syntax}
non-range-expression:
   expression
\end{syntax}

\begin{syntax}
array-literal:
  rectangular-array-literal
  associative-array-literal
\end{syntax}

\begin{syntax}
rectangular-array-literal:
  [ expression-list ]
  [ expression-list , ]
\end{syntax}

\begin{syntax}
expression-list:
  expression
  expression , expression-list
\end{syntax}

\begin{syntax}
associative-array-literal:
  [ associative-expr-list ]
  [ associative-expr-list , ]
\end{syntax}

\begin{syntax}
associative-expr-list:
  index-expr => value-expr
  index-expr => value-expr, associative-expr-list
\end{syntax}

\begin{syntax}
index-expr:
  expression
\end{syntax}

\begin{syntax}
value-expr:
  expression
\end{syntax}

\begin{syntax}
nil-expression:
  `nil'
\end{syntax}

\begin{syntax}
enum-constant-expression:
  enum-type . identifier
\end{syntax}

\begin{syntax}
enum-type:
  identifier
\end{syntax}

\begin{syntax}
iteratable-call-expression:
  call-expression
\end{syntax}

\begin{syntax}
query-expression:
  ? identifier[OPT]
\end{syntax}

\begin{syntax}
cast-expression:
  expression : type-specifier
\end{syntax}

\begin{syntax}
type-specifier:
  primitive-type
  enum-type
  structured-type
  dataparallel-type
  synchronization-type
\end{syntax}

\begin{syntax}
primitive-type:
  `void'
  `bool' primitive-type-parameter-part[OPT]
  `int' primitive-type-parameter-part[OPT]
  `uint' primitive-type-parameter-part[OPT]
  `real' primitive-type-parameter-part[OPT]
  `imag' primitive-type-parameter-part[OPT]
  `complex' primitive-type-parameter-part[OPT]
  `string'
\end{syntax}

\begin{syntax}
primitive-type-parameter-part:
  ( integer-parameter-expression )
\end{syntax}

\begin{syntax}
integer-parameter-expression:
  expression
\end{syntax}

\begin{syntax}
structured-type:
  class-type
  record-type
  union-type
  tuple-type
\end{syntax}

\begin{syntax}
class-type:
  identifier
  identifier ( named-expression-list )
\end{syntax}

\begin{syntax}
named-expression-list:
  named-expression
  named-expression , named-expression-list
\end{syntax}

\begin{syntax}
named-expression:
  expression
  identifier = expression
\end{syntax}

\begin{syntax}
record-type:
  identifier
  identifier ( named-expression-list )
\end{syntax}

\begin{syntax}
union-type:
  identifier
\end{syntax}

\begin{syntax}
tuple-type:
  ( type-specifier , type-list )
\end{syntax}

\begin{syntax}
type-list:
  type-specifier
  type-specifier , type-list
\end{syntax}

\begin{syntax}
dataparallel-type:
  range-type
  domain-type
  mapped-domain-type
  array-type
  index-type
\end{syntax}

\begin{syntax}
range-type:
  `range' ( named-expression-list )
\end{syntax}

\begin{syntax}
domain-type:
  base-domain-type
  simple-subdomain-type
  sparse-subdomain-type
\end{syntax}

\begin{syntax}
base-domain-type:
  rectangular-domain-type
  associative-domain-type
\end{syntax}

\begin{syntax}
rectangular-domain-type:
  `domain' ( named-expression-list )
\end{syntax}

\begin{syntax}
associative-domain-type:
  `domain' ( associative-index-type )
  `domain' ( enum-type )
  `domain' ( `opaque' )
\end{syntax}

\begin{syntax}
associative-index-type:
  type-specifier
\end{syntax}

\begin{syntax}
simple-subdomain-type:
  `subdomain' ( domain-expression )
\end{syntax}

\begin{syntax}
domain-expression:
  domain-literal
  domain-name
  domain-assignment-expression
  domain-striding-expression
  domain-alignment-expression
  domain-slice-expression
\end{syntax}

\begin{syntax}
domain-name:
  identifier
\end{syntax}

\begin{syntax}
domain-assignment-expression:
  domain-name = domain-expression
\end{syntax}

\begin{syntax}
domain-striding-expression:
  domain-expression `by' expression
\end{syntax}

\begin{syntax}
domain-alignment-expression:
  domain-expression `align' expression
\end{syntax}

\begin{syntax}
domain-slice-expression:
  domain-expression [ slicing-index-set ]
  domain-expression ( slicing-index-set )
\end{syntax}

\begin{syntax}
slicing-index-set:
  domain-expression
  range-expression-list
\end{syntax}

\begin{syntax}
sparse-subdomain-type:
  `sparse' `subdomain'[OPT] ( domain-expression )
\end{syntax}

\begin{syntax}
mapped-domain-type:
  domain-type `dmapped' dmap-value
\end{syntax}

\begin{syntax}
dmap-value:
  expression
\end{syntax}

\begin{syntax}
array-type:
  [ domain-expression ] type-specifier
\end{syntax}

\begin{syntax}
index-type:
  `index' ( domain-expression )
\end{syntax}

\begin{syntax}
synchronization-type:
  sync-type
  single-type
  atomic-type
\end{syntax}

\begin{syntax}
sync-type:
  `sync' type-specifier
\end{syntax}

\begin{syntax}
single-type:
  `single' type-specifier
\end{syntax}

\begin{syntax}
atomic-type:
  `atomic' type-specifier
\end{syntax}

\begin{syntax}
lvalue-expression:
  variable-expression
  member-access-expression
  call-expression
  parenthesized-expression
\end{syntax}

\begin{syntax}
parenthesized-expression:
  ( expression )
\end{syntax}

\begin{syntax}
unary-expression:
  unary-operator expression
\end{syntax}

\begin{syntax}
unary-operator: one of
  + $ $ $ $ - $ $ $ $ ~ $ $ $ $ !
\end{syntax}

\begin{syntax}
binary-expression:
  expression binary-operator expression
\end{syntax}

\begin{syntax}
binary-operator: one of
  + $ $ $ $ - $ $ $ $ * $ $ $ $ / $ $ $ $ % $ $ $ $ ** $ $ $ $ & $ $ $ $ | $ $ $ $ ^ $ $ $ $ << $ $ $ $ >> $ $ $ $ && $ $ $ $ || $ $ $ $ == $ $ $ $ != $ $ $ $ <= $ $ $ $ >= $ $ $ $ < $ $ $ $ > $ $ $ $ `by' $ $ $ $ #
\end{syntax}

\begin{syntax}
if-expression:
  `if' expression `then' expression `else' expression
  `if' expression `then' expression
\end{syntax}

\begin{syntax}
for-expression:
  `for' index-var-declaration `in' iteratable-expression `do' expression
  `for' iteratable-expression `do' expression
\end{syntax}

\begin{syntax}
forall-expression:
  `forall' index-var-declaration `in' iteratable-expression task-intent-clause[OPT] `do' expression
  `forall' iteratable-expression task-intent-clause[OPT] `do' expression
  [ index-var-declaration `in' iteratable-expression task-intent-clause[OPT] ] expression
  [ iteratable-expression task-intent-clause[OPT] ] expression
\end{syntax}

\begin{syntax}
index-var-declaration:
  identifier
  tuple-grouped-identifier-list
\end{syntax}

\begin{syntax}
tuple-grouped-identifier-list:
  ( identifier-list )
\end{syntax}

\begin{syntax}
identifier-list:
  identifier
  identifier , identifier-list
  tuple-grouped-identifier-list
  tuple-grouped-identifier-list , identifier-list
\end{syntax}

\begin{syntax}
iteratable-expression:
  expression
  `zip' ( expression-list )
\end{syntax}

\begin{syntax}
task-intent-clause:
  `with' ( task-intent-list )
\end{syntax}

\begin{syntax}
task-intent-list:
  formal-intent identifier
  formal-intent identifier, task-intent-list
\end{syntax}

\begin{syntax}
formal-intent:
  `const'
  `const in'
  `const ref'
  `in'
  `out'
  `inout'
  `ref'
  `param'
  `type'
\end{syntax}

\begin{syntax}
reduce-expression:
  reduce-scan-operator `reduce' iteratable-expression
  class-type `reduce' iteratable-expression
\end{syntax}

\begin{syntax}
reduce-scan-operator: one of
  + $ $ $ $ * $ $ $ $ && $ $ $ $ || $ $ $ $ & $ $ $ $ | $ $ $ $ ^ $ $ $ $ `min' $ $ $ $ `max' $ $ $ $ `minloc' $ $ $ $ `maxloc'
\end{syntax}

\begin{syntax}
scan-expression:
  reduce-scan-operator `scan' iteratable-expression
  class-type `scan' iteratable-expression
\end{syntax}

\begin{syntax}
module-access-expression:
  module-identifier-list . identifier
\end{syntax}

\begin{syntax}
module-identifier-list:
  module-identifier
  module-identifier . module-identifier-list
\end{syntax}

\begin{syntax}
tuple-expression:
  ( tuple-component , )
  ( tuple-component , tuple-component-list )
  ( tuple-component , tuple-component-list , )
\end{syntax}

\begin{syntax}
tuple-component:
  expression
  `_'
\end{syntax}

\begin{syntax}
tuple-component-list:
  tuple-component
  tuple-component , tuple-component-list
\end{syntax}

\begin{syntax}
tuple-expand-expression:
  ( ... expression )
\end{syntax}

\begin{syntax}
locale-access-expression:
  expression . `locale'
\end{syntax}

\begin{syntax}
mapped-domain-expression:
  domain-expression `dmapped' dmap-value
\end{syntax}

\begin{syntax}
method-call-expression:
  receiver-clause[OPT] expression ( named-expression-list )
  receiver-clause[OPT] expression [ named-expression-list ]
  receiver-clause[OPT] parenthesesless-function-identifier
\end{syntax}

\begin{syntax}
parenthesesless-function-identifier:
  identifier
\end{syntax}

\begin{syntax}
call-expression:
  lvalue-expression ( named-expression-list )
  lvalue-expression [ named-expression-list ]
  parenthesesless-function-identifier
\end{syntax}

\begin{syntax}
constructor-call-expression:
  `new' class-name ( argument-list )
\end{syntax}

\begin{syntax}
class-name:
  identifier
\end{syntax}

\begin{syntax}
argument-list:
  ( formals[OPT] )
\end{syntax}

\begin{syntax}
formals:
  formal
  formal , formals
\end{syntax}

\begin{syntax}
formal:
  formal-intent[OPT] identifier formal-type[OPT] default-expression[OPT]
  formal-intent[OPT] identifier formal-type[OPT] variable-argument-expression
  formal-intent[OPT] tuple-grouped-identifier-list formal-type[OPT] default-expression[OPT]
  formal-intent[OPT] tuple-grouped-identifier-list formal-type[OPT] variable-argument-expression
\end{syntax}

\begin{syntax}
default-expression:
  = expression
\end{syntax}

\begin{syntax}
formal-type:
  : type-specifier
  : ? identifier[OPT]
\end{syntax}

\begin{syntax}
variable-argument-expression:
  ... expression
  ... ? identifier[OPT]
  ...
\end{syntax}

\begin{syntax}
let-expression:
  `let' variable-declaration-list `in' expression
\end{syntax}

\begin{syntax}
assignment-statement:
  lvalue-expression assignment-operator expression
\end{syntax}

\begin{syntax}
assignment-operator: one of
   = $ $ $ $ += $ $ $ $ -= $ $ $ $ *= $ $ $ $ /= $ $ $ $ %= $ $ $ $ **= $ $ $ $ &= $ $ $ $ |= $ $ $ $ ^= $ $ $ $ &&= $ $ $ $ ||= $ $ $ $ <<= $ $ $ $ >>=
\end{syntax}

\begin{syntax}
swap-statement:
  lvalue-expression swap-operator lvalue-expression
\end{syntax}

\begin{syntax}
swap-operator:
  <=>
\end{syntax}

\begin{syntax}
io-statement:
  io-expression io-operator expression
\end{syntax}

\begin{syntax}
io-expression:
  expression
  io-expression io-operator expression
\end{syntax}

\begin{syntax}
io-operator:
  <`(*$\sim$*)'>
\end{syntax}

\begin{syntax}
conditional-statement:
  `if' expression `then' statement else-part[OPT]
  `if' expression block-statement else-part[OPT]
\end{syntax}

\begin{syntax}
else-part:
  `else' statement
\end{syntax}

\begin{syntax}
select-statement:
  `select' expression { when-statements }
\end{syntax}

\begin{syntax}
when-statements:
  when-statement
  when-statement when-statements
\end{syntax}

\begin{syntax}
when-statement:
  `when' expression-list `do' statement
  `when' expression-list block-statement
  `otherwise' statement
  `otherwise' `do' statement
\end{syntax}

\begin{syntax}
while-do-statement:
  `while' expression `do' statement
  `while' expression block-statement
\end{syntax}

\begin{syntax}
do-while-statement:
  `do' statement `while' expression ;
\end{syntax}

\begin{syntax}
for-statement:
  `for' index-var-declaration `in' iteratable-expression `do' statement
  `for' index-var-declaration `in' iteratable-expression block-statement
  `for' iteratable-expression `do' statement
  `for' iteratable-expression block-statement
\end{syntax}

\begin{syntax}
label-statement:
  `label' identifier statement
\end{syntax}

\begin{syntax}
break-statement:
  `break' identifier[OPT] ;
\end{syntax}

\begin{syntax}
continue-statement:
  `continue' identifier[OPT] ;
\end{syntax}

\begin{syntax}
param-for-statement:
  `for' `param' identifier `in' param-iteratable-expression `do' statement
  `for' `param' identifier `in' param-iteratable-expression block-statement
\end{syntax}

\begin{syntax}
param-iteratable-expression:
  range-literal
  range-literal `by' integer-literal
\end{syntax}

\begin{syntax}
use-statement:
  `use' module-or-enum-name-list ;
\end{syntax}

\begin{syntax}
module-or-enum-name-list:
  module-or-enum-name limitation-clause[OPT]
  module-or-enum-name , module-or-enum-name-list
\end{syntax}

\begin{syntax}
limitation-clause:
  `except' exclude-list
  `only' rename-list[OPT]
\end{syntax}

\begin{syntax}
exclude-list:
  identifier-list
  $ * $
\end{syntax}

\begin{syntax}
rename-list:
  rename-base
  rename-base , rename-list
\end{syntax}

\begin{syntax}
rename-base:
  identifier `as' identifier
  identifier
\end{syntax}

\begin{syntax}
module-or-enum-name:
  identifier
  identifier . module-or-enum-name
\end{syntax}

\begin{syntax}
empty-statement:
  ;
\end{syntax}

\begin{syntax}
return-statement:
  `return' expression[OPT] ;
\end{syntax}

\begin{syntax}
yield-statement:
  `yield' expression ;
\end{syntax}

\begin{syntax}
module-declaration-statement:
  privacy-specifier[OPT] `module' module-identifier block-statement
\end{syntax}

\begin{syntax}
procedure-declaration-statement:
  privacy-specifier[OPT] linkage-specifier[OPT] `proc' function-name argument-list[OPT] return-intent[OPT] return-type[OPT] where-clause[OPT]
    function-body
\end{syntax}

\begin{syntax}
linkage-specifier:
  `inline'
\end{syntax}

\begin{syntax}
function-name:
  identifier
  operator-name
\end{syntax}

\begin{syntax}
operator-name: one of
  + $ $ $ $ - $ $ $ $ * $ $ $ $ / $ $ $ $ % $ $ $ $ ** $ $ $ $ ! $ $ $ $ == $ $ $ $ != $ $ $ $ <= $ $ $ $ >= $ $ $ $ < $ $ $ $ > $ $ $ $ << $ $ $ $ >> $ $ $ $ & $ $ $ $ | $ $ $ $ ^ $ $ $ $ ~
  += $ $ $ $ -= $ $ $ $ *= $ $ $ $ /= $ $ $ $ %= $ $ $ $ **= $ $ $ $ &= $ $ $ $ |= $ $ $ $ ^= $ $ $ $ <<= $ $ $ $ >>= $ $ $ $ <=> $ $ $ $ <~>
\end{syntax}

\begin{syntax}
return-intent:
  `const'
  `const ref'
  `ref'
  `param'
  `type'
\end{syntax}

\begin{syntax}
return-type:
  : type-specifier
\end{syntax}

\begin{syntax}
where-clause:
  `where' expression
\end{syntax}

\begin{syntax}
function-body:
  block-statement
  return-statement
\end{syntax}

\begin{syntax}
external-procedure-declaration-statement:
  `extern' external-name[OPT] `proc' function-name argument-list return-intent[OPT] return-type[OPT]
\end{syntax}

\begin{syntax}
exported-procedure-declaration-statement:
  `export' external-name[OPT] `proc' function-name argument-list return-intent[OPT] return-type[OPT]
    function-body
\end{syntax}

\begin{syntax}
iterator-declaration-statement:
  privacy-specifier[OPT] `iter' iterator-name argument-list[OPT] return-intent[OPT] return-type[OPT] where-clause[OPT]
  iterator-body
\end{syntax}

\begin{syntax}
iterator-name:
  identifier
\end{syntax}

\begin{syntax}
iterator-body:
  block-statement
  yield-statement
\end{syntax}

\begin{syntax}
method-declaration-statement:
  linkage-specifier[OPT] proc-or-iter this-intent[OPT] type-binding[OPT] function-name argument-list[OPT] 
    return-intent[OPT] return-type[OPT] where-clause[OPT] function-body
\end{syntax}

\begin{syntax}
proc-or-iter:
  `proc'
  `iter'
\end{syntax}

\begin{syntax}
this-intent:
  `param'
  `ref'
  `type'
\end{syntax}

\begin{syntax}
type-binding:
  identifier .
\end{syntax}

\begin{syntax}
type-declaration-statement:
  enum-declaration-statement
  class-declaration-statement
  record-declaration-statement
  union-declaration-statement
  type-alias-declaration-statement
\end{syntax}

\begin{syntax}
enum-declaration-statement:
  `enum' identifier { enum-constant-list }
\end{syntax}

\begin{syntax}
enum-constant-list:
  enum-constant
  enum-constant , enum-constant-list[OPT]
\end{syntax}

\begin{syntax}
enum-constant:
  identifier init-part[OPT]
\end{syntax}

\begin{syntax}
init-part:
  = expression
\end{syntax}

\begin{syntax}
class-declaration-statement:
  simple-class-declaration-statement
  external-class-declaration-statement
\end{syntax}

\begin{syntax}
simple-class-declaration-statement:
  `class' identifier class-inherit-list[OPT] { class-statement-list[OPT] }
\end{syntax}

\begin{syntax}
class-inherit-list:
  : class-type-list
\end{syntax}

\begin{syntax}
class-type-list:
  class-type
  class-type , class-type-list
\end{syntax}

\begin{syntax}
class-statement-list:
  class-statement
  class-statement class-statement-list
\end{syntax}

\begin{syntax}
class-statement:
  variable-declaration-statement
  method-declaration-statement
  type-declaration-statement
  empty-statement
\end{syntax}

\begin{syntax}
external-class-declaration-statement:
  `extern' external-name[OPT] simple-class-declaration-statement
\end{syntax}

\begin{syntax}
external-name:
  identifier
  string-literal
\end{syntax}

\begin{syntax}
record-declaration-statement:
  simple-record-declaration-statement
  external-record-declaration-statement
\end{syntax}

\begin{syntax}
simple-record-declaration-statement:
  `record' identifier record-inherit-list[OPT] { record-statement-list }
\end{syntax}

\begin{syntax}
record-inherit-list:
  : record-type-list
\end{syntax}

\begin{syntax}
record-type-list:
  record-type
  record-type , record-type-list
\end{syntax}

\begin{syntax}
record-statement-list:
  record-statement
  record-statement record-statement-list
\end{syntax}

\begin{syntax}
record-statement:
  variable-declaration-statement
  method-declaration-statement
  type-declaration-statement
  empty-statement
\end{syntax}

\begin{syntax}
external-record-declaration-statement:
  `extern' external-name[OPT] simple-record-declaration-statement
\end{syntax}

\begin{syntax}
union-declaration-statement:
  `extern'[OPT] `union' identifier { union-statement-list }
\end{syntax}

\begin{syntax}
union-statement-list:
  union-statement
  union-statement union-statement-list
\end{syntax}

\begin{syntax}
union-statement:
  type-declaration-statement
  procedure-declaration-statement
  iterator-declaration-statement
  variable-declaration-statement
  empty-statement
\end{syntax}

\begin{syntax}
type-alias-declaration-statement:
  privacy-specifier[OPT] `config'[OPT] `type' type-alias-declaration-list ;
  external-type-alias-declaration-statement
\end{syntax}

\begin{syntax}
type-alias-declaration-list:
  type-alias-declaration
  type-alias-declaration , type-alias-declaration-list
\end{syntax}

\begin{syntax}
type-alias-declaration:
  identifier = type-specifier
  identifier
\end{syntax}

\begin{syntax}
external-type-alias-declaration-statement:
  `extern' `type' type-alias-declaration-list ;
\end{syntax}

\begin{syntax}
variable-declaration-statement:
  privacy-specifier[OPT] config-or-extern[OPT] variable-kind variable-declaration-list ;
\end{syntax}

\begin{syntax}
config-or-extern: one of
  `config' $ $ $ $ `extern'
\end{syntax}

\begin{syntax}
variable-kind:
  `param'
  `const'
  `var'
  `ref'
  `const ref'
\end{syntax}

\begin{syntax}
variable-declaration-list:
  variable-declaration
  variable-declaration , variable-declaration-list
\end{syntax}

\begin{syntax}
variable-declaration:
  identifier-list type-part[OPT] initialization-part
  identifier-list type-part no-initialization-part[OPT]
  array-alias-declaration
\end{syntax}

\begin{syntax}
initialization-part:
  = expression
\end{syntax}

\begin{syntax}
type-part:
  : type-specifier
\end{syntax}

\begin{syntax}
no-initialization-part:
  = `noinit'
\end{syntax}

\begin{syntax}
array-alias-declaration:
  identifier reindexing-expression[OPT] => array-expression ;
\end{syntax}

\begin{syntax}
reindexing-expression:
  : [ domain-expression ]
\end{syntax}

\begin{syntax}
array-expression:
  expression
\end{syntax}

\begin{syntax}
remote-variable-declaration-statement:
  `on' expression variable-declaration-statement
\end{syntax}

\begin{syntax}
on-statement:
  `on' expression `do' statement
  `on' expression block-statement
\end{syntax}

\begin{syntax}
cobegin-statement:
  `cobegin' task-intent-clause[OPT] block-statement
\end{syntax}

\begin{syntax}
coforall-statement:
  `coforall' index-var-declaration `in' iteratable-expression task-intent-clause[OPT] `do' statement
  `coforall' index-var-declaration `in' iteratable-expression task-intent-clause[OPT] block-statement
  `coforall' iteratable-expression task-intent-clause[OPT] `do' statement
  `coforall' iteratable-expression task-intent-clause[OPT] block-statement
\end{syntax}

\begin{syntax}
begin-statement:
  `begin' task-intent-clause[OPT] statement
\end{syntax}

\begin{syntax}
sync-statement:
  `sync' statement
  `sync' block-statement
\end{syntax}

\begin{syntax}
serial-statement:
  `serial' expression[OPT] `do' statement
  `serial' expression[OPT] block-statement
\end{syntax}

\begin{syntax}
atomic-statement:
  `atomic' statement
\end{syntax}

\begin{syntax}
forall-statement:
  `forall' index-var-declaration `in' iteratable-expression task-intent-clause[OPT] `do' statement
  `forall' index-var-declaration `in' iteratable-expression task-intent-clause[OPT] block-statement
  `forall' iteratable-expression task-intent-clause[OPT] `do' statement
  `forall' iteratable-expression task-intent-clause[OPT] block-statement
  [ index-var-declaration `in' iteratable-expression task-intent-clause[OPT] ] statement
  [ iteratable-expression task-intent-clause[OPT] ] statement
\end{syntax}

\begin{syntax}
delete-statement:
  `delete' expression ;
\end{syntax}


\cleardoublepage
\documentclass[10pt,twoside,titlepage]{article}
\usepackage{color}
\usepackage{times}
\usepackage{fullpage}
\usepackage{graphicx}
\usepackage{listings}
\usepackage{longtable}
\usepackage[nottoc]{tocbibind}
\lstdefinelanguage{chapel}
  {
    morekeywords={
      and, array, atomic,
      begin, bool, break,
      call, class, cobegin, complex, config, const, constructor, continue,
      def, distribute, do, domain,
      else, enum, except,
      for, forall,
      goto,
      if, imag, implements, in, int, inout, _invariant, iterator,
      let, like,
      module,
      nil, not,
      on, or, ordered, otherwise, out,
      param, _private, private, public,
      real, record, _release, repeat, return,
      select, serial, single, subtype, sync
      then, to, type, typeselect,
      uint, union, until, _unordered,
      var, _view,
      when, where, while, with,
      yield
    },
    sensitive=false,
    mathescape=false,
    morecomment=[l]{//},
    morecomment=[s]{/*}{*/},
    morestring=[b]",
}

\lstset{
    basicstyle=\footnotesize\tt,
    keywordstyle=\bf,
    commentstyle=\em,
    showstringspaces=false,
    flexiblecolumns=false,
    numbers=left,
    numbersep=5pt,
    numberstyle=\tiny,
    numberblanklines=false,
    stepnumber=0
  }

\newcommand{\chpl}[1]{\lstinline[language=chapel,basicstyle=\normalsize\tt,keywordstyle=]!#1!}

\lstnewenvironment{chapel}{\lstset{language=chapel,xleftmargin=2pc}}{}

\lstdefinelanguage{syntax}
  {
    sensitive=false,
    mathescape=true,
    basewidth = 0.50em,
    fontadjust=true,
    columns=fullflexible,
    basicstyle=\small,
    keywordstyle=\footnotesize\ttfamily,
    commentstyle=\footnotesize\ttfamily,
    identifierstyle=\small\slshape,
    moredelim=[is][\small\bf]{`}{'},
    literate={-}{{\ttfamily -}}{1}
             {[OPT]}{{{\scriptsize $_{opt}$}}}{2}
  }

\lstnewenvironment{syntax}{\lstset{language=syntax,xleftmargin=2pc}}{}

\newcommand{\sntx}[1]{\lstinline[language=syntax]!#1!}


%% High section numbers require different number widths
\usepackage[titles]{tocloft}
\usepackage{ifpdf}
\ifpdf
\usepackage[pdftex,
            bookmarks,
            plainpages=false,
            breaklinks,
            pdftitle={Chapel Language Specification},
            pdfauthor={Cray Inc, 901 Fifth Avenue Suite 1000, Seattle, WA 98164},
            pdfsubject={Chapel High Productivity Language}
           ]{hyperref}
\else
\usepackage[ps2pdf]{hyperref}
\fi
\setlength{\cftsecnumwidth}{1.7em}
\setlength{\cftsubsecnumwidth}{2.6em}
\setlength{\cftsubsubsecnumwidth}{3.4em}
\setlength{\cftsubsecindent}{1.7em}
\setlength{\cftsubsubsecindent}{4.3em}

\newcommand{\ie}{\emph{i.e.}}
\newcommand{\eg}{\emph{e.g.}}

\newenvironment{TODO} {
\begin{quote}
{\it TODO:}
}{
\end{quote}
}

\newenvironment{example}{
\begin{quote}
{\it Example}.
}{
\end{quote}
}

\newenvironment{note}{
\begin{quote}
{\it Implementors' note}.
}{
\end{quote}
}

\newenvironment{rationale}{
\begin{quote}
{\it Rationale}.
}{
\end{quote}
}

\newenvironment{openissue}{
\begin{quote}
{\it Open issue}.
}{
\end{quote}
}

\newenvironment{craychapel}{
\begin{quote}
{\it Cray's Chapel Implementation}.
}{
\end{quote}
}

\newenvironment{suggestionbox}{
\begin{quote}
{\it Suggestions?}
}{
\end{quote}
}

\newcommand{\rsec}[1]
           {\S\ref{#1}}

% courtesy: http://www.iam.ubc.ca/~newbury/tex/page-set-up.html
\newcommand{\sekshun}[1]
           {
             \section{#1}
             \markboth{Chapel Language Specification}{#1}
           }

\oddsidemargin 0.0in
\evensidemargin 0.5in
\textwidth 6in
\headheight 0.2in
\topmargin 0in
\headsep 0.3in
\textheight 8.5in

\makeindex
\title{Chapel Language Specification 0.782}

\author{Cray Inc\\
901 Fifth Avenue, Suite 1000\\
Seattle, WA 98164}

\date{}

\setcounter{tocdepth}{3}

\begin{document}

\pagestyle{empty}
\pagenumbering{alph}

\ifpdf
\pdfbookmark[1]{Title}{titlepage}
\fi
\maketitle

\cleardoublepage
\include{tm}
\cleardoublepage

\pagestyle{myheadings}
\markboth{Chapel Language Specification}{Chapel Language Specification}
\pagenumbering{roman}

\ifpdf
\pdfbookmark[1]{Table of Contents}{tablecontents}
\fi
\tableofcontents

\cleardoublepage

\pagestyle{myheadings}
\pagenumbering{arabic}

\setlength{\parindent}{0in}
\setlength{\parskip}{4mm plus2mm minus1mm}

\sekshun{Scope}
\label{Scope}

Chapel is a new parallel programming language that is under
development at Cray Inc. in the context of the DARPA High Productivity
Language Systems initiative and the DARPA High Productivity Computing
Systems initiative.

This document specifies the Chapel language.  It is a work in progress
and is not definitive.  In particular, it is not a standard.

\cleardoublepage
\sekshun{Notation}
\label{Notation}

Special notations are used in this specification to denote Chapel code
and to denote Chapel syntax.

Chapel code is represented with a fixed-width font where keywords are
bold and comments are italicized.
\begin{example}
\begin{chapel}
for i in D do   // iterate over domain D
  writeln(i);   // output indices in D
\end{chapel}
\end{example}

Chapel syntax is represented with standard syntax notation in which
productions define the syntax of the language.  A production is
defined in terms of non-terminal ({\it italicized}) and terminal
(non-italicized) symbols.  The complete syntax defines all of the
non-terminal symbols in terms of one another and terminal symbols.

A definition of a non-terminal symbol is a multi-line construct.  The
first line shows the name of the non-terminal that is being defined
followed by a colon.  The next lines before an empty line define the
alternative productions to define the non-terminal.
\begin{example}
The production
\begin{syntax_donotcollect}
bool-literal:
  `true'
  `false'
\end{syntax_donotcollect}
defines \sntx{bool-literal} to be either the symbol \sntx{`true'} or
\sntx{`false'}.
\end{example}
In the event that a single line of a definition needs to break across
multiple lines of text, more indentation is used to indicate that it
is a continuation of the same alternative production.

As a short-hand for cases where there are many alternatives that
define one symbol, the first line of the definition of the
non-terminal may be followed by ``one of'' to indicate that the single
line in the production defines alternatives for each symbol.
\begin{example}
The production
\begin{syntax_donotcollect}
unary-operator: one of
  + - ~ !
\end{syntax_donotcollect}
is equivalent to
\begin{syntax_donotcollect}
unary-operator:
  +
  -
  ~
  !
\end{syntax_donotcollect}
\end{example}

As a short-hand to indicate an optional symbol in the definition of a
production, the subscript ``opt'' is suffixed to the symbol.
\begin{example}
The production
\begin{syntax_donotcollect}
formal:
  formal-tag identifier formal-type[OPT] default-expression[OPT]
\end{syntax_donotcollect}
is equivalent to
\begin{syntax_donotcollect}
formal:
  formal-tag identifier formal-type default-expression
  formal-tag identifier formal-type
  formal-tag identifier default-expression
  formal-tag identifier
\end{syntax_donotcollect}
\end{example}

\cleardoublepage
\sekshun{Organization}
\label{Organization}

This specification is organized as follows:
\begin{itemize}

\item
Chapter~\ref{Scope}, Scope, describes the scope of this specification.

\item
Chapter~\ref{Notation}, Notation, introduces the notation that is used
throughout this specification.

\item
Chapter~\ref{Organization}, Organization, describes the contents of
each of the chapters within this specification.

\item
Chapter~\ref{Acknowledgments}, Acknowledgments, offers a note of
thanks to people and projects.

\item
Chapter~\ref{Language_Overview}, Language Overview, describes Chapel
at a high level.

\item
Chapter~\ref{Lexical_Structure}, Lexical Structure, describes the
lexical components of Chapel.

\item
Chapter~\ref{Types}, Types, describes the types in Chapel and defines
the primitive and enumerated types.

\item
Chapter~\ref{Variables}, Variables, describes variables and constants
in Chapel.

\item
Chapter~\ref{Conversions}, Conversions, describes the legal implicit
and explicit conversions allowed between values of different types.
Chapel does not allow for user-defined conversions.

\item
Chapter~\ref{Expressions}, Expressions, describes the non-parallel
expressions in Chapel.

\item
Chapter~\ref{Statements}, Statements, describes the non-parallel
statements in Chapel.

\item
Chapter~\ref{Modules}, Modules, describes modules in Chapel., Chapel
modules allow for name space management.

\item
Chapter~\ref{Functions}, Functions, describes functions and function
resolution in Chapel.

\item
Chapter~\ref{Tuples}, Tuples, describes tuples in Chapel.

\item
Chapter~\ref{Classes}, Classes, describes reference classes in Chapel.

\item
Chapter~\ref{Records}, Records, describes records or value classes in
Chapel.

\item
Chapter~\ref{Unions}, Unions, describes unions in Chapel.

\item
Chapter~\ref{Ranges}, Ranges, describes ranges in Chapel.

\item
Chapter~\ref{Domains}, Domains, describes domains in Chapel.  Chapel
domains are first-class index sets that support the description of
iteration spaces, array sizes and shapes, and sets of indices.

\item
Chapter~\ref{Arrays}, Arrays, describes arrays in Chapel.  Chapel arrays are
more general than in most languages including support for
multidimensional, sparse, associative, and unstructured arrays.

\item
Chapter~\ref{Iterators}, Iterators, describes iterator functions.

\item
Chapter~\ref{Generics}, Generics, describes Chapel's support for
generic functions and types.

\item
Chapter~\ref{Input_and_Output}, Input and Output, describes support
for input and output in Chapel, including file input and output..

\item
Chapter~\ref{Task_Parallelism_and_Synchronization}, Task Parallelism
and Synchronization, describes task-parallel expressions and
statements in Chapel as well as synchronization constructs and the atomic
statement.

\item
Chapter~\ref{Data_Parallelism}, Data Parallelism, describes
data-parallel expressions and statements in Chapel including
reductions and scans, whole array assignment, and promotion.

\item
Chapter~\ref{Locales}, Locales, describes constructs for managing
locality and executing Chapel programs on distributed-memory systems.

\item
Chapter~\ref{Domain_Maps}, Domain Maps, describes
Chapel's \emph{domain map} construct for defining the layout of
domains and arrays within a single locale and/or the distribution of
domains and arrays across multiple locales.

\item
Chapter~\ref{User_Defined_Reductions_and_Scans}, User-Defined
Reductions and Scans, describes how Chapel programmers can define
their own reduction and scan operators.

\item
Chapter~\ref{User_Defined_Domain_Maps}, User-Defined Domain Maps,
describes how Chapel programmers can define their own domain maps to
implement domains and arrays.

\item
  Chapter~\ref{Memory_Consistency_Model}, Memory Consistency Model,
  describes Chapel's rules for ordering the reads and writes performed
  by a program's tasks.

\item
Chapter~\ref{Interoperability} describes Chapel's interoperability
features for combining Chapel programs with code written in different
languages.

\item
Chapter~\ref{Standard_Modules}, Standard Modules, describes the
standard modules that are provided with the Chapel language.

\item
Chapter~\ref{Standard_Distributions}, Standard Distributions,
describes the standard distributions (multi-locale domain maps) that
are provided with the Chapel language.

\item
Chapter~\ref{Standard_Layouts}, Standard Layouts, describes the
standard layouts (single locale domain maps) that are provided with
the Chapel language.

\item
Appendix~\ref{Syntax}, Collected Lexical and Syntax Productions,
contains the syntax productions listed throughout this specification
in both alphabetical and depth-first order.

\end{itemize}

\cleardoublepage
This is a stub.  This portion of the document does not exist.

\cleardoublepage
\sekshun{Language Overview}
\label{Language_Overview}

In HPC applications, the current dominant parallel programming paradigm 
is characterized by a localized
view of the computation combined with explicit control
over message passing, as exemplified by a combination
of Fortran or C/C++ with MPI. Such a fragmented memory
model provides the programmer with full control over data
distribution and communication, at the expense of productivity,
conciseness, and clarity.

Chapel is a new parallel programming language that 
strives to improve the programmability of parallel computer systems.
It provides a higher level of expression 
than current parallel languages do and it improves the separation between 
algorithmic expression and data structure implementation details. 

Chapel supports a global-view parallel programming model at a high level by 
supporting abstractions for data parallelism, task parallelism, and nested parallelism. 
It supports optimization for the locality of data and computation in the program 
via abstractions for data distribution and data-driven placement of subcomputations. 
It supports code reuse and generality via object-oriented concepts and generic 
programming features. While Chapel borrows concepts from many preceding languages, 
its parallel concepts are most closely based on ideas from High-Performance Fortran 
(HPF), ZPL, and the Cray MTA's extensions to Fortran/C. 

The key features of the Chapel language for productive parallel programming are as 
follows: 
\begin{itemize}
\item {\bf Locale type} - an opaque type used for organizing and referring to 
units of machine locality.
\item {\bf Domains} - first-class index sets that can potentially be distributed 
between multiple locales.  Domains are Chapel's primary vehicle for global-view 
data parallelism.
\item {\bf Arrays} - generalized support for distributed data aggregates, including 
dynamic multidimensional rectilinear arrays, potentially strided and/or sparse 
in each dimension; associative arrays; set- and graph-based arrays.
\item {\bf User-defined distributions} - a capability for users to specify the 
low-level distributed implementation of domains and arrays orthogonally 
to the computations that operate on these concepts.
\item {\bf \chpl{forall} loops and iterators} - concepts for specifying parallel 
iteration in a manner that separates algorithm and implementation.
\item {\bf Index types} - types representing domain indices to support code 
clarity and bounds-checking optimizations.    
\item {\bf User-defined reductions and scans} - a framework for expressing parallel 
prefix operations over data aggregates cleanly and efficiently.
\item {\bf \chpl{cobegin} and \chpl{begin} statements} - statement types for 
supporting task-parallel computations.
\item {\bf Sync and single-assignment variables} - variable types that support 
synchronization between parallel tasks.
\item {\bf Atomic sections} - compound statements that support atomic execution 
from the perspective of other threads.
\item {\bf \chpl{On} clauses} - specifications that support explicit placement of 
data values and computation on the machine's locales.
\item {\bf Value and reference classes} - object-oriented software containers 
that support encapsulation of state and the separation of interfaces from 
implementations.
\item {\bf Function and operator overloading, multiple dispatch, pass-by-argument name, 
default argument values} - concepts that support modern and productive function call 
capabilities.
\item {\bf Type variables and latent types} - capabilities for writing algorithms 
independently of types to support code reuse, exploratory programming, and 
generic functions and data structures.
\item {\bf Modules} - software containers for namespace management and programming 
in-the-large.
\item {\bf Other features for productive programming} - tuples, type-safe unions, 
sequences, etc.
\end{itemize}

\subsection{Motivating Principles}
\label{Motivating_Principles}

Chapel pushes the state-of-the-art in parallel programming
by focusing on productivity and not just performance. In particular
Chapel combines the goal of highest possible
object code performance with that of programmability
by supporting a high level interface resulting in
shorter time-to-solution and reduced application development
cost. The design of Chapel is guided by four key
areas of programming language technology: global-view programming,
locality-awareness, object-orientation, and generic
programming.

\subsubsection{Global View Programming Model}
\label{Global_View_Programming_Model}

Parallel programming models can be divided into two types of models:
{\em fragmented} and {\em global-view}.  Fragmented programming models
require programmers to express algorithms on a task-by-task basis so that
the tasks can execute in parallel.  Global-view programming models
allow programmers to express a parallel algorithm as a whole, similar to
a serial algorithm.  The compiler and runtime libraries identify and assign
tasks to run in parallel across processors.  Chapel provides a global-view 
programming model.

The global-view programming model is is able:
\begin{itemize}
\item to operate on distributed data structures monolithically as 
though they were local to the executing thread's memory, and 
\item to express parallelism within a single source text without requiring 
multiple executables to be run simultaneously. 
\end{itemize}
While the global-view programming model can be implemented on any distributed memory 
machine, specific architectures provide an ideal target for this model.  
Global-view models map particularly well to architectures that 
support a global address space, DGAS and PGAS memory segments, a high performance 
network, lightweight synchronization, and latency-tolerant processors.  This 
synergy results in improved performance as compared to implementations on less-
productive architectures.

We believe that the dominance of the fragmented programming model is the primary
inhibitor of parallel programmability today, and therefore recommend that new
productivity-oriented languages focus on supporting a global view of parallel
programming.  





\subsubsection{Locality Aware Programming}
\label{Locality_Aware_Programming}

Locality-aware programming, in the style of HPF and
ZPL, provides distribution of shared data structures without
requiring a fragmentation of control structure. The programmer
reasons about load-balance and locality by specifying
the placement of data objects and threads.

\subsubsection{Object-Oriented Programming}
\label{Object-Oriented_Programming}

Object-oriented programming helps in managing complexity
by separating common function from specific implementation
to facilitate reuse.

\subsubsection{Generic Programming}
\label{Generic_Programming}

Generic programming and type-inference simplify the
type systems presented to users. High-performance computing
requires type systems to provide data structure details
that allow for efficient implementation. Generic programming
avoids the need for explicit specification of such
details when they can be inferred from the source or from
specialization of program templates.

\subsection{Basic Language Features}
\label{Basic_Language_Features}

Chapel is an imperative programming language.  The basic concepts of the
language should be familiar to users of C, Fortran, Java, Modula and Ada.
However, the syntax of the Chapel language does not directly build upon any of 
these existing languages.   Programmers should start afresh when programming in 
Chapel and not be limited to the constructs of existing languages.

\subsubsection{Getting Started}

Consider the classic first program.  The program file,
\chpl{helloworld.chpl} contains:
\begin{chapel}
def main() {
  writeln("hello, world");
}
\end{chapel}

The syntax of this simple program somewhat resembles C.  There are a
few items to note.
The program contains one module, \chpl{helloworld}, which is implicitly
named from the name of the file.  The keyword \chpl{def} indicates that the
definition of a function follows.

To compile and run this program, execute the following
commands at the system prompt:
\begin{verbatim}
> chpl helloworld.chpl
> ./a.out
\end{verbatim}
The following output is shown:
\begin{verbatim}
> hello, world
\end{verbatim}

\subsubsection{Programs and Modules}
\label{Programs_and_Modules}

All Chapel code is organized using \emph{modules} which serve as code
containers to help manage code complexity as programs grow in size.
One module may ``\chpl{use}'' another, giving it access to that
module's public global symbols.  In the following example, the
standard Chapel module \chpl{Types} is used.  This module contains
the \chpl{numBits} function that returns the size of a Chapel numeric
type in bits. 
%% test:  modulesexample.chpl
\begin{chapel}
use Types;

writeln("The default size of Chapel integers is ",numBits(int)," bits.");
\end{chapel}
The output of this example is:
\begin{verbatim}
The default size of Chapel integers is 32 bits.
\end{verbatim}

For convenience in exploratory programming,
explicit module declarations are not required.  If code is specified 
without a module declaration, the code's
filename is used as the module name for the code that it contains.

All Chapel programs must define a single subroutine named
\fnname{main()} that specifies the entry point for the program.  
This entry point is executed by a single logical thread.

Chapel provides standard modules for bit level operations, computing
random numbers and quering the system time.  See~\rsec{Standard_Modules}
for more details about these modules and how to use them.

\subsubsection{Data Types and Variables}
\label{Data_Types_and_Variables}

The following example demonstrates some variable declarations in Chapel.
\begin{chapel}
config const n = 10;

var x = 1.0,
    y = n:real,
    z: real;
\end{chapel}

The constant \chpl{n} can be set at runtime, as indicated by \chpl{config}, 
or it is set to its default value of \chpl{10}.  It is inferred to be of 
type \chpl{int} from this integer default value.  Similarly, \chpl{x} and 
\chpl{y} are inferred to be of type \chpl{real}.  The variable \chpl{x}
is initialized to \chpl{1.0} and the variable \chpl{y} is initialized to
the value of \chpl{n}, converted from an integer value to a real value.
The variable \chpl{z} has an explicit type declaration.  
Because \chpl{z} is not initialized, it has a default intial value of \chpl{0.0}.

Variable declarations in Chapel include the kind of variable, the variable's
name, type and initial value.  A variable's initialization may be omitted, 
in which case it will be initialized to an value dependent on its definition for 
safety (\eg, ``zero'' for numerical types).  Alternatively, a variable's type 
may be omitted, in which case it will be inferred from its
initializer.

There are three kinds of variables in Chapel specified by the following 
keywords:  \chpl{var}, \chpl{const},
and \chpl{param}.  The optional keyword \chpl{config} may precede
any of these variable keywords.  The \chpl{var} keyword indicates that a
variable is truly ``variable'' and may be modified throughout its
lifetime.  The \chpl{const} keyword indicates that a variable is a
constant, meaning that it \emph{must} be initialized and that its
value cannot change during its lifetime.  Unlike many languages,
Chapel's constant initializers need not be known at compile-time.  The
\chpl{param} keyword is used to define a \emph{parameter}, which is a
compile-time constant.  Parameter values are
required in certain language contexts, such as when specifying a
scalar type's bit-width or an array's rank.  In other contexts,
parameter values can be used to assert to the compiler that a
variable's value is known and unchanging.

Labeling a variable declaration with the optional \chpl{config}
keyword allows its value to be specified on the command line of the
compiler-generated executable (for \chpl{config const} and
\chpl{config var} declarations), or on the command-line of the
compiler itself (for \chpl{config param} declarations).

Variable declarations may also be specified in a variety of
comma-separated ways which allow multiple variables to share the same
variable-kind, type definition or initializer.  

Chapel has support for boolean, integer and floating point primitive types,
including the support for unsigned integers and complex types.  There is
also support for strings as primitive types.  The following table
lists the set of primitive types.  For each type, the default size and
all possible sizes are given.

\begin{center}
\begin{tabular}{|l|l|l|}
\hline
{\bf Type} & {\bf Default Size} & {\bf Supported Sizes}  \\
\hline
\begin{chapel}
int
\end{chapel}
& 32 bits & 
\begin{chapel}
int(8)
int(16) 
int(32) 
int(64) 
\end{chapel} \\
\hline
\begin{chapel}
uint
\end{chapel}
& 32 bits & 
\begin{chapel}
uint(8)
uint(16) 
uint(32) 
uint(64) 
\end{chapel} \\
\hline
\begin{chapel}
real
\end{chapel}
& 64 bits & 
\begin{chapel}
real(32)
real(64)
real(128)
\end{chapel} \\
\hline
\begin{chapel}
imag
\end{chapel} 
& 64 bits & 
\begin{chapel}
imag(32) 
imag(64)
imag(128)
\end{chapel} \\
\hline
\begin{chapel}
complex
\end{chapel}
& 128 bits & 
\begin{chapel}
complex(64)
complex(128)
complex(256)
\end{chapel} \\
\hline
\begin{chapel}
bool
\end{chapel} 
& 1 bit & 
\begin{chapel}
bool
\end{chapel}  \\
\hline
\begin{chapel}
string
\end{chapel}
& unbounded & 
\begin{chapel}
string 
\end{chapel} \\
\hline
\end{tabular}
\end{center}

%% Should locale be listed in the above table?

Chapel is a type-safe language.  When assigning from one type to another, explicit 
casts are often required by the compiler.  Details of implicit and explicit
conversions are discussed in~\rsec{Conversions}.

%% Not sure how much to say about coercions and conversions here.

Beyond these primitive types, there is support for enumerated types, tuples and
unions.  Additionally, arrays, domains, sequences, classes and records are
used in variable declarations as type definitions.

Chapel supports the ability to created named type definitions using
the \chpl{type} keyword.  Like parameter variables, type definitions
must be known at compile-time.  The example below demonstrates a use of a type
definition.

\begin{chapel}
type elemType = real(32);
var alpha: elemType;
\end{chapel}

The first line of the example code defines the identifier 
\typename{elemType} to be
an alias for a \chpl{real(32)}---Chapel's 32-bit floating point type.
The identifier \typename{elemType} may be used to specify a variable's
definition or anywhere else that a type is allowed.  

\subsubsection{Statements and Expressions}
\label{Statements_and_Expressions}

Examples of Chapel statements are given in the following table.

\begin{center}
\begin{tabular}{|l|l|}
\hline
{\bf Statement} & {\bf Example} \\
\hline
Block Statement &
\begin{chapel} % test:  block.chpl
var tau, s, c: real;
const a = 2.0, b = 5.5;
const b = 5.5;
{
tau = -a/b;
s = 1/sqrt(1 + tau*tau);
c = s*tau;
}
writeln("Givens rotation = ", s, " ", c);
\end{chapel} \\
\hline
Expression Statement & 
\begin{chapel} % test:  expstmt.chpl
var denom = 1.0;
var x: real;

testForZero(denom);
testForZero(x);
testForZero(0.0);

def testForZero(x: real) {
  if (x == 0.0) then halt("Value is zero.");
  else writeln("Non-zero value.  Continuing.");   
}
\end{chapel} \\
\hline
Assignment Statement & 
\begin{chapel} % test: assign.chpl
var i: int;

i = 0;
i = i + 1;
i += 1;
writeln(i);
\end{chapel} \\
\hline
Conditional Statement &
\begin{chapel} % test:  cond.chpl
const D = [1..5];
var x, y: [D] real;
var alpha = 2.0;

[i in D] y(i) = 3.0*i;
scale(x, y, alpha);
writeln(x);

def scale(x, y, alpha: real) {
  if (x.numElements != y.numElements) {
    writeln("Error:  Input vectors are not the same length.");
    return;
  }
  if (alpha == 0.0) {
    x = 0.0;
  } else if (alpha == 1.0) {
    x = y;
  } else {
    x = alpha*y;
  }
}
\end{chapel} \\
\hline
Select Statement &
\begin{chapel} % test:  select.chpl
const D = [1..5];
var A: [D] real;

[i in D] A(i) = i;

writeln(getvalue("first",A));
writeln(getvalue("last",A));
writeln(getvalue("middle",A));

def getvalue(pos:string,y) {
  var x = 0.0;
  select pos {
    when "first" do x = y(1);
    when "last" do x = y(y.numElements);
    when "middle" do x = y((y.numElements/2):int + y.numElements%2);
    otherwise writeln("Unrecognized element position");
  }
  return x;
}
\end{chapel} \\
\hline
\end{tabular}

\begin{tabular}{|l|l|}
\hline
{\bf Statement} & {\bf Example} \\
\hline
While and Do While Loops &
\begin{chapel} % test: while.chpl
var t = 11;

writeln("Scope of do while loop:");
do {
  t += 1;
  writeln(t);
} while (t <= 10);

t = 11;
writeln("Scope of while loop:");
while (t <= 10) {
  t += 1;
  writeln(t);
}
\end{chapel} \\
\hline
For Loop &
\begin{chapel} % test: for.chpl
const D = [1..5];
var A: [D] real;

[i in D] A(i) = -i*i;
writeln(norm1(A));

def norm1(x) {
  var norm = 0.0;
  for i in x.domain {
    norm += abs(x(i));
  }
  return norm;
}
\end{chapel} \\
\hline
Use Statement &
\begin{chapel} % test:  use.chpl
use Time;
var programTimer:Timer;

programTimer.start();
writeln("Write one line.");
programTimer.stop();
writeln(programTimer.accumulated);
\end{chapel} \\
\hline
Type Select Statement &
\begin{chapel} % test:  typeselect.chpl
var x = 32, y = 15.5;
var z: int(8);
var coord = (0.0,0.0);
var yes: bool;

writetype(x);
writetype(y);
writetype(z);
writetype(coord);
writetype(yes);
writetype("no");

def writetype(x) {
  type select x {
    when int do writeln("Integer type");
    when uint do writeln("Unsigned integer type");
    when real do writeln("Real type");
    when complex do writeln("Complex type");
    when string do writeln("String type");
    when bool do writeln("Boolean type");
    otherwise writeln("Non-primitive type");
  }
}
\end{chapel} \\
\hline
Empty Statement &
\begin{chapel}
;
\end{chapel} \\
\hline
\end{tabular}
\end{center}

\begin{center}
\begin{tabular}{|l|l|}
\hline
{\bf Expression} & {\bf Example} \\
\hline
Query Expression &
\begin{chapel} % query.chpl
writeln(sumOfThree(1,2,3));
writeln(sumOfThree(4.0,5.0,3.0));

def sumOfThree(x: ?t, y:t, z:t):t {
   var sum: t;

   sum = x + y + z;
   return sum;
}
\end{chapel} \\
\hline
Casts &
\begin{chapel} % casts.chpl
var x, y: complex;
x = 2.56 + 9.0i;
y = (3.12, 8.7): complex;
var z = (4.2, 6.1);

writeln(x);
writeln(y);
writeln(z);

var m = 2: int(64);
var n = 2;
var i = 1;
var j = 1;

while (n > 0) do {
  n *= 2;
  i += 1;
}
while (m > 0) do {
  m *= 2;
  j += 1;
}

writeln("For 32-bit integers, 2 ** (",i,") overflows.");
writeln("For 64-bit integers, 2 ** (",j,") overflows.");
\end{chapel} \\
\hline
Let Expression &
\begin{chapel} % let.chpl
quadsol(3.0,8.0,5.0);
quadsol(3.0,4.0,5.0);

def quadsol(a:real, b:real, c:real) {
  writeln("The solution of ",a,"x^2 + ",b,"x + ",c," = 0 is:");
  if (b*b > 4.0*a*c) {
    var x:  (real, real);

    x = let temp1 = sqrt(b*b - 4.0*a*c), temp2 = 2.0*a in
        ((-b + temp1)/temp2, (-b - temp1)/temp2);

    writeln(x);
  } else {
    var x: (complex, complex);

    x = let temp1 = sqrt(4.0*a*c - b*b)/(2.0*a), temp2 = -b/(2.0*a) in
        ((temp2,temp1):complex,(temp2,-temp1):complex);

    writeln(x);
  }
}
\end{chapel} \\
\hline
Conditional Expression &
\begin{chapel} % condexp.chpl
writehalf(8);
writehalf(21);
writehalf(1000);

def writehalf(i: int) {
  var half = if (i % 2) then i/2 +1 else i/2;
  writeln("Half of ",i," is ",half);
}
\end{chapel} \\
\hline
\end{tabular}
\end{center}

\subsubsection{Structured Data Types}
\label{Structured_Data_Types}

\subsubsection{Functions and Methods}
\label{Functions_and_Methods}

\subsubsection{Sequences and Iterators}
\label{Sequences_and_Iterators}


\subsubsection{Arrays and Domains}
\label{Arrays_and_Domains}

In Chapel, arrays are reference types that are declared using domains.
A domain is a first-class representation of an index space, potentially 
defined to be distributed across multiple locales.   All arrays
declared with a particular domain are indexed and distributed according 
to that domain's specifications.  

The following example shows three arrays that are declared to be
vectors of length \chpl{m} and then used to compute and store a 
scaled addition.
\begin{chapel}
const VectorD: domain(1) = [1..m];
var A, B, C: [VectorD] real;

A = B + alpha * C;
\end{chapel}
The first line declares a constant named \chpl{VectorD} that
is defined to be a \chpl{domain} that is  
1-dimensional, describing indices $\{ 1, 2, \ldots, m \}$.
The next line uses the \chpl{VectorD} domain to declare three
arrays \chpl{A}, \chpl{B}, and \chpl{C} of type \chpl{real}.  The
domain's index set defines the size and shape of these arrays. 

The final line uses whole-array syntax to specify the elementwise 
multiplications, additions, and assignments.  In this case since
all three arrays are declared with the same domain, the compiler
knows that the arrays are the right shape and size to successfully
compute the array addition and can generate the appropriate elementwise
additions.

Whole-array operations like this one are implicitly parallel, if the \chpl{VectorD}
domain were distributed across a set of processors.  For example, 
a block distribution of \chpl{VectorD} would be specified as follows.    
\begin{chapel}
const VectorD: domain(1) distributed(Block) = [1..m];
\end{chapel}
Each processor would perform the operations for the array elements that it owns, 
as defined by \chpl{VectorD}'s distribution since that was the domain
used to define all three arrays.

Arrays may be multi-dimensional if declared with multi-dimensional domains,
and they may be of any type.  
%% Need to qualify previous statement.
Since arrays are reference types, they are passed by reference to functions
where they may be modified and remain modified upon return.  However, assigning
from one array to another merely copies the values from one to the other.  The
two arrays each continue to reference individual arrays. 
%% Is previous statement correct?

The above example uses arithmetic domains and arrays.  Domains may
also be sparse, indefinite, enumerated or opaque.  Subdomains may be
defined to specify a subset of the domain's indices, as in the case of
inner, non-boundary points of a grid.  See~\rsec{Domains_and_Arrays} for
a complete description of domains and arrays.
 
\subsection{Parallel Features}
\label{Parallel_Features}


\subsubsection{Data Parallel Constructs}
\label{Data_Parallel_Constructs}


\subsubsection{Task Parallel Constructs}
\label{Task_Parallel_Constructs}


\subsubsection{Exploiting Data Locality}
\label{Exploiting_Data_Locality}


\subsubsection{Synchronizing and Serializing Tasks}
\label{Synchronizing_and_Serializing_Tasks}


\subsection{Data Distributions}
\label{Data_Distributions}


\cleardoublepage
\sekshun{Lexical Structure}
\label{Lexical_Structure}

This is a stub.  This portion of the document does not exist.

\subsection{Programs}
\label{Programs}

This is a stub.  This portion of the document does not exist.

\subsection{Comments}
\label{Comments}

This is a stub.  This portion of the document does not exist.

\subsection{White Space}
\label{White_Space}

This is a stub.  This portion of the document does not exist.

\subsection{Case Sensitivity}
\label{Case_Sensitivity}

This is a stub.  This portion of the document does not exist.

\subsection{Tokens}
\label{Tokens}

This is a stub.  This portion of the document does not exist.

\subsubsection{Identifiers}
\label{Identifiers}

This is a stub.  This portion of the document does not exist.

\subsubsection{Keywords}
\label{Keywords}

This is a stub.  This portion of the document does not exist.

\subsubsection{Literals}
\label{Literals}

This is a stub.  This portion of the document does not exist.

\subsubsection{Operators and Punctuation}
\label{Operators_and_Punctuation}

This is a stub.  This portion of the document does not exist.

\subsubsection{Grouping Tokens}
\label{Grouping_Tokens}

This is a stub.  This portion of the document does not exist.

\subsection{Compile-Time Conditionals}
\label{Compile-Time_Conditionals}

This is a stub.  This portion of the document does not exist.

\subsection{User-Defined Compiler Errors}
\label{User-Defined_Compiler_Errors}

This is a stub.  This portion of the document does not exist.

\cleardoublepage
\sekshun{Types}
\label{Types}

Chapel is a statically typed language with a rich set of types.  These
include a set of predefined primitive types, enumerated types,
locale types, structured types (classes, records, unions, tuples),
data parallel types (ranges, domains, arrays), and synchronization
types (sync, single).

% This section defines the primitive
% types, enumerated types, and type aliases.  

The syntax of a type is as follows:

\begin{syntax}
type-specifier:
  primitive-type
  enum-type
  locale-type
  structured-type
  dataparallel-type
  synchronization-type
\end{syntax}

Programmers can define their own enumerated types, classes, records,
unions, and type aliases using type declaration statements:

\begin{syntax}
type-declaration-statement:
  enum-declaration-statement
  class-declaration-statement
  record-declaration-statement
  union-declaration-statement
  type-alias-declaration-statement
\end{syntax}

These statements are defined in Sections \rsec{Enumerated_Types},
\rsec{Class_Declarations}, \rsec{Record_Declarations},
\rsec{Union_Declarations}, and \rsec{Type_Aliases}, respectively.

\section{Primitive Types}
\label{Primitive_Types}
\index{types!primitive}

The primitive types include the following types: \chpl{void}, chpl{bool},
\chpl{int}, \chpl{uint}, \chpl{real}, \chpl{imag}, \chpl{complex},
\chpl{string}, and \chpl{locale}.  These primitive types are defined
in this section.

The primitive types are summarized by the following syntax:
\begin{syntax}
primitive-type:
  `void'
  `bool' primitive-type-parameter-part[OPT]
  `int' primitive-type-parameter-part[OPT]
  `uint' primitive-type-parameter-part[OPT]
  `real' primitive-type-parameter-part[OPT]
  `imag' primitive-type-parameter-part[OPT]
  `complex' primitive-type-parameter-part[OPT]
  `string'

primitive-type-parameter-part:
  ( integer-parameter-expression )

integer-parameter-expression:
  expression
\end{syntax}

If present, the parenthesized \sntx{integer-parameter-expression} must
evaluate to a compile-time constant of integer type.  See~\rsec{Compile-Time_Constants}

\begin{openissue}
There is an expectation of future support for larger bit width
primitive types depending on a platform's native support for those
types.
\end{openissue}

\subsection{The Void Type}
\label{The_Void_Type}
\index{void@\chpl{void}}

The \chpl{void} type is used to represent the lack of a value, for
example when a function has no arguments and/or no return type.  

There may be storage associated with a value of type \chpl{void}, in which
case its lifetime obeys the same rules as a value of type \chpl{int}.

\subsection{The Bool Type}
\label{The_Bool_Type}
\index{bool@\chpl{bool}}

Chapel defines a logical data type designated by the symbol
\chpl{bool} with the two predefined values \chpl{true} and
\chpl{false}.  This default boolean type is stored using an
implementation-defined number of bits.  A particular number of bits
can be specified using a parameter value following the \chpl{bool}
keyword, such as \chpl{bool(8)} to request an 8-bit boolean value.
Legal sizes are 8, 16, 32, and 64 bits.

%% The relational operators return values of \chpl{bool} type and the
%% logical operators operate on values of \chpl{bool} type.

Some statements require expressions of \chpl{bool} type and Chapel
supports a special conversion of values to \chpl{bool} type when used
in this context~(\rsec{Implicit_Statement_Bool_Conversions}).

\subsection{Signed and Unsigned Integral Types}
\label{Signed_and_Unsigned_Integral_Types}
\index{uint@\chpl{uint}}
\index{int@\chpl{int}}

The integral types can be parameterized by the number of bits used to
represent them.  Valid bit-sizes are 8, 16, 32, and 64.  
The default signed integral type, \chpl{int}, and the
default unsigned integral type, \chpl{uint}, are 32 bits.

The integral types and their ranges are given in the following table:

\begin{center}
\begin{tabular}{|l|r|r|}
\hline
{\bf Type} & {\bf Minimum Value} & {\bf Maximum Value} \\
\hline
{\tt int(8)} & -128 & 127 \\
{\tt uint(8)} & 0 & 255 \\
{\tt int(16)} & -32768 & 32767 \\
{\tt uint(16)} & 0 & 65535 \\
{\tt int(32)}, {\tt int} & -2147483648 & 2147483647 \\
{\tt uint(32)}, {\tt uint} & 0 & 4294967295 \\
{\tt int(64)} & -9223372036854775808 & 9223372036854775807 \\
{\tt uint(64)} & 0 & 18446744073709551615 \\
\hline
\end{tabular}
\end{center}

The unary and binary operators that are pre-defined over the integral
types operate with 32- and 64-bit precision.  Using these operators on
integral types represented with fewer bits results in a coercion
according to the rules defined in~\rsec{Implicit_Conversions}.

\begin{openissue}
There is on going discussion on whether the default size of the
integral types should be changed to 64 bits.
\end{openissue}


\subsection{Real Types}
\label{Real_Types}
\index{real@\chpl{real}}

Like the integral types, the real types can be parameterized by the
number of bits used to represent them.  The default real
type, \chpl{real}, is 64 bits.  The real types that are supported are
machine-dependent, but usually include \chpl{real(32)} (single
precision) and \chpl{real(64)} (double precision) following the IEEE
754 standard.  

\subsection{Imaginary Types}
\label{Imaginary_Types}
\index{imaginary@\chpl{imaginary}}

The imaginary types can be parameterized by the number of bits used to
represent them.  The default imaginary type, \chpl{imag}, is 64 bits.
The imaginary types that are supported are machine-dependent, but
usually include \chpl{imag(32)} and \chpl{imag(64)}.

\begin{rationale}
The imaginary type is included to avoid numeric instabilities and
under-optimized code stemming from always coercing real values to
complex values with a zero imaginary part.
\end{rationale}

\subsection{Complex Types}
\label{Complex_Types}
\index{complex@\chpl{complex}}

Like the integral and real types, the complex types can be
parameterized by the number of bits used to represent them.  A complex
number is composed of two real numbers so the number of bits used to
represent a complex is twice the number of bits used to represent the
real numbers.  The default complex type, \chpl{complex}, is 128 bits;
it consists of two 64-bit real numbers.  The complex types that are
supported are machine-dependent, but usually
include \chpl{complex(64)} and \chpl{complex(128)}.

The real and imaginary components can be accessed via the methods
\chpl{re} and \chpl{im}.  The type of these components is real.
See~\rsec{Math} for math routines for complex types.

\begin{example}
Given a complex number \chpl{c} with the value \chpl{3.14+2.72i}, the
expressions \chpl{c.re} and \chpl{c.im} refer to \chpl{3.14}
and \chpl{2.72} respectively.
\end{example}

\subsection{The String Type}
\label{The_String_Type}
\index{string@\chpl{string}}

Strings are a primitive type designated by the symbol \chpl{string}
comprised of ASCII characters.  Their length is unbounded.
See~\rsec{Standard} for routines for manipulating strings.


\begin{openissue}
There is an expectation of future support for fixed-length strings.
\end{openissue}

\begin{openissue}
There is an expectation of future support for different character
sets, possibly including internationalization.
\end{openissue}

\section{Enumerated Types}
\label{Enumerated_Types}
\index{enumerated types}

Enumerated types are declared with the following syntax:

\begin{syntax}
enum-declaration-statement:
  `enum' identifier { enum-constant-list }

enum-constant-list:
  enum-constant
  enum-constant , enum-constant-list[OPT]

enum-constant:
  identifier init-part[OPT]

init-part:
  = expression
\end{syntax}

The enumerated type can then be referenced by its name, as summarized
by the following syntax:

\begin{syntax}
enum-type:
  identifier
\end{syntax}

An enumerated type defines a set of named constants that can be
referred to via a member access on the enumerated type.
These constants are treated as parameters of integral type.  Each
enumerated type is a distinct type. If the \sntx{init-part} is
omitted, the \sntx{enum-constant} has an integral value one higher
than the previous \sntx{enum-constant} in the enum, with the first
having the value \chpl{1}.

\begin{chapelexample}{enum.chpl}
The code
\begin{chapel}
enum statesman { Aristotle, Roosevelt, Churchill, Kissinger }
\end{chapel}
defines an enumerated type with four constants.  The function
\begin{chapel}
proc quote(s: statesman) {
  select s {
    when statesman.Aristotle do
       writeln("All paid jobs absorb and degrade the mind.");
    when statesman.Roosevelt do
       writeln("Every reform movement has a lunatic fringe.");
    when statesman.Churchill do
       writeln("A joke is a very serious thing.");
    when statesman.Kissinger do
       { write("No one will ever win the battle of the sexes; ");
         writeln("there's too much fraternizing with the enemy."); }
  }
}
\end{chapel}
\begin{chapelnoprint}
for s in statesman.Aristotle..statesman.Kissinger do
  quote(s:statesman);
\end{chapelnoprint}
\begin{chapeloutput}
All paid jobs absorb and degrade the mind.
Every reform movement has a lunatic fringe.
A joke is a very serious thing.
No one will ever win the battle of the sexes; there's too much fraternizing with the enemy.
\end{chapeloutput}
outputs a quote from the given statesman.  Note that enumerated
constants must be prefixed by the enumerated type and a dot.
\end{chapelexample}


\section{Locale Types}
\label{Locale_Types}
\index{types!locale}

Locale types are summarized by the following syntax:

\begin{syntax}
locale-type:
  `locale'
\end{syntax}

The \chpl{locale} type is defined in~\rsec{The_Locale_Type}.

\begin{openissue}
We expect to support \emph{realms} as another locale type.
\end{openissue}

\section{Structured Types}
\label{Structured_Types}
\index{types!structured}

The structured types are summarized by the following syntax:

\begin{syntax}
structured-type:
  class-type
  record-type
  union-type
  tuple-type
\end{syntax}
% in README.firstClassFns: function-type

Classes are discussed in \rsec{Classes}.  Records are discussed
in \rsec{Records}.  Unions are discussed in \rsec{Unions}.  Tuples are
discussed in \rsec{Tuples}.

\subsection{Class Types}

The class type defines a type that contains variables and constants,
called fields, and functions, called methods.  Classes are defined
in~\rsec{Classes}.  The class type can also contain type aliases and
parameters.  Such a class is generic and is defined
in~\rsec{Generic_Types}.

\subsection{Record Types}

The record type is similar to a class type; the primary difference is
that a record is a value rather than a reference.  Records are defined
in~\rsec{Records}.

\subsection{Union Types}

The union type defines a type that contains one of a set of variables.
Like classes and records, unions may also define methods.  Unions are
defined in~\rsec{Unions}.

\subsection{Tuple Types}

A tuple is a light-weight record that consists of one or more
anonymous fields.  If all the fields are of the same type, the tuple
is homogeneous.  Tuples are defined in~\rsec{Tuples}.

\section{Data Parallel Types}
\label{Data_Parallel_Types}
\index{types!dataparallel}

The data parallel types are summarized by the following syntax:

\begin{syntax}
dataparallel-type:
  range-type
  domain-type
  mapped-domain-type
  array-type
  index-type
\end{syntax}

Ranges and their index types are discussed in \rsec{Ranges}.
Domains and their index types are discussed in \rsec{Domains}.
Arrays are discussed in \rsec{Arrays}.

\subsection{Range Types}

A range defines an integral sequence of some integral type.  Ranges
are defined in \rsec{Ranges}.

\subsection{Domain, Array, and Index Types}
\label{Domain_and_Array_Types}

A domain defines a set of indices. An array defines a set of
elements that correspond to the indices in its domain.
A domain's indicies can be of any type.
Domains, arrays, and their index
types are defined in \rsec{Domains} and \rsec{Arrays}.

\section{Synchronisation Types}
\label{Synchronisation_Types}
\index{types!synchronisation}

The synchronization types are summarized by the following syntax:

\begin{syntax}
synchronization-type:
  sync-type
  single-type
\end{syntax}

Sync and single types are discussed in \rsec{Sync_Variables}
and \rsec{Single_Variables}.

\section{Type Aliases}
\label{Type_Aliases}
\index{type aliases}

Type aliases are declared with the following syntax:
\begin{syntax}
type-alias-declaration-statement:
  `config'[OPT] `type' type-alias-declaration-list ;

type-alias-declaration-list:
  type-alias-declaration
  type-alias-declaration , type-alias-declaration-list

type-alias-declaration:
  identifier = type-specifier
  identifier
\end{syntax}
A type alias is a symbol that aliases the type specified in the
\sntx{type-part}.  A use of a type alias has the same meaning as using
the type specified by \sntx{type-part} directly.

If the keyword \chpl{config} precedes the keyword \chpl{type}, the
type alias is called a configuration type alias.  Configuration type
aliases can be set at compilation time via compilation flags or other
implementation-defined means.  The \chpl{type-specifier} in the
program is ignored if the type-alias is alternatively set.

The \sntx{type-part} is optional in the definition of a class or
record.  Such a type alias is called an unspecified type
alias. Classes and records that contain type aliases, specified or
unspecified, are generic~(\rsec{Type_Aliases_in_Generic_Types}).

\begin{openissue}
There is on going discussion on whether a type alias is a new
type or simply an alias.  The former should enable redefinition of
default values, identity elements, etc.
%hilde
% Would inheritance work?
\end{openissue}

\cleardoublepage
\sekshun{Variables}
\label{Variables}

A variable is a symbol that represents memory.  Chapel is a
statically-typed, type-safe language so every variable has a type that
is known at compile-time and the compiler enforces that values
assigned to the variable can be stored in that variable as specified
by its type.

\subsection{Variable Declarations}
\label{Variable_Declarations}
\index{variables!declarations}

Variables are declared with the following syntax:
\begin{syntax}
variable-declaration-statement:
  `config'[OPT] variable-kind variable-declaration ;

variable-kind: one of
  `param' `const' `var'

variable-declaration-list:
  variable-declaration
  variable-declaration , variable-declaration-list

variable-declaration:
  identifier-list type-part[OPT] initialization-part
  identifier-list type-part

identifier-list:
  identifier
  identifier , identifier-list

type-part:
  : type
  : synchronization-type type

initialization-part:
  = expression
\end{syntax}
A \sntx{variable-declaration-statement} is used to define one or more
variables.  If the statement is a top-level module statement, the
variables are global; otherwise they are local.  Global variables are
discussed in~\rsec{Global_Variables}.  Local variables are discussed
in~\rsec{Local_Variables}.

The optional keyword \chpl{config} specifies that the variables are
configuration variables, described in
Section~\rsec{Configuration_Variables}.

The \sntx{variable-kind} specifies whether the variables are
parameters (\chpl{param}), constants (\chpl{const}), or regular
variables (\chpl{var}).  Parameters are compile-time constants whereas
constants are runtime constants.  Both levels of constants are
discussed in~\rsec{Constants}.

Multiple variables can be defined in the same variable declaration
list.  All variables defined in the same \sntx{identifier-list} are
defined to have the same type and initialization expression.

The \sntx{type-part} of a variable declaration specifies the type of
the variable.  It is optional if the \sntx{initialization-part} is
specified.  If the \sntx{type-part} is omitted, the type of the
variable is inferred using local type inference described
in~\rsec{Local_Type_Inference}.

The \sntx{initialization-part} of a variable declaration specifies an
initial expression to assign to the variable.  If
the \sntx{initialization-part} is omitted, the variable is initialized
to a default value described in~\rsec{Default_Initialization}.

\subsubsection{Default Initialization}
\label{Default_Initialization}
\index{variables!default initialization}

If a variable declaration has no initialization expression, a variable
is initialized to the default value of its type.  The default values
are as follows:
\begin{center}
\begin{tabular}{|l|l|}
\hline
{\bf Type} & {\bf Default Value} \\
\hline
{\tt bool} & {\tt false} \\
{\tt int(*)} & {\tt 0} \\
{\tt uint(*)} & {\tt 0} \\
{\tt real(*)} & {\tt 0.0} \\
{\tt imag(*)} & {\tt 0.0i} \\
{\tt complex(*)} & {\tt 0.0 + 0.0i} \\
{\tt string} & {\tt ""} \\
enums & first enum constant \\
classes & {\tt nil} \\
records & default constructed record \\
sequences & empty sequence \\
arrays & elements are default values \\
tuples & components are default values \\
\hline
\end{tabular}
\end{center}

\subsubsection{Local Type Inference}
\label{Local_Type_Inference}
\index{type inference}

If the type is omitted from a variable declaration, the type of the
variable becomes the type of the initialization expression.

\subsection{Global Variables}
\label{Global_Variables}
\index{variables!global}

Variables declared in statements that are in a module but not in a
function or block within that module are global variables.  Global
variables can be accessed anywhere within that module after the
declaration of that variable.  They can also be accessed in other
modules that use that module.

\subsection{Local Variables}
\label{Local_Variables}
\index{variables!local}

Local variables are variables that are not global.  Local variables
are declared within block statements.  They can only be accessed
within the scope of that block statement (including all inner nested
block statements and functions).

A local variable only exists during the execution of code that lies
within that block statement.  This time is called the lifetime of the
variable.  When execution has finished within that block statement,
the local variable and the storage it represents is removed.
Variables of class type are the sole exception.  Constructors of class
types create storage that is not associated with any scope.  Such
storage is managed automatically as discussed
in~\rsec{Automatic_Memory_Management}.

\subsection{Constants}
\label{Constants}

Constants are divided into two categories: parameters, specified with
the keyword \chpl{param}, are compile-time constants and constants,
specified with the keyword \chpl{const}, are runtime constants.

\subsubsection{Compile-Time Constants}
\label{Compile-Time_Constants}
\index{constants!compile-time}
\index{param@\chpl{param}}
\index{parameters}

A compile-time constant or parameter must have a single value that is
known statically by the compiler.  Parameters are restricted to
primitive and enumerated types.

Parameters can be assigned expressions that are parameter expressions.
Parameter expressions are restricted to the following constructs:
\begin{itemize}
\item
 Literals of primitive type.
\item
 Parenthesized parameter expressions.
\item
 Casts of parameter expressions to primitive or enumerated types.
\item
 Applications of the unary operators \verb@+@, \verb@-@, \verb@!@,
 and \verb@~@ on operands that are bool or integral parameter
 expressions.
\item
 Applications of the binary operators \verb@+@, \verb@-@, \verb@*@, \verb@/@, \verb@%@, \verb@**@, \verb@&&@, \verb@||@, \verb@!@, \verb@&@, \verb@|@, \verb@^@, \verb@~@, \verb@<<@, \verb@>>@, \verb@==@, \verb@!=@, \verb@<=@, \verb@>=@, \verb@<@, and \verb@>@ on operands that are bool or integral parameter expressions.
\item
 The conditional expression where the condition is a parameter and the
 then- and else-expressions are parameters.
\end{itemize}

There is an expectation that parameters will be expanded to more types
and more operations, and that functions that return parameters will be
introduced, in the future.

\subsubsection{Runtime Constants}
\label{Runtime_Constants}
\index{constants!runtime}
\index{const@\chpl{const}}

Constants, as opposed to parameters, do not have the restrictions that
are associated with parameters.  Constants can be any type.  They
require an initialization expression and contain the value of that
expression throughout their lifetime.

Variables of class type that are constants are constant references.
The fields of the class can be modified, but the variable always
points to the object that it was initialized to reference.

\subsection{Configuration Variables}
\label{Configuration_Variables}
\index{variables!configuration}
\index{config@\chpl{config}}

If the keyword \chpl{config} precedes the
keyword \chpl{var}, \chpl{const}, or \chpl{param}, the variable,
constant, or parameter is called a configuration variable,
configuration constant, or configuration parameter respectively.  Such
variables, constants, and parameters must be global.

The initialization of these variables can be set via implementation
dependent means, such as command-line switches or environment
variables.  The initialization expression in the program is ignored if
the initialization is alternatively set.

\index{parameters!configuration}
Configuration parameters are set during compilation time via
compilation flags or other implementation dependent means.
\begin{example}
A configuration parameter is set via a compiler flag.  It may be used
to control the target that is being compiled.  For example, the code
\begin{chapel}
config param target: string = "XT3";
\end{chapel}
sets a string parameter \chpl{target} to \chpl{"XT3"}.  This can be
checked to compile different code for this target.
\end{example}

\cleardoublepage
\sekshun{Conversions}
\label{Conversions}
\index{conversions}

A \emph{conversion} converts an expression of one type to another type,
possibly changing its value.
\index{conversions!source type}
\index{conversions!target type}
We refer to these two types the \emph{source} and \emph{target} types.
Conversions can be either
implicit~(\rsec{Implicit_Conversions}) or
explicit~(\rsec{Explicit_Conversions}).


\section{Implicit Conversions}
\label{Implicit_Conversions}
\index{conversions!implicit}

An \emph{implicit conversion} is a conversion that occurs implicitly,
that is, not due to an explicit specification in the program.
Implicit conversions occur at the locations in the program listed below.
Each location determines the target type.
The source and target types of an implicit conversion must be allowed.
They determine whether and how the expression's value changes.

Implicit conversions are not applied when initializing \chpl{ref} or
\chpl{type} values or for actual arguments passed to \chpl{ref} or
\chpl{type} formal arguments.

\index{conversions!implicit!occurs at}
An implicit conversion occurs at each of the following program locations:

\begin{itemize}
\item In an assignment, the expression on the right-hand side of
      the assignment is converted to the type of the variable
      or another lvalue on the left-hand side of the assignment.

\item The actual argument of a function call or an operator is converted
      to the type of the corresponding formal argument, if the formal's
      intent is \chpl{param}, \chpl{in}, \chpl{const in}, or an abstract intent
      (\rsec{Abstract_Intents}) with the semantics of
      \chpl{in} or \chpl{const in}.

% MPF: This rule doesn't seem to be implemented right now,
%      but rather reflects ideal language design.
\item If the formal argument's intent is \chpl{out}, the formal argument
      is converted to the type of the corresponding actual argument
      upon function return.

\item The return or yield expression within a function without a \chpl{ref}
      return intent is converted to the return type of that function.

\item The condition of a conditional expression,
      conditional statement, while-do or do-while loop statement
      is converted to the boolean type~(\rsec{Implicit_Statement_Bool_Conversions}).
      A special rule defines the allowed source types and
      how the expression's value changes in this case.
\end{itemize}

\index{conversions!implicit!allowed types}
Implicit conversions \emph{are allowed} between
the following source and target types,
as defined in the referenced subsections:

\begin{itemize}
\item numeric and boolean types~(\rsec{Implicit_NumBool_Conversions}),
\item class types~(\rsec{Implicit_Class_Conversions}),
\item integral types in the special case when the expression's value
      is a compile-time constant~(\rsec{Implicit_Compile_Time_Constant_Conversions}), and
\item from an integral or class type to \chpl{bool}
      in certain cases~(\rsec{Implicit_Statement_Bool_Conversions}).
\end{itemize}

In addition,
an implicit conversion from a type to the same type is allowed for any type.
Such conversion does not change the value of the expression.

% TODO: If an implicit conversion is not allowed, it is an error.

Implicit conversion is not transitive. That is, if an implicit conversion
is allowed from type \chpl{T1} to \chpl{T2} and from \chpl{T2} to \chpl{T3},
that by itself does not allow an implicit conversion
from \chpl{T1} to \chpl{T3}.

\subsection{Implicit Numeric and Bool Conversions}
\label{Implicit_NumBool_Conversions}

\index{conversions!numeric}
\index{conversions!implicit!numeric}
Implicit conversions among numeric types are allowed when
all values representable in the source type can also be represented
in the target type, retaining their full precision.
%
%REVIEW: vass: I did not understand the point of the following,
% so I am commenting it out for now.
%When the implicit conversion is from an integral to a real type, source
%types are converted to type \chpl{int} before determining if the
%conversion is valid.
%
In addition, implicit conversions from
types \chpl{int(64)} and \chpl{uint(64)} to types \chpl{real(64)}
and \chpl{complex(128)} are allowed, even though they may result in a loss of
precision.

%REVIEW: hilde
% Unless we are supporting some legacy behavior, I would recommend removing this
% provision.  A loss of precision is a loss of precision, so I would favor
% consistent behavior that does not lead to surprising results.  EVERY ``if''
% costs money: which is to say that if a behavior can be described simply, it can
% be implemented simply.

\begin{rationale}
We allow these additional conversions because they are an important
convenience for application programmers. Therefore we are willing to
lose precision in these cases. The largest real and complex types
are chosen to retain precision as often as as possible.
\end{rationale}

\index{conversions!boolean}
\index{conversions!implicit!boolean}
Any boolean type can be implicitly converted to any other boolean type,
retaining the boolean value.
Any boolean type can be implicitly converted to any integral type
by representing \chpl{false} as 0 and \chpl{true} as 1,
except (if applicable)
a boolean cannot be converted to \chpl{int(1)}.
% Rationale: because 1 cannot be represented by \chpl{int(1)}.

\begin{rationale}
We disallow implicit conversion of a boolean to
a real, imaginary, or complex type because of the following.
We expect that the cases where such a conversion is needed
will more likely be unintended by the programmer.
Marking those cases as errors will draw the programmer's attention.
If such a conversion is actually desired, a cast \rsec{Explicit_Conversions}
can be inserted.
\end{rationale}

Legal implicit conversions with numeric and boolean types
may thus be tabulated as follows:

\begin{center}
\begin{tabular}{l|llllll}
& \multicolumn{6}{c}{Target Type} \\ [4pt]

Source Type  & bool($t$) & uint($t$) & int($t$) & real($t$) & imag($t$) & complex($t$) \\  [3pt]

\cline{1-7} \\

bool($s$)    & all $s,t$ & all $s,t$   & all $s$; $2 \le t$ & & & \\ [7pt]

uint($s$)    & & $s \le t$ & $s < t$   & $s \le mant(t)$   & & $s \le mant(t/2)$   \\ [7pt]

uint(64)     & &           &           & real(64)          & & complex(128)        \\ [7pt]

int($s$)     & &           & $s \le t$ & $s \le mant(t)+1$ & & $s \le mant(t/2)+1$ \\ [7pt]

int(64)      & &           &           & real(64)          & & complex(128)        \\ [7pt]

real($s$)    & & & & $s \le t$ &           & $s \le t/2$ \\ [7pt]

imag($s$)    & & & &           & $s \le t$ & $s \le t/2$ \\ [7pt]

complex($s$) & & & &           &           & $s \le t$   \\ [5pt]

\end{tabular}
\end{center}
Here, $mant(i)$ is the number of bits in the (unsigned) mantissa of
the $i$-bit floating-point type.\footnote{For the IEEE 754 format,
$mant(32)=24$ and $mant(64)=53$.}
%
Conversions for the default integral and real types (\chpl{uint},
\chpl{complex}, etc.) are the same as for their
explicitly-sized counterparts.

\subsection{Implicit Compile-Time Constant Conversions}
\label{Implicit_Compile_Time_Constant_Conversions}
\index{conversions!numeric!parameter}
\index{conversions!implicit!parameter}

The following implicit conversion of a parameter is allowed:
\begin{itemize}
\item A parameter of type \chpl{int(64)} can be implicitly converted
to \chpl{int(8)}, \chpl{int(16)}, \chpl{int(32)}, or any unsigned integral type if the
value of the parameter is within the range of the target type.
\end{itemize}

\subsection{Implicit Statement Bool Conversions}
\label{Implicit_Statement_Bool_Conversions}
\index{conversions!boolean!in a statement}
\index{conversions!implicit!boolean}

In the condition of an if-statement, while-loop, and do-while-loop,
the following implicit conversions to \chpl{bool} are supported:
\begin{itemize}
\item An expression of integral type is taken to be false if it is zero and is true otherwise.
\item An expression of a class type is taken to be false if it is nil and is true otherwise.
\end{itemize}

\section{Explicit Conversions}
\label{Explicit_Conversions}
\index{conversions!explicit}

Explicit conversions require a cast in the code.  Casts are defined
in~\rsec{Casts}.  Explicit conversions are supported between more
types than implicit conversions, but explicit conversions are not
supported between all types.

The explicit conversions are a superset of the implicit conversions.
In addition to the following definitions,
an explicit conversion from a type to the same type is allowed for any type.
Such conversion does not change the value of the expression.

\subsection{Explicit Numeric Conversions}
\label{Explicit_Numeric_Conversions}
\index{conversions!numeric}
\index{conversions!explicit!numeric}

Explicit conversions are allowed from any numeric type or boolean to bytes or
string, and vice-versa.

% A valid \chpl{bool} value behaves like a single unsigned bit.  
When a \chpl{bool} is converted to a \chpl{bool}, \chpl{int}
or \chpl{uint} of equal or larger size, its value is zero-extended to fit the
new representation.  When a \chpl{bool} is converted to a
smaller \chpl{bool}, \chpl{int} or \chpl{uint}, its most significant
bits are truncated (as appropriate) to fit the new representation.

When a \chpl{int}, \chpl{uint}, or \chpl{real} is converted to a \chpl{bool}, the result is \chpl{false} if the number was equal to 0 and \chpl{true} otherwise.
% This has the odd effect that a bool stored in a signed one-bit bitfield would
% change sign without generating a conversion error.  But its subsequent
% conversion back to a bool would yield the original value.
% In regard to supporting bitfields: Be careful what you wish for.

% The source type determines whether a value is zero- or sign-extended.
When an \chpl{int} is converted to a larger \chpl{int} or \chpl{uint}, its value is
sign-extended to fit the new representation.  
When a \chpl{uint} is converted to a larger \chpl{int} or \chpl{uint}, its value
is zero-extended.
When an \chpl{int} or \chpl{uint} is converted to an \chpl{int} or \chpl{uint}
of the same size, its binary representation is unchanged.
When an \chpl{int} or \chpl{uint} is converted to a smaller \chpl{int}
or \chpl{uint}, its value is truncated to fit the new representation.

\begin{future}
There are several kinds of integer conversion which can result in a loss of
precision.  Currently, the conversions are performed as specified, and no error
is reported.  In the future, we intend to improve type checking, so the user can
be informed of potential precision loss at compile time, and actual precision
loss at run time.  Such cases include:
%
% An exception is thrown if the source value cannot be represented in the target type.
When an \chpl{int} is converted to a \chpl{uint} and the original value is
negative;
When a \chpl{uint} is converted to an \chpl{int} and the sign bit of the result
is true;
When an \chpl{int} is converted to a smaller \chpl{int} or \chpl{uint} and any
of the truncated bits differs from the original sign bit;
%
When a \chpl{uint} is converted to a smaller \chpl{int} or \chpl{uint} and any
of the truncated bits is true;
\end{future}

\begin{rationale}
For integer conversions, the default behavior of a program should be to produce
a run-time error if there is a loss of precision.  Thus, cast expressions not only
give rise to a value conversion at run time, but amount to an assertion
that the required precision is preserved.  Explicit conversion procedures would be
available in the run-time library so that one can perform explicit conversions
that result in a loss of precision but do not generate a run-time diagnostic.
\end{rationale}

When converting from a \chpl{real} type to a larger \chpl{real} type, the
represented value is preserved.  When converting from a \chpl{real} type to a
smaller \chpl{real} type, the closest representation in the target type is
chosen.\footnote{When converting to a smaller real type, a loss of precision is \emph{expected}.
Therefore, there is no reason to produce a run-time diagnostic.}

When converting to a \chpl{real} type from an integer type, integer types
smaller than \chpl{int} are first converted to \chpl{int}.  Then, the closest
representation of the converted value in the target type is chosen.  The exact
behavior of this conversion is implementation-defined.

When converting from \chpl{real($k$)} to \chpl{complex($2k$)}, the original
value is copied into the real part of the result, and the imaginary part of the
result is set to zero.  When converting from a \chpl{real($k$)} to
a \chpl{complex($\ell$)} such that $\ell > 2k$, the conversion is performed as
if the original value is first converted to \chpl{real($\ell/2$)} and then
to \chpl{$\ell$}.

The rules for converting from \chpl{imag} to \chpl{complex} are the same as for
converting from real, except that the imaginary part of the result is set using
the input value, and the real part of the result is set to zero.

\subsection{Explicit Tuple to Complex Conversion}
\label{Explicit_Tuple_to_Complex_Conversion}
\index{conversions!tuple to complex}
\index{conversions!explicit!tuple to complex}

A two-tuple of numerical values may be converted to a \chpl{complex} value.  If
the destination type is \chpl{complex(128)}, each member of the two-tuple must
be convertible to \chpl{real(64)}.  If the destination type
is \chpl{complex(64)}, each member of the two-tuple must be convertible
to \chpl{real(32)}.  The first member of the tuple becomes the real part of the
resulting complex value; the second member of the tuple becomes the imaginary
part of the resulting complex value.

\subsection{Explicit Enumeration Conversions}
\label{Explicit_Enumeration_Conversions}
\index{conversions!enumeration}
\index{conversions!explicit!enumeration}

Explicit conversions are allowed from any enumerated type to any
\chpl{bytes} or \chpl{string} and vice-versa, and include \chpl{param} conversions.
For enumerated types that are either concrete or semi-concrete
(\rsec{Enumerated_Types}), conversions are supported between the
enum's constants and any numeric type or \chpl{bool},
including \chpl{param} conversions.  For a semi-concrete enumerated
type, if a conversion is attempted involving a constant with no
underlying integer value, it will generate a compile-time error for
a \chpl{param} conversion or an execution-time error otherwise.

When the target type is an integer type, the value is first converted to its
underlying integer type and then to the target type, following the rules above
for converting between integers.

When the target type is a real, imaginary, or complex type, the value
is first converted to its underlying integer type and then to the
target type.

When the target type is \chpl{bool}, the value is first converted to its
underlying integer type.  If the result is zero, the value of the \chpl{bool}
is \chpl{false}; otherwise, it is \chpl{true}.

When the target type is \chpl{bytes} or \chpl{string}, the value becomes the
name of the enumerator.

When the source type is \chpl{bool}, enumerators corresponding to the values 0
and 1 in the underlying integer type are selected, corresponding to input values
of \chpl{false} and \chpl{true}, respectively.

%REVIEW: hilde
% As with default values for variables of enumerated types, I am pushing for the
% simplest implementation -- in which the conversion does not actually change
% the stored value.  This means that it may be possible for an enumerated variable
% to assume a value that does not correspond to any of its enumerators.  Further
% encouragement to always supply a default clause in your switch statements!

When the source type is a real or integer type, the value is converted to the
target type's underlying integer type.  

The conversion from \chpl{complex} and \chpl{imag} types to an enumerated type is not
permitted.

When the source type is \chpl{bytes} or \chpl{string}, the enumerator whose name
matches value of the input is selected.  If no such enumerator exists, an
\chpl{IllegalArgumentError} is thrown.

\subsection{Explicit Class Conversions}
\label{Explicit_Class_Conversions}
\index{conversions!class}
\index{conversions!explicit!class}

An expression of static class type \chpl{C} can be explicitly
converted to a class type \chpl{D} provided that \chpl{C} is derived
from \chpl{D} or \chpl{D} is derived from \chpl{C}.

When at run time the source expression refers to an instance of \chpl{D}
or it subclass, its value is not changed.  Otherwise, the cast fails and
the result depends on whether or not the destination type is nilable. If
the cast fails and the destination type is not nilable, the cast
expression will throw a \chpl{classCastError}. If the cast fails and the
destination type is nilable, as with \chpl{D?}, then the result will be
\chpl{nil}.

In some cases a new variant of a class type needs to be computed that has
different nilability or memory management strategy. Supposing that
\chpl{T} represents a class type, then these casts may compute a new type:

\begin{itemize}
\item
\chpl{T:owned} - new management is \chpl{owned}, nilability from \chpl{T}

\item
\chpl{T:shared} - new management \chpl{shared}, nilability from \chpl{T}

\item
\chpl{T:borrowed} - new management \chpl{borrowed}, nilability from \chpl{T}

\item
\chpl{T:unmanaged} - new management \chpl{unmanaged}, nilability from \chpl{T}

\item
\chpl{T:class} - non-nilable type with specific concrete or generic management from \chpl{T}

\item
\chpl{T:class?} - nilable type with specific concrete or generic management from \chpl{T}

\item
\chpl{T:owned class} - non-nilable type with \chpl{owned} management
\item
\chpl{T:owned class?} - nilable type with \chpl{owned} management

\item
\chpl{T:shared class} - non-nilable type with \chpl{shared} management
\item
\chpl{T:shared class?} - nilable type with \chpl{shared} management

\item
\chpl{T:borrowed class} - non-nilable type with \chpl{borrowed} management
\item
\chpl{T:borrowed class?} - nilable type with \chpl{borrowed} management

\item
\chpl{T:unmanaged class} - non-nilable type with \chpl{unmanaged} management
\item
\chpl{T:unmanaged class?} - nilable type with \chpl{unmanaged} management

\end{itemize}

\subsection{Explicit Range Conversions}
\label{Explicit_Range_Conversions}
\index{conversions!range}
\index{conversions!explicit!range}

An expression of stridable range type can be explicitly converted
to an unstridable range type, changing the stride to 1 in the process.

\subsection{Explicit Domain Conversions}
\label{Explicit_Domain_Conversions}
\index{conversions!domain}
\index{conversions!explicit!domain}

An expression of stridable domain type can be explicitly converted
to an unstridable domain type, changing all strides to 1 in the process.

\subsection{Explicit String to Bytes Conversions}
\label{Explicit_String_to_Bytes_Conversions}
\index{conversions!string to bytes}
\index{conversions!explicit!string to bytes}

An expression of \chpl{string} type can be explicitly converted to a
\chpl{bytes}. However, the reverse is not possible as a \chpl{bytes} can contain
arbitrary bytes. Instead, \chpl{bytes.decode()} method should be used to produce
a \chpl{string} from a \chpl{bytes}.

\subsection{Explicit Type to String Conversions}
\label{Explicit_Type_to_String_Conversions}
\index{conversions!type to string}
\index{conversions!explicit!type to string}

A type expression can be explicitly converted to a \chpl{string}. The resultant
\chpl{string} is the name of the type.

\begin{chapelexample}{explicit-type-to-string.chpl}
For example:
\begin{chapel}
var x: real(64) = 10.0;
writeln(x.type:string);
\end{chapel}
\begin{chapeloutput}
real(64)
\end{chapeloutput}
This program will print out the string \chpl{"real(64)"}.
\end{chapelexample}

\cleardoublepage
\sekshun{Expressions}
\label{Expressions}
\index{expressions}

Chapel provides the following expressions:

\begin{syntax}
expression:
  literal-expression
  nil-expression
  variable-expression
  enum-constant-expression
  call-expression
  iteratable-call-expression
  member-access-expression
  constructor-call-expression
  query-expression
  cast-expression
  lvalue-expression
  parenthesized-expression
  unary-expression
  binary-expression
  let-expression
  if-expression
  for-expression
  forall-expression
  reduce-expression
  scan-expression
  module-access-expression
  tuple-expression
  tuple-expand-expression
  locale-access-expression
  mapped-domain-expression
\end{syntax}
% in README.firstClassFns: lambda-declaration-expression

Individual expressions are defined in the remainder of this chapter
and additionally as follows:

\begin{itemize}
\item forall, reduce, and scan \rsec{Data_Parallelism}
\item module access \rsec{Explicit_Naming}
\item tuple and tuple expand \rsec{Tuples}
\item locale access \rsec{Querying_the_Locale_of_a_Variable}
\item mapped domain \rsec{Domain_Maps}
\item constructor calls \rsec{Class_New}
\item \chpl{nil} \rsec{Class_nil_value}
\end{itemize}

\section{Literal Expressions}
\label{Literal_Expressions}
\index{literal expressions}
\index{expressions!literal}

A literal value for any of the predefined
types~(\rsec{Primitive_Type_Literals}) is a literal expression.
Literal expressions are given by the following syntax:
\begin{syntax}
literal-expression:
  bool-literal
  integer-literal
  real-literal
  imaginary-literal
  string-literal
  range-literal
  domain-literal
  array-literal
\end{syntax}

\section{Variable Expressions}
\label{Variable_Expressions}
\index{expressions!variable}

A use of a variable, constant, parameter, or formal argument, is
itself an expression.  The syntax of a variable expression is given
by:
\begin{syntax}
variable-expression:
  identifier
\end{syntax}

\section{Enumeration Constant Expression}
\label{Enumeration_Constant_Expression}
\index{expressions!enumeration constant}

A use of an enumeration constant is itself an expression.  Such a
constant must be preceded by the enumeration type name.  The syntax of
an enumeration constant expression is given by:
\begin{syntax}
enum-constant-expression:
  enum-type . identifier
\end{syntax}

For an example of using enumeration constants,
see~\rsec{Enumerated_Types}.

\section{Parenthesized Expressions}
\label{Parenthesized_Expressions}
\index{expressions!parenthesized}

A \sntx{parenthesized-expression} is an expression that is delimited
by parentheses as given by:
\begin{syntax}
parenthesized-expression:
  ( expression )
\end{syntax}
Such an expression evaluates to the expression.  The parentheses are
ignored and have only a syntactical effect.

\section{Call Expressions}
\label{Call_Expressions}
\index{function calls}
\index{expressions!call}

Functions and function calls are defined in~\rsec{Functions}.

\section{Indexing Expressions}
\label{Indexing_Expressions}
\index{indexing}
\index{expressions!indexing}

Indexing, for example into arrays, tuples, and domains,
has the same syntax as a call expression.
 
Indexing is performed by an implicit invocation of the \chpl{this} method
on the value being indexed,
passing the indices as the actual arguments.

\section{Member Access Expressions}
\label{Member_Access_Expressions}
\index{member access}
\index{expressions!member access}

Member access expressions provide access to a field or invoke a method
of an instance of a class, record, or union.
They are defined in \rsec{Class_Field_Accesses} and
\rsec{Class_Method_Calls}, respectively.

\begin{syntax}
member-access-expression:
  field-access-expression
  method-call-expression
\end{syntax}

\section{The Query Expression}
\label{The_Query_Expression}
\index{expressions!type query}
\index{? (type query)@\chpl{?} (type query)}
\index{operators!? (type query)@\chpl{?} (type query)}

A query expression is used to query a type or value within a formal
argument type expression.  The syntax of a query expression is given
by:
\begin{syntax}
query-expression:
  ? identifier[OPT]
\end{syntax}
Querying is restricted to querying the type of a formal argument, the
element type of a formal argument that is an array, the domain of a
formal argument that is an array, the size of a primitive type, or a
type or parameter field of a formal argument type.

The identifier can be omitted.  This is useful for ensuring the
genericity of a generic type that defines default values for all of
its generic fields when specifying a formal argument as discussed
in~\rsec{Formal_Arguments_of_Generic_Type}.

\begin{chapelexample}{query.chpl}
The following code defines a generic function where the type of the
first argument is queried and stored in the type alias \chpl{t} and
the domain of the second argument is queried and stored in the
variable \chpl{D}:
\begin{chapelnoprint}
{ // }
\end{chapelnoprint}
\begin{chapel}
proc foo(x: ?t, y: [?D] t) {
  for i in D do
    y[i] = x;
}
\end{chapel}
\begin{chapelnoprint}
// {
var x = 1.5;
var y: [1..4] x.type;
foo(x, y);
writeln(y);
}
\end{chapelnoprint}
This allows a generic specification of assigning a
particular value to all elements of an array.  The value and the
elements of the array are constrained to be the same type.  This
function can be rewritten without query expression as follows:
\begin{chapelnoprint}
{ // }
\end{chapelnoprint}
\begin{chapel}
proc foo(x, y: [] x.type) {
  for i in y.domain do
    y[i] = x;
}
\end{chapel}
\begin{chapelnoprint}
// {
var x = 1.5;
var y: [1..4] x.type;
foo(x, y);
writeln(y);
}
\end{chapelnoprint}
\begin{chapeloutput}
1.5 1.5 1.5 1.5
1.5 1.5 1.5 1.5
\end{chapeloutput}
\end{chapelexample}

There is an expectation that query expressions will be allowed in more
places in the future.

\section{Casts}
\label{Casts}
\index{casts}
\index{expressions!cast}
\index{: (cast)@\chpl{:} (cast)}
\index{operators!: (cast)@\chpl{:} (cast)}

A cast is specified with the following syntax:
\begin{syntax}
cast-expression:
  expression : type-specifier
\end{syntax}
The expression is converted to the specified type.  A cast expression invokes
the corresponding explicit conversion~(\rsec{Explicit_Conversions}).  A
resolution error occurs if no such conversion exists.

\section{LValue Expressions}
\label{LValue_Expressions}
\index{lvalues}
\index{expressions!lvalue}

An {\em lvalue} is an expression that can be used on the left-hand
side of an assignment statement or on either side of a swap statement,
that can be passed to a formal argument of a function that
has \chpl{out}, \chpl{inout} or \chpl{ref} intent, or that can be returned by a
variable function.  Valid lvalue expressions include the following:
\begin{itemize}
\item
 Variable expressions.
\item
 Member access expressions.
\item
 Call expressions of variable functions.
\item
 Indexing expressions.
\end{itemize}

LValue expressions are given by the following syntax:
\begin{syntax}
lvalue-expression:
  variable-expression
  member-access-expression
  call-expression
  parenthesized-expression
\end{syntax}
The syntax is less restrictive than the definition above.  For
example, not all \sntx{call-expression}s are lvalues.

\section{Precedence and Associativity}
\label{Operator_Precedence_and_Associativity}
\index{operators!precedence}
\index{operators!associativity}
\index{expressions!precedence}
\index{expressions!associativity}

The following table summarizes operator and expression precedence and
associativity.  Operators and expressions listed earlier have higher
precedence than those listed later.
\begin{center}
\begin{tabular}{|l|l|l|}
\hline
{\bf Operator} & {\bf Associativity} & {\bf Use} \\
\hline
\verb@.@ & \multirow{3}{*}{left} & member access \\
\verb@()@ & & function call or access \\
\verb@[]@ & & function call or access \\
\hline
\verb@new@ & right & constructor call \\
\hline
\verb@:@ & left & cast \\
\hline
\verb@**@ & right & exponentiation \\
\hline
\verb@reduce@ & \multirow{3}{*}{left} & reduction \\
\verb@scan@ & & scan \\
\verb@dmapped@ & & domain map application \\
\hline
\verb@!@ & \multirow{2}{*}{right} & logical negation \\
\verb@~@ & & bitwise negation \\
\hline
\verb@*@ & \multirow{3}{*}{left} & multiplication \\
\verb@/@ & & division \\
\verb@%@ & & modulus \\
\hline
unary \verb@+@ & \multirow{2}{*}{right} & positive identity \\
unary \verb@-@ & & negation \\
\hline
\verb@+@ & \multirow{2}{*}{left} & addition \\
\verb@-@ & & subtraction \\
\hline
\verb@<<@ & \multirow{2}{*}{left} & left shift \\
\verb@>>@ & & right shift \\
\hline
\verb@<=@ & \multirow{4}{*}{left} & less-than-or-equal-to comparison \\
\verb@>=@ & & greater-than-or-equal-to comparison \\
\verb@<@ & & less-than comparison \\
\verb@>@ & & greater-than comparison \\
\hline
\verb@==@ & \multirow{2}{*}{left} & equal-to comparison \\
\verb@!=@ & & not-equal-to comparison \\
\hline
\verb@&@ & left & bitwise/logical and \\
\hline
\verb@^@ & left & bitwise/logical xor \\
\hline
\verb@|@ & left & bitwise/logical or \\
\hline
\verb@&&@ & left & short-circuiting logical and \\
\hline
\verb@||@ & left & short-circuiting logical or \\
\hline
\verb@..@ & left & range construction \\
\hline
\verb@in@ & left & forall expression \\
\hline
\verb@by@ & \multirow{2}{*}{left} & range/domain stride application \\
\verb@#@ & & range count application \\
\hline
\verb@if then else@ & \multirow{5}{*}{left} & conditional expression \\
\verb@forall do@ & & forall expression \\
\verb@[ ]@ & & forall expression \\
\verb@for do@ & & for expression \\
\verb@sync single@ & & sync and single type \\
\hline
\verb@,@ & left & comma separated expressions \\
\hline
\end{tabular}
\end{center}

\begin{rationale}
In general, our operator precedence is based on that of the C family
of languages including C++, Java, Perl, and C\#.  We comment on a few
of the differences and unique factors here.

We find that there is tension between the relative precedence of
exponentiation, unary minus/plus, and casts.  The following three
expressions show our intuition for how these expressions should be
parenthesized.

\begin{center}
\begin{tabular}{lcl}
\chpl{-2**4} & wants & \chpl{-(2**4)} \\
\chpl{-2:uint} & wants & \chpl{(-2):uint} \\
\chpl{2:uint**4:uint} & wants & \chpl{(2:uint)**(4:uint)} \\
\end{tabular}
\end{center}

Trying to support all three of these cases results in a
circularity---exponentiation wants precedence over unary minus, unary
minus wants precedence over casts, and casts want precedence over
exponentiation.  We chose to break the circularity by making unary
minus have a lower precedence.  This means that for the second case
above:

\begin{center}
\begin{tabular}{lcl}
\chpl{-2:uint} & requires & \chpl{(-2):uint} \\
\end{tabular}
\end{center}

We also chose to depart from the C family of languages by making unary
plus/minus have lower precedence than binary multiplication, division,
and modulus as in Fortran.  We have found very few cases that
distinguish between these cases.  An interesting one is:

\begin{center}
\begin{tabular}{l}
\chpl{const minint = min(int(32));}\\
\chpl{...-minint/2...}
\end{tabular}
\end{center}

Intuitively, this should result in a positive value, yet C's
precedence rules results in a negative value due to asymmetry in
modern integer representations.  If we learn of cases that argue in
favor of the C approach, we would likely reverse this decision in
order to more closely match C.

We were tempted to diverge from the C precedence rules for the binary
bitwise operators to make them bind less tightly than comparisons.
This would allow us to interpret:

\begin{center}
\begin{tabular}{lcl}
\chpl{a | b == 0} & as & \chpl{(a | b) == 0} \\
\end{tabular}
\end{center}

However, given that no other popular modern language has made this
change, we felt it unwise to stray from the pack.  The typical
rationale for the C ordering is to allow these operators to be used as
non-short-circuiting logical operations.

One final area of note is the precedence of reductions.  Two common
cases tend to argue for making reductions very low or very high in the
precedence table:

\begin{center}
\begin{tabular}{lcl}
\chpl{max reduce A - min reduce A} & wants & \chpl{(max reduce A) - (min reduce A)} \\
\chpl{max reduce A * B} & wants & \chpl{max reduce (A * B)} \\
\end{tabular}
\end{center}

The first statement would require reductions to have a higher
precedence than the arithmetic operators while the second would
require them to be lower.  We opted to make reductions have high
precedence due to the argument that they tend to resemble unary
operators.  Thus, to support our intuition:

\begin{center}
\begin{tabular}{lcl}
\chpl{max reduce A * B} & requires & \chpl{max reduce (A * B)} \\
\end{tabular}
\end{center}

This choice also has the (arguably positive) effect of making the
unparenthesized version of this statement result in an aggregate value
if A and B are both aggregates---the reduction of A results in a
scalar which promotes when being multiplied by B, resulting in an
aggregate.  Our intuition is that users who forget the parenthesis
will learn of their error at compilation time because the resulting
expression is not a scalar as expected.

\end{rationale}

\section{Operator Expressions}
\label{Binary_Expressions}
\label{Unary_Expressions}
\index{expressions!operator}

\index{operators!unary}
\index{expressions!unary operator}
The application of operators to expressions is itself an expression.
The syntax of a unary expression is given by:
\begin{syntax}
unary-expression:
  unary-operator expression

unary-operator: one of
  + - ~ !
\end{syntax}

\index{operators!binary}
\index{expressions!binary operator}
The syntax of a binary expression is given by:
\begin{syntax}
binary-expression:
  expression binary-operator expression

binary-operator: one of
  + - * / % ** & | ^ << >> && || == != <= >= < > `by' #
\end{syntax}

The operators are defined in subsequent sections.

\section{Arithmetic Operators}
\label{Arithmetic_Operators}
\index{operators!arithmetic}

This section describes the predefined arithmetic operators.  These
operators can be redefined over different types using operator
overloading~(\rsec{Function_Overloading}).

For each operator, implicit conversions are applied to the operands of
an operator such that they are compatible with one of the function
forms listed, those listed earlier in the list being given
preference.  If no compatible implicit conversions exist, then a
compile-time error occurs.  In these cases, an explicit cast is required.

All integral arithmetic operators are implemented over integral types
of size 32 and 64 bits only.  For example, adding two 8-bit integers
is done by first converting them to 32-bit integers and then adding
the 32-bit integers.  The result is a 32-bit integer.

\subsection{Unary Plus Operators}
\label{Unary_Plus_Operators}
\index{+ (unary)@\chpl{+} (unary)}
\index{operators!+ (unary)@\chpl{+} (unary)}

The unary plus operators are predefined as follows:
\begin{chapel}
proc +(a: int(8)): int(8)
proc +(a: int(16)): int(16)
proc +(a: int(32)): int(32)
proc +(a: int(64)): int(64)

proc +(a: uint(8)): uint(8)
proc +(a: uint(16)): uint(16)
proc +(a: uint(32)): uint(32)
proc +(a: uint(64)): uint(64)

proc +(a: real(32)): real(32)
proc +(a: real(64)): real(64)

proc +(a: imag(32)): imag(32)
proc +(a: imag(64)): imag(64)

proc +(a: complex(64)): complex(64)
proc +(a: complex(128)): complex(128)
\end{chapel}
For each of these definitions, the result is the value of the operand.

\subsection{Unary Minus Operators}
\label{Unary_Minus_Operators}
\index{operators!negation}
\index{- (unary)@\chpl{-} (unary)}
\index{operators!- (unary)@\chpl{-} (unary)}

The unary minus operators are predefined as follows:
\begin{chapel}
proc -(a: int(8)): int(8)
proc -(a: int(16)): int(16)
proc -(a: int(32)): int(32)
proc -(a: int(64)): int(64)

proc -(a: real(32)): real(32)
proc -(a: real(64)): real(64)

proc -(a: imag(32)): imag(32)
proc -(a: imag(64)): imag(64)

proc -(a: complex(64)): complex(64)
proc -(a: complex(128)): complex(128)
\end{chapel}
For each of these definitions that return a value, the result is the
negation of the value of the operand.  For integral types, this
corresponds to subtracting the value from zero.  For real and
imaginary types, this corresponds to inverting the sign.  For complex
types, this corresponds to inverting the signs of both the real and
imaginary parts.

It is an error to try to negate a value of type \chpl{uint(64)}.  Note
that negating a value of type \chpl{uint(32)} first converts the type
to \chpl{int(64)} using an implicit conversion.

\subsection{Addition Operators}
\label{Addition_Operators}
\index{operators!addition}
\index{+@\chpl{+}}
\index{operators!+@\chpl{+}}

The addition operators are predefined as follows:
\begin{chapel}
proc +(a: int(8), b: int(8)): int(8)
proc +(a: int(16), b: int(16)): int(16)
proc +(a: int(32), b: int(32)): int(32)
proc +(a: int(64), b: int(64)): int(64)

proc +(a: uint(8), b: uint(8)): uint(8)
proc +(a: uint(16), b: uint(16)): uint(16)
proc +(a: uint(32), b: uint(32)): uint(32)
proc +(a: uint(64), b: uint(64)): uint(64)

proc +(a: real(32), b: real(32)): real(32)
proc +(a: real(64), b: real(64)): real(64)

proc +(a: imag(32), b: imag(32)): imag(32)
proc +(a: imag(64), b: imag(64)): imag(64)

proc +(a: complex(64), b: complex(64)): complex(64)
proc +(a: complex(128), b: complex(128)): complex(128)

proc +(a: real(32), b: imag(32)): complex(64)
proc +(a: imag(32), b: real(32)): complex(64)
proc +(a: real(64), b: imag(64)): complex(128)
proc +(a: imag(64), b: real(64)): complex(128)

proc +(a: real(32), b: complex(64)): complex(64)
proc +(a: complex(64), b: real(32)): complex(64)
proc +(a: real(64), b: complex(128)): complex(128)
proc +(a: complex(128), b: real(64)): complex(128)

proc +(a: imag(32), b: complex(64)): complex(64)
proc +(a: complex(64), b: imag(32)): complex(64)
proc +(a: imag(64), b: complex(128)): complex(128)
proc +(a: complex(128), b: imag(64)): complex(128)
\end{chapel}
For each of these definitions that return a value, the result is the
sum of the two operands.

It is a compile-time error to add a value of type \chpl{uint(64)} and
a value of type \chpl{int(64)}.

Addition over a value of real type and a value of imaginary type
produces a value of complex type.  Addition of values of complex type
and either real or imaginary types also produces a value of complex
type.

\subsection{Subtraction Operators}
\label{Subtraction_Operators}
\index{operators!subtraction}
\index{-@\chpl{-}}
\index{operators!-@\chpl{-}}

The subtraction operators are predefined as follows:
\begin{chapel}
proc -(a: int(8), b: int(8)): int(8)
proc -(a: int(16), b: int(16)): int(16)
proc -(a: int(32), b: int(32)): int(32)
proc -(a: int(64), b: int(64)): int(64)

proc -(a: uint(8), b: uint(8)): uint(8)
proc -(a: uint(16), b: uint(16)): uint(16)
proc -(a: uint(32), b: uint(32)): uint(32)
proc -(a: uint(64), b: uint(64)): uint(64)

proc -(a: real(32), b: real(32)): real(32)
proc -(a: real(64), b: real(64)): real(64)

proc -(a: imag(32), b: imag(32)): imag(32)
proc -(a: imag(64), b: imag(64)): imag(64)

proc -(a: complex(64), b: complex(64)): complex(64)
proc -(a: complex(128), b: complex(128)): complex(128)

proc -(a: real(32), b: imag(32)): complex(64)
proc -(a: imag(32), b: real(32)): complex(64)
proc -(a: real(64), b: imag(64)): complex(128)
proc -(a: imag(64), b: real(64)): complex(128)

proc -(a: real(32), b: complex(64)): complex(64)
proc -(a: complex(64), b: real(32)): complex(64)
proc -(a: real(64), b: complex(128)): complex(128)
proc -(a: complex(128), b: real(64)): complex(128)

proc -(a: imag(32), b: complex(64)): complex(64)
proc -(a: complex(64), b: imag(32)): complex(64)
proc -(a: imag(64), b: complex(128)): complex(128)
proc -(a: complex(128), b: imag(64)): complex(128)
\end{chapel}
For each of these definitions that return a value, the result is the
value obtained by subtracting the second operand from the first
operand.

It is a compile-time error to subtract a value of type \chpl{uint(64)}
from a value of type \chpl{int(64)}, and vice versa.

Subtraction of a value of real type from a value of imaginary type,
and vice versa, produces a value of complex type.  Subtraction of
values of complex type from either real or imaginary types, and vice
versa, also produces a value of complex type.

\subsection{Multiplication Operators}
\label{Multiplication_Operators}
\index{operators!multiplication}
\index{operators!*@\chpl{*}}
\index{*@\chpl{*}}

The multiplication operators are predefined as follows:
\begin{chapel}
proc *(a: int(8), b: int(8)): int(8)
proc *(a: int(16), b: int(16)): int(16)
proc *(a: int(32), b: int(32)): int(32)
proc *(a: int(64), b: int(64)): int(64)

proc *(a: uint(8), b: uint(8)): uint(8)
proc *(a: uint(16), b: uint(16)): uint(16)
proc *(a: uint(32), b: uint(32)): uint(32)
proc *(a: uint(64), b: uint(64)): uint(64)

proc *(a: real(32), b: real(32)): real(32)
proc *(a: real(64), b: real(64)): real(64)

proc *(a: imag(32), b: imag(32)): real(32)
proc *(a: imag(64), b: imag(64)): real(64)

proc *(a: complex(64), b: complex(64)): complex(64)
proc *(a: complex(128), b: complex(128)): complex(128)

proc *(a: real(32), b: imag(32)): imag(32)
proc *(a: imag(32), b: real(32)): imag(32)
proc *(a: real(64), b: imag(64)): imag(64)
proc *(a: imag(64), b: real(64)): imag(64)

proc *(a: real(32), b: complex(64)): complex(64)
proc *(a: complex(64), b: real(32)): complex(64)
proc *(a: real(64), b: complex(128)): complex(128)
proc *(a: complex(128), b: real(64)): complex(128)

proc *(a: imag(32), b: complex(64)): complex(64)
proc *(a: complex(64), b: imag(32)): complex(64)
proc *(a: imag(64), b: complex(128)): complex(128)
proc *(a: complex(128), b: imag(64)): complex(128)
\end{chapel}
For each of these definitions that return a value, the result is the
product of the two operands.

It is a compile-time error to multiply a value of type \chpl{uint(64)} and
a value of type \chpl{int(64)}.

Multiplication of values of imaginary type produces a value of real
type.  Multiplication over a value of real type and a value of
imaginary type produces a value of imaginary type.  Multiplication of
values of complex type and either real or imaginary types produces a
value of complex type.

\subsection{Division Operators}
\label{Division_Operators}
\index{operators!division}
\index{/@\chpl{/}}
\index{operators!/@\chpl{/}}

The division operators are predefined as follows:
\begin{chapel}
proc /(a: int(8), b: int(8)): int(8)
proc /(a: int(16), b: int(16)): int(16)
proc /(a: int(32), b: int(32)): int(32)
proc /(a: int(64), b: int(64)): int(64)

proc /(a: uint(8), b: uint(8)): uint(8)
proc /(a: uint(16), b: uint(16)): uint(16)
proc /(a: uint(32), b: uint(32)): uint(32)
proc /(a: uint(64), b: uint(64)): uint(64)

proc /(a: real(32), b: real(32)): real(32)
proc /(a: real(64), b: real(64)): real(64)

proc /(a: imag(32), b: imag(32)): real(32)
proc /(a: imag(64), b: imag(64)): real(64)

proc /(a: complex(64), b: complex(64)): complex(64)
proc /(a: complex(128), b: complex(128)): complex(128)

proc /(a: real(32), b: imag(32)): imag(32)
proc /(a: imag(32), b: real(32)): imag(32)
proc /(a: real(64), b: imag(64)): imag(64)
proc /(a: imag(64), b: real(64)): imag(64)

proc /(a: real(32), b: complex(64)): complex(64)
proc /(a: complex(64), b: real(32)): complex(64)
proc /(a: real(64), b: complex(128)): complex(128)
proc /(a: complex(128), b: real(64)): complex(128)

proc /(a: imag(32), b: complex(64)): complex(64)
proc /(a: complex(64), b: imag(32)): complex(64)
proc /(a: imag(64), b: complex(128)): complex(128)
proc /(a: complex(128), b: imag(64)): complex(128)
\end{chapel}
For each of these definitions that return a value, the result is the
quotient of the two operands.

It is a compile-time error to divide a value of type \chpl{uint(64)} by
a value of type \chpl{int(64)}, and vice versa.

Division of values of imaginary type produces a value of real type.
Division over a value of real type and a value of imaginary type
produces a value of imaginary type.  Division of values of complex
type and either real or imaginary types produces a value of complex
type.

When the operands are integers, the result (quotient) is also an integer.  If \chpl{b}
does not divide \chpl{a} exactly, then there are two candidate quotients $q1$ and $q2$
such that $b * q1$ and $b * q2$ are the two multiples of \chpl{b} closest to \chpl{a}.
The integer result $q$ is the candidate quotient which lies closest to zero.

\subsection{Modulus Operators}
\label{Modulus_Operators}
\index{operators!modulus}
\index{\%@\chpl{\%}}
\index{operators!\%@\chpl{\%}}

The modulus operators are predefined as follows:
\begin{chapel}
proc %(a: int(8), b: int(8)): int(8)
proc %(a: int(16), b: int(16)): int(16)
proc %(a: int(32), b: int(32)): int(32)
proc %(a: int(64), b: int(64)): int(64)

proc %(a: uint(8), b: uint(8)): uint(8)
proc %(a: uint(16), b: uint(16)): uint(16)
proc %(a: uint(32), b: uint(32)): uint(32)
proc %(a: uint(64), b: uint(64)): uint(64)
\end{chapel}
For each of these definitions that return a value, the result is the
remainder when the first operand is divided by the second operand.

The sign of the result is the same as the sign of the dividend \chpl{a}, and the
magnitude of the result is always smaller than that of the divisor \chpl{b}.
For integer operands, the \chpl{\%} and \chpl{/} operators are related by the
following identity:
\begin{chapel}
var q = a / b;
var r = a % b;
writeln(q * b + r == a);    // true
\end{chapel}

It is a compile-time error to take the remainder of a value of
type \chpl{uint(64)} and a value of type \chpl{int(64)}, and vice
versa.

There is an expectation that the predefined modulus operators will be
extended to handle real, imaginary, and complex types in the future.

\subsection{Exponentiation Operators}
\label{Exponentiation_Operators}
\index{operators!exponentiation}
\index{**@\chpl{**}}
\index{operators!**@\chpl{**}}

The exponentiation operators are predefined as follows:
\begin{chapel}
proc **(a: int(8), b: int(8)): int(8)
proc **(a: int(16), b: int(16)): int(16)
proc **(a: int(32), b: int(32)): int(32)
proc **(a: int(64), b: int(64)): int(64)

proc **(a: uint(8), b: uint(8)): uint(8)
proc **(a: uint(16), b: uint(16)): uint(16)
proc **(a: uint(32), b: uint(32)): uint(32)
proc **(a: uint(64), b: uint(64)): uint(64)

proc **(a: real(32), b: real(32)): real(32)
proc **(a: real(64), b: real(64)): real(64)
\end{chapel}
For each of these definitions that return a value, the result is the
value of the first operand raised to the power of the second operand.

It is a compile-time error to take the exponent of a value of
type \chpl{uint(64)} by a value of type \chpl{int(64)}, and vice
versa.

There is an expectation that the predefined exponentiation operators
will be extended to handle imaginary and complex types in the future.

\section{Bitwise Operators}
\label{Bitwise_Operators}
\index{operators!bitwise}

This section describes the predefined bitwise operators.  These
operators can be redefined over different types using operator
overloading~(\rsec{Function_Overloading}).

\subsection{Bitwise Complement Operators}
\label{Bitwise_Complement_Operators}
\index{operators!bitwise!complement}
\index{\~@\chpl{\~}}
\index{operators!\~@\chpl{\~}}

The bitwise complement operators are predefined as follows:
\begin{chapel}
proc ~(a: bool): bool

proc ~(a: int(8)): int(8)
proc ~(a: int(16)): int(16)
proc ~(a: int(32)): int(32)
proc ~(a: int(64)): int(64)

proc ~(a: uint(8)): uint(8)
proc ~(a: uint(16)): uint(16)
proc ~(a: uint(32)): uint(32)
proc ~(a: uint(64)): uint(64)
\end{chapel}
For each of these definitions, the result is the bitwise complement of
the operand.

\subsection{Bitwise And Operators}
\label{Bitwise_And_Operators}
\index{operators!bitwise!and}
\index{&@\chpl{&}}
\index{operators!&@\chpl{&}}

The bitwise and operators are predefined as follows:
\begin{chapel}
proc &(a: bool, b: bool): bool

proc &(a: int(?w), b: int(w)): int(w)
proc &(a: uint(?w), b: uint(w)): uint(w)

proc &(a: int(?w), b: uint(w)): uint(w)
proc &(a: uint(?w), b: int(w)): uint(w)
\end{chapel}
For each of these definitions, the result is
computed by applying the logical and operation to the bits of the
operands.

Chapel allows mixing signed and unsigned integers of the same size
when passing them as arguments to bitwise and.
In the mixed case the result is of the same size as the arguments
and is unsigned.
No run-time error is issued, even if the apparent sign changes as the
required conversions are performed.

\begin{rationale}
The mathematical meaning of integer arguments
is discarded when they are passed to bitwise operators.
Instead the arguments are treated simply as bit vectors.
The bit-vector meaning is preserved when converting
between signed and unsigned of the same size.
The choice of unsigned over signed as the result type in the mixed case
reflects the semantics of standard C.
\end{rationale}

\subsection{Bitwise Or Operators}
\label{Bitwise_Or_Operators}
\index{operators!bitwise!or}
\index{|@\chpl{|}}
\index{operators!|@\chpl{|}}

The bitwise or operators are predefined as follows:
\begin{chapel}
proc |(a: bool, b: bool): bool

proc |(a: int(?w), b: int(w)): int(w)
proc |(a: uint(?w), b: uint(w)): uint(w)

proc |(a: int(?w), b: uint(w)): uint(w)
proc |(a: uint(?w), b: int(w)): uint(w)
\end{chapel}

For each of these definitions, the result is
computed by applying the logical or operation to the bits of the
operands.
Chapel allows mixing signed and unsigned integers of the same size
when passing them as arguments to bitwise or.
No run-time error is issued, even if the apparent sign changes as the
required conversions are performed.

\begin{rationale}
The same as for bitwise and (\rsec{Bitwise_And_Operators}).
\end{rationale}

\subsection{Bitwise Xor Operators}
\label{Bitwise_Xor_Operators}
\index{operators!bitwise!exclusive or}
\index{^@\chpl{^}}
\index{operators!^@\chpl{^}}

The bitwise xor operators are predefined as follows:
\begin{chapel}
proc ^(a: bool, b: bool): bool

proc ^(a: int(?w), b: int(w)): int(w)
proc ^(a: uint(?w), b: uint(w)): uint(w)

proc ^(a: int(?w), b: uint(w)): uint(w)
proc ^(a: uint(?w), b: int(w)): uint(w)
\end{chapel}

For each of these definitions, the result is
computed by applying the XOR operation to the bits of the operands.
Chapel allows mixing signed and unsigned integers of the same size
when passing them as arguments to bitwise xor.
No run-time error is issued, even if the apparent sign changes as the required
conversions are performed.

\begin{rationale}
The same as for bitwise and (\rsec{Bitwise_And_Operators}).
\end{rationale}

\section{Shift Operators}
\label{Shift_Operators}
\index{operators!shift}
\index{<<@\chpl{<<}}
\index{operators!<<@\chpl{<<}}
\index{>>@\chpl{>>}}
\index{operators!>>@\chpl{>>}}

This section describes the predefined shift operators.  These
operators can be redefined over different types using operator
overloading~(\rsec{Function_Overloading}).

The shift operators are predefined as follows:
\begin{chapel}
proc <<(a: int(8), b): int(8)
proc <<(a: int(16), b): int(16)
proc <<(a: int(32), b): int(32)
proc <<(a: int(64), b): int(64)

proc <<(a: uint(8), b): uint(8)
proc <<(a: uint(16), b): uint(16)
proc <<(a: uint(32), b): uint(32)
proc <<(a: uint(64), b): uint(64)

proc >>(a: int(8), b): int(8)
proc >>(a: int(16), b): int(16)
proc >>(a: int(32), b): int(32)
proc >>(a: int(64), b): int(64)

proc >>(a: uint(8), b): uint(8)
proc >>(a: uint(16), b): uint(16)
proc >>(a: uint(32), b): uint(32)
proc >>(a: uint(64), b): uint(64)
\end{chapel}
The type of the second actual argument must be any integral type.

The \chpl{<<} operator shifts the bits of \chpl{a} left by the
integer \chpl{b}.  The new low-order bits are set to zero.

The \chpl{>>} operator shifts the bits of \chpl{a} right by the
integer \chpl{b}.  When \chpl{a} is negative, the new high-order bits
are set to one; otherwise the new high-order bits are set to zero.

The value of \chpl{b} must be non-negative.

\section{Logical Operators}
\label{Logical_Operators}
\index{operators!logical}

This section describes the predefined logical operators.  These
operators can be redefined over different types using operator
overloading~(\rsec{Function_Overloading}).

\subsection{The Logical Negation Operator}
\label{Logical_Negation_Operators}
\index{operators!logical!not}
\index{\!@\chpl{!}}
\index{operators!\!@\chpl{!}}

The logical negation operator is predefined as follows:
\begin{chapel}
proc !(a: bool): bool
\end{chapel}
The result is the logical negation of the operand.

\subsection{The Logical And Operator}
\label{Logical_And_Operators}
\index{operators!logical!and}
\index{&&@\chpl{&&}}
\index{operators!&&@\chpl{&&}}

The logical and operator is predefined over bool type.  It returns
true if both operands evaluate to true; otherwise it returns false.
If the first operand evaluates to false, the second operand is not
evaluated and the result is false.
%% hilde sez: In the interest of supporting parallel execution, we should leave
%% unspecified whether the right operand is evaluated.
%% Where sufficient processing resources are available, it is faster on average
%% to evaluate both the left and right operands and perform the conjunction or
%% disjunction than to block until the value of the left operand is known and
%% only then commence to evaluate the right operand.

The logical and operator over expressions \chpl{a} and \chpl{b} given
by
\begin{chapel}
a && b
\end{chapel}
is evaluated as the expression
\begin{chapel}
if isTrue(a) then isTrue(b) else false
\end{chapel}

The function \chpl{isTrue} is predefined over bool type as follows:
\begin{chapel}
proc isTrue(a:bool) return a;
\end{chapel}
Overloading the logical and operator over other types is accomplished
by overloading the \chpl{isTrue} function over other types.

\subsection{The Logical Or Operator}
\label{Logical_Or_Operators}
\index{operators!logical!or}
\index{||@\chpl{||}}
\index{operators!||@\chpl{||}}


The logical or operator is predefined over bool type.  It returns
true if either operand evaluate to true; otherwise it returns false.
If the first operand evaluates to true, the second operand is not
evaluated and the result is true.

The logical or operator over expressions \chpl{a} and \chpl{b} given
by
\begin{chapel}
a || b
\end{chapel}
is evaluated as the expression
\begin{chapel}
if isTrue(a) then true else isTrue(b)
\end{chapel}

The function \chpl{isTrue} is predefined over bool type as described
in~\rsec{Logical_And_Operators}.  Overloading the logical or operator
over other types is accomplished by overloading the \chpl{isTrue}
function over other types.

\section{Relational Operators}
\label{Relational_Operators}
\index{operators!relational}

This section describes the predefined relational operators.  These
operators can be redefined over different types using operator
overloading~(\rsec{Function_Overloading}).

\subsection{Ordered Comparison Operators}
\label{Ordered_Comparison_Operators}

\index{operators!less than}
\index{<@\chpl{<}}
\index{operators!<@\chpl{<}}
The ``less than'' comparison operators are predefined over numeric
types as follows:
\begin{chapel}
proc <(a: int(8), b: int(8)): bool
proc <(a: int(16), b: int(16)): bool
proc <(a: int(32), b: int(32)): bool
proc <(a: int(64), b: int(64)): bool

proc <(a: uint(8), b: uint(8)): bool
proc <(a: uint(16), b: uint(16)): bool
proc <(a: uint(32), b: uint(32)): bool
proc <(a: uint(64), b: uint(64)): bool

proc <(a: real(32), b: real(32)): bool
proc <(a: real(64), b: real(64)): bool

proc <(a: imag(32), b: imag(32)): bool
proc <(a: imag(64), b: imag(64)): bool
\end{chapel}
The result of \chpl{a < b} is true if \chpl{a} is less than \chpl{b};
otherwise the result is false.

\index{operators!greater than}
\index{>@\chpl{>}}
\index{operators!>@\chpl{>}}
The ``greater than'' comparison operators are predefined over numeric
types as follows:
\begin{chapel}
proc >(a: int(8), b: int(8)): bool
proc >(a: int(16), b: int(16)): bool
proc >(a: int(32), b: int(32)): bool
proc >(a: int(64), b: int(64)): bool

proc >(a: uint(8), b: uint(8)): bool
proc >(a: uint(16), b: uint(16)): bool
proc >(a: uint(32), b: uint(32)): bool
proc >(a: uint(64), b: uint(64)): bool

proc >(a: real(32), b: real(32)): bool
proc >(a: real(64), b: real(64)): bool

proc >(a: imag(32), b: imag(32)): bool
proc >(a: imag(64), b: imag(64)): bool
\end{chapel}
The result of \chpl{a > b} is true if \chpl{a} is greater
than \chpl{b}; otherwise the result is false.

\index{operators!less than or equal}
\index{<=@\chpl{<=}}
\index{operators!<=@\chpl{<=}}
The ``less than or equal to'' comparison operators are predefined over
numeric types as follows:
\begin{chapel}
proc <=(a: int(8), b: int(8)): bool
proc <=(a: int(16), b: int(16)): bool
proc <=(a: int(32), b: int(32)): bool
proc <=(a: int(64), b: int(64)): bool

proc <=(a: uint(8), b: uint(8)): bool
proc <=(a: uint(16), b: uint(16)): bool
proc <=(a: uint(32), b: uint(32)): bool
proc <=(a: uint(64), b: uint(64)): bool

proc <=(a: real(32), b: real(32)): bool
proc <=(a: real(64), b: real(64)): bool

proc <=(a: imag(32), b: imag(32)): bool
proc <=(a: imag(64), b: imag(64)): bool
\end{chapel}
The result of \chpl{a <= b} is true if \chpl{a} is less than or equal
to \chpl{b}; otherwise the result is false.

\index{operators!greater than or equal}
\index{>=@\chpl{>=}}
\index{operators!>=@\chpl{>=}}
The ``greater than or equal to'' comparison operators are predefined
over numeric types as follows:
\begin{chapel}
proc >=(a: int(8), b: int(8)): bool
proc >=(a: int(16), b: int(16)): bool
proc >=(a: int(32), b: int(32)): bool
proc >=(a: int(64), b: int(64)): bool

proc >=(a: uint(8), b: uint(8)): bool
proc >=(a: uint(16), b: uint(16)): bool
proc >=(a: uint(32), b: uint(32)): bool
proc >=(a: uint(64), b: uint(64)): bool

proc >=(a: real(32), b: real(32)): bool
proc >=(a: real(64), b: real(64)): bool

proc >=(a: imag(32), b: imag(32)): bool
proc >=(a: imag(64), b: imag(64)): bool
\end{chapel}
The result of \chpl{a >= b} is true if \chpl{a} is greater than or
equal to \chpl{b}; otherwise the result is false.

The ordered comparison operators are predefined over strings as follows:
\begin{chapel}
proc <(a: string, b: string): bool
proc >(a: string, b: string): bool
proc <=(a: string, b: string): bool
proc >=(a: string, b: string): bool
\end{chapel}
Comparisons between strings are defined based on the ordering of the
character set used to represent the string, which is applied
elementwise to the string's characters in order.


\subsection{Equality Comparison Operators}
\label{Equality_Comparison_Operators}
\index{operators!equality}
\index{==@\chpl{==}}
\index{operators!==@\chpl{==}}
\index{"!=@\chpl{"\"!=}}
\index{operators!"!=@\chpl{"\"!=}}

The equality comparison operators \chpl{==} and \chpl{\!=} are predefined over bool and the
numeric types as follows:
\begin{chapel}
proc ==(a: int(8), b: int(8)): bool
proc ==(a: int(16), b: int(16)): bool
proc ==(a: int(32), b: int(32)): bool
proc ==(a: int(64), b: int(64)): bool

proc ==(a: uint(8), b: uint(8)): bool
proc ==(a: uint(16), b: uint(16)): bool
proc ==(a: uint(32), b: uint(32)): bool
proc ==(a: uint(64), b: uint(64)): bool

proc ==(a: real(32), b: real(32)): bool
proc ==(a: real(64), b: real(64)): bool

proc ==(a: imag(32), b: imag(32)): bool
proc ==(a: imag(64), b: imag(64)): bool

proc ==(a: complex(64), b: complex(64)): bool
proc ==(a: complex(128), b: complex(128)): bool

proc !=(a: int(8), b: int(8)): bool
proc !=(a: int(16), b: int(16)): bool
proc !=(a: int(32), b: int(32)): bool
proc !=(a: int(64), b: int(64)): bool

proc !=(a: uint(8), b: uint(8)): bool
proc !=(a: uint(16), b: uint(16)): bool
proc !=(a: uint(32), b: uint(32)): bool
proc !=(a: uint(64), b: uint(64)): bool

proc !=(a: real(32), b: real(32)): bool
proc !=(a: real(64), b: real(64)): bool

proc !=(a: imag(32), b: imag(32)): bool
proc !=(a: imag(64), b: imag(64)): bool

proc !=(a: complex(64), b: complex(64)): bool
proc !=(a: complex(128), b: complex(128)): bool
\end{chapel}
The result of \chpl{a == b} is true if \chpl{a} and \chpl{b} contain
the same value; otherwise the result is false.  The result of \chpl{a
\!= b} is equivalent to \chpl{\!(a == b)}.

The equality comparison operators are predefined over classes as
follows:
\begin{chapel}
proc ==(a: object, b: object): bool
proc !=(a: object, b: object): bool
\end{chapel}
The result of \chpl{a == b} is true if \chpl{a} and \chpl{b} reference
the same storage location; otherwise the result is false.  The result
of \chpl{a \!= b} is equivalent to \chpl{\!(a == b)}.

Default equality comparison operators are generated for records if the
user does not define them.  These operators are described
in~\rsec{Record_Comparison_Operators}.

\index{== (string)@\chpl{==} (string)}
\index{operators!== (string)@\chpl{==} (string)}
\index{"!= (string)@\chpl{"\"!=} (string)}
\index{operators!"!= (string)@\chpl{"\"!=} (string)}
The equality comparison operators are predefined over strings as
follows:
\begin{chapel}
proc ==(a: string, b: string): bool
proc !=(a: string, b: string): bool
\end{chapel}
The result of \chpl{a == b} is true if the sequence of characters
in \chpl{a} matches exactly the sequence of characters in \chpl{b};
otherwise the result is false.  The result of \chpl{a \!= b} is
equivalent to \chpl{\!(a == b)}.

\section{Miscellaneous Operators}
\label{Miscellaneous_Operators}

This section describes several miscellaneous operators.  These
operators can be redefined over different types using operator
overloading~(\rsec{Function_Overloading}).

\subsection{The String Concatenation Operator}
\label{The_String_Concatenation_Operator}
\index{operators!string concatenation}
\index{operators!concatenation!string}
\index{operators!+ (string)@\chpl{+} (string)}

The string concatenation operator \chpl{+} is predefined over numeric, boolean,
and enumerated types with strings. It casts its operands to string type and
concatenates them together.

\begin{chapelexample}{string-concat.chpl}
The code
\begin{chapelnoprint}
var i:int = 3;
writeln(
\end{chapelnoprint}
\begin{chapel}
"result: "+i
\end{chapel}
\begin{chapelnoprint}
);
\end{chapelnoprint}
\begin{chapeloutput}
result: 3
\end{chapeloutput}
where \chpl{i} is an integer appends the string representation of \chpl{i} to the
string literal \chpl{"result: "}.  If \chpl{i} is \chpl{3}, then the resulting string
would be \chpl{"result: 3"}.
\begin{chapelnoprint}
\end{chapelnoprint}
\end{chapelexample}

\subsection{The By Operator}
\label{The_By_Operator}
\index{by@\chpl{by}}
\index{operators!by@\chpl{by}}

The operator \chpl{by} is predefined on ranges and rectangular domains.
It is described in~\rsec{By_Operator_For_Ranges} for ranges
and~\rsec{Domain_Striding} for domains.

\subsection{The Range Count Operator}
\label{The_Range_Count_Operator}
\index{operators!range!count}
\index{#@\chpl{#}}
\index{operators!#@\chpl{#}}

The operator \chpl{#} is predefined on ranges. It is described
in ~\rsec{Count_Operator}.

\section{Let Expressions}
\label{Let_Expressions}
\index{let@\chpl{let}}
\index{operators!let@\chpl{let}}

A let expression allows variables to be declared at the expression
level and used within that expression.  The syntax of a let expression
is given by:
\begin{syntax}
let-expression:
  `let' variable-declaration-list `in' expression
\end{syntax}
The scope of the variables is the let-expression.
\begin{chapelexample}{let.chpl}
Let expressions are useful for defining variables in the context of
an expression.  In the code
\begin{chapelnoprint}
  var a = 4;
  var b = 5;
  var l =
\end{chapelnoprint}
\begin{chapel}
  let x: real = a*b, y = x*x in 1/y
\end{chapel}
the value determined by \chpl{a*b} is computed and converted to type
real if it is not already a real.  The square of the real is then
stored in \chpl{y} and the result of the expression is the reciprocal
of that value.
\begin{chapelnoprint}
  ;
  writeln(l);
\end{chapelnoprint}
\begin{chapeloutput}
0.0025
\end{chapeloutput}
\end{chapelexample}

\section{Conditional Expressions}
\label{Conditional_Expressions}
\index{conditional expressions}
\index{expressions!conditional}
\index{expressions!if-then-else}
\index{if@\chpl{if}}
\index{then@\chpl{then}}
\index{else@\chpl{else}}

A conditional expression is given by the following syntax:
\begin{syntax}
if-expression:
  `if' expression `then' expression `else' expression
  `if' expression `then' expression
\end{syntax}
The conditional expression is evaluated in two steps.  First, the
expression following the \chpl{if} keyword is evaluated.  Then, if the
expression evaluated to true, the expression following the \chpl{then}
keyword is evaluated and taken to be the value of this expression.
Otherwise, the expression following the \chpl{else} keyword is
evaluated and taken to be the value of this expression.  In both
cases, the unselected expression is not evaluated.

The `else' clause can be omitted only when the conditional expression
is nested immediately inside a for or forall expression.  Such an expression
is used to filter predicates as described
in~\rsec{Filtering_Predicates_For} and~\rsec{Filtering_Predicates_Forall},
respectively.

\begin{chapelexample}{condexp.chpl}
This example shows how if-then-else can be used in a context in which an
expression is expected.
\begin{chapel}
writehalf(8);
writehalf(21);
writehalf(1000);

proc writehalf(i: int) {
  var half = if (i % 2) then i/2 +1 else i/2;
  writeln("Half of ",i," is ",half); 
}
\end{chapel}
\begin{chapelprintoutput}
Half of 8 is 4\\
Half of 21 is 11\\
Half of 1000 is 500\\
\end{chapelprintoutput}
\end{chapelexample}

\section{For Expressions}
\label{For_Expressions}
\index{for@\chpl{for}}
\index{expressions!for@\chpl{for}}

A for expression is given by the following syntax:
\begin{syntax}
for-expression:
  `for' index-var-declaration `in' iteratable-expression `do' expression
  `for' iteratable-expression `do' expression
\end{syntax}
The for expression executes a for loop (\rsec{The_For_Loop}),
evaluates the body expression on each iteration of the loop,
and returns the resulting values as a collection.
The size and shape of that collection
are determined by the iteratable-expression.

\subsection{Filtering Predicates in For Expressions}
\label{Filtering_Predicates_For}
\index{for@\chpl{for}!filtering predicates}
\index{expressions!for@\chpl{for}!filtering predicates}

A conditional expression that is immediately enclosed in a for
expression and does not require an else clause filters the iterations of the for expression.
The iterations for which the condition does not hold
are not reflected in the result of the for expression.

\begin{chapelexample}{yieldPredicates.chpl}
The code
\begin{chapel}
var A = for i in 1..10 do if i % 3 != 0 then i;
\end{chapel}
\begin{chapelpost}
writeln(A);
\end{chapelpost}
\begin{chapeloutput}
1 2 4 5 7 8 10
\end{chapeloutput}
declares an array A that is initialized to the integers between
1 and 10 that are not divisible by 3.
\end{chapelexample}

\cleardoublepage
\sekshun{Statements}
\label{Statements}

\index{statement}

Chapel is an imperative language with statements that may have side
effects.  Statements allow for the sequencing of program execution.
They are as follows:
\begin{syntax}
statement:
  block-statement
  expression-statement
  assignment-statement
  swap-statement
  conditional-statement
  select-statement
  while-do-statement
  do-while-statement
  for-statement
  label-statement
  break-statement
  continue-statement
  param-for-statement
  return-statement
  yield-statement
  module-declaration-statement
  function-declaration-statement
  method-declaration-statement
  type-declaration-statement
  variable-declaration-statement
  remote-variable-declaration-statement
  use-statement
  type-select-statement
  empty-statement
  parallel-statement
  on-statement
\end{syntax}

The declaration statements are discussed in the sections that define
what they declare.  Module declaration statements are defined
in~\rsec{Modules}.  Function declaration statements are defined
in~\rsec{Functions}.  Method declaration statements are defined
in~\rsec{Class_Methods}.  Type declaration statements are defined
in~\rsec{Types}.  Variable declaration statements are defined
in~\rsec{Variables}.  Remote variable declaration statements are
defined in~\rsec{remote_variable_declarations}.  Tuple variable
declaration statements are defined
in~\rsec{Variable_Declarations_in_a_Tuple}.  Return statements are
defined in~\rsec{The_Return_Statement}.  Yield statements are defined
in~\rsec{The_Yield_Statement}.  The \sntx{parallel-statement} consists
of statements that create or limit parallelism.  These statements are
described in~\rsec{Task_Parallelism_and_Synchronization}
and~\rsec{Data_Parallelism}.  The \sntx{on-statement} is defined
in~\rsec{On}.  The compiler error statement is defined
in~\rsec{User_Defined_Compiler_Errors}.

\subsection{Blocks}
\label{Blocks}

\index{block}

A block is a statement or a possibly empty list of statements that
form their own scope.  A block is given by
\begin{syntax}
block-statement:
  { statements[OPT] }

statements:
  statement
  statement statements
\end{syntax}

Variables defined within a block are local
variables~(\rsec{Local_Variables}).

The statements within a block are executed serially unless the block
is in a cobegin statement~(\rsec{Cobegin}).

\subsection{Expression Statements}
\label{Expression_Statements}

\index{expression statement}
\index{expression!as a statement}
The expression statement evaluates an expression solely for side
effects. The syntax for an expression statement is given by
\begin{syntax}
expression-statement:
  expression ;
\end{syntax}

\subsection{Assignment Statements}
\label{Assignment_Statements}
\index{assignment}

An assignment statement assigns the value of an expression to another
expression that can appear on the left-hand side of the operator, for
example, a variable.  Assignment statements are given by

\index{=@\chpl{=}}
\index{+=@\chpl{+=}}
\index{-=@\chpl{-=}}
\index{*=@\chpl{*=}}
\index{/=@\chpl{/=}}
\index{\%=@\chpl{\%=}}
\index{**=@\chpl{**=}}
\index{&=@\chpl{&=}}
\index{|=@\chpl{|=}}
\index{^=@\chpl{^=}}
\index{||=@\chpl{||=}}
\index{&&=@\chpl{&&=}}
\index{<<=@\chpl{<<=}}
\index{>>=@\chpl{>>=}}
\begin{syntax}
assignment-statement:
  lvalue-expression assignment-operator expression

assignment-operator: one of
   = += -= *= /= %= **= &= |= ^= &&= ||= <<= >>=
\end{syntax}

The expression on the left-hand side of the assignment operator must
be a valid lvalue~(\rsec{lvalue}).  It is evaluated before the
expression on the right-hand side of the assignment operator, which
can be any expression.

The assignment operators that contain a binary operator as a prefix is
a short-hand for applying the binary operator to the left and
right-hand side expressions and then assigning the value of that
application to the already evaluated left-hand side.  Thus, for
example, \chpl{x += y} is equivalent to \chpl{x = x + y} where the
expression \chpl{x} is evaluated once.

In a compound assignment, a cast to the type on the left-hand side is
inserted before the simple assignment if the operator is a shift or
both the right-hand side expression can be assigned to the left-hand
side expression and the type of the left-hand side is a primitive
type.

\begin{rationale}
This cast is necessary to handle \chpl{+=} where the type of the
left-hand side is, for example, \chpl{int(8)} because the \chpl{+}
operator is defined on \chpl{int(32)}, not \chpl{int(8)}.
\end{rationale}

Values of one primitive or enumerated type can be assigned to another
primitive or enumerated type if an implicit coercion exists between
those types~(\rsec{Implicit_Conversions}).

The validity and semantics of assigning between
classes~(\rsec{Class_Assignment}), records~(\rsec{Record_Assignment}),
unions~(\rsec{Union_Assignment}), tuples~(\rsec{Tuple_Assignment}),
ranges~(\rsec{Range_Assignment}),
domains~(\rsec{Domain_Assignment}), and arrays~(\rsec{Array_Assignment})
is discussed in these later sections.

\subsection{The Swap Statement}
\label{The_Swap_Statement}
\index{swap!statement}
\index{swap!operator}
The swap statement indicates to swap the values in the expressions
on either side of the swap operator.  Since both expressions are assigned
to, each must be a valid lvalue expression~(\rsec{lvalue}).
\begin{syntax}
swap-statement:
  lvalue-expression swap-operator lvalue-expression

swap-operator:
  <=>
\end{syntax}

To implement the swap operation, the compiler uses temporary variables
as necessary.

\begin{example}
The following swap statement
\begin{chapel}
var a, b: real;

a <=> b;
\end{chapel}
is semantically equivalent to:
\begin{chapel}
const t = b;
b = a;
a = t;
\end{chapel}
\end{example}

\subsection{The Conditional Statement}
\label{The_Conditional_Statement}

\index{if@\chpl{if}}
\index{then@\chpl{then}}
\index{else@\chpl{else}}
\index{conditional!statement}
The conditional statement allows execution to choose between two
statements based on the evaluation of an expression of \chpl{bool}
type. The syntax for a conditional statement is given by
\begin{syntax}
conditional-statement:
  `if' expression `then' statement else-part[OPT]
  `if' expression block-statement else-part[OPT]

else-part:
  `else' statement
\end{syntax}

A conditional statement evaluates an expression of bool type. If the
expression evaluates to true, the first statement in the conditional
statement is executed.  If the expression evaluates to false and the
optional else-clause exists, the statement following the
\chpl{else} keyword is executed.

If the expression is a parameter, the conditional statement is folded
by the compiler. If the expression evaluates to true, the first
statement replaces the conditional statement. If the expression
evaluates to false, the second statement, if it exists, replaces the
conditional statement; if the second statement does not exist, the
conditional statement is removed.

\index{conditional statement!dangling else}
If the statement that immediately follows the optional \chpl{then}
keyword is a conditional statement and it is not in a block, the
else-clause is bound to the nearest preceding conditional statement
without an else-clause.

Each statement embedded in the {\em conditional-statement} has its own
scope whether or not an explicit block surrounds it.

\subsection{The Select Statement}
\label{The_Select_Statement}

\index{select@\chpl{select}}
\index{when@\chpl{when}}

The select statement is a multi-way variant of the conditional
statement.  The syntax is given by:
\begin{syntax}
select-statement:
  `select' expression { when-statements }

when-statements:
  when-statement
  when-statement when-statements

when-statement:
  `when' expression-list `do' statement
  `when' expression-list block-statement
  `otherwise' statement

expression-list:
  expression
  expression , expression-list
\end{syntax}
The expression that follows the keyword \chpl{select}, the select
expression, is compared with the list of expressions following the
keyword \chpl{when}, the case expressions, using the equality
operator \chpl{==}.  If the expressions cannot be compared with the
equality operator, a compile-time error is generated.  The first case
expression that contains an expression where that comparison
is \chpl{true} will be selected and control transferred to the
associated statement.  If the comparison is always \chpl{false}, the
statement associated with the keyword \chpl{otherwise}, if it exists,
will be selected and control transferred to it.  There may be at most
one \chpl{otherwise} statement and its location within the select
statement does not matter.

Each statement embedded in the {\em when-statement} has its own scope
whether or not an explicit block surrounds it.

\subsection{The While and Do While Loops}
\label{The_While_and_Do_While_Loops}

\index{while loops}
\index{while@\chpl{while}}

There are two variants of the while loop in Chapel.  The syntax of the
while-do loop is given by:
\begin{syntax}
while-do-statement:
  `while' expression `do' statement
  `while' expression block-statement
\end{syntax}
The syntax of the do-while loop is given by:
\begin{syntax}
do-while-statement:
  `do' statement `while' expression ;
\end{syntax}
In both variants, the expression evaluates to a value of type \chpl{bool}
which determines when the loop terminates and control continues with
the statement following the loop.

The while-do loop is executed as follows:
\begin{enumerate}
\item The expression is evaluated.
\item If the expression evaluates to \chpl{false}, the statement is
  not executed and control continues to the statement following the
  loop.
\item If the expression evaluates to \chpl{true}, the statement is
  executed and control continues to step 1, evaluating the expression
  again.
\end{enumerate}

The do-while loop is executed as follows:
\begin{enumerate}
\item The statement is executed.
\item The expression is evaluated.
\item If the expression evaluates to \chpl{false}, control continues
  to the statement following the loop.
\item If the expression evaluates to \chpl{true}, control continues to
  step 1 and the the statement is executed again.
\end{enumerate}
In this second form of the loop, note that the statement is executed
unconditionally the first time.

\subsection{The For Loop}
\label{The_For_Loop}

\index{for@\chpl{for}}
\index{for loops}

The for loop iterates over ranges, domains, arrays, iterators, or any
class that implements an iterator named \chpl{these}.  The syntax of
the for loop is given by:
\begin{syntax}
for-statement:
  `for' loop-control-part loop-body-part

loop-control-part:
  index-expression `in' iterator-expression
  iterator-expression

loop-body-part:
  `do' statement
  block-statement

index-expression:
  expression

iterator-expression:
  expression
\end{syntax}

The index-expression declares new variables for the scope of the loop.
It may specify a new identifier.  Alternatively, the index-expression
may specify multiple identifiers grouped using a tuple notation in
order to destructure the values returned by the iterator expression,
as described in~\rsec{Indices_in_a_Tuple}.

The index-expression is optional and may be omitted if the indices do
not need to be referenced in the loop.

If the iterator-expression is a tuple delimited by parentheses, the
components of the tuple must support iteration, e.g., a tuple of
arrays, and those components are iterated over using a zipper
iteration defined in~\rsec{Zipper_Iteration}.  If the
iterator-expression is a tuple delimited by brackets, the components
of the tuple must support iteration and these components are iterated
over using a tensor product iteration defined
in~\rsec{Tensor_Product_Iteration}.

\subsubsection{Zipper Iteration}
\label{Zipper_Iteration}
\index{zipper iteration}

When multiple iterators are iterated over in a zipper context, on each
iteration, each expression is iterated over, the values are returned
by the iterators in a tuple and assigned to the index, and the
statement is executed.

The shape of each iterator, the rank and the extents in each
dimension, must be identical.

\begin{example}
The output of
\begin{chapel}
for (i, j) in (1..3, 4..6) do
  write(i, " ", j, " ");
\end{chapel}
is ``1 4 2 5 3 6 ''.
\end{example}

\subsubsection{Tensor Product Iteration}
\label{Tensor_Product_Iteration}
\index{tensor product iterator}
When multiple iterators are iterated over in a tensor product context,
they are iterated over as if they were nested in distinct for loops.
There is no constraint on the iterators as there is in the zipper
context.

\begin{example}
The output of
\begin{chapel}
for (i, j) in [1..3, 4..6] do
  write(i, " ", j, " ");
\end{chapel}
is ``1 4 1 5 1 6 2 4 2 5 2 6 3 4 3 5 3 6 ''. The statement is
equivalent to
\begin{chapel}
for i in 1..3 do
  for j in 4..6 do
    write(i, " ", j, " ");
\end{chapel}
\end{example}

\subsubsection{Parameter For Loops}
\label{Parameter_For_Loops}

\index{for loops!parameters}
\index{for@\chpl{for}}
\index{param@\chpl{param}}

Parameter for loops are unrolled by the compiler so that the index
variable is a parameter rather than a variable.  The syntax for a
parameter for loop statement is given by:
\begin{syntax}
param-iterator-expression:
  range-literal
  range-literal `by' integer-literal

param-for-statement:
  `for' `param' identifier `in' param-iterator-expression `do' statement
  `for' `param' identifier `in' param-iterator-expression block-statement
\end{syntax}
Parameter for loops are restricted to iteration over range literals
with an optional by expression where the bounds and stride must be
parameters.  The loop is then unrolled for each iteration.

\subsection{The Label, Break, and Continue Statements}
\label{Label_Break_Continue}
\index{label@\chpl{label}}
\index{break@\chpl{break}}
\index{continue@\chpl{continue}}

The label-statement is used to name a specific loop which can then
be the target of a break- or continue-statement.  If a break-
or continue-statement has no label, the target is the lexically
inner-most loop. Labels can only be given to for-, while-do- and
do-while-statements.

The syntax for label, break, and continue statements is given by:
\begin{syntax}
label-statement:
  `label' identifier statement

break-statement:
  `break' identifier[OPT] ;

continue-statement:
  `continue' identifier[OPT] ;
\end{syntax}

If a break-statement is encountered, control will be transferred to
after the associated loop.  If a continue-statement is encountered,
control will be transferred to the end of the associated loop, but
still inside the loop.  Break-statements cannot be used to break out of
parallel loops.  Neither break- nor continue-statements can
cross out of cobegin-, coforall-, begin-, or sync-statements.

\begin{example}
In the following code, the index of the first element in each row of
\chpl{A} that is equal to \chpl{findVal} is printed.  Once a match is
found, the continue statement is executed causing the outer loop to
move to the next row.
\begin{chapel}
label outer for i in 1..n {
  for j in 1..n {
    if A[i, j] == findVal {
      writeln("index: ", (i, j), " matches.");
      continue outer;
    }
  }
}
\end{chapel}
\end{example}

\subsection{The Use Statement}
\label{The_Use_Statement}
\index{use@\chpl{use}}
\index{modules!using}

The use statement makes symbols in modules available without accessing
them via the module name.  The syntax of the use statement is given
by:
\begin{syntax}
use-statement:
  `use' module-name-list ;

module-name-list:
  module-name
  module-name , module-name-list

module-name:
  identifier
  module-name . module-name
\end{syntax}
The use statement makes symbols in each listed module's scope available
from the scope where the use statement occurs.

Symbols injected by a use statement are at an outer scope from those
defined directly in the scope where the use statement occurs, but at a
nearer scope than symbols defined in the scope containing the scope where
the use statement occurs.

If used modules themselves use other modules, symbols are scoped according
the depth of use statements followed to find them. It is an error for two
variables, types, or modules to be defined at the same depth.

\begin{openissue}
There is an expectation that this statement will be extended to allow
the programmer to restrict which symbols are 'used' as well as to
rename symbols that are used.
\end{openissue}

\subsection{The Type Select Statement}
\label{The_Type_Select_Statement}

\index{type select statements}

A type select statement has two uses.  It can be used to determine the
type of a union, as discussed
in~\rsec{The_Type_Select_Statement_and_Unions}.  In its more general
form, it can be used to determine the types of one or more values
using the same mechanisms used to disambiguate function definitions.
It syntax is given by:
\begin{syntax}
type-select-statement:
  `type' `select' expression-list { type-when-statements }

type-when-statements:
  type-when-statement
  type-when-statement type-when-statements

type-when-statement:
  `when' type-list `do' statement
  `when' type-list block-statement
  `otherwise' statement

expression-list:
  expression
  expression , expression-list

type-list:
  type-specifier
  type-specifier , type-list
\end{syntax}

Call the expressions following the keyword \chpl{select}, the select
expressions.  The number of select expressions must be equal to the
number of types following each of the \chpl{when} keywords.  Like the
select statement, one of the statements associated with a \chpl{when}
will be executed.  In this case, that statement is chosen by the
function resolution mechanism.  The select expressions are the actual
arguments, the types following the \chpl{when} keywords are the types
of the formal arguments for different anonymous functions.  The
function that would be selected by function resolution determines the
statement that is executed.  If none of the functions are chosen, the
the statement associated with the keyword \chpl{otherwise}, if it
exists, will be selected.

As with function resolution, this can result in an ambiguous
situation.  Unlike with function resolution, in the event of an
ambiguity, the first statement in the list of when statements is
chosen.

\subsection{The Empty Statement}
\label{The_Empty_Statement}

An empty statement has no effect.  The syntax of an empty statement is
given by
\begin{syntax}
empty-statement:
  ;
\end{syntax}

\cleardoublepage
\sekshun{Modules}
\label{Modules}
\index{modules}

Chapel supports modules to manage name spaces.  A program consists of
one or more modules.  Every symbol, including variables, functions,
and types, is associated with some module.

Module definitions are described in~\rsec{Module_Definitions}.  The
relation between files and modules is described
in~\rsec{Implicit_Modules}.  Nested modules are described
in~\rsec{Nested_Modules}.  The visibility of a module's symbols by
users of the module is described in~\rsec{Visibility_Of_Symbols}.  The execution
of a program and module initialization is described
in~\rsec{Program_Execution}.

\section{Module Definitions}
\label{Module_Definitions}
\index{module@\chpl{module}}
\index{modules!definitions}

A module is declared with the following syntax:
\begin{syntax}
module-declaration-statement:
  privacy-specifier[OPT] `module' module-identifier block-statement

privacy-specifier:
  `private'
  `public'

module-identifier:
  identifier
\end{syntax}

A module's name is specified after the \chpl{module} keyword.
The \sntx{block-statement} opens the module's scope.  Symbols defined
in this block statement are defined in the module's scope and are
called \emph{top-level module symbols}.  The visibility of a module is
defined by its \sntx{privacy-specifier}~(\rsec{Visibility_Of_A_Module}).

Module declaration statements must be top-level statements within a
module.  A module that is declared within another module is called a
nested module~(\rsec{Nested_Modules}).

\section{Files and Implicit Modules}
\label{Implicit_Modules}
\index{modules!and files}

Multiple modules can be defined in the same file and need not bear any
relation to the file in terms of their names.

\begin{chapelexample}{two-modules.chpl}
The following file contains two explicitly named modules
(\rsec{Explicit_Naming}), MX and MY.
\begin{chapel}
module MX {
  var x: string = "Module MX";
  proc printX() {
    writeln(x);
  }
}

module MY {
  var y: string = "Module MY";
  proc printY() {
    writeln(y);
  }
}
\end{chapel}
\begin{chapelpost}
MX.printX();
MY.printY();
\end{chapelpost}
\begin{chapeloutput}
Module MX
Module MY
\end{chapeloutput}
Module MX defines top-level module symbols x and printX, while MY
defines top-level module symbols y and printY.
\end{chapelexample}

For any file that contains top-level statements other than module
declarations, the file itself is treated as the module declaration.
In this case,
\index{implicit modules}
\index{modules!implicit}
the module is implicit and takes its name from the base filename.  In
particular, the module name is defined as the remaining string after
removing the \chpl{.chpl} suffix and any path specification from the
specified filename.  If the resulting name is not a legal Chapel
identifier, it cannot be referenced in a use statement.

\begin{chapelexample}{implicit.chpl}
The following file, named implicit.chpl, defines an implicitly named
module called implicit.
\begin{chapel}
var x: int = 0;
var y: int = 1;

proc printX() {
  writeln(x);
}
proc printY() {
  writeln(y);
}
\end{chapel}
\begin{chapelpost}
printX();
printY();
\end{chapelpost}
\begin{chapeloutput}
0
1
\end{chapeloutput}
Module implicit defines the top-level module symbols x, y, printX, and
printY.
\end{chapelexample}


\section{Nested Modules}
\label{Nested_Modules}
\index{modules!nested}

A nested module is a module that is defined within another module, the
outer module.  Nested modules automatically have access to all of the
symbols in the outer module.  However, the outer module needs to
explicitly use a nested module to have access to its symbols.

A nested module can be used without using the outer module by
explicitly naming the outer module in the use statement.
\begin{chapelexample}{nested-use.chpl}
The code
\begin{chapelpre}
module libsci {
  writeln("Initializing libsci");
  module blas {
    writeln("\\tInitializing blas");
  }
}
module testmain { // used to avoid warnings
}
\end{chapelpre}
\begin{chapel}
use libsci.blas;
\end{chapel}
\begin{chapeloutput}
Initializing libsci
	Initializing blas
\end{chapeloutput}
uses a module named \chpl{blas} that is nested inside a module
named \chpl{libsci}.
\end{chapelexample}

Files with both module declarations and top-level statements result in
nested modules.

\begin{chapelexample}{nested.chpl}
The following file, named nested.chpl, defines an
implicitly named module called nested, with nested modules
MX and MY.
\begin{chapel}
module MX {
  var x: int = 0;
}

module MY {
  var y: int = 0;
}

use MX, MY;

proc printX() {
  writeln(x);
}

proc printY() {
  writeln(y);
}
\end{chapel}
\begin{chapelpost}
printX();
printY();
\end{chapelpost}
\begin{chapeloutput}
0
0
\end{chapeloutput}
\end{chapelexample}


\section{Access of Module Contents}
\label{Access_Of_Module_Contents}
\index{modules!access}

A module's contents can be accessed by code outside of that module
depending on the visibility of the module
itself~(\rsec{Visibility_Of_A_Module}) and the visibility of each
individual symbol~(\rsec{Visibility_Of_Symbols}).  This can be done
via explicit naming~(\rsec{Explicit_Naming}) or the use
statement~(\rsec{Using_Modules}).

\subsection{Visibility Of A Module}
\label{Visibility_Of_A_Module}
\index{modules!access}

A module defined at file scope is visible anywhere. The visibility of a nested
module is subject to the rules of~\rsec{Visibility_Of_Symbols}. There,
the nested module is considered a "symbol defined at the top level
scope" of its outer module.

\subsection{Visibility Of A Module's Symbols}
\label{Visibility_Of_Symbols}
\index{modules!access}

A symbol defined at the top level scope of a module is \emph{visible}
from outside the module when the \sntx{privacy-specifier} of its
definition is \chpl{public} or is omitted (i.e. by default). When a
symbol defined at the top level scope of a module is declared
\chpl{private}, it is not visible outside of that module. A
symbol's visibility inside its module is controlled by normal lexical
scoping and is not affected by its \sntx{privacy-specifier}.  A
module's visible symbols are accessible via explicit
naming~(\rsec{Explicit_Naming}) or the use
statement~(\rsec{Using_Modules}) only where the module's symbol is
visible~(\rsec{Visibility_Of_A_Module}).

\subsection{Explicit Naming}
\label{Explicit_Naming}
\index{modules!explicitly named}

All publicly visible top-level module symbols can be named explicitly
with the following syntax:
\begin{syntax}
module-access-expression:
  module-identifier-list . identifier

module-identifier-list:
  module-identifier
  module-identifier . module-identifier-list

\end{syntax}
This allows two variables that have the same name to be distinguished
based on the name of their module.  Using explicit module naming in a
function call restricts the set of candidate functions to those in the
specified module.

If code refers to symbols that are defined by multiple modules, the
compiler will issue an error.  Explicit naming can be used to
disambiguate the symbols in this case.

\begin{openissue}
It is currently unspecified whether the
first-named module is always at the outermost module level scope, or whether a
scope-search mechanism is used starting at the scope containing the
usage.
\end{openissue}

\begin{chapelexample}{ambiguity.chpl}
In the following example,
\begin{chapel}
module M1 {
  var x: int = 1;
  var y: int = -1;
  proc printX() {
    writeln("M1's x is: ", x);
  }
  proc printY() {
    writeln("M1's y is: ", y);
  }
}
 
module M2 {
  use M3;
  use M1;

  var x: int = 2;

  proc printX() {
    writeln("M2's x is: ", x);
  }

  proc main() {
    M1.x = 4;
    M1.printX();
    writeln(x);
    printX(); // This is not ambiguous
    printY(); // ERROR: This is ambiguous
  }
}

module M3 {
  var x: int = 3;
  var y: int = -3;
  proc printY() {
    writeln("M3's y is: ", y);
  }
}
\end{chapel}
\begin{chapeloutput}
ambiguity.chpl:22: In function 'main':
ambiguity.chpl:27: error: ambiguous call 'printY()'
ambiguity.chpl:34: note: candidates are: printY()
ambiguity.chpl:7: note:                 printY()
\end{chapeloutput}
The call to printX() is not ambiguous because M2's definition shadows
that of M1.  On the other hand, the call to printY() is ambiguous
because it is defined in both M1 and M3.  This will result in a
compiler error.
\end{chapelexample}

\subsection{Using Modules}
\label{Using_Modules}
\index{modules!using}

If a module is visible to the scope in which accessing its symbols is desirable,
then a use statement on that module may be employed.  Use statements
make a module's visible symbols available without requiring them to be
prefixed by the module's name.  For information about use statements in general,
see~\rsec{The_Use_Statement}.

If a type is specified in the \sntx{limitation-clause}, then the type's fields
and methods are treated similarly to the type name.  These fields and methods
cannot be specified in a \sntx{limitation-clause} on their own.

% We need to figure out what to do about functions that return types which due
% to the limitation-clause are not visible without prefix.


\subsection{Module Initialization}
\label{Module_Initialization}
\index{modules!initialization}

Module initialization occurs at program start-up.  All top-level
statements in a module other than function and type declarations are
executed during module initialization.

\begin{chapelexample}{init.chpl}
In the code,
\begin{chapelpre}
proc foo() {
    return 1;
}
\end{chapelpre}
\begin{chapel}
var x = foo();       // executed at module initialization
writeln("Hi!");      // executed at module initialization
proc sayGoodbye {
  writeln("Bye!");   // not executed at module initialization
}
\end{chapel}
\begin{chapeloutput}
Hi!
\end{chapeloutput}
The function foo() will be invoked and its result assigned to x.  Then
``Hi!'' will be printed.
\end{chapelexample}

Module initialization order is discussed
in~\rsec{Module_Initialization_Order}.


\section{Program Execution}
\label{Program_Execution}
\index{program execution}
\index{program initialization}

Chapel programs start by initializing all modules and then executing
the main function~(\rsec{The_main_Function}).

\subsection{The {\em main} Function}
\label{The_main_Function}

\index{main@\chpl{main}}
\index{functions!main@\chpl{main}}
The main function must be called \chpl{main} and must have zero
arguments.  It can be specified with or without parentheses.  In any
Chapel program, there is a single main function that defines the
program's entry point.  If a program defines multiple potential entry
points, the implementation may provide a compiler flag that
disambiguates between main functions in multiple modules.

\begin{craychapel}
In the Cray Chapel compiler implementation, the \emph{--
--main-module} flag can be used to specify the module from which the
main function definition will be used.
\end{craychapel}

\begin{chapelexample}{main-module.chpl}
Because it defines two \chpl{main} functions, the following code will yield an
error unless a main module is specified on the command line.
\begin{chapel}
module M1 {
  const x = 1;
  proc main() {
    writeln("M", x, "'s main");
  }
}
 
module M2 {
  use M1;

  const x = 2;
  proc main() {
    M1.main();
    writeln("M", x, "'s main");
  }
}
\end{chapel}
\begin{chapelcompopts}
--main-module M1 \# main\_module.M1.good
--main-module M2 \# main\_module.M2.good
\end{chapelcompopts}
If M1 is specified as the main module, the program will output:
\begin{chapelprintoutput}{main_module.M1.good}
M1's main
\end{chapelprintoutput}
If M2 is specified as the main module the program will output:
\begin{chapelprintoutput}{main_module.M2.good}
M1's main
M2's main
\end{chapelprintoutput}
Notice that main is treated like just another function if it is not in
the main module and can be called as such.
\end{chapelexample}

\index{exploratory programming}

%subsubsection{Programs with a Single Module}
%% \label{Programs_with_a_Single_Module}

To aid in exploratory programming, a default main function is
created if the program does not contain a user-defined main function.  The
default main function is equivalent to
\begin{chapel}
proc main() {}
\end{chapel}

\begin{chapelexample}{no-main.chpl}
The code
\begin{chapel}
writeln("hello, world");
\end{chapel}
\begin{chapeloutput}
hello, world
\end{chapeloutput}
is a legal and complete Chapel program.  The startup code for a Chapel program
first calls the module initialization code for the main module and then
calls \chpl{main()}.  This program's initialization function is the top-level
writeln() statement.  The module declaration is taken to be the entire file,
as described in~\rsec{Implicit_Modules}.
\end{chapelexample}


\subsection{Module Initialization Order}
\label{Module_Initialization_Order}
\index{modules!initialization order}

Module initialization is performed using the following algorithm.

Starting from the module that defines the main function, the modules named in
its use statements are visited depth-first and initialized in post-order.  If a
use statement names a module that has already been visited, it is not visited a
second time.  Thus, infinite recursion is avoided.

Modules used by a given module are visited in the order in which
they appear in the program text.  For nested modules, the
parent module and its uses are initialized before the nested module and its uses.

\begin{chapelexample}{init-order.chpl}
The code
\begin{chapel}
module M1 {
  use M2.M3;
  use M2;
  writeln("In M1's initializer");
  proc main() {
    writeln("In main");
  }
}

module M2 {
  use M4;
  writeln("In M2's initializer");
  module M3 {
    writeln("In M3's initializer");
  }
}

module M4 {
  writeln("In M4's initializer");
}
\end{chapel}
prints the following
\begin{chapelprintoutput}{}
In M4's initializer
In M2's initializer
In M3's initializer
In M1's initializer
In main
\end{chapelprintoutput}
M1, the main module, uses M2.M3 and then M2, thus M2.M3 must be
initialized.  Because M2.M3 is a nested module, M4 (which is used by
M2) must be initialized first.  M2 itself is initialized, followed by
M2.M3.  Finally M1 is initialized, and the main function is run.
\end{chapelexample}

\cleardoublepage
\sekshun{Functions}
\label{Functions}
\index{functions}

This section defines functions.  Methods and iterators are functions
and most of this section applies to them as well.  They are defined
separately in~\rsec{Iterators} and~\rsec{Class_Methods}.

\subsection{Function Definitions}
\label{Function_Definitions}
\index{functions!syntax}

\index{def@\chpl{def}}
Functions are declared with the following syntax:
\begin{syntax}
function-declaration-statement:
  `def' function-name argument-list[OPT] var-param-clause[OPT]
    return-type[OPT] where-clause[OPT] block-level-statement

function-name:
  identifier
  operator-name

operator-name: one of
  + - * / % ** ! == <= >= < > << >> & | ^ ~

argument-list:
  ( formals[OPT] )

formals:
  formal
  formal , formals

formal:
  formal-tag identifier formal-type[OPT] default-expression[OPT]
  formal-tag identifier formal-type[OPT] variable-argument-expression

formal-type:
  : type
  : TQUESTION identifier

default-expression:
  = expression

variable-argument-expression:
  ... expression
  ... TQUESTION identifier

formal-tag: one of
  in out inout param type

var-param-clause:
  `var'
  `const'
  `param'

where-clause:
  `where' expression
\end{syntax}

Operator overloading is supported in Chapel on the operators listed
above under operator name.  Operator and function overloading is
discussed in~\rsec{Function_Overloading}.

The intents \chpl{in}, \chpl{out}, and \chpl{inout} are discussed
in~\rsec{Intents}.  The formal tags \chpl{param} and \chpl{type} make
a function generic and are discussed in~\rsec{Generics}.  If the
formal argument's type is elided, generic, or prefixed with a question
mark, the function is also generic and is discussed
in~\rsec{Generics}.

Default expressions allow for the omission of actual arguments at the
call site, resulting in the implicit passing of a default value.
Default values are discussed in~\rsec{Default_Values}.

Functions do not require parentheses if they have no arguments.  Such
functions are described in~\rsec{Functions_without_Parentheses}.

Return types are optional and are discussed in~\rsec{Return_Types}.

Functions can take a variable number of arguments.  Such functions are
discussed in~\rsec{Variable_Length_Argument_Lists}.

The optional \sntx{var-param-clause} defines a variable function,
discussed in~\rsec{Variable_Functions}, or a parameter function,
discussed in~\rsec{Parameter_Functions}.  By default, a function call
cannot be treated as an lvalue and is constant.  This may be
explicitly specified via the keyword~\chpl{const}.

The optional where clause is only applicable if the function is
generic.  It is discussed in~\rsec{Where_Expressions}.

\subsection{The Return Statement}
\label{The_Return_Statement}
\index{return@\chpl{return}}

The return statement can only appear in a function.  It exits that
function, returning control to the point at which that function was
called.  It can optionally return a value.  The syntax of the return
statement is given by
\begin{syntax}
return-statement:
  `return' expression[OPT] ;
\end{syntax}

\begin{example}
The following code defines a function that returns the sum of three
integers:
\begin{chapel}
def sum(i1: int, i2: int, i3: int)
  return i1 + i2 + i3;
\end{chapel}
\end{example}

\subsection{Function Calls}
\label{Function_Calls}
\index{function calls}

Functions are called in call expressions described
in~\rsec{Call_Expressions}.  The function that is called is resolved
according to the algorithm described in~\rsec{Function_Resolution}.

\subsection{Formal Arguments}
\label{Formal_Arguments}
\index{formal arguments}

Chapel supports an intuitive formal argument passing mechanism.  An
argument is passed by value unless it is a class, array, or domain in
which case it is passed by reference.

Intents~(\rsec{Intents}) result in potential assignments to temporary
variables during a function call.  For example, passing an array by
intent \chpl{in}, a temporary array will be created.

\subsubsection{Named Arguments}
\label{Named_Arguments}
\index{named arguments}
\index{formal arguments!naming}

A formal argument can be named at the call site to explicitly map an
actual argument to a formal argument.

\begin{example}
In the code
\begin{chapel}
def foo(x: int, y: int) { ... }

foo(x=2, y=3);
foo(y=3, x=2);
\end{chapel}
named argument passing is used to map the actual arguments to the
formal arguments.  The two function calls are equivalent.
\end{example}

Named arguments are sometimes necessary to disambiguate calls or
ignore arguments with default values.  For a function that has many
arguments, it is sometimes good practice to name the arguments at the
call-site for compiler-checked documentation.

\subsubsection{Default Values}
\label{Default_Values}
\index{default values}
\index{formal arguments!defaults}

Default values can be specified for a formal argument by appending the
assignment operator and a default expression the declaration of the
formal argument.  If the actual argument is omitted from the function
call, the default expression is evaluated when the function call is
made and the evaluated result is passed to the formal argument as if
it were passed from the call site.

\begin{example}
In the code
\begin{chapel}
def foo(x: int = 5, y: int = 7) { ... }

foo();
foo(7);
foo(y=5);
\end{chapel}
default values are specified for the formal arguments \chpl{x}
and \chpl{y}.  The three calls to \chpl{foo} are equivalent to the
following three calls where the actual arguments are
explicit: \chpl{foo(5, 7)}, \chpl{foo(7, 7)}, and \chpl{foo(5, 5)}.
Note that named arguments are necessary to pass actual arguments to
formal arguments but use default values for arguments that are
specified earlier in the formal argument list.
\end{example}

\subsection{Intents}
\label{Intents}
\index{intents}

Intents allow the actual arguments to be copied to a formal argument
and also to be copied back.

\subsubsection{The Blank Intent}
\label{The_Blank_Intent}

If the intent is omitted, it is called a blank intent.  In such a
case, the value is copied in using the assignment operator.  Thus
classes are passed by reference and records are passed by value.
Arrays and domains are an exception because assignment does not apply
from the actual to the formal.  Instead, arrays and domains are passed
by reference.

With the exception of arrays, any argument that has blank intent
cannot be assigned within the function.

\subsubsection{The In Intent}
\label{The_In_Intent}
\index{in@\chpl{in}}
\index{intents!in@\chpl{in}}

If \chpl{in} is specified as the intent, the actual argument is copied
to the formal argument as usual, but it may also be assigned to within
the function.  This assignment is not reflected back at the call site.

If an array is passed to a formal argument that has \chpl{in} intent,
a copy of the array is made via assignment.  Changes to the elements
within the array are thus not reflected back at the call site.
Domains cannot be passed to a function via the \chpl{in} intent.

\subsubsection{The Out Intent}
\label{The_Out_Intent}
\index{out@\chpl{out}}
\index{intents!out@\chpl{out}}

If \chpl{out} is specified as the intent, the actual argument is
ignored when the call is made, but after the call, the formal argument
is assigned to the actual argument at the call site.  The actual
argument must be a valid lvalue.  The formal argument can be assigned
to and read from within the function.

The formal argument cannot not be generic and is treated as a variable
declaration.  Domains cannot be passed to a function via
the \chpl{out} intent.

\subsubsection{The Inout Intent}
\label{The_Inout_Intent}
\index{inout@\chpl{inout}}
\index{intents!inout@\chpl{inout}}

If \chpl{inout} is specified as the intent, the actual argument is
both passed to the formal argument as if the \chpl{in} intent applied
and then copied back as if the \chpl{out} intent applied.  The formal
argument can be generic and takes its type from the actual argument.
Domains cannot be passed to a function via the \chpl{inout} intent.
The formal argument can be assigned to and read from within the
function.

\subsection{Return Types}
\label{Return_Types}
\index{return@\chpl{return}!types}

A function can optionally return a value.  If the function does not
return a value, then no return type can be specified.  If the function
does return a value, the return type is optional.

\subsubsection{Explicit Return Types}
\label{Explicit_Return_Types}

If a return type is specified, the values that the function returns
via return statements must be assignable to a value of the return
type.  For variable functions~(\rsec{Variable_Functions}), the return
type must match the type returned in all of the return statements
exactly.

\subsubsection{Implicit Return Types}
\label{Implicit_Return_Types}
\index{type inference!of return types}

If a return type is not specified, it will be inferred from the return
statements.  Given the types that are returned by the different
statements, if exactly one of those types can be a target, via
implicit conversions, of every other type, then that is the inferred
return type.  Otherwise, it is an error.  For variable
functions~(\rsec{Variable_Functions}), every return statement must
return the same exact type and it becomes the inferred type.

\subsection{Variable Functions}
\label{Variable_Functions}
\index{functions!as lvalues}

A variable function is a function that can be assigned a value.  Note
that a variable function does not return a reference.  That is, the
reference cannot be captured.

A variable function is specified by following the argument list with
the \chpl{var} keyword.  A variable function must return an lvalue.

When a variable function is called on the left-hand side of an
assignment statement or in the context of a call to a formal argument
by out or inout intent, the lvalue that is returned by the function is
assigned a value.

Variable functions support an implicit argument \chpl{setter} of type
bool.  If the variable function is called in a context such that the
returned lvalue is assigned a value, the argument \chpl{setter}
is \chpl{true}; otherwise it is \chpl{false}.  This argument is useful
for controlling different behavior depending on the call site.

\begin{example}
The following code creates a function that can be interpreted as a
simple two-element array where the elements are actually global
variables:
\begin{chapel}
var x, y = 0;

def A(i: int) var {
  if i < 0 || i > 1 then
    halt("array access out of bounds");
  if i == 0 then
    return x;
  else
    return y;
}
\end{chapel}
This function can be assigned to in order to write to the ``elements''
of the array as in
\begin{chapel}
A(0) = 1;
A(1) = 2;
\end{chapel}
It can be called as an expression to access the ``elements'' as in
\begin{chapel}
writeln(A(0) + A(1));
\end{chapel}
This code outputs the number \chpl{3}.

The implicit \chpl{setter} argument can be used to ensure, for
example, that the second element in the pseudo-array is only assigned
a value if the first argument is positive.  To do this, the line
\begin{chapel}
if setter && i == 1 && x <= 0 then
  halt("cannot assign value to A(1) if A(0) <= 0");
\end{chapel}
\end{example}

\subsection{Parameter Functions}
\label{Parameter_Functions}
\index{functions!as parameters}

A parameter function is a function that returns a parameter
expression.  It is specified by following the function's argument list
by the keyword \chpl{param}.  It is often, but not necessarily,
generic.

It is a compile-time error if a parameter function does not return a
parameter expression.  The result of a parameter function is computed
during compilation and the result is inlined at the call site.

\begin{example}
In the code
\begin{chapel}
def sumOfSquares(param a: int, param b: int) param
  return a**2 + b**2;

var x: sumOfSquares(2, 3)*int;
\end{chapel}
the function \chpl{sumOfSquares} is a parameter function that takes
two parameters as arguments.  Calls to this function can be used in
places where a parameter expression is required.  In this example, the
call is used in the declaration of a homogeneous and so is required to
be a parameter.
\end{example}.

\subsection{Function Overloading}
\label{Function_Overloading}
\index{functions!overloading}
\index{operators!overloading}

Functions that have the same name but different argument lists are
called overloaded functions.  Function calls to overloaded functions
are resolved according to the algorithm in~\rsec{Function_Resolution}.

Operator overloading is achieved by defining a function with a name
specified by that operator.  The operators that may be overloaded are
listed in the following table:

\begin{center}
\begin{tabular}{|l|l|}
\hline
{\bf arity} & {\bf operators} \\
\hline
unary & \verb@+ - ! ~@ \\
binary & \verb@+ - * / % ** ! == <= >= < > << >> & | ^ @ \\
\hline
\end{tabular}
\end{center}

The arity and precedence of the operator must be maintained when it is
overloaded.  Operator resolution follows the same algorithm as
function resolution.

\subsection{Function Resolution}
\label{Function_Resolution}

Given a function call, the function that the call resolves to is
determined according to the following algorithm:
\begin{itemize}
\item
Identify the set of visible functions.  A visible function is any
function with the same name that satisfies the criteria
in~\rsec{Identifying_Visible_Functions}.
\item
From the set of visible functions, determine the set of candidate
functions.  A function is a candidate if the function is a valid
application of the function call's actual arguments as determined
in~\rsec{Determining_Candidate_Functions}.  A compiler error occurs if
there are no candidate functions.
\item
From the set of candidate functions, the most specific function is
determined.  The most specific function is a candidate function that
is more specific than every other candidate function.  If there is no
function that is more specific than every other candidate function,
the function call is ambiguous and a compiler error occurs.  The term
{\em more specific function} is defined
in~\rsec{Determining_More_Specific_Functions}.
\end{itemize}.

\subsubsection{Identifying Visible Functions}
\label{Identifying_Visible_Functions}
\index{functions!visible}

A function is a visible function to a function call if the name of the
function is the same as the name of the function call and the function
is defined or used in a lexical outer scope.

\index{functions!with class arguments}
Additionally, functions that have arguments of class type are
considered globally visible and so are always visible regardless of
the location of their definition.

\subsubsection{Determining Candidate Functions}
\label{Determining_Candidate_Functions}
\index{functions!candidates}

A function is a candidate function if there is a {\em valid mapping}
from the function call to the function and each actual argument is
mapped to a formal argument that is a {\em legal argument mapping}.

\paragraph{Valid Mapping}

A function call is mapped to a function according to the following
steps:
\begin{itemize}
\item
Each actual argument that is passed by name is matched to the formal
argument with that name.  If there is no formal argument with that
name, there is no valid mapping.
\item
The remaining actual arguments are mapped in order to the remaining
formal arguments in order.  If there are more actual arguments then
formal arguments, there is no valid mapping.  If any formal argument
that is not mapped to by an actual argument does not have a default
value, there is no valid mapping.
\item
The valid mapping is the mapping of actual arguments to formal
arguments plus default values to formal arguments that are not mapped
to by actual arguments.
\end{itemize}

\paragraph{Legal Argument Mapping}

An actual argument of type $T_A$ can be mapped to a formal argument of
type $T_F$ if any of the following conditions hold:
\begin{itemize}
\item $T_A$ and $T_F$ are the same type.
\item There is an implicit coercion from $T_A$ to $T_F$.
\item $T_A$ is derived from $T_F$.
\item $T_A$ is scalar promotable to $T_F$.
\end{itemize}

\subsubsection{Determining More Specific Functions}
\label{Determining_More_Specific_Functions}
\index{functions!most specific}

Given two functions $F_1$ and $F_2$, $F_1$ is determined to be more
specific than $F_2$ by the following steps:
\begin{itemize}
\item
If at least one of the legal argument mappings to $F_1$ is a {\em more
specific argument mapping} than the corresponding legal argument
mapping to $F_2$ and none of the legal argument mappings to $F_2$ is a
more specific argument mapping than the corresponding legal argument
mapping to $F_1$, then $F_1$ is more specific.
\item If $F_1$ does not require promotion and $F_2$ does require promotion, then $F_1$ is more specific.
\item If $F_1$ shadows $F_2$, then $F_1$ is more specific.
\item Otherwise, $F_1$ is not more specific than $F_2$.
\end{itemize}

Given an argument mapping, $M_1$, from an actual argument, $A$, of
type $T_A$ to a formal argument, $F1$, of type $T_{F1}$ and an
argument mapping, $M_2$, from the same actual argument to a formal
argument, $F2$, of type $T_{F2}$, the more specific argument mapping
is determined by the following steps:
\begin{itemize}
\item
 If $T_{F1}$ and $T_{F2}$ are the same type and $F1$ is an
 instantiated parameter, $M_1$ is more specific.
\item
 If $T_{F1}$ and $T_{F2}$ are the same type and $F2$ is an
 instantiated parameter, $M_2$ is more specific.
\item
 If $M_1$ requires scalar promotion and $M_2$ does not require scalar
 promotion, $M_2$ is more specific.
\item
 If $M_2$ requires scalar promotion and $M_1$ does not require scalar
 promotion, $M_1$ is more specific.
\item
 If $F1$ is generic over all types and $F2$ is not generic over all
 types, $M_2$ is more specific.
\item
 If $F2$ is generic over all types and $F1$ is not generic over all
 types, $M_1$ is more specific.
\item
 If $T_{F1}$ and $T_{F2}$ are the same type, neither mapping is more
 specific.
\item
 If $T_A$ and $T_{F1}$ are the same type, $M_1$ is more specific.
\item
 If $T_A$ and $T_{F2}$ are the same type, $M_2$ is more specific.
\item
 If $T_{F1}$ is derived from $T_{F2}$, then $M_1$ is more specific.
\item
 If $T_{F2}$ is derived from $T_{F1}$, then $M_2$ is more specific.
\item
 If there is an implicit coercion from $T_{F1}$ to $T_{F2}$, then
 $M_1$ is more specific.
\item
 If there is an implicit coercion from $T_{F2}$ to $T_{F1}$, then
 $M_2$ is more specific.
\item
 If $T_{F1}$ is any \chpl{int} type and $T_{F2}$ is any \chpl{uint}
 type, $M_1$ is more specific.
\item
 If $T_{F2}$ is any \chpl{int} type and $T_{F1}$ is any \chpl{uint}
 type, $M_2$ is more specific.
\item
 Otherwise neither mapping is more specific.
\end{itemize}

\subsection{Functions without Parentheses}
\label{Functions_without_Parentheses}
\index{functions!functions without parentheses}

Functions do not require parentheses if they have empty argument
lists.  Functions declared without parentheses around empty argument
lists must be called without parentheses.

\begin{example}
Given the definitions
\begin{chapel}
def foo { }
def bar() { }
\end{chapel}
the function \chpl{foo} can be called by writing \chpl{foo} and the
function \chpl{bar} can be called by writing \chpl{bar()}.  It is an
error to apply parentheses to \chpl{foo} or omit them from \chpl{bar}.
\end{example}

\subsection{Nested Functions}
\label{Nested_Functions}
\index{functions!nested}

A function defined in another function is called a nested function.
Nesting of functions may be done to arbitrary degrees, i.e., a
function can be nested in a nested function.

Nested functions are only visible to function calls within the scope
in which they are defined.  An exception is to a function that has an
argument that is a class type.  Such functions are globally visible.

\subsubsection{Accessing Outer Variables}
\label{Accessing_Outer_Variables}

Nested functions may refer to variables defined in the function in
which they are nested.  If the function has class arguments, and is
thus globally visible, it is a compiler error to refer to a variable
in the outer function.

\begin{rationale}
It may be too strict to make this a compiler error.  Are there
advantages to making this a runtime error?
\end{rationale}

\subsection{Variable Length Argument Lists}
\label{Variable_Length_Argument_Lists}
\index{functions!variable number of arguments}

Functions can be defined to take a variable number of arguments.  This
allows the call site to pass a different number of actual arguments.

If the variable argument expression is an identifier prepended by a
question mark, the number of arguments is variable.  Alternatively,
the variable expression can evaluate to an integer parameter value
requiring the call site to pass that number of arguments to the
function.

In the function, the formal argument is a tuple of the actual
arguments.

\begin{example}
The code
\begin{chapel}
def mywriteln(x: int ...?k) {
  for param i in 1..k do
    writeln(x(i));
}
\end{chapel}
defines a function called \chpl{mywriteln} that takes a variable
number of arguments and then writes them out on separate lines.  The
parameter for-loop~(\rsec{Parameter_For_Loops}) is unrolled by the
compiler so that \chpl{i} is a parameter, rather than a variable.
This function can be made generic~(\rsec{Generics}) to take arguments
of different types by eliding the type.
\end{example}

A tuple of variables arguments can be passed to a function that takes
variable arguments by destructuring the tuple in a tuple destructuring
expression.  The syntax of this expression is given by
\begin{syntax}
tuple-destructuring-expression:
  ( ... expression )
\end{syntax}
In this expression, the tuple defined by \sntx{expression} is expanded
in place to represent its components.  This allows for the forwarding
of variable arguments as variable arguments.

\cleardoublepage
\sekshun{Classes}
\label{Classes}

Classes are an abstraction of a data structure where the storage
location is allocated independent of the scope of the variable of
class type.  Each call to the constructor creates a new data object
and returns a reference to the object.  Storage is reclaimed
automatically as described in~\rsec{Automatic_Memory_Management}.

\subsection{Class Declarations}
\label{Class_Declarations}

A class is defined with the following syntax:
\begin{syntax}
class-declaration-statement:
  `class' identifier class-inherit-type-list[OPT] {
    class-statement-list }

class-inherit-expression-list:
  class-type
  class-type , inherit-expression-list

class-statement-list:
  class-statement
  class-statement class-statement-list

class-statement:
  type-declaration-statement
  function-declaration-statement
  variable-declaration-statement
\end{syntax}
A \sntx{class-declaration-statement} defines a new type symbol
specified by the identifier.  Classes inherit data and functionality
from other classes if the \sntx{inherit-type-list} is specified.
Inheritance is described in~\rsec{Inheritance}.

The body of a class declaration consists of a sequence of statements
where each of the statements either defines a variable, called a
field, a function, called a method, or a type.

If a class contains a type alias or a parameter, the class is generic.
Generic classes are described in~\rsec{Generics}.

\subsection{Class Assignment}
\label{Class_Assignment}

Classes are assigned by reference.  After an assignment from one
variable of class type to another, the variables reference the same
storage location.

\subsection{Class Fields}
\label{Class_Fields}

Variables and constants declared within class declarations define
fields within that class.  (Parameters make a class generic.)  Fields
define the storage associated with a class.

\begin{example}
The code
\begin{chapel}
class Actor {
  var name: string;
  var age: uint;
}
\end{chapel}
defines a new class type called \chpl{Actor} that has two fields: the
string field \chpl{name} and the unsigned integer field \chpl{age}.
\end{example}

\subsubsection{Class Field Accesses}
\label{Class_Field_Accesses}

The field in a class is accessed via a member access expression as
described in~\rsec{Member_Access_Expressions}.  Fields in a class can
be modified via an assignment statement where the left-hand side of
the assignment is a member access.
\begin{example}
Given a variable \chpl{anActor} of type \chpl{Actor}, defined above,
the code
\begin{chapel}
var s: string = anActor.name;
anActor.age = 27;
\end{chapel}
reads the field \chpl{name} and assigns the value to the variable
\chpl{s}, and assigns the storage location in the object
\chpl{anActor} associated with the field \chpl{age} the value
\chpl{27}.
\end{example}

\subsection{Class Methods}
\label{Class_Methods}

A method is a function that is bound to a class.  A method is called
by passing an instance of the class to the method via a special
syntax that is similar to a field access.

\subsubsection{Class Method Declarations}
\label{Class_Method_Declarations}

Methods are declared with the following syntax:
\begin{syntax}
method-declaration-statement:
  `def' type-binding function-name argument-list[OPT] var-clause[OPT]
    return-type[OPT] where-clause[OPT] block-level-statement

type-binding:
  identifier .
\end{syntax}
If a method is declared within the lexical scope of a class, record,
or union, the type binding can be omitted and is taken to be the
innermost class, record, or union that the method is defined in.

\subsubsection{Class Method Calls}
\label{Class_Method_Calls}

A method is called by using the member access syntax as described
in~\rsec{Member_Access_Expressions} where the accessed expression is
the name of the method.

\begin{example}
A method to output information about an instance of the \chpl{Actor}
class can be defined as follows:
\begin{chapel}
def Actor.print() {
  writeln("Actor ", name, " is ", age, " years old");
}
\end{chapel}
This method can be called on an instance of the \chpl{Actor}
class, \chpl{anActor}, by writing \chpl{anActor.print()}.
\end{example}

\subsubsection{The {\em this} Reference}
\label{The_em_this_Reference}

The instance of a class is passed to a method using special syntax.
It does not appear in the argument list to the method.  The
reference \chpl{this} is an alias to the instance of the class on
which the method is called.

\begin{example}
Let class \chpl{C}, method \chpl{foo}, and function \chpl{bar} be
defined as
\begin{chapel}
class C {
  def foo() {
    bar(this);
  }
}
def bar(c: C) { }
\end{chapel}
Then given an instance of \chpl{C} called \chpl{c}, the method
call \chpl{c.foo()} results in a call to \chpl{bar} where the argument
is \chpl{c}.
\end{example}

\subsubsection{Class Methods without Parentheses}
\label{Class_Methods_without_Parentheses}

Methods do not require parentheses if they have empty argument lists.
Methods declared without parentheses around empty argument lists must
be called without parentheses.

\begin{example}
Given the definitions
\begin{chapel}
class C {
  def foo { }
  def bar() { }
}
\end{chapel}
and an instance of \chpl{C} called \chpl{c}, then the
method \chpl{foo} can be called by writing \chpl{c.foo} and the
method \chpl{bar} can be called by writing \chpl{c.bar()}.  It is an
error to apply parentheses to \chpl{foo} or omit them from \chpl{bar}.
\end{example}

\subsubsection{The {\em this} Method}
\label{The_em_this_Method}

A method declared with the name \chpl{this} allows a class to be
``indexed'' similarly to how a tuple, sequence, or array is indexed.
Indexing into a class has the semantics of calling a method on the
class named \chpl{this}.  There is no other way to call a method
called \chpl{this}.  The \chpl{this} method must be declared with
parentheses even if the argument list is empty.

\begin{example}
In the following code, the \chpl{this} method is used to create a
class that acts like a simple array that contains three integers
indexed by one, two, and three.
\begin{chapel}
class ThreeArray {
  var x1, x2, x3: int;
  def this(i: int) var {
    select i {
      when 1 do return x1;
      when 2 do return x2;
      when 3 do return x3;
    }
    halt("ThreeArray index out of bounds: ", i);
  }
}
\end{chapel}
\end{example}

\subsection{Class Constructors}
\label{Class_Constructors}

A class constructor is defined by declaring a method with the same
name as the class.  The constructor is used to create instances of the
class.  When the constructor is called, memory is allocated to store a
class instance.

\subsubsection{The Default Constructor}
\label{The_Default_Constructor}

A default constructor is automatically created for every class in the
Chapel program.  This constructor is defined such that it has one
argument for every field in the class.  Each of the arguments has a
default value.

The default constructor is very useful but its generality in terms of
having one argument for each field all of which have default values
makes it slightly difficult for the user to create their own
constructor.  It is expected that in many simple cases, the default
constructor will be all that is necessary.

\begin{example}
Given the class
\begin{chapel}
class C {
  def x: int;
  def y: real = 3.14;
  def z: string = "Hello, World!";
}
\end{chapel}
then instances of the class can be created by calling the default
constructor as follows:
\begin{itemize}
\item The call \chpl{C()} is equivalent to \chpl{C(0,3.14,"Hello, World!")}.
\item The call \chpl{C(2)} is equivalent to \chpl{C(2,3.14,"Hello, World!")}.
\item The call \chpl{C(z="")} is equivalent to \chpl{C(0,3.14,"")}.
\item The call \chpl{C(0,0.0,"")} is equivalent to \chpl{C(0,0.0,"")}.
\end{itemize}
\end{example}

\subsection{Getters and Setters}
\label{Getters_and_Setters}

All field accesses are resolved via getter and setter methods that are
defined in the class with the same name as the field.  A setter is
defined as an explicit setter
function~(\rsec{Explicit_Setter_Functions}).  Default getters and
setters are defined that simply access or set the field if the user
does not define their own.

\begin{example}
In the code
\begin{chapel}
class C {
  var x: int;
  def =x(value: int) {
    if value < 0 then
      halt("x assigned negative value");
    x = value;
  }
}
\end{chapel}
a setter is defined for field \chpl{x} that ensures that \chpl{x} is
never assigned a negative value.
\end{example}

\subsection{Inheritance}
\label{Inheritance}

A ``derived'' class can inherit from one or more other classes by
specifying those classes, the base classes, following the name of the
derived class in the declaration of the derived class.  When
inheriting from multiple base classes, only one of the base classes
may contain fields.  The other classes can only define methods.  Note
that a class can still be derived from a class that contains fields
which is itself derived from a class that contains fields.

\subsubsection{Accessing Base Class Fields}
\label{Accessing_Base_Class_Fields}

A derived class contains data associated with the fields in its base
classes.  The fields can be accessed in the same way that they are
accessed in their base class unless the getter or setter methods is
overridden in the derived class, as discussed
in~\rsec{Overriding_Base_Class_Methods}.

\subsubsection{Derived Class Constructors}
\label{Derived_Class_Constructors}

Derived class constructors automatically call the default constructor
of the base class.  There is an expectation that a more standard way
of chaining constructor calls will be supported.

\subsubsection{Shadowing Base Class Fields}
\label{Shadowing_Base_Class_Fields}

A field in the derived class can be declared with the same name as a
field in the base class.  Such a field shadows the field in the base
class in that it is always referenced when it is accessed in the
context of the derived class.  There is an expectation that there will
be a way to reference the field in the base class but this is not
defined at this time.

\subsubsection{Overriding Base Class Methods}
\label{Overriding_Base_Class_Methods}

If a method in a derived class is declared with the identical
signature as a method in a base class, then it is said to override the
base class's method.  Such a method is a candidate for dynamic
dispatch in the event that a variable that has the base class type
references a variable that has the derived class type.

The identical signature requires that the names, types, and order of
the formal arguments be identical.

\subsubsection{Inheriting from Multiple Classes}
\label{Inheriting_from_Multiple_Classes}

\begin{implementation}
Multiple inheritance is not yet supported.
\end{implementation}

A class can be derived from multiple base classes provided that only
one of the base classes contains fields either directly or from base
classes that it is derived from.  The methods defined by the other
base classes can be overridden.

\subsection{Class Promotion of Scalar Functions}
\label{Scalar Promotion}

A class can be defined to promote scalar functions by defining an
iterator in the class named \chpl{this} and specifying a return type.
The return type indicates the type that the class promotes.  The body
of the \chpl{this} iterator is ignored.  The class must also implement
the iterator interface as described in~\rsec{Iterator_Interface}.

There is an expectation that class promotion will be implemented in a
different way in the future.

\subsection{Nested Classes}
\label{Nested_Classes}

\begin{implementation}
Nested classes are not yet supported.
\end{implementation}

A class defined within another class is a nested class.

\subsection{Automatic Memory Management}
\label{Automatic_Memory_Management}

\begin{implementation}
Memory allocated to store class objects is not yet reclaimed.
\end{implementation}

Memory associated with class instances is reclaimed automatically when
there is no way for the current program to reference this memory.  The
programmer does not need to free the memory associated with class
instances.

\cleardoublepage
\sekshun{Records}
\label{Records}
\index{records}

A record is a data structure that is similar to a class except it has
value semantics, similar to primitive types.  Value semantics mean that
assignment, argument passing and function return values are by default
all done by copying.  Value semantics also imply that a variable of
record type is associated with only one piece of storage and has only one
type throughout its lifetime.  Storage is allocated for a variable of
record type when the variable declaration is executed, and the record
variable is also initialized at that time. When the record variable goes
out of scope, or at the end of the program if it is a global, it is
deinitialized and its storage is deallocated.

A record declaration statement creates a record
type~\rsec{Record_Declarations}.  A variable of record type contains all
and only the fields defined by that type (\rsec{Record_Types}).  Value
semantics imply that the type of a record variable is known at compile
time (i.e. it is statically typed).

A record can be created using the \chpl{new} operator, which allocates
storage, initializes it via a call to a record constructor, and returns
it.  A record is also created upon a variable declaration of a record
type.

A record type is generic if it contains generic fields.  Generic record types
are discussed in detail in~\rsec{Generic_Types}.

\section{Record Declarations}
\label{Record_Declarations}
\index{records!declarations}
\index{declarations!records}
\index{record@\chpl{record}}

A record type is defined with the following syntax:
\begin{syntax}
record-declaration-statement:
  simple-record-declaration-statement
  external-record-declaration-statement

simple-record-declaration-statement:
  `record' identifier { record-statement-list }

record-statement-list:
  record-statement
  record-statement record-statement-list

record-statement:
  variable-declaration-statement
  method-declaration-statement
  type-declaration-statement
  empty-statement
\end{syntax}

A \sntx{record-declaration-statement} defines a new type symbol specified
by the identifier. As in a class declaration, the body of a record declaration
can contain variable, method, and type declarations.

If a record declaration contains a type alias or parameter field, or it
contains a variable or constant field without a specified type and
without an initialization expression, then it declares a generic record
type.  Generic record types are described in~\rsec{Generic_Types}.

If the \chpl{extern} keyword appears before the \chpl{record} keyword, then an
external record type is declared. An external record is used within Chapel
for type and field resolution, but no corresponding backend definition is
generated.  It is presumed that the definition of an external record is supplied
by a library or the execution environment.  See the chapter on interoperability
(\rsec{Interoperability}) for more information on external records.

\begin{future}
Privacy controls for classes and records are currently not specified,
as discussion is needed regarding its impact on inheritance, for
instance.
\end{future}

\subsection{Record Types}
\label{Record_Types}
\index{records!record types}
\index{records!types}
\index{types!records}

A record type specifier simply names a record type, using
the following syntax:
\begin{syntax}
record-type:
  identifier
  identifier ( named-expression-list )
\end{syntax}
A record type specifier may appear anywhere a type specifier is permitted.

For non-generic records, the record name by itself is sufficient to specify the
type.  Generic records must be instantiated to serve as a fully-specified
type, for example to declare a variable.  This is done with
type constructors, which are defined in Section~\ref{Type_Constructors}.

\subsection{Record Fields}
\label{Record_Fields}
\index{records!fields}
\index{fields!records}

Variable declarations within a record type declaration define fields within that
record type.  The presence of at least one parameter field causes the record
type to become generic.  Variable fields define the storage associated with a
record.

\begin{chapelexample}{defineActorRecord.chpl}
The code
\begin{chapel}
record ActorRecord {
  var name: string;
  var age: uint;
}
\end{chapel}
\begin{chapeloutput}
\end{chapeloutput}
defines a new record type called \chpl{ActorRecord} that has two fields: the
string field \chpl{name} and the unsigned integer field \chpl{age}.  The data
contained by a record of this type is exactly the same as that contained by
an instance of the \chpl{Actor} class defined in the preceding
chapter~\rsec{Class_Fields}.
\end{chapelexample}

\subsection{Record Methods}
\label{Record_Methods}
\index{records!methods}
\index{methods!records}

A record method is a function or iterator that is bound to a record.
See the methods section~\rsec{Methods} for more information about
methods.

Note that the receiver of a record method is passed by \chpl{ref} or
\chpl{const ref} intent by default, depending on whether or not
\chpl{this} is modified in the body of the method.

\subsection{Nested Record Types}
\label{Nested_Record_Types}
\index{nested records}
\index{records!nested}

Record type declarations may be nested within other class, record and union
declarations.  Methods defined in a nested record type may access fields
declared in the containing aggregate type either implicitly, or explicitly by
means of an \chpl{outer} reference.

\section{Record Variable Declarations}
\label{Record_Variable_Declarations}
\index{records!variable declarations}
\index{variables!records}

A record variable declaration is a variable declaration using a record type.
When a variable of record type is declared, storage is allocated sufficient to
store all of the fields defined in that record type.

In the context of a class or record or union declaration, the fields are
allocated within the object as if they had been declared individually.  In this
sense, records provide a way to group related fields within a containing class
or record type.

In the context of a function body, a record variable declaration
causes storage to be allocated sufficient to store all of the fields in that
record type.  The record variable is initialized through a call to its
default initializer.  The default initializer for a record is defined in the
same way as the default initializer for a class (\rsec{Default_Initialization}).

\subsection{Storage Allocation}
\label{Record_Storage}
\index{records!allocation}

Storage for a record variable directly contains the data associated
with the fields in the record, in the same manner as variables
of primitive types directly contain the primitive values.
Record storage is reclaimed when the record variable goes out of scope.
No additional storage for a record is allocated or reclaimed.
Field data of one variable's record is not shared with data
of another variable's record.

\subsection{Record Initialization}
\label{Record_Initialization}
\index{records!initialization}
\index{initialization!record}

A variable of a record type declared without an initialization expression
is initialized through a call to the record's default initializer,
passing no arguments.  The default initializer for a record is defined in
the same way as the default initializer for a class
(\rsec{Default_Initialization}).

To construct a record as an expression,
i.e. without binding it to a variable, the \chpl{new} operator is
required.  In this case, storage is allocated and reclaimed as for a record
variable declaration (\rsec{Record_Storage}), except that the temporary record
goes out of scope at the end of the enclosing statement.
The constructors for a record are
defined in the same way as those for a class (\rsec{Class_Constructors}).

\begin{rationale}

The \chpl{new} keyword disambiguates types from values. This is needed
because of the syntactic similarity between constructors and type
specifiers for classes and records.

\end{rationale}

\begin{chapelexample}{recordCreation.chpl}
The program
\begin{chapel}
record TimeStamp {
  var time: string = "1/1/1011";
}

var timestampDefault: TimeStamp;                  // use the default for 'time'
var timestampCustom = new TimeStamp("2/2/2022");  // ... or a different one
writeln(timestampDefault);
writeln(timestampCustom);

var idCounter = 0;
record UniqueID {
  var id: int;
  proc UniqueID() { idCounter += 1; id = idCounter; }
}

writeln(new UniqueID());  // create and use a record value without a variable
writeln(new UniqueID());
\end{chapel}
\begin{chapelcompopts}
--no-warn-constructors
\end{chapelcompopts}
produces the output
\begin{chapelprintoutput}{}
(time = 1/1/1011)
(time = 2/2/2022)
(id = 1)
(id = 2)
\end{chapelprintoutput}
The variable \chpl{timestampDefault} is initialized with \chpl{TimeStamp}'s
default initializer. The expression \chpl{new TimeStamp} creates a record that
is assigned to \chpl{timestampCustom}.  It effectively
initializes \chpl{timestampCustom} via a call to the constructor with desired
arguments. The records created with \chpl{new UniqueID()} are discarded after
they are used.
\end{chapelexample}

As with classes, the user can provide his own constructors
(\rsec{User_Defined_Constructors}).  If any user-defined constructors are
supplied, the default initializer cannot be called directly.

\subsection{Record Deinitializer}
\label{Record_Deinitializer}
\index{records!deinitializer}
\index{deinitializer!records}

A record author may specify additional actions to be performed before record storage is
reclaimed by defining a record deinitializer.  A record deinitializer is a method named
\chpl{deinit()}.  A record deinitializer takes no arguments
(aside from the implicit \chpl{this} argument).  If defined, the deinitializer is called
on a record object after it goes out of scope and before its memory is reclaimed.

% TODO: The above ambiguous language is intended to allow optimizations involving extending
% the lifetime of an object.  However, we leave unspecified the means by which a user may
% demand avid running of the deinitializer and reclamation of memory (as in C\#).  We need to
% specify this so the above language can be tightened up for that case.

% For now, the actual lifetime of a record object is under the control of the compiler.  For
% example, as an optimization, ownership of an object may be transferred between variables with
% non-overlapping lifetimes.  When this happens, there will be no observable deinitialization of
% one of those variables.  The compiler may also choose to insert temporary copies e.g. of
% record formals or of a record return value.

% The compiler guarantees that every record constructor call will have exactly one
% corresponding record deinitializer call.  However, the exact number of constructor-deinitializer
% pairs is determined by the compiler, and may also be influenced by various compiler
% options.

\begin{chapelexample}{recordDeinitializer.chpl}
\begin{chapel}
class C { var x: int; } // A class with nonzero size.
// If the class were empty, whether or not its memory was reclaimed
// would not be observable.

// Defines a record implementing simple memory management.
record R {
  var c: unmanaged C;
  proc init() {
    c = new unmanaged C(0);
  }
  proc deinit() {
    delete c; c = nil;
  }
}

proc foo()
{
  var r: R; // Initialized using default constructor.
  writeln(r);
  // r will go out of scope here.
  // Its deinitializer will be called to free the C object it contains.
}

foo();
\end{chapel}
\begin{chapeloutput}
(c = {x = 0})

====================
Leaked Memory Report
==============================================================
Number of leaked allocations
           Total leaked memory (bytes)
                      Description of allocation
==============================================================
==============================================================
\end{chapeloutput}
\begin{chapelexecopts}
--memLeaksByType
\end{chapelexecopts}
\end{chapelexample}


\section{Record Arguments}
\label{Record_Arguments}
\index{records!arguments}
\index{arguments!records}

When records are copied into or out of a function's formal argument,
the copy is performed consistently with the semantics described for
record assignment (\rsec{Record_Assignment}).

\begin{chapelexample}{paramPassing.chpl}
The program
\begin{chapel}
record MyColor {
  var color: int;
}
proc printMyColor(in mc: MyColor) {
  writeln("my color is ", mc.color);
  mc.color = 6;   // does not affect the caller's record
}
var mc1: MyColor;        // 'color' defaults to 0
var mc2: MyColor = mc1;  // mc1's value is copied into mc2
mc1.color = 3;           // mc1's value is modified
printMyColor(mc2);       // mc2 is not affected by assignment to mc1
printMyColor(mc2);       // ... or by assignment in printMyColor()

proc modifyMyColor(inout mc: MyColor, newcolor: int) {
  mc.color = newcolor;
}
modifyMyColor(mc2, 7);   // mc2 is affected because of the 'inout' intent
printMyColor(mc2);
\end{chapel}
produces
\begin{chapelprintoutput}{}
my color is 0
my color is 0
my color is 7
\end{chapelprintoutput}
The assignment to \chpl{mc1.color} affects only the record stored
in \chpl{mc1}. The record in \chpl{mc2} is not affected by
the assignment to \chpl{mc1} or by the assignment in \chpl{printMyColor}.
\chpl{mc2} is affected by the assignment in \chpl{modifyMyColor}
because the intent \chpl{inout} is used.
\end{chapelexample}

\section{Record Field Access}
\label{Record_Field_Access}
\index{records!field access}
\index{field access}

A record field is accessed the same way as a class field
(\rsec{Class_Field_Accesses}).  When a field access is used as an
rvalue, the value of that field is returned.  When it is used as
an lvalue, the value of the record field is updated.

Accessing a parameter or type field returns a parameter or type,
respectively. Also, parameter and type fields can be accessed from
an instantiated record type in addition to from a record value.


\subsection{Field Getter Methods}
\label{Field_Getter_Methods}
\index{records!getters}

As in classes, field accesses are performed via getter methods
(\rsec{Getter_Methods}).  By default, these methods simply return a reference to
the specified field (so they can be written as well as read).  The user may
redefine these as needed.

\section{Record Method Calls}
\label{Record_Method_Access}
\index{records!method calls}
\index{method calls}

Record method calls are written the same way as other method calls
(\rsec{Method_Calls}). Unlike class methods, record methods are
always resolved at compile time.

\section{Common Operations}

\subsection{Record Assignment}
\label{Record_Assignment}
\index{records!assignment}

A variable of record type may be updated by assignment.  The compiler
provides a default assignment operator for each record type \chpl{R}
having the signature:

\begin{chapel}
proc =(ref lhs:R, rhs) : void ;
\end{chapel}

In it, the value of each field of the record on the right-hand side is assigned
to the corresponding field of the record on the left-hand side. It is
a type error if the left-hand side and the right-hand side do not have
the same set of field names. It is also a type error if two fields with
the same name do not have assignable types.

The compiler-provided assignment operator may be overridden as described
in \ref{Assignment_Statements}.

The following example demonstrates record assignment.
\begin{chapelexample}{assignment.chpl}
\begin{chapel}
record R {
  var i: int;
  var x: real;
  proc print() { writeln("i = ", this.i, ", x = ", this.x); }
}
var A: R;
A.i = 3;
A.print();	// "i = 3, x = 0.0"

var C: R;
A = C;
A.print();	// "i = 0, x = 0.0"

C.x = 3.14;
A.print();	// "i = 0, x = 0.0"
\end{chapel}
\begin{chapeloutput}
i = 3, x = 0.0
i = 0, x = 0.0
i = 0, x = 0.0
\end{chapeloutput}
Prior to the first call to \chpl{R.print}, the record \chpl{A} is created and
initialized to all zeroes.  Then, its \chpl{i} field is set to \chpl{3}.
For the second call to \chpl{R.print}, the record \chpl{C} is created assigned
to \chpl{A}.  Since \chpl{C} is default-initialized to all zeroes, those zero
values overwrite both values in \chpl{A}.

The next clause demonstrates that \chpl{A} and \chpl{C} are distinct entities,
rather than two references to the same object.  Assigning \chpl{3.14}
to \chpl{C.x} does not affect the \chpl{x} field in \chpl{A}.
\end{chapelexample}

\subsection{Default Comparison Operators}
\label{Record_Comparison_Operators}
\index{records!equality}
\index{records!inequality}
\index{records!==@\chpl{==}}
\index{records!"!=@\chpl{"\"!=}}
\index{== (record)@\chpl{==} (record)}
\index{"!= (record)@\chpl{"\"!=} (record)}
Default functions to overload \chpl{==} and \chpl{\!=} are defined for
records if none are explicitly defined.
The default implementation of \chpl{==} applies \chpl{==} to each
field of the two argument records and reduces the result with
the \chpl{&&} operator.  The default implementation of \chpl{\!=}
applies \chpl{\!=} to each field of the two argument records and
reduces the result with the \chpl{||} operator.

\section{Differences between Classes and Records}
\label{Class_and_Record_Differences}
\index{records!differences with classes}

The key differences between records and classes are listed below.

\subsection{Declarations}
\label{Declaration_Differences}
\index{records!declarations!differences with classes}

Syntactically, class and record type declarations are identical, except that
they begin with the \chpl{class} and \chpl{record} keywords, respectively.
In contrast to classes, records do not support inheritance.

\subsection{Storage Allocation}
\label{Storage_Allocation_Differences}
\index{classes!allocation}
\index{records!allocation}

For a variable of record type, storage necessary to contain the data fields
has a lifetime equivalent to the scope in which it is declared.  No two record
variables share the same data.  It is not necessary to call \chpl{new} to create
a record.

By contrast, a class variable contains only a reference to a
class instance.  A class instance is created through a call to its \chpl{new}
operator.  Storage for a class instance, including storage for
the data associated with the fields in the class, is allocated and reclaimed
separately from variables referencing that instance.  The same class instance
can be referenced by multiple class variables.

\subsection{Assignment}
\label{Assignment_Differences}
\index{classes!assignment}
\index{records!assignment}

Assignment to a class variable is performed by reference, whereas assignment to
a record is performed by value.  When a variable of class type is assigned to
another variable of class type, they both become names for the same object.  In
contrast, when a record variable is assigned to another record variable, then
contents of the source record are copied into the target record field-by-field.

When a variable of class type is assigned to a record, matching fields (matched
by name) are copied from the class instance into the corresponding record
fields.  Subsequent changes to the fields in the target record have no effect
upon the class instance.

Assignment of a record to a class variable is not permitted.

\subsection{Arguments}
\label{Argument_Differences}
\index{classes!arguments}
\index{records!arguments}

Record arguments use the \chpl{const ref} intent by default - in contrast
to class arguments which pass by \chpl{const in} intent by default.

Similarly, the \chpl{this} receiver argument is passed by \chpl{const in} by
default for class methods. In contrast, it is passed by \chpl{ref} or
\chpl{const ref} by default for record methods.

\subsection{No {\em nil} Value}
\index{nil@\chpl{nil}!not provided for records}

Records do not provide a counterpart of the \chpl{nil} value.  A variable of
record type is associated with storage throughout its lifetime, so \chpl{nil}
has no meaning with respect to records.

\subsection{The {\em delete} operator}
\label{Record_Delete_Illegal}
\index{records!delete illegal}
\index{delete!illegal for records}

Calling \chpl{delete} on a record is illegal.

%REVIEW: we could discuss this:
%An explicit call to \chpl{delete} with a record argument has no effect.  The
%compiler may treat this as a hint that the record should not be accessed later
%within its scope and diagnose that case.

\subsection{Default Comparison Operators}
\label{Comparison_Operator_Differences}
\index{classes!comparison}
\index{records!comparison}

For records, the compiler will supply default comparison operators if
they are not supplied by the user.  In contrast, the user cannot redefine
\chpl{==} and \chpl{!=} for classes.  The default comparison operators
for a record examine the arguments' fields, while the comparison
operators for classes check whether the l.h.s. and r.h.s. refer to the
same class instance or are both \chpl{nil}.

\cleardoublepage
This is a stub.  This portion of the document does not exist.

\cleardoublepage
\sekshun{Tuples}
\label{Tuples}
\index{tuples}

A tuple is an ordered set of components that allows for the
specification of a light-weight record with anonymous fields.

\subsection{Tuple Expressions}
\label{Tuple_Expressions}

A tuple expression is a comma-separated list of expressions that is
enclosed in parentheses.  The number of expressions is the size of the
tuple and the types of the expressions determine the component types
of the tuple.

The syntax of a tuple expression is given by:
\begin{syntax}
tuple-expression:
  ( expression-list )

expression-list:
  expression
  expression , expression-list
\end{syntax}

\begin{example}
The statement
\begin{chapel}
var x = (1, 2);
\end{chapel}
defines a variable \chpl{x} that is a 2-tuple containing the values
\chpl{1} and \chpl{2}.
\end{example}

\subsection{Tuple Type Definitions}
\label{Tuple_Type_Definitions}
\index{tuples!types}

A tuple type is a comma-separated list of types.  The number of types
in the list defines the size of the tuple, which is part of the
tuple's type.  The syntax of a tuple type is given by:
\begin{syntax}
tuple-type:
  ( type-list )

type-list:
  type
  type , type-list
\end{syntax}

\begin{example}
Given a tuple expression \chpl{(1, 2)}, the type of the tuple value is
\chpl{(int, int)}, referred to as a 2-tuple of integers.
\end{example}

\subsection{Tuple Assignment}
\label{Tuple_Assignment}
\index{assignment!tuples}
\index{tuples!assignment}

In tuple assignment, the components of the tuple on the left-hand
side of the assignment operator are each assigned the components of
the tuple on the right-hand side of the assignment.  The assignments
are simultaneous so that each component expression on the right-hand
side is fully evaluated before being assigned to the left-hand side.

\subsection{Tuple Operators}
\label{Tuple_Operators}
\index{tuples!operators}

The arithmetic ~(\rsec{Arithmetic_Operators}), bitwise
~(\rsec{Bitwise_Operators}), shift ~(\rsec{Shift_Operators}), and
relational ~(\rsec{Relational_Operators}) operators are also defined
over tuples.

With the exception of relational operators, operations applied to two
tuples result in element-by-element application of the operation.

Relational operators over tuples apply in an "alphabetical" manner.
Each component is compared to the corresponding component or to the
scalar value until the relation is found to be true or false.

\begin{example}
In the code:
\begin{chapel}
var x = ("c", "h", "p", "l") > ("c", "h", "a", "t"); 
\end{chapel}
The value of \chpl{x} is \chpl{true}. After comparing \chpl{"c"} to
\chpl{"c"}, and \chpl{"h"} to \chpl{"h"}, \chpl{"p"} is found to be
greater than \chpl{"a"}, so the expression is \chpl{true}. 
\end{example}

\subsection{Tuple Destructuring}
\label{Tuple_Destructuring}
\index{tuples!destructuring}

When a tuple expression appears on the left-hand side of an assignment
statement, the expression on the right-hand side is said to be {\em
destructured}.  The components of the tuple on the right-hand side are
assigned to each of the component expressions on the left-hand side.
This assignment is simultaneous in that the right-hand side is
evaluated before the assignments are made.
\begin{example}
Given two variables of the same type, x and y, they can be swapped by
the following single assignment statement:
\begin{chapel}
(x, y) = (y, x);
\end{chapel}
\end{example}

\subsubsection{Variable Declarations in a Tuple}
\label{Variable_Declarations_in_a_Tuple}
\index{tuples!variable declarations}

Variables can be defined in a tuple to facilitate capturing the values
from a function that returns a tuple.  The extension to the syntax of
variable declarations is as follows:
\begin{syntax}
tuple-variable-declaration-statement:
  `config'[OPT] variable-kind tuple-variable-declaration ;

tuple-variable-declaration:
  ( tuple-identifier-list ) type-part[OPT] initialization-part
  ( tuple-identifier-list ) type-part

tuple-identifier-list:
  tuple-identifier
  tuple-identifier , tuple-identifier-list

tuple-identifier:
  identifier
  ( tuple-identifier-list )
\end{syntax}
The identifiers defined within the \sntx{tuple-identifier-list} are declared
to be new variables in the scope of the statement.  The
\sntx{type-part} and/or \sntx{initialization-part} defines a tuple
that is destructured when assigned to the new variables. The shape of the
\sntx{tuple-identifier-list} must match the shape of any specified
\sntx{type-part} or \sntx{initialization-part}.

\subsubsection{Ignoring Values with Underscore}
\label{Ignoring_Values_with_Underscore}
\index{_@\chpl{_}}

If an underscore appears as a component in a tuple expression in a
destructuring context, the expression on the right-hand side is
ignored, though it is still evaluated.

\subsection{Homogeneous Tuples}
\label{Homogeneous_Tuples}
\index{tuples!homogeneous}

A homogeneous tuple is a special-case of a general tuple where the
types of the components are identical.  Homogeneous tuples have fewer
restrictions for how they can be indexed~(\rsec{Tuple_Indexing}).

\subsubsection{Declaring Homogeneous Tuples}
\label{Declaring_Homogeneous_Tuples}

\index{* (tuples)@\chpl{*} tuples}

A homogeneous tuple type may be specified with the following syntax if
it appears as a top-level type in a variable declaration, formal
argument declaration, return type specification, or type alias
declaration:
\begin{syntax}
homogeneous-tuple-type:
  integer-parameter-expression * type

integer-parameter-expression:
  expression
\end{syntax}
The homogeneous tuple type specification is syntactic sugar for the
type explicitly replicated a number of times equal to the
\sntx{integer-parameter-expression}.
\begin{example}
The following types are equivalent:
\begin{center}
\chpl{3*int} \hspace{2pc} \chpl{(int, int, int)}
\end{center}
\end{example}

\subsection{Tuple Indexing}
\label{Tuple_Indexing}
\index{tuples!indexing}

A tuple may be indexed into by an integer.  Indexing a tuple is given
by the following syntax:
\begin{syntax}
tuple-indexing-expression:
  expression ( integer-expression )
\end{syntax}

The result of indexing a tuple by integer $k$ is the value of the
$k$th component.  If the tuple is not homogeneous, the tuple can only
be indexed by an integer parameter.  This ensures that the type of the
indexing expression is known at compile-time.

\subsection{Formal Arguments of Tuple Type}
\label{Formal_Arguments_of_Tuple_Type}

\index{formal arguments!tuples}

\begin{status}
Formal arguments of tuple type are treated as if they were records.
Conversions are not applied to the components.
\end{status}

\subsubsection{Formal Argument Declarations in a Tuple}
\label{Formal_Argument_Declarations_in_a_Tuple}
\index{formal arguments!tuples}

Formal argument declarations can be grouped into a tuple similarly to
variable declarations to facilitate passing the result of a function that
returns a tuple directly to another function.

\begin{status}Formal arguments grouped in a tuple cannot be explicitly
typed. A function with formal arguments grouped in a tuple is
therefore generic.
\end{status}

\cleardoublepage
\sekshun{Ranges}
\label{Ranges}
\index{ranges}

Ranges represent a sequence of integral values.  Ranges are
either \emph{bounded} or \emph{unbounded}.

Bounded ranges are characterized by a low bound~$l$, a high bound~$h$,
and a stride~$s$.  If the stride is positive, the values described by
the range are $l, l+s, l+2s, l+3s, ...$ such that all of the values in
the sequence are less than or equal to $h$.  If the stride is negative,
the values described by the range are $h, h+s, h+2s, h+3s, ...$ such
that all of the values in the sequence are greater than or equal to
$l$.  If $l > h$, the range is considered degenerate and represents an
empty sequence. Ranges support iteration over the values they represent
as described in ~\rsec{The_For_Loop}.

Unbounded ranges are those in which the low and/or high bounds are
omitted.  Unbounded ranges conceptually represent a countably infinite
number of values.

\subsection{Range Types}
\label{Range_Types}
\index{ranges!types}

The type of a range is characterized by three things:
(1)~the type of the values being represented, (2)~the boundedness of
the range, and (3)~whether or not the range is \emph{stridable}.

The type of the range's values is represented using a type parameter
named \emph{idxType}.  This must be one of the \chpl{int} or
\chpl{uint} types.  The default type is \chpl{int}.

\begin{openissue}
It has been hypothesized that ranges of other types, such as floating
point values, might also be of interest to represent a range of legal
tolerances, for example.  If you believe such support would be of
interest to you, please let us know.
\end{openissue}

The boundedness of the range is represented using an enumerated
parameter named \emph{boundedType} of type \chpl{BoundedRangeType}.
Legal values are \chpl{bounded}, \chpl{boundedLow},
\chpl{boundedHigh}, and \chpl{boundedNone}.  The first value specifies
a bounded range while the other three values specify a range in which
the high bound is omitted, the low bound is omitted, or both bounds
are omitted, respectively.  The default value is \chpl{bounded}.

The stridability of a range is represented by a boolean parameter
named \emph{stridable}.  If this parameter is set to true, the range's
stride can take on any signed integer value other than 0 of the same
bit-width as \chpl{idxType}.  If set to false, the range's stride is
fixed to 1.  The default value is \chpl{false}.

\begin{rationale}
The \emph{boundedType} and \emph{stridable} values of a range are used
to optimize the generated code for common cases of ranges, as well as
to optimize the implementation of domains and arrays defined using
ranges.
\end{rationale}

The syntax of a range type is summarized as follows:
\begin{syntax}
range-type:
  `range' ( named-expression-list )
\end{syntax}

\begin{example}
The following declaration declares a variable \chpl{r}
of range type that can represent ranges of 64-bit integers, with both
high and low bounds specified, and the ability to have a stride other
than 1.
\begin{chapelpre}
% test_rangeVariable.chpl
\end{chapelpre}
\begin{chapel}
var r: range(int(64), BoundedRangeType.bounded, stridable=true);
\end{chapel}
\begin{chapelpost}
writeln(r);
var i64: int(64) = 3;
r = i64..13 by 3;
writeln(r);
\end{chapelpost}
\begin{chapeloutput}
1..0
3..12 by 3
\end{chapeloutput}
\end{example}

The default value for a range is \chpl{1..0}.

\subsection{Literal Range Values}
\label{Range_Literals}
\index{ranges!literals} 

Range literals are specified as follows:
\begin{syntax}
range-literal:
  bounded-range-literal
  unbounded-range-literal
\end{syntax}

\subsubsection{Bounded Range Literals}
\label{Bounded_Ranges}
\index{ranges!bounded}

A bounded range is specified by the syntax
\begin{syntax}
bounded-range-literal:
  expression .. expression
\end{syntax}
The first expression is taken to be the lower bound $l$ and the second
expression is taken to be the upper bound $h$.  The stride of the
range is 1 and can be modified with the \chpl{by} operator as described
in~\rsec{By_Operator_For_Ranges}.

\index{ranges!integral element type}
The element type of the range type is determined by the type of the
low and high bound.  It is either \chpl{int}, \chpl{uint},
\chpl{int(64)}, or \chpl{uint(64)}.  The type is determined by
conceptually adding the low and high bounds together.  The boundedness
of such a range is \chpl{BoundedRangeType.bounded}.  The stridability of
the range is \chpl{false}.

\subsubsection{Unbounded Range Literals}
\label{Unbounded_Ranges}
\index{ranges!unbounded}

An unbounded range is specified by the syntax
\begin{syntax}
unbounded-range-literal:
  expression ..
  .. expression
  ..
\end{syntax}

The first form results in a \chpl{BoundedRangeType.boundedLow} range, the
second in a \chpl{BoundedRangeType.boundedHigh} range, and the third in
a \chpl{BoundedRangeType.boundedNone} range.

Unbounded ranges can be iterated over with zipper iteration
(~\rsec{Zipper_Iteration}) and their shape conforms to the shape of the
other iterators they are being iterated over with.
\begin{example}
The code
\begin{chapelpre}
% test_zipWithUnbounded.chpl
\end{chapelpre}
\begin{chapel}
for i in (1..5, 3..) do
  write(i, "; ");
\end{chapel}
\begin{chapelpost}
writeln();
\end{chapelpost}
\begin{chapeloutput}
(1, 3); (2, 4); (3, 5); (4, 6); (5, 7); 
\end{chapeloutput}
produces the output ``(1, 3); (2, 4); (3, 5); (4, 6); (5, 7); ''.
\end{example}

It is an error to iterate over a \chpl{BoundedRangeType.boundedNone} range,
a \chpl{BoundedRangeType.boundedLow} range with negative stride or a
\chpl{BoundedRangeType.boundedHigh} range with positive stride.

Unbounded ranges can also be used to index into ranges, domains,
arrays, and strings.  In these cases, elided bounds are inherited
from the bounds of the expression being indexed.


\subsection{Range Assignment}
\label{Range_Assignment}
\index{ranges!assignment}

Assigning one range to another results in its low, high, and stride
values being copied from the source range to the destination range.

In order for range assignment to be legal, the element type of the
source range must be implicitly coercible to the element type of the
destination range.  The two range types must have the same boundedness
parameter.  It is legal to assign a non-stridable range to a stridable
range, but illegal to assign a stridable range to a non-stridable
range unless the stridable range has a stride value of 1.


\subsection{Range Operators}
\label{Range_Operators}
\index{ranges!operators}

\subsubsection{By Operator}
\label{By_Operator_For_Ranges}
\index{ranges!strided}
\index{ranges!by operator}
\index{by@\chpl{by}}

The \chpl{by} operator can be applied to any range to create a strided
range.

The \chpl{by} operator takes a range and an integer value to yield a
new range that is strided by the integer.  Striding a strided range
results in a stride whose value is the product of the two strides.
The stride argument can either be of type \chpl{idxType} or some other
integer value that can coerce to a signed integer value of the same
bit-width as \chpl{idxType}.

\begin{example}
In the following declarations, range \chpl{r1} represents the odd integers
between 1 and 20. Range \chpl{r2} strides \chpl{r1} by two and represents
every other odd integer between 1 and 20: 1, 5, 9, ...
\begin{chapelpre}
% test_rangeByOperator.chpl
\end{chapelpre}
\begin{chapel}
var r1 = 1..20 by 2;
var r2 = r1 by 2;
\end{chapel}
\begin{chapelpost}
writeln(r1);
writeln(r2);
\end{chapelpost}
\begin{chapeloutput}
1..19 by 2
1..17 by 4
\end{chapeloutput}
\end{example}

\begin{rationale}
{\it Why isn't the high bound specified first if the stride is
negative?}  The reason for this choice is that the \chpl{by} operator
is binary, not ternary.  Given a range \chpl{R} initialized
to \chpl{1..3}, we want \chpl{R by -1} to contain the ordered sequence
$3,2,1$.  But then \chpl{R by -1} would be different than \chpl{3..1
by -1} even though it should be identical by substituting the value in
R into the expression.
\end{rationale}

\subsubsection{Count Operator}
\label{Count_Operator}
\index{ranges!count operator}

The \chpl{#} operator can be applied to a range that has a high bound,
a low bound, or both.

The \chpl{#} operator takes a range and an integral count and creates
a new range with \emph{count} elements. The stride of the resulting range is
the same as that of the initial range. It is an error for the count to
be negative.  The \emph{idxType} of the resulting range is the same
type that would be obtained by adding the integral count value to a value
with the range's \emph{idxType}.

When applied to a \chpl{BoundedRangeType.bounded} range with a positive
stride, \emph{count} elements are taken starting from the low
bound. When the stride is negative, \emph{count} elements are taken
starting from the high bound. It is an error for \emph{count} to be larger
than the length of the range.

When applied to a \chpl{BoundedRangeType.boundedLow} range, the low bound
is fixed and and the high bound is set based on the count and the
absolute value of the stride.

When applied to a \chpl{BoundedRangeType.boundedHigh} range, the high
bound is fixed and the low bound is set based on the count and the
absolute value of the stride.

It is an error to apply the count operator to a
\chpl{BoundedRangeType.boundedNone} range.

\begin{example}
The following declarations result in equivalent ranges.
\begin{chapelpre}
% test_rangeCountOperator.chpl
\end{chapelpre}
\begin{chapel}
var r1 = 2.. by -2 # 3;
var r2 = ..6 by -2 # 3;
var r3 = 0..6 by -2 # 3;
var r4 = 1..#6 by -2;
\end{chapel}
\begin{chapelpost}
writeln(r1 == r2 \&\& r2 == r3 \&\& r3 == r4);
writeln((r1, r2, r3, r4));
\end{chapelpost}
\begin{chapeloutput}
true
(2..6 by -2, 2..6 by -2, 2..6 by -2, 2..6 by -2)
\end{chapeloutput}
Each of these ranges represents the ordered set of three values: 6, 4, 2.
\end{example}

\subsubsection{Arithmetic Operators}
\label{Range_Arithmetic}
\index{ranges!arithmetic operators}

The following arithmetic operators are defined on ranges and integral
types:

\begin{chapel}
def +(r: range, s: integral): range
def +(s: integral, r: range): range
def -(r: range, s: integral): range
\end{chapel}

The \chpl{+} and \chpl{-} operators apply the scalar via the operator
to the range's low and high bounds, producing a shifted version of the
range.  The element type of the resulting range is the type of the value
that would result from an addition between the scalar value and a value
with the range's element type.  The bounded and stridable parameters for
the result range are the same as for the input range.

\begin{example}
The following code creates a bounded, non-stridable range \chpl{r}
which has an element type of \chpl{int} representing the values ${0,
  1, 2, 3}$.  It then uses the \chpl{+} operator to
create a second range \chpl{r2} representing the values ${1, 2, 3,
  4}$.  The \chpl{r2} range is bounded, non-stridable, and represents
values of type \chpl{int}.
\begin{chapelpre}
% test_rangeAdd.chpl
\end{chapelpre}
\begin{chapel}
var r = 0..3;
var r2 = r + 1;
\end{chapel}
\begin{chapelpost}
writeln((r, r2));
\end{chapelpost}
\begin{chapeloutput}
(0..3, 1..4)
\end{chapeloutput}
\end{example}


\subsubsection{Range Slicing}
\label{Range_Slicing}
\index{ranges!slicing}

Ranges can be \emph{sliced} using other ranges to create new
sub-ranges.  The resulting range represents the intersection between
the two ranges.  Range slicing is defined by using the range as a
function in a call expression where the argument is another range.
If the slicing range is unbounded in one or both directions, it
inherits its missing bounds from the range being sliced.

\begin{example}
In the following example, \chpl{r} represents the integers from 1 to
20 inclusive.  Ranges \chpl{r2} and \chpl{r3} are defined using range
slices and represent the indices from 3 to 20 and the odd integers
between 1 and 20 respectively. Range \chpl{r4} represents the odd
integers between 1 and 20 that are also divisible by 3.
\begin{chapelpre}
% test_rangeSlicing.chpl
\end{chapelpre}
\begin{chapel}
var r = 1..20;
var r2 = r[3..];
var r3 = r[1.. by 2];
var r4 = r3[0.. by 3];
\end{chapel}
\begin{chapelpost}
writeln((r, r2, r3, r4));
\end{chapelpost}
\begin{chapeloutput}
(1..20, 3..20, 1..19 by 2, 3..15 by 6)
\end{chapeloutput}
\end{example}

\subsection{Predefined Functions and Methods on Ranges}
\index{ranges!predefined functions}
\begin{protohead}
def $range$.low : idxType
\end{protohead}
\begin{protobody}
Returns the low bound of the range.
\end{protobody}

\begin{protohead}
def $range$.high : idxType
\end{protohead}
\begin{protobody}
Returns the high bound of the range.
\end{protobody}

\begin{protohead}
def $range$.stride : int(numBits(idxType))
\end{protohead}
\begin{protobody}
Returns the stride of the range.
\end{protobody}

\begin{protohead}
def $range$.length : idxType
\end{protohead}
\begin{protobody}
Returns the number of elements in the range.
\end{protobody}

\begin{protohead}
def $range$.member(i: idxType): bool
\end{protohead}
\begin{protobody}
Returns whether or not \chpl{i} is in the range.
\end{protobody}

\begin{protohead}
def $range$.member(other: range): bool
\end{protohead}
\begin{protobody}
Returns whether or not every element in other is also in this.
\end{protobody}

\begin{protohead}
def $range$.order(i: idxType): idxType
\end{protohead}
\begin{protobody}
If \chpl{i} is a member of the range, returns an integer value giving
the ordinal value of \chpl{i} within the range using 0-based indexing.
Otherwise, it returns \chpl{(-1):idxType}.
\end{protobody}

\begin{example}
The following calls show the order of index 4 in each of the given
ranges:
\begin{chapel}
(0..10).order(4) == 4
(1..10).order(4) == 3
(3..5).order(4) == 1
(0..10 by 2).order(4) == 2
(3..5 by 2).order(4) == -1
\end{chapel}
\end{example}

\cleardoublepage
\sekshun{Domains and Arrays}
\label{Domains_and_Arrays}

A {\em domain} is a description of a collection of names for data.
These names are referred to as the {\em indices} of the domain.  All
indices for a particular domain are values with some common type.
Valid types for indices are primitive types and class references or
unions, tuples or records whose fields are valid types for indices.
This excludes sequences, domains, and arrays.  Like sequences, domains
have a rank and a total order on their elements.  An {\em array} is
generically a function that maps from a {\em domain} to a collection
of variables.  Chapel supports a variety of kinds of domains and
arrays defined over those domains as well as a mechanism to allow
application specific implementations of arrays.

Arrays abstract mappings from sets of values to variables.  This key
use of data structures coupled with the generic syntactic support for
array usage increases software reusability.  By separating the sets of
values into their own abstraction, domains, distributions can be
associated with sets rather than variables.  This enables the
orthogonality of data distributions.  Distributions are discussed
in~\rsec{Locality_and_Distribution}.

\subsection{Domains}
\label{Domains}

Domains are first-class ordered sets of indices.  There are five kinds
of domains:
\begin{itemize}
\item
Arithmetic domains are rectilinear sets of Cartesian indices that can
have an arbitrary rank.
\item
Sparse domains are subsets of indices in arithmetic domains.
\item
Indefinite domains are sets of indices where the type of the index is
some type that is not an array, domain, or sequence.  Indefinite
domains define dictionaries or associative arrays implemented via hash
tables.
\item
Opaque domains are sets of anonymous indices.  Opaque domains define
graphs and unspecified sets.
\item
Enumerated domains are sets of constants defined by some enumerated
type.
\end{itemize}

\subsubsection{Domain Types}
\label{Domain_Types}

Domain types vary based on the kind of the domain.  The type of an
arithmetic domain is parameterized by the rank of the domain and the
integral type of the indices.  The type of a sparse domain is
parameterized by the type of the arithmetic domain that defines the
superset of its indices.  The type of an indefinite domain is
parameterized by the type of the index.  The type of an opaque domain
is unique.  The type of an enumerated domain is parameterized by the
enumerated type.

\begin{example}
In the code
\begin{chapel}
var D: domain(2) = [1..n, 1..n];
\end{chapel}
\chpl{D} is defined as a two-dimensional arithmetic domain and is
initialized to contain the set of indices $(i,j)$ for all $i$ and $j$
such that $i \in {1, 2, \ldots, n}$ and $j \in {1, 2, \ldots, n}$.
\end{example}

\subsubsection{Index Types}
\label{Index_Types}

Each domain has a corresponding {\em index} type which is the type of
the domain's indices qualified by its status as an index.  Variables
of this type can be declared using the following syntax:
\begin{syntax}
index-type:
  `index' ( domain-expression )
\end{syntax}
If the type of the indices of the domain is \chpl{int}, then the index
type can be converted into this type.

A value with a type that is the same as the type of the indices in a
domain but is not the index type can be converted into the index type
using a special ``method'' called \chpl{index}.

\begin{example}
In the code
\begin{chapel}
var j = D.index(i);
\end{chapel}
the type of the variable \chpl{j} is the index type of
domain \chpl{D}.  The variable \chpl{i}, which must have the same type
as the underlying type of the indices of \chpl{D}, is verified to be
in domain \chpl{D} before it is assigned to \chpl{j}.
\end{example}

Values of index type are known to be valid and may have specialized
representations to facilitate accessing arrays defined for that
domain. It may therefore be less expensive to access arrays using
values of appropriate index type rather than values of the more
general type the domain is defined over.

\begin{implementation}
In the current implementation, the index type is not distinguished
from the underlying type of the indices.  The index method is not yet
implemented.
\end{implementation}

\subsubsection{Domain Assignment}
\label{Domain_Assignment}

Domain assignment is by value.  If arrays are declared over a domain,
domain assignment impacts these arrays as discussed
in~\rsec{Association_of_Arrays_to_Domains}, but the arrays remain
associated with the same domain regardless of the assignment.

\subsubsection{Formal Arguments of Domain Type}
\label{Formal_Arguments_of_Domain_Type}

Domains are passed to functions by reference.  Formal arguments that
receive domains are aliases of the actual arguments.  It is a
compile-time error to pass a domain to a formal argument that has a
non-blank intent.

\subsubsection{Iteration over Domains}
\label{Iteration_over_Domains}

All domains support iteration via forall and for loops over the
indices in the set that the domain defines.  The type of the indices
returned by iterating over a domain is the index type of the domain.

\subsubsection{Domain Promotion of Scalar Functions}
\label{Domain_Promotion_of_Scalar_Functions}
\index{domains!promotion}

Domain promotion of a scalar function is defined over the domain type
and the type of the indices of the domain (not the index type).
Domain promotion has the same semantics as sequence promotion where
the scalar type is the indices of the domain and the promotion type is
the domain type.

\begin{example}
Given an array \chpl{A} with element type \chpl{int} declared over a
one-dimensional domain \chpl{D} with integral type \chpl{int}, then
the array can be assigned the values given by the indices in the
domain by writing
\begin{chapel}
A = D;
\end{chapel}
\end{example}

\subsection{Arrays}
\label{Arrays}

Arrays associate variables or elements with the sets of indices in a
domain.  Arrays must be declared over domains and have a specified
element type.

\subsubsection{Array Types}
\label{Array_Types}

The type of an array is parameterized by the type of the domain that
it is declared over and the element type of the array.  Array types
are given by the following syntax:
\begin{syntax}
array-type:
  [ domain-expression ] type

domain-expression:
  expression
\end{syntax}
The \sntx{domain-expression} must specify a domain that the array can
be declared over.  This can be a domain literal.  If it is a domain
literal, the square brackets around the domain literal can be omitted.

\begin{example}
In the code
\begin{chapel}
var A: [D] real;
\end{chapel}
\chpl{A} is declared to be an array over domain \chpl{D} with
elements of type \chpl{real}.
\end{example}

\begin{implementation}
Arrays of arrays are not currently supported.
\end{implementation}

\subsubsection{Array Indexing}
\label{Array_Indexing}

Arrays can be indexed by indices in the domain they are declared over.
The indexing results in an access of the element that is mapped by
this index.

\begin{example}
If \chpl{A} is an array with element type \chpl{real} declared over a
one-dimensional arithmetic domain \chpl{[1..n]}, then the first
element in \chpl{A} can be accessed via the expression \chpl{A(1)} and
set to zero via the assignment \chpl{A(1) = 0.0}.
\end{example}

Indexing into an array with a domain is call array slicing and is
discussed in the next section.

Arithmetic arrays also support indexing over the components of their
indices for multidimensional arithmetic domains (where the indices are
tuples), as described in~\rsec{Arithmetic_Array_Indexing}.

\subsubsection{Array Slicing}
\label{Array_Slicing}

An array can be indexed by a domain that has the same type as the
domain which the array was declared over.  Indexing in this manner has
the effect of array slicing.  The result is a new array declared over
the indexing domain where the elements in the array alias the elements
in the array being indexed.

\begin{example}
Given the definitions
\begin{chapel}
var OuterD: domain(2) = [0..n+1, 0..n+1];
var InnerD: domain(2) = [1..n, 1..n];
var A, B: [OuterD] real;
\end{chapel}
the assignment given by
\begin{chapel}
A(InnerD) = B(InnerD);
\end{chapel}
assigns the elements in the interior of \chpl{B} to the elements in
the interior of \chpl{A}.
\end{example}

Arithmetic arrays also support slicing by indexing into them with
arithmetic sequences or tuples of arithmetic sequences as described
in~\rsec{Arithmetic_Array_Slicing}.

\subsubsection{Array Assignment}
\label{Array_Assignment}

Array assignment is by value.  Arrays can be assigned arrays,
sequences, or domains.  If \chpl{A} is an lvalue of array type
and \chpl{B} is an expression of either array, sequence, or domain
type, then the assignment
\begin{chapel}
A = B;
\end{chapel}
is equivalent to
\begin{chapel}
forall (i,e) in (A.domain,B) do
  A(i) = e;
\end{chapel}
If the zipper iteration is illegal, then the assignment is illegal.
Notice that the assignment is implemented with the semantics of
a \chpl{forall} loop.

Arrays can also be assigned single values of their element type.  In
this case, each element in the array is assigned this value.
If \chpl{e} is an expression of the element type of the array or a
type that can be implicitly converted to the element type of the
array, then the assignment
\begin{chapel}
A = e;
\end{chapel}
is equivalent to
\begin{chapel}
forall i in A.domain do
  A(i) = e;
\end{chapel}

\subsubsection{Formal Arguments of Array Type}
\label{Formal_Arguments_of_Array_Type}

Arrays are passed to functions by reference.  Formal arguments that
receive arrays are aliases of the actual arguments.  The ordinary rule
that disallows assignment to formal arguments of blank intent does not
apply to arrays.

\subsubsection{Iteration over Arrays}
\label{Iteration_over_Arrays}

All arrays support iteration via forall and for loops over the
elements mapped to by the indices in the array's domain.

\subsubsection{Array Promotion of Scalar Functions}
\label{Array_Promotion_of_Scalar_Functions}

Array promotion of a scalar function is defined over the array type
and the element type of the array.  Array promotion has the same
semantics as sequence promotion where the scalar type is the element
type of the array and the promotion type is the array type.  The only
difference between sequence promotion and array promotion is that if a
function returns a value, the promoted function returns an array of
those values rather than a sequence of those values.  The array is
defined over the same domain as the array that was passed to the
function.  In the event of zipper promotion over multiple arrays or
both arrays and sequences, the promoted function returns a value based
on the first argument to the function that enables promotion.

\begin{implementation}
In the current implementation, promotion always returns sequences.
\end{implementation}

\begin{example}
Whole array operations is a special case of array promotion of scalar
functions.  In the code
\begin{chapel}
A = B + C;
\end{chapel}
if \chpl{A}, \chpl{B}, and \chpl{C} are arrays, this code assigns each
element in \chpl{A} the element-wise sum of the elements in \chpl{B}
and \chpl{C}.
\end{example}

\subsubsection{Array Initialization}
\label{Array_Initialization}

By default, the elements in an array are initialized to the default
values associated with the element type of the array.  There is an
expectation that this default initialization can be overridden for
performance reasons by explicitly marking the array type or variable.

The initialization expression in the declaration of an array can be
based on the indices in the domain using special array declaration
syntax that replaces both the type and initialization specifications
of the declaration:
\begin{syntax}
special-array-declaration:
  identifier-list indexed-array-type-part initialization-part

indexed-array-type-part:
  : array-type-forall-expression type

array-type-forall-expression:
  [ identifier `in' domain-expression ]

initialization-part:
  = expression
\end{syntax}
In this code, the \sntx{array-type-forall-expression} is syntactic
sugar for surrounding the \sntx{initialization-part} with this basic
forall-expression.

Given a domain expression \chpl{D}, an element type \chpl{t}, an
expression \chpl{e} that is of type \chpl{t} or that can be implicitly
converted to type \chpl{t}, then the declaration of array \chpl{A}
given by
\begin{chapel}
var A: [i in D] t = e;
\end{chapel}
is equivalent to
\begin{chapel}
var A: [D] t = [i in D] e;
\end{chapel}
The scope of the forall expression is as in the rewritten part so the
expression \chpl{e} can include references to index \chpl{i}.

\subsection{Arithmetic Domains and Arrays}
\label{Arithmetic_Domains_and_Arrays}

An arithmetic domain is a rectilinear set of Cartesian indices.
Arithmetic domains are specified as a tuple of arithmetic sequences
enclosed in square brackets.

\subsubsection{Arithmetic Domain Literals}
\label{Arithmetic_Domain_Literals}

An arithmetic domain literal is specified by the following syntax:
\begin{syntax}
arithmetic-domain-literal:
  [ arithmetic-sequence-expression-list ]

arithmetic-sequence-expression-list:
  arithmetic-sequence-expression
  arithmetic-sequence-expression , arithmetic-sequence-expression-list

arithmetic-sequence-expression:
  expression
\end{syntax}

\begin{example}
The expression \chpl{[1..5, 1..5]} defines a $5 \times 5$ arithmetic
domain with the indices $(1, 1), (1, 2), \ldots, (5, 5)$.
\end{example}

\subsubsection{Arithmetic Domain Types}
\label{Arithmetic_Domain_Types}

The type of an arithmetic domain is determined from the rank of the
arithmetic domain (the number of arithmetic sequences that define it)
and by an underlying integeral type called the {\em dimensional index
type} which must be identical to each of the integral element types of
the arithmetic sequences that define the arithmetic domain.  By
default, the dimensional index type of an arithmetic domain
is \chpl{int}.

The arithmetic domain type is specified by the syntax of a function
call to the keyword \chpl{domain} that takes at least an argument
called \chpl{rank} that is a parameter of type \chpl{int} and
optionally an integral type named \chpl{dim_type}.

\begin{example}
The expression \chpl{[1..5, 1..5]} defines an arithmetic domain with
type \chpl{domain(2,int)}.
\end{example}

\subsubsection{Arithmetic Domain Indexing}
\label{Arithmetic_Domain_Indexing}

Arithmetic domains support indexing by a value of type \chpl{int} that
is at least one and no more than the rank of the array.  Indexing into
an arithmetic domain returns the arithmetic sequence associated with
that dimension.

\begin{example}
In the code
\begin{chapel}
for i in D(1) do
  for j in D(2) do
    writeln(A(i,j));
\end{chapel}
domain \chpl{D} is iterated over by two nested loops.  The first
dimension of \chpl{D} is iterated over in the outer loop.  The second
dimension is iterated over in the inner loop.
\end{example}

\subsubsection{Arithmetic Array Indexing}
\label{Arithmetic_Array_Indexing}

In addition to being indexed by indices defined by their arithmetic
domains, arithmetic arrays can be indexed directly by values of the
dimensional index type where the number of values is equal to the rank
of the array.  This has the semantics of first moving the values into
a tuple and then indexing into the array.

\begin{example}
Given the definition
\begin{chapel}
  var ij = (i,j);
\end{chapel}
the indexing expressions \chpl{A(ij)} and \chpl{A(i,j)} are
equivalent.
\end{example}

\subsubsection{Arithmetic Array Slicing}
\label{Arithmetic_Array_Slicing}

This is a stub.  This portion of the document does not exist.

\subsubsection{Formal Arguments of Arithmetic Array Type}
\label{Formal_Arguments_of_Arithmetic_Array_Type}

This is a stub.  This portion of the document does not exist.

\subsection{Sparse Domains and Arrays}
\label{Sparse_Domains_and_Arrays}

This is a stub.  This portion of the document does not exist.

\subsubsection{Adding Indices to Sparse Domains}
\label{Adding_Indices_to_Sparse_Domains}

This is a stub.  This portion of the document does not exist.

\subsubsection{Removing Indices from Sparse Domains}
\label{Removing_Indices_from_Sparse_Domains}

This is a stub.  This portion of the document does not exist.

\subsection{Indefinite Domains and Arrays}
\label{Indefinite_Domains_and_Arrays}

This is a stub.  This portion of the document does not exist.

\subsubsection{Indefinite Domain and Array Types}
\label{Indefinite_Domain_and_Array_Types}

This is a stub.  This portion of the document does not exist.

\subsubsection{Indefinite Domain Index Types}
\label{Indefinite_Domain_Index_Types}

This is a stub.  This portion of the document does not exist.

\subsubsection{Adding Indices to Indefinite Domains}
\label{Adding_Indices_to_Indefinite_Domains}

This is a stub.  This portion of the document does not exist.

\subsubsection{Removing Indices from Indefinite Domains}
\label{Removing_Indices_from_Indefinite_Domains}

This is a stub.  This portion of the document does not exist.

\subsection{Opaque Domains and Arrays}
\label{Opaque_Domains_and_Arrays}

This is a stub.  This portion of the document does not exist.

\subsubsection{Opaque Domain and Array Types}
\label{Opaque_Domain_and_Array_Types}

This is a stub.  This portion of the document does not exist.

\subsubsection{Opaque Domain Index Types}
\label{Opaque_Domain_Index_Types}

This is a stub.  This portion of the document does not exist.

\subsubsection{Adding Indices to Opaque Domains}
\label{Adding_Indices_to_Opaque_Domains}

This is a stub.  This portion of the document does not exist.

\subsubsection{Removing Indices from Opaque Domains}
\label{Removing_Indices_from_Opaque_Domains}

This is a stub.  This portion of the document does not exist.

\subsection{Enumerated Domains and Arrays}
\label{Enumerated_Domains_and_Arrays}

This is a stub.  This portion of the document does not exist.

\subsubsection{Enumerated Domain and Array Types}
\label{Enumerated_Domain_and_Array_Types}

This is a stub.  This portion of the document does not exist.

\subsubsection{Enumerated Domain Index Types}
\label{Enumerated_Domain_Index_Types}

This is a stub.  This portion of the document does not exist.

\subsection{Association of Arrays to Domains}
\label{Association_of_Arrays_to_Domains}

This is a stub.  This portion of the document does not exist.

\subsubsection{Preservative Reallocation of Arrays}
\label{Preservative_Reallocation_of_Arrays}

This is a stub.  This portion of the document does not exist.

\subsubsection{Destructive Reallocation of Arrays}
\label{Destructive_Reallocation_of_Arrays}

This is a stub.  This portion of the document does not exist.

\subsection{Subdomains}
\label{Subdomains}

This is a stub.  This portion of the document does not exist.

\subsubsection{Subdomain Definition}
\label{Subdomain_Definition}

This is a stub.  This portion of the document does not exist.

\subsubsection{Association of Subdomains to Domains}
\label{Association_of_Subdomains_to_Domains}

This is a stub.  This portion of the document does not exist.

\subsection{Predefined Functions and Methods on Domains and Arrays}

{\bf Functions and Methods on Domains}

\begin{chapel}
def Domain.numIndices: dim_type
\end{chapel}
Returns the number of indices in the domain.

{\bf Functions and Methods on Arrays}

\begin{chapel}
def Array.numElements: this.domain.dim_type
\end{chapel}
Returns the number of elements in the array.

\cleardoublepage
\sekshun{Iterators}
\label{Iterators}
\index{iterators}

An iterator is a function that conceptually returns multiple values
rather than simply a single value.

\begin{openissue}
The parallel iterator story is under development.  It is expected that
the specification will be expanded regarding parallel iterators soon.
\end{openissue}

\subsection{Iterator Function Definitions}
\label{Iterator_Function_Definitions}
\index{iterator function definitions}

The syntax to declare an iterator function (or simply, ``iterator''), is given
by:
\begin{syntax}
iterator-declaration-statement:
  `iter' iterator-name argument-list[OPT] var-param-type-clause[OPT] where-clause[OPT]
  iterator-body

iterator-name:
  identifier

iterator-body:
  block-statement
  yield-statement
\end{syntax}

The syntax of an iterator declaration is similar to a function declaration, with
some key differences:
\begin{itemize}
\item The keyword \chpl{iter} is used instead of the keyword \chpl{proc}.
\item The name of the iterator cannot overload any operator.
\item \chpl{yield} statements may appear in the body of an iterator, but not in
a regular function.
\end{itemize}

\subsection{The Yield Statement}
\label{The_Yield_Statement}
\index{yield@\chpl{yield}}

The yield statement can only appear in iterators.  The syntax of the
yield statement is given by
\begin{syntax}
yield-statement:
  `yield' expression ;
\end{syntax}

When an iterator is executed and a \chpl{yield} is encountered, the value of the yield
expression is returned.  However, the state of execution of the iterator is
saved.  On its next invocation, execution resumes from the point immediately
following that \chpl{yield} statement.

When a \chpl{return} is encountered, the iterator finishes without yielding another
index value.  The \chpl{return} statements appearing in an iterator are not
permitted to have a return value.
An iterator also completes after the last
statement in the iterator function is executed.
An iterator need not contain any yield statements.

\subsection{Iterator Calls}
\label{Iterator_Calls}

The syntax used to call an interator is given by:
\begin{syntax}
iterator-call-expression:
  call-expression
\end{syntax}
This is identical to the function-call syntax.
%REVIEW: hilde
% Can iterator definitions and uses have field syntax?

All details of the \sntx{iterator-call-expression} semantics --- including
resolution, the use of parentheses versus brackets to delimit the parameter
list, calling the iterator without an argument list and named arguments ---
are identical with the corresponding semantics for function calls.
See~\rsec{Function_Calls}.

However, the result of an iterator call depends upon its context, as described below.

\subsubsection{Iterators in For and Forall Loops}
\label{Iterators_in_For_and_Forall_Loops}

When an iterator is accessed via a for or forall loop, the iterator is
evaluated alongside the loop body in an interleaved manner.  For each
iteration, the iterator yields a value and the body is executed.

\subsubsection{Iterators as Arrays}
\label{Iterators_as_Arrays}
\index{iterators!and arrays}

If an iterator function is captured into a new variable declaration or
assigned to an array, the iterator is iterated over in total and the
expression evaluates to a one-dimensional arithmetic array that
contains the values returned by the iterator on each iteration.
\begin{example}
Given an iterator
\begin{chapel}
iter squares(n: int): int {
  for i in 1..n do
    yield i * i;
}
\end{chapel}
\begin{chapelpost}
writeln(squares(5));
\end{chapelpost}
\begin{chapeloutput}
1 4 9 16 25
\end{chapeloutput}
the expression \chpl{squares(5)} evaluates to the array \chpl{1, 4, 9, 16, 25}.
\end{example}

\subsubsection{Iterators and Generics}
\label{Iterators_and_Generics}
\index{iterators!and generics}

An iterator call expression can be passed to a generic function argument that
has neither a declared type nor default value
(\rsec{Formal_Arguments_without_Types}).
In this case the iterator is passed without being evaluated.
Within the generic function the corresponding formal argument
can be used as an iterator, e.g. in for loops.
The arguments to the iterator call expression, if any, are evaluated
at the call site, i.e. prior to passing the iterator to the generic function.

\subsubsection{Recursive Iterators}
\label{Recursive_Iterators}
\index{iterators!recursive}

Recursive iterators are allowed. A recursive iterator invocation is
typically made by iterating over it in a loop.


\begin{example}
A post-order traversal of a tree data structure could be written like this:
\begin{chapel}
iter postorder(tree: Tree): string {
  if tree != nil {
    for child in postorder(tree.left) do
      yield child;
    for child in postorder(tree.right) do
      yield child;
    yield tree.data;
  }
}
\end{chapel}
By contrast, using calls \chpl{postorder(tree.left)}
and \chpl{postorder(tree.right)} as stand-alone statements would
result in generating temporary arrays containing the outcomes of these
recursive calls, which would then be discarded.
\end{example}

\subsection{Parallel Iterators}
\label{Parallel_Iterators}

Iterators used in explicit forall-statements or -expressions must be
parallel iterators.  Reductions, scans and promotion over serial
iterators will be serialized.

The definition of parallel iterators is forthcoming.  Parallel
iterators are defined over standard constructs in Chapel such as
ranges, domains, and arrays (including Block- and Cyclic-distributed
domains and arrays), thereby allowing these constructs to be used with
forall-statements and -expressions.

\cleardoublepage
\sekshun{Generics}
\label{Generics}

Chapel supports generic functions and types that are parameterizable
over both types and parameters.  The generic functions and types look
similar to non-generic functions and types already discussed.

\section{Generic Functions}
\label{Generic_Functions}
\index{functions!generic}
\index{generics!functions}

A function is generic if any of the following conditions hold:
\begin{itemize}
\item
Some formal argument is specified with an intent of \chpl{type} or
\chpl{param}.
\item
Some formal argument has no specified type and no default value.
\item
Some formal argument is specified with a queried type.
\item
The type of some formal argument is a generic type, e.g., \chpl{List}.
Queries may be inlined in generic types, e.g., \chpl{List(?eltType)}.
\item
The type of some formal argument is an array type where either the
element type is queried or omitted or the domain is queried or
omitted.
\end{itemize}
These conditions are discussed in the next sections.

\subsection{Formal Type Arguments}
\label{Formal_Type_Arguments}
\index{intents!type@\chpl{type}}

If a formal argument is specified with intent \chpl{type}, then a type
must be passed to the function at the call site.  A copy of the
function is instantiated for each unique type that is passed to this
function at a call site.  The formal argument has the semantics of a
type alias.
\begin{example}
The following code defines a function that takes two types at the call
site and returns a 2-tuple where the types of the components of the
tuple are defined by the two type arguments and the values are
specified by the types default values.
\begin{chapelpre}
% build2tuple.chpl
\end{chapelpre}
\begin{chapel}
proc build2Tuple(type t, type tt) {
  var x1: t;
  var x2: tt;
  return (x1, x2);
}
\end{chapel}
This function is instantiated with ``normal'' function call syntax
where the arguments are types:
\begin{chapel}
var t2 = build2Tuple(int, string);
t2 = (1, "hello");
\end{chapel}
\begin{chapelpost}
writeln(t2);
\end{chapelpost}
\begin{chapeloutput}
(1, hello)
\end{chapeloutput}
\end{example}

\subsection{Formal Parameter Arguments}
\label{Formal_Parameter_Arguments}
\index{intents!param@\chpl{param}}

If a formal argument is specified with intent \chpl{param}, then a
parameter must be passed to the function at the call site.  A copy of
the function is instantiated for each unique parameter that is passed
to this function at a call site.  The formal argument is a parameter.
\begin{example}
The following code defines a function that takes an integer parameter
\chpl{p} at the call site as well as a regular actual argument of
integer type \chpl{x}.  The function returns a homogeneous tuple of
size \chpl{p} where each component in the tuple has the value of
\chpl{x}.
\begin{chapelpre}
% fillTuple.chpl
\end{chapelpre}
\begin{chapel}
proc fillTuple(param p: int, x: int) {
  var result: p*int;
  for param i in 1..p do
    result(i) = x;
  return result;
}
\end{chapel}
\begin{chapelpost}
writeln(fillTuple(3,3));
\end{chapelpost}
\begin{chapeloutput}
(3, 3, 3)
\end{chapeloutput}
The function call \chpl{fillTuple(3, 3)} returns a 3-tuple where each
component contains the value \chpl{3}.
\end{example}

\subsection{Formal Arguments without Types}
\label{Formal_Arguments_without_Types}
\index{formal arguments!without types}

If the type of a formal argument is omitted, the type of the formal
argument is taken to be the type of the actual argument passed to the
function at the call site.  A copy of the function is instantiated for
each unique actual type.
\begin{example}
The example from the previous section can be extended to be generic on
a parameter as well as the actual argument that is passed to it by
omitting the type of the formal argument \chpl{x}.  The following code
defines a function that returns a homogeneous tuple of size \chpl{p}
where each component in the tuple is initialized to \chpl{x}:
\begin{chapelpre}
% fillTuple2.chpl
\end{chapelpre}
\begin{chapel}
proc fillTuple(param p: int, x) {
  var result: p*x.type;
  for param i in 1..p do
    result(i) = x;
  return result;
}
\end{chapel}
\begin{chapelpost}
var x = fillTuple(3, 3.14);
writeln(x);
writeln(typeToString(x.type));
\end{chapelpost}
\begin{chapeloutput}
(3.14, 3.14, 3.14)
3*real
\end{chapeloutput}
In this function, the type of the tuple is taken to be the type of the
actual argument.  The call \chpl{fillTuple(3, 3.14)} returns a 3-tuple
of real values \chpl{(3.14, 3.14, 3.14)}.  The return type is
\chpl{(real, real, real)}.
\end{example}

\subsection{Formal Arguments with Queried Types}
\label{Formal_Arguments_with_Queried_Types}
\index{formal arguments!queried types}

If the type of a formal argument is specified as a queried type, the
type of the formal argument is taken to be the type of the actual
argument passed to the function at the call site.  A copy of the
function is instantiated for each unique actual type.  The queried
type has the semantics of a type alias.
\begin{example}
The example from the previous section can be rewritten to use a
queried type for clarity:
\begin{chapelpre}
% fillTuple3.chpl
\end{chapelpre}
\begin{chapel}
proc fillTuple(param p: int, x: ?t) {
  var result: p*t;
  for param i in 1..p do
    result(i) = x;
  return result;
}
\end{chapel}
\begin{chapelpost}
var x = fillTuple(3, 3.14);
writeln(x);
writeln(typeToString(x.type));
\end{chapelpost}
\begin{chapeloutput}
(3.14, 3.14, 3.14)
3*real
\end{chapeloutput}
\end{example}

\subsection{Formal Arguments of Generic Type}
\label{Formal_Arguments_of_Generic_Type}
\index{formal arguments!generic types}

If the type of a formal argument is a generic type, the type of the
formal argument is taken to be the type of the actual argument passed
to the function at the call site with the constraint that the type of
the actual argument is an instantiation of the generic type.  A copy
of the function is instantiated for each unique actual type.
\begin{example}
The following code defines a function \chpl{writeTop} that takes an
actual argument that is a generic stack
(see~\rsec{Example_Generic_Stack}) and outputs the top element of the
stack.  The function is generic on the type of its argument.
\begin{chapel}
proc writeTop(s: Stack) {
  write(s.top.item);
}
\end{chapel}
\end{example}

Types and parameters may be queried from the top-level types of formal
arguments as well.  In the example above, the formal argument's type
could also be specified as \chpl{Stack(?type)} in which case the
symbol \chpl{type} is equivalent to \chpl{s.itemType}.

Note that generic types which have default values for all of their
generic fields, \emph{e.g. range}, are not generic when simply
specified and require a query to mark the argument as generic.  For
simplicity, the identifier may be omitted.
\begin{example}
The following code defines a class with a type field that has a
default value.  Function \chpl{f} is defined to take an argument of
this class type where the type field is instantiated to the default.
Function \chpl{g}, on the other hand, is generic on its argument
because of the use of the question mark.
\begin{chapel}
class C {
  type t = int;
}
proc f(c: C) {
  // c.type is always int
}
proc g(c: C(?)) {
  // c.type may not be int
}
\end{chapel}
\end{example}

\index{where@\chpl{where}!implicit}
The generic type may be specified with some queries and some exact
values.  Thesse exact values result in \emph{implicit where clauses}
for the purpose of function resolution.
\begin{example}
Given the class definition
\begin{chapel}
class C {
  type t;
  type tt;
}
\end{chapel}
then the function definition
\begin{chapel}
proc f(c: C(?t,real)) {
  // body
}
\end{chapel}
is equivalent to
\begin{chapel}
proc f(c: C(?t,?tt)) where tt == real {
  // body
}
\end{chapel}
\end{example}
For tuples with query arguments, an implicit where clause is always
created to ensure that the size of the actual tuple matches the
implicitly specified size of the formal tuple.
\begin{example}
The function definition
\begin{chapel}
proc f(tuple: (?t,real)) {
  // body
}
\end{chapel}
is equivalent to
\begin{chapel}
proc f(tuple: (?t,?tt)) where tuple.size == 2 && tt == real {
  // body
}
\end{chapel}
\end{example}

\index{integral@\chpl{integral}}
\index{numeric@\chpl{numeric}}
\index{enumerated@\chpl{enumerated}}
The generic types \chpl{integral}, \chpl{numeric} and \chpl{enumerated}
are generic types that can only be instantiated with, respectively, the
signed and unsigned integral types, all of the numeric types, and
enumerated types.

\subsection{Formal Arguments of Generic Array Types}
\label{Formal_Arguments_of_Generic_Array_Types}
\index{formal arguments!array types}

If the type of a formal argument is an array where either the domain
or the element type is queried or omitted, the type of the formal
argument is taken to be the type of the actual argument passed to the
function at the call site.  If the domain is omitted, the domain of
the formal argument is taken to be the domain of the actual argument.

A queried domain may not be modified via the name to which it is bound
(see~\rsec{Association_of_Arrays_to_Domains} for rationale).

\section{Function Visibility in Generic Functions}
\label{Function_Visibility_in_Generic_Functions}
\index{generics!function visibility}

Function visibility in generic functions is altered depending on the
instantiation.  When resolving function calls made within generic
functions, the visible functions are taken from any call site at which
the generic function is instantiated for each particular
instantiation.  The specific call site chosen is arbitrary and it is
referred to as the \emph{point of instantiation}.

For function calls that specify the module
explicitly~(\rsec{Explicit_Naming}), an implicit use of the specified
module exists at the call site.

\begin{example}
Consider the following code which defines a generic
function \chpl{bar}:
\begin{chapelpre}
% point_of_instantiation.chpl
\end{chapelpre}
\begin{chapel}
module M1 {
  record R {
    var x: int;
    proc foo() { }
  }
}

module M2 {
  proc bar(x) {
    x.foo();
  }
}

module M3 {
  use M1, M2;
  proc main() {
    var r: R;
    bar(r);
  }
}
\end{chapel}
\begin{chapeloutput}
\end{chapeloutput}
In the function \chpl{main}, the variable \chpl{r} is declared to be
of type \chpl{R} defined in module \chpl{M1} and a call is made to the
generic function \chpl{bar} which is defined in module \chpl{M2}.
This is the only place where \chpl{bar} is called in this program and
so it becomes the point of instantiation for \chpl{bar} when the
argument \chpl{x} is of type \chpl{R}.  Therefore, the call to
the \chpl{foo} method in \chpl{bar} is resolved by looking for visible
functions from within \chpl{main} and going through the use of
module \chpl{M1}.
\end{example}

If the generic function is only called indirectly through dynamic
dispatch, the point of instantiation is defined as the point at which
the derived type (the type of the implicit \chpl{this} argument) is
defined or instantiated (if the derived type is generic).

\begin{rationale}
Visible function lookup in Chapel's generic functions is handled
differently than in C++'s template functions in that there is no split
between dependent and independent types.

Also, dynamic dispatch and instantiation is handled differently.
Chapel supports dynamic dispatch over methods that are generic in some
of its formal arguments.

Note that the Chapel lookup mechanism is still under development and
discussion.  Comments or questions are appreciated.
\end{rationale}

\section{Generic Types}
\label{Generic_Types}
\index{generics!types}
\index{types!generic}
\index{generics!classes}
\index{classes!generic}
\index{generics!records}
\index{records!generic}

Generic types are generic classes and generic records.
A class or record is generic if it contains one or more
\index{generics!fields}
\index{fields!generic}
generic fields. A generic field is one of:
\begin{itemize}
\item a specified or unspecified type alias,
\item a parameter field, or
\item a \chpl{var} or \chpl{const} field that has no type and no initialization
expression.
\end{itemize}

\mbox{} % push the following line to the next page

For each generic field, the class or record is parameterized over:
\begin{itemize}
\item the type bound to the type alias,
\item the value of the parameter field, or
\item the type of the \chpl{var} or \chpl{const} field, respectively.
\end{itemize}
Correspondingly, the class or record is instantiated with a set
of types and parameter values, one type or value per generic field.

% Here are the aspects to be defined for each kind of generic field:
% - what it makes the class/record generic over
% - the type constructor arg that gets created
% - the default constructor arg that gets created
% - the requirements on the corresponding user-defined constructor arg
% - for each of the above args:
%    - what kind of actual it accepts (type, param, value)
%    - what is the semantics;
%      i.e. how it corresponds to the class/record's genericity
%    - what is the arg's default, if any
% 
% In the presentation below, some of these aspects are discussed
% in the field-kind-specific subsections, some in the constructor-specific
% subsections, some in both.  I.e. there is an overlap between
% field-kind and constructor subsections; that should be OK but feel free
% to clean up.
% 
% It would be cool to summarize that in a table
% (one dimension: field kinds; the other dimension: aspects).

\subsection{Type Aliases in Generic Types}
\label{Type_Aliases_in_Generic_Types}
\index{type aliases!in classes or records}
\index{fields!type alias}

If a class or record defines a type alias, the class or record
is generic over the type that is bound to that alias.
% Type aliases defined in a class or a record can be unspecified type
% aliases; type aliases that are not bound to a type.  If a class or
% record contains an unspecified type alias, the aliased type must be
% specified whenever the type is used.
Such a type alias is accessed as if it were a field;
similar to a parameter field, it cannot be assigned
except in its declaration.

The type alias becomes an argument with intent \chpl{type} to
the compiler-generated constructor (\rsec{Generic_Compiler_Generated_Constructors})
for its class or record. This makes the compiler-generated constructor generic.
The type alias also becomes an argument with intent \chpl{type} to
the type constructor (\rsec{Type_Constructors}).
If the type alias declaration binds it to a type, that type
becomes the default for these arguments, otherwise they have no defaults.

The class or record is instantiated by binding the type alias
to the actual type passed to the corresponding argument of
a user-defined (\rsec{Generic_User_Constructors})
or compiler-generated constructor or type constructor.
If that argument has a default, the actual type can be omitted, in
which case the default will be used instead.

\begin{example}
The following code defines a class called \chpl{Node} that implements
a linked list data structure.  It is generic over the type of the
element contained in the linked list.
\begin{chapelpre}
% NodeClass.chpl
\end{chapelpre}
\begin{chapel}
class Node {
  type eltType;
  var data: eltType;
  var next: Node(eltType);
}
\end{chapel}
\begin{chapelpost}
var n: Node(real) = new Node(real, 3.14);
writeln(n.data);
writeln(n.next);
writeln(typeToString(n.next.type));
\end{chapelpost}
\begin{chapeloutput}
3.14
nil
Node(real)
\end{chapeloutput}
The call \chpl{new Node(real, 3.14)} creates a node in the linked list
that contains the value \chpl{3.14}.  The \chpl{next} field is set to
nil.  The type specifier \chpl{Node} is a generic type and cannot be
used to define a variable.  The type specifier \chpl{Node(real)}
denotes the type of the \chpl{Node} class instantiated over
\chpl{real}.  Note that the type of the \chpl{next} field is specified
as \chpl{Node(eltType)}; the type of \chpl{next} is the same type as
the type of the object that it is a field of.
\end{example}

\subsection{Parameters in Generic Types}
\label{Parameters_in_Generic_Types}
\index{parameters!in classes or records}
\index{fields!parameter}

If a class or record defines a parameter field, the class or record
is generic over the value that is bound to that field.
% A parameter defined in a class or record is accessed as if it were a
% field.  This access returns a parameter.  
The parameter becomes an argument with intent \chpl{param} to the
compiler-generated constructor (\rsec{Generic_Compiler_Generated_Constructors})
for that class or record.  This makes the compiler-generated
constructor generic.  The parameter also becomes an argument
with intent \chpl{param} to the type  constructor (\rsec{Type_Constructors}).
If the parameter declaration has an initialization expression, that expression
becomes the default for these arguments, otherwise they have no defaults.

The class or record is instantiated by binding the parameter
to the actual value passed to the corresponding argument of
a user-defined (\rsec{Generic_User_Constructors})
or compiler-generated constructor or type constructor.
If that argument has a default, the actual value can be omitted, in
which case the default will be used instead.

\begin{example}
The following code defines a class called \chpl{IntegerTuple} that is
generic over an integer parameter which defines the number of
components in the class.
\begin{chapelpre}
% IntegerTuple.chpl
\end{chapelpre}
\begin{chapel}
class IntegerTuple {
  param size: int;
  var data: size*int;
}
\end{chapel}
\begin{chapelpost}
var x = new IntegerTuple(3);
writeln(x.data);
\end{chapelpost}
\begin{chapeloutput}
(0, 0, 0)
\end{chapeloutput}
The call \chpl{new IntegerTuple(3)} creates an instance of the
\chpl{IntegerTuple} class that is instantiated over parameter
\chpl{3}.  The field \chpl{data} becomes a 3-tuple of integers.  The
type of this class instance is \chpl{IntegerTuple(3)}.  The type
specified by \chpl{IntegerTuple} is a generic type.
\end{example}

\subsection{Fields without Types}
\label{Fields_without_Types}
\index{fields!variable and constant, without types}
\index{variables!in classes or records}
\index{constants!in classes or records}

If a \chpl{var} or \chpl{const} field in a class or record has no specified type or
initialization expression, the class or record is generic over the
type of that field.  The field becomes an argument with blank intent to
the compiler-generated constructor (\rsec{Generic_Compiler_Generated_Constructors}).
That argument has no specified type and no default
value. This makes the compiler-generated constructor generic.
The field also becomes an argument with \chpl{type} intent and no default
to the type constructor (\rsec{Type_Constructors}).
Correspondingly, an actual value must always be passed to the default
constructor argument and an actual type to the type constructor argument.

The class or record is instantiated by binding the type of the field
to the type of the value passed to the corresonding argument
of a user-defined (\rsec{Generic_User_Constructors}) or compiler-generated constructor (\rsec{Generic_Compiler_Generated_Constructors}).
When the type constructor is invoked, the class or record is instantiated
by binding the type of the field to the actual type passed to
the corresponding argument.

\begin{example}
The following code defines another class called \chpl{Node} that
implements a linked list data structure.  It is generic over the type
of the element contained in the linked list.  This code does not
specify the element type directly in the class as a type alias but
rather omits the type from the \chpl{data} field.
\begin{chapelpre}
% fieldWithoutType.chpl
\end{chapelpre}
\begin{chapel}
class Node {
  var data;
  var next: Node(data.type) = nil;
}
\end{chapel}
A node with integer element type can be defined in the call to the
constructor.  The call \chpl{new Node(1)} defines a node with the
value \chpl{1}.  The code
\begin{chapel}
var list = new Node(1);
list.next = new Node(2);
\end{chapel}
\begin{chapelpost}
writeln(list.data);
writeln(list.next.data);
\end{chapelpost}
\begin{chapeloutput}
1
2
\end{chapeloutput}
defines a two-element list with nodes containing the values \chpl{1}
and \chpl{2}.  The type of each object could be specified
as \chpl{Node(int)}.
\end{example}

\subsection{The Type Constructor}
\label{Type_Constructors}
\index{generics!type constructor}
\index{constructors!type constructors}

A type constructor is automatically created for each class or record.
A type constructor is a type function (\rsec{Type_Functions}) that has
the same name as the class or record.  It takes one argument per the
class's or record's generic field, including fields inherited from the
superclasses, if any.
The formal argument has intent \chpl{type} for a type alias field and is a
parameter for a parameter field. It accepts the type to be bound
to the type alias and the value to be bound to the parameter, respectively.
For a generic \chpl{var} or \chpl{const} field, the corresponding
formal argument also has intent \chpl{type}. It accepts the type
of the field, as opposed to a value as is the case for a parameter field.
The formal arguments occur in the same order as the fields are
declared and have the same names as the corresponding fields.
Unlike the compiler-generated constructor, the type constructor has only
those arguments that correspond to generic fields.

A call to a type constructor accepts actual types and parameter values
and returns the type of the class or record that is instantiated
appropriately for each field
(\rsec{Type_Aliases_in_Generic_Types}, \rsec{Parameters_in_Generic_Types},
\rsec{Fields_without_Types}).
\index{generics!instantiated type}
Such an instantiated type must be used as the type of a variable,
array element, non-generic formal argument, and in other cases
where uninstantiated generic class or record types are not allowed.

When a generic field declaration has an initialization expression
or a type alias is specified, that initializer becomes the default value
for the corresponding type constructor argument.  Uninitialized
fields, including all generic \chpl{var} and \chpl{const} fields,
and unspecified type aliases result in arguments with no defaults;
actual types or values for these arguments must always be provided
when invoking the type constructor.

\subsection{Generic Methods}
\label{Generic_Methods}
\index{generics!methods}

All methods bound to generic classes or records, including
constructors, are generic over the implicit \chpl{this} argument.
This is in addition to being generic over any other argument that is generic.

\subsection{The Compiler-Generated Constructor}
\label{Generic_Compiler_Generated_Constructors}
\index{generics!constructors!compiler-generated}
\index{constructors!compiler-generated!for generic classes or records}

If no user-defined constructors are supplied for a given generic class, the
compiler generates one following in a manner similar to that for concrete
classes (\rsec{The_Compiler_Generated_Constructor}).
However, the compiler-generated constructor for a generic class or record
(\rsec{The_Compiler_Generated_Constructor}) is generic over each argument that
corresponds to a generic field, as specified above.
The argument has intent \chpl{type} for a type alias field and is a
parameter for a parameter field. It accepts the type to be bound
to the type alias and the value to be bound to the parameter, respectively.
This is the same as for the type constructor.
For a generic \chpl{var} or \chpl{const} field, the corresponding
formal argument has the blank intent and accepts the value
for the field to be initialized with. The type of the field
is inferred automatically to be the type of the initialization value.

The default values for the generic arguments of the compiler-generated constructor
are the same as for the type constructor (\rsec{Type_Constructors}).
For example, the arguments corresponding to the generic \chpl{var}
and \chpl{const} fields, if any, never have defaults, so the corresponding
actual values must always be provided.

\subsection{User-Defined Constructors}
\label{Generic_User_Constructors}
\index{generics!constructors!user-defined}
\index{constructors!user-defined!for generic classes or records}

If a generic field of a class does not have an initialization expression
or a type alias is unspecified, each user-defined constructor for that
class must provide a formal argument whose name
matches the name of the field.

If the name of a formal argument in a user-defined constructor matches the name
of a generic field that does not have an initialization
expression, is a type alias, or is a parameter field, the field is
automatically initialized at the beginning of the constructor invocation
to the actual value of that argument.
This is done by passing that formal argument to the implicit invocation
of the compiler-generated constructor during default-initialization (\rsec{Default_Initialization}).

%%  The following story is nicer but it's not how it is implemented:
%If the name of a formal argument in a class constructor
%matches the name of a generic field, the field is automatically initialized
%to the actual value for that argument upon the constructor invocation.
%If the generic field does not have an initialization expression,
%such a matching formal argument must be provided in each constructor
%for that class.

\begin{example}
In the following code:
\begin{chapelpre}
% constructorsForGenericFields.chpl
\end{chapelpre}
\begin{chapel}
class MyGenericClass {
  type t1;
  param p1;
  const c1;
  var v1;
  var x1: t1; // this field is not generic

  type t5 = real;
  param p5 = "a string";
  const c5 = 5.5;
  var v5 = 555;
  var x5: t5; // this field is not generic

  proc MyGenericClass(c1, v1, type t1, param p1) { }
  proc MyGenericClass(type t5, param p5, c5, v5, x5,
                     type t1, param p1, c1, v1, x1) { }
}  // class MyGenericClass

var g1 = new MyGenericClass(11, 111, int, 1);
var g2 = new MyGenericClass(int, "this is g2", 3.3, 333, 3333,
                            real, 2, 222, 222.2, 22);
\end{chapel}
\begin{chapelpost}
writeln(g1);
writeln(g2);
\end{chapelpost}
\begin{chapeloutput}
{p1 = 1, c1 = 11, v1 = 111, x1 = 0, p5 = a string, c5 = 5.5, v5 = 555, x5 = 0.0}
{p1 = 2, c1 = 222, v1 = 222.2, x1 = 0.0, p5 = this is g2, c5 = 5.5, v5 = 555, x5 = 0}
\end{chapeloutput}
The arguments \chpl{t1}, \chpl{p1}, \chpl{c1}, and \chpl{v1} are
required in all constructors for \chpl{MyGenericClass}. They can appear
in any order. Both \chpl{MyGenericClass} constructors initialize the
corresponding fields implicitly because these fields do not have initialization
expressions. The second constructor also initializes implicitly
the fields \chpl{t5} and \chpl{p5} because they are a type field
and a parameter field. It does not initialize the fields \chpl{c5}
and \chpl{v5} because they have initialization expressions, or
the fields \chpl{x1} and \chpl{x5} because they are not generic fields
(even though they are of generic types).
\end{example}

\begin{openissue}
The design of constructors, especially for generic classes, is
under development, so the above specification may change.
\end{openissue}

\section{Where Expressions}
\label{Where_Expressions}
\index{where@\chpl{where}}
\index{generics!where@\chpl{where}}

The instantiation of a generic function can be constrained by {\em
where clauses}.  A where clause is specified in the definition of a
function~(\rsec{Function_Definitions}).  When a function is
instantiated, the expression in the where clause must be a parameter
expression and must evaluate to either \chpl{true} or \chpl{false}.
If it evaluates to \chpl{false}, the instantiation is rejected and the
function is not a possible candidate for function resolution.
Otherwise, the function is instantiated.
\begin{example}
Given two overloaded function definitions
\begin{chapelpre}
% whereClause.chpl
\end{chapelpre}
\begin{chapel}
proc foo(x) where x.type == int { writeln("int"); }
proc foo(x) where x.type == real { writeln("real"); }
\end{chapel}
\begin{chapelpost}
foo(3);
foo(3.14);
\end{chapelpost}
\begin{chapeloutput}
int
real
\end{chapeloutput}
the call foo(3) resolves to the first definition because when the
second function is instantiated the where clause evaluates to false.
\end{example}

\section{User-Defined Compiler Diagnostics}
\label{User_Defined_Compiler_Errors}
\index{compiler diagnostics!user-defined}
\index{compiler errors!user-defined}
\index{compiler warnings!user-defined}
\index{compilerError}
\index{compilerWarning}

The special compiler diagnostic function calls \chpl{compilerError}
and \chpl{compilerWarning} generate compiler diagnostic of the
indicated severity if the function containing these calls may be
called when the program is executed and the function call is not
eliminated by parameter folding.

The compiler diagnostic is defined by the actual arguments which must
be string parameters.  The diagnostic points to the spot in the Chapel
program from which the function containing the call is called.
Compilation halts if a \chpl{compilerError} is encountered whereas it
will continue after encountering a \chpl{compilerWarning}.

Note that when a variable function is called in a context where the
implicit \chpl{setter} argument is true or false, both versions of the
variable function are resolved by the compiler.  Consequently,
the \chpl{setter} argument cannot be effectively used to guard a
compiler diagnostic statements.

\begin{example}
The following code shows an example of using user-defined compiler
diagnostics to generate warnings and errors:
\begin{chapelpre}
% compilerDiagnostics.chpl
\end{chapelpre}
\begin{chapel}
proc foo(x, y) {
  if (x.type != y.type) then
    compilerError("foo() called with non-matching types: ", 
                  typeToString(x.type), " != ", typeToString(y.type));
  writeln("In 2-argument foo...");
}

proc foo(x) {
  compilerWarning("1-argument version of foo called");
  writeln("In generic foo!");
}
\end{chapel}
\begin{chapelpost}
foo(3.4);
foo("hi");
foo(1, 2);
foo(1.2, 3.4);
foo("hi", "bye");
\end{chapelpost}
\begin{chapeloutput}
compilerDiagnostics.chpl:12: warning: 1-argument version of foo called
compilerDiagnostics.chpl:13: warning: 1-argument version of foo called
In generic foo!
In generic foo!
In 2-argument foo...
In 2-argument foo...
In 2-argument foo...
\end{chapeloutput}

The first routine generates a compiler error whenever the compiler
encounters a call to it where the two arguments have different types.
It prints out an error message indicating the types of the arguments.
The second routine generates a compiler warning whenver the compiler
encounters a call to it.

Thus, if the program foo.chpl contained the following calls:

\begin{numberedchapel}
foo(3.4);
foo("hi");
foo(1, 2);
foo(1.2, 3.4);
foo("hi", "bye");
foo(1, 2.3);
foo("hi", 2.3);
\end{numberedchapel}

\noindent compiling the program would generate output like:

\begin{commandline}
foo.chpl:1: warning: 1-argument version of foo called with type: real
foo.chpl:2: warning: 1-argument version of foo called with type: string
foo.chpl:6: error: foo() called with non-matching types: int != real
\end{commandline}

\end{example}

\section{Example: A Generic Stack}
\label{Example_Generic_Stack}
\begin{chapelpre}
% genericStack.chpl
\end{chapelpre}
\begin{chapel}
class MyNode {
  type itemType;              // type of item
  var item: itemType;         // item in node
  var next: MyNode(itemType); // reference to next node (same type)
}

record Stack {
  type itemType;             // type of items
  var top: MyNode(itemType); // top node on stack linked list

  proc push(item: itemType) {
    top = new MyNode(itemType, item, top);
  }

  proc pop() {
    if isEmpty then
      halt("attempt to pop an item off an empty stack");
    var oldTop = top;
    top = top.next;
    return oldTop.item;
  }

  proc isEmpty return top == nil;
}
\end{chapel}
\begin{chapelpost}
var s: Stack(int);
s.push(1);
s.push(2);
s.push(3);
while !s.isEmpty do
  writeln(s.pop());
\end{chapelpost}
\begin{chapeloutput}
3
2
1
\end{chapeloutput}

\cleardoublepage
\sekshun{Parallelism and Synchronization}
\label{Parallelism_and_Synchronization}

Chapel is an explicitly parallel programming language.  Parallelism is
introduced into a program by the user with the following three
constructs: \chpl{forall}, \chpl{cobegin}, and \chpl{begin}.  In
addition, some operations on arrays and domains, as well as
invocations of promotion, are executed in parallel.  Synchronization
is provided with \emph{synchronization variables} and \emph{atomic}
statements.  To avoid any unintended implications, the
terms \emph{computation} and \emph{sub-computation} will be used to
refer to distinct, concurrently executing portions of the program.

\subsection{The Forall Loop}
\label{Forall}
\index{forall@\chpl{forall}}
\index{forall loops}

The forall loop is a variant of the for loop that allows for the
concurrent execution of the loop body. The for loop is described
in~\rsec{The_For_Loop}. The syntax for the forall loop is given by
\begin{syntax}
forall-statement:
   `forall' loop-control-part loop-body-part
\end{syntax}

The forall loop evaluates the loop body once for each element in
the \sntx{iterator-expression}.  Each instance of the forall loop's
statement may be executed concurrently with each other, but this is
not guaranteed.  The compiler and runtime determine the actual
concurrency based on the specification of the iterator of the loop.
The keyword \chpl{ordered}, described in~\rsec{Ordered_Forall}, can be
used to constrain the parallelism to give a partial order on the
iterator.

Control continues with the statement following the forall loop only
after each iteration has been completely evaluated.  Control transfers
out of a loop body via \chpl{break}, \chpl{continue},
and \chpl{return} are not permitted.  Control can be transferred out
of the loop via a \chpl{yield} statement.

\begin{example}
In the code
\begin{chapel}
forall i in 1..N do
  a(i) = b(i);
\end{chapel}
the user has stated that the element-wise assignments can execute
concurrently.  This loop may be performed serially, with maximum
concurrency where each loop body iteration instance is executed in a
separate computation, or somewhere in between.
\end{example}

\begin{status}
The forall loop is currently executed serially.
\end{status}

\subsubsection{Alternative Forall Loop Syntax}
\label{Alternative_Forall_Loop_Syntax}
\index{forall loops!alternative syntax}

The forall loop may be alternatively specified with a more concise
syntax given by:
\begin{syntax}
alternative-forall-statement:
  [loop-control-part] statement
\end{syntax}
The semantics are unchanged.

\begin{example}
The previous \chpl{forall} example can be alternatively written as:
\begin{chapel}
[i in 1..N] a(i) = b(i);
\end{chapel}
\end{example}

\subsubsection{The Ordered Forall Loop}
\label{Ordered_Forall}
\index{forall loops!ordered}
\index{ordered@\chpl{ordered}}

By default a forall loop allows complete concurrent evaluation of the
iterator expression and among the loop instances. The
keyword \chpl{ordered} can be used to constrain the general
parallelism among instances of the loop to that expressed by an
iterator. This allows an iterator to both define an array of values
and to impose a partial order on that iterator.  This has the same
semantics as with the ordered expression which is explained
in \rsec{Ordered_Expressions}.  The syntax is:
\begin{syntax}
ordered-forall-statement:
   `ordered' `forall' loop-control-part loop-body-part
\end{syntax}

\begin{example}
In the code
\begin{chapel}
ordered forall i in walk(root) do
  work(i);

def walk(n: node) {
  yield n;
  forall c in 0..n.numOfChildren {
      yield n.child[c];
  }
}
\end{chapel}
there is a constraint on the parallel execution such that the
function \chpl{work} is evaluated on a node before any of its
immediate children nodes.  The work on sibling nodes can be executed
concurrently.
\end{example}

\begin{status}
The ordered forall loop is currently executed serially.
\end{status}

\subsection{The Forall Expression}
\label{Forall_Expressions}
\index{forall expressions}

A forall expression can be used to enable concurrent evaluation of
sub-expressions.  The sub-expressions are evaluated once for each
element in the iterator expression.  The syntax of a forall expression
is given by
\begin{syntax}
forall-expression:
   `forall' loop-control-part `do' expression
   [loop-control-part] expression
\end{syntax}

A forall expression is semantically equivalent to an iterator that
yields the expressions.

\begin{example}
The code
\begin{chapel}
[i in S] f(i);
\end{chapel}
is equivalent to
\begin{chapel}
def ff() {
  for i in S do
    yield f(i);
}
ff();
\end{chapel}
\end{example}

\begin{status}
Forall expressions are evaluated serially.
\end{status}

\subsubsection{Filtering Predicates in Forall Expressions}
\label{Filtering_Predicates_Forall}
\index{forall expressions!and conditional expressions}

An if expression that is immediately enclosed by a forall expression
does not require an else part.
\begin{example}
The following expression returns every other element starting with the
first:
\begin{chapel}
[i in 1..s.length] if i % 2 == 1 then s(i)
\end{chapel}
\end{example}

\subsection{The Cobegin Statement}
\label{Cobegin}
\index{cobegin@\chpl{cobegin}}

The cobegin statement is used to create parallelism among statements
within a block statement. The \chpl{cobegin} statement syntax is
\begin{syntax}
cobegin-statement:
  `cobegin' block-statement
\end{syntax}
Each statement within the block statement is executed concurrently and
is considered a separate computation.  Control continues after all of
the statements within the block statement have been evaluated.

As with the forall loop, control transfers are not permitted
either into or out of the cobegin's block statement. Similarly,
yield statements are allowed.

Variables declared in the cobegin statement are {\em single variables},
described in~\rsec{Single_Variables}.

\subsection{The Coforall Loop}
\label{Coforall}
\index{coforall@\chpl{coforall}}
\index{coforall loops}

The coforall loop is a variant of the cobegin statement and the forall
loop.  The syntax for the coforall loop is given by
\begin{syntax}
coforall-statement:
   `coforall' loop-control-part loop-body-part
\end{syntax}

The semantics of the \chpl{coforall} loop are identical to
the \chpl{forall} loop except that each iteration is guaranteed to run
concurrently.  It thus has potentially higher overhead than a forall
loop, but in cases where concurrency is required for correctness, it
is essential.

The semantics of the \chpl{coforall} loop are also identical to
a \chpl{cobegin} statement where each iteration of the \chpl{coforall}
loop is equivalent to a separate statement in a \chpl{cobegin} block.

Control continues with the statement following the \chpl{coforall}
loop only after each iteration has been completely evaluated.  Control
transfers out of a loop body via \chpl{break}, \chpl{continue},
and \chpl{return} are not permitted.  Control can be transferred out
of the loop via a \chpl{yield} statement.

\subsection{The Begin Statement}
\label{Begin}
\index{begin@\chpl{begin}}

The begin statement spawns a computation to execute a statement.
Control continues simultaneously with the statement following the
begin statement. The begin statement is an unstructured way to create
a new computation that is executed only for its side-effects. The
syntax for the begin statement is given by
\begin{syntax}
begin-statement:
  `begin' statement
\end{syntax}

The following statements cannot be contained in begin-statements:
break-statements, continue-statements, yield-statements, and
return-statements.

\subsection{The Ordered Expression}
\label{Ordered_Expressions}
\index{ordered@\chpl{ordered}}

\begin{status}
The ordered expression is not yet implemented.
\end{status}

The \chpl{ordered} keyword can be used as an unary operator to suppress
parallel execution among instances of an expression that can involve
side-effects to memory.  The \chpl{ordered} keyword does not inhibit
parallelism within the sub-expression.  The syntax is:
\begin{syntax}
ordered-expression:
   `ordered' expression
\end{syntax}

\begin{example}
In the code
\begin{chapel}
ordered [i in S] f(i) 
\end{chapel}
\chpl{f} is a function and \chpl{S} is an iterator expression. Each
instance of \chpl{f(i)} is executed once for each value in \chpl{S}
and in serial order. The \chpl{ordered} constraint does not propagate
to inhibit parallelism within \chpl{f}.
\end{example}

\subsection{The Serial Statement}
\label{Serial}
\index{serial@\chpl{serial}}

The \chpl{serial} statement can be used to dynamically control the
degree of parallelism.  The syntax is:
\begin{syntax}
serial-statement:
  `serial' expression block-level-statement
\end{syntax}
where the expression evaluates to a bool type.  Independent of that
value, the \sntx{block-level-statement} is evaluated. If the
expression is true, any dynamically encountered forall loop or cobegin
statement is executed serially within the current computation.  Any
dynamically encountered begin-statement is executed serially with the
current computation; no new computation is spawned.  Control continues
to the statement following the begin-statement after the
begin-statement finishes.

\begin{example}
In the code
\begin{chapel}
ordered forall i in walk(root) do
  work(i);

def walk(n: node) {
  yield n;
  serial n.depth > 4 forall c in 0..n.numOfChildren {
      yield n.child[c];
  }
}
\end{chapel}
the serial statement inhibits concurrent execution on the tree for
nodes that are deeper than four levels in the tree.
\end{example}

There is an expectation that functions that may be executed in a
serial context are cloned to avoid the overhead of testing and
suppressing parallelism.

\subsection{Synchronization Variables}
\label{Synchronization_Variables}
\index{synchronization variables}

{\em Synchronization variables} are used to coordinate computations
that share data.  The use of and assignment to these variables
implicitly controls the execution order of the computation.  There are
two kinds of synchronization variables, {\em single} and {\em sync}
variables.  A single variable can only be assigned once during its
lifetime.  A sync variable can be assigned multiple times during its
lifetime.

The normal use of and assignment to a synchronization variable is well
suited for producer-consumer data sharing.  Additional functions on
synchronization variable are provided such that other traditional
synchronization primitives, such as semaphores and mutexes, can be
constructed.

\subsubsection{Single Variables}
\label{Single_Variables}
\index{synchronization variables!single@\chpl{single}}
\index{single@\chpl{single}}

A single (assignment) variable can only be assigned once during its
lifetime.  A use of a single variable before it is assigned causes the
computation's execution to be suspended until the variable is
assigned. Otherwise, the use proceeds as with normal variables and the
computation continues.  After a single assignment variable is assigned,
all computations with pending uses resume in an unspecified order.  A
single variable is specified with a single type given by the following
syntax:
\begin{syntax}
 single-type:
   `single' type
\end{syntax}

\begin{example}
In the code
\begin{chapel}
class Tree {
  var is_leaf : bool;
  var left    : Tree;
  var right   : Tree;
  var value   : int;

  def sum() {
    if (is_leaf) then 
       return value;

    var x : single int;
    begin x = left.sum();
    var y = right.sum();
    return x+y;
  }
}
\end{chapel}
the single variable \chpl{x} is assigned by an asynchronous
computation created with the begin statement. The computation
returning the sum waits on the reading of \chpl{x} until it has been
assigned.

While a \chpl{cobegin} might be a more suitable formulation, this
fragment creates an asynchronous computation to compute the sum of the
left sub-tree while the main computation continues with the right
sub-tree. The final reference to variable x will be delayed until the
assignment to x completes and that value will be used as a summand.
\end{example}

When a single variable has an initializer, the evaluation of
that initializer is implicitly performed as an asynchronous computation. 
\begin{example}
The code
\begin{chapel}
var x: single int = left.sum;
\end{chapel}
is equivalent to
\begin{chapel}
var x: single int;
x = left.sum;
\end{chapel}
\end{example}

\index{synchronization variables!implicit in cobegin@implicit in \chpl{cobegin}}
Any variable declaration within a cobegin statement is implicitly
treated as a single variable for references in other statements of the
cobegin statement.
\begin{example}
In the code
\begin{chapel}
def sum() {
  if (is_leaf) then 
    return value;
  var z;
  cobegin {
    var x = left.sum();
    var y = right.sum();
    z = x+y;
  }
  return z;
}
\end{chapel}
the computation with assignment to \chpl{z} waits for the other
computations to assign to \chpl{x} and \chpl{y} before it
references \chpl{x} and \chpl{y} in order to assign to \chpl{z}.  The
variables \chpl{x} and \chpl{y} are implicitly single.
\end{example}

\subsubsection{Sync Variables}
\label{Sync_Variables}
\index{synchronization variables!sync@\chpl{sync}}
\index{sync@\chpl{sync}}

A sync variable generalizes the single assignment variable to permit
multiple assignments to the variable. A sync variable is logically
either {\em full} or {\em empty}. When it is empty, computations that
attempt to read that variable are suspended until the variable becomes
full by the next assignment to it, which atomically changes the state
to full. When the variable is full, a read of that variable consumes
the value and atomically transitions the state to empty. If there is
more than one computation waiting on a sync variable, one is
non-deterministically selected to use the variable and resume
execution.  The other computations continue to wait for the next
assignment.

If a computation attempts to assign to a sync variable that is full,
the computation is suspended and the assignment is delayed. When the
sync variable becomes empty, the computation is resumed and the
assignment proceeds, transitioning the state back to full. If there
are multiple computations attempting such an assignment, one is
non-deterministically selected to proceed and the other assignments
continue to wait until the sync variable is emptied again.

A sync variable is specified with a sync type given by the following
syntax:
\begin{syntax}
sync-type:
  `sync'
\end{syntax}

\subsubsection{Additional Synchronization Variable Functions}
\label{Functions_on_Synchronization_Variables}
\index{synchronization variables!built-in functions on}
\index{readFE@\chpl{readFE}}
\index{readFF@\chpl{readFF}}
\index{readXX@\chpl{readXX}}
\index{writeEF@\chpl{writeEF}}
\index{writeFF@\chpl{writeFF}}
\index{writeXF@\chpl{writeXF}}
\index{reset@\chpl{reset}}
\index{isFull@\chpl{isFull}}

Synchronization variables support additional methods that
can be used to bypass their semantics to provide new ones.
Let \chpl{sync_class} be a (compiler-generated) class
for a sync variable \chpl{s} of arbitrary type \chpl{t}.
The following methods are defined in this class:

\begin{protohead}
def $sync_class$.readFE(): t
\end{protohead}
\begin{protobody}
Wait for full, leave empty, and return \chpl{s}'s value.
\end{protobody}

\begin{protohead}
def $sync_class$.readFF(): t
\end{protohead}
\begin{protobody}
Wait for full, leave full, and return \chpl{s}'s value.
\end{protobody}

\begin{protohead}
def $sync_class$.readXX(): t
\end{protohead}
\begin{protobody}
No wait, leave F/E unchanged, and return \chpl{s}'s value.
\end{protobody}

\begin{protohead}
def $sync_class$.writeEF(v: t)
\end{protohead}
\begin{protobody}
Wait for empty, assign \chpl{s=v}, and leave full.
\end{protobody}

\begin{protohead}
def $sync_class$.writeFF(v: t)
\end{protohead}
\begin{protobody}
Wait for full, assign \chpl{s=v}, and leave full.
\end{protobody}

\begin{protohead}
def $sync_class$.writeXF(v: t)
\end{protohead}
\begin{protobody}
No wait, assign \chpl{s=v}, and leave full.
\end{protobody}

\begin{protohead}
def $sync_class$.reset()
\end{protohead}
\begin{protobody}
No wait, assign to \chpl{s} \chpl{t}'s default value, and leave empty.
\end{protobody}

\begin{protohead}
def $sync_class$.isFull: bool
\end{protohead}
\begin{protobody}
No wait, returns \chpl{true} if \chpl{s} is full, \chpl{false} otherwise.
\end{protobody}

For single variables \chpl{s},
only \chpl{readFF} and \chpl{writeEF} are defined.

\begin{example}
Given the following declarations
\begin{chapel}
var x: single int;
var y: int;
\end{chapel}
the code
\begin{chapel}
x = 5;
y = x;
\end{chapel}
is equivalent to
\begin{chapel}
x.writeEF(5);
y = x.readFF();
\end{chapel}
\end{example}

\begin{rationale}
Although the \chpl{readFE}, \chpl{readFF}, and \chpl{writeEF} methods
are implicitly called
when referencing or assigning to sync or single variables,
making these methods available supports programmers who wish to
make the semantics of these operations more explicit.
It might be desirable to have a compiler option
that disables the implicit use of these methods.
\end{rationale}

\subsubsection{Synchronization Variables of Record and Class Types}
\label{Synchronization_Variables_of_Record_Type}
\index{synchronization variables!of record type}
\index{synchronization variables!of class type}

A variable of record or class type can be a single or sync
variable. The semantics of single and sync variables are applied only
to the variable and not to accesses of individual fields.  A record or
class type may have synchronization variable fields to get
synchronization semantics on individual field accesses.

\subsubsection{Synchronization Formal Arguments}
\label{Synchronization_Formal_Arguments}
\index{synchronization variables!formal arguments}

If an argument is a sync or single type, the actual is passed by
reference and the argument itself is a valid lvalue.

The unqualified types \chpl{sync} and \chpl{single} can also be used
to specify a generic formal argument.  In this case, the actual must
be a synchronization variable and it is passed by reference.  For
generic formal arguments of any type, an actual that is \chpl{sync}
or \chpl{single} is ``read'' before being passed to the function and
the generic formal argument is not a \chpl{sync} or \chpl{single}
type.

\subsection{Memory Consistency Model}
\label{Memory_Consistency}
\index{memory consistency model}

This section is forthcoming.

\subsection{Atomic Statement}
\label{Atomic_Transactions}
\index{atomic transactions}
\index{atomic@\chpl{atomic}}

\begin{status}
Atomic statements are not yet implemented.
\end{status}

The atomic statement creates an atomic transaction of a statement. The
statement is executed with transaction semantics in that the statement
executes entirely, the statement appears to have completed in a single
order and serially with respect to other atomic statements, and no
variable assignment is visible until the statement has completely
executed.

This definition of an atomic statement is sometimes called {\em strong
atomicity} because the semantics are atomic to the entire program.
{\em Weak atomicity} is defined so that an atomic statement is atomic
only with respect to other atomic statements.  If the performance
implications of strong atomicity are not tolerable, the semantics of
atomic transactions may be revisited, and could become weaker.

The syntax for the atomic statement is given by:
\begin{syntax}
atomic-statement:
  `atomic' statement
\end{syntax}

\begin{example}
The following code illustrates one possible use of atomic statements:
\begin{chapel}
var found = false;
atomic {
  if head == obj {
    found = true;
    head = obj.next;
  } else  {
    var last = head;
    while last != null {
    if last.next == obj {
      found = true;
      last.next = object.next;
      break;
    }
    last = last.next;
  }
}
\end{chapel}
Inside the atomic statement is a sequential implementation of
removing a particular object denoted by \chpl{obj} from a singly
linked list.  This is an operation that is well-defined, assuming only
one computation is attempting it at a time. The atomic statement
ensures that, for example, the value of \chpl{head} does not change
after it is first in the first comparison and subsequently read to
initialize \chpl{last}. The variables eventually owned by this
computation are \chpl{found}, \chpl{head}, \chpl{obj}, and the various
\chpl{next} fields on examined objects.
\end{example}

The effect of an atomic statement is dynamic.

\begin{example}
If there is a method associated with a list that removes an object,
that method may not be parallel safe, but could be invoked safely inside an
atomic statement:
\begin{chapel}
atomic found = head.remove(obj);
\end{chapel}
\end{example}

\cleardoublepage
\sekshun{Locality and Distribution}
\label{Locality_and_Distribution}

\begin{implementation}
Programs can currently only run on a single locale.  The abstractions
described here are not yet implemented.
\end{implementation}

Chapel provides high-level abstractions that allow programmers to
exploit locality by defining the affinity of data and computation.
This is accomplished by associating both data objects and computations
with abstract {\em locales}. To provide a higher-level mechanism,
Chapel allows a mapping from domains to locales to be specified. This
mapping is called a {\em distribution} and it guides that placement of
variables associated with arrays and the placement of subcomputations
defined over the domain.

\index{local}
\index{remote}
Throughout this section, the term {\em local} refers to data that is
associated with the locale that a computation is running on and {\em
remote} refers to data that is not. We assume that there is some
overhead associated with accessing data that may be remote compared to
data known to be local.

\subsection{Locales}
\label{Locales}
\index{locales}

A locale abstracts a processor or node in a parallel computer system,
or the basic component in the computer system where local memory can
be accessed uniformly.

\subsubsection{The Locale Type}
\label{The_Locale_Type}
\index{locale@\chpl{locale}}

The identifier \chpl{locale} is a primitive type that abstracts a
locale as described above.  Both data and computations can be
associated with a value of locale type. The only operators defined
over locales are the equality and inequality comparison operators.

\subsubsection{Predefined Locales Array}
\label{Predefined_Locales_Array}
\index{Global@\chpl{Global}}
\index{Locales@\chpl{Locales}}
\index{numLocales@\chpl{numLocales}}
\index{execution environment}

A predefined configuration variable defines the {\em execution
environment} for a program.  This environment is defined by the
following definitions:
\begin{chapel}
config const numLocales: int;
const Locales: [1..numLocales] locale;
const Global: locale;
\end{chapel}
The environment consists of constants which are fixed when the program
begins execution.  The variable \chpl{Global} holds a special value
of \chpl{locale} type that can be distinct from the values stored
in \chpl{Locales}. This value is used to denote an object or
computation that has no defined affinity.

When a program starts, a single thread is executing.  It is running on
the locale given by \chpl{Locales(1)}.

\subsubsection{Querying the Locale of a Variable}
\label{Querying_the_Locale_of_a_Variable}
\index{locale@\chpl{locale}}

Every variable \chpl{v} is associated with some locale which can be
queried using the following syntax:
\begin{syntax}
locale-access:
  expression . `locale'
\end{syntax}
When the \sntx{expression} is a class type, the locale is where the
object is located rather than where the \sntx{expression} may be
located.

\subsection{Specifying Locales for Computation}
\label{Specifying_Locales_for_Computation}

When execution is proceeding on some locale, a computation can be
associated with a different locale in two ways: via distributions as
discussed in~\rsec{Distributions} or with an \sntx{on-statement} as
discussed below.

\subsubsection{On}
\label{On}
\index{on@\chpl{on}}

The on statement controls on which locale a computation or data should
be placed.  The syntax of the on statement is given by
\begin{syntax}
on-statement:
  `on' expression `do' statement
  `on' expression block-level-statement
\end{syntax}
If the \sntx{expression} is a value of \chpl{locale} type,
the \sntx{statement} or \sntx{block-level-statement} is executed on
the locale specified directly by the expression.  Otherwise, the
expression must be a variable and the locale is taken to be the locale
where the variable is located.  Execution continues after
the \chpl{on-statement} after execution of the \sntx{statement}
or \sntx{block-level-statement} completes.

\index{Global@\chpl{Global}}
If the locale that the \sntx{expression} refers to is equal
to \chpl{Global}, then the locale is unspecified and is determined by
the runtime and/or compiler.

\begin{example}
A common idiom is to use \chpl{on} in conjunction with \chpl{forall}
to access an array decomposed over multiple locales.  The code
\begin{chapel}
forall i in D do on A(i) {
  // some computation
}
\end{chapel}
executes each iteration of the forall loop on the locale where the
element of \chpl{A(i)} is located.
\end{example}

By default, when new variables and data objects are created, they are
created in the locale where the computation is running.  This locale
can be changed by using the \chpl{on} keyword.  Variables can be
defined within an \sntx{on-statement} to define them on a particular
locale.

\subsubsection{On and Iterators}
\label{On_and_Iterators}
\index{iterators!on@\chpl{on}}

When a loop iterates over a sequence specified by an iterator,
on-statements inside the iterator control where the corresponding loop
body is executed.

\begin{example}
An iterator over a distributed tree might include an iterator over the
nodes as defined in the following code:
\begin{chapel}
class Tree {
  var left, right: Tree;
  iterator nodes {
    on this yield this;
    if left then
      forall t in left.nodes do
        yield t;
    if right then
      forall t in right.nodes do
        yield t;
  }
}
\end{chapel}
Given this code and a binary tree of type \chpl{Tree} stored in
variable \chpl{tree}, then we can use the nodes iterator to iterate
over the tree with the following code:
\begin{chapel}
forall t in tree.nodes {
  // body executed on t as specified in nodes
}
\end{chapel}
Here, each instance of the body of the \chpl{forall} loop is executed
on the locale where the corresponding object \chpl{t} is located.
This is specified in the \chpl{nodes} iterator where the \chpl{on}
keyword is used.  In the case of zipper or tensor product iteration,
the location of execution is taken from the first iterator.  This can
be overridden by explictly using \chpl{on} in the body of the loop or
by reordering the product of iteration.
\end{example}

\subsection{Distributions}
\label{Distributions}
\index{distributions}

A mapping from domain index values to locales is called a {\em
distribution}.

\subsubsection{Distributed Domains}
\label{Distributed_Domains}

\index{domains!distributed}
A domain for which a distribution is specified is referred to as a
{\em distributed domain}.  A domain supports a method, \chpl{locale},
that maps index values in the domain to locales that correspond to the
domain's distribution.

Iteration over a distributed domain implicitly executes the control
computation in the domain of the associated locale.  Similarly, when
iterating over the elements of an array defined over a distributed
domain, the controlled computations are determined by the distribution
of the domain.  If there are conflicting distributions in product
iterations, the locale of the computation is taken to be the first
component in the product.

\begin{example}
If \chpl{D} is a distributed domain, then in the code
\begin{chapel}
forall d in D {
  // body
}
\end{chapel}
the body of the loop is executed in the locale where the
index \chpl{d} maps to by the distribution of \chpl{D}.
\end{example}

\subsubsection{Distributed Arrays}
\label{Distributed_Arrays}
\index{arrays!distributed}

Arrays defined over a distributed domain will have the element
variables stored on the locale determined by the distribution.  Thus,
if \chpl{d} is an index of distributed domain \chpl{D} and \chpl{A} is
an array defined over that domain, then \chpl{A(d).locale} is the
locale to which \chpl{d} maps to according to \chpl{D}.

\subsubsection{Undistributed Domains and Arrays}
\label{Undistributed_Domains_and_Arrays}

If a domain or an array does not have a distributed part, the domain
or array is not distributed and exists only on the locale on which it
is defined.

\subsection{Standard Distributions}
\label{Standard_Distributions}

Standard distributions include the following:
\begin{itemize}
\item The block distribution \chpl{Block}
\item The cyclic distribution \chpl{Cyclic}
\item The block-cyclic distribution \chpl{BlockCyclic}
\item The cut distribution \chpl{Cut}
\end{itemize}

A design goal is that all standard distributions are defined with the
same mechanisms that user-defined
distributions~(\rsec{User_Defined_Distributions}) are defined with.

\subsection{User-Defined Distributions}
\label{User_Defined_Distributions}

This section is forthcoming.

\cleardoublepage
This is a stub.  This portion of the document does not exist.

\cleardoublepage
\sekshun{Input and Output}
\label{Input_and_Output}

Chapel provides a built-in \chpl{file} class to handle input and
output to files using functions and methods
called \chpl{read}, \chpl{readln},
\chpl{write}, and \chpl{writeln}.

\section{The {\em file} type}
\index{file type}

The file class contains the following fields:
\begin{itemize}
\item
The \chpl{filename} field is a \chpl{string} that contains the name of
the file.
\item
The \chpl{mode} field is a \chpl{FileAccessMode} enum value that indicates
whether the file is being read or written.
\item
The \chpl{path} field is a \chpl{string} that contains the path of the
file.
\item
The \chpl{style} field can be set to \chpl{text} or \chpl{binary} to
specify that reading from or writing to the file should be done with
text or binary formats.
\end{itemize}
These fields can be modified any time that the file is closed.

The \chpl{mode} field supports the following \chpl{FileAccessMode} values:
\begin{itemize}
\item
\chpl{FileAccessMode.read} The file can be read.
\item
\chpl{FileAccessMode.write} The file can be written.
\end{itemize}

The file type supports the following methods:
\index{file type!methods}
\begin{itemize}
\item
The \chpl{open()} method opens the file for reading and/or writing.
\item
The \chpl{close()} method closes the file for reading and/or writing.
\item
The \chpl{isOpen} method returns true if the file is open for reading
and/or writing, and otherwise returns false.
\item
The \chpl{eof} method returns true if the file is at its end-of-file
position and returns false otherwise.
\item
The \chpl{flush()} method flushes the file, finishing outstanding
reading and writing.
\end{itemize}

Additionally, the file type supports the
methods \chpl{read}, \chpl{readln}, \chpl{write}, and \chpl{writeln} for 
input and output as discussed in~\rsec{filewrite} and~\rsec{fileread}.

\section{Standard files {\em stdout}, {\em stdin}, and {\em stderr}}
\index{file type!standard files stdin, stdout, stderr}
\index{stdin@\chpl{stdin}}
\index{stdout@\chpl{stdout}}
\index{stderr@\chpl{stderr}}

The files \chpl{stdout}, \chpl{stdin}, and \chpl{stderr} are
predefined and map to standard output, standard input, and standard
error as implemented in a platform dependent fashion.

\section{The {\em write}, {\em writeln}, {\em read}, and {\em readln} 
functions}
\index{writeln@\chpl{writeln}}
\index{write@\chpl{write}}
\index{read@\chpl{read}}
\index{readln@\chpl{readln}}
\index{read}
\index{write}

The built-in function \chpl{write} takes an arbitrary number of
arguments and prints each out in turn to \chpl{stdout}.  The built-in
function \chpl{writeln} is identical to \chpl{write} except that it
outputs an additional {\em end-of-line} character after writing out
the argument expressions.  Both of these functions will generate their
output atomically with respect to other calls to these functions from
other tasks.

The built-in function \chpl{read} takes an arbitrary number of
variable expressions and reads into each in turn from \chpl{stdin}.
Any whitespace is skipped over and is used only to separate one
argument from the next.  The built-in function \chpl{readln} is
identical except that upon reading all of its arguments it scans ahead
in the input stream until just after the next {\em end-of-line}
character.

The \chpl{read} and \chpl{readln} functions are also defined to take
an arbitrary number of types as arguments.  In this case, the
functions read an expression of each argument type.  In the event that
a single type is specified, the return value is the value that was
read; if multiple types are specified, a tuple of the values is
returned.

These functions are wrappers for the methods on files described next.

\begin{example}
The \chpl{writeln} wrapper function allows for a simple implementation
of the {\em Hello-World} program:
\begin{chapel}
writeln("Hello, World!");
\end{chapel}
\end{example}

\begin{example}
The following code shows three ways to read values into a pair of
variables \chpl{x} and \chpl{y}:
\begin{chapel}
var x: int;
var y: real;

/* reading into variable expressions */
read(x, y);

/* reading via a single type argument */
x = read(int);
y = read(real);

/* reading via multiple type arguments */
(x, y) = read(int, real);
\end{chapel}
\end{example}

\section{User-Defined {\em writeThis} methods}

To define the output for a given type, the user must define a method
called \chpl{writeThis} on that type that takes a single argument of
\chpl{Writer} type.  If such a method does not exist, a default method is
created.

\section{The {\em write} and {\em writeln} method on files}
\label{filewrite}
\index{write!on files}

The \chpl{file} type supports methods \chpl{write} and \chpl{writeln}
for output.  These methods are defined to take an arbitrary number of
arguments.  Each argument is written in turn by calling
the \chpl{writeThis} method on that argument.
Default \chpl{writeThis} methods are bound to any type that the user
does not explicitly create one for.

A lock is used to ensure that output is serialized across multiple
tasks.

\subsection{The {\em write} and {\em writeln} method on strings}
\label{stringwrite}
\index{write!on strings}

The \chpl{write} and \chpl{writeln} methods can also be called on
strings to write the output to a string instead of a file.

\subsection{Generalized {\em write} and {\em writeln}}
\label{writer}
\index{Writer@\chpl{Writer}}

The \chpl{Writer} class contains no arguments and serves as a base
class to allow user-defined classes to be written to.  If a class is
defined to be a subclass of Writer, it must override
the \chpl{writeIt} method that takes a \chpl{string} as an argument.

\begin{example}
The following code defines a subclass of \chpl{Writer} that overrides
the \chpl{writeIt} method to allow it to be written to.  It also
overrides the \chpl{writeThis} method to override the default way that
it is written.
\begin{chapel}
class C: Writer {
  var data: string;
  proc writeIt(s: string) {
    data += s.substring(1);
  }
  proc writeThis(x: Writer) {
    x.write(data);
  }
}

var c = new C();
c.write(41, 32, 23, 14);
writeln(c);
\end{chapel}
The \chpl{C} class filters the arguments sent to it, printing out only
the first letter.  The output to the above is thus \chpl{4321}.
\end{example}


\section{The {\em read} and {\em readln} methods on files}
\label{fileread}
\index{read!on files}

The \chpl{file} type supports \chpl{read} and \chpl{readln} methods.
The \chpl{read} method takes an arbitrary number of arguments, reading
in each argument from file.  The \chpl{readln} method also
takes an arbitrary number of arguments, reading in each argument
from a single line or multiple lines in the file and 
advancing the file pointer to the next line after the last argument 
is read.

The \chpl{file} type also supports overloaded methods \chpl{read}
and \chpl{readln} that take an arbitrary number of types as arguments.
These methods read values of the specified types from the file and
return them in a tuple.  If only one type is read, the value is not
returned in a tuple, but is returned directly.

\begin{example}
The following line of code reads a value of type \chpl{int} from
\chpl{stdin} and uses it to initialize variable \chpl{x} (causing
\chpl{x} to have an inferred type of \chpl{int}):
\begin{chapel}
var x = stdin.read(int);
\end{chapel}
\end{example}


\section{Default {\em read} and {\em write} methods}
\index{write!default methods}
\index{read!default methods}

Default \chpl{write} methods are created for all types for which a user
\chpl{write} method is not defined.  They have the following semantics:
\begin{itemize}
\item
{\bf arrays} Outputs the elements of the array in row-major order
where rows are separated by line-feeds and blank lines are used to
separate other dimensions.
\item
{\bf domains} Outputs the dimensions of the domain enclosed
by \chpl{[} and \chpl{]}.
\item
{\bf ranges} Outputs the lower bound of the range followed
by \chpl{..} followed by the upper bound of the range.  If the stride
of the range is not one, the output is additionally followed by the
word \chpl{by} followed by the stride of the range.
\item
{\bf tuples} Outputs the components of the tuple in order delimited
by \chpl{(} and \chpl{)}, and separated by commas.
\item
{\bf classes} Outputs the values within the fields of the class
prefixed by the name of the field and the character \chpl{=}.  Each
field is separated by a comma.  The output is delimited by \chpl{\{}
and \chpl{\}}.
\item
{\bf records} Outputs the values within the fields of the class
prefixed by the name of the field and the character \chpl{=}.  Each
field is separated by a comma.  The output is delimited by \chpl{(}
and \chpl{)}.
\end{itemize}

Default \chpl{read} methods are created for all types for which a user
\chpl{read} method is not defined.  The default \chpl{read} methods are
defined to read in the output of the default \chpl{write} method.

\cleardoublepage
\sekshun{Standard Modules}
\label{Standard_Modules}
\index{standard modules}

This section describes modules that are automatically used by every
Chapel program as well as a set of standard modules that can be used
manually, provididing standard library support.  The automatic modules
are as follows:

\begin{tabular}{lll}
\hspace{1pc} & \chpl{Math} & Math routines \\
             & \chpl{Standard} & Basic routines \\
             & \chpl{Types} & Routines related to primitive types \\
\end{tabular}

\noindent The standard modules include:

\begin{tabular}{lll}
\hspace{1pc} & \chpl{BitOps} & Bit manipulation routines \\
             & \chpl{Functions}   & Common higher-order functions \\
             & \chpl{Norm}   & Routines for computing vector and matrix norms \\
             & \chpl{Random} & Random number generation routines \\
             & \chpl{Search} & Generic searching routines \\
             & \chpl{Sort} & Generic sorting routines \\
             & \chpl{Time} & Types and routines related to time \\
\end{tabular}

There is an expectation that each of these modules will be extended
and that more standard modules will be defined over time.

\subsection{Automatic Modules}
\index{automatic modules}

Automatic modules are used by a Chapel program automatically.  There
is currently no way to avoid their use by a program though we
anticipate adding such a capability in the future.

\subsubsection{Functions}
\label{Functions}
\index{standard modules!Functions}

The module \chpl{Functions} defines functions (mostly higher-order functions) that 
commonly occur in functional programming.

\vspace{1pc}

\begin{protohead}
array(x...?n)
\end{protohead}
\begin{protobody}
A helper function for creating single dimension arrays by using the given values as the initial values.
\end{protobody}

\begin{protohead}
drop(arr, n)
\end{protohead}
\begin{protobody}
An iterator which returns elements of a sequence, having dropped the first n
\end{protobody}

\begin{protohead}
dropWhile(arr, op)
\end{protohead}
\begin{protobody}
An iterator which returns elements of a sequence, having dropped the first elements which meet the given conditional operator
\end{protobody}

\begin{protohead}
filter(arr, op)
\end{protohead}
\begin{protobody}
An iterator which returns only elements in the sequence which meet the given conditional operator
\end{protobody}

\begin{protohead}
foldLeft(arr, init, op)
\end{protohead}
\begin{protobody}
Reduces a sequence with an operator pair-wise from the left starting with an initial value.
\end{protobody}

\begin{protohead}
foldRight(arr, init, op)
\end{protohead}
\begin{protobody}
Reduces a sequence with an operator pair-wise from the right starting with an initial value.
\end{protobody}

\begin{protohead}
map(arr, op)
\end{protohead}
\begin{protobody}
An iterator which returns the result of an operator applied to each element of a sequence
\end{protobody}

\begin{protohead}
reverse(arr)
\end{protohead}
\begin{protobody}
An iterator which reverses the sequence given.
\end{protobody}

\begin{protohead}
splitAt(arr, n)
\end{protohead}
\begin{protobody}
Returns a tuple splitting a sequence at the position n.
\end{protobody}

\begin{protohead}
take(arr, n)
\end{protohead}
\begin{protobody}
An iterator which returns the first n elements of a sequence
\end{protobody}

\begin{protohead}
takeWhile(arr, op)
\end{protohead}
\begin{protobody}
An iterator which returns the first elements of a sequence which meet the given conditional operator
\end{protobody}

\subsubsection{Math}
\label{Math}
\index{automatic modules!Math}

The module \chpl{Math} defines routines for mathematical computations.
This module is used by default; there is no need to explicitly use
this module.  The Math module defines routines that are derived from
and implemented via the standard C routines defined in \chpl{math.h}.

\vspace{1pc}

\begin{protohead}
def abs(i: int(?w)): int(w)
def abs(i: uint(?w)): uint(w)
def abs(x: real): real
def abs(x: real(32)): real(32)
def abs(x: complex): real
\end{protohead}
\begin{protobody}
Returns the absolute value of the argument.
\end{protobody}

\begin{protohead}
def acos(x: real): real
def acos(x: real(32)): real(32)
\end{protohead}
\begin{protobody}
Returns the arc cosine of the argument.  It is an error if \chpl{x} is
less than $-1$ or greater than $1$.
\end{protobody}

\begin{protohead}
def acosh(x: real): real
def acosh(x: real(32)): real(32)
\end{protohead}
\begin{protobody}
Returns the inverse hyperbolic cosine of the argument.  It is an error
if \chpl{x} is less than $1$.
\end{protobody}

\begin{protohead}
def asin(x: real): real
def asin(x: real(32)): real(32)
\end{protohead}
\begin{protobody}
Returns the arc sine of the argument.  It is an error if \chpl{x} is
less than $-1$ or greater than $1$.
\end{protobody}

\begin{protohead}
def asinh(x: real): real
def asinh(x: real(32)): real(32)
\end{protohead}
\begin{protobody}
Returns the inverse hyperbolic sine of the argument.
\end{protobody}

\begin{protohead}
def atan(x: real): real
def atan(x: real(32)): real(32)
\end{protohead}
\begin{protobody}
Returns the arc tangent of the argument.
\end{protobody}

\begin{protohead}
def atan2(y: real, x: real): real
def atan2(y: real(32), x: real(32)): real(32)
\end{protohead}
\begin{protobody}
Returns the arc tangent of the two arguments.  This is equivalent to
the arc tangent of \chpl{y / x} except that the signs of \chpl{y}
and \chpl{x} are used to determine the quadrant of the result.
\end{protobody}

\begin{protohead}
def atanh(x: real): real
def atanh(x: real(32)): real(32)
\end{protohead}
\begin{protobody}
Returns the inverse hyperbolic tangent of the argument.  It is an error
if \chpl{x} is less than $-1$ or greater than $1$.
\end{protobody}

\begin{protohead}
def cbrt(x: real): real
def cbrt(x: real(32)): real(32)
\end{protohead}
\begin{protobody}
Returns the cube root of the argument.
\end{protobody}

\begin{protohead}
def ceil(x: real): real
def ceil(x: real(32)): real(32)
\end{protohead}
\begin{protobody}
Returns the value of the argument rounded up to the nearest integer.
\end{protobody}

\begin{protohead}
def conjg(a: complex(?w)): complex(w)
\end{protohead}
\begin{protobody}
Returns the conjugate of \chpl{a}.
\end{protobody}

\begin{protohead}
def cos(x: real): real
def cos(x: real(32)): real(32)
\end{protohead}
\begin{protobody}
Returns the cosine of the argument.
\end{protobody}

\begin{protohead}
def cosh(x: real): real
def cosh(x: real(32)): real(32)
\end{protohead}
\begin{protobody}
Returns the hyperbolic cosine of the argument.
\end{protobody}

\begin{protohead}
def erf(x: real): real
def erf(x: real(32)): real(32)
\end{protohead}
\begin{protobody}
Returns the error function of the argument defined as
$$\frac{2}{\sqrt{\pi}}\int^x_0e^{-t^2}dt$$
for the argument $x$.
\end{protobody}

\begin{protohead}
def erfc(x: real): real
def erfc(x: real(32)): real(32)
\end{protohead}
\begin{protobody}
Returns the complementary error function of the argument.  This is
equivalent to \chpl{1.0 - erf(x)}.
\end{protobody}

\begin{protohead}
def exp(x: real): real
def exp(x: real(32)): real(32)
\end{protohead}
\begin{protobody}
Returns the value of $e$ raised to the power of the argument.
\end{protobody}

\begin{protohead}
def exp2(x: real): real
def exp2(x: real(32)): real(32)
\end{protohead}
\begin{protobody}
Returns the value of $2$ raised to the power of the argument.
\end{protobody}

\begin{protohead}
def expm1(x: real): real
def expm1(x: real(32)): real(32)
\end{protohead}
\begin{protobody}
Returns one less than the value of $e$ raised to the power of the argument.
\end{protobody}

\begin{protohead}
def floor(x: real): real
def floor(x: real(32)): real(32)
\end{protohead}
\begin{protobody}
Returns the value of the argument rounded down to the nearest integer.
\end{protobody}

\begin{protohead}
def lgamma(x: real): real
def lgamma(x: real(32)): real(32)
\end{protohead}
\begin{protobody}
Returns the natural logarithm of the absolute value of the gamma
function of the argument.
\end{protobody}

\begin{protohead}
def log(x: real): real
def log(x: real(32)): real(32)
\end{protohead}
\begin{protobody}
Returns the natural logarithm of the argument.  It is an error if the
argument is less than or equal to zero.
\end{protobody}

\begin{protohead}
def log10(x: real): real
def log10(x: real(32)): real(32)
\end{protohead}
\begin{protobody}
Returns the base 10 logarithm of the argument.  It is an error if the
argument is less than or equal to zero.
\end{protobody}

\begin{protohead}
def log1p(x: real): real
def log1p(x: real(32)): real(32)
\end{protohead}
\begin{protobody}
Returns the natural logarithm of \chpl{x+1}.  It is an error
if \chpl{x} is less than or equal to $-1$.
\end{protobody}

\begin{protohead}
def log2(i: int(?w)): int(w)
def log2(i: uint(?w)): uint(w)
def log2(x: real): real
def log2(x: real(32)): real(32)
\end{protohead}
\begin{protobody}
Returns the base 2 logarithm of the argument.  It is an error if the
argument is less than or equal to zero.
\end{protobody}

\begin{protohead}
def nearbyint(x: real): real
def nearbyint(x: real(32)): real(32)
\end{protohead}
\begin{protobody}
Returns the rounded integral value of the argument determined by the
current rounding direction.
\end{protobody}

\begin{protohead}
def rint(x: real): real
def rint(x: real(32)): real(32)
\end{protohead}
\begin{protobody}
Returns the rounded integral value of the argument determined by the
current rounding direction.
\end{protobody}

\begin{protohead}
def round(x: real): real
def round(x: real(32)): real(32)
\end{protohead}
\begin{protobody}
Returns the rounded integral value of the argument.  Cases halfway
between two integral values are rounded towards zero.
\end{protobody}

\begin{protohead}
def sin(x: real): real
def sin(x: real(32)): real(32)
\end{protohead}
\begin{protobody}
Returns the sine of the argument.
\end{protobody}

\begin{protohead}
def sinh(x: real): real
def sinh(x: real(32)): real(32)
\end{protohead}
\begin{protobody}
Returns the hyperbolic sine of the argument.
\end{protobody}

\begin{protohead}
def sqrt(x: real): real
def sqrt(x: real(32)): real(32)
\end{protohead}
\begin{protobody}
Returns the square root of the argument.  It is an error if the
argument is less than zero.
\end{protobody}

\begin{protohead}
def tan(x: real): real
def tan(x: real(32)): real(32)
\end{protohead}
\begin{protobody}
Returns the tangent of the argument.
\end{protobody}

\begin{protohead}
def tanh(x: real): real
def tanh(x: real(32)): real(32)
\end{protohead}
\begin{protobody}
Returns the hyperbolic tangent of the argument.
\end{protobody}

\begin{protohead}
def tgamma(x: real): real
def tgamma(x: real(32)): real(32)
\end{protohead}
\begin{protobody}
Returns the gamma function of the argument defined as
$$\int_0^\infty t^{x-1} e^{-t} dt$$
for the argument $x$.
\end{protobody}

\begin{protohead}
def trunc(x: real): real
def trunc(x: real(32)): real(32)
\end{protohead}
\begin{protobody}
Returns the nearest integral value to the argument that is not larger
than the argument in absolute value.
\end{protobody}


\subsubsection{Standard}
\label{Standard}
\index{automatic modules!Standard}

\begin{protohead}
def ascii(s: string): int
\end{protohead}
\begin{protobody}
Returns the ASCII code number of the first letter in the
argument \chpl{s}.
\end{protobody}

\begin{protohead}
def assert(test: bool) {
\end{protohead}
\begin{protobody}
Exits the program if \chpl{test} is false and prints to standard error
the location in the Chapel code of the call to \chpl{assert}.
If \chpl{test} is true, no action is taken.
\end{protobody}

\begin{protohead}
def assert(test: bool, args ...?numArgs) {
\end{protohead}
\begin{protobody}
Exits the program if \chpl{test} is false and prints to standard error
the location in the Chapel code of the call to \chpl{assert} as well
as the rest of the arguments to the call.  If \chpl{test} is true, no
action is taken.
\end{protobody}

\begin{protohead}
def complex.re: real
\end{protohead}
\begin{protobody}
Returns the real component of the complex number.
\end{protobody}

\begin{protohead}
def complex.im: real
\end{protohead}
\begin{protobody}
Returns the imaginary component of the complex number.
\end{protobody}

\begin{protohead}
def complex.=re(f: real)
\end{protohead}
\begin{protobody}
Sets the real component of the complex number to \chpl{f}.
\end{protobody}

\begin{protohead}
def complex.=im(f: real)
\end{protohead}
\begin{protobody}
Sets the imaginary component of the complex number to \chpl{f}.
\end{protobody}

\begin{protohead}
def exit(status: int)
\end{protohead}
\begin{protobody}
Exits the program with code \chpl{status}.
\end{protobody}

\begin{protohead}
def halt()
\end{protohead}
\begin{protobody}
Exits the program and prints to standard error the location in the
Chapel code of the call to \chpl{halt}.
\end{protobody}

\begin{protohead}
def halt(args ...?numArgs)
\end{protohead}
\begin{protobody}
Exits the program and prints to standard error the location in the
Chapel code of the call to \chpl{halt} as well as the rest of the
arguments to the call.
\end{protobody}

\begin{protohead}
def length(s: string): int
\end{protohead}
\begin{protobody}
Returns the number of characters in the argument \chpl{s}.
\end{protobody}

\begin{protohead}
def max(x, y...?k)
\end{protohead}
\begin{protobody}
Returns the maximum of the arguments when compared using the
``greater-than'' operator.  The return type is inferred from the types
of the arguments as allowed by implicit coercions.
\end{protobody}

\begin{protohead}
def min(x, y...?k)
\end{protohead}
\begin{protobody}
Returns the minimum of the arguments when compared using the
``less-than'' operator.  The return type is inferred from the types of
the arguments as allowed by implicit coercions.
\end{protobody}

\begin{protohead}
def string.substring(x): string
\end{protohead}
\begin{protobody}
Returns a value of string type that is a substring of the base
expression.  If \chpl{x} is $i$, a value of type \chpl{int}, then the
result is the $i$th character.  If \chpl{x} is a range, the result is
the substring where the characters in the substring are given by the
values in the range.
\end{protobody}

\begin{protohead}
def typeToString(type t) param : string
\end{protohead}
\begin{protobody}
Returns a string parameter that represents the name of the
type \chpl{t}.
\end{protobody}

\subsubsection{Types}

\begin{protohead}
def numBits(type t) param : int
\end{protohead}
\begin{protobody}
Returns the number of bits used to store the values of type \chpl{t}.
This is implemented for all numeric types and fixed-width \chpl{bool} types.
It is not implemented for default-width \chpl{bool}.
\end{protobody}


\begin{protohead}
def numBytes(type t) param : int
\end{protohead}
\begin{protobody}
Returns the number of bytes used to store the values of type \chpl{t}.
This is implemented for all numeric types and fixed-width \chpl{bool} types.
It is not implemented for default-width \chpl{bool}.
\end{protobody}

\begin{protohead}
def max(type t): t
\end{protohead}
\begin{protobody}
Returns the maximum value that can be stored in type \chpl{t}.  This
is implemented for all numeric types.
\end{protobody}

\begin{protohead}
def min(type t): t
\end{protohead}
\begin{protobody}
Returns the minimum value that can be stored in type \chpl{t}.  This
is implemented for all numeric types.
\end{protobody}



\subsection{Standard Modules}

Standard modules can be used by a Chapel program via the \chpl{use}
keyword.

\subsubsection{BitOps}
\label{BitOps}
\index{standard modules!BitOps}

The module \chpl{BitOps} defines routines that manipulate the bits of
values of integral types.

\vspace{1pc}

\begin{protohead}
def bitPop(i: integral): int
\end{protohead}
\begin{protobody}
Returns the number of bits set to one in the integral
argument \chpl{i}.
\end{protobody}

\begin{protohead}
def bitMatMultOr(i: uint(64), j: uint(64)): uint(64)
\end{protohead}
\begin{protobody}
Returns the bitwise matrix multiplication of \chpl{i} and \chpl{j}
where the values of \chpl{uint(64)} type are treated as $8 \times 8$
bit matrices and the combinator function is bitwise or.
\end{protobody}

\begin{protohead}
def bitRotLeft(i: integral, shift: integral): i.type
\end{protohead}
\begin{protobody}
Returns the value of the integral argument \chpl{i} after rotating the
bits to the left \chpl{shift} number of times.
\end{protobody}

\begin{protohead}
def bitRotRight(i: integral, shift: integral): i.type
\end{protohead}
\begin{protobody}
Returns the value of the integral argument \chpl{i} after rotating the
bits to the right \chpl{shift} number of times.
\end{protobody}


\subsubsection{Norm}
\label{Norm}
\index{standard modules!Norm}

The module \chpl{Norm} supports the computation of standard vector and
matrix norms on Chapel arrays.  The current interface is minimal and
should be expected to grow and evolve over time.

\begin{protohead}
enum normType {norm1, norm2, normInf, normFrob};
\end{protohead}
\begin{protobody}
An enumerated type indicating the different types of norms supported
by this module: 1-norm, 2-norm, infinity norm and Frobenius norm,
respectively.
\end{protobody}

\begin{protohead}
def norm(x: [], p: normType) where x.rank == 1 || x.rank == 2
\end{protohead}
\begin{protobody}
Compute the norm indicated by \chpl{p} on the 1D or 2D array \chpl{x}.
\end{protobody}

\begin{protohead}
def norm(x: [])
\end{protohead}
\begin{protobody}
Compute the default norm on array \chpl{x}.  For a 1D array this is
the 2-norm, for a 2D array, this is the Frobenius norm.
\end{protobody}

\subsubsection{Random}
\label{Random}
\index{standard modules!Random}

The module \chpl{Random} supports the generation of pseudo-random
values and streams of values.  The current interface is minimal and
should be expected to grow and evolve over time.  In particular, we
expect to support other pseudo-random number generation algorithms,
more random value types (\eg, int), and both serial and parallel
iterators over the RandomStream class.

\begin{protohead}
class RandomStream
def RandomStream(seed: int(64), param parSafe: bool = true)
def RandomStream(seedGenerator: SeedGenerator = SeedGenerator.currentTime,
                 param parSafe: bool = true)
\end{protohead}
\begin{protobody}
Implements a pseudo-random stream of values based on a seed value.
The current implementation generates the values using a linear
congruential generator.  In future versions of this module, the
RandomStream class will offer a wider variety of algorithms for
generating pseudo-random values.

To construct a RandomStream class, the seed may be explicitly passed.
It must be an odd integer between $1$ and $2^{46}-1$.  Alternatively,
the RandomStream class can be constructed by passing a value of the
enumerated type SeedGenerator to choose an algorithm to use to set the
seed.  If neither a seed nor a SeedGenerator value is passed to the
RandomStream class, the seed will be initialized based on the current
time in microseconds (rounded via modular arithmetic to the nearest
odd integer between $1$ and $2^{46}-1$.

The parSafe parameter defaults to true and allows for safe use of this
class by concurrent tasks.  This can be overridden when calling
methods to make them safe when called by concurrent tasks.  This
mechanism allows for lower overhead calls when there is no threat of
concurrent calls, but correct calls when there is.
\end{protobody}

\begin{protohead}
enum SeedGenerator { currentTime };
\end{protohead}
\begin{protobody}
Values of this enumerated type may be used to choose a method for
initializing the seed in the RandomStream class.  The only value
supported at present is \chpl{currentTime} which can be used to
initialize the seed based on the current time in microseconds (rounded
via modular arithmetic to the nearest odd integer between $1$ and
$2^{46}-1$.
\end{protobody}

\begin{protohead}
def RandomStream.fillRandom(x:[?D], param parSafe = this.parSafe)
\end{protohead}
\begin{protobody}
Fill the argument array, \chpl{x}, with the next $|$\chpl{D}$|$ values
of the pseudo-random stream in row-major order.  The array must be an
array of real(64), imag(64), or complex(128) elements.  For complex
arrays, each complex element is initialized with two values from the
stream of random numbers.
\end{protobody}

\begin{protohead}
def RandomStream.skipToNth(in n: integral, param parSafe = this.parSafe)
\end{protohead}
\begin{protobody}
Skips ahead or back to the \chpl{n}-th value in the
random stream.  The value of n is assumed to be positive, such that
\chpl{n}~==~1 represents the initial value in the stream.
\end{protobody}

\begin{protohead}
def RandomStream.getNext(param parSafe = this.parSafe): real
\end{protohead}
\begin{protobody}
Returns the next value in the random stream as a real.
\end{protobody}

\begin{protohead}
def RandomStream.getNth(n: integral, param parSafe = this.parSafe): real
\end{protohead}
\begin{protobody}
Returns the \chpl{n}-th value in the random stream as a real.  Equivalent to
calling \chpl{skipToNth(n)} followed by \chpl{getNext()}.
\end{protobody}

\begin{protohead}
def fillRandom(x:[], initseed: int(64))
\end{protohead}
\begin{protobody}
A routine provided for convenience to support the functionality of the
fillRandom method (above) without explicitly constructing an instance
of the \chpl{RandomStream} class.  This is useful for filling a single
array or multiple arrays which require no coherence between them.
The \chpl{initseed} parameter corresponds to the \chpl{seed} member of
the \chpl{RandomStream} class.  If unspecified, the default for the
class will be used.
\end{protobody}

\subsubsection{Search}
\label{Search}
\index{standard modules!Search}

The \chpl{Search} module is designed to support standard search
routines.  The current interface is minimal and should be expected to
grow and evolve over time.

\begin{protohead}
def LinearSearch(Data: [?Dom], val): (bool, index(Dom))
\end{protohead}
\begin{protobody}
Searches through the pre-sorted array \chpl{Data} looking for the
value \chpl{val} using a sequential linear search.  Returns a tuple
indicating (1) whether or not the value was found and (2) the location
of the value if it was found, or the location where the value should
have been if it was not found.
\end{protobody}


\begin{protohead}
def BinarySearch(Data: [?Dom], val, in lo = Dom.low, in hi = Dom.high)
\end{protohead}
\begin{protobody}
Searches through the pre-sorted array \chpl{Data} looking for the
value \chpl{val} using a sequential binary search.  If provided, only
the indices \chpl{lo} through \chpl{hi} will be considered, otherwise
the whole array will be searched.  Returns a tuple indicating (1)
whether or not the value was found and (2) the location of the value
if it was found, or the location where the value should have been if
it was not found.
\end{protobody}


\subsubsection{Sort}
\label{Sort}
\index{standard modules!Sort}

The \chpl{Sort} module is designed to support standard sorting
routines.  The current interface is minimal and should be expected to
grow and evolve over time.

\begin{protohead}
def InsertionSort(Data: [?Dom]) where Dom.rank == 1
\end{protohead}
\begin{protobody}
Sorts the 1D array \chpl{Data} in-place using a sequential insertion
sort algorithm.
\end{protobody}

\begin{protohead}
def QuickSort(Data: [?Dom]) where Dom.rank == 1
\end{protohead}
\begin{protobody}
Sorts the 1D array \chpl{Data} in-place using a sequential
implementation of the QuickSort algorithm.
\end{protobody}

\subsubsection{Time}
\label{Time}
\index{standard modules!Time}

The module \chpl{Time} defines routines that query the system time and
a record \chpl{Timer} that is useful for timing portions of code.

\vspace{1pc}

\begin{protohead}
record Timer
\end{protohead}
\begin{protobody}
A timer is used to time portions of code.  Its semantics are similar
to a stopwatch.
\end{protobody}

\begin{protohead}
enum TimeUnits { microseconds, milliseconds, seconds, minutes, hours };
\end{protohead}
\begin{protobody}
The enumeration TimeUnits defines units of time.  These units can be
supplied to routines in this module to specify the desired time units.
\end{protobody}

\begin{protohead}
enum Day { sunday=0, monday, tuesday, wednesday, thursday, friday, saturday };
\end{protohead}
\begin{protobody}
The enumeration Day defines the days of the week, with Sunday defined to be 0.
\end{protobody}


\begin{protohead}
def getCurrentDate(): (int, int, int)
\end{protohead}
\begin{protobody}
Returns the year, month, and day of the month as integers.  The year
is the year since 0.  The month is in the range 1 to 12.  The day is
in the range 1 to 31.
\end{protobody}

\begin{protohead}
def getCurrentDayOfWeek(): Day
\end{protohead}
\begin{protobody}
Returns the current day of the week.
\end{protobody}


\begin{protohead}
def getCurrentTime(unit: TimeUnits = TimeUnits.seconds): real
\end{protohead}
\begin{protobody}
Returns the elapsed time since midnight in the units specified.
\end{protobody}

\begin{protohead}
def Timer.clear()
\end{protohead}
\begin{protobody}
Clears the elapsed time stored in the Timer.
\end{protobody}

\begin{protohead}
def Timer.elapsed(unit: TimeUnits = TimeUnits.seconds): real
\end{protohead}
\begin{protobody}
Returns the cumulative elapsed time, in the units specified, between
calls to \chpl{start} and \chpl{stop}.  If the timer is running, the
elapsed time since the last call to \chpl{start} is added to the
return value.
\end{protobody}

\begin{protohead}
def Timer.start()
\end{protohead}
\begin{protobody}
Starts the timer.  It is an error to start a timer that is already
running.
\end{protobody}

\begin{protohead}
def Timer.stop()
\end{protohead}
\begin{protobody}
Stops the timer.  It is an error to stop a timer that is not running.
\end{protobody}

\begin{protohead}
def sleep(t: uint)
\end{protohead}
\begin{protobody}
Delays a task for \chpl{t} seconds.
\end{protobody}


\cleardoublepage
\appendix
%%
%% Do not modify this file.  This file is automatically
%% generated by collect_syntax.pl.
%%

\sekshun{Collected Lexical and Syntax Productions}
\label{Syntax}

This appendix collects the syntax productions listed throughout the specification.  There are no new syntax productions in this appendix.  The productions are listed both alphabetically and in depth-first order for convenience.

\section{Alphabetical Lexical Productions}

\begin{syntax}
\end{syntax}

\begin{syntax}
binary-digit: one of
  `0' `1'
\end{syntax}

\begin{syntax}
binary-digits:
  binary-digit
  binary-digit binary-digits
\end{syntax}

\begin{syntax}
bool-literal: one of
  `true' $ $ $ $ `false'
\end{syntax}

\begin{syntax}
digit: one of
  `0' `1' `2' `3' `4' `5' `6' `7' `8' `9'
\end{syntax}

\begin{syntax}
digits:
  digit
  digit digits
\end{syntax}

\begin{syntax}
double-quote-delimited-characters:
  string-character double-quote-delimited-characters[OPT]
  ' double-quote-delimited-characters[OPT]
\end{syntax}

\begin{syntax}
exponent-part:
  `e' sign[OPT] digits
  `E' sign[OPT] digits
\end{syntax}

\begin{syntax}
hexadecimal-digit: one of
  `0' `1' `2' `3' `4' `5' `6' `7' `8' `9' `A' `B' `C' `D' `E' `F' `a' `b' `c' `d' `e' `f'
\end{syntax}

\begin{syntax}
hexadecimal-digits:
  hexadecimal-digit
  hexadecimal-digit hexadecimal-digits
\end{syntax}

\begin{syntax}
hexadecimal-escape-character:
  `$\backslash$x' hexadecimal-digits
\end{syntax}

\begin{syntax}
identifier:
  letter-or-underscore legal-identifier-chars[OPT]
\end{syntax}

\begin{syntax}
imaginary-literal:
  real-literal `i'
  integer-literal `i'
\end{syntax}

\begin{syntax}
integer-literal:
  digits
  `0x' hexadecimal-digits
  `0X' hexadecimal-digits
  `0o' octal-digits
  `0O' octal-digits
  `0b' binary-digits
  `0B' binary-digits
\end{syntax}

\begin{syntax}
legal-identifier-char:
  letter-or-underscore
  digit
  `(*\texttt{\$}*)'
\end{syntax}

\begin{syntax}
legal-identifier-chars:
  legal-identifier-char legal-identifier-chars[OPT]
\end{syntax}

\begin{syntax}
letter-or-underscore:
  letter
  `_'
\end{syntax}

\begin{syntax}
letter: one of
  `A' `B' `C' `D' `E' `F' `G' `H' `I' `J' `K' `L' `M' `N' `O' `P' `Q' `R' `S' `T' `U' `V' `W' `X' `Y' `Z'
  `a' `b' `c' `d' `e' `f' `g' `h' `i' `j' `k' `l' `m' `n' `o' `p' `q' `r' `s' `t' `u' `v' `w' `x' `y' `z'
\end{syntax}

\begin{syntax}
octal-digit: one of
  `0' `1' `2' `3' `4' `5' `6' `7'
\end{syntax}

\begin{syntax}
octal-digits:
  octal-digit
  octal-digit octal-digits
\end{syntax}

\begin{syntax}
p-exponent-part:
  `p' sign[OPT] digits
  `P' sign[OPT] digits
\end{syntax}

\begin{syntax}
real-literal:
  digits[OPT] . digits exponent-part[OPT]
  digits .[OPT] exponent-part
  `0x' hexadecimal-digits[OPT] . hexadecimal-digits p-exponent-part[OPT]
  `0X' hexadecimal-digits[OPT] . hexadecimal-digits p-exponent-part[OPT]
  `0x' hexadecimal-digits .[OPT] p-exponent-part
  `0X' hexadecimal-digits .[OPT] p-exponent-part
\end{syntax}

\begin{syntax}
sign: one of
  + $ $ $ $ -
\end{syntax}

\begin{syntax}
simple-escape-character: one of
  `$\backslash\mbox{\bf '}\hspace{5pt}$' `$\backslash$"$\hspace{5pt}$' `$\backslash$?$\hspace{5pt}$' `$\backslash$$\backslash$$\hspace{5pt}$' `$\backslash$a$\hspace{5pt}$' `$\backslash$b$\hspace{5pt}$' `$\backslash$f$\hspace{5pt}$' `$\backslash$n$\hspace{5pt}$' `$\backslash$r$\hspace{5pt}$' `$\backslash$t$\hspace{5pt}$' `$\backslash$v$\hspace{5pt}$'
\end{syntax}

\begin{syntax}
single-quote-delimited-characters:
  string-character single-quote-delimited-characters[OPT]
  " single-quote-delimited-characters[OPT]
\end{syntax}

\begin{syntax}
string-character:
  `any character except the double quote, single quote, or new line'
  simple-escape-character
  hexadecimal-escape-character
\end{syntax}

\begin{syntax}
string-literal:
  " double-quote-delimited-characters[OPT] "
  ' single-quote-delimited-characters[OPT] '
\end{syntax}

\section{Alphabetical Syntax Productions}

\begin{syntax}
\end{syntax}

\begin{syntax}
aligned-range-expression:
  range-expression `align' expression
\end{syntax}

\begin{syntax}
argument-list:
  ( formals[OPT] )
\end{syntax}

\begin{syntax}
array-alias-declaration:
  identifier reindexing-expression[OPT] => array-expression ;
\end{syntax}

\begin{syntax}
array-expression:
  expression
\end{syntax}

\begin{syntax}
array-literal:
  rectangular-array-literal
  associative-array-literal
\end{syntax}

\begin{syntax}
array-type:
  [ domain-expression ] type-specifier
\end{syntax}

\begin{syntax}
assignment-operator: one of
   = $ $ $ $ += $ $ $ $ -= $ $ $ $ *= $ $ $ $ /= $ $ $ $ %= $ $ $ $ **= $ $ $ $ &= $ $ $ $ |= $ $ $ $ ^= $ $ $ $ &&= $ $ $ $ ||= $ $ $ $ <<= $ $ $ $ >>=
\end{syntax}

\begin{syntax}
assignment-statement:
  lvalue-expression assignment-operator expression
\end{syntax}

\begin{syntax}
associative-array-literal:
  [ associative-expr-list ]
  [ associative-expr-list , ]
\end{syntax}

\begin{syntax}
associative-domain-literal:
   { associative-expression-list }
\end{syntax}

\begin{syntax}
associative-domain-type:
  `domain' ( associative-index-type )
  `domain' ( enum-type )
  `domain' ( `opaque' )
\end{syntax}

\begin{syntax}
associative-expr-list:
  index-expr => value-expr
  index-expr => value-expr, associative-expr-list
\end{syntax}

\begin{syntax}
associative-expression-list:
   non-range-expression
   non-range-expression, associative-expression-list
\end{syntax}

\begin{syntax}
associative-index-type:
  type-specifier
\end{syntax}

\begin{syntax}
atomic-statement:
  `atomic' statement
\end{syntax}

\begin{syntax}
atomic-type:
  `atomic' type-specifier
\end{syntax}

\begin{syntax}
base-domain-type:
  rectangular-domain-type
  associative-domain-type
\end{syntax}

\begin{syntax}
begin-statement:
  `begin' task-intent-clause[OPT] statement
\end{syntax}

\begin{syntax}
binary-expression:
  expression binary-operator expression
\end{syntax}

\begin{syntax}
binary-operator: one of
  + $ $ $ $ - $ $ $ $ * $ $ $ $ / $ $ $ $ % $ $ $ $ ** $ $ $ $ & $ $ $ $ | $ $ $ $ ^ $ $ $ $ << $ $ $ $ >> $ $ $ $ && $ $ $ $ || $ $ $ $ == $ $ $ $ != $ $ $ $ <= $ $ $ $ >= $ $ $ $ < $ $ $ $ > $ $ $ $ `by' $ $ $ $ #
\end{syntax}

\begin{syntax}
block-statement:
  { statements[OPT] }
\end{syntax}

\begin{syntax}
break-statement:
  `break' identifier[OPT] ;
\end{syntax}

\begin{syntax}
call-expression:
  lvalue-expression ( named-expression-list )
  lvalue-expression [ named-expression-list ]
  parenthesesless-function-identifier
\end{syntax}

\begin{syntax}
cast-expression:
  expression : type-specifier
\end{syntax}

\begin{syntax}
class-declaration-statement:
  simple-class-declaration-statement
  external-class-declaration-statement
\end{syntax}

\begin{syntax}
class-inherit-list:
  : class-type-list
\end{syntax}

\begin{syntax}
class-name:
  identifier
\end{syntax}

\begin{syntax}
class-statement-list:
  class-statement
  class-statement class-statement-list
\end{syntax}

\begin{syntax}
class-statement:
  variable-declaration-statement
  method-declaration-statement
  type-declaration-statement
  empty-statement
\end{syntax}

\begin{syntax}
class-type-list:
  class-type
  class-type , class-type-list
\end{syntax}

\begin{syntax}
class-type:
  identifier
  identifier ( named-expression-list )
\end{syntax}

\begin{syntax}
cobegin-statement:
  `cobegin' task-intent-clause[OPT] block-statement
\end{syntax}

\begin{syntax}
coforall-statement:
  `coforall' index-var-declaration `in' iteratable-expression task-intent-clause[OPT] `do' statement
  `coforall' index-var-declaration `in' iteratable-expression task-intent-clause[OPT] block-statement
  `coforall' iteratable-expression task-intent-clause[OPT] `do' statement
  `coforall' iteratable-expression task-intent-clause[OPT] block-statement
\end{syntax}

\begin{syntax}
conditional-statement:
  `if' expression `then' statement else-part[OPT]
  `if' expression block-statement else-part[OPT]
\end{syntax}

\begin{syntax}
config-or-extern: one of
  `config' $ $ $ $ `extern'
\end{syntax}

\begin{syntax}
constructor-call-expression:
  `new' class-name ( argument-list )
\end{syntax}

\begin{syntax}
continue-statement:
  `continue' identifier[OPT] ;
\end{syntax}

\begin{syntax}
counted-range-expression:
  range-expression # expression
\end{syntax}

\begin{syntax}
dataparallel-type:
  range-type
  domain-type
  mapped-domain-type
  array-type
  index-type
\end{syntax}

\begin{syntax}
default-expression:
  = expression
\end{syntax}

\begin{syntax}
delete-statement:
  `delete' expression ;
\end{syntax}

\begin{syntax}
dmap-value:
  expression
\end{syntax}

\begin{syntax}
do-while-statement:
  `do' statement `while' expression ;
\end{syntax}

\begin{syntax}
domain-alignment-expression:
  domain-expression `align' expression
\end{syntax}

\begin{syntax}
domain-assignment-expression:
  domain-name = domain-expression
\end{syntax}

\begin{syntax}
domain-expression:
  domain-literal
  domain-name
  domain-assignment-expression
  domain-striding-expression
  domain-alignment-expression
  domain-slice-expression
\end{syntax}

\begin{syntax}
domain-literal:
  rectangular-domain-literal
  associative-domain-literal
\end{syntax}

\begin{syntax}
domain-name:
  identifier
\end{syntax}

\begin{syntax}
domain-slice-expression:
  domain-expression [ slicing-index-set ]
  domain-expression ( slicing-index-set )
\end{syntax}

\begin{syntax}
domain-striding-expression:
  domain-expression `by' expression
\end{syntax}

\begin{syntax}
domain-type:
  base-domain-type
  simple-subdomain-type
  sparse-subdomain-type
\end{syntax}

\begin{syntax}
else-part:
  `else' statement
\end{syntax}

\begin{syntax}
empty-statement:
  ;
\end{syntax}

\begin{syntax}
enum-constant-expression:
  enum-type . identifier
\end{syntax}

\begin{syntax}
enum-constant-list:
  enum-constant
  enum-constant , enum-constant-list[OPT]
\end{syntax}

\begin{syntax}
enum-constant:
  identifier init-part[OPT]
\end{syntax}

\begin{syntax}
enum-declaration-statement:
  `enum' identifier { enum-constant-list }
\end{syntax}

\begin{syntax}
enum-type:
  identifier
\end{syntax}

\begin{syntax}
exclude-list:
  identifier-list
  $ * $
\end{syntax}

\begin{syntax}
exported-procedure-declaration-statement:
  `export' external-name[OPT] `proc' function-name argument-list return-intent[OPT] return-type[OPT]
    function-body
\end{syntax}

\begin{syntax}
expression-list:
  expression
  expression , expression-list
\end{syntax}

\begin{syntax}
expression-statement:
  variable-expression ;
  member-access-expression ;
  call-expression ;
  constructor-call-expression ;
  let-expression ; 
\end{syntax}

\begin{syntax}
expression:
  literal-expression
  nil-expression
  variable-expression
  enum-constant-expression
  call-expression
  iteratable-call-expression
  member-access-expression
  constructor-call-expression
  query-expression
  cast-expression
  lvalue-expression
  parenthesized-expression
  unary-expression
  binary-expression
  let-expression
  if-expression
  for-expression
  forall-expression
  reduce-expression
  scan-expression
  module-access-expression
  tuple-expression
  tuple-expand-expression
  locale-access-expression
  mapped-domain-expression
\end{syntax}

\begin{syntax}
external-class-declaration-statement:
  `extern' external-name[OPT] simple-class-declaration-statement
\end{syntax}

\begin{syntax}
external-name:
  identifier
  string-literal
\end{syntax}

\begin{syntax}
external-procedure-declaration-statement:
  `extern' external-name[OPT] `proc' function-name argument-list return-intent[OPT] return-type[OPT]
\end{syntax}

\begin{syntax}
external-record-declaration-statement:
  `extern' external-name[OPT] simple-record-declaration-statement
\end{syntax}

\begin{syntax}
external-type-alias-declaration-statement:
  `extern' `type' type-alias-declaration-list ;
\end{syntax}

\begin{syntax}
field-access-expression:
  receiver-clause[OPT] identifier
\end{syntax}

\begin{syntax}
for-expression:
  `for' index-var-declaration `in' iteratable-expression `do' expression
  `for' iteratable-expression `do' expression
\end{syntax}

\begin{syntax}
for-statement:
  `for' index-var-declaration `in' iteratable-expression `do' statement
  `for' index-var-declaration `in' iteratable-expression block-statement
  `for' iteratable-expression `do' statement
  `for' iteratable-expression block-statement
\end{syntax}

\begin{syntax}
forall-expression:
  `forall' index-var-declaration `in' iteratable-expression task-intent-clause[OPT] `do' expression
  `forall' iteratable-expression task-intent-clause[OPT] `do' expression
  [ index-var-declaration `in' iteratable-expression task-intent-clause[OPT] ] expression
  [ iteratable-expression task-intent-clause[OPT] ] expression
\end{syntax}

\begin{syntax}
forall-statement:
  `forall' index-var-declaration `in' iteratable-expression task-intent-clause[OPT] `do' statement
  `forall' index-var-declaration `in' iteratable-expression task-intent-clause[OPT] block-statement
  `forall' iteratable-expression task-intent-clause[OPT] `do' statement
  `forall' iteratable-expression task-intent-clause[OPT] block-statement
  [ index-var-declaration `in' iteratable-expression task-intent-clause[OPT] ] statement
  [ iteratable-expression task-intent-clause[OPT] ] statement
\end{syntax}

\begin{syntax}
formal-intent:
  `const'
  `const in'
  `const ref'
  `in'
  `out'
  `inout'
  `ref'
  `param'
  `type'
\end{syntax}

\begin{syntax}
formal-type:
  : type-specifier
  : ? identifier[OPT]
\end{syntax}

\begin{syntax}
formal:
  formal-intent[OPT] identifier formal-type[OPT] default-expression[OPT]
  formal-intent[OPT] identifier formal-type[OPT] variable-argument-expression
  formal-intent[OPT] tuple-grouped-identifier-list formal-type[OPT] default-expression[OPT]
  formal-intent[OPT] tuple-grouped-identifier-list formal-type[OPT] variable-argument-expression
\end{syntax}

\begin{syntax}
formals:
  formal
  formal , formals
\end{syntax}

\begin{syntax}
function-body:
  block-statement
  return-statement
\end{syntax}

\begin{syntax}
function-name:
  identifier
  operator-name
\end{syntax}

\begin{syntax}
identifier-list:
  identifier
  identifier , identifier-list
  tuple-grouped-identifier-list
  tuple-grouped-identifier-list , identifier-list
\end{syntax}

\begin{syntax}
if-expression:
  `if' expression `then' expression `else' expression
  `if' expression `then' expression
\end{syntax}

\begin{syntax}
index-expr:
  expression
\end{syntax}

\begin{syntax}
index-type:
  `index' ( domain-expression )
\end{syntax}

\begin{syntax}
index-var-declaration:
  identifier
  tuple-grouped-identifier-list
\end{syntax}

\begin{syntax}
init-part:
  = expression
\end{syntax}

\begin{syntax}
initialization-part:
  = expression
\end{syntax}

\begin{syntax}
integer-parameter-expression:
  expression
\end{syntax}

\begin{syntax}
io-expression:
  expression
  io-expression io-operator expression
\end{syntax}

\begin{syntax}
io-operator:
  <`(*$\sim$*)'>
\end{syntax}

\begin{syntax}
io-statement:
  io-expression io-operator expression
\end{syntax}

\begin{syntax}
iteratable-call-expression:
  call-expression
\end{syntax}

\begin{syntax}
iteratable-expression:
  expression
  `zip' ( expression-list )
\end{syntax}

\begin{syntax}
iterator-body:
  block-statement
  yield-statement
\end{syntax}

\begin{syntax}
iterator-declaration-statement:
  privacy-specifier[OPT] `iter' iterator-name argument-list[OPT] return-intent[OPT] return-type[OPT] where-clause[OPT]
  iterator-body
\end{syntax}

\begin{syntax}
iterator-name:
  identifier
\end{syntax}

\begin{syntax}
label-statement:
  `label' identifier statement
\end{syntax}

\begin{syntax}
let-expression:
  `let' variable-declaration-list `in' expression
\end{syntax}

\begin{syntax}
limitation-clause:
  `except' exclude-list
  `only' rename-list[OPT]
\end{syntax}

\begin{syntax}
linkage-specifier:
  `inline'
\end{syntax}

\begin{syntax}
literal-expression:
  bool-literal
  integer-literal
  real-literal
  imaginary-literal
  string-literal
  range-literal
  domain-literal
  array-literal
\end{syntax}

\begin{syntax}
locale-access-expression:
  expression . `locale'
\end{syntax}

\begin{syntax}
lvalue-expression:
  variable-expression
  member-access-expression
  call-expression
  parenthesized-expression
\end{syntax}

\begin{syntax}
mapped-domain-expression:
  domain-expression `dmapped' dmap-value
\end{syntax}

\begin{syntax}
mapped-domain-type:
  domain-type `dmapped' dmap-value
\end{syntax}

\begin{syntax}
member-access-expression:
  field-access-expression
  method-call-expression
\end{syntax}

\begin{syntax}
method-call-expression:
  receiver-clause[OPT] expression ( named-expression-list )
  receiver-clause[OPT] expression [ named-expression-list ]
  receiver-clause[OPT] parenthesesless-function-identifier
\end{syntax}

\begin{syntax}
method-declaration-statement:
  linkage-specifier[OPT] proc-or-iter this-intent[OPT] type-binding[OPT] function-name argument-list[OPT] 
    return-intent[OPT] return-type[OPT] where-clause[OPT] function-body
\end{syntax}

\begin{syntax}
module-access-expression:
  module-identifier-list . identifier
\end{syntax}

\begin{syntax}
module-declaration-statement:
  privacy-specifier[OPT] `module' module-identifier block-statement
\end{syntax}

\begin{syntax}
module-identifier-list:
  module-identifier
  module-identifier . module-identifier-list
\end{syntax}

\begin{syntax}
module-identifier:
  identifier
\end{syntax}

\begin{syntax}
module-or-enum-name-list:
  module-or-enum-name limitation-clause[OPT]
  module-or-enum-name , module-or-enum-name-list
\end{syntax}

\begin{syntax}
module-or-enum-name:
  identifier
  identifier . module-or-enum-name
\end{syntax}

\begin{syntax}
named-expression-list:
  named-expression
  named-expression , named-expression-list
\end{syntax}

\begin{syntax}
named-expression:
  expression
  identifier = expression
\end{syntax}

\begin{syntax}
nil-expression:
  `nil'
\end{syntax}

\begin{syntax}
no-initialization-part:
  = `noinit'
\end{syntax}

\begin{syntax}
non-range-expression:
   expression
\end{syntax}

\begin{syntax}
on-statement:
  `on' expression `do' statement
  `on' expression block-statement
\end{syntax}

\begin{syntax}
operator-name: one of
  + $ $ $ $ - $ $ $ $ * $ $ $ $ / $ $ $ $ % $ $ $ $ ** $ $ $ $ ! $ $ $ $ == $ $ $ $ != $ $ $ $ <= $ $ $ $ >= $ $ $ $ < $ $ $ $ > $ $ $ $ << $ $ $ $ >> $ $ $ $ & $ $ $ $ | $ $ $ $ ^ $ $ $ $ ~
  += $ $ $ $ -= $ $ $ $ *= $ $ $ $ /= $ $ $ $ %= $ $ $ $ **= $ $ $ $ &= $ $ $ $ |= $ $ $ $ ^= $ $ $ $ <<= $ $ $ $ >>= $ $ $ $ <=> $ $ $ $ <~>
\end{syntax}

\begin{syntax}
param-for-statement:
  `for' `param' identifier `in' param-iteratable-expression `do' statement
  `for' `param' identifier `in' param-iteratable-expression block-statement
\end{syntax}

\begin{syntax}
param-iteratable-expression:
  range-literal
  range-literal `by' integer-literal
\end{syntax}

\begin{syntax}
parenthesesless-function-identifier:
  identifier
\end{syntax}

\begin{syntax}
parenthesized-expression:
  ( expression )
\end{syntax}

\begin{syntax}
primitive-type-parameter-part:
  ( integer-parameter-expression )
\end{syntax}

\begin{syntax}
primitive-type:
  `void'
  `bool' primitive-type-parameter-part[OPT]
  `int' primitive-type-parameter-part[OPT]
  `uint' primitive-type-parameter-part[OPT]
  `real' primitive-type-parameter-part[OPT]
  `imag' primitive-type-parameter-part[OPT]
  `complex' primitive-type-parameter-part[OPT]
  `string'
\end{syntax}

\begin{syntax}
privacy-specifier:
  `private'
  `public'
\end{syntax}

\begin{syntax}
proc-or-iter:
  `proc'
  `iter'
\end{syntax}

\begin{syntax}
procedure-declaration-statement:
  privacy-specifier[OPT] linkage-specifier[OPT] `proc' function-name argument-list[OPT] return-intent[OPT] return-type[OPT] where-clause[OPT]
    function-body
\end{syntax}

\begin{syntax}
query-expression:
  ? identifier[OPT]
\end{syntax}

\begin{syntax}
range-expression-list:
  range-expression
  range-expression, range-expression-list
\end{syntax}

\begin{syntax}
range-expression:
  expression
  strided-range-expression
  counted-range-expression
  aligned-range-expression
  sliced-range-expression
\end{syntax}

\begin{syntax}
range-literal:
  expression .. expression
  expression ..
  .. expression
  ..
\end{syntax}

\begin{syntax}
range-type:
  `range' ( named-expression-list )
\end{syntax}

\begin{syntax}
receiver-clause:
  expression .
\end{syntax}

\begin{syntax}
record-declaration-statement:
  simple-record-declaration-statement
  external-record-declaration-statement
\end{syntax}

\begin{syntax}
record-inherit-list:
  : record-type-list
\end{syntax}

\begin{syntax}
record-statement-list:
  record-statement
  record-statement record-statement-list
\end{syntax}

\begin{syntax}
record-statement:
  variable-declaration-statement
  method-declaration-statement
  type-declaration-statement
  empty-statement
\end{syntax}

\begin{syntax}
record-type-list:
  record-type
  record-type , record-type-list
\end{syntax}

\begin{syntax}
record-type:
  identifier
  identifier ( named-expression-list )
\end{syntax}

\begin{syntax}
rectangular-array-literal:
  [ expression-list ]
  [ expression-list , ]
\end{syntax}

\begin{syntax}
rectangular-domain-literal:
  { range-expression-list }
\end{syntax}

\begin{syntax}
rectangular-domain-type:
  `domain' ( named-expression-list )
\end{syntax}

\begin{syntax}
reduce-expression:
  reduce-scan-operator `reduce' iteratable-expression
  class-type `reduce' iteratable-expression
\end{syntax}

\begin{syntax}
reduce-scan-operator: one of
  + $ $ $ $ * $ $ $ $ && $ $ $ $ || $ $ $ $ & $ $ $ $ | $ $ $ $ ^ $ $ $ $ `min' $ $ $ $ `max' $ $ $ $ `minloc' $ $ $ $ `maxloc'
\end{syntax}

\begin{syntax}
reindexing-expression:
  : [ domain-expression ]
\end{syntax}

\begin{syntax}
remote-variable-declaration-statement:
  `on' expression variable-declaration-statement
\end{syntax}

\begin{syntax}
rename-base:
  identifier `as' identifier
  identifier
\end{syntax}

\begin{syntax}
rename-list:
  rename-base
  rename-base , rename-list
\end{syntax}

\begin{syntax}
return-intent:
  `const'
  `const ref'
  `ref'
  `param'
  `type'
\end{syntax}

\begin{syntax}
return-statement:
  `return' expression[OPT] ;
\end{syntax}

\begin{syntax}
return-type:
  : type-specifier
\end{syntax}

\begin{syntax}
scan-expression:
  reduce-scan-operator `scan' iteratable-expression
  class-type `scan' iteratable-expression
\end{syntax}

\begin{syntax}
select-statement:
  `select' expression { when-statements }
\end{syntax}

\begin{syntax}
serial-statement:
  `serial' expression[OPT] `do' statement
  `serial' expression[OPT] block-statement
\end{syntax}

\begin{syntax}
simple-class-declaration-statement:
  `class' identifier class-inherit-list[OPT] { class-statement-list[OPT] }
\end{syntax}

\begin{syntax}
simple-record-declaration-statement:
  `record' identifier record-inherit-list[OPT] { record-statement-list }
\end{syntax}

\begin{syntax}
simple-subdomain-type:
  `subdomain' ( domain-expression )
\end{syntax}

\begin{syntax}
single-type:
  `single' type-specifier
\end{syntax}

\begin{syntax}
sliced-range-expression:
  range-expression ( range-expression )
  range-expression [ range-expression ]
\end{syntax}

\begin{syntax}
slicing-index-set:
  domain-expression
  range-expression-list
\end{syntax}

\begin{syntax}
sparse-subdomain-type:
  `sparse' `subdomain'[OPT] ( domain-expression )
\end{syntax}

\begin{syntax}
statement:
  block-statement
  expression-statement
  assignment-statement
  swap-statement
  io-statement
  conditional-statement
  select-statement
  while-do-statement
  do-while-statement
  for-statement
  label-statement
  break-statement
  continue-statement
  param-for-statement
  use-statement
  empty-statement
  return-statement
  yield-statement
  module-declaration-statement
  procedure-declaration-statement
  external-procedure-declaration-statement
  exported-procedure-declaration-statement
  iterator-declaration-statement
  method-declaration-statement
  type-declaration-statement
  variable-declaration-statement
  remote-variable-declaration-statement
  on-statement
  cobegin-statement
  coforall-statement
  begin-statement
  sync-statement
  serial-statement
  atomic-statement
  forall-statement
  delete-statement
\end{syntax}

\begin{syntax}
statements:
  statement
  statement statements
\end{syntax}

\begin{syntax}
step-expression:
  expression
\end{syntax}

\begin{syntax}
strided-range-expression:
  range-expression `by' step-expression
\end{syntax}

\begin{syntax}
structured-type:
  class-type
  record-type
  union-type
  tuple-type
\end{syntax}

\begin{syntax}
swap-operator:
  <=>
\end{syntax}

\begin{syntax}
swap-statement:
  lvalue-expression swap-operator lvalue-expression
\end{syntax}

\begin{syntax}
sync-statement:
  `sync' statement
  `sync' block-statement
\end{syntax}

\begin{syntax}
sync-type:
  `sync' type-specifier
\end{syntax}

\begin{syntax}
synchronization-type:
  sync-type
  single-type
  atomic-type
\end{syntax}

\begin{syntax}
task-intent-clause:
  `with' ( task-intent-list )
\end{syntax}

\begin{syntax}
task-intent-list:
  formal-intent identifier
  formal-intent identifier, task-intent-list
\end{syntax}

\begin{syntax}
this-intent:
  `param'
  `ref'
  `type'
\end{syntax}

\begin{syntax}
tuple-component-list:
  tuple-component
  tuple-component , tuple-component-list
\end{syntax}

\begin{syntax}
tuple-component:
  expression
  `_'
\end{syntax}

\begin{syntax}
tuple-expand-expression:
  ( ... expression )
\end{syntax}

\begin{syntax}
tuple-expression:
  ( tuple-component , )
  ( tuple-component , tuple-component-list )
  ( tuple-component , tuple-component-list , )
\end{syntax}

\begin{syntax}
tuple-grouped-identifier-list:
  ( identifier-list )
\end{syntax}

\begin{syntax}
tuple-type:
  ( type-specifier , type-list )
\end{syntax}

\begin{syntax}
type-alias-declaration-list:
  type-alias-declaration
  type-alias-declaration , type-alias-declaration-list
\end{syntax}

\begin{syntax}
type-alias-declaration-statement:
  privacy-specifier[OPT] `config'[OPT] `type' type-alias-declaration-list ;
  external-type-alias-declaration-statement
\end{syntax}

\begin{syntax}
type-alias-declaration:
  identifier = type-specifier
  identifier
\end{syntax}

\begin{syntax}
type-binding:
  identifier .
\end{syntax}

\begin{syntax}
type-declaration-statement:
  enum-declaration-statement
  class-declaration-statement
  record-declaration-statement
  union-declaration-statement
  type-alias-declaration-statement
\end{syntax}

\begin{syntax}
type-list:
  type-specifier
  type-specifier , type-list
\end{syntax}

\begin{syntax}
type-part:
  : type-specifier
\end{syntax}

\begin{syntax}
type-specifier:
  primitive-type
  enum-type
  structured-type
  dataparallel-type
  synchronization-type
\end{syntax}

\begin{syntax}
unary-expression:
  unary-operator expression
\end{syntax}

\begin{syntax}
unary-operator: one of
  + $ $ $ $ - $ $ $ $ ~ $ $ $ $ !
\end{syntax}

\begin{syntax}
union-declaration-statement:
  `extern'[OPT] `union' identifier { union-statement-list }
\end{syntax}

\begin{syntax}
union-statement-list:
  union-statement
  union-statement union-statement-list
\end{syntax}

\begin{syntax}
union-statement:
  type-declaration-statement
  procedure-declaration-statement
  iterator-declaration-statement
  variable-declaration-statement
  empty-statement
\end{syntax}

\begin{syntax}
union-type:
  identifier
\end{syntax}

\begin{syntax}
use-statement:
  `use' module-or-enum-name-list ;
\end{syntax}

\begin{syntax}
value-expr:
  expression
\end{syntax}

\begin{syntax}
variable-argument-expression:
  ... expression
  ... ? identifier[OPT]
  ...
\end{syntax}

\begin{syntax}
variable-declaration-list:
  variable-declaration
  variable-declaration , variable-declaration-list
\end{syntax}

\begin{syntax}
variable-declaration-statement:
  privacy-specifier[OPT] config-or-extern[OPT] variable-kind variable-declaration-list ;
\end{syntax}

\begin{syntax}
variable-declaration:
  identifier-list type-part[OPT] initialization-part
  identifier-list type-part no-initialization-part[OPT]
  array-alias-declaration
\end{syntax}

\begin{syntax}
variable-expression:
  identifier
\end{syntax}

\begin{syntax}
variable-kind:
  `param'
  `const'
  `var'
  `ref'
  `const ref'
\end{syntax}

\begin{syntax}
when-statement:
  `when' expression-list `do' statement
  `when' expression-list block-statement
  `otherwise' statement
  `otherwise' `do' statement
\end{syntax}

\begin{syntax}
when-statements:
  when-statement
  when-statement when-statements
\end{syntax}

\begin{syntax}
where-clause:
  `where' expression
\end{syntax}

\begin{syntax}
while-do-statement:
  `while' expression `do' statement
  `while' expression block-statement
\end{syntax}

\begin{syntax}
yield-statement:
  `yield' expression ;
\end{syntax}

\section{Depth-First Lexical Productions}

\begin{syntax}
bool-literal: one of
  `true' $ $ $ $ `false'
\end{syntax}

\begin{syntax}
identifier:
  letter-or-underscore legal-identifier-chars[OPT]
\end{syntax}

\begin{syntax}
letter-or-underscore:
  letter
  `_'
\end{syntax}

\begin{syntax}
letter: one of
  `A' `B' `C' `D' `E' `F' `G' `H' `I' `J' `K' `L' `M' `N' `O' `P' `Q' `R' `S' `T' `U' `V' `W' `X' `Y' `Z'
  `a' `b' `c' `d' `e' `f' `g' `h' `i' `j' `k' `l' `m' `n' `o' `p' `q' `r' `s' `t' `u' `v' `w' `x' `y' `z'
\end{syntax}

\begin{syntax}
legal-identifier-chars:
  legal-identifier-char legal-identifier-chars[OPT]
\end{syntax}

\begin{syntax}
legal-identifier-char:
  letter-or-underscore
  digit
  `(*\texttt{\$}*)'
\end{syntax}

\begin{syntax}
digit: one of
  `0' `1' `2' `3' `4' `5' `6' `7' `8' `9'
\end{syntax}

\begin{syntax}
imaginary-literal:
  real-literal `i'
  integer-literal `i'
\end{syntax}

\begin{syntax}
real-literal:
  digits[OPT] . digits exponent-part[OPT]
  digits .[OPT] exponent-part
  `0x' hexadecimal-digits[OPT] . hexadecimal-digits p-exponent-part[OPT]
  `0X' hexadecimal-digits[OPT] . hexadecimal-digits p-exponent-part[OPT]
  `0x' hexadecimal-digits .[OPT] p-exponent-part
  `0X' hexadecimal-digits .[OPT] p-exponent-part
\end{syntax}

\begin{syntax}
digits:
  digit
  digit digits
\end{syntax}

\begin{syntax}
exponent-part:
  `e' sign[OPT] digits
  `E' sign[OPT] digits
\end{syntax}

\begin{syntax}
sign: one of
  + $ $ $ $ -
\end{syntax}

\begin{syntax}
hexadecimal-digits:
  hexadecimal-digit
  hexadecimal-digit hexadecimal-digits
\end{syntax}

\begin{syntax}
hexadecimal-digit: one of
  `0' `1' `2' `3' `4' `5' `6' `7' `8' `9' `A' `B' `C' `D' `E' `F' `a' `b' `c' `d' `e' `f'
\end{syntax}

\begin{syntax}
p-exponent-part:
  `p' sign[OPT] digits
  `P' sign[OPT] digits
\end{syntax}

\begin{syntax}
integer-literal:
  digits
  `0x' hexadecimal-digits
  `0X' hexadecimal-digits
  `0o' octal-digits
  `0O' octal-digits
  `0b' binary-digits
  `0B' binary-digits
\end{syntax}

\begin{syntax}
octal-digits:
  octal-digit
  octal-digit octal-digits
\end{syntax}

\begin{syntax}
octal-digit: one of
  `0' `1' `2' `3' `4' `5' `6' `7'
\end{syntax}

\begin{syntax}
binary-digits:
  binary-digit
  binary-digit binary-digits
\end{syntax}

\begin{syntax}
binary-digit: one of
  `0' `1'
\end{syntax}

\begin{syntax}
string-literal:
  " double-quote-delimited-characters[OPT] "
  ' single-quote-delimited-characters[OPT] '
\end{syntax}

\begin{syntax}
double-quote-delimited-characters:
  string-character double-quote-delimited-characters[OPT]
  ' double-quote-delimited-characters[OPT]
\end{syntax}

\begin{syntax}
string-character:
  `any character except the double quote, single quote, or new line'
  simple-escape-character
  hexadecimal-escape-character
\end{syntax}

\begin{syntax}
simple-escape-character: one of
  `$\backslash\mbox{\bf '}\hspace{5pt}$' `$\backslash$"$\hspace{5pt}$' `$\backslash$?$\hspace{5pt}$' `$\backslash$$\backslash$$\hspace{5pt}$' `$\backslash$a$\hspace{5pt}$' `$\backslash$b$\hspace{5pt}$' `$\backslash$f$\hspace{5pt}$' `$\backslash$n$\hspace{5pt}$' `$\backslash$r$\hspace{5pt}$' `$\backslash$t$\hspace{5pt}$' `$\backslash$v$\hspace{5pt}$'
\end{syntax}

\begin{syntax}
hexadecimal-escape-character:
  `$\backslash$x' hexadecimal-digits
\end{syntax}

\begin{syntax}
single-quote-delimited-characters:
  string-character single-quote-delimited-characters[OPT]
  " single-quote-delimited-characters[OPT]
\end{syntax}

\section{Depth-First Syntax Productions}

\begin{syntax}
module-declaration-statement:
  privacy-specifier[OPT] `module' module-identifier block-statement
\end{syntax}

\begin{syntax}
privacy-specifier:
  `private'
  `public'
\end{syntax}

\begin{syntax}
module-identifier:
  identifier
\end{syntax}

\begin{syntax}
block-statement:
  { statements[OPT] }
\end{syntax}

\begin{syntax}
statements:
  statement
  statement statements
\end{syntax}

\begin{syntax}
statement:
  block-statement
  expression-statement
  assignment-statement
  swap-statement
  io-statement
  conditional-statement
  select-statement
  while-do-statement
  do-while-statement
  for-statement
  label-statement
  break-statement
  continue-statement
  param-for-statement
  use-statement
  empty-statement
  return-statement
  yield-statement
  module-declaration-statement
  procedure-declaration-statement
  external-procedure-declaration-statement
  exported-procedure-declaration-statement
  iterator-declaration-statement
  method-declaration-statement
  type-declaration-statement
  variable-declaration-statement
  remote-variable-declaration-statement
  on-statement
  cobegin-statement
  coforall-statement
  begin-statement
  sync-statement
  serial-statement
  atomic-statement
  forall-statement
  delete-statement
\end{syntax}

\begin{syntax}
expression-statement:
  variable-expression ;
  member-access-expression ;
  call-expression ;
  constructor-call-expression ;
  let-expression ; 
\end{syntax}

\begin{syntax}
variable-expression:
  identifier
\end{syntax}

\begin{syntax}
member-access-expression:
  field-access-expression
  method-call-expression
\end{syntax}

\begin{syntax}
field-access-expression:
  receiver-clause[OPT] identifier
\end{syntax}

\begin{syntax}
receiver-clause:
  expression .
\end{syntax}

\begin{syntax}
expression:
  literal-expression
  nil-expression
  variable-expression
  enum-constant-expression
  call-expression
  iteratable-call-expression
  member-access-expression
  constructor-call-expression
  query-expression
  cast-expression
  lvalue-expression
  parenthesized-expression
  unary-expression
  binary-expression
  let-expression
  if-expression
  for-expression
  forall-expression
  reduce-expression
  scan-expression
  module-access-expression
  tuple-expression
  tuple-expand-expression
  locale-access-expression
  mapped-domain-expression
\end{syntax}

\begin{syntax}
literal-expression:
  bool-literal
  integer-literal
  real-literal
  imaginary-literal
  string-literal
  range-literal
  domain-literal
  array-literal
\end{syntax}

\begin{syntax}
range-literal:
  expression .. expression
  expression ..
  .. expression
  ..
\end{syntax}

\begin{syntax}
domain-literal:
  rectangular-domain-literal
  associative-domain-literal
\end{syntax}

\begin{syntax}
rectangular-domain-literal:
  { range-expression-list }
\end{syntax}

\begin{syntax}
range-expression-list:
  range-expression
  range-expression, range-expression-list
\end{syntax}

\begin{syntax}
range-expression:
  expression
  strided-range-expression
  counted-range-expression
  aligned-range-expression
  sliced-range-expression
\end{syntax}

\begin{syntax}
strided-range-expression:
  range-expression `by' step-expression
\end{syntax}

\begin{syntax}
step-expression:
  expression
\end{syntax}

\begin{syntax}
counted-range-expression:
  range-expression # expression
\end{syntax}

\begin{syntax}
aligned-range-expression:
  range-expression `align' expression
\end{syntax}

\begin{syntax}
sliced-range-expression:
  range-expression ( range-expression )
  range-expression [ range-expression ]
\end{syntax}

\begin{syntax}
associative-domain-literal:
   { associative-expression-list }
\end{syntax}

\begin{syntax}
associative-expression-list:
   non-range-expression
   non-range-expression, associative-expression-list
\end{syntax}

\begin{syntax}
non-range-expression:
   expression
\end{syntax}

\begin{syntax}
array-literal:
  rectangular-array-literal
  associative-array-literal
\end{syntax}

\begin{syntax}
rectangular-array-literal:
  [ expression-list ]
  [ expression-list , ]
\end{syntax}

\begin{syntax}
expression-list:
  expression
  expression , expression-list
\end{syntax}

\begin{syntax}
associative-array-literal:
  [ associative-expr-list ]
  [ associative-expr-list , ]
\end{syntax}

\begin{syntax}
associative-expr-list:
  index-expr => value-expr
  index-expr => value-expr, associative-expr-list
\end{syntax}

\begin{syntax}
index-expr:
  expression
\end{syntax}

\begin{syntax}
value-expr:
  expression
\end{syntax}

\begin{syntax}
nil-expression:
  `nil'
\end{syntax}

\begin{syntax}
enum-constant-expression:
  enum-type . identifier
\end{syntax}

\begin{syntax}
enum-type:
  identifier
\end{syntax}

\begin{syntax}
iteratable-call-expression:
  call-expression
\end{syntax}

\begin{syntax}
query-expression:
  ? identifier[OPT]
\end{syntax}

\begin{syntax}
cast-expression:
  expression : type-specifier
\end{syntax}

\begin{syntax}
type-specifier:
  primitive-type
  enum-type
  structured-type
  dataparallel-type
  synchronization-type
\end{syntax}

\begin{syntax}
primitive-type:
  `void'
  `bool' primitive-type-parameter-part[OPT]
  `int' primitive-type-parameter-part[OPT]
  `uint' primitive-type-parameter-part[OPT]
  `real' primitive-type-parameter-part[OPT]
  `imag' primitive-type-parameter-part[OPT]
  `complex' primitive-type-parameter-part[OPT]
  `string'
\end{syntax}

\begin{syntax}
primitive-type-parameter-part:
  ( integer-parameter-expression )
\end{syntax}

\begin{syntax}
integer-parameter-expression:
  expression
\end{syntax}

\begin{syntax}
structured-type:
  class-type
  record-type
  union-type
  tuple-type
\end{syntax}

\begin{syntax}
class-type:
  identifier
  identifier ( named-expression-list )
\end{syntax}

\begin{syntax}
named-expression-list:
  named-expression
  named-expression , named-expression-list
\end{syntax}

\begin{syntax}
named-expression:
  expression
  identifier = expression
\end{syntax}

\begin{syntax}
record-type:
  identifier
  identifier ( named-expression-list )
\end{syntax}

\begin{syntax}
union-type:
  identifier
\end{syntax}

\begin{syntax}
tuple-type:
  ( type-specifier , type-list )
\end{syntax}

\begin{syntax}
type-list:
  type-specifier
  type-specifier , type-list
\end{syntax}

\begin{syntax}
dataparallel-type:
  range-type
  domain-type
  mapped-domain-type
  array-type
  index-type
\end{syntax}

\begin{syntax}
range-type:
  `range' ( named-expression-list )
\end{syntax}

\begin{syntax}
domain-type:
  base-domain-type
  simple-subdomain-type
  sparse-subdomain-type
\end{syntax}

\begin{syntax}
base-domain-type:
  rectangular-domain-type
  associative-domain-type
\end{syntax}

\begin{syntax}
rectangular-domain-type:
  `domain' ( named-expression-list )
\end{syntax}

\begin{syntax}
associative-domain-type:
  `domain' ( associative-index-type )
  `domain' ( enum-type )
  `domain' ( `opaque' )
\end{syntax}

\begin{syntax}
associative-index-type:
  type-specifier
\end{syntax}

\begin{syntax}
simple-subdomain-type:
  `subdomain' ( domain-expression )
\end{syntax}

\begin{syntax}
domain-expression:
  domain-literal
  domain-name
  domain-assignment-expression
  domain-striding-expression
  domain-alignment-expression
  domain-slice-expression
\end{syntax}

\begin{syntax}
domain-name:
  identifier
\end{syntax}

\begin{syntax}
domain-assignment-expression:
  domain-name = domain-expression
\end{syntax}

\begin{syntax}
domain-striding-expression:
  domain-expression `by' expression
\end{syntax}

\begin{syntax}
domain-alignment-expression:
  domain-expression `align' expression
\end{syntax}

\begin{syntax}
domain-slice-expression:
  domain-expression [ slicing-index-set ]
  domain-expression ( slicing-index-set )
\end{syntax}

\begin{syntax}
slicing-index-set:
  domain-expression
  range-expression-list
\end{syntax}

\begin{syntax}
sparse-subdomain-type:
  `sparse' `subdomain'[OPT] ( domain-expression )
\end{syntax}

\begin{syntax}
mapped-domain-type:
  domain-type `dmapped' dmap-value
\end{syntax}

\begin{syntax}
dmap-value:
  expression
\end{syntax}

\begin{syntax}
array-type:
  [ domain-expression ] type-specifier
\end{syntax}

\begin{syntax}
index-type:
  `index' ( domain-expression )
\end{syntax}

\begin{syntax}
synchronization-type:
  sync-type
  single-type
  atomic-type
\end{syntax}

\begin{syntax}
sync-type:
  `sync' type-specifier
\end{syntax}

\begin{syntax}
single-type:
  `single' type-specifier
\end{syntax}

\begin{syntax}
atomic-type:
  `atomic' type-specifier
\end{syntax}

\begin{syntax}
lvalue-expression:
  variable-expression
  member-access-expression
  call-expression
  parenthesized-expression
\end{syntax}

\begin{syntax}
parenthesized-expression:
  ( expression )
\end{syntax}

\begin{syntax}
unary-expression:
  unary-operator expression
\end{syntax}

\begin{syntax}
unary-operator: one of
  + $ $ $ $ - $ $ $ $ ~ $ $ $ $ !
\end{syntax}

\begin{syntax}
binary-expression:
  expression binary-operator expression
\end{syntax}

\begin{syntax}
binary-operator: one of
  + $ $ $ $ - $ $ $ $ * $ $ $ $ / $ $ $ $ % $ $ $ $ ** $ $ $ $ & $ $ $ $ | $ $ $ $ ^ $ $ $ $ << $ $ $ $ >> $ $ $ $ && $ $ $ $ || $ $ $ $ == $ $ $ $ != $ $ $ $ <= $ $ $ $ >= $ $ $ $ < $ $ $ $ > $ $ $ $ `by' $ $ $ $ #
\end{syntax}

\begin{syntax}
if-expression:
  `if' expression `then' expression `else' expression
  `if' expression `then' expression
\end{syntax}

\begin{syntax}
for-expression:
  `for' index-var-declaration `in' iteratable-expression `do' expression
  `for' iteratable-expression `do' expression
\end{syntax}

\begin{syntax}
forall-expression:
  `forall' index-var-declaration `in' iteratable-expression task-intent-clause[OPT] `do' expression
  `forall' iteratable-expression task-intent-clause[OPT] `do' expression
  [ index-var-declaration `in' iteratable-expression task-intent-clause[OPT] ] expression
  [ iteratable-expression task-intent-clause[OPT] ] expression
\end{syntax}

\begin{syntax}
index-var-declaration:
  identifier
  tuple-grouped-identifier-list
\end{syntax}

\begin{syntax}
tuple-grouped-identifier-list:
  ( identifier-list )
\end{syntax}

\begin{syntax}
identifier-list:
  identifier
  identifier , identifier-list
  tuple-grouped-identifier-list
  tuple-grouped-identifier-list , identifier-list
\end{syntax}

\begin{syntax}
iteratable-expression:
  expression
  `zip' ( expression-list )
\end{syntax}

\begin{syntax}
task-intent-clause:
  `with' ( task-intent-list )
\end{syntax}

\begin{syntax}
task-intent-list:
  formal-intent identifier
  formal-intent identifier, task-intent-list
\end{syntax}

\begin{syntax}
formal-intent:
  `const'
  `const in'
  `const ref'
  `in'
  `out'
  `inout'
  `ref'
  `param'
  `type'
\end{syntax}

\begin{syntax}
reduce-expression:
  reduce-scan-operator `reduce' iteratable-expression
  class-type `reduce' iteratable-expression
\end{syntax}

\begin{syntax}
reduce-scan-operator: one of
  + $ $ $ $ * $ $ $ $ && $ $ $ $ || $ $ $ $ & $ $ $ $ | $ $ $ $ ^ $ $ $ $ `min' $ $ $ $ `max' $ $ $ $ `minloc' $ $ $ $ `maxloc'
\end{syntax}

\begin{syntax}
scan-expression:
  reduce-scan-operator `scan' iteratable-expression
  class-type `scan' iteratable-expression
\end{syntax}

\begin{syntax}
module-access-expression:
  module-identifier-list . identifier
\end{syntax}

\begin{syntax}
module-identifier-list:
  module-identifier
  module-identifier . module-identifier-list
\end{syntax}

\begin{syntax}
tuple-expression:
  ( tuple-component , )
  ( tuple-component , tuple-component-list )
  ( tuple-component , tuple-component-list , )
\end{syntax}

\begin{syntax}
tuple-component:
  expression
  `_'
\end{syntax}

\begin{syntax}
tuple-component-list:
  tuple-component
  tuple-component , tuple-component-list
\end{syntax}

\begin{syntax}
tuple-expand-expression:
  ( ... expression )
\end{syntax}

\begin{syntax}
locale-access-expression:
  expression . `locale'
\end{syntax}

\begin{syntax}
mapped-domain-expression:
  domain-expression `dmapped' dmap-value
\end{syntax}

\begin{syntax}
method-call-expression:
  receiver-clause[OPT] expression ( named-expression-list )
  receiver-clause[OPT] expression [ named-expression-list ]
  receiver-clause[OPT] parenthesesless-function-identifier
\end{syntax}

\begin{syntax}
parenthesesless-function-identifier:
  identifier
\end{syntax}

\begin{syntax}
call-expression:
  lvalue-expression ( named-expression-list )
  lvalue-expression [ named-expression-list ]
  parenthesesless-function-identifier
\end{syntax}

\begin{syntax}
constructor-call-expression:
  `new' class-name ( argument-list )
\end{syntax}

\begin{syntax}
class-name:
  identifier
\end{syntax}

\begin{syntax}
argument-list:
  ( formals[OPT] )
\end{syntax}

\begin{syntax}
formals:
  formal
  formal , formals
\end{syntax}

\begin{syntax}
formal:
  formal-intent[OPT] identifier formal-type[OPT] default-expression[OPT]
  formal-intent[OPT] identifier formal-type[OPT] variable-argument-expression
  formal-intent[OPT] tuple-grouped-identifier-list formal-type[OPT] default-expression[OPT]
  formal-intent[OPT] tuple-grouped-identifier-list formal-type[OPT] variable-argument-expression
\end{syntax}

\begin{syntax}
default-expression:
  = expression
\end{syntax}

\begin{syntax}
formal-type:
  : type-specifier
  : ? identifier[OPT]
\end{syntax}

\begin{syntax}
variable-argument-expression:
  ... expression
  ... ? identifier[OPT]
  ...
\end{syntax}

\begin{syntax}
let-expression:
  `let' variable-declaration-list `in' expression
\end{syntax}

\begin{syntax}
assignment-statement:
  lvalue-expression assignment-operator expression
\end{syntax}

\begin{syntax}
assignment-operator: one of
   = $ $ $ $ += $ $ $ $ -= $ $ $ $ *= $ $ $ $ /= $ $ $ $ %= $ $ $ $ **= $ $ $ $ &= $ $ $ $ |= $ $ $ $ ^= $ $ $ $ &&= $ $ $ $ ||= $ $ $ $ <<= $ $ $ $ >>=
\end{syntax}

\begin{syntax}
swap-statement:
  lvalue-expression swap-operator lvalue-expression
\end{syntax}

\begin{syntax}
swap-operator:
  <=>
\end{syntax}

\begin{syntax}
io-statement:
  io-expression io-operator expression
\end{syntax}

\begin{syntax}
io-expression:
  expression
  io-expression io-operator expression
\end{syntax}

\begin{syntax}
io-operator:
  <`(*$\sim$*)'>
\end{syntax}

\begin{syntax}
conditional-statement:
  `if' expression `then' statement else-part[OPT]
  `if' expression block-statement else-part[OPT]
\end{syntax}

\begin{syntax}
else-part:
  `else' statement
\end{syntax}

\begin{syntax}
select-statement:
  `select' expression { when-statements }
\end{syntax}

\begin{syntax}
when-statements:
  when-statement
  when-statement when-statements
\end{syntax}

\begin{syntax}
when-statement:
  `when' expression-list `do' statement
  `when' expression-list block-statement
  `otherwise' statement
  `otherwise' `do' statement
\end{syntax}

\begin{syntax}
while-do-statement:
  `while' expression `do' statement
  `while' expression block-statement
\end{syntax}

\begin{syntax}
do-while-statement:
  `do' statement `while' expression ;
\end{syntax}

\begin{syntax}
for-statement:
  `for' index-var-declaration `in' iteratable-expression `do' statement
  `for' index-var-declaration `in' iteratable-expression block-statement
  `for' iteratable-expression `do' statement
  `for' iteratable-expression block-statement
\end{syntax}

\begin{syntax}
label-statement:
  `label' identifier statement
\end{syntax}

\begin{syntax}
break-statement:
  `break' identifier[OPT] ;
\end{syntax}

\begin{syntax}
continue-statement:
  `continue' identifier[OPT] ;
\end{syntax}

\begin{syntax}
param-for-statement:
  `for' `param' identifier `in' param-iteratable-expression `do' statement
  `for' `param' identifier `in' param-iteratable-expression block-statement
\end{syntax}

\begin{syntax}
param-iteratable-expression:
  range-literal
  range-literal `by' integer-literal
\end{syntax}

\begin{syntax}
use-statement:
  `use' module-or-enum-name-list ;
\end{syntax}

\begin{syntax}
module-or-enum-name-list:
  module-or-enum-name limitation-clause[OPT]
  module-or-enum-name , module-or-enum-name-list
\end{syntax}

\begin{syntax}
limitation-clause:
  `except' exclude-list
  `only' rename-list[OPT]
\end{syntax}

\begin{syntax}
exclude-list:
  identifier-list
  $ * $
\end{syntax}

\begin{syntax}
rename-list:
  rename-base
  rename-base , rename-list
\end{syntax}

\begin{syntax}
rename-base:
  identifier `as' identifier
  identifier
\end{syntax}

\begin{syntax}
module-or-enum-name:
  identifier
  identifier . module-or-enum-name
\end{syntax}

\begin{syntax}
empty-statement:
  ;
\end{syntax}

\begin{syntax}
return-statement:
  `return' expression[OPT] ;
\end{syntax}

\begin{syntax}
yield-statement:
  `yield' expression ;
\end{syntax}

\begin{syntax}
module-declaration-statement:
  privacy-specifier[OPT] `module' module-identifier block-statement
\end{syntax}

\begin{syntax}
procedure-declaration-statement:
  privacy-specifier[OPT] linkage-specifier[OPT] `proc' function-name argument-list[OPT] return-intent[OPT] return-type[OPT] where-clause[OPT]
    function-body
\end{syntax}

\begin{syntax}
linkage-specifier:
  `inline'
\end{syntax}

\begin{syntax}
function-name:
  identifier
  operator-name
\end{syntax}

\begin{syntax}
operator-name: one of
  + $ $ $ $ - $ $ $ $ * $ $ $ $ / $ $ $ $ % $ $ $ $ ** $ $ $ $ ! $ $ $ $ == $ $ $ $ != $ $ $ $ <= $ $ $ $ >= $ $ $ $ < $ $ $ $ > $ $ $ $ << $ $ $ $ >> $ $ $ $ & $ $ $ $ | $ $ $ $ ^ $ $ $ $ ~
  += $ $ $ $ -= $ $ $ $ *= $ $ $ $ /= $ $ $ $ %= $ $ $ $ **= $ $ $ $ &= $ $ $ $ |= $ $ $ $ ^= $ $ $ $ <<= $ $ $ $ >>= $ $ $ $ <=> $ $ $ $ <~>
\end{syntax}

\begin{syntax}
return-intent:
  `const'
  `const ref'
  `ref'
  `param'
  `type'
\end{syntax}

\begin{syntax}
return-type:
  : type-specifier
\end{syntax}

\begin{syntax}
where-clause:
  `where' expression
\end{syntax}

\begin{syntax}
function-body:
  block-statement
  return-statement
\end{syntax}

\begin{syntax}
external-procedure-declaration-statement:
  `extern' external-name[OPT] `proc' function-name argument-list return-intent[OPT] return-type[OPT]
\end{syntax}

\begin{syntax}
exported-procedure-declaration-statement:
  `export' external-name[OPT] `proc' function-name argument-list return-intent[OPT] return-type[OPT]
    function-body
\end{syntax}

\begin{syntax}
iterator-declaration-statement:
  privacy-specifier[OPT] `iter' iterator-name argument-list[OPT] return-intent[OPT] return-type[OPT] where-clause[OPT]
  iterator-body
\end{syntax}

\begin{syntax}
iterator-name:
  identifier
\end{syntax}

\begin{syntax}
iterator-body:
  block-statement
  yield-statement
\end{syntax}

\begin{syntax}
method-declaration-statement:
  linkage-specifier[OPT] proc-or-iter this-intent[OPT] type-binding[OPT] function-name argument-list[OPT] 
    return-intent[OPT] return-type[OPT] where-clause[OPT] function-body
\end{syntax}

\begin{syntax}
proc-or-iter:
  `proc'
  `iter'
\end{syntax}

\begin{syntax}
this-intent:
  `param'
  `ref'
  `type'
\end{syntax}

\begin{syntax}
type-binding:
  identifier .
\end{syntax}

\begin{syntax}
type-declaration-statement:
  enum-declaration-statement
  class-declaration-statement
  record-declaration-statement
  union-declaration-statement
  type-alias-declaration-statement
\end{syntax}

\begin{syntax}
enum-declaration-statement:
  `enum' identifier { enum-constant-list }
\end{syntax}

\begin{syntax}
enum-constant-list:
  enum-constant
  enum-constant , enum-constant-list[OPT]
\end{syntax}

\begin{syntax}
enum-constant:
  identifier init-part[OPT]
\end{syntax}

\begin{syntax}
init-part:
  = expression
\end{syntax}

\begin{syntax}
class-declaration-statement:
  simple-class-declaration-statement
  external-class-declaration-statement
\end{syntax}

\begin{syntax}
simple-class-declaration-statement:
  `class' identifier class-inherit-list[OPT] { class-statement-list[OPT] }
\end{syntax}

\begin{syntax}
class-inherit-list:
  : class-type-list
\end{syntax}

\begin{syntax}
class-type-list:
  class-type
  class-type , class-type-list
\end{syntax}

\begin{syntax}
class-statement-list:
  class-statement
  class-statement class-statement-list
\end{syntax}

\begin{syntax}
class-statement:
  variable-declaration-statement
  method-declaration-statement
  type-declaration-statement
  empty-statement
\end{syntax}

\begin{syntax}
external-class-declaration-statement:
  `extern' external-name[OPT] simple-class-declaration-statement
\end{syntax}

\begin{syntax}
external-name:
  identifier
  string-literal
\end{syntax}

\begin{syntax}
record-declaration-statement:
  simple-record-declaration-statement
  external-record-declaration-statement
\end{syntax}

\begin{syntax}
simple-record-declaration-statement:
  `record' identifier record-inherit-list[OPT] { record-statement-list }
\end{syntax}

\begin{syntax}
record-inherit-list:
  : record-type-list
\end{syntax}

\begin{syntax}
record-type-list:
  record-type
  record-type , record-type-list
\end{syntax}

\begin{syntax}
record-statement-list:
  record-statement
  record-statement record-statement-list
\end{syntax}

\begin{syntax}
record-statement:
  variable-declaration-statement
  method-declaration-statement
  type-declaration-statement
  empty-statement
\end{syntax}

\begin{syntax}
external-record-declaration-statement:
  `extern' external-name[OPT] simple-record-declaration-statement
\end{syntax}

\begin{syntax}
union-declaration-statement:
  `extern'[OPT] `union' identifier { union-statement-list }
\end{syntax}

\begin{syntax}
union-statement-list:
  union-statement
  union-statement union-statement-list
\end{syntax}

\begin{syntax}
union-statement:
  type-declaration-statement
  procedure-declaration-statement
  iterator-declaration-statement
  variable-declaration-statement
  empty-statement
\end{syntax}

\begin{syntax}
type-alias-declaration-statement:
  privacy-specifier[OPT] `config'[OPT] `type' type-alias-declaration-list ;
  external-type-alias-declaration-statement
\end{syntax}

\begin{syntax}
type-alias-declaration-list:
  type-alias-declaration
  type-alias-declaration , type-alias-declaration-list
\end{syntax}

\begin{syntax}
type-alias-declaration:
  identifier = type-specifier
  identifier
\end{syntax}

\begin{syntax}
external-type-alias-declaration-statement:
  `extern' `type' type-alias-declaration-list ;
\end{syntax}

\begin{syntax}
variable-declaration-statement:
  privacy-specifier[OPT] config-or-extern[OPT] variable-kind variable-declaration-list ;
\end{syntax}

\begin{syntax}
config-or-extern: one of
  `config' $ $ $ $ `extern'
\end{syntax}

\begin{syntax}
variable-kind:
  `param'
  `const'
  `var'
  `ref'
  `const ref'
\end{syntax}

\begin{syntax}
variable-declaration-list:
  variable-declaration
  variable-declaration , variable-declaration-list
\end{syntax}

\begin{syntax}
variable-declaration:
  identifier-list type-part[OPT] initialization-part
  identifier-list type-part no-initialization-part[OPT]
  array-alias-declaration
\end{syntax}

\begin{syntax}
initialization-part:
  = expression
\end{syntax}

\begin{syntax}
type-part:
  : type-specifier
\end{syntax}

\begin{syntax}
no-initialization-part:
  = `noinit'
\end{syntax}

\begin{syntax}
array-alias-declaration:
  identifier reindexing-expression[OPT] => array-expression ;
\end{syntax}

\begin{syntax}
reindexing-expression:
  : [ domain-expression ]
\end{syntax}

\begin{syntax}
array-expression:
  expression
\end{syntax}

\begin{syntax}
remote-variable-declaration-statement:
  `on' expression variable-declaration-statement
\end{syntax}

\begin{syntax}
on-statement:
  `on' expression `do' statement
  `on' expression block-statement
\end{syntax}

\begin{syntax}
cobegin-statement:
  `cobegin' task-intent-clause[OPT] block-statement
\end{syntax}

\begin{syntax}
coforall-statement:
  `coforall' index-var-declaration `in' iteratable-expression task-intent-clause[OPT] `do' statement
  `coforall' index-var-declaration `in' iteratable-expression task-intent-clause[OPT] block-statement
  `coforall' iteratable-expression task-intent-clause[OPT] `do' statement
  `coforall' iteratable-expression task-intent-clause[OPT] block-statement
\end{syntax}

\begin{syntax}
begin-statement:
  `begin' task-intent-clause[OPT] statement
\end{syntax}

\begin{syntax}
sync-statement:
  `sync' statement
  `sync' block-statement
\end{syntax}

\begin{syntax}
serial-statement:
  `serial' expression[OPT] `do' statement
  `serial' expression[OPT] block-statement
\end{syntax}

\begin{syntax}
atomic-statement:
  `atomic' statement
\end{syntax}

\begin{syntax}
forall-statement:
  `forall' index-var-declaration `in' iteratable-expression task-intent-clause[OPT] `do' statement
  `forall' index-var-declaration `in' iteratable-expression task-intent-clause[OPT] block-statement
  `forall' iteratable-expression task-intent-clause[OPT] `do' statement
  `forall' iteratable-expression task-intent-clause[OPT] block-statement
  [ index-var-declaration `in' iteratable-expression task-intent-clause[OPT] ] statement
  [ iteratable-expression task-intent-clause[OPT] ] statement
\end{syntax}

\begin{syntax}
delete-statement:
  `delete' expression ;
\end{syntax}


\cleardoublepage
\documentclass[10pt,twoside,titlepage]{article}
\usepackage{color}
\usepackage{times}
\usepackage{fullpage}
\usepackage{graphicx}
\usepackage{listings}
\usepackage{longtable}
\usepackage[nottoc]{tocbibind}
\input{chapel_listing}
\input{syntax_listing}

%% High section numbers require different number widths
\usepackage[titles]{tocloft}
\usepackage{ifpdf}
\ifpdf
\usepackage[pdftex,
            bookmarks,
            plainpages=false,
            breaklinks,
            pdftitle={Chapel Language Specification},
            pdfauthor={Cray Inc, 901 Fifth Avenue Suite 1000, Seattle, WA 98164},
            pdfsubject={Chapel High Productivity Language}
           ]{hyperref}
\else
\usepackage[ps2pdf]{hyperref}
\fi
\setlength{\cftsecnumwidth}{1.7em}
\setlength{\cftsubsecnumwidth}{2.6em}
\setlength{\cftsubsubsecnumwidth}{3.4em}
\setlength{\cftsubsecindent}{1.7em}
\setlength{\cftsubsubsecindent}{4.3em}

\newcommand{\ie}{\emph{i.e.}}
\newcommand{\eg}{\emph{e.g.}}

\newenvironment{TODO} {
\begin{quote}
{\it TODO:}
}{
\end{quote}
}

\newenvironment{example}{
\begin{quote}
{\it Example}.
}{
\end{quote}
}

\newenvironment{note}{
\begin{quote}
{\it Implementors' note}.
}{
\end{quote}
}

\newenvironment{rationale}{
\begin{quote}
{\it Rationale}.
}{
\end{quote}
}

\newenvironment{openissue}{
\begin{quote}
{\it Open issue}.
}{
\end{quote}
}

\newenvironment{craychapel}{
\begin{quote}
{\it Cray's Chapel Implementation}.
}{
\end{quote}
}

\newenvironment{suggestionbox}{
\begin{quote}
{\it Suggestions?}
}{
\end{quote}
}

\newcommand{\rsec}[1]
           {\S\ref{#1}}

% courtesy: http://www.iam.ubc.ca/~newbury/tex/page-set-up.html
\newcommand{\sekshun}[1]
           {
             \section{#1}
             \markboth{Chapel Language Specification}{#1}
           }

\oddsidemargin 0.0in
\evensidemargin 0.5in
\textwidth 6in
\headheight 0.2in
\topmargin 0in
\headsep 0.3in
\textheight 8.5in

\makeindex
\title{Chapel Language Specification 0.782}

\author{Cray Inc\\
901 Fifth Avenue, Suite 1000\\
Seattle, WA 98164}

\date{}

\setcounter{tocdepth}{3}

\begin{document}

\pagestyle{empty}
\pagenumbering{alph}

\ifpdf
\pdfbookmark[1]{Title}{titlepage}
\fi
\maketitle

\cleardoublepage
\include{tm}
\cleardoublepage

\pagestyle{myheadings}
\markboth{Chapel Language Specification}{Chapel Language Specification}
\pagenumbering{roman}

\ifpdf
\pdfbookmark[1]{Table of Contents}{tablecontents}
\fi
\tableofcontents

\cleardoublepage

\pagestyle{myheadings}
\pagenumbering{arabic}

\setlength{\parindent}{0in}
\setlength{\parskip}{4mm plus2mm minus1mm}

\input{Scope}
\cleardoublepage
\input{Notation}
\cleardoublepage
\input{Organization}
\cleardoublepage
\input{Acknowledgments}
\cleardoublepage
\input{Language_Overview}
\cleardoublepage
\input{Lexical_Structure}
\cleardoublepage
\input{Types}
\cleardoublepage
\input{Variables}
\cleardoublepage
\input{Conversions}
\cleardoublepage
\input{Expressions}
\cleardoublepage
\input{Statements}
\cleardoublepage
\input{Modules}
\cleardoublepage
\input{Functions}
\cleardoublepage
\input{Classes}
\cleardoublepage
\input{Records}
\cleardoublepage
\input{Unions}
\cleardoublepage
\input{Tuples}
\cleardoublepage
\input{Ranges}
\cleardoublepage
\input{Domains_and_Arrays}
\cleardoublepage
\input{Iterators}
\cleardoublepage
\input{Generics}
\cleardoublepage
\input{Parallelism_and_Synchronization}
\cleardoublepage
\input{Locality_and_Distribution}
\cleardoublepage
\input{Reductions_and_Scans}
\cleardoublepage
\input{Input_and_Output}
\cleardoublepage
\input{Standard_Modules}
\cleardoublepage
\appendix
\input{Syntax}
\cleardoublepage
\input{spec.ind}

\end{document}


\end{document}


\end{document}

\end{document}
