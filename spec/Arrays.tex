\sekshun{Arrays}
\label{Arrays}
\index{arrays}

An \emph{array} is a map from a domain's indices to a collection of
variables of homogenous type.  Since Chapel domains support a rich
variety of index sets, Chapel arrays are also richer than the
traditional linear or rectilinear array types in conventional
languages.  Like domains, arrays may be distributed across multiple
locales without explicitly partitioning them using Chapel's Domain
Maps~(\rsec{Domain_Maps}).


\subsection{Array Types}
\label{Array_Types}
\index{arrays!types}

An array type is specified by the identity of the domain that it is
declared over and the element type of the array.  Array types are
given by the following syntax:

\begin{syntax}
array-type:
  [ domain-expression ] type-specifier

domain-expression:
  domain-literal
  expression
\end{syntax}
The \sntx{domain-expression} must specify a domain that the array can
be declared over.  This can be a domain literal.  If it is a domain
literal, the duplicate square brackets around the domain literal can
be omitted.

\begin{example}
In the code
\begin{chapel}
const D: domain(2) = [1..10, 1..10];
var A: [D] real;
\end{chapel}
\chpl{A} is declared to be an arithmetic array over arithmetic
domain \chpl{D} with elements of type \chpl{real}.  As a result, it
represents a 2-dimensional $10 \times 10$ real floating point
variables indexed using the indices $(1, 1), (1, 2), \ldots, (1, 10),
(2, 1), \ldots, (10, 10)$.
\end{example}

%
% should the following be moved elsewhere?  Should handle these
% param/type queries consistently between this chapter and domains
% (and ranges?)
%

An array's element type can be referred to using the member symbol
\chpl{eltType}.

\begin{example}
In the following example, \chpl{x} is declared to be of type
\chpl{real} since that is the element type of array \chpl{A}.
\begin{chapel}
var A: [D] real;
var x: A.eltType;
\end{chapel}
\end{example}

\subsection{Array Values}
\label{Array_Values}
\index{arrays!values}
\index{arrays!initialization}

An array's value is the collection of its elements' values.
Assignments between array variables are performed by value as
described in~\rsec{Array_Assignment}.  Chapel semantics are defined so
that the compiler will never need to insert temporary arrays of the
same size as a user array variable.

\index{arrays!literals, lack thereof}

Arrays do not have a literal format in Chapel, but arithmetic array
variables can be initialized using tuple values.  These tuples must
match the size and shape of the array itself.

\begin{example}
The following example declares a $2 \times 3$ array \chpl{A} using an
anonymous domain value and initializes the elements of the array using
a 2-tuple of 3-tuples which matches the array's size and shape.
\begin{chapel}
var A: [1..2, 1..3] real = ((1.1, 1.2, 1.3), (2.1, 2.2, 2.3));
\end{chapel}
\end{example}

\index{arrays!default values}

An array's default value is to have its elements all initialized to
the default values for their types.

\subsubsection{Runtime Representation of Array Values}
\label{Array_Runtime_Representation}
\index{arrays!runtime representation}
\index{arrays!domain maps}

The runtime representation of an array in memory is controlled by its
domain's domain map.  Through this mechanism, users can reason about
and control the runtime representation of an array's elements.  See
~\rsec{Domain_Maps} for more details.


\subsection{Array Indexing}
\label{Array_Indexing}
\index{arrays!indexing}
\index{indexing!arrays}

Arrays can be indexed using index values from the domain over which
they are declared.  Array indexing is expressed using either
parenthesis or square brackets.  This results in a reference to the
element that corresponds to the index value.

% NEED SYNTAX DIAGRAM HERE?

\begin{example}
Given:
\begin{chapel}
var A: [1..10] real;
\end{chapel}

the first two elements of A can be assigned the value 1.2 and 3.4
respectively using the assignment:
\begin{chapel}
A(1) = 1.2;
A[2] = 3.4;
\end{chapel}
\end{example}

If an array is indexed using an index that is not part of its domain's
index set, the reference is considered out-of-bounds and a runtime
error will occur, halting the program.


\subsubsection{Arithmetic Array Indexing}
\label{Arithmetic_Array_Indexing}
\index{indexing!arithmetic arrays}
\index{arithmetic arrays!indexing}

Since the indices for multidimensional arithmetic domains are tuples,
for convenience, arithmetic arrays can be indexed using the list of
integer values that make up the tuple index.  This is semantically
equivalent to creating a tuple value out of the integer values and
using that tuple value to index the array.  For symmetry,
1-dimensional arithmetic arrays can be accessed using 1-tuple indices
even though their index type is an integral value.  This is
semantically equivalent to de-tupling the integral value from the
1-tuple and using it to index the array.

\begin{example}
Given:
\begin{chapel}
var A: [1..5, 1..5] real;
var ij: 2*int = (1, 1);
\end{chapel}
the elements of array A can be indexed using any of the following
idioms:
\begin{chapel}
A(ij) = 1.1;
A((1, 2)) = 1.2;
A(1, 3) = 1.3;
A[ij] = -1.1;
A[(1, 4)] = 1.4;
A[1, 5] = 1.5;
\end{chapel}
\end{example}

\begin{example}
The code
\begin{chapel}
def f(A: [], is...)
  return A(is);
\end{chapel}
defines a function that takes an array as the first argument and a
variable-length argument list.  It then indexes into the array using
the tuple that captures the actual arguments.  This function works
even for one-dimensional arrays because one-dimensional arrays can be
indexed into by 1-tuples.
\end{example}


\subsection{Iteration over Arrays}
\label{Iteration_over_Arrays}
\index{arrays!iteration}
\index{iteration!over arrays}

All arrays support iteration via for, forall and coforall loops.
These loops iterate over all of the array elements as described by its
domain.  A loop of the form:

\begin{chapel}
[co]for[all] a in A
  ...a...
\end{chapel}

is semantically equivalent to:

\begin{chapel}
[co]for[all] i in A.domain
  ...A(i)...
\end{chapel}

Thus, the iterator variable for an array traversal is a reference to
the array element type.


\subsection{Array Assignment}
\label{Array_Assignment}
\index{arrays!assignment}

Array assignment is by value.  Arrays can be assigned arrays, ranges,
domains, iterators, or tuples.  If \chpl{A} is an lvalue of array type
and \chpl{B} is an expression of either array, range, or domain type,
or an iterator, then the assignment
\begin{chapel}
A = B;
\end{chapel}
is equivalent to
\begin{chapel}
forall (a,b) in (A,B) do
  a = b;
\end{chapel}
If the zipper iteration is illegal, then the assignment is illegal.
Notice that the assignment is implemented with the semantics of
a \chpl{forall} loop.

Arrays can be assigned tuples of values of their element type if the
tuple contains the same number of elements as the array.  For
multidimensional arrays, the tuple must be a nested tuple such that
the nesting depth is equal to the rank of the array and the shape of
this nested tuple must match the shape of the array.  The values are
assigned element-wise.

% Is the above true for unordered array types?  Should it be?

Arrays can also be assigned single values of their element type.  In
this case, each element in the array is assigned this value.
If \chpl{e} is an expression of the element type of the array or a
type that can be implicitly converted to the element type of the
array, then the assignment
\begin{chapel}
A = e;
\end{chapel}
is equivalent to
\begin{chapel}
forall a in A do
  a = e;
\end{chapel}


\subsection{Array Slicing}
\label{Array_Slicing}
\index{arrays!slicing}
\index{slicing!arrays}

An array can be sliced using a domain that has the same type as the
domain over which it was declared.  The result of an array slice is an
alias to the subset of the array elements from the original array
corresponding to the slicing domain's index set.
 
\begin{example}
Given the definitions
\begin{chapel}
var OuterD: domain(2) = [0..n+1, 0..n+1];
var InnerD: domain(2) = [1..n, 1..n];
var A, B: [OuterD] real;
\end{chapel}
the assignment given by
\begin{chapel}
A[InnerD] = B[InnerD];
\end{chapel}
assigns the elements in the interior of \chpl{B} to the elements in
the interior of \chpl{A}.
\end{example}

\subsubsection{Arithmetic Array Slicing}
\label{Arithmetic_Array_Slicing}
\index{arrays!slice}

An arithmetic array can be sliced by any arithmetic domain that is a
subdomain of the array's defining domain.  If the subdomain
relationship is not met, an out-of-bounds error will occur.  The
result is a subarray whose indices are those of the slicing domain and
whose elements are an alias of the original array's.

Arithmetic arrays also support slicing by ranges directly.  If each
dimension is indexed by a range, this is equivalent to slicing the
array by the arithmetic domain defined by those ranges.  These
range-based slices may also be expressed using partially unbounded or
completely unbounded ranges.  This is equivalent to slicing the
array's defining domain by the specified ranges to create a subdomain
as described in~\rsec{Slicing} and then using that subdomain to slice
the array.

For multidimensional arithmetic arrays, slicing with a rank change is
supported by substituting integral values within a dimension's range
for an actual range.  The resulting array will have a rank less than
the arithmetic array's rank and equal to the number of ranges that are
passed in to take the slice.

\begin{example}
Given an array
\begin{chapel}
var A: [1..n, 1..n] int;
\end{chapel}
the slice \chpl{A[1..n, 1]} is a one-dimensional array whose elements
are the first column of \chpl{A}.
\end{example}



\subsection{Array Arguments to Functions}
\label{Array_Arguments_To_Functions}
\index{arrays!function arguments}

Arrays are passed to functions by reference.  Formal arguments that
receive arrays are aliases of the actual arguments.

% Do we really need to say this?  Should it be said here -- seems like
% there is no normal rule and that the cases should be described in
% the function intents section.
%
%  The ordinary rule
%that disallows assignment to formal arguments of blank intent does not
%apply to arrays.

When a formal argument has array type, the element type of the array
can be omitted and/or the domain of the array can be queried or
omitted.  In such cases, the argument is generic and is discussed
in~\rsec{Formal_Arguments_of_Generic_Array_Types}.

If a non-queried domain is specified in the array type of a formal
argument, the domain must match the domain of the actual argument.
This is verified at runtime.  There is an exception if the domain is
an arithmetic domain, described
in~\rsec{Formal_Arguments_of_Arithmetic_Array_Type}.


\subsubsection{Formal Arguments of Arithmetic Array Type}
\label{Formal_Arguments_of_Arithmetic_Array_Type}
\index{formal arguments!arithmetic arrays}
\index{arrays!as formal arguments}

Formal arguments of arithmetic array type allow an arithmetic domain
to be specified that does not match the arithmetic domain of the
actual arithmetic array that is passed to the formal argument.  In
this case, the shape (size in each dimension and rank) of the domain
of the actual array must match the shape of the domain of the formal
array.  The indices are translated in the formal array, which is a
reference to the actual array.

\begin{example}
In the code
\begin{chapel}
def foo(X: [1..5] int) { ... }
var A: [1..10 by 2] int;
foo(A);
\end{chapel}
the array \chpl{A} is strided and its elements can be indexed by the
odd integers between one and nine.  In the function \chpl{foo}, the
array \chpl{X} references array \chpl{A} and the same elements can be
indexed by the integers between one and five.
\end{example}


\subsubsection{Array Promotion of Scalar Functions}
\label{Array_Promotion_of_Scalar_Functions}
\index{arrays!promotion}

Array promotion of a scalar function is defined over the array type
and the element type of the array.  The domain of the returned array,
if an array is captured by the promotion, is the domain of the array
that promoted the function.  In the event of zipper promotion over
multiple arrays, the promoted function returns an array with a domain
that is equal to the domain of the first argument to the function that
enables promotion.  If the first argument is an iterator or a range,
the result is a one-based one-dimensional array.

\begin{example}
Whole array operations is a special case of array promotion of scalar
functions.  In the code
\begin{chapel}
A = B + C;
\end{chapel}
if \chpl{A}, \chpl{B}, and \chpl{C} are arrays, this code assigns each
element in \chpl{A} the element-wise sum of the elements in \chpl{B}
and \chpl{C}.
\end{example}

%
% TODO: should have an example of promoting an actual function here
%


\subsubsection{Array Aliases}
\label{Array_Aliases}
\index{=>@\chpl{=>}}

Array slices alias the data in arrays rather than copying it.  Such
array aliases can be captured and optionally reindexed with the array
alias operator \chpl{=>}.  The syntax for capturing an alias to an
array requires a new variable declaration:
\begin{syntax}
array-alias-declaration:
  identifier reindexing-expression[OPT] => array-expression ;

reindexing-expression:
  [ domain-expression ]

array-expression:
  expression
\end{syntax}
The identifier is an alias to the array specified in
the \sntx{array-expression}.

The optional \sntx{reindexing-expression} allows the domain of the
array alias to be reindexed.  The shape of the domain in
the \sntx{reindexing-expression} must match the shape of the domain of
the \sntx{array-expression}.  Indexing via the alias is governed by
the new indices.

\begin{example}
In the code
\begin{chapel}
var A: [1..5, 1..5] int;
var AA: [0..2, 0..2] => A[2..4, 2..4];
\end{chapel}
an array alias \chpl{AA} is created to alias the interior of
array \chpl{A} given by the slice \chpl{A[2..4, 2..4]}.  The
reindexing expression changes the indices defined by the domain of the
alias to be zero-based in both dimensions.  Thus \chpl{AA(1,1)} is
equivalent to \chpl{A(3,3)}.
\end{example}

%
% TODO: need to insert something about using alias operator in
% constructors as well.  Ran out of time for 1.1 release
%


\subsection{Sparse Arrays}
\label{Sparse_Arrays}
\index{arrays!sparse}

Sparse arrays in Chapel are those whose domain is a sparse array.  A
sparse array differs from other array types in that it stores a single
value corresponding to multiple indices.  This value is commonly
referred to as the \emph{zero value}, but we refer to it as the
\emph{implicitly replicated value} or \emph{IRV} since it can take
on any value of the array's element type in practice including
non-zero numeric values, a class reference, a record or tuple value,
etc.

An array declared over a sparse domain can be indexed using any of the
indices in the sparse domain's parent domain.  If it is read using an
index that is not part of the sparse domain's index set, the IRV value
is returned.  Otherwise, the array element corresponding to the index
is returned.

Sparse arrays can only be written at locations corresponding to
indices in their domain's index set.  In general, writing to other
locations corresponding to the IRV value will result in a runtime
error.

By default a sparse array's IRV is defined as the default value for
the array's element type.  The IRV can be set to any value of the
array's element type by assigning to a pseudo-field named \chpl{IRV}
in the array.

\begin{example}
The following code example declares a sparse array, \chpl{SpsA} using
the sparse domain \chpl{SpsD} (For this example, assume that
\chpl{n}$>$1).  Line~2 assigns two indices to \chpl{SpsD}'s index set
and then lines 3--4 store the values 1.1 and 9.9 to the corresponding
values of \chpl{SpsA}.  The IRV of \chpl{SpsA} will initially be 0.0
since its element type is \chpl{real}.  However, the fifth line sets
the IRV to be the value 5.5, causing \chpl{SpsA} to represent the
value 1.1 in its low corner, 9.9 in its high corner, and 5.5
everywhere else.  The final statement is an error since it attempts to
assign to \chpl{SpsA} at an index not described by its domain,
\chpl{SpsD}.

\begin{chapel}
var SpsA: [SpsD] real;
SpsD = ((1,1), (n,n));
SpsA(1,1) = 1.1;
SpsA(n,n) = 9.9;
SpsA.IRV = 5.5;
SpsA(1,n) = 0.0;  // ERROR!
\end{chapel}
\end{example}



\subsection{Association of Arrays to Domains}
\label{Association_of_Arrays_to_Domains}
\index{domains!association to arrays}
\index{arrays!association to domains}

%
% Be sure to talk about resetting array values & assigning IRVs
%

When an array is declared, it is linked during execution to the domain
identity over which it was declared.  This linkage is invariant for
the array's lifetime and cannot be changed.

When indices are added or removed from a domain, the change impacts
the arrays declared over this particular domain.  In the case of
adding an index, an element is added to the array and initialized to
the IRV for sparse arrays, and to the default value for the element
type for dense arrays.  In the case of removing an index, the element
in the array is removed.

When a domain is reassigned a new value, its arrays are also impacted.
Values that correspond to indices in the intersection of the old and
new domain are preserved in the arrays.  Values that could only be
indexed by the old domain are lost.  Values that can only be indexed
by the new domain have elements added to the new array, initialized to
the IRV for sparse arrays, and to the element type's default value for
other array types.

For performance reasons, there is an expectation that a method will be
added to domains to allow non-preserving assignment, \emph{i.e.}, all
values in the arrays associated with the assigned domain will be lost.
Today this can be achieved by assigning the array's domain an empty
index set (causing all array elements to be deallocated) and then
re-assigning the new index set to the domain.

An array's domain can only be modified directly, via the domain's name
or an alias created by passing it to a function via blank intent.  In
particular, the domain may not be modified via the array's
\chpl{.domain} method, nor by using the domain query syntax on a
function's formal array
argument~(\rsec{Formal_Arguments_of_Generic_Array_Types}).

\begin{rationale}
When multiple arrays are declared using a single domain, modifying the
domain affects all of the arrays.  Allowing an array's domain to be
queried and then modified suggests that the change should only affect
that array.  By requiring the domain to be modified directly, the user
is encouraged to think in terms of the domain distinctly from a
particular array.

In addition, this choice has the beneficial effect that arrays
declared via an anonymous domain have a constant domain.  Constant
domains are considered a common case and have potential compilation
benefits such as eliminating bounds checks.  Therefore making this
convenient syntax support a common, optimizable case seems prudent.
\end{rationale}


\subsection{Predefined Functions and Methods on Arrays}
\index{arrays!predefined functions}

There is an expectation that this list of predefined functions and
methods will grow.

\begin{protohead}
def $Array$.eltType type
\end{protohead}
\begin{protobody}
Returns the element type of the array.
\end{protobody}

\begin{protohead}
def $Array$.rank param
\end{protohead}
\begin{protobody}
Returns the rank of the array.
\end{protobody}

\begin{protohead}
def $Array$.domain: this.domain
\end{protohead}
\begin{protobody}
Returns the domain of the given array.  This domain is constant,
implying that the domain cannot be resized by assigning to its domain
field, only by modifying the domain directly.
\end{protobody}

\begin{protohead}
def $Array$.numElements: this.domain.dim_type
\end{protohead}
\begin{protobody}
Returns the number of elements in the array.
\end{protobody}

\begin{protohead}
def reshape(A: $Array$, D: $Domain$): $Array$
\end{protohead}
\begin{protobody}
Returns a copy of the array containing the same values but in the
shape of the new domain.  The number of indices in the domain must
equal the number of elements in the array.  The elements of the array
are copied into the new array using the default iteration orders over
both arrays.
\end{protobody}
