\sekshun{Arrays}
\label{Arrays}
\index{arrays}

An \emph{array} is a map from a domain's indices to a collection of
variables of homogeneous type.  Since Chapel domains support a rich
variety of index sets, Chapel arrays are also richer than the
traditional linear or rectilinear array types in conventional
languages.  Like domains, arrays may be distributed across multiple
locales without explicitly partitioning them using Chapel's Domain
Maps~(\rsec{Domain_Maps}).


\section{Array Types}
\label{Array_Types}
\index{arrays!types}

An array type is specified by the identity of the domain that it is
declared over and the element type of the array.  Array types are
given by the following syntax:

\begin{syntax}
\begin{verbatim}
array-type:
  [ domain-expression ] type-expression
\end{verbatim}
\end{syntax}
The \sntx{domain-expression} must specify a domain that the array can
be declared over.  If the \sntx{domain-expression} is a domain
literal, the curly braces around the literal may be omitted.

\begin{chapelexample}{decls.chpl}
In the code
\begin{chapel}
\begin{verbatim}
const D: domain(2) = {1..10, 1..10};
var A: [D] real;
\end{verbatim}
\end{chapel}
\begin{chapelpost}
\begin{verbatim}
writeln(D);
writeln(A);
\end{verbatim}
\end{chapelpost}
\begin{chapeloutput}
\begin{verbatim}
{1..10, 1..10}
0.0 0.0 0.0 0.0 0.0 0.0 0.0 0.0 0.0 0.0
0.0 0.0 0.0 0.0 0.0 0.0 0.0 0.0 0.0 0.0
0.0 0.0 0.0 0.0 0.0 0.0 0.0 0.0 0.0 0.0
0.0 0.0 0.0 0.0 0.0 0.0 0.0 0.0 0.0 0.0
0.0 0.0 0.0 0.0 0.0 0.0 0.0 0.0 0.0 0.0
0.0 0.0 0.0 0.0 0.0 0.0 0.0 0.0 0.0 0.0
0.0 0.0 0.0 0.0 0.0 0.0 0.0 0.0 0.0 0.0
0.0 0.0 0.0 0.0 0.0 0.0 0.0 0.0 0.0 0.0
0.0 0.0 0.0 0.0 0.0 0.0 0.0 0.0 0.0 0.0
0.0 0.0 0.0 0.0 0.0 0.0 0.0 0.0 0.0 0.0
\end{verbatim}
\end{chapeloutput}
\chpl{A} is declared to be an arithmetic array over rectangular
domain \chpl{D} with elements of type \chpl{real}.  As a result, it
represents a 2-dimensional $10 \times 10$ real floating point
variables indexed using the indices
$(1, 1), (1, 2), \ldots, (1, 10), (2, 1), \ldots, (10, 10)$.
\end{chapelexample}

%
% should the following be moved elsewhere?  Should handle these
% param/type queries consistently between this chapter and domains
% (and ranges?)
%
\index{arrays!element type}
An array's element type can be referred to using the member symbol
\chpl{eltType}.

\begin{chapelexample}{eltType.chpl}
In the following example, \chpl{x} is declared to be of type
\chpl{real} since that is the element type of array \chpl{A}.
\begin{chapelpre}
\begin{verbatim}
const D: domain(2) = {1..10, 1..10};
\end{verbatim}
\end{chapelpre}
\begin{chapel}
\begin{verbatim}
var A: [D] real;
var x: A.eltType;
\end{verbatim}
\end{chapel}
\begin{chapelpost}
\begin{verbatim}
writeln(x.type:string);
writeln(A.eltType:string);
\end{verbatim}
\end{chapelpost}
\begin{chapeloutput}
\begin{verbatim}
real(64)
real(64)
\end{verbatim}
\end{chapeloutput}
\end{chapelexample}

\section{Array Values}
\label{Array_Values}
\index{arrays!values}
\index{arrays!initialization}
\index{initialization!arrays}

An array's value is the collection of its elements' values.
Assignments between array variables are performed by value as
described in~\rsec{Array_Assignment}.  Chapel semantics are defined so
that the compiler will never need to insert temporary arrays of the
same size as a user array variable.

\index{arrays!literals}

Array literal values can be either rectangular or associative, corresponding to
the underlying domain which defines its indices. 

\begin{syntax}
\begin{verbatim}
array-literal:
  rectangular-array-literal
  associative-array-literal
\end{verbatim}
\end{syntax}

\subsection{Rectangular Array Literals}
\label{Rectangular_Array_Literals}
\index{rectangular array literals}
\index{arrays!rectangular!literals}

Rectangular array literals are specified by enclosing a comma separated list of 
expressions representing values in square brackets. A 1-based domain will 
automatically be generated for the given array literal.  The type of the array's 
values will be the type of the first element listed. A trailing comma is
allowed.

\begin{syntax}
\begin{verbatim}
rectangular-array-literal:
  [ expression-list ]
  [ expression-list , ]
\end{verbatim}
\end{syntax}

\begin{chapelexample}{adecl-literal.chpl}
The following example declares a 5 element rectangular array literal 
containing strings, then subsequently prints each string element to the console.
\begin{chapel}
\begin{verbatim}
var A = ["1", "2", "3", "4", "5"];

for i in 1..5 do
  writeln(A[i]);
\end{verbatim}
\end{chapel}
\begin{chapeloutput}
\begin{verbatim}
1
2
3
4
5
\end{verbatim}
\end{chapeloutput}
\end{chapelexample}

\begin{future}
Provide syntax which allows users to specify the domain for a rectangular 
array literal.
\end{future}

\begin{future}
Determine the type of a rectangular array literal based on the most promoted 
type, rather than the first element's type.
\end{future}

\begin{chapelexample}{decl-with-anon-domain.chpl}
The following example declares a 2-element array \chpl{A} containing 3-element
arrays of real numbers.  \chpl{A} is initialized using array literals.
\begin{chapel}
\begin{verbatim}
var A: [1..2] [1..3] real = [[1.1, 1.2, 1.3], [2.1, 2.2, 2.3]];
\end{verbatim}
\end{chapel}
\begin{chapelpost}
\begin{verbatim}
writeln(A.domain);
\end{verbatim}
\end{chapelpost}
\begin{chapeloutput}
\begin{verbatim}
{1..2}
\end{verbatim}
\end{chapeloutput}
\end{chapelexample}

\begin{openissue}
We would like to differentiate syntactically between array literals for an array
of arrays and a multi-dimensional array. 
\end{openissue}

\index{arrays!rectangular!default values}
An rectangular array's default value is for each array element to be initialized to
the default value of the element type.

\subsection{Associative Array Literals}
\label{Associative_Array_Literals}
\index{associative array literals}
\index{arrays!associative!literals}

Associative array values are specified by enclosing a comma separated list of
index-to-value bindings within square brackets. It is expected that the indices 
in the listing match in type and, likewise, the types of values in the listing 
also match. A trailing comma is allowed.

\begin{syntax}
\begin{verbatim}
associative-array-literal:
  [ associative-expr-list ]
  [ associative-expr-list , ]

associative-expr-list:
  index-expr => value-expr
  index-expr => value-expr, associative-expr-list

index-expr:
  expression

value-expr:
  expression
\end{verbatim}
\end{syntax}

\begin{openissue}
Currently it is not possible to use other associative domains as values within
an associative array literal.
\end{openissue}

\begin{chapelexample}{adecl-assocLiteral.chpl}
The following example declares a 5 element associative array literal which maps
integers to their corresponding string representation. The indices and their
corresponding values are then printed. 
\begin{chapel}
\begin{verbatim}
var A = [1 => "one", 10 => "ten", 3 => "three", 16 => "sixteen"];

for da in zip (A.domain, A) do
  writeln(da);
\end{verbatim}
\end{chapel}
\begin{chapelprediff}
\begin{verbatim}
#!/usr/bin/env sh
testname=$1
outfile=$2
sort $outfile > $outfile.2
mv $outfile.2 $outfile
\end{verbatim}
\end{chapelprediff}
\begin{chapeloutput}
\begin{verbatim}
(1, one)
(10, ten)
(16, sixteen)
(3, three)
\end{verbatim}
\end{chapeloutput}
\end{chapelexample}

\subsection{Runtime Representation of Array Values}
\label{Array_Runtime_Representation}
\index{arrays!runtime representation}
\index{arrays!domain maps}

The runtime representation of an array in memory is controlled by its
domain's domain map.  Through this mechanism, users can reason about
and control the runtime representation of an array's elements.  See
~\rsec{Domain_Maps} for more details.


\section{Array Indexing}
\label{Array_Indexing}
\index{arrays!indexing}
\index{indexing!arrays}

Arrays can be indexed using index values from the domain over which
they are declared.  Array indexing is expressed using either
parentheses or square brackets.  This results in a reference to the
element that corresponds to the index value.

% NEED SYNTAX DIAGRAM HERE?

\begin{chapelexample}{array-indexing.chpl}
Given:
\begin{chapel}
\begin{verbatim}
var A: [1..10] real;
\end{verbatim}
\end{chapel}
the first two elements of A can be assigned the value 1.2 and 3.4
respectively using the assignment:
\begin{chapel}
\begin{verbatim}
A(1) = 1.2;
A[2] = 3.4;
\end{verbatim}
\end{chapel}
\begin{chapelpost}
\begin{verbatim}
writeln(A.domain);
writeln(A);
\end{verbatim}
\end{chapelpost}
\begin{chapeloutput}
\begin{verbatim}
{1..10}
1.2 3.4 0.0 0.0 0.0 0.0 0.0 0.0 0.0 0.0
\end{verbatim}
\end{chapeloutput}
\end{chapelexample}

Except for associative arrays, if an array is indexed using an index that
is not part of its domain's index set, the reference is considered
out-of-bounds and a runtime error will occur, halting the program.

\subsection{Rectangular Array Indexing}
\label{Rectangular_Array_Indexing}
\index{indexing!rectangular arrays}
\index{rectangular arrays!indexing}

Since the indices for multidimensional rectangular domains are tuples,
for convenience, rectangular arrays can be indexed using the list of
integer values that make up the tuple index.  This is semantically
equivalent to creating a tuple value out of the integer values and
using that tuple value to index the array.  For symmetry,
1-dimensional rectangular arrays can be accessed using 1-tuple indices
even though their index type is an integral value.  This is
semantically equivalent to de-tupling the integral value from the
1-tuple and using it to index the array.

\begin{chapelexample}{array-indexing-2.chpl}
Given:
\begin{chapel}
\begin{verbatim}
var A: [1..5, 1..5] real;
var ij: 2*int = (1, 1);
\end{verbatim}
\end{chapel}
the elements of array A can be indexed using any of the following
idioms:
\begin{chapel}
\begin{verbatim}
A(ij) = 1.1;
A((1, 2)) = 1.2;
A(1, 3) = 1.3;
A[ij] = -1.1;
A[(1, 4)] = 1.4;
A[1, 5] = 1.5;
\end{verbatim}
\end{chapel}
\begin{chapelpost}
\begin{verbatim}
writeln(ij);
writeln(A);
\end{verbatim}
\end{chapelpost}
\begin{chapeloutput}
\begin{verbatim}
(1, 1)
-1.1 1.2 1.3 1.4 1.5
0.0 0.0 0.0 0.0 0.0
0.0 0.0 0.0 0.0 0.0
0.0 0.0 0.0 0.0 0.0
0.0 0.0 0.0 0.0 0.0
\end{verbatim}
\end{chapeloutput}
\end{chapelexample}

\begin{chapelexample}{index-using-var-arg-tuple.chpl}
The code
\begin{chapel}
\begin{verbatim}
proc f(A: [], is...)
  return A(is);
\end{verbatim}
\end{chapel}
\begin{chapelpost}
\begin{verbatim}
var B: [1..5] int;
[i in 1..5] B(i) = i;
var C: [1..5,1..5] int;
[(i,j) in {1..5,1..5}] C(i,j) = i+i*j;
writeln(f(B, 3));
writeln(f(C, 3, 3));
\end{verbatim}
\end{chapelpost}
\begin{chapeloutput}
\begin{verbatim}
3
12
\end{verbatim}
\end{chapeloutput}
defines a function that takes an array as the first argument and a
variable-length argument list.  It then indexes into the array using
the tuple that captures the actual arguments.  This function works
even for one-dimensional arrays because one-dimensional arrays can be
indexed into by 1-tuples.
\end{chapelexample}

\subsection{Associative Array Indexing}
\label{Associative_Array_Indexing}
\index{indexing!associative arrays}
\index{associative arrays!indexing}

Indices can be added to associative arrays through the array's domain.
\begin{chapelexample}{assoc-add-index.chpl}
Given:
\begin{chapel}
\begin{verbatim}
var D : domain(string);
var A : [D] int;
\end{verbatim}
\end{chapel}

the array A initially contains no elements. We can change that by adding
indices to the domain D:
\begin{chapel}
\begin{verbatim}
D.add("a");
D.add("b");
\end{verbatim}
\end{chapel}

The array A can now be indexed with indices "a" and "b":

\begin{chapel}
\begin{verbatim}
A["a"] = 1;
A["b"] = 2;
var x = A["a"];
\end{verbatim}
\end{chapel}
\end{chapelexample}


\section{Iteration over Arrays}
\label{Iteration_over_Arrays}
\index{arrays!iteration}
\index{iteration!array}

% FYI: Similar to text regarding tuple iteration.  Slightly less
% similar for domain iteration.
All arrays support iteration via standard \chpl{for}, \chpl{forall}
and \chpl{coforall} loops.  These loops iterate over all of the array
elements as described by its domain.  A loop of the form:

% This is difficult to capture in a test program
\begin{chapel}
\begin{verbatim}
[for|forall|coforall] a in A do
  ...a...
\end{verbatim}
\end{chapel}

is semantically equivalent to:

% This is difficult to capture in a test program
\begin{chapel}
\begin{verbatim}
[for|forall|coforall] i in A.domain do
  ...A[i]...
\end{verbatim}
\end{chapel}

The iterator variable for an array iteration is a reference to the
array element type.


\pagebreak
\section{Array Assignment}
\label{Array_Assignment}
\index{arrays!assignment}
\index{assignment!array}

Array assignment is by value.  Arrays can be assigned arrays, ranges,
domains, iterators, or tuples as long as the two expressions are
compatible in terms of number of dimensions and shape.

\begin{chapelexample}{assign.chpl}
If \chpl{A} is an array variable and \chpl{B} is an expression
of array, range, domain, or tuple type, or an iterator, then the
assignment
\begin{chapelpre}
\begin{verbatim}
var A: [1..3] int;
var B: [1..3] int;
A = -1;
B = 1;
\end{verbatim}
\end{chapelpre}
\begin{chapelnoprint}
\begin{verbatim}
writeln(A);
writeln(B);
\end{verbatim}
\end{chapelnoprint}
\begin{chapel}
\begin{verbatim}
A = B;
\end{verbatim}
\end{chapel}
\begin{chapelnoprint}
\begin{verbatim}
writeln(A);
writeln(B);
A = -2;
B = 2;
writeln(A);
writeln(B);
\end{verbatim}
\end{chapelnoprint}
is equivalent to
\begin{chapel}
\begin{verbatim}
[(a,b) in zip(A,B)] a = b;
\end{verbatim}
\end{chapel}
\begin{chapelpost}
\begin{verbatim}
writeln(A);
writeln(B);
\end{verbatim}
\end{chapelpost}
\begin{chapeloutput}
\begin{verbatim}
-1 -1 -1
1 1 1
1 1 1
1 1 1
-2 -2 -2
2 2 2
2 2 2
2 2 2
\end{verbatim}
\end{chapeloutput}
If the zipper iteration is illegal, then the assignment is illegal.
This means, for example, that a range cannot be assigned to a
multidimensional rectangular array because the two expressions don't
match in shape and can't be zipped together.  Notice that the
assignment is implemented using parallelism when possible, and
serially otherwise.
\end{chapelexample}

Arrays can be assigned tuples of values of their element type if the
tuple contains the same number of elements as the array.  For
multidimensional arrays, the tuple must be a nested tuple such that
the nesting depth is equal to the rank of the array and the shape of
this nested tuple must match the shape of the array.  The values are
assigned element-wise.

% Is the above true for unordered array types?  Should it be?

Arrays can also be assigned single values of their element type.  In
this case, each element in the array is assigned this value.
\begin{chapelexample}{assign-2.chpl}
If \chpl{e} is an expression of the element type of the array or a
type that can be implicitly converted to the element type of the
array, then the assignment
\begin{chapelpre}
\begin{verbatim}
var A: [1..4] uint;
writeln(A);
var e: uint = 77;
\end{verbatim}
\end{chapelpre}
\begin{chapel}
\begin{verbatim}
A = e;
\end{verbatim}
\end{chapel}
\begin{chapelnoprint}
\begin{verbatim}
writeln(A);
e = 33;
\end{verbatim}
\end{chapelnoprint}
is equivalent to
\begin{chapel}
\begin{verbatim}
forall a in A do
  a = e;
\end{verbatim}
\end{chapel}
\begin{chapelpost}
\begin{verbatim}
writeln(A);
\end{verbatim}
\end{chapelpost}
\begin{chapeloutput}
\begin{verbatim}
0 0 0 0
77 77 77 77
33 33 33 33
\end{verbatim}
\end{chapeloutput}
\end{chapelexample}

\section{Array Slicing}
\label{Array_Slicing}
\index{arrays!slicing}
\index{slicing!array}

An array can be sliced using a domain that has the same type as the
domain over which it was declared.  The result of an array slice is an
alias to the subset of the array elements from the original array
corresponding to the slicing domain's index set.
 
\begin{chapelexample}{slicing.chpl}
Given the definitions
\begin{chapelpre}
\begin{verbatim}
config const n = 2;
\end{verbatim}
\end{chapelpre}
\begin{chapel}
\begin{verbatim}
var OuterD: domain(2) = {0..n+1, 0..n+1};
var InnerD: domain(2) = {1..n, 1..n};
var A, B: [OuterD] real;
\end{verbatim}
\end{chapel}
\begin{chapelnoprint}
\begin{verbatim}
writeln(OuterD);
writeln(InnerD);
B = 1;
\end{verbatim}
\end{chapelnoprint}
the assignment given by
\begin{chapel}
\begin{verbatim}
A[InnerD] = B[InnerD];
\end{verbatim}
\end{chapel}
\begin{chapelpost}
\begin{verbatim}
writeln(A);
writeln(B);
\end{verbatim}
\end{chapelpost}
\begin{chapeloutput}
\begin{verbatim}
{0..3, 0..3}
{1..2, 1..2}
0.0 0.0 0.0 0.0
0.0 1.0 1.0 0.0
0.0 1.0 1.0 0.0
0.0 0.0 0.0 0.0
1.0 1.0 1.0 1.0
1.0 1.0 1.0 1.0
1.0 1.0 1.0 1.0
1.0 1.0 1.0 1.0
\end{verbatim}
\end{chapeloutput}
assigns the elements in the interior of \chpl{B} to the elements in
the interior of \chpl{A}.
\end{chapelexample}

\subsection{Rectangular Array Slicing}
\label{Rectangular_Array_Slicing}
\index{arrays!slicing!rectangular}
\index{slicing!arrays!rectangular}

A rectangular array can be sliced by any rectangular domain that is a
subdomain of the array's defining domain.  If the subdomain
relationship is not met, an out-of-bounds error will occur.  The
result is a subarray whose indices are those of the slicing domain and
whose elements are an alias of the original array's.

Rectangular arrays also support slicing by ranges directly.  If each
dimension is indexed by a range, this is equivalent to slicing the
array by the rectangular domain defined by those ranges.  These
range-based slices may also be expressed using partially unbounded or
completely unbounded ranges.  This is equivalent to slicing the
array's defining domain by the specified ranges to create a subdomain
as described in~\rsec{Array_Slicing} and then using that subdomain to slice
the array.

\subsection{Rectangular Array Slicing with a Rank Change}
\label{Rectangular_Array_Slicing_With_Rank_Change}
\index{arrays!slicing!rectangular!rank change}

For multidimensional rectangular arrays, slicing with a rank change is
supported by substituting integral values within a dimension's range
for an actual range.  The resulting array will have a rank less than
the rectangular array's rank and equal to the number of ranges that are
passed in to take the slice.

\begin{chapelexample}{array-decl.chpl}
Given an array
\begin{chapelpre}
\begin{verbatim}
config const n = 4;
\end{verbatim}
\end{chapelpre}
\begin{chapel}
\begin{verbatim}
var A: [1..n, 1..n] int;
\end{verbatim}
\end{chapel}
\begin{chapelpost}
\begin{verbatim}
writeln(A);
\end{verbatim}
\end{chapelpost}
\begin{chapeloutput}
\begin{verbatim}
0 0 0 0
0 0 0 0
0 0 0 0
0 0 0 0
\end{verbatim}
\end{chapeloutput}
the slice \chpl{A[1..n, 1]} is a one-dimensional array whose elements
are the first column of \chpl{A}.
\end{chapelexample}


\section{Count Operator}
\label{Count_Operator_Arrays}
\index{arrays!count operator}
\index{operators!# (on arrays)@\chpl{\#} (on arrays)}
The \chpl{\#} operator can be applied to dense rectangular arrays with
a tuple argument whose size matches the rank of the array (or
optionally an integer in the case of a 1D array).  The operator is
equivalent to applying the \chpl{\#} operator to the array's domain and
using the result to slice the array as described in
Section~\ref{Rectangular_Array_Slicing}.


\section{Array Arguments to Functions}
\label{Array_Arguments_To_Functions}
\index{arrays!actual arguments}
\index{arguments!array}

By default, arrays are passed to function by \chpl{ref} or \chpl{const ref}
depending on whether or not the formal argument is modified. The
\chpl{in}, \chpl{inout}, and \chpl{out} intent can create copies of
arrays.

When a formal argument has array type, the element type of the array
can be omitted and/or the domain of the array can be queried or
omitted.  In such cases, the argument is generic and is discussed
in~\rsec{Formal_Arguments_of_Generic_Array_Types}.

If a formal array argument specifies a domain as part of its type
signature, the domain of the actual argument must represent the same
index set.  If the formal array's domain was declared using an
explicit domain map, the actual array's domain must use an equivalent
domain map.

\subsection{Array Promotion of Scalar Functions}
\label{Array_Promotion_of_Scalar_Functions}
\index{arrays!promotion}
\index{promotion!arrays}

Arrays may be passed to a scalar function argument whose type
matches the array's element type.  This results in a promotion of the
scalar function as defined in~\rsec{Promotion}.

\begin{chapelexample}{whole-array-ops.chpl}
Whole array operations is a special case of array promotion of scalar
functions.  In the code
\begin{chapelpre}
\begin{verbatim}
var A, B, C: [1..3] real;
A = -1;
B = 2;
C = 3;
\end{verbatim}
\end{chapelpre}
\begin{chapel}
\begin{verbatim}
A = B + C;
\end{verbatim}
\end{chapel}
\begin{chapelpost}
\begin{verbatim}
writeln(A);
\end{verbatim}
\end{chapelpost}
\begin{chapeloutput}
\begin{verbatim}
5.0 5.0 5.0
\end{verbatim}
\end{chapeloutput}
if \chpl{A}, \chpl{B}, and \chpl{C} are arrays, this code assigns each
element in \chpl{A} the element-wise sum of the elements in \chpl{B}
and \chpl{C}.
\end{chapelexample}


\section{Returning Arrays from Functions}
\label{Returning_Arrays_from_Functions}
\index{arrays!returning}
\index{returning!array}

Arrays return by value by default. The \chpl{ref} and \chpl{const ref}
return intents can be used to return a reference to an array.

Similarly to array arguments, the element type and/or domain of an array return
type can be omitted.


\section{Sparse Arrays}
\label{Sparse_Arrays}
\index{arrays!sparse}

Sparse arrays in Chapel are those whose domain is sparse.  A
sparse array differs from other array types in that it stores a single
value corresponding to multiple indices.  This value is commonly
referred to as the \emph{zero value}, but we refer to it as the
\emph{implicitly replicated value} or \emph{IRV} since it can take
on any value of the array's element type in practice including
non-zero numeric values, a class reference, a record or tuple value,
etc.

An array declared over a sparse domain can be indexed using any of the
indices in the sparse domain's parent domain.  If it is read using an
index that is not part of the sparse domain's index set, the IRV value
is returned.  Otherwise, the array element corresponding to the index
is returned.

Sparse arrays can only be written at locations corresponding to
indices in their domain's index set.  In general, writing to other
locations corresponding to the IRV value will result in a runtime
error.

By default a sparse array's IRV is defined as the default value for
the array's element type.  The IRV can be set to any value of the
array's element type by assigning to a pseudo-field named \chpl{IRV}
in the array.

\begin{chapelexample}{sparse-error.chpl}
The following code example declares a sparse array, \chpl{SpsA} using
the sparse domain \chpl{SpsD} (For this example, assume that
\chpl{n}$>$1).  Line~2 assigns two indices to \chpl{SpsD}'s index set
and then lines 3--4 store the values 1.1 and 9.9 to the corresponding
values of \chpl{SpsA}.  The IRV of \chpl{SpsA} will initially be 0.0
since its element type is \chpl{real}.  However, the fifth line sets
the IRV to be the value 5.5, causing \chpl{SpsA} to represent the
value 1.1 in its low corner, 9.9 in its high corner, and 5.5
everywhere else.  The final statement is an error since it attempts to
assign to \chpl{SpsA} at an index not described by its domain,
\chpl{SpsD}.

\begin{chapelpre}
\begin{verbatim}
config const n = 5;
const D = {1..n, 1..n};
\end{verbatim}
\end{chapelpre}
\begin{chapel}
\begin{verbatim}
var SpsD: sparse subdomain(D);
var SpsA: [SpsD] real;
SpsD = ((1,1), (n,n));
SpsA(1,1) = 1.1;
SpsA(n,n) = 9.9;
SpsA.IRV = 5.5;
SpsA(1,n) = 0.0;  // ERROR!
\end{verbatim}
\end{chapel}
\begin{chapeloutput}
\begin{verbatim}
sparse-error.chpl:9: error: halt reached - attempting to assign a 'zero' value in a sparse array: (1, 5)
\end{verbatim}
\end{chapeloutput}
\end{chapelexample}



\section{Association of Arrays to Domains}
\label{Association_of_Arrays_to_Domains}
\index{domains!association with arrays}
\index{arrays!association with domains}

%
% Be sure to talk about resetting array values & assigning IRVs
%

When an array is declared, it is linked during execution to the domain
identity over which it was declared.  This linkage is invariant for
the array's lifetime and cannot be changed.

When indices are added or removed from a domain, the change impacts
the arrays declared over this particular domain.  In the case of
adding an index, an element is added to the array and initialized to
the IRV for sparse arrays, and to the default value for the element
type for dense arrays.  In the case of removing an index, the element
in the array is removed.

When a domain is reassigned a new value, its arrays are also impacted.
Values that correspond to indices in the intersection of the old and
new domain are preserved in the arrays.  Values that could only be
indexed by the old domain are lost.  Values that can only be indexed
by the new domain have elements added to the new array, initialized to
the IRV for sparse arrays, and to the element type's default value for
other array types.

For performance reasons, there is an expectation that a method will be
added to domains to allow non-preserving assignment, \emph{i.e.}, all
values in the arrays associated with the assigned domain will be lost.
Today this can be achieved by assigning the array's domain an empty
index set (causing all array elements to be deallocated) and then
re-assigning the new index set to the domain.

An array's domain can only be modified directly, via the domain's name
or an alias created by passing it to a function via default intent.  In
particular, the domain may not be modified via the array's
\chpl{.domain} method, nor by using the domain query syntax on a
function's formal array
argument~(\rsec{Formal_Arguments_of_Generic_Array_Types}).

\begin{rationale}
When multiple arrays are declared using a single domain, modifying the
domain affects all of the arrays.  Allowing an array's domain to be
queried and then modified suggests that the change should only affect
that array.  By requiring the domain to be modified directly, the user
is encouraged to think in terms of the domain distinctly from a
particular array.

In addition, this choice has the beneficial effect that arrays
declared via an anonymous domain have a constant domain.  Constant
domains are considered a common case and have potential compilation
benefits such as eliminating bounds checks.  Therefore making this
convenient syntax support a common, optimizable case seems prudent.
\end{rationale}


\section{Predefined Functions and Methods on Arrays}
\label{Predefined_Functions_and_Methods_on_Arrays}
\index{arrays!predefined functions}
\index{predefined functions!arrays}
\index{functions!arrays!predefined}

There is an expectation that this list of predefined methods will grow.

\index{arrays!eltType@\chpl{eltType}}
\index{predefined functions!eltType (array)@\chpl{eltType} (array)}
\begin{protohead}
\begin{verbatim}
proc Array.eltType type
\end{verbatim}
\end{protohead}
\begin{protobody}
Returns the element type of the array.
\end{protobody}

\index{arrays!rank@\chpl{rank}}
\index{predefined functions!rank (array)@\chpl{rank} (array)}
\begin{protohead}
\begin{verbatim}
proc Array.rank param
\end{verbatim}
\end{protohead}
\begin{protobody}
Returns the rank of the array.
\end{protobody}

\index{arrays!domain@\chpl{domain}}
\index{predefined functions!domain (array)@\chpl{domain} (array)}
\begin{protohead}
\begin{verbatim}
proc Array.domain: this.domain
\end{verbatim}
\end{protohead}
\begin{protobody}
Returns the domain of the given array.  This domain is constant,
implying that the domain cannot be resized by assigning to its domain
field, only by modifying the domain directly.
\end{protobody}

\index{arrays!numElements@\chpl{numElements}}
\index{predefined functions!numElements (array)@\chpl{numElements} (array)}
\begin{protohead}
\begin{verbatim}
proc Array.numElements: this.domain.dim_type
\end{verbatim}
\end{protohead}
\begin{protobody}
Returns the number of elements in the array.
\end{protobody}

\index{arrays!reshape@\chpl{reshape}}
\index{predefined functions!reshape (array)@\chpl{reshape} (array)}
\begin{protohead}
\begin{verbatim}
proc reshape(A: Array, D: Domain): Array
\end{verbatim}
\end{protohead}
\begin{protobody}
Returns a copy of the array containing the same values but in the
shape of the new domain.  The number of indices in the domain must
equal the number of elements in the array.  The elements of the array
are copied into the new array using the default iteration orders over
both arrays.
\end{protobody}

\index{arrays!size@\chpl{size}}
\index{predefined functions!size (array)@\chpl{size} (array)}
\begin{protohead}
\begin{verbatim}
proc Array.size: this.domain.dim_type
\end{verbatim}
\end{protohead}
\begin{protobody}
Same as $Array$.numElements.
\end{protobody}
