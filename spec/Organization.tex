\sekshun{Organization}
\label{Organization}

This specification is organized as follows:
\begin{itemize}

\item
Chapter~\ref{Scope}, Scope, describes the scope of this specification.

\item
Chapter~\ref{Notation}, Notation, introduces the notation that is used
throughout this specification.

\item
Chapter~\ref{Organization}, Organization, describes the contents of
each of the chapters within this specification.

\item
Chapter~\ref{Acknowledgments}, Acknowledgments, offers a note of
thanks to people and projects.

\item
Chapter~\ref{Language_Overview}, Language Overview, describes Chapel
at a high level.

\item
Chapter~\ref{Lexical_Structure}, Lexical Structure, describes the
lexical components of Chapel.

\item
Chapter~\ref{Types}, Types, describes the types in Chapel and defines
the primitive and enumerated types.

\item
Chapter~\ref{Variables}, Variables, describes variables and constants
in Chapel.

\item
Chapter~\ref{Conversions}, Conversions, describes the legal implicit
and explicit conversions allowed between values of different types.
Chapel does not allow for user-defined conversions.

\item
Chapter~\ref{Expressions}, Expressions, describes the non-parallel
expressions in Chapel.

\item
Chapter~\ref{Statements}, Statements, describes the non-parallel
statements in Chapel.

\item
Chapter~\ref{Modules}, Modules, describes modules in Chapel., Chapel
modules allow for name space management.

\item
Chapter~\ref{Functions}, Functions, describes functions and function
resolution in Chapel.

\item
Chapter~\ref{Tuples}, Tuples, describes tuples in Chapel.

\item
Chapter~\ref{Classes}, Classes, describes reference classes in Chapel.

\item
Chapter~\ref{Records}, Records, describes records or value classes in
Chapel.

\item
Chapter~\ref{Unions}, Unions, describes unions in Chapel.

\item
Chapter~\ref{Ranges}, Ranges, describes ranges in Chapel.

\item
Chapter~\ref{Domains}, Domains, describes domains in Chapel.  Chapel
domains are first-class index sets that support the description of
iteration spaces, array sizes and shapes, and sets of indices.

\item
Chapter~\ref{Arrays}, Arrays, describes arrays in Chapel.  Chapel arrays are
more general than in most languages including support for
multidimensional, sparse, associative, and unstructured arrays.

\item
Chapter~\ref{Iterators}, Iterators, describes iterator functions.

\item
Chapter~\ref{Generics}, Generics, describes Chapel's support for
generic functions and types.

\item
Chapter~\ref{Input_and_Output}, Input and Output, describes support
for input and output in Chapel, including file input and output..

\item
Chapter~\ref{Task_Parallelism_and_Synchronization}, Task Parallelism
and Synchronization, describes task-parallel expressions and
statements in Chapel as well as synchronization constructs and the atomic
statement.

\item
Chapter~\ref{Data_Parallelism}, Data Parallelism, describes
data-parallel expressions and statements in Chapel including
reductions and scans, whole array assignment, and promotion.

\item
Chapter~\ref{Locales}, Locales, describes constructs for managing
locality and executing Chapel programs on distributed-memory systems.

\item
Chapter~\ref{Domain_Maps}, Domain Maps, describes
Chapel's \emph{domain map} construct for defining the layout of
domains and arrays within a single locale and/or the distribution of
domains and arrays across multiple locales.

\item
Chapter~\ref{User_Defined_Reductions_and_Scans}, User-Defined
Reductions and Scans, describes how Chapel programmers can define
their own reduction and scan operators.

\item
Chapter~\ref{User_Defined_Domain_Maps}, User-Defined Domain Maps,
describes how Chapel programmers can define their own domain maps to
implement domains and arrays.

\item
  Chapter~\ref{Memory_Consistency_Model}, Memory Consistency Model,
  describes Chapel's rules for ordering the reads and writes performed
  by a program's tasks.

\item
Chapter~\ref{Interoperability} describes Chapel's interoperability
features for combining Chapel programs with code written in different
languages.

\item
Chapter~\ref{Standard_Modules}, Standard Modules, describes the
standard modules that are provided with the Chapel language.

\item
Chapter~\ref{Standard_Distributions}, Standard Distributions,
describes the standard distributions (multi-locale domain maps) that
are provided with the Chapel language.

\item
Chapter~\ref{Standard_Layouts}, Standard Layouts, describes the
standard layouts (single locale domain maps) that are provided with
the Chapel language.

\item
Appendix~\ref{Syntax}, Collected Lexical and Syntax Productions,
contains the syntax productions listed throughout this specification
in both alphabetical and depth-first order.

\end{itemize}
