\sekshun{Input and Output}
\label{Input_and_Output}

Chapel provides a built-in \chpl{file} type to handle input and output
to files using functions and methods called \chpl{read}, \chpl{readln}, 
\chpl{write}, and \chpl{writeln}.

\subsection{The {\em file} type}
\index{file type}

The file type contains the following fields:
\begin{itemize}
\item
The \chpl{filename} field is a \chpl{string} that contains the name of
the file.
\item
The \chpl{path} field is a \chpl{string} that contains the path of the
file.
\item
The \chpl{mode} field is a \chpl{string} that indicates whether the
file is being read or written.
\item
The \chpl{style} field can be set to \chpl{text} or \chpl{binary} to
specify that reading from or writing to the file should be done with
text or binary formats.
\end{itemize}
These fields can be modified any time that the file is closed.

The \chpl{mode} field supports the following strings:
\begin{itemize}
\item
\chpl{"r"} The file can be read.
\item
\chpl{"w"} The file can be written.
\end{itemize}

The file type supports the following methods:
\index{file type!methods}
\begin{itemize}
\item
The \chpl{open()} method opens the file for reading and/or writing.
\item
The \chpl{close()} method closes the file for reading and/or writing.
\item
The \chpl{isOpen} method returns true if the file is open for reading
and/or writing, and otherwise returns false.
\item
The \chpl{flush()} method flushes the file, finishing outstanding
reading and writing.
\end{itemize}

Additionally, the file type supports the
methods \chpl{read}, \chpl{readln}, \chpl{write}, and \chpl{writeln} for 
input and output as discussed in~\rsec{filewrite} and~\rsec{fileread}.

\subsection{Standard files {\em stdout}, {\em stdin}, and {\em stderr}}
\index{file type!standard files}

The files \chpl{stdout}, \chpl{stdin}, and \chpl{stderr} are
predefined and map to standard output, standard input, and standard
error as implemented in a platform dependent fashion.

\subsection{The {\em write}, {\em writeln}, {\em read}, and {\em readln} 
functions}
\index{writeln@\chpl{writeln}}
\index{write@\chpl{write}}
\index{read@\chpl{read}}
\index{readln@\chpl{readln}}
\index{read}
\index{write}

The built-in function \chpl{write} can take an arbitrary number of
arguments and writes each of the arguments out in turn
to \chpl{stdout}.  The built-in function \chpl{writeln} has the same
semantics as \chpl{write} but outputs an {\em end-of-line} character
after writing out the arguments.  The built-in function \chpl{read}
can take an arbitrary number of arguments and reads each of the
arguments in turn from \chpl{stdin}.  The built-in function \chpl{readln}
also takes an arbitrary number of arguments, reading each argument from
\chpl{stdin}.  These arguments may be entered on a single line or on multiple
lines.  After all arguments of the \chpl{readln} call are read, an 
end-of-line character is expected to be read, ignoring any additional 
input between the last argument read and the end-of-line character.

These functions are wrappers for the methods on files described next.

\begin{example}
The \chpl{writeln} wrapper function allows for a simple implementation
of the {\em Hello-World} program:
\begin{chapel}
writeln("Hello, World!");
\end{chapel}
\end{example}

\subsection{User-Defined {\em writeThis} methods}

To define the output for a given type, the user must define a method
called \chpl{writeThis} on that type that takes a single argument of
\chpl{Writer} type.  If such a method does not exist, a default method is
created.

\subsection{The {\em write} and {\em writeln} method on files}
\label{filewrite}
\index{write!on files}

The \chpl{file} type supports methods \chpl{write} and \chpl{writeln}
for output.  These methods are defined to take an arbitrary number of
arguments.  Each argument is written in turn by calling
the \chpl{writeThis} method on that argument.
Default \chpl{writeThis} methods are bound to any type that the user
does not explicitly create one for.

A lock is used to ensure that output is serialized across multiple
computations.

\subsubsection{The {\em write} and {\em writeln} method on strings}
\label{stringwrite}
\index{write!on strings}

The \chpl{write} and \chpl{writeln} methods can also be called on
strings to write the output to a string instead of a file.

\subsubsection{Generalized {\em write} and {\em writeln}}
\label{writer}
\index{Writer@\chpl{Writer}}

The \chpl{Writer} class contains no arguments and serves as a base
class to allow user-defined classes to be written to.  If a class is
defined to be a subclass of Writer, it must override
the \chpl{writeIt} method that takes a \chpl{string} as an argument.

\begin{example}
The following code defines a subclass of \chpl{Writer} that overrides
the \chpl{writeIt} method to allow it to be written to.  It also
overrides the \chpl{writeThis} method to override the default way that
it is written.
\begin{chapel}
class C: Writer {
  var data: string;
  def writeIt(s: string) {
    data += s.substring(1);
  }
  def writeThis(x: Writer) {
    x.write(data);
  }
}

var c = C();
c.write(41, 32, 23, 14);
writeln(c);
\end{chapel}
The \chpl{C} class filters the arguments sent to it, printing out only
the first letter.  The output to the above is thus \chpl{4321}.
\end{example}


\subsection{The {\em read} and {\em readln} methods on files}
\label{fileread}
\index{read!on files}

The \chpl{file} type supports \chpl{read} and \chpl{readln} methods.
The \chpl{read} method takes an arbitrary number of arguments, reading
in each argument from file.  The \chpl{readln} method also
takes an arbitrary number of arguments, reading in each argument
from a single line or multiple lines in the file and 
advancing the file pointer to the next line after the last argument 
is read.

The \chpl{file} type also supports overloaded methods \chpl{read}
and \chpl{readln} that have a single generic type argument.  These methods
read a single value of the specified type from the file and returns it.

\begin{example}
The following line of code reads a value of type \chpl{int} from
\chpl{stdin} and uses it to initialize variable \chpl{x} (causing
\chpl{x} to have an inferred type of \chpl{int}):
\begin{chapel}
var x = stdin.read(int);
\end{chapel}
\end{example}


\subsection{Default {\em read} and {\em write} methods}
\index{write!default methods}
\index{read!default methods}

Default \chpl{write} methods are created for all types for which a user
\chpl{write} method is not defined.  They have the following semantics:
\begin{itemize}
\item
{\bf arrays} Outputs the elements of the array in row-major order
where rows are separated by line-feeds and blank lines are used to
separate other dimensions.
\item
{\bf domains} Outputs the dimensions of the domain enclosed
by \chpl{[} and \chpl{]}.
\item
{\bf ranges} Outputs the lower bound of the range followed
by \chpl{..} followed by the upper bound of the range.  If the stride
of the range is not one, the output is additionally followed by the
word \chpl{by} followed by the stride of the range.
\item
{\bf tuples} Outputs the components of the tuple in order delimited
by \chpl{(} and \chpl{)}, and separated by commas.
\item
{\bf classes} Outputs the values within the fields of the class
prefixed by the name of the field and the character \chpl{=}.  Each
field is separated by a comma.  The output is delimited by \chpl{\{}
and \chpl{\}}.
\item
{\bf records} Outputs the values within the fields of the class
prefixed by the name of the field and the character \chpl{=}.  Each
field is separated by a comma.  The output is delimited by \chpl{(}
and \chpl{)}.
\end{itemize}

Default \chpl{read} methods are created for all types for which a user
\chpl{read} method is not defined.  The default \chpl{read} methods are
defined to read in the output of the default \chpl{write} method.
