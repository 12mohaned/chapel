\sekshun{Language Overview}
\label{Language_Overview}

\subsection{Motivating Principles}

\subsection{Getting Started}

\subsection{Walk-Through: Jacobi}

\subsection{Walk-Through: Norm}

\subsection{Walk-Through: Producer-Consumer}

\begin{numberedchapel}
use Time;

config var numIterations: int = 5;
config var sleepTime: uint = 2;

var s: sync int;

begin {  // fork consumer computation
  for c in 1..numIterations do
    writeln("consumer got ", s);
}

// producer computation
for p in 1..numIterations {
  sleep(sleepTime);
  s = p;
}
\end{numberedchapel}

\subsection{Walk-Through: Generic Stacks}

\subsubsection{Generic Stack with Linked List Implementation}

\begin{numberedchapel}
// A class that is used by the generic stack to implement nodes for a
// linked-list implementation.
class MyNode {
  type itemType;              // type of item
  var item: itemType;         // item in node
  var next: MyNode(itemType); // reference to next node (same type)
}

// A stack class that is generic over the type of data that it
// contains.  The implementation uses a linked list implemented with
// the node class above.
record Stack {
  type itemType;             // type of items
  var top: MyNode(itemType); // top node on stack linked list

  // push method: add an item to the top of the stack
  def push(item: itemType) {
    top = MyNode(itemType, item, top);
  }

  // pop method: remove an item from the top of the stack
  // note: it is a runtime error if the stack is empty
  def pop() {
    if isEmpty? then
      halt("attempt to pop an item off an empty stack");
    var oldTop = top;
    top = top.next;
    return oldTop.item;
  }

  // isEmpty? method: true if the stack is empty; otherwise false
  def isEmpty? return top == nil;
}
\end{numberedchapel}

\subsubsection{Generic Stack with Array Implementation}

\begin{numberedchapel}
// A stack class that is generic over the type of data that it
// contains.  The implementation uses an array to store the elements
// in the stack.
record Stack {
  type itemType;            // type of items
  var numItems: int = 0;    // number of items in the stack
  var data: [1..2] itemType; // array of items

  // push method: add an item to the top of the stack
  // note: the array is doubled if it is full
  def push(item: itemType) {
    var height = data.numElements;
    if numItems == height then
      data.domain = [1..height*2];
    data(numItems+1) = item;
    numItems += 1;
  }

  // pop method: remove an item from the top of the stack
  // note: it is a runtime error if the stack is empty
  def pop() {
    if isEmpty? then
      halt("attempt to pop an item off an empty stack");
    numItems -= 1;
    return data(numItems+1);
  }

  // isEmpty? method: true if the stack is empty; otherwise false
  def isEmpty? return numItems == 0;
}
\end{numberedchapel}
