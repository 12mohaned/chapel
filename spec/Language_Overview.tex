\sekshun{Language Overview}
\label{Language_Overview}

Chapel is a new programming language under development at Cray Inc. as
part of the DARPA High Productivity Computing Systems (HPCS) program
to improve the productivity of parallel programmers.

This section provides a brief overview of the Chapel language by first
discussing the guiding principles behind the design of the language
and then providing introductory discussions of the main language
features.  These discussions are written at an intuitive level of
detail.  They are neither formal nor complete, but rather are intended
as starting points for the beginning Chapel programmer.

\subsection{Guiding Principles}

The following four main principles have guided the design of the
Chapel language since conception:
\begin{enumerate}
\item General parallel programming
\item Control of locality
\item Object oriented programming
\item Generic programming
\end{enumerate}
The first two principles were motivated by a desire to support
general, performance-oriented parallel programming through high-level
abstractions.  The second two principles were motivated by a desire to
narrow the gulf between high-performance parallel programming
languages and mainstream programming and scripting languages.

\subsubsection{General Parallel Programming}

First and foremost, Chapel is designed to support general parallel
programming through the use of high-level language abstractions.
Chapel supports a \emph{global-view programming model} that raises the
level of abstraction of expressing both data and control flow when
compared to parallel programming models currently used in production.
A global-view programming model is best defined in terms
of \emph{global-view data structures} and a \emph{global view of
control}.

\emph{Global-view data structures} are arrays and other data
aggregates whose sizes and indices are expressed globally even though
their implementations may distribute them across the \emph{locales} of
a parallel system.  A locale is an abstraction of a unit of uniform
memory access on a target architecture.  That is, within a locale, all
threads exhibit similar access times to any specific memory address.
For example, a locale in a commodity cluster could be defined to be a
single core of a processor, a multicore processor or an SMP node of
multiple processors.

Such a global view of data contrasts with most parallel languages
which tend to require users to partition distributed data aggregates
into per-processor chunks either manually or using language
abstractions.  As a simple example, consider creating a 0-based vector
with $n$ elements distributed between $p$ locales.  A language like
Chapel that supports global-view data structures allows the user to
declare the array to contain $n$ elements and to refer to the array
using the indices $0 \ldots n-1$.  In contrast, most traditional
approaches require the user to declare the array as $p$ chunks of
$n/p$ elements each and to specify and manage inter-processor
communication and synchronization explicitly (and the details can be
messy if $p$ does not divide $n$ evenly).  Moreover, the chunks are
typically accessed using local indices on each processor
(\eg,~$0..n/p$), requiring the user to explicitly translate between
logical indices and those used by the implementation.

A \emph{global view of control} means that a user's program commences
execution with a single logical thread of control and then introduces
additional parallelism through the use of certain language concepts.
All parallelism in Chapel is implemented via multithreading, though
these threads are created via high-level language concepts and managed
by the compiler and runtime, rather than through explicit
fork/join-style programming.  An impact of this approach is that
Chapel can express parallelism that is more general than the Single
Program, Multiple Data~(SPMD) model that today's most common parallel
programming approaches use as the basis for their programming and
execution models.  Chapel's general support for parallelism does not
preclude users from coding in an SPMD style if they wish.

Supporting general parallel programming also means targeting a broad
range of parallel architectures.  Chapel is designed to target a wide
spectrum of HPC hardware including clusters of commodity processors
and SMPs; vector, multithreading, and multicore processors; custom
vendor architectures; distributed-memory, shared-memory, and shared
address space architectures; and networks of any topology.  Our
portability goal is to have any legal Chapel program run correctly on
all of these architectures, and for Chapel programs that express
parallelism in an architecturally-neutral way to perform reasonably on
all of them.  Naturally, Chapel programmers can tune their codes to
more closely match a particular machine's characteristics, though
doing so may cause the program to be a poorer match for other
architectures.

\subsubsection{Control of Locality}

A second principle in Chapel is to allow the user to optionally and
incrementally specify where data and computation should be placed on
the physical machine.  Such control over program locality is essential
to achieve scalable performance on large machine sizes.  Such control
contrasts with shared-memory programming models which present the user
with a flat memory model.  It also contrasts with SPMD-based
programming models in which such details are explicitly specified by
the programmer on a process-by-process basis via the multiple
cooperating program instances.

\subsubsection{Object-Oriented Programming}

A third principle in Chapel is support for object-oriented
programming.  Object-oriented programming has been instrumental in
raising productivity in the mainstream programming community due to
its encapsulation of related data and functions into a single software
component, its support for specialization and reuse, and its use as a
clean mechanism for defining and implementing interfaces.  Chapel
supports objects in order to make these benefits available in a
parallel language setting, and to provide a familiar paradigm for
members of the mainstream programming community.  Chapel supports
traditional reference-based classes as well as value classes that are
assigned and passed by value.

Chapel does not require the programmer to use an object-oriented style
in their code, so that traditional Fortran and C programmers in the
HPC community need not adopt a new programming paradigm in order to
use Chapel effectively.  Many of Chapel's standard library
capabilities are implemented using objects, so such programmers may
need to utilize a method-invocation style of syntax to use these
capabilities.  However, using such libraries does not necessitate
broader adoption of object-oriented methodologies.

\subsubsection{Generic Programming}

Chapel's fourth principle is support for generic programming and
polymorphism.  These features allow code to be written in a style that
is generic across types, making it applicable to variables of multiple
types, sizes, and precisions.  The goal of these features is to
support exploratory programming as in popular interpreted and
scripting languages, and to support code reuse by allowing algorithms
to be expressed without explicitly replicating them for each possible
type.  This flexibility at the source level is implemented by having
the compiler create versions of the code for each required type
signature rather than by relying on dynamic typing which would result
in unacceptable runtime overheads for the HPC community.

\subsection{Getting Started}

A Chapel version of the standard ``hello, world'' computation is as
follows:
\vspace{0.5pc}
\begin{chapel}
writeln("hello, world");
\end{chapel}
This complete Chapel program contains a single line of code that makes
a call to the standard \chpl{writeln} function.

\index{modules}
\index{main@\chpl{main}}

In general, Chapel programs define code using one or more named
\emph{modules}, each of which supports top-level initialization code
that is invoked the first time the module is used.  Programs also
define a single entry point via a function named \chpl{main}.  To
facilitate exploratory programming, Chapel allows programmers to
define modules using files rather than an explicit module declaration
and to omit the program entry point when the program only has a single
user module.

Chapel code is stored in files with the extension \chpl{.chpl}.
Assuming the ``hello, world'' program is stored in a file
called \chpl{hello.chpl}, it would define a single user
module, \chpl{hello}, whose name is taken from the filename.  Since
the file defines a module, the top-level code in the file defines the
module's initialization code.  And since the program is composed of
the single \chpl{hello} module, the \chpl{main} function is omitted.
Thus, when the program is executed, the single \chpl{hello} module
will be initialized by executing its top-level code thus invoking the
call to the \chpl{writeln} function.  Modules are described in more
detail in~\rsec{Modules}.

To compile and run the ``hello world'' program, execute the following
commands at the system prompt:
\begin{commandline} 
> chpl -o hello hello.chpl
> ./hello
\end{commandline}
The following output will be printed to the console:
\begin{commandline}
hello, world
\end{commandline}

\subsection{Variables}
\index{param@\chpl{param}}
\index{const@\chpl{const}}

There are three kinds of Chapel variables: \chpl{var},
\chpl{const} (runtime constants), and \chpl{param} (compile time constants).
Chapel variables are discussed in detail in~\rsec{Variables}.

\index{variables!declarations}
Chapel is a strongly typed language.  Each variable declaration must
include a type specification or an initialization assignment from
which the type can be inferred.  Variable declarations include the
kind of variable, a type specification, and an initialization
assignment.  The type specification can be omitted if the type can be
inferred from the initial value.  If the variable is not initialized
in its declaration, then it contains a default initial value according
to its type.

\begin{example}
The following code gives examples of variable declarations in Chapel:
\begin{chapel}
var x: real = 1.0; // x is real, init to 1.0
var y, z: real;    // y, z are real, init to 0.0

var n = 10; // n inferred to be int, init to 10
const size = n; // size is runtime constant int,
                // set to 10
\end{chapel}
\end{example}

\index{variables!configuration}
The keyword \chpl{config} may precede any global variable declaration,
indicating that the variable may be set at compile time or runtime.
To set configuration variables at compile time, use ``\chpl{-s}''
followed by the variable name and value on the compiler command line.
To set configuration variables at runtime, use ``\chpl{--}'' followed
by the variable name and value on the execution command line.  More
details about configuration variables are given
in~\rsec{Configuration_Variables}.

\begin{example}
The following code gives an example of a configuration compile time
constant \chpl{debug} and a configuration runtime constant \chpl{n}:
\begin{chapel}
config param debug: bool;
config const n = 100;
\end{chapel}

If the program is stored in a file called ``test.chpl,'' then to
set \chpl{debug} at compile time, use the following command:
\begin{commandline}
> chpl test.chpl -s debug=true
\end{commandline}

To set \chpl{n} at runtime, use the following command:
\begin{commandline}
> ./a.out --n=1000
\end{commandline}
\end{example}

\subsection{Types}

Chapel is a strongly typed language.  Some implicit conversions 
between data types are supported, but many type conversions require
an explicit cast.  See~\rsec{Conversions} for more information. 

Type aliases and generics are unified by the syntax of Chapel.  Type
aliases are described in more detail in~\rsec{Type_Aliases}.

\subsubsection{Primitive Types}
\index{types!primitive}

Chapel provides primitive data types for signed and unsigned integers,
real, imaginary and complex floating-point numbers, logical values,
strings, files, and locales.  The following table provides information
about the names of these data types in Chapel, and their default sizes
and initial values.  If the size of the integer and floating point
data types are not specified, then the default size is used.  When
declaring variables, if an initial value is not supplied, then the
default initial value for that variable's type is used.
\begin{center}
\begin{tabular}{|l|l|l|l|}
\hline
{\bf Type} & {\bf Default Size} & {\bf Types with Specified Sizes} & {\bf Default Initial Value}\\
\hline
\chpl{int} & 32 bits &
\chpl{int(8)}, \chpl{int(16)}, \chpl{int(32)}, \chpl{int(64)} &
\chpl{0} \\
\hline
\chpl{uint} & 32 bits &
\chpl{uint(8)}, \chpl{uint(16)}, \chpl{uint(32)}, \chpl{uint(64)} &
\chpl{0} \\
\hline
\chpl{real} & 64 bits &
\chpl{real(32)}, \chpl{real(64)}, \chpl{real(128)} &
\chpl{0.0} \\
\hline
\chpl{imag} & 64 bits &
\chpl{imag(32)}, \chpl{imag(64)}, \chpl{imag(128)} &
\chpl{0.0i} \\
\hline
\chpl{complex} & 128 bits &
\chpl{complex(64)}, \chpl{complex(128)}, \chpl{complex(256)} &
\chpl{0.0 + 0.0i} \\
\hline
\chpl{bool} &  & &
\chpl{false} \\
\hline
\chpl{string} &  & &
\chpl{""} \\
\hline
\chpl{file}  & & &
\emph{closed file with no name} \\
\hline
\chpl{locale} & & &
\emph{predefined} \\
\hline
\end{tabular}
\end{center}

\subsubsection{Ranges}
\index{ranges}

Ranges are bounded and unbounded strided sequences of integral types.
Ranges can be specified by the literal expression, \chpl{low..high by
stride} where \chpl{low}, \chpl{high} and \chpl{stride} are integral
expressions.  The bounds of the range, \chpl{low} and \chpl{high}, can
be left unspecified to indicate an unbounded range.  The stride is
also optional.  If \chpl{by stride} is omitted, the default stride of
the range is one.

Ranges are most commonly used in the definition of arithmetic domains
and in iterator expressions.  Ranges are described further
in~\rsec{Ranges}.

\begin{example}
The following example defines a range \chpl{span1D} which is the
ordered set \chpl{1} to \chpl{n}.  This range is then used in the
definition of the arithmetic two-dimensional domain, \chpl{D}, which
is then used to declare the $n \times n$ array \chpl{A}.  See the
following paragraph for an introduction to domain and array types.
 
\begin{chapel}
var span1D: range = 1..n; // span1D represents set of ints 1 to n
var D = [span1D,span1D];  // D is 2-D, n x n domain
var A: [D] real;          // A is 2-D, n x n array
\end{chapel}
\end{example}

\subsubsection{Domains and Arrays}
\index{domains}
\index{arrays}
In Chapel, the indices and values for a data set are maintained
in two distinct types of data structures.  \emph{Domains} specify indices
and \emph{arrays} store the values for sets of data.  Arrays and domains may
be arithmetic, sparse, associative, enumerated or opaque.

\begin{example}
The code below gives simple examples of how domains and arrays are
declared and used.  The variable \chpl{D2} is declared to have
type \chpl{domain(2)} indicating that it is a two-dimensional domain.
The domain declaration also includes an initialization for the domain
using the range literal expression \chpl{1..n}, indicating
that \chpl{D2} is an arithmetic domain.  The arrays declared
with \chpl{D2}, \chpl{A}, \chpl{B}, \chpl{C} are two-dimensional
arithmetic integer arrays.

The \chpl{for} loop statement demonstrates how domains are used to
index into and iterate over entries in arrays.  And, finally the
assignment statement, \chpl{C = A + B} shows that whole array
operations are supported in Chapel.

\begin{chapel}
const D2: domain(2) = [1..n, 1..n];
var A, B, C: [D2] int;

for (i,j) in D2 {
  A(i,j) = j;
  B(i,j) = i;
}
C = A + B;
\end{chapel}
\end{example}

An array must be specified with a domain in its variable declaration,
and it is linked with this domain during execution.  When a domain is
modified by adding or removing indices, all arrays declared with this
domain reflect this change.  Arrays and domains are passed by
reference into functions but are assigned by value.

A domain can be declared with a distribution function which indicates
how the data values in arrays declared with this domain are to be
distributed across processors.

\begin{example}
In the code below, the domain \chpl{DistD2} is declared to be a 
block distributed two-dimensional arithmetic domain.
The arrays \chpl{A}, \chpl{B}, and \chpl{C} are then block distributed
two dimensional integer arrays.
\begin{chapel}
const DistD2: domain(2) distributed(Block) = [1..n, 1..n];
var A, B, C: [DistD2] int;
\end{chapel}
\end{example}

The Chapel language also supports constructs that allow 
access to a specific part of an array or domain through slicing,
reindexing, and array aliasing.   For more information about 
domains and arrays see~\rsec{Domains_and_Arrays}.
 
\subsubsection{Enumerations}
\index{enum@\chpl{enum}}
An \emph{enumerated} type defines an ordered set of named constants.
By default, these constants have an associated 
integral value, starting with one for the first constant. 
An integral value may be assigned to one or more of the constants
to override the default behavior.  For more information about
enumerated types see~\rsec{Enumerated_Types}.

\begin{example}
\begin{chapel}
enum day {sun, mon, tue, wed, thu, fri, sat};
var d = day.wed;

writeln(d, " is day number ", d:int, " of the week");
\end{chapel}

This example would output:
\begin{commandline}
wed is day number 4 of the week
\end{commandline}
\end{example}

\subsubsection{Tuples}

\index{tuples}
A \emph{tuple} is an ordered collection of types.
\begin{example}
The following code shows two ways of declaring tuples.  When the tuple
contains multiple types, as in the variable \chpl{pt}, the tuple
declaration specifies the component types of the tuple, separated by
commas and contained within parentheses.  For the homogeneous
tuple \chpl{x}, a short-hand notation can be used for its declaration.
The number of elements of the tuple, followed by \chpl{*} and the type
can be used in this case.

\begin{chapel}
var pt: (int,real);
var x: 3*int;
\end{chapel}
\end{example}

A tuple expression is a comma-separated list of expressions that is
enclosed in parentheses.  Tuples can be declared and assigned to with
tuple expressions.  When a tuple expression appears on the left-hand
side of an assignment statement, the expression on the right-hand side
is a \emph{destructured} tuple expression.

\begin{example}

The example below shows how the tuple variables \chpl{pt} and \chpl{x}
are declared with tuple expression literals.
\begin{chapel}
var pt = (1, 3.0);
var x = (1, 1, 1);
\end{chapel}

The elements of the tuple \chpl{x} can be assigned to integer variables
through the use of tuple destructuring.
\begin{chapel}
var a, b, c: int;

(a, b, c) = x;
\end{chapel}
\end{example}

There are a set of operators defined for variables of tuple type
and elements of a tuple are accessed through indexing and destructuring.  
For more detailed information about tuples see~\rsec{Tuples}.

Tuples are very useful constructs.  They can be used in functions with
variable length argument lists, index expressions for \chpl{for} loops
and in the assignment to complex variables.  One such example follows.

\begin{example}
Tuples may be used in the assignment to a complex variable.
\begin{chapel}
var x, y: real;
var z: complex;

z = (x,y):complex;
\end{chapel}
\end{example}

\subsubsection{Classes}
\index{classes}
\emph{Classes} are data structures defined with fields and methods.
A variable that is declared to be of a class type is a reference to an
object, or instance, of that class.
 
An instance of a class is created by calling its constructor in a
variable declaration.  Each call to the constructor instantiates a new
object of the class and returns a reference to the object.  Chapel
provides a default class constructor for each class.  For more
information about classes see~\rsec{Classes}.

\begin{example}
The following code gives an example of the circle class.  The
variable, \chpl{x}, is an instance of the circle class with a radius
of \chpl{1.0}.  The variable \chpl{y} is assigned the reference to the
same object as \chpl{x}.  When the radius of \chpl{y} is modified, the
radius of \chpl{x} is modified as well.  Writing the area of \chpl{x}
and \chpl{y} prints the same value, \chpl{12.56}.
\begin{chapel}
class circle {
  var radius: real;
  def area {
    param pi = 3.14;
    return pi*(radius**2);
  }
}
var x = circle(radius=1.0);
var y = x;

y.radius = 2.0;

writeln((x.area, y.area));
\end{chapel}

This program has the output
\begin{commandline}
(12.56, 12.56)
\end{commandline}
\end{example}

\subsubsection{Records}
\index{records}
\emph{Records} are similar to classes.  They contain fields and methods
and can inherit fields and methods from other records.  A record,
however, directly contains the data associated with the fields in the
record.  A record is not a reference to storage location as classes
are.  Thus, records are assigned by value.  For more information about
records see~\rsec{Records}.

\begin{example}
In the example code below, \chpl{p} and \chpl{q} are both variables of
the record \chpl{point}.  They are distinct storage locations that are
updated independently of each other.  When \chpl{p} is assigned
to \chpl{q}, \chpl{q} then contains the same values as \chpl{p}.  But,
when \chpl{q} is updated, \chpl{p} is not since they are referring to
the same storage location.

\begin{chapel}
record point {
  var x,y: real;
  def magnitude {
    return max(abs(x),abs(y));
  }
}
var p = point(0.0,1.0);
var q = p;

q.x = 2.0;

writeln((p.magnitude, q.magnitude));
\end{chapel}

This program has the output:

\begin{commandline}
(1.0, 2.0)
\end{commandline}
\end{example}

\subsubsection{Summary of Non-Primitive Types}

The non-primitive types are listed in the table below, along
with their default initial values for variables of that type. 
\begin{center}
\begin{tabular}{|l|l|}
\hline
{\bf Type} & {\bf Default Initial Value}\\
\hline
\chpl{range} & \chpl{1..0} \\
\hline
\chpl{domain} & \emph{empty range for each dimension} \\
\hline
\chpl{array} & \emph{each element has its default initial value} \\
\hline
\chpl{enum} & \emph{first enumeration constant} \\
\hline
\chpl{tuple} & \emph{each component has its default initial value} \\
\hline
\chpl{class} & \chpl{nil} \\
\hline
\chpl{record} & \emph{default constructor with zero arguments} \\
\hline
\end{tabular}
\end{center}

\subsection{Expressions}

Chapel provides a rich set of expressions described
in~\rsec{Expressions}.

\index{operators!precedence}
The following table provides a summary of Chapel's expressions.  They
are listed in order of precedence (high to low) and with an
associativity:
\begin{center}
\begin{tabular}{|l|l|l|}
\hline
{\bf Operators} & {\bf Associativity} & {\bf Use} \\
\hline
\verb@.@ & left & member access \\
\verb@() []@ & left & function call, index expression \\
\verb@**@ & right & exponentiation \\
unary \verb@+ - ~@ & right & sign and bitwise negation \\
\verb@:@ & left & cast\\
\verb@* / %@ & left & multiply, divide, and modulus \\
\verb@+ -@ & left & plus and minus \\
\verb@&@ & left & bitwise and \\
\verb@^@ & left & bitwise xor \\
\verb@<< >>@ & left & shift left and shift right \\
\verb@|@ & left & bitwise or \\
\verb@<= >= < >@ & left & ordered comparison \\
\verb@== !=@ & left & equality comparison \\
\verb@!@ & right & logical negation \\
\verb@&&@ & left & logical and \\
\verb@||@ & left & logical or \\
\verb@..@ & left & ranges  \\
\verb@in@ & left & forall expressions \\
\verb@by@ & left & striding ranges and domains \\
\verb@if@ & left & conditional expressions \\
\verb@reduce scan@ & left & reductions and scans\\
\verb@,@ & left & comma separated expressions \\
\hline
\end{tabular}
\end{center}

\paragraph{Scans and Reductions}
Chapel supports standard and user-defined reduction and scan
operators.  These operators precede the keyword \chpl{scan}
or \chpl{reduce}, depending on the desired operation.

The built-in Chapel scan and reduction operators are;
\begin{chapel}
+ * && || & | ^ min max
\end{chapel}
For more information about reductions, scans and user-defined
operators for either see~\rsec{Reductions_and_Scans}.

\paragraph{Query Expression}
\index{query expression}
The query expression is used to query the type of a generic function argument
and assign that type to a variable.  For more information about
query expressions and generic functions see~\rsec{The_Query_Expression} 
and~\rsec{Generic_Functions}.

\begin{example}
The following definition of the generic function \chpl{sumOfThree} 
ensures that all arguments and the return value are of the same type, 
without specifying the type.
The type of the first argument in the \chpl{sumOfThree}
function is queried and assigned to the variable \chpl{t}.  The remaining
arguments and the return value are then specified to also be of type \chpl{t}.
\begin{chapel}
def sumOfThree(x:?t, y:t, z:t):t {
   return x + y + z;
}
\end{chapel}
\end{example}

\paragraph{Let Expression} 
\index{let@\chpl{let}}
The \chpl{let}--\chpl{in} expression defines variables to
be used only in the expression following the \chpl{in} keyword.
For more information about the \chpl{let} expression 
see~\rsec{Let_Expressions}.

\begin{example}
\begin{chapel}
x = let t1 = sqrt(b*b - 4.0*a*c), t2 = 2.0*a
    in ((-b + t1)/t2, (-b - t1)/t2);
\end{chapel}
\end{example}

\paragraph{Conditional Expression} 
\index{conditional expressions}
Chapel provides a conditional expression that can be evaluated
at the expression level.  For more information about conditional
expressions see~\rsec{Conditional_Expressions}.

\begin{example}
\begin{chapel}
var half = if (i % 2) then i/2 +1 else i/2;
\end{chapel}
\end{example}

\subsection{Statements}

\paragraph{Assignment}
\index{operators!assignment}
Assignment in Chapel is supported by the following operators:
\begin{chapel}
 = += -= *= /= %= **= &= |= ^= &&= ||= <<= >>=
\end{chapel}

An assignment statement contains a left hand-side side expression
followed by an assignment operator and a right-hand side expression.
An assignment operator that has a binary operator as a prefix is a
short-hand notation for applying the binary operator to the left and
right-hand side expressions assigning the result to the left-hand
side.  For example,
\chpl{x += 1} is an alternative way of writing \chpl{x = x + 1}.

\paragraph{Swap}
\index{operators!swap}
The swap operator, \chpl{<=>} swaps the values between
the left and right-hand side expressions.
\begin{chapel}
x <=> y; // swaps the value in x with the value in y
\end{chapel}

\paragraph{The Array Alias Operator}
\index{arrays!aliases to}
\index{operators!array alias}
The array alias operator, \chpl{=>}, used within a
variable declaration, creates an alias to an array or 
an array slice.  
\begin{chapel}
var A11 => A[block1,block1];
\end{chapel}
For more information about array aliasing see~\rsec{Array_Aliases}.

\paragraph{Block Statement}  
\index{block level statement}
A block statement is delimited by braces, \{ \} and it
contains a set of Chapel statements separated by semi-colons.  
Variables declared within a block statement are local to that block.

\begin{example}
The following code shows a block of statements for computing a Givens
rotation.  The variable \chpl{tau} is declared within the block as is
local to this block.
\begin{chapel}

var a, b: real;
var s, c: real;

{
  var tau = -a/b;
  s = 1/sqrt(1 + tau*tau);
  c = s*tau;
}

\end{chapel} 
\end{example} 

\paragraph{For Loop Statements} 
\index{for loops}
In Chapel, a \chpl{for} loop statement defines an index and
specifies either a function, expression or a variable to
iterate over.  For more information about \chpl{for} loops
see~\rsec{The_For_Loop}.

\begin{example}
The following code gives an example of a \chpl{for} loop
statement that reads in data from the files \chpl{xdatain}
and \chpl{ydatain} to initialize the arrays \chpl{x} and
\chpl{y}.  

The next \chpl{for} loop iterates over the elements of 
\chpl{x} and \chpl{y}, an element of \chpl{x} and an element
of \chpl{y} on each line.  This loop iterates over arrays rather
than a domain.  Also, since the body of this loop is not a block
statement, the keyword \chpl{do} is needed to delimit the loop
body. 
\begin{chapel}
var x, y: [D] real;

for ind in D { 
  xdatain.read(x(ind));
  ydatain.read(y(ind));
}

for (ex, ey) in (x, y) do 
  writeln((ex, ey));
\end{chapel}
\end{example}

\paragraph{Conditional Statement} 
The Chapel conditional statement uses the standard
\chpl{if}--\chpl{then}--\chpl{else} structure.  When
the then clause is a block of statements, delimited with
`\{' and `\}', the \chpl{then} keyword may be omitted.
For more information about conditional statements 
see~\rsec{The_Conditional_Statement}.

\begin{example}
\begin{chapel}
if (a == 0.0) then 
  x = 0.0;
else 
  x = a*y;

if (b != 0.0) {
  z = b*x;
  y = z; 
}
\end{chapel}
The example demonstrates two conditional statements, one with
both \chpl{then} and \chpl{else} clauses, and one with just a
\chpl{then} clause.
\end{example}

\paragraph{Select Statement} 
\index{select@\chpl{select}}
Chapel provides a \chpl{select} statement that allows the option of
executing different statement blocks, depending on the value of the
select variable.  For more information about the \chpl{select}
statement see~\rsec{The_Select_Statement}.

\begin{example}
\begin{chapel}
enum day {sun, mon, tue, wed, thu, fri, sat};
var today: day;

select today {
  when fri, sat do rate = 250;
  when sun do rate = 180;
  otherwise rate = 140;
}
\end{chapel}
In the above example, the \chpl{select} statement assigns
a value to \chpl{peakRate} depending on the value of the 
variable \chpl{today}.
\end{example}

\paragraph{Type Select Statement} 
\index{type select statements}
The \chpl{type select} statement allows the choice of
executing different statement blocks depending on the type
of the select variable.  For more information about the
\chpl{type select} statement see~\rsec{The_Type_Select_Statement}.

\begin{example}
\begin{chapel}
type select ind {
  when int do y = x(ind);
  when real, uint do y = x(ind:int);
  otherwise y = 0.0;
}
\end{chapel}
In the above example, the variable \chpl{ind} is used
to index into the array \chpl{x} if it has the appropriate
type.
\end{example}

\subsection{Functions and Iterators}
\paragraph{Function Definitions}

Functions are defined with the keyword \chpl{def}, a name and a list
of function arguments.  Optionally, intents and types for the function
arguments, return types and a where clause can be specified.

\begin{example}
The example below defines a simple function that computes the area of
a rectangle.  The \chpl{areaRect} function is defined to take two
function arguments of type \chpl{real} and return a value of
type \chpl{real}.  If this function is called with function arguments
that have a type other than \chpl{real}, it will result in a function
resolution error at compile time.
\begin{chapel}
def areaRect(x:real, y:real): real {
  return x*y;
}
\end{chapel}
\end{example}

The Chapel language supports function overloading.  At compile time,
function resolution will identify the most specific function for each
function call from a set of candidate functions depending on the
number of function arguments, the types specified for the input and
return values, and any where clauses.

\paragraph{Function Intents}
\index{functions!intents}
Intents of function arguments specify how an argument can be
modified during and after the function call. 
The intents can be specified as \chpl{in}, \chpl{out} or \chpl{inout}.

If the intent of an argument is \chpl{in}, the actual argument is copied 
to the formal argument and it may be modified within the function.  However, 
the actual argument at the call site does not reflect any modifications 
made within the function.  The \chpl{out} intent indicates that the 
actual argument is ignored when the call is made, but after the
call the formal argument is assigned to the actual argument at the call site.
The \chpl{inout} intent indicates that the actual argument be copied to 
the formal argument and then copied back after the call.

If the intent is omitted, then the argument has a blank intent,
copying in the value using the assignment operator.  Arguments with a
blank intent cannot be assigned within a function.  Actual arguments
of array and domain types are handled differently when they have a
blank intent.  In this case, array and domain arguments are passed by
reference.  Thus, they can be modified within the function and those
modifications are reflected at the call site.  Arrays arguments with
non-blank intents are handled according to definition, being copied to
and from the formal arguments.  Actual arguments of domain types can
only have a blank intent.

Further information about function intents is in~\rsec{Intents}.

\paragraph{Generic Functions}
\index{generics!functions}
The Chapel language supports \emph{generic functions}, allowing the user
to define a function without specifying the types of the formal arguments
or return variables.  A generic function definition can have formal arguments  
of generic type, formal arguments without a specified type or 
with a queried type, or formal arguments that are tagged with 
\chpl{type} or \chpl{param} keywords.  In addition, a function is
considered to be generic if it contains an array argument with its
domain or element type unspecified or queried.

\begin{example}
\begin{chapel}
def absSum(x) {
  return + reduce abs(x); 
}
\end{chapel}
This routine, \chpl{absSum}, can be called
with arrays of any element type for which an \chpl{abs} function is
defined.  Further, this routine can be called with arrays of 
any domain type, so it can be called with arrays of arbitrary dimension.
\end{example}

Generic functions use \emph{query expressions} to query the types or
domains of arguments.  The query expression uses a \chpl{?} to
indicate that the type of the preceding variable be queried and
assigned to the following variable.  For more information about
generic functions, see~\rsec{Generic_Functions}.

\begin{example}
The generic function \chpl{average2} returns the average of the
two input arguments.  The type of the first argument \chpl{x} is
queried and assigned to \chpl{t}.  The second argument \chpl{y}
and the return value and specified to be of same type as the first
argument.  In the function, the constant \chpl{two} is defined to
be \chpl{2}, casted to the same type as the input arguments and result.

Two calls to \chpl{average2} are shown, one with integer arguments and one
with real arguments.  
\begin{chapel}
def average2(x: ?t, y: t): t {
  const two = 2.0:t;
  return (x + y)/two;
}

writeln(average2(1,4));
writeln(average2(1.0,4.0);
\end{chapel}

This program has the output:
\begin{commandline}
2
2.5
\end{commandline}
\end{example}

\paragraph{Variable Length Argument Lists}
Functions in Chapel can take variable length argument lists.
Tuples can be used in the definition of such functions.
For more information about variable length argument lists 
see~\rsec{Variable_Length_Argument_Lists}.

\begin{example}
In the code below, the function \chpl{writeLines} is defined to write
each string argument on a separate line.  The symbol \chpl{...} in the
argument list indicates the number of arguments is variable.  The
notation \chpl{?n} indicates to query the number of arguments and
assign that number to \chpl{n}.  The variable \chpl{eachLine} is a
homogeneous tuple that contains \chpl{n} elements of
type \chpl{string}.  The function contains a \chpl{for} loop that
writes each element of
\chpl{eachLine} on a separate line.

\begin{chapel}
def writeLines(eachLine:string...?n) {
  for i in 1..n do 
    writeln(eachLine(i));
}

writeLines("Write","one","word","on","each","line"); 
\end{chapel}

The output of this example is:
\begin{commandline}
Write
one
word
on
each
line
\end{commandline}
\end{example}



\paragraph{Iterators}
\index{iterators}
Iterators generate sequences of values.  They are defined similarly to
functions, except that they contain at least one \chpl{yield}
statement.  Iterators may be called in \chpl{for} and \chpl{forall}
loop statements.  For each iteration of the loop, a value from the
iterator is yielded and the body of the loop executed.  More
information about iterators is in~\rsec{Iterators}.
\begin{example}
\begin{chapel}
def evens(n) {
  for i in 1..n do yield (i,2*i);
}

for (i,i2) in evens(m) {
  x(i) = y(i2);
}
\end{chapel}
The above code gives an example of an iterator definition and use.
The iterator, \chpl{evens} yields a tuple of integers at each iteration.
\end{example}

\subsection{Input and Output}
\label{Intro_Input_and_Output}

\paragraph{File I/O}
\index{file type}
The Chapel language provides a \chpl{file} type for use in reading and
writing to files.  To read or write from a file, a variable of \chpl{file}
type must be declared and then assigned the name and path of the file 
and whether the file is to be read or written.  Methods to open, to close,
to read and to write from that file can be invoked with this file
variable.

\begin{example}
In the following example, the array \chpl{A} is initialized with
values read from the file \chpl{inMatrix.dat}.
\begin{chapel}
var A: [D] real;

var inputFile = file("inMatrix.dat");
inputFile.open();
for (i, j) in D do inputFile.read(A(i,j));
inputFile.close();
\end{chapel}
\end{example}

\paragraph{Chapel Standard I/O Files and Functions}
Chapel provides three standard files, \chpl{stdout}, \chpl{stdin}
and \chpl{stderr} which map to standard output, standard input and
standard error, respectively.

Chapel provides built-in functions \chpl{write} and \chpl{writeln}
to write to \chpl{stdout}, and the built-in functions \chpl{read}
and \chpl{readln} that read from \chpl{stdin}.

\paragraph{Default I/O Methods for Chapel Types}
Default \chpl{read}, \chpl{readln}, \chpl{write}, and \chpl{writeln} 
methods are provided for all Chapel types.  

\begin{example}
The following code shows the definition of an arithmetic domain \chpl{D}
and array \chpl{A}, and calling the \chpl{writeln} function
to print both variables to standard output.  
\begin{chapel}
var D = [1..5, 1..5];
var A: [D] real;

class circle {
  var radius: real;
  const diameter = 2.0*radius;
  def area {
    param pi = 3.14;
    return pi*(radius**2);
  }
}

var x = circle(1.0);
writeln(x);

writeln("Writing a variable of domain type:");
writeln(D);
writeln("Writing a variable of array type:");
writeln(A);
\end{chapel}

The output is shown below.  Variables of arithmetic domain type are 
formatted to print the ranges for each dimension of the domain.  Variables
of arithmetic array type are formatted to print the values of each row
separated by a carriage return.
\begin{commandline}
Writing a variable of domain type:
[1..5, 1..5]
Writing a variable of array type:
0.0 0.0 0.0 0.0 0.0
0.0 0.0 0.0 0.0 0.0
0.0 0.0 0.0 0.0 0.0
0.0 0.0 0.0 0.0 0.0
\end{commandline}
\end{example}

In addition, \chpl{read}, \chpl{readln}, \chpl{write} and 
\chpl{writeln} methods
are provided for user-defined types.  Users may override the default
write method by providing a \chpl{writeThis} method
for that type.

\begin{example}
In the following example, the class \chpl{circle} is defined and
an instance \chpl{x} is declared and then written.
When an instance of class is written, by default, the names and values 
of that object's fields are written.  So, the \chpl{radius} and
\chpl{diameter} fields of \chpl{x} are written.

\begin{chapel}
class circle {
  var radius: real;
  const diameter = 2.0*radius;
  def area {
    param pi = 3.14;
    return pi*(radius**2);
  }
}

var x = circle(1.0);
writeln(x);
\end{chapel}

This code produces the output:
\begin{commandline}
{radius = 1.0, diameter = 2.0}
\end{commandline}
\end{example}

\subsection{Parallelism}

The Chapel language is designed to ease the challenges of parallel
programming.  Since the language has specific constructs to support
parallel computations, the user can easily express the parallel work,
and the compiler can easily identify it.

Chapel supports both data parallelism and task parallelism.  To
efficiently manage both types of parallelism on a multi-processor
system, users can specify how data and work is to be distributed
across processors by using distribution and locality features.

\paragraph{Data Parallelism}
Data parallelism refers to work where the same operations can be
executed concurrently on a set of data.  To support data parallel
work, the Chapel language provides the parallel \chpl{forall}
statement.  The iterations of a \chpl{forall} loop execute
concurrently, as determined by the compiler and the runtime library.
For more information about \chpl{forall} loops see~\rsec{Forall}.

The Chapel language provides an alternative, short-hand notation for 
defining a parallel loop.  This notation omits the \chpl{forall} keyword
and uses brackets to delimit the rest of the loop statement which includes
the index and iterator expressions.  This notation is convenient to use
when the loop body is a single statement.  

The compiler will also parallelize whole array and domain statements, 
where possible.

\begin{example}
The following code demonstrates three ways to assign, in parallel, 
the scaled elements of the array \chpl{B} to the elements of 
the array \chpl{A}.  
\begin{chapel}  
D = [1..m, 1..n];
A, B: [D] real;
alpha: real;

forall ij in D {
  A(ij) = alpha*B(ij);
} 

[ij in D] A(ij) = alpha*B(ij);

A = alpha*B;
\end{chapel}
The first loop uses the \chpl{forall} loop statement.  
The second loop uses the short-hand bracketed notation and the 
third statement expresses the computation using whole-array operations.  
All three computations will execute concurrently in the same manner,
as determined by the compiler and runtime library.
\end{example}


\paragraph{Task Parallelism}
The Chapel language also supports parallelism where different types of
operations are being executed concurrently with different data sets.
In this case, computations are spawned through the use of the \chpl{begin}
or \chpl{co-begin} statements.  Synchronization between these computations
is managed through the use of \chpl{sync} and
\chpl{single} variables.

The \chpl{begin} statement spawns a computation to execute a
statement. Control continues simultaneously with the statement following
the begin statement.  A begin statement cannot contain break,
yield, or return statements.  For more information about the \chpl{begin}
statement see~\rsec{Begin}. 

The \chpl{cobegin} statement is used to create parallelism within
a block statement.  All statements within the block statement
are executed concurrently.  Any variable declared within the cobegin
statement is a \chpl{single} variable.  For more information about
the \chpl{cobegin} statement see~\rsec{Cobegin}.

\begin{example}
\begin{chapel}
var done: sync bool;
 
begin 
  while(!done) do work(x);

otherwork(x);
done = true;

cobegin {
  init(A);
  init(B);
}
\end{chapel}
\end{example}

In addition, the \chpl{coforall} loop statement is provided in order
to express the concurrent statements within a \chpl{cobegin} in an
easier way.  For more details about the \chpl{coforall} loop statement
see~\rsec{Coforall}.

\begin{example}
If three consumer computations needed to be started, a \chpl{cobegin}
statement could be used.
\begin{chapel}
cobegin {
  runConsumer();
  runConsumer();
  runConsumer();
}
\end{chapel}

By using a \chpl{coforall} statement instead, the code is shorter
and less redundant.
\begin{chapel}
coforall i in 1..3 do
  runConsumer();
\end{chapel}
\end{example}

\paragraph{Synchronization Variables}
Synchronization between computations is managed through \chpl{sync}
and \chpl{single} variables.  Sync variables have a {\em full} or {\em
empty} state associated with them that is modified when read and
written to and control access to these variables between threads of
computation.  Single variables are similar to sync variables, but they
are assigned once.  Using a single variable before it is assigned,
suspends the computation until another task assigns a value to it.
For more information about synchronization variables
see~\rsec{Sync_Variables} and~\rsec{Single_Variables}.
\begin{chapel}
var x: single int;
var y: sync real;
\end{chapel}

\paragraph{Suppressing Parallelism}
It is often necessary to suppress concurrency during sections of
parallel programs.  Chapel provides a \chpl{serial} statement to
serialize statements within a parallel region.  For more information
about the \chpl{serial} statement see~\rsec{Serial}.


\paragraph{Scans and Reductions}
Chapel supports operations that execute over multiple locales through
the use of \chpl{scan} and \chpl{reduce} expressions.  The language
provides built-in \chpl{scan} and \chpl{reduce} operators.  For more
information about scans and reductions see~\rsec{Reductions_and_Scans}.

\paragraph{Data Distributions and Locality}

\index{locale@\chpl{locale}}
In Chapel, the term {\em locale} refers to the processing unit in a
parallel computer system.  Chapel provides a predefined
array \chpl{Locales} where each entry is of the \chpl{locale} type.
User-defined variables of \chpl{locale} type can be declared to store
entries from this \chpl{Locales} array or the entries can be accessed
directly to indicate where data should be reside in memory or where
computations should be executed.  For more information about locales,
see~\rsec{Locales} and~\rsec{Specifying_Locales_for_Computation}.

Chapel manages the distribution of data through the use of domains
and distributions.  A distribution is a mapping of a domain's indices to 
locales.  When a domain is declared with a distribution, then any iteration
over that domain or its associated arrays will execute in parallel across the 
locales according to how the data is distributed.

The \chpl{on} statement controls on which locale a computation or data
should be placed.

For more information about locality and distributions 
see~\rsec{Locality_and_Distribution}.

