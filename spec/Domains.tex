\sekshun{Domains}
\label{Domains}
\index{domains}

A \emph{domain} is a first-class representation of an index set.
Domains are used to specify iteration spaces, to define the size and
shape of arrays (\rsec{Arrays}), and to specify aggregate operations
like slicing.  The indices described by a domain may be regular and
structured or they may be irregular and unstructured.  Chapel also
supports the ability to create \emph{subdomains} and \emph{sparse
subdomains} to represent subsets of a domain's index set.  A domain's
indices may potentially be distributed across multiple locales as
described in~\rsec{Domain_Maps}, supporting global-view data
structures.


\subsection{Domain Taxonomy}
\label{Domain_Taxonomy}
\index{domains!taxonomy}

This section describes Chapel's taxonomy of domain types.

\subsubsection{Root Domains and Subdomains}
\index{domains!root}

% could also call root domains base domains, except that we currently
% use that term for subdomains -- a place where it seems natural, but
% where we could also lean more heavily on parent domain...

A domain is either a \emph{root domain} or a \emph{subdomain}.  This
is represented as follows:

\begin{syntax}
domain-type:
  root-domain-type
  subdomain-type
\end{syntax}

 \noindent A root domain has no parent domain and can represent an
arbitrary set of indices of its index type.  A subdomain has an
associated parent domain value and is constrained to only store
indices that are also described by its parent domain.


\subsubsection{Regular and Irregular Domain Types}
\index{domains!regular}
\index{domains!irregular}

Domain types can be thought of as falling into two major categories:
regular and irregular.  This is represented for root domain types as
follows:

\begin{syntax}
root-domain-type:
  regular-domain-type
  irregular-domain-type
\end{syntax}

Regular domains, known as \emph{arithmetic domains}, describe
multidimensional rectangular index sets.  They are characterized by a
tensor product of ranges and represent indices that are tuples of an
integral type.  Regular domains can be represented using $O(1)$ space.
They are useful for representing multidimensional rectangular index
sets and arrays.

% TODO: Do we depend on the term arithmetic anywhere in the spec or
% code?

An irregular domain can store an arbitrary set of indices of an
arbitrary but homogenous index type.  Irregular domains typically
require space proportional to the number of indices being
representeed.

\index{domains!associative}
\index{domains!opaque}

The two major classes of irregular domains in Chapel are associative
domains and opaque domains:

\begin{syntax}
irregular-domain-type:
  associative-domain-type
  opaque-domain-type
\end{syntax}

\noindent Associative domains represent an arbitrary set of indices
of a given type and can be used to describe sets or to create
dictionary-style arrays.  Opaque domains are those for which the
indices have no inherent names and are therefore anonymous.  They can
be used for representing sets and for building unstructured arrays,
similar to pointer-based data structures in conventional languages.

Sparse subdomains, described in~\rsec{Sparse_Subdomains}, are also
considered to be irregular domains.  A non-sparse subdomain inherits
the regularity or irregularity of its parent domain.


\subsection{Domain Characteristics}
\label{Domain_Characteristics}
\index{domains!characteristics}

\subsubsection{Domain Types}
\label{Domain_Types}
\index{domains!types}

All domain types are characterized by the type of indices that they
store (see~\rsec{Index_Types}).  The way in which these index types
are specified in the domain's type signature varies across domain
types.  It is defined for root domain types
in~\rsec{Root_Domain_Types} and for subdomains
in~\rsec{Subdomain_Types}.

\subsubsection{Domain Values}
\label{Domain_Values}
\index{domains!values}
\index{domains!ordered}
\index{domains!unordered}

A domain's value is the index set that it represents.  A domain's
index set can be considered either \emph{ordered} or \emph{unordered},
indicating whether or not there is a well-defined order defined for
its indices for the purposes of things like serial iteration and I/O.
The domain values for the root domain types are defined
in~\rsec{Domain_Values}.  The domain values for sparse subdomain types
are defined in~\rsec{Sparse_Domain_Values}.

\subsubsection{Domain Identity}
\label{Domain_Identity}
\index{domains!identity}

In addition to storing a value, domain variables have an identity that
distinguishes them from other domains of the same type with the same
value.  This identity is used to define the domain's relationship
with subdomains~(\rsec{Subdomains}), index types~(\rsec{Index_Types}),
and arrays~(\rsec{Association_of_Arrays_to_Domains}).  The identity of
a domain is represented by its name.

\begin{openissue}
In the future, it is likely that we will support a means of creating
domain aliases, much as we support array aliases currently.
\end{openissue}

\subsubsection{Runtime Representation of Domain Values}
\label{Domain_Runtime_Representation}
\index{domains!runtime representation}
\index{domains!domain maps}

While domains are a high-level abstraction, users have control over
the runtime representation of a domain's index set through Chapel's
support for domain maps~(\rsec{Domain_Maps}), both
standard~(\rsec{Standard_Layouts} and \rsec{Standard_Distributions})
and user-defined~(\rsec{User_Defined_Domain_Maps}).  Chapel
implementations should also document their choice of implicit domain
maps (used to implement domains with no domain map specifiers).


\subsection{Root Domain Types}
\label{Root_Domain_Types}
\index{domains!root types}
\index{types!domains, root}


\subsubsection{Arithmetic Domain Types}
\index{arithmetic domains!types}
\index{types!arithmetic domains}

Arithmetic domain types are parameterized by three things:
\begin{itemize}

 \item \chpl{rank}, a positive \chpl{int} value indicating the number
of dimensions that the domain represents;

 \item \chpl{idxType}, a type member representing the index type for
each dimension.  If unspecified, \chpl{idxType} defaults
to \chpl{int}.

% BLC: we should potentially rename idxType to idxDimType to make it
% more consistent with the irregular case

 \item \chpl{stridable}, a \chpl{bool} value indicating whether or not
any of the domain's dimensions will be characterized by a strided
range.  If unspecified, \chpl{stridable} defaults to \chpl{false}.

\end{itemize}

If \chpl{rank} is $1$, the index type represented by an arithmetic
domain is \chpl{idxType}.  Otherwise, the index type is the homogenous
tuple type \chpl{rank*idxType}.

% BLC: longer-term, we've talked about potentially making it be
% rank*idxType even if rank is 1.

The syntax of an arithmetic domain type is summarized as follows:

\begin{syntax}
regular-domain-type:
  `domain' ( named-expression-list )
\end{syntax}

\noindent where \sntx{named-expression-list} permits the values of
\chpl{rank}, \chpl{idxType}, and \chpl{stridable} to be specified
using standard function call syntax.

%\begin{example}
%
% PLACEHOLDER: INSERT EXAMPLE HERE
%
%\end{example}


\subsubsection{Associative Domain Types}
\label{Associative_Domain_Types}
\index{types!associative domains}
\index{associative domains!types}

An associative domain type is parameterized by \chpl{idxType}, the
type of the index that it stores.  The syntax is as follows:

\begin{syntax}
associative-domain-type:
  `domain' ( associative-index-type )

associative-index-type:
  type-specifier
\end{syntax}

%
% should there be an opt_idxTypeEquals in the domain type above?
%


If the \sntx{associative-index-type} is an enumerated type, the
associative domain is called an \emph{enumerated domain type}---a
variant of associative domain types that has some distinct
characteristics, described in subsequent sections.

% TODO: is there a more explicit reference I could put here?

\subsubsection{Opaque Domain Types}
\label{Opaque_Domain_Types}
\index{types!opaque domains}
\index{opaque domains!types}

An opaque domain type is parameterized by the type \chpl{opaque},
indicating that the index values are anonymous and have no obvious
representational name or value.  The opaque domain type is given by
the following syntax:

\begin{syntax}
opaque-domain-type:
  `domain' ( `opaque' )
\end{syntax}


\subsection{Root Domain Values}
\label{Root_Domain_Values}
\label{Root_Domain Literals}
\label{Root_Default_Domain_Values}

This section describes the values, literal formats (if applicable),
and default values for each root domain type.

\subsubsection{Arithmetic Domain Values}
\label{Arithmetic_Domain_Values}
\index{arithmetic domains!values}

An arithmetic domain's value is represented as \chpl{rank} ranges of
 type \chpl{range(idxType,} \chpl{ BoundedRangeType.bounded,} \chpl{
 stridable)}.  The index set for a rank~1 domain is the set of indices
 described by its singleton range.  The index set for a rank~$n$
 domain is the set of all \chpl{n*idxType} tuples described by the
 tensor product of its ranges.  Arithmetic domain indices are ordered
 according to the lexicographic order of their values.

\index{arithmetic domains!literals}

Literal arithmetic domain values are represented by a comma-separated
list of range expressions of matching \chpl{idxType} enclosed in
square brackets:

%
% TODO: Is it correct that they must be matching?
%

\begin{syntax}
domain-literal:
  [ range-expression-list ]

range-expression-list:
  range-expression
  range-expression, range-expression-list

range-expression:
  expression
\end{syntax}

\noindent The type of an arithmetic domain literal is defined as follows:

\begin{itemize}

\item \chpl{rank} = the number of range expressions in the literal

\item \chpl{idxType} = the type of the range expressions

\item \chpl{stridable} = \chpl{true} if any of the range expressions
are stridable, otherwise \chpl{false}

\end{itemize}

\begin{example}
The expression \chpl{[1..5, 1..5]} defines an arithmetic domain with
type \chpl{domain(rank=2,} \chpl{ idxType=int,} \chpl{ strided=false)}.
\end{example}

\begin{example}
The expression \chpl{[1..5, 1..5]} defines a $5 \times 5$ arithmetic
domain with the indices $(1, 1), (1, 2), \ldots, (1, 5), (2, 5), \ldots (5, 5)$.
\end{example}

\begin{example}
In the code
\begin{chapel}
var D: domain(2) = [1..n, 1..n];
\end{chapel}

\chpl{D} is defined as a two-dimensional, nonstridable arithmetic
domain with an index type of \chpl{2*int} and is initialized to
contain the set of indices $(i,j)$ for all $i$ and $j$ such that
$i \in {1, 2, \ldots, n}$ and $j \in {1, 2, \ldots, n}$.
\end{example}

\index{arithmetic domains!default value}

The default value of a range type is the \chpl{rank} default range
values for type \chpl{range(idxType,} \chpl{ BoundedRangeType.bounded,}
\chpl{stridable)}.

% TODO: example here

\subsubsection{Associative Domain Values}
\label{Associative_Domain_Values}
\index{associative domains!values}

An associative domain's value is simply the set of all index values
that the domain describes.  The indices of an associative domain are
typically unordered.  The only exception is associative domains over
enumerated types which are ordered according to the order in which the
enumeration's identifiers were declared.

\index{associative domains!literals, lack thereof}
\index{associative domains!initializations}

There is no literal syntax for an associative domain, though a tuple
of values of type \chpl{idxType} can be used to initialize a variable
of associative domain type.

% TODO: assign too?

\index{associative domains!default values}

The default value for an associative domain is the empty set unless
\chpl{idxType} is an enumerated type in which case the default value
is the set of all identifiers in the enumeration.

\begin{rationale}

The decision to have enumerated domains start fully populated was
based on the observation that enumerations have a finite, typically
small number of elements and that it would be common to declare arrays
with values corresponding to each identifier in the enumeration.
Furthermore, we considered it simpler to clear a fully-populated
domain than to fully populate an empty one.

In addition, we believe that fully-populated constant enumerated
domains are an important case for compiler optimizations, particularly
if the numeric values of the enumeration are consecutive.

\end{rationale}


\subsubsection{Opaque Domain Values}
\label{Opaque_Domain_Values}
\index{opaque domains!values}

An opaque domain's value is simply the unordered set of anonymous
indices that the domain describes.

\index{opaque domains!literals, lack thereof}

There is no literal syntax for an opaque domain due to the fact that
the indices have no inherent names.

\index{opaque domains!default values}

The default value for an opaque domain is the empty set.


\subsection{Subdomains}
\label{Subdomains}
\index{subdomains}
\index{domains!subdomains}

A subdomain is a domain whose indices are guaranteed to be a subset of
those described by another domain known as its \emph{parent domain}.
Subdomains have the same type as their parent domain, and by default
they inherit the domain map of their parent domain.  All domain types
support subdomains.

\begin{rationale}
Subdomains are provided in Chapel for a number of reasons: to
faciliate the ability of the compiler or a reader to reason about the
inter-relationship of distinct domain variables; to support the
author's ability to elide redundant domain mapping specifications; to
support the compiler's ability to reason about the relative alignment
of multiple domains; and to improve the compiler's ability to prove
away bounds checks for array accesses.
\end{rationale}

\subsubsection{Subdomain Types}
\label{Subdomain_Types}
\index{subdomains!types}
\index{types!subdomains}

A subdomain type is specified using the following syntax:
\begin{syntax}
subdomain-type:
  `sparse'[OPT] `subdomain' ( domain-expression )
\end{syntax}
This declares that \sntx{domain-expression} is the parent domain of
this subdomain type.  The subdomain type has the same type as its
parent domain.  By default it will share the parent domain's domain
map.  The optional \sntx{sparse} keyword permits the ability to
create sparse subdomains described in ~\rsec{Sparse_Subdomains}.

\begin{openissue}

An open semantic issue for subdomains is when a subdomain's subset
property should be re-verified when its parent domain is reassigned
and whether this should be done aggressively or lazily.

\end{openissue}


\subsection{Sparse Subdomains}
\label{Sparse_Subdomains}
\index{domains!sparse}
\index{subdomains!sparse}

 \emph{Sparse subdomains} are irregular domains that describe an
arbitrary subset of a domain, even if the parent domain is a regular
domain.  Sparse subdomains are useful in Chapel for
defining \emph{sparse arrays} in which a single element value occurs
frequently enough that it is worthwhile to avoid storing it
redundantly.  The difference between a sparse subdomain's index set
and its parent domain's describes the set of indices for which the
sparse array will store this replicated value.
See~\rsec{Sparse_Arrays} for details about sparse arrays.

\subsubsection{Sparse Domain Types}
\index{sparse domains!values}

Each root domain type has a unique corresponding sparse subdomain
type.  Sparse subdomains whose parent domains are also sparse
subdomains share the same type.

\subsubsection{Sparse Domain Values}
\label{Sparse_Domain_Values}
\index{sparse domains!values}

A sparse subdomain's value is simply the set of all index values that
the domain describes.  If the parent domain's indices were ordered,
the sparse subdomain's are as well.

\index{sparse domains!literals, lack thereof}
\index{sparse domains!initializations}
There is no literal syntax for an associative domain, though for a
domain \chpl{D}, a tuple of values of type \chpl{index(D)} can be used
to initialize a variable of sparse domain type.

\index{sparse domains!default value}
The default value for a sparse subdomain value is the empty set.

\begin{example}
The following code declares a 2D dense domain \chpl{D}, followed by a
2D sparse subdomain of \chpl{D} named \chpl{SpsD}.  Since \chpl{SpsD}
is uninitialized, it will initially describe the empty set of indices
from \chpl{D}.
\begin{chapel}
const D: domain(2) = [1..n, 1..n];
var SpsD: sparse subdomain(D);
\end{chapel}
\end{example}



\subsection{Index Types}
\label{Index_Types}
\index{domains!index types}

Each domain value has a corresponding compiler-provided \emph{index
type} which can be used to represent values belonging to that domain's
index set.  Index types are described using the following syntax:

\begin{syntax}
index-type:
  `index' ( domain-expression )
\end{syntax}

\begin{rationale}

Index types are included in Chapel with two goals in mind. The first
is to improve readability of the Chapel program by declaring variables
to be members of specific domains with the intention of giving them
more semantic meaning to a reader as compared to, say, storing all
indices as \chpl{int} types where the semantic meanings blur.

The second goal is to provide the compiler with the ability to prove
away bounds checks by giving the user the capability to assure the
compiler that a given variable belongs to a particular domain and is
therefore in bounds for its arrays and its parent domains' arrays.

Since index types are known to be legal for a given domain, they may
also afford the opportunity to represent an index using an optimized
format that doesn't simply store the index variable's value in order
to support accelerated access to arrays declared over that domain.
For both this reason and the previous, it may be less expensive to
index into arrays using index type variables of their domains or
subdomains.

\end{rationale}

\begin{openissue}

An open issue for index types is what the semantics should be for an
index type value that is live across a modification to its domain's
index set---particularly one that shrinks the index set.  Our
hypothesis is that most stored indices will either have short
lifespans or belong to constant or monotonically growing domains.  But
these semantics need to be defined nevertheless.

\end{openissue}


\subsection{Domain Assignment}
\label{Domain_Assignment}
\index{domains!assignment}

All domain types support domain assignment.  Domain assignment is by
value and causes the target domain variable to take on the index set
of the right-hand side expression.  In practice, the right-hand side
expression is often another domain value, a tuple of indices, or a
loop that enumerates indices.  If the domain variable being assigned
was used to declare arrays, these arrays are reallocated as discussed
in~\rsec{Association_of_Arrays_to_Domains}.

\begin{example}
The following three assignments show ways of assigning indices to a
sparse domain, \chpl{SpsD}.  The first assigns the domain two index
values, \chpl{(1,1)} and \chpl{(n,n)}.  The second assigns the domain
all of the indices along the diagonal from
\chpl{(1,1)}$\ldots$\chpl{(n,n)}.  The third invokes an iterator that
is written to \chpl{yield} indices read from a file named
``inds.dat''.  Each of these assignments has the effect of replacing
the previous index set with a completely new set of values.
\begin{chapel}
SpsD = ((1,1), (n,n));
SpsD = [i in 1..n] (i,i);
SpsD = readIndicesFromFile("inds.dat");
\end{chapel}
\end{example}

\subsection{Domain Index Set Manipulation}
\label{Domain_Index_Set_Manipulation}

\subsubsection{Querying Index Set Membership}
\label{Domain_Membership}
\index{domains!membership}

Every domain type supports a \chpl{member(i)} method that returns a
boolean value indicating whether or not the given index \chpl{i} is a
member of the domain's index set.

%
% would like to call the type of i above idxType, but it's not true
% for arithmetic...  motivation to unify
%

\subsubsection{Clearing a Domain's Index Set}
\label{Clearing_Domain_Indices}
\index{domains!clearing}

Every domain type supports a \chpl{clear()} method that resets a
domain's index set to its default value as specified
in~\rsec{Root_Default_Domain_Values}.

\begin{example}
The following call will cause the sparse domain \chpl{SpsD} to
describe an empty set of indices as it was when initially declared.
\begin{chapel}
SpsD.clear();
\end{chapel}
\end{example}

\begin{example}
The following call causes the associative domain \chpl{HashD} to
describe an empty set of indices as it did when it was initially
declared.
\begin{chapel}
HashD.clear();
\end{chapel}
\end{example}

\subsubsection{Adding and Removing Domain Indices}
\label{Adding_and_Removing_Domain_Indices}
\index{domains!adding indices}
\index{domains!removing indices}

All irregular domain types support the ability to incrementally add
and remove indices from their index sets.  This can either be done
using \chpl{add(i:idxType)} and \chpl{remove(i:idxType)} methods on a
domain variable or by using the \chpl{+=} and \chpl{-=} assignment
operators.  It is legal to add the same index to an irregular domain's
index set twice, but illegal to remove an index that does not belong
to the domain's index set.

%
% These remove semantics seem dangerous in a parallel context; maybe
% add flags to both the method versions of the call that say whether
% they should balk or not?  Or add exceptions...
%

As with normal domain assignments, arrays declared in terms of a
domain being modified in this way will be reallocated as discussed
in~\rsec{Association_of_Arrays_to_Domains}.



\subsection{Iteration over Domains}
\label{Iteration_over_Domains}
\index{domains!iteration}

All domains support iteration via standard for, forall, and coforall
loops.  These loops iterate over all of the indices that the domain
describes.  The type of the iterator variable for an iteration over a
domain named \chpl{D} is that domain's index type, \chpl{index(D)}.
If the domain's indices are ordered, a for loop will traverse the
indices in order.


\subsection{Slicing}
\label{Slicing}
\index{slicing}

In Chapel, slicing is the application of an index set to another
variable using either parentheses or square brackets.

% TODO: syntax diagram?!?

\subsubsection{Domain-based Slicing}
\index{slicing!domains}
\index{domains!slicing}

The index set used to express a slice can be represented using a
domain value.

Slicing an array results in an alias to a subset of the array's
elements as described in~\rsec{Array_Slicing}.

Slicing a domain evaluates to a new domain value whose index set is
the intersection of the domain's index set and the slicing index set.
The type and domain map of the result match the domain being sliced.

%
% TODO: can we say that it is an alias in the case of sparse/associative?
%

\subsubsection{Range-based Slicing}
\label{Range_Based_Slicing}
\index{slicing!range-based}
\index{range-based slicing}
When slicing arithmetic domains or arrrays, the slice can be expressed
as a list of \chpl{rank} ranges.  These ranges can either be bounded
or unbounded.  When unbounded, they inherit their bounds from the
domain or array being sliced.

\begin{example}
The following code declares a 2D arithmetic domain \chpl{D}, and then
a number of subdomains of \chpl{D} by slicing into \chpl{D} using
bounded and unbounded ranges.  The \chpl{InnerD} domain describes the
inner indices of D, \chpl{Col2OfD} describes the 2nd column of
\chpl{D}, and \chpl{AllButLastRow} describes all of \chpl{D} except
for the last row.

\begin{chapel}
const D: domain(2) = [1..n, 1..n],
      InnerD = D[2..n-1, 2..n-1],
      Col2OfD = D[.., 2..2],
      AllButLastRow = D[..n-1, ..];
\end{chapel}
\end{example}

\begin{openissue}
For slices that use a list of ranges, our intention is to use zipper
semantics vs. tensor semantics when evaluating the ranges depending on
whether square brackets or parentheses are used.  Currently all slices
are defined using tensor semantics for simplicity.  Since this may
change in the future, we recommend using square brackets to express
array-based slicing.
\end{openissue}

\subsubsection{Rank-Change Slicing}
\label{Rank_Change_Slicing}
\index{slicing!rank-change}
\index{rank-change slicing}

For multidimensional arithmetic domains and arrays, substituting
integral values for one or more of the ranges in a range-based slice
will result in domain or array of lower rank.

The result of a rank-change slice on an array is an alias to a subset
of the array's elements as described
in~\rsec{Arithmetic_Array_Slicing}.

The result of rank-change slice on a domain is a subdomain of the
domain being sliced, as described in~\rsec{Subdomains}.  The resulting
subdomain's type will be the same as the original domain, but with
a \chpl{rank} equal to the number of dimensions that were sliced by
ranges rather than integers.


\subsection{Domain Arguments to Functions}
\label{Domain_Arguments_to_Functions}
\index{domains!function arguments}
\index{function arguments!domains}

This section describes the semantics of passing domains as arguments
to functions.

\subsubsection{Formal Arguments of Domain Type}

When a domain value is passed to a formal argument of compatible
domain type by blank intent, it is passed by reference in order to
preserve the domain's identity.

% Need some syntax diagrams here maybe?


\subsubsection{Domain Promotion of Scalar Functions}
\label{Domain_Promotion_of_Scalar_Functions}
\index{domains!promotion}

Domain values may be passed to a scalar function argument whose type
matches the domain's index type.  This results in a promotion of the
scalar function as defined in~\rsec{Promotion}.

\begin{example}
Given a function \chpl{foo()} that accepts real floating point values
and an associative domain \chpl{D} of
type \chpl{domain(real)}, \chpl{foo} can be called with \chpl{D} as
its actual argument which will result in the function being invoked
for each value in the index set of \chpl{D}.
\end{example}

\begin{example}
Given an array \chpl{A} with element type \chpl{int} declared over a
one-dimensional domain \chpl{D} with \chpl{idxType} \chpl{int}, the
array elements can be assigned their corresponding index values by
writing:
\begin{chapel}
A = D;
\end{chapel}
This is equivalent to:
\begin{chapel}
forall (a,i) in (A,D) do
  a = i;
\end{chapel}
\end{example}


\subsection{Domain Operators}

\subsubsection{By Operator}
\label{By_Operator_For_Domains}
\index{by@\chpl{by}!on arithmetic domains}

The \chpl{by} operator can be applied to an arithmetic domain value in
order to create a strided arithmetic domain value.  The right-hand
operand to the \chpl{by} operator can either be an integral value or
an integral tuple whose size matches the domain's rank.

% syntax diagram?

The type of the resulting domain is the same as the original domain
but with \chpl{stridable} set to true.  In the case of an integer
stride value, the value of the resulting domain is computed by
applying the integer value to each range in the value using the
\chpl{by} operator.  In the case of a tuple stride value, the resulting
domain's value is computed by applying each tuple component to the
corresponding range using the \chpl{by} operator.



\subsection{Predefined Functions and Methods on Domains}
\index{domains!predefined functions}

\begin{protohead}
def $Domain$.numIndices: dim_type
\end{protohead}
\begin{protobody}
Returns the number of indices in the domain.
\end{protobody}

%% \begin{protohead} **/
%% def $Domain$.order(i: index($Domain$)): dim_type **/
%% \end{protohead} **/
%% \begin{protobody} **/
%% If \chpl{i} is a member of the domain, returns the ordinal value of **/
%% \chpl{i} using a total ordering of the domain's indices using 0-based **/
%% indexing.  Otherwise, it returns \chpl{(-1):dim_type}.  For arithmetic **/
%% domains, this ordering will be based on a row-major ordering of the **/
%% indices; for other domains, the ordering may be **/
%% implementation-dependent and unstable as elements are added and **/
%% removed from the domain. **/
%% \end{protobody} **/


\subsubsection{Predefined Functions and Methods on Arithmetic Domains}

\begin{protohead}
def $Domain$.dim(d: int): range
\end{protohead}
\begin{protobody}
Returns the range of indices described by dimension \chpl{d} of the
domain.
\end{protobody}

\begin{example}
In the code
\begin{chapel}
for i in D.dim(1) do
  for j in D.dim(2) do
    writeln(A(i,j));
\end{chapel}
domain \chpl{D} is iterated over by two nested loops.  The first
dimension of \chpl{D} is iterated over in the outer loop.  The second
dimension is iterated over in the inner loop.
\end{example}

% BLC: ``integral'' isn't really correct in the two 1D cases below,
% however, we don't really seem to have a user-level name for the
% per-dimension index type in the language that I can see.

\begin{protohead}
def $Domain$.low: index($Domain$)
\end{protohead}
\begin{protobody}
Returns the low index of the domain as a value of the domain's index
type.
\end{protobody}

\begin{protohead}
def $Domain$.high: index($Domain$)
\end{protohead}
\begin{protobody}
Returns the high index of the domain as a value of the domain's index
type.
\end{protobody}

\begin{protohead}
def $Domain$.translate(off: integral): domain
def $Domain$.translate(off: rank*integral): domain
\end{protohead}
\begin{protobody}
Returns a new domain that is the current domain translated
by \chpl{off} or \chpl{off(d)} for each dimension \chpl{d}.
\end{protobody}

\begin{protohead}
def $Domain$.expand(off: integral): domain
def $Domain$.expand(off: rank*integral): domain
\end{protohead}
\begin{protobody}
Returns a new domain that is the current domain adjusted by \chpl{off}
or \chpl{off(d)} for each dimension \chpl{d}.
\end{protobody}

\begin{protohead}
def $Domain$.exterior(off: integral): domain
def $Domain$.exterior(off: rank*integral): domain
\end{protohead}
\begin{protobody}
Returns a new domain that is the exterior portion of the domain
with \chpl{off} or \chpl{off(d)} indices for each dimension \chpl{d}.
If \chpl{off} or \chpl{off(d)} is negative, compute the exterior from
the low bound of the dimension; if positive, compute the exterior from
the high bound.
\end{protobody}

\begin{protohead}
def $Domain$.interior(off: integral): domain
def $Domain$.interior(off: rank*integral): domain
\end{protohead}
\begin{protobody}
Returns a new domain that is the interior portion of the domain
with \chpl{off} or \chpl{off(d)} indices for each dimension \chpl{d}.
If \chpl{off} or \chpl{off(d)} is negative, compute the interior from
the low bound of the dimension; if positive, compute the interior from
the high bound.
\end{protobody}

%% \begin{protohead} **/
%% def $Domain$.position(i: index($Domain$)): rank*dim_type **/
%% \end{protohead} **/
%% \begin{protobody} **/
%% Returns a tuple holding the order of index i in each range defining **/
%% the domain. **/
%% \end{protobody} **/
