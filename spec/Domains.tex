\sekshun{Domains}
\label{Domains}
\index{domains}

A \emph{domain} describes a collection of names for data.  These names
are referred to as the \emph{indices} of the domain.  All indices for
a particular domain are values with some common type.  Valid types for
indices are primitive types and class references or unions, tuples or
records whose fields are valid types for indices.  This excludes
ranges, domains, and arrays.  Domains have a rank and a total order on
their elements.

Chapel supports a variety of kinds of domains and arrays defined over
those domains as well as a mechanism to allow application-specific
implementations of arrays.

\subsection{Domains}
%\label{Domains}

Domains are first-class ordered sets of indices.  There are five kinds
of domains:
\begin{itemize}
\item
Arithmetic domains are rectilinear sets of Cartesian indices that can
have an arbitrary rank.
\item
Sparse domains are subdomains that support a notion of an implicit
``zero element'' for array elements described by its base domain but
not the domain itself.
\item
Associative domains are sets of indices where the type of the index is
some type that is not an array, domain, or range.  Associative domains
define dictionaries or associative arrays implemented via hash tables.
\item
Enumerated domains are a special case of associative domains where the
indices are defined by an enumerated type.
\item
Opaque domains are sets of anonymous indices.  Opaque domains define
graphs and unspecified sets.
\end{itemize}

\subsubsection{Domain Types}
\label{Domain_Types}
\index{domains!types}

Domain types vary based on the kind of the domain.  The type of an
arithmetic domain is parameterized by the rank of the domain and the
integral type of the indices.  The type of a sparse domain is
parameterized by the type of the domain that defines its bounding
index set.  The type of an associative domain is parameterized by the
type of the index.  The type of an opaque domain is unique.  The type
of an enumerated domain is parameterized by the enumerated type.

The syntax of a domain type is summarized as follows:
\begin{syntax}
domain-type:
  arithmetic-domain-type
  associative-domain-type
  opaque-domain-type
  enumerated-domain-type
  sparse-domain-type
  subdomain-type
\end{syntax}  

\begin{example}
In the code
\begin{chapel}
var D: domain(2) = [1..n, 1..n];
\end{chapel}
\chpl{D} is defined as a two-dimensional arithmetic domain and is
initialized to contain the set of indices $(i,j)$ for all $i$ and $j$
such that $i \in {1, 2, \ldots, n}$ and $j \in {1, 2, \ldots, n}$.
\end{example}

\subsubsection{Index Types}
\label{Index_Types}
\index{domains!index types}

Each domain has a corresponding \emph{index} type which is the type of
the domain's indices qualified by its status as an index.  Variables
of this type can be declared using the following syntax:
\begin{syntax}
index-type:
  `index' ( domain-expression )
\end{syntax}
If the type of the indices of the domain is \chpl{int}, then the index
type can be converted into this type.

A value with a type that is the same as the type of the indices in a
domain but is not the index type can be converted into the index type
using a special ``method'' called \chpl{index}.

\begin{example}
In the code
\begin{chapel}
var j = D.index(i);
\end{chapel}
the type of the variable \chpl{j} is the index type of
domain \chpl{D}.  The variable \chpl{i}, which must have the same type
as the underlying type of the indices of \chpl{D}, is verified to be
in domain \chpl{D} before it is assigned to \chpl{j}.
\end{example}

Values of index type are known to be valid and may have specialized
representations to facilitate accessing arrays defined for that
domain. It may therefore be less expensive to access arrays using
values of appropriate index type rather than values of the more
general type the domain is defined over.

\subsubsection{Domain Assignment}
\label{Domain_Assignment}
\index{domains!assignment}

Domain assignment is by value.  If arrays are declared over a domain,
domain assignment impacts these arrays as discussed
in~\rsec{Association_of_Arrays_to_Domains}, but the arrays remain
associated with the same domain regardless of the assignment.

\subsubsection{Formal Arguments of Domain Type}
\label{Formal_Arguments_of_Domain_Type}
\index{formal arguments!domains}
\index{domains!as formal arguments}

Domains are passed to functions by reference.  Formal arguments that
receive domains are aliases of the actual arguments.  It is a
compile-time error to pass a domain to a formal argument that has a
non-blank intent.

\subsubsection{Iteration over Domains}
\label{Iteration_over_Domains}

All domains support iteration via forall and for loops over the
indices in the set that the domain defines.  The type of the indices
returned by iterating over a domain is the index type of the domain.

\subsubsection{Domain Promotion of Scalar Functions}
\label{Domain_Promotion_of_Scalar_Functions}
\index{domains!promotion}

Domain promotion of a scalar function is defined over the domain type
and the type of the indices of the domain (not the index type).

\begin{example}
Given an array \chpl{A} with element type \chpl{int} declared over a
one-dimensional domain \chpl{D} with integral type \chpl{int}, then
the array can be assigned the values given by the indices in the
domain by writing
\begin{chapel}
A = D;
\end{chapel}
\end{example}


\subsection{Arithmetic Domains}
\label{Arithmetic_Domains}
\index{domains!arithmetic}

An arithmetic domain is a rectilinear set of Cartesian indices.
Arithmetic domains are specified as a tuple of ranges enclosed in
square brackets.

\subsubsection{Arithmetic Domain Literals}
\label{Arithmetic_Domain_Literals}
\index{domains!arithmetic literals}

An arithmetic domain literal is a square tuple of ranges.
\begin{example}
The expression \chpl{[1..5, 1..5]} defines a $5 \times 5$ arithmetic
domain with the indices $(1, 1), (1, 2), \ldots, (5, 5)$.
\end{example}

\subsubsection{Arithmetic Domain Types}
\label{Arithmetic_Domain_Types}

The type of an arithmetic domain is determined by three components:
(1) the rank of the arithmetic domain (the number of ranges that
define it); (2) an underlying integral type called the
\emph{dimensional index type} which must be identical to each of the
integral element types of the ranges that define the arithmetic
domain; (3) a boolean value indicating whether any of the ranges that
define the domain are stridable or not.  By default, the dimensional
index type of an arithmetic domain is \chpl{int} and the stridability
value is set to false.

The arithmetic domain type is specified by the syntax of a function
call to the keyword \chpl{domain} that takes at least an argument
called \chpl{rank} that is a parameter of type \chpl{int} and
optionally an integral type named \chpl{dim_type} and a boolean value
named \chpl{stridable}.  Its syntax is summarized as follows:
\begin{syntax}
arithmetic-domain-type:
  `domain' ( named-expression-list )
\end{syntax}

\begin{example}
The expression \chpl{[1..5, 1..5]} defines an arithmetic domain with
type \chpl{domain(2, int, false)}.
\end{example}

\subsubsection{Strided Arithmetic Domains}
\label{Strided_Arithmetic_Domains}
\index{domains!arithmetic, strided}

If the ranges that define an arithmetic domain are strided, then the
arithmetic domain is said to be strided and the stridable parameter
must be set to true.  For domains with inferred type, if the
initializing expression uses stridable ranges, the domain will be
inferred to have a stridable parameter of true.

The \chpl{by} operator can be applied to any arithmetic domain to
create a strided arithmetic domain.  It is predefined over an
arithmetic domain and an integer or a tuple of integers.  In the
integer case, the ranges in each dimension are strided by the integer.
In the tuple of integers case, the size of the tuple must match the
rank of the domain; the integers stride each dimension of the domain.
If the ranges are already strided, the strides applied by
the \chpl{by} operator are multiplied to the strides of the ranges.

\subsubsection{Arithmetic Domain Slicing}
\label{Arithmetic_Domain_Slicing}

Arithmetic domains support slicing by indexing into them specifying a
range per dimension.  Square brackets should be used for
multidimensional domains, while either square brackets or parenthesis
can be used for 1D domains.  

For multi-dimensional arithmetic domains, slicing with a rank change
is supported by substituting integral values within a dimension's
range for an actual range.  The resulting domain will have a rank less
than the arithmetic domain's rank and equal to the number of ranges
that are passed in to take the slice.

The result is a subdomain of the domain being sliced, as described
in~\rsec{Subdomains}, as defined by the intersection of the two
domains.  Partially unbounded or completely unbounded ranges may be
used to specify that the slice should extend to the domain's lower
and/or upper bound.

\begin{example}
The following code declares a 2D arithmetic domain \chpl{D}, and then
a number of subdomains of \chpl{D} by slicing into \chpl{D} using
bounded and unbounded ranges.  The \chpl{InnerD} domain describes the
inner indices of D, \chpl{Col2OfD} describes the 2nd column of
\chpl{D}, and \chpl{AllButLastRow} describes all of \chpl{D} except
for the last row.

\begin{chapel}
const D: domain(2) = [1..n, 1..n],
      InnerD = D[2..n-1, 2..n-1],
      Col2OfD = D[.., 2..2],
      AllButLastRow = D[..n-1, ..];
\end{chapel}
\end{example}


\subsection{Sparse Domains}
\label{Sparse_Domains}
\index{domains!sparse}

Sparse domains are used in Chapel to describe irregular index subsets
and to define sparse arrays.  Sparse arrays are typically used to
represent data aggregates in which a value occurs so frequently that
it would be wasteful to store it explicitly for each occurrence.  This
value is commonly described as the ``zero value'', though we refer to
it as the \emph{implicitly replicated value} or \emph{IRV} since it
may be a value other than zero.

\subsubsection{Sparse Domain Types}

A sparse domain type is specified by the syntax
\begin{syntax}
sparse-domain-type:
  `sparse' `subdomain' ( domain-expression )
\end{syntax}
This syntax specifies that the domain is a sparse subset of the
indices in the domain specified by the \sntx{domain-expression},
sometimes called the \emph{base domain} or \emph{parent domain}.

\begin{example}
The following code declares a 2D dense domain \chpl{D}, followed by a
2D sparse subdomain of \chpl{D} named \chpl{SpsD}.  Since \chpl{SpsD}
is uninitialized, it will initially describe the empty set of indices
from \chpl{D}.
\begin{chapel}
const D: domain(2) = [1..n, 1..n];
var SpsD: sparse subdomain(D);
\end{chapel}
\end{example}


\subsubsection{Sparse Domain Assignment}

Sparse domains can be assigned aggregates of indices from their parent
domain.  Common methods for expressing such aggregates are to use a
tuple of indices, a forall expression that enumerates indices, or an
iterator that generates indices.

\begin{example}
The following three assignments show ways of assigning indices to a
sparse domain, \chpl{SpsD}.  The first assigns the domain two index
values, \chpl{(1,1)} and \chpl{(n,n)}.  The second assigns the domain
all of the indices along the diagonal from
\chpl{(1,1)}$\ldots$\chpl{(n,n)}.  The third invokes an iterator that
is written to \chpl{yield} indices read from a file named
``inds.dat''.  Each of these assignments has the effect of replacing
the previous index set with a completely new set of values.
\begin{chapel}
SpsD = ((1,1), (n,n));
SpsD = [i in 1..n] (i,i);
SpsD = readIndicesFromFile("inds.dat");
\end{chapel}
\end{example}

Sparse domains can be emptied by using a method \chpl{clear} that
clears out its index set.

\begin{example}
The following call will cause the sparse domain \chpl{SpsD} to
describe an empty set of indices as it was when initially declared.
\begin{chapel}
SpsD.clear();
\end{chapel}
\end{example}

As with other domain types, reassigning a domain's index set will
cause arrays declared in terms of that domain to store elements
corresponding to the new indices of the domain.  These elements will
be initialized to the array's IRV by default.



\subsubsection{Modifying a Sparse Domain}

Indices can be incrementally added to or removed from sparse domains.
Sparse domains support a method \chpl{add} that takes an index and
adds it to the sparse domain's index set.  All arrays declared over
this sparse domain will now store an element corresponding to this
index, initialized to be its IRV.

Sparse domains support a method \chpl{remove} that takes an index and
removes this index from the sparse domain.  The values in the arrays
indexed by the removed index are lost.

The operators \chpl{+=} and \chpl{-=} have special semantics for
sparse domains; they are interpreted as calls to the \chpl{add}
and \chpl{remove} methods respectively.  The statement
\begin{chapel}
D += i;
\end{chapel}
is equivalent to
\begin{chapel}
D.add(i);
\end{chapel}
Similarly, the statement
\begin{chapel}
D -= i;
\end{chapel}
is equivalent to
\begin{chapel}
D.remove(i);
\end{chapel}

As with other methods and operators, the \chpl{add}, \chpl{remove},
\chpl{+=}, and \chpl{-=} operators can be invoked in a promoted manner
by specifying an aggregate of indices rather than a single index at a
time.

\subsection{Associative Domains}
\label{Associative_Domains}
\index{domains!associative}

An associative domain type can be defined over any scalar type and is
given by the following syntax:
\begin{syntax}
associative-domain-type:
  `domain' ( scalar-type )

scalar-type:
  type-specifier
\end{syntax}
A scalar type is any primitive type, tuple of scalar types, or class,
record, or union where all of the fields have scalar types.
Enumerated types are scalar types but domains declared over enumerated
types are described in~\rsec{Enumerated_Domains}.  Arrays
declared over associative domains are dictionaries mapping from values
to variables.

\subsubsection{Changing the Indices in Associative Domains}

As with sparse domains, indices can be added or removed to associative
domains.  Associative domains support a method \chpl{add} that takes
an index and adds this index to the associative domain.  All arrays
declared over this associative domain can now access elements
corresponding to this index.

Associative domains support a method \chpl{remove} that takes an index
and removes this index from the associative domain.  The values in the
arrays indexed by the removed index are lost.

The operators \chpl{+=} and \chpl{-=} have special semantics for
associative domains; they are interpreted as calls to the \chpl{add}
and \chpl{remove} methods respectively.  The statement
\begin{chapel}
D += i;
\end{chapel}
is equivalent to
\begin{chapel}
D.add(i);
\end{chapel}
Similarly, the statement
\begin{chapel}
D -= i;
\end{chapel}
is equivalent to
\begin{chapel}
D.remove(i);
\end{chapel}

Like sparse domains, associative domains can be emptied by using a
method \chpl{clear} that clears out its index set.

\begin{example}
The following call will cause the associative domain \chpl{HashD} to
describe an empty set of indices as it was when initially declared.
\begin{chapel}
HashD.clear();
\end{chapel}
\end{example}

\subsubsection{Testing Membership in Associative Domains}

An associative domain supports a \chpl{member} method that can test
whether a particular value is part of the index set. It
returns \chpl{true} if the index is in the associative domain and
otherwise returns \chpl{false}.

\subsection{Opaque Domains}
\label{Opaque_Domains}
\index{domains!opaque}

An opaque domain is a form of associative domain where there is no
algebra on the types of the indices.  The indices are, in essence,
opaque.  The opaque domain type is given by the following syntax:
\begin{syntax}
opaque-domain-type:
  `domain' ( `opaque' )
\end{syntax}

New index values for opaque domains are explicitly requested via a
method called \chpl{create}.  Indices can be removed by a method
called \chpl{remove}.

Opaque domains permit more efficient implementations than associative
domains under the assumption that creation of new domain index values
is rare.

\subsection{Enumerated Domains}
\label{Enumerated_Domains}
\index{domains!enumerated}

Enumerated domains are a special case of associative domains where the
indices are the constants defined by an enumerated type.  The syntax
of an enumerated domain type is summarized as follows:
\begin{syntax}
enumerated-domain-type:
  `domain' ( enum-type )
\end{syntax}

Enumerated domains are initialized with indices for each constant
defined in the enumerated type they are declared over.

An enumerated domain is specified identically to the associative
domain type, except that the type is an enumerated type rather than
some other value type.

\subsection{Subdomains}
\label{Subdomains}
\index{subdomains}
\index{domains!subdomains}

A subdomain is a domain whose indices are a subset of those described
by a \emph{base domain}.  A subdomain is specified by the following
syntax:
\begin{syntax}
subdomain-type:
  `subdomain' ( domain-expression )
\end{syntax}
The ordering of the indices in the subdomain is consistent with the
ordering of the indices in the base domain.

Subdomains are verified during execution even as domains are
reassigned.  The indices in a subdomain are known to be indices in a
domain, allowing for fast bounds-checking.

\subsection{Predefined Functions and Methods on Domains}
\index{domains!predefined functions}

There is an expectation that this list of predefined functions and
methods will grow.

\begin{protohead}
def $Domain$.numIndices: dim_type
\end{protohead}
\begin{protobody}
Returns the number of indices in the domain.
\end{protobody}

\begin{protohead}
def $Domain$.member(i: index($Domain$)): bool
\end{protohead}
\begin{protobody}
Returns whether or not index \chpl{i} is a member of the domain's
index set.
\end{protobody}

\begin{protohead}
def $Domain$.order(i: index($Domain$)): dim_type
\end{protohead}
\begin{protobody}
If \chpl{i} is a member of the domain, returns the ordinal value of
\chpl{i} using a total ordering of the domain's indices using 0-based
indexing.  Otherwise, it returns \chpl{(-1):dim_type}.  For arithmetic
domains, this ordering will be based on a row-major ordering of the
indices; for other domains, the ordering may be
implementation-dependent and unstable as elements are added and
removed from the domain.
\end{protobody}


\subsubsection{Predefined Functions and Methods on Arithmetic Domains}

We expect that this list of predefined functions and methods will
grow.

\begin{protohead}
def $Domain$.rank param
\end{protohead}
\begin{protobody}
Returns the rank of the domain.
\end{protobody}

\begin{protohead}
def $Domain$.dim(d: int): range
\end{protohead}
\begin{protobody}
Returns the range of indices described by dimension \chpl{d} of the
domain.
\end{protobody}

\begin{example}
In the code
\begin{chapel}
for i in D.dim(1) do
  for j in D.dim(2) do
    writeln(A(i,j));
\end{chapel}
domain \chpl{D} is iterated over by two nested loops.  The first
dimension of \chpl{D} is iterated over in the outer loop.  The second
dimension is iterated over in the inner loop.
\end{example}

% BLC: ``integral'' isn't really correct in the two 1D cases below,
% however, we don't really seem to have a user-level name for the
% per-dimension index type in the language that I can see.

\begin{protohead}
def $Domain$.low: $integral$        // for 1D domains
def $Domain$.low: index($Domain$)   // for multidimensional domains
\end{protohead}
\begin{protobody}
Returns the low index of the domain as a scalar value for 1D domains
and as an index value for a multidimensional domain.
\end{protobody}

\begin{protohead}
def $Domain$.high: $integral$        // for 1D domains
def $Domain$.high: index($Domain$)   // for multidimensional domains
\end{protohead}
\begin{protobody}
Returns the high index of the domain as a scalar value for 1D domains
and as an index value for a multidimensional domain.
\end{protobody}

\begin{protohead}
def $Domain$.position(i: index($Domain$)): rank*dim_type
\end{protohead}
\begin{protobody}
Returns a tuple holding the order of index i in each range defining
the domain.
\end{protobody}


