\sekshun{Memory Consistency Model}
\label{Memory_Consistency_Model}
\index{memory consistency model}

\begin{openissue}
  This chapter is a work-in-progress and represents an area where we
  are particularly interested in feedback from, and collaboration
  with, the broader community.
\end{openissue}

Chapel's memory consistency model is well-defined for programs that
are {\em data-race-free}.  Such programs are sequentially consistent.
For other programs, no specific guarantees can be made about the
program's execution.

\begin{rationale}
  Chapel presents a memory consistency model that is less strict than
  Java's.  It does so because it does not strive to provide the same
  dynamic security requirements as Java does.
\end{rationale}

Accessing a synchronization variable (\chpl{sync} or \chpl{single}, \rsec{Synchronization_Variables}) is
the only way to correctly synchronize between two Chapel tasks.
Such reads and writes serve as memory fences, preventing
reordering of reads and writes to traditional variables across the
synchronization variable's access.  When the same memory location
is written by one task and read by another, the ordering of the
read relative to the write is undefined unless there is an intervening
access to a synchronization variable.

%\begin{chapelexample}{syncFenceFlag}
\begin{example}
  In this example, a synchronization variable is used to (a) ensure that
  all writes to an array of unsynchronized variables are complete, (b)
  signal that fact to a second task, and (c) pass along the number of
  values that are valid for reading.

\begin{chapel}
var A: [1..100] real;
var done(*\texttt{\$}*): sync int;           // initially empty
cobegin {
  { // Reader task
    const numToRead = done(*\texttt{\$}*);   // block until writes are complete
    for i in 1..numToRead do
      writeln("A[", i, "] = ", A[i]);
  }
  {  // Writer task
    const numToWrite = 23;     // an arbitrary number
    for i in 1..numToWrite do
      A[i] = i/10.0;
    done(*\texttt{\$}*) = numToWrite;        // fence writes to A and signal done
  }
}
\end{chapel}
\begin{chapeloutput}
A[1] = 0.1
A[2] = 0.2
A[3] = 0.3
A[4] = 0.4
A[5] = 0.5
A[6] = 0.6
A[7] = 0.7
A[8] = 0.8
A[9] = 0.9
A[10] = 1.0
A[11] = 1.1
A[12] = 1.2
A[13] = 1.3
A[14] = 1.4
A[15] = 1.5
A[16] = 1.6
A[17] = 1.7
A[18] = 1.8
A[19] = 1.9
A[20] = 2.0
A[21] = 2.1
A[22] = 2.2
A[23] = 2.3
\end{chapeloutput}
%\end{chapelexample}
\end{example}


\begin{chapelexample}{syncSpinWait.chpl}
One consequence of Chapel's memory consistency model is that a task cannot spin-wait on a
variable waiting for another task to write to that variable.  The behavior of
the following code is undefined:

\begin{chapelpre}
if false { // }
\end{chapelpre}
\begin{chapel}
var x: int;
cobegin {
  while x != 1 do ;  // spin wait
  x = 1;
}
\end{chapel}
\begin{chapelnoprint}
// {
}
\end{chapelnoprint}
In contrast, spinning on a synchronization variable has well-defined
behavior:
\begin{chapel}
var x(*\texttt{\$}*): sync int;
cobegin {
  while x(*\texttt{\$}*).readXX() != 1 do ;  // spin wait
  x(*\texttt{\$}*).writeXF(1);
}
\end{chapel}
\begin{chapeloutput}
\end{chapeloutput}

In this code, the first statement in the cobegin statement executes a
loop until the variable is set to one.  The second statement in the
cobegin statement sets the variable to one.  Neither of these
statements block.
\end{chapelexample}


\begin{future}
Upon completion, Chapel's atomic statement~(\rsec{Atomic_Statement}) will serve as
an additional means of correctly synchronizing between tasks.
\end{future}

