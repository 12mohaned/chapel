\index{if@\chpl{if}}
\index{then@\chpl{then}}
\index{else@\chpl{else}}
\index{conditional statement}
The conditional statement allows execution to choose between two
statements based on the evaluation of an expression of bool type. The
syntax for a conditional statement is given by
\begin{syntax}
conditional-statement:
  `if' expression `then' statement [[ `else' statement ]]
  `if' expression block-level-statement [[ `else' statement ]]
\end{syntax}

A conditional statement evaluates an expression of bool type. If the
expression evaluates to true, the first statement or block level statement of
the conditional statement is executed.  If the expression evaluates to
false and the optional else-clause exists, the statement following the
\chpl{else} keyword is executed.

If the boolean expression contains only parameters that may be evaluated at
compile time, the compiler will replace the entire conditional statement with
the appropriate statement depending on the value of the boolean 
expression.

\index{conditional statement!dangling else}
If the conditional statement contains nested conditional statements, then 
the else-clause is bound to the nearest preceding conditional statement
without an else-clause.

Each statement embedded in the {\em conditional-statement} has its own
scope whether or not an explicit block surrounds it.
