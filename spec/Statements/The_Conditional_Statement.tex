\index{if@\chpl{if}}
\index{then@\chpl{then}}
\index{else@\chpl{else}}
\index{conditional statement}
The conditional statement allows execution to choose between two
statements based on the evaluation of an expression of bool type. The
syntax for a conditional statement is given by
\begin{syntax}
conditional-statement:
  `if' expression `then' statement [[ `else' statement ]]
  `if' expression block-level-statement [[ `else' statement ]]
\end{syntax}

A conditional statement evaluates an expression of bool type. If the
expression evaluates to true, the statement that immediately follows
the \chpl{then} keyword or, in the case where there is no \chpl{then}
keyword, the expression, is executed. If the expression evaluates to
false and the optional else-clause exists, the statement following the
\chpl{else} keyword is executed.

If the expression is a parameter, the conditional statement is folded
by the compiler. If the expression evaluates to true, the first
statement replaces the conditional statement. If the expression
evaluates to false, the second statement, if it exists, replaces the
conditional statement; if the second statement does not exist, the
conditional statement is removed.

\index{conditional statement!dangling else}

If the statement that immediately follows the optional \chpl{then}
keyword is a conditional statement and it is not in a block, the
else-clause is bound to the nearest preceding conditional statement
without an else-clause.

Each statement embedded in the {\em conditional-statement} has its own
scope whether or not an explicit block surrounds it.
