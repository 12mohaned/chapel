There are three kinds of type declaration statements given by
\begin{syntax}
type-declaration-statement:
  enum-declaration-statement
  structural-type-declaration-statement
  type-alias-declaration-statement
\end{syntax}

Enumerated types are declared in type declaration statements with the
following syntax:
\begin{syntax}
enum-declaration-statement:
  `enum' identifier { enum-constant [[,...]] } ;

enum-constant:
  identifier [[ = expression ]]
\end{syntax}
Enumerated types are discussed in Section~\ref{Enumerated_Types}.

Classes, records, and unions, collectively referred to as structural
types are declared in type declaration statements with the following
syntax:
\begin{syntax}
structural-type-declaration-statement:
  structural-type identifier [[ inherit-expression ]] {
    structural-type-statement [[...]] }

structural-type-statement:
  type-declaration-statement
  function-declaration-statement
  variable-declaration-statement
  `type' identifier ;

structural-type: one of
  `class' `record' `union'
\end{syntax}
Classes are discussed in Section~\ref{Classes}, records are discussed
in Section~\ref{Records}, and unions are discussed in
Section~\ref{Unions}.

Type aliases are declared in type declaration statements with the
following syntax:
\begin{syntax}
type-alias-declaration-statement:
  `type' identifier = type ;
\end{syntax}
Type aliases are discussed in Section~\ref{Type_Aliases}.
