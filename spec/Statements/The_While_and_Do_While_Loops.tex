There are two variants of the while loop in Chapel.  The syntax of the
while-do loop is given by:
\begin{syntax}
while-do-statement:
  `while' expression `do' statement
  `while' expression block-level-statement
\end{syntax}
The syntax of the do-while loop is given by:
\begin{syntax}
do-while-statement:
  `do' statement `while' expression ;
\end{syntax}
In both variants, the expression evaluates to a value of type \chpl{bool}
which determines when the loop terminates and control continues with
the statement following the loop.

The while-do loop is executed as follows:
\begin{enumerate}
\item The expression is evaluated.
\item If the expression evaluates to \chpl{false}, the statement is
  not executed and control continues to the statement following the
  loop.
\item If the expression evaluates to \chpl{true}, the statement is
  executed and control continues to step 1, evaluating the expression
  again.
\end{enumerate}

The do-while loop is executed as follows:
\begin{enumerate}
\item The statement is executed.
\item The expression is evaluated.
\item If the expression evaluates to \chpl{false}, control continues
  to the statement following the loop.
\item If the expression evaluates to \chpl{true}, control continues to
  step 1 and the the statement is executed again.
\end{enumerate}
In this second form of the loop, note that the statement is executed
unconditionally once.
