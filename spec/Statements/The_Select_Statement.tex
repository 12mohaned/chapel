\index{select@\chpl{select}}
\index{when@\chpl{when}}

The select statement is a multi-way variant of the conditional
statement.  The syntax is given by:
\begin{syntax}
select-statement:
  `select' expression { when-statement [[...]] }

when-statement:
  `when' expression [[,...]] `do' statement
  `when' expression [[,...]] block-level-statement
  `otherwise' statement
\end{syntax}
The expression that follows the keyword \chpl{select}, the select
expression, is compared with the list of expressions following the
keyword \chpl{when}, the case expressions, using the equality
operator \chpl{==}.  If the expressions cannot be compared with the
equality operator, a compile-time error is generated.  The first case
expression that contains an expression where that comparison
is \chpl{true} will be selected and control transferred to the
associated statement.  If the comparison is always \chpl{false}, the
statement associated with the keyword \chpl{otherwise}, if it exists,
will be selected and control transferred to it.  There may be at most
one \chpl{otherwise} statement and its location within the select
statement does not matter.

Each statement embedded in the {\em when-statement} has its own scope
whether or not an explicit block surrounds it.
