\lstdefinelanguage{chapel}
  {
    morekeywords={
      align, atomic,
      begin, bool, break, by,
      class, cobegin, coforall, complex, config, const, continue,
      delete, dmapped, do, domain,
      else, enum, extern, export,
      false, for, forall,
      if, imag, in, index, inline, inout, int, iter,
      label, lambda, let, local, locale,
      module,
      new, nil, noinit,
      on, opaque, otherwise, out,
      param, proc,
      range, real, record, reduce, ref, return,
      scan, select, serial, single, sparse, string, subdomain, sync,
      then, true, type,
      uint, union, use,
      var,
      when, where, while, with,
      yield,
      zip
    },
    sensitive=false,
    mathescape=true,
    morecomment=[l]{//},
    morecomment=[s]{/*}{*/},
    morestring=[b]",
}

\lstset{
    basicstyle=\footnotesize\ttfamily,
    keywordstyle=\bfseries,
    commentstyle=\em,
    showstringspaces=false,
    flexiblecolumns=false,
    numbers=left,
    numbersep=5pt,
    numberstyle=\tiny,
    numberblanklines=false,
    stepnumber=0,
    escapeinside={(*}{*)},
    language=chapel,
  }

%\newcommand{\chpl}[1]{\lstinline[language=chapel,basicstyle=\ttfamily,keywordstyle=\bfseries]!#1!}
\newcommand{\chpl}[1]{\lstinline[language=chapel,basicstyle=\small\ttfamily,keywordstyle=]!#1!}
\newcommand{\varname}[1]{\emph{#1}}
\newcommand{\typename}[1]{\emph{#1}}
\newcommand{\fnname}[1]{\chpl{#1}}

\lstnewenvironment{chapel}{\lstset{language=chapel,xleftmargin=2pc,stepnumber=0}}{}
\lstnewenvironment{invisible}{\lstset{language=chapel,xleftmargin=2pc,stepnumber=0,keywordstyle=\bfseries\color{white},basicstyle=\small\ttfamily\color{white}}}{}
\lstnewenvironment{chapel0}{\lstset{language=chapel,stepnumber=0}}{}

\lstnewenvironment{numberedchapel}{\lstset{language=chapel,xleftmargin=15pt,stepnumber=1}}{}

\lstnewenvironment{chapelcode}{\lstset{language=chapel,stepnumber=1}}{}

% Uses the same listing style as the {chapel} environment, but keyword
% formatting is turned off.  The argument is ignored in LaTeX
% but used to name the .good file during test extraction.
% The argument must be supplied but may be empty.
% If empty it defaults to null, which signals the test extractor to 
% autogenerate the .good file name as ``<test_name>.good''.
\lstnewenvironment{chapelprintoutput}[1]
  {\lstset{language=chapel,xleftmargin=2pc,stepnumber=0,keywordstyle=}}{}

\lstnewenvironment{commandline}{\lstset{keywordstyle=,xleftmargin=2pc}}{}

\lstnewenvironment{protohead}{\lstset{language=chapel,xleftmargin=0pc,belowskip=-10pt,stepnumber=0}}{}

\newenvironment{protobody}{\begin{description}\item[\quad\quad] }{\end{description}}
