\sekshun{Standard Modules}
\label{Standard_Modules}
\index{standard modules}
\index{modules!standard}
\index{modules!automatic}
\index{modules!optional}

Standard modules provide standard library support
and are available to every Chapel program.
The functions and other definitions of \emph{automatic} modules are always
available to a Chapel program.
\emph{Optional} modules can be accessed via use statements
(\rsec{Using_Modules}).  

The automatic modules
are as follows:

\begin{tabular}{lll}
\hspace{1pc} & \chpl{Base} & Basic routines \\
             & \chpl{IO}   & Input/output support \\
             & \chpl{Math} & Math routines \\
             & \chpl{Types} & Routines related to primitive types \\
\end{tabular}

There is an expectation that each of these modules will be extended
and that more standard modules will be defined over time.

% Should this be "craychapel"?
The \chpl{Base} module is described below. The other standard modules
are listed and documented here:

\hspace{.5in}
\url{http://chapel.cray.com/docs/latest/}


\section{Base}
\label{Standard}
\index{modules!Base@\chpl{Base}}

The \chpl{Base} module provides the following functions,
listed in alphabetical order.

\begin{protohead}
proc ascii(s: string): int
\end{protohead}
\begin{protobody}
Returns the ASCII code number of the first letter in the
argument \chpl{s}.
\end{protobody}

\begin{protohead}
proc assert(test: bool)
proc assert(test: bool, args...)
\end{protohead}
\begin{protobody}
If \chpl{test} is true, no action is taken.
If \chpl{test} is false,
prints an error message to stderr giving the location of the call to \chpl{assert}
in the Chapel source, followed by the remaining arguments to the call, if any,
then exits the program.
\end{protobody}

\begin{protohead}
proc complex.re: real
\end{protohead}
\begin{protobody}
Returns the real component of the complex number.
\end{protobody}

\begin{protohead}
proc complex.im: real
\end{protohead}
\begin{protobody}
Returns the imaginary component of the complex number.
\end{protobody}

\begin{protohead}
proc complex.=re(f: real)
\end{protohead}
\begin{protobody}
Sets the real component of the complex number to \chpl{f}.
\end{protobody}

\begin{protohead}
proc complex.=im(f: real)
\end{protohead}
\begin{protobody}
Sets the imaginary component of the complex number to \chpl{f}.
\end{protobody}

\begin{protohead}
proc exit(status: int)
\end{protohead}
\begin{protobody}
Exits the program with code \chpl{status}.
\end{protobody}

\begin{protohead}
proc halt()
proc halt(args...)
\end{protohead}
\begin{protobody}
Prints an error message to stderr giving the location of the call to \chpl{halt}
in the Chapel source, followed by the arguments to the call, if any,
then exits the program.
\end{protobody}

\begin{protohead}
proc string.length: int
\end{protohead}
\begin{protobody}
Returns the number of characters in the base expression of type string.
\end{protobody}

\begin{protohead}
proc max(x, y...?k)
\end{protohead}
\begin{protobody}
Returns the maximum of the arguments when compared using the
``greater-than'' operator.  The return type is inferred from the types
of the arguments as allowed by implicit conversions.
\end{protobody}

\begin{protohead}
proc min(x, y...?k)
\end{protohead}
\begin{protobody}
Returns the minimum of the arguments when compared using the
``less-than'' operator.  The return type is inferred from the types of
the arguments as allowed by implicit conversions.
\end{protobody}

\begin{protohead}
proc string.size: int
\end{protohead}
\begin{protobody}
Same as string.length.
\end{protobody}

\begin{protohead}
proc string.substring(x): string
\end{protohead}
\begin{protobody}
Returns a value of string type that is a substring of the base
expression.  If \chpl{x} is $i$, a value of type \chpl{int}, then the
result is the $i$th character.  If \chpl{x} is a range, the result is
the substring where the characters in the substring are given by the
values in the range.
\end{protobody}

\begin{protohead}
proc typeToString(type t) param : string
\end{protohead}
\begin{protobody}
Returns a string parameter that represents the name of the
type \chpl{t}.
\end{protobody}

\begin{protohead}
proc warning(s:string)
proc warning(args...)
\end{protohead}
\begin{protobody}
Prints a warning to stderr giving the location of the call to \chpl{warning}
in the Chapel source, followed by the argument(s) to the call.
\end{protobody}
