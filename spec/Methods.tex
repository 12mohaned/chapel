\sekshun{Methods}
\label{Methods}
\index{methods}
\index{methods!primary}
\index{methods!secondary}
\index{primary methods}
\index{secondary methods}

A \emph{method} is a procedure or iterator that is associated with an
expression known as the \emph{receiver}.

Methods are declared with the following syntax:
\begin{syntax}
method-declaration-statement:
  linkage-specifier[OPT] proc-or-iter this-intent[OPT] type-binding[OPT] function-name argument-list[OPT] 
    return-intent[OPT] return-type[OPT] where-clause[OPT] function-body

proc-or-iter:
  `proc'
  `iter'

this-intent:
  `param'
  `type'
  `ref'
  `const ref'
  `const'

type-binding:
  identifier .
  `(' expr `)' .

\end{syntax}

% Note: at the time `const ref` and `const` this intents were
% added, we didn't see any conceptual reason not to support
% `const in` and `in` this intents; it's just not a pressing concern.

Methods defined within the lexical scope of a class, record, or union
are referred to as \emph{primary methods}.  For such methods,
the \sntx{type-binding} is omitted and is taken to be the
innermost class, record, or union in which the method is defined.

Methods defined outside of such scopes are known as \emph{secondary
methods} and must have a \sntx{type-binding} (otherwise, they would
simply be standalone functions rather than methods).  Note that
secondary methods can be defined not only for classes, records, and
unions, but also for any other type (e.g., integers, reals, strings).

Secondary methods can be declared with a type expression instead of a
type identifier. In particular, if the \sntx{type-binding} is a
parenthesized expression, the compiler will evaluate that expression to
find the receiver type for the method. In that case, the method applies
only to receivers of that type. See also
\rsec{Creating_General_and_Specialized_Versions_of_a_Function}.

Method calls are described in \rsec{Method_Calls}.

The use of \sntx{this-intent} is described in \rsec{Method_receiver_and_this}.

\section{Method Calls}
\label{Method_Calls}
\index{method calls}
\index{methods!calling}
\index{methods!receiver}

A method is invoked with a method call, which is similar to a non-method
call expression, but it can include a receiver clause. The receiver
clause syntactically identifies a single argument by
putting it before the method name. That argument is the method receiver.
When calling a method from another method, or from within a class or
record declaration, the receiver clause can be omitted.

\begin{syntax}
method-call-expression:
  receiver-clause[OPT] expression ( named-expression-list )
  receiver-clause[OPT] expression [ named-expression-list ]
  receiver-clause[OPT] parenthesesless-function-identifier
\end{syntax}

The receiver-clause (or its absence) specifies the method's receiver
\rsec{Method_receiver_and_this}.

\begin{chapelexample}{defineMethod.chpl}
A method to output information about an instance of the \chpl{Actor}
class can be defined as follows:
\begin{chapelpre}
use useActor1;
\end{chapelpre}
\begin{chapel}
proc Actor.print() {
  writeln("Actor ", name, " is ", age, " years old");
}
\end{chapel}
\begin{chapelpost}
anActor.print();
delete anActor;
\end{chapelpost}
\begin{chapeloutput}
(Tommy, {name = Tommy, age = 27})
Actor Tommy is 27 years old
\end{chapeloutput}
This method can be called on an instance of the \chpl{Actor}
class, \chpl{anActor}, with the call expression \chpl{anActor.print()}.
\end{chapelexample}

The actual arguments supplied in the method call are bound to the
formal arguments in the method declaration following the rules specified for
procedures (\rsec{Functions}). The exception is the receiver
\rsec{Method_receiver_and_this}.

\section{The Method Receiver and the {\em this} Argument}
\label{Method_receiver_and_this}
\index{methods!receiver}
\index{this@\chpl{this}}
\index{classes!this@\chpl{this}}
\index{receiver}
\index{type methods}
\index{instance methods}
\index{methods!type}
\index{methods!instance}

A method's \emph{receiver} is an implicit formal argument
named \chpl{this} representing the expression on which the method is
invoked.  The receiver's actual argument is specified by the
\sntx{receiver-clause} of a method-call-expression as specified
in \rsec{Class_Field_Accesses}.  



% TODO: specify how the receiver affects the choice of the method.

\begin{chapelexample}{implicitThis.chpl}
Let class \chpl{C}, method \chpl{foo}, and function \chpl{bar} be
defined as
\begin{chapel}
class C {
  proc foo() {
    bar(this);
  }
}
proc bar(c: C) { writeln(c); }
\end{chapel}
\begin{chapelpost}
var c1: C = new C();
c1.foo();
delete c1;
\end{chapelpost}
\begin{chapeloutput}
{}
\end{chapeloutput}
Then given an instance of \chpl{C} called \chpl{c1}, the method
call \chpl{c1.foo()} results in a call to \chpl{bar} where the
argument is \chpl{c1}.  Within primary method \chpl{C.foo()}, the
(implicit) receiver formal has static type \chpl{C} and is referred to
as \chpl{this}.
\end{chapelexample}

Methods whose receivers are objects are called \emph{instance
methods}.  Methods may also be defined to have \chpl{type}
receivers---these are known as \emph{type methods}.

The optional \sntx{this-intent} is used to specify type methods, to
constrain a receiver argument to be a \chpl{param}, or to specify how
the receiver argument should be passed to the method.

A method whose \sntx{this-intent} is \chpl{type} defines a \emph{type
method}.  It can only be called on the type itself rather than on an
instance of the type.  When \sntx{this-intent} is \chpl{param}, it
specifies that the function can only be applied to param objects of
the given type binding.

\begin{chapelexample}{paramTypeThisIntent.chpl}
In the following code, the \chpl{isOdd} method is defined with
a \sntx{this-intent} of \chpl{param}, permitting it to be called on
params only.  The \chpl{size} method is defined with
a \sntx{this-intent} of \chpl{type}, requiring it to be called on
the \chpl{int} type itself, not on integer values.
\begin{chapel}
proc param int.isOdd() param {
  return this & 0x1 == 0x1;
}

proc type int.size() param {
  return 64;
}

param three = 3;
var seven = 7;

writeln(42.isOdd());          // prints false
writeln(three.isOdd());       // prints true
writeln((42+three).isOdd());  // prints true
// writeln(seven.isOdd());    // illegal since 'seven' is not a param

writeln(int.size());          // prints 64
// writeln(42.size());        // illegal since 'size()' is a type method
\end{chapel}
\begin{chapeloutput}
false
true
true
64
\end{chapeloutput}
\end{chapelexample}

Type methods can also be iterators.

\begin{chapelexample}{typeMethodIter.chpl}
In the following code, the class \chpl{C} defines a type method
iterator which can be invoked on the type itself:
\begin{chapel}
class C {
  var x: int;
  var y: string;

  iter type myIter() {
    yield 3;
    yield 5;
    yield 7;
    yield 11;
  }
}

for i in C.myIter() do
  writeln(i);
\end{chapel}
\begin{chapeloutput}
3
5
7
11
\end{chapeloutput}
\end{chapelexample}

When \sntx{this-intent} is \chpl{ref}, the receiver argument will be
passed by reference, allowing modifications to \chpl{this}.  If
\sntx{this-intent} is \chpl{const ref}, the receiver argument is passed
by reference but it cannot be modified inside the method. The
\sntx{this-intent} can also describe an abstract intent as follows. If it is
\chpl{const}, the receiver argument will be passed with \chpl{const}
intent. If it is left out entirely, the receiver will be passed with
a default intent. For records, that default intent is \chpl{ref} if
\chpl{this} is modified within the function and \chpl{const ref}
otherwise.  For other types, the default \chpl{this} intent matches the
default argument intent described in \rsec{The_Default_Intent}.

\begin{chapelexample}{refThisIntent.chpl}
In the following code, the \chpl{doubleMe} function is defined with a
\sntx{this-intent} of \chpl{ref}, allowing variables of type \chpl{int} to
double themselves.
\begin{chapel}
proc ref int.doubleMe() { this *= 2; }
\end{chapel}
\begin{chapelpost}
var x: int = 2;
x.doubleMe();
writeln(x);
\end{chapelpost}
\begin{chapeloutput}
4
\end{chapeloutput}
Given a variable \chpl{x = 2}, a call to \chpl{x.doubleMe()} will set \chpl{x}
to \chpl{4}.
\end{chapelexample}

\section{The {\em this} Method}
\label{The_this_Method}
\index{methods!indexing}
\index{this@\chpl{this}}
\index{methods!this@\chpl{this}}

A procedure method declared with the name \chpl{this} allows the receiver to be
``indexed'' similarly to how an array is indexed.  Indexing (as with
\chpl{A[1]}) has the semantics of calling a method
named \chpl{this}.  There is no other way to call a method
called \chpl{this}.  The \chpl{this} method must be declared with
parentheses even if the argument list is empty.

\begin{chapelexample}{thisMethod.chpl}
In the following code, the \chpl{this} method is used to create a
class that acts like a simple array that contains three integers
indexed by 1, 2, and 3.
\begin{chapel}
class ThreeArray {
  var x1, x2, x3: int;
  proc this(i: int) ref {
    select i {
      when 1 do return x1;
      when 2 do return x2;
      when 3 do return x3;
    }
    halt("ThreeArray index out of bounds: ", i);
  }
}
\end{chapel}
\begin{chapelpost}
var ta = new ThreeArray();
ta(1) = 1;
ta(2) = 2;
ta(3) = 3;
for i in 1..3 do
  writeln(ta(i));
ta(4) = 4;
\end{chapelpost}
\begin{chapeloutput}
1
2
3
thisMethod.chpl:9: error: halt reached - ThreeArray index out of bounds: 4
\end{chapeloutput}
\end{chapelexample}

\section{The {\em these} Method}
\label{The_these_Method}
\index{methods!iterating}
\index{these@\chpl{these}}
\index{methods!these@\chpl{these}}

An iterator method declared with the name \chpl{these} allows the
receiver to be ``iterated over'' similarly to how a domain or array
supports iteration.  Using a value in the context of a loop where an
\sntx{iteratable-expression} is expected has the semantics of calling a
method on the value named \chpl{these}.

\begin{chapelexample}{theseIterator.chpl}
In the following code, the \chpl{these} method is used to create a
class that acts like a simple array that can be iterated over and
contains three integers.
\begin{chapel}
class ThreeArray {
  var x1, x2, x3: int;
  iter these() ref {
    yield x1;
    yield x2;
    yield x3;
  }
}
\end{chapel}
\begin{chapelpost}
var ta = new ThreeArray();
for (i, j) in zip(ta, 1..) do
  i = j;

for i in ta do
  writeln(i);
delete ta;
\end{chapelpost}
\begin{chapeloutput}
1
2
3
\end{chapeloutput}

\end{chapelexample}

An iterator type method with the name \chpl{these} supports iteration
over the class type itself.

\begin{chapelexample}{typeMethodIterThese.chpl}
In the following code, the class \chpl{C} defines a type method
iterator named \chpl{these}, supporting direct iteration over the type:
\begin{chapel}
class C {
  var x: int;
  var y: string;

  iter type these() {
    yield 1;
    yield 2;
    yield 4;
    yield 8;
  }
}

for i in C do
  writeln(i);
\end{chapel}
\begin{chapeloutput}
1
2
4
8
\end{chapeloutput}
\end{chapelexample}
