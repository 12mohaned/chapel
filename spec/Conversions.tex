\sekshun{Conversions}
\label{Conversions}

A conversion allows an expression of one type to be converted into
another type.  Conversions can be either
implicit~(\rsec{Implicit_Conversions}) or
explicit~(\rsec{Explicit_Conversions}).

\section{Implicit Conversions}
\label{Implicit_Conversions}
\index{conversions!implicit}

Implicit conversions can occur during an assignment (from the
expression on the right-hand side to the variable on the left-hand
side) or during a function call (from the actual expression to the
formal argument).  An implicit conversion does not require a cast.

Implicit conversions are allowed between numeric
types~(\rsec{Implicit_Numeric_Conversions}), from enumerated types to
numeric types~(\rsec{Implicit_Enumeration_Conversions}), between class
types~(\rsec{Implicit_Class_Conversions}), and between record
types~(\rsec{Implicit_Record_Conversions}).  A special set of implicit
conversions are allowed from compile-time constants of type \chpl{int}
and \chpl{int(64)} to other smaller numeric types if the value is in
the range of the smaller numeric
type~(\rsec{Implicit_Compile_Time_Constant_Conversions}).  Lastly,
implicit conversions are supported from integral and class types to
bool in the context of a
statement~(\rsec{Implicit_Statement_Bool_Conversions}).

\subsection{Implicit Bool and Numeric Conversions}
\label{Implicit_Numeric_Conversions}
\index{conversions!numeric}
\index{conversions!bool}

Implicit conversions among fundamental numeric types are always allowed, so long as all
values that can be represented in the source type can also be represented in the
target type without loss of precision.
When the implicit conversion is from an integral to a real type, source
types are converted to type \chpl{int} before determining if the
conversion is valid.

In addition, implicit conversions from
types \chpl{int(64)} and \chpl{uint(64)} to types \chpl{real(64)}
and \chpl{complex(128)} are allowed, even though they may result in a loss of
precision.

%REVIEW: hilde
% Unless we are supporting some legacy behavior, I would recommend removing this
% provision.  A loss of precision is a loss of precision, so I would favor
% consistent behavior that does not lead to surprising results.  EVERY ``if''
% costs money: which is to say that if a behavior can be described simply, it can
% be implemented simply.
% A general description like the one above makes the spec comprehensive.  It
% will continue to be correct, even if additional types are added later -- while
% stating each permissible conversion explicitly is not only verbose, it is a
% path to the spec's obsolescence.
% The ``In addition'' clause above is inconsistent with the section on
% Explicit conversions, below.  One or the other should be fixed.

\begin{rationale}
In C\#, implicit conversions from \chpl{int(32)} or \chpl{int(64)}
to \chpl{real(32)} are supported and allow for a loss of precision.
Since the default \chpl{real} size is 64 and the default \chpl{int}
size is 32 in Chapel, we did not follow the lead of C\# in this regard
since it seemed unfortunate to favor \chpl{real(32)} over \chpl{real}
in the default case.  That is, given the \chpl{sqrt} function defined
over \chpl{real(32)} and \chpl{real}, it is preferable to choose the
version over \chpl{real} when calling with an actual of
type \chpl{int} rather than lose precision and half of the bits to
call the \chpl{real(32)} version.

Additionally, we don't allow implicit conversions from \chpl{int(8)}
or \chpl{int(16)} to \chpl{real(32)} because to do so would result in
an ambiguity when computing, e.g., \chpl{int(8) + int(8)}.
%NOTE:
%Both of these restrictions proceed naturally from the above, since conversion
%of an int(8) or int(16) to any real type involves first converting to int
%(which is int(32) by default), whereby the first legal implicit conversion to
%real is real(64).  Implicit conversions cannot imply a loss of precision, so
%there is no implicit conversion path to real(32).
\end{rationale}

Implicit numeric conversions may thus be tabulated as follows:
\begin{center}
\begin{tabular}{l|llllll|}
\cline{2-7}
& \multicolumn{6}{c|}{Target Type} \\
Source Type & bool($j$) & int($j$) & uint($j$) & real($j$) & imag($j$) & complex($j$) \\ \cline{2-7}
bool($i$) & all $i,j$ & all $i,j>=2$ & all $i,j$ & all $i,j$ & all $i,j$ & all $i,j$ \\
int($i$) & & $i<=j$ & $i<j$ & $i<=mant(j)$ & & $i<=mant(j)$ \\
uint($i$) & & $i<j$ & $i<=j$ & $i<smant(j)$ & & $i<mant(j)$ \\
real($i$) & & $mant(i)<=j$ & $mant(i)<j$ & $i<=j$ & & $2i<=j$ \\
imag($i$) & & & & & $i<=j$ & $2i<=j$ \\
complex($i$) & & & & & & $i<=j$ \\ \cline{2-7}
\end{tabular}
\end{center}
where $mant(i)$ represents the effective number of bits in the signed mantissa
of the corresponding floating-point type.  In this table the
default \chpl{int}, \chpl{uint} and \chpl{real} types map to their corresponding
explicitly-sized types.

\subsection{Implicit Enumeration Conversions}
\label{Implicit_Enumeration_Conversions}
\index{conversions!enumeration}

An expression that is an enumerated type can be implicitly converted
to any integral type as long as all of the constants defined by the
enumerated type are within range of the integral type.

% Note that the inverse conversion in not implicit, an explicit cast is required
% to convert an integer to an enumerated type.
% This is consistent with C# and later versions of C++.

\subsection{Implicit Class Conversions}
\label{Implicit_Class_Conversions}
\index{conversions!class}

An expression of class type \chpl{D} can be implicitly converted to
another class type \chpl{C} provided that \chpl{D} is a subclass
of \chpl{C}.

\subsection{Implicit Record Conversions}
\label{Implicit_Record_Conversions}
\index{conversions!record}

An expression of record type \chpl{D} can be implicitly converted to
another record type \chpl{C} provided that \chpl{D} is a nominal
subtype of \chpl{C}.

\subsection{Implicit Compile-Time Constant Conversions}
\label{Implicit_Compile_Time_Constant_Conversions}
\index{conversions!parameter}

The following two implicit conversions of parameters are supported:
\begin{itemize}
\item A parameter of type \chpl{int(32)} can be implicitly converted
to \chpl{int(8)}, \chpl{int(16)}, or any unsigned integral type if the
value of the parameter is within the range of the target type.
\item A parameter of type \chpl{int(64)} can be implicitly converted
to \chpl{uint(64)} if the value of the parameter is nonnegative.
\end{itemize}

\subsection{Implicit Statement Bool Conversions}
\label{Implicit_Statement_Bool_Conversions}
\index{conversions!bool}

In the condition of an if-statement, while-loop, and do-while-loop,
the following implicit conversions are supported:
\begin{itemize}
\item An expression of integral type is taken to be false if it is zero and is true otherwise.
\item An expression of a class type is taken to be false if it is nil and is true otherwise.
\end{itemize}

\section{Explicit Conversions}
\label{Explicit_Conversions}
\index{conversions!explicit}

Explicit conversions require a cast in the code.  Casts are defined
in~\rsec{Casts}.  Explicit conversions are supported between more
types than implicit conversions, but explicit conversions are not
supported between all types.

The explicit conversions are a superset of the implicit conversions.

\subsection{Explicit Numeric Conversions}
\label{Explicit_Numeric_Conversions}
\index{conversions!numeric}

Explicit conversions are allowed from any numeric type, bool, or
string to any other numeric type, bool, or string.  

% A valid \chpl{bool} value behaves like a single unsigned bit.  
When a \chpl{bool} is converted to a \chpl{bool}, \chpl{int}
or \chpl{uint} of equal or larger size, its value is zero-extended to fit the
new representation.  When a \chpl{bool} is converted to a
smaller \chpl{bool}, \chpl{int} or \chpl{uint}, its most significant
bits are truncated (as appropriate) to fit the new representation.
% This has the odd effect that a bool stored in a signed one-bit bitfield would
% change sign without generating a conversion error.  But its subsequent
% conversion back to a bool would yield the original value.
% In regard to supporting bitfields: Be careful what you wish for.

% The source type determines whether a value is zero- or sign-extended.
When an \chpl{int} is converted to a larger \chpl{int} or \chpl{uint}, its value is
sign-extended to fit the new representation.  
When a \chpl{uint} is converted to a larger \chpl{int} or \chpl{uint}, its value
is zero-extended.
When an \chpl{int} or \chpl{uint} is converted to an \chpl{int} or \chpl{uint}
of the same size, its binary representation is unchanged.
When an \chpl{int} or \chpl{uint} is converted to a smaller \chpl{int}
or \chpl{uint}, its value is truncated to fit the new representation.

\begin{future}
There are several kinds of integer conversion which can result in a loss of
precision.  Currently, the conversions are performed as specified, and no error
is reported.  In the future, we intend to improve type checking, so the user can
be informed of potential precision loss at compile time, and actual precision
loss at run time.  Such cases include:
%
% An exception is thrown if the source value cannot be represented in the target type.
When an \chpl{int} is converted to a \chpl{uint} and the original value is
negative;
When a \chpl{uint} is converted to an \chpl{int} and the sign bit of the result
is true;
When an \chpl{int} is converted to a smaller \chpl{int} or \chpl{uint} and any
of the truncated bits differs from the original sign bit;
%
When a \chpl{uint} is converted to a smaller \chpl{int} or \chpl{uint} and any
of the truncated bits is true;
\end{future}

\begin{rationale}
For integer conversions, the default behavior of a program should be to produce
a run-time error if there is a loss of precision.  Thus, cast expressions not only
give rise to a value conversion at run time, but amount to an assertion
that the required precision is preserved.  Explicit conversion procedures would be
available in the run-time library so that one can perform explicit conversions
that result in a loss of precision but do not generate a run-time diagnostic.
\end{rationale}

When converting from a \chpl{real} type to a larger \chpl{real} type, the
represented value is preserved.  When converting from a \chpl{real} type to a
smaller \chpl{real} type, the closest representation in the target type is
chosen.\footnote{When converting to a smaller real type, a loss of precision is \emph{expected}.
Therefore, there is no reason to produce a run-time diagnostic.}

When converting to a \chpl{real} type from an integer type, integer types
smaller than \chpl{int} are first converted to \chpl{int}.  Then, the closest
representation of the converted value in the target type is chosen.  The exact
behavior of this conversion is implementation-defined.

When converting from \chpl{real($k$)} to \chpl{complex($2k$)}, the original
value is copied into the real part of the result, and the imaginary part of the
result is set to zero.  When converting from a \chpl{real($k$)} to
a \chpl{complex($\ell$)} such that $\ell > 2k$, the conversion is performed as
if the original value is first converted to \chpl{real($\ell/2$)} and then
to \chpl{$\ell$}.

The rules for converting from \chpl{imag} to \chpl{complex} are the same as for
converting from real, except that the imaginary part of the result is set using
the input value, and the real part of the result is set to zero.

\subsection{Explicit Enumeration Conversions}
\label{Explicit_Enumeration_Conversions}
\index{conversions!enumeration}

Explicit conversions are allowed from any enumerated type to any
integer or real type, \chpl{bool}, or \chpl{string}, and vice versa.

When the target type is an integer type, the value is first converted to its
underlying integer type and then to the target type, following the rules above
for converting between integers.

When the target type is a real or complex type, the value is first converted to
its underlying integer type and then to the target type.

The conversion of an enumerated type to \chpl{imag} is not permitted.

When the target type is \chpl{bool}, the value is first converted to its
underlying integer type.  If the result is zero, the value of the \chpl{bool}
is \chpl{false}; otherwise, it is \chpl{true}.

When the target type is \chpl{string}, the value becomes the name of the
enumerator.  % in the execution character set.

When the source type is \chpl{bool}, enumerators corresponding to the values 0
and 1 in the underlying integer type are selected, corresponding to input values
of \chpl{false} and \chpl{true}, respectively.

%REVIEW: hilde
% As with default values for variables of enumerated types, I am pushing for the
% simplest implementation -- in which the conversion does not actually change
% the stored value.  This means that it may be possible for an enumerated variable
% to assume a value that does not correspond to any of its enumerators.  Further
% encouragement to always supply a default clause in your switch statements!

When the source type is a real or integer type, the value is converted to the
target type's underlying integer type.  

The conversion from \chpl{complex} and \chpl{imag} types to an enumerated type is not
permitted.

When the source type is string, the enumerator whose name matches value of the input
string is selected.  If no such enumerator exists, a runtime error occurs.

\subsection{Explicit Class Conversions}
\label{Explicit_Class_Conversions}
\index{conversions!class}

An expression of static class type \chpl{C} can be explicitly
converted to a class type \chpl{D} provided that \chpl{C} is derived
from \chpl{D} or \chpl{D} is derived from \chpl{C}.  In the event
that \chpl{D} is derived from \chpl{C}, it is a runtime error if the
the dynamic class type of \chpl{C} is not derived from or equal
to \chpl{D}.

\subsection{Explicit Record Conversions}
\label{Explicit_Record_Conversions}
\index{conversions!record}

An expression of record type \chpl{C} can be explicitly converted to
another record type \chpl{D} provided that \chpl{C} is derived
from \chpl{D}.  There are no explicit record conversions that are not
also implicit record conversions.
