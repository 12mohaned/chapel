\sekshun{Conversions}
\label{Conversions}
\index{conversions}

A \emph{conversion} converts an expression of one type to another type,
possibly changing its value.
\index{conversions!source type}
\index{conversions!target type}
We refer to these two types the \emph{source} and \emph{target} types.
Conversions can be either
implicit~(\rsec{Implicit_Conversions}) or
explicit~(\rsec{Explicit_Conversions}).


\section{Implicit Conversions}
\label{Implicit_Conversions}
\index{conversions!implicit}

An \emph{implicit conversion} is a conversion that occurs implicitly,
that is, not due to an explicit specification in the program.
Implicit conversions occur at the locations in the program listed below.
Each location determines the target type.
The source and target types of an implicit conversion must be allowed.
They determine whether and how the expression's value changes.

Implicit conversions are not applied when initializing \chpl{ref} or
\chpl{type} values or for actual arguments passed to \chpl{ref} or
\chpl{type} formal arguments.

\index{conversions!implicit!occurs at}
An implicit conversion occurs at each of the following program locations:

\begin{itemize}
\item In an assignment, the expression on the right-hand side of
      the assignment is converted to the type of the variable
      or another lvalue on the left-hand side of the assignment.

\item The actual argument of a function call or an operator is converted
      to the type of the corresponding formal argument, if the formal's
      intent is \chpl{param}, \chpl{in}, \chpl{const in}, or an abstract intent
      (\rsec{Abstract_Intents}) with the semantics of
      \chpl{in} or \chpl{const in}.

% MPF: This rule doesn't seem to be implemented right now,
%      but rather reflects ideal language design.
\item If the formal argument's intent is \chpl{out}, the formal argument
      is converted to the type of the corresponding actual argument
      upon function return.

\item The return or yield expression within a function without a \chpl{ref}
      return intent is converted to the return type of that function.

\item The condition of a conditional expression,
      conditional statement, while-do or do-while loop statement
      is converted to the boolean type~(\rsec{Implicit_Statement_Bool_Conversions}).
      A special rule defines the allowed source types and
      how the expression's value changes in this case.
\end{itemize}

\index{conversions!implicit!allowed types}
Implicit conversions \emph{are allowed} between
the following source and target types,
as defined in the referenced subsections:

\begin{itemize}
\item numeric, boolean, and enumerated types~(\rsec{Implicit_NumBoolEnum_Conversions}),
\item class types~(\rsec{Implicit_Class_Conversions}),
\item integral types in the special case when the expression's value
      is a compile-time constant~(\rsec{Implicit_Compile_Time_Constant_Conversions}), and
\item from an integral or class type to \chpl{bool}
      in certain cases~(\rsec{Implicit_Statement_Bool_Conversions}).
\end{itemize}

In addition,
an implicit conversion from a type to the same type is allowed for any type.
Such conversion does not change the value of the expression.

% TODO: If an implicit conversion is not allowed, it is an error.

Implicit conversion is not transitive. That is, if an implicit conversion
is allowed from type \chpl{T1} to \chpl{T2} and from \chpl{T2} to \chpl{T3},
that by itself does not allow an implicit conversion
from \chpl{T1} to \chpl{T3}.

\subsection{Implicit Numeric, Bool and Enumeration Conversions}
\label{Implicit_NumBoolEnum_Conversions}

\index{conversions!numeric}
\index{conversions!implicit!numeric}
Implicit conversions among numeric types are allowed when
all values representable in the source type can also be represented
in the target type, retaining their full precision.
%
%REVIEW: vass: I did not understand the point of the following,
% so I am commenting it out for now.
%When the implicit conversion is from an integral to a real type, source
%types are converted to type \chpl{int} before determining if the
%conversion is valid.
%
In addition, implicit conversions from
types \chpl{int(64)} and \chpl{uint(64)} to types \chpl{real(64)}
and \chpl{complex(128)} are allowed, even though they may result in a loss of
precision.

%REVIEW: hilde
% Unless we are supporting some legacy behavior, I would recommend removing this
% provision.  A loss of precision is a loss of precision, so I would favor
% consistent behavior that does not lead to surprising results.  EVERY ``if''
% costs money: which is to say that if a behavior can be described simply, it can
% be implemented simply.

\begin{rationale}
We allow these additional conversions because they are an important
convenience for application programmers. Therefore we are willing to
lose precision in these cases. The largest real and complex types
are chosen to retain precision as often as as possible.
\end{rationale}

\index{conversions!boolean}
\index{conversions!implicit!boolean}
Any boolean type can be implicitly converted to any other boolean type,
retaining the boolean value.
Any boolean type can be implicitly converted to any integral type
by representing \chpl{false} as 0 and \chpl{true} as 1,
except (if applicable)
a boolean cannot be converted to \chpl{int(1)}.
% Rationale: because 1 cannot be represented by \chpl{int(1)}.

\begin{rationale}
We disallow implicit conversion of a boolean to
a real, imaginary, or complex type because of the following.
We expect that the cases where such a conversion is needed
will more likely be unintended by the programmer.
Marking those cases as errors will draw the programmer's attention.
If such a conversion is actually desired, a cast \rsec{Explicit_Conversions}
can be inserted.
\end{rationale}

\index{conversions!enumerated types}
\index{conversions!implicit!enumerated types}
An expression of an enumerated type can be implicitly converted
to an integral type, provided that all of the constants defined by the
enumerated type are representable by the integral type.

% Requiring an explicit cast to convert an integer to an enumerated type
% is consistent with C# and later versions of C++.

Legal implicit conversions with numeric, boolean and enumerated types
may thus be tabulated as follows:

\begin{center}
\begin{tabular}{l|llllll}
& \multicolumn{6}{c}{Target Type} \\ [4pt]

Source Type  & bool($t$) & uint($t$) & int($t$) & real($t$) & imag($t$) & complex($t$) \\  [3pt]

\cline{1-7} \\

bool($s$)    & all $s,t$ & all $s,t$   & all $s$; $2 \le t$ & & & \\ [7pt]

enum         &           & (see rules) & (see rules)        & & & \\ [7pt]

uint($s$)    & & $s \le t$ & $s < t$   & $s \le mant(t)$   & & $s \le mant(t/2)$   \\ [7pt]

uint(64)     & &           &           & real(64)          & & complex(128)        \\ [7pt]

int($s$)     & &           & $s \le t$ & $s \le mant(t)+1$ & & $s \le mant(t/2)+1$ \\ [7pt]

int(64)      & &           &           & real(64)          & & complex(128)        \\ [7pt]

real($s$)    & & & & $s \le t$ &           & $s \le t/2$ \\ [7pt]

imag($s$)    & & & &           & $s \le t$ & $s \le t/2$ \\ [7pt]

complex($s$) & & & &           &           & $s \le t$   \\ [5pt]

\end{tabular}
\end{center}
Here, $mant(i)$ is the number of bits in the (unsigned) mantissa of
the $i$-bit floating-point type.\footnote{For the IEEE 754 format,
$mant(32)=24$ and $mant(64)=53$.}
%
Conversions for the default integral and real types (\chpl{uint},
\chpl{complex}, etc.) are the same as for their
explicitly-sized counterparts.

\subsection{Implicit Compile-Time Constant Conversions}
\label{Implicit_Compile_Time_Constant_Conversions}
\index{conversions!numeric!parameter}
\index{conversions!implicit!parameter}

The following implicit conversion of a parameter is allowed:
\begin{itemize}
\item A parameter of type \chpl{int(64)} can be implicitly converted
to \chpl{int(8)}, \chpl{int(16)}, \chpl{int(32)}, or any unsigned integral type if the
value of the parameter is within the range of the target type.
\end{itemize}

\subsection{Implicit Statement Bool Conversions}
\label{Implicit_Statement_Bool_Conversions}
\index{conversions!boolean!in a statement}
\index{conversions!implicit!boolean}

In the condition of an if-statement, while-loop, and do-while-loop,
the following implicit conversions to \chpl{bool} are supported:
\begin{itemize}
\item An expression of integral type is taken to be false if it is zero and is true otherwise.
\item An expression of a class type is taken to be false if it is nil and is true otherwise.
\end{itemize}

\section{Explicit Conversions}
\label{Explicit_Conversions}
\index{conversions!explicit}

Explicit conversions require a cast in the code.  Casts are defined
in~\rsec{Casts}.  Explicit conversions are supported between more
types than implicit conversions, but explicit conversions are not
supported between all types.

The explicit conversions are a superset of the implicit conversions.
In addition to the following definitions,
an explicit conversion from a type to the same type is allowed for any type.
Such conversion does not change the value of the expression.

\subsection{Explicit Numeric Conversions}
\label{Explicit_Numeric_Conversions}
\index{conversions!numeric}
\index{conversions!explicit!numeric}

Explicit conversions are allowed from any numeric type, boolean, or
string to any other numeric type, boolean, or string.  

% A valid \chpl{bool} value behaves like a single unsigned bit.  
When a \chpl{bool} is converted to a \chpl{bool}, \chpl{int}
or \chpl{uint} of equal or larger size, its value is zero-extended to fit the
new representation.  When a \chpl{bool} is converted to a
smaller \chpl{bool}, \chpl{int} or \chpl{uint}, its most significant
bits are truncated (as appropriate) to fit the new representation.

When a \chpl{int}, \chpl{uint}, or \chpl{real} is converted to a \chpl{bool}, the result is \chpl{false} if the number was equal to 0 and \chpl{true} otherwise.
% This has the odd effect that a bool stored in a signed one-bit bitfield would
% change sign without generating a conversion error.  But its subsequent
% conversion back to a bool would yield the original value.
% In regard to supporting bitfields: Be careful what you wish for.

% The source type determines whether a value is zero- or sign-extended.
When an \chpl{int} is converted to a larger \chpl{int} or \chpl{uint}, its value is
sign-extended to fit the new representation.  
When a \chpl{uint} is converted to a larger \chpl{int} or \chpl{uint}, its value
is zero-extended.
When an \chpl{int} or \chpl{uint} is converted to an \chpl{int} or \chpl{uint}
of the same size, its binary representation is unchanged.
When an \chpl{int} or \chpl{uint} is converted to a smaller \chpl{int}
or \chpl{uint}, its value is truncated to fit the new representation.

\begin{future}
There are several kinds of integer conversion which can result in a loss of
precision.  Currently, the conversions are performed as specified, and no error
is reported.  In the future, we intend to improve type checking, so the user can
be informed of potential precision loss at compile time, and actual precision
loss at run time.  Such cases include:
%
% An exception is thrown if the source value cannot be represented in the target type.
When an \chpl{int} is converted to a \chpl{uint} and the original value is
negative;
When a \chpl{uint} is converted to an \chpl{int} and the sign bit of the result
is true;
When an \chpl{int} is converted to a smaller \chpl{int} or \chpl{uint} and any
of the truncated bits differs from the original sign bit;
%
When a \chpl{uint} is converted to a smaller \chpl{int} or \chpl{uint} and any
of the truncated bits is true;
\end{future}

\begin{rationale}
For integer conversions, the default behavior of a program should be to produce
a run-time error if there is a loss of precision.  Thus, cast expressions not only
give rise to a value conversion at run time, but amount to an assertion
that the required precision is preserved.  Explicit conversion procedures would be
available in the run-time library so that one can perform explicit conversions
that result in a loss of precision but do not generate a run-time diagnostic.
\end{rationale}

When converting from a \chpl{real} type to a larger \chpl{real} type, the
represented value is preserved.  When converting from a \chpl{real} type to a
smaller \chpl{real} type, the closest representation in the target type is
chosen.\footnote{When converting to a smaller real type, a loss of precision is \emph{expected}.
Therefore, there is no reason to produce a run-time diagnostic.}

When converting to a \chpl{real} type from an integer type, integer types
smaller than \chpl{int} are first converted to \chpl{int}.  Then, the closest
representation of the converted value in the target type is chosen.  The exact
behavior of this conversion is implementation-defined.

When converting from \chpl{real($k$)} to \chpl{complex($2k$)}, the original
value is copied into the real part of the result, and the imaginary part of the
result is set to zero.  When converting from a \chpl{real($k$)} to
a \chpl{complex($\ell$)} such that $\ell > 2k$, the conversion is performed as
if the original value is first converted to \chpl{real($\ell/2$)} and then
to \chpl{$\ell$}.

The rules for converting from \chpl{imag} to \chpl{complex} are the same as for
converting from real, except that the imaginary part of the result is set using
the input value, and the real part of the result is set to zero.

\subsection{Explicit Tuple to Complex Conversion}
\label{Explicit_Tuple_to_Complex_Conversion}
\index{conversions!tuple to complex}
\index{conversions!explicit!tuple to complex}

A two-tuple of numerical values may be converted to a \chpl{complex} value.  If
the destination type is \chpl{complex(128)}, each member of the two-tuple must
be convertible to \chpl{real(64)}.  If the destination type
is \chpl{complex(64)}, each member of the two-tuple must be convertible
to \chpl{real(32)}.  The first member of the tuple becomes the real part of the
resulting complex value; the second member of the tuple becomes the imaginary
part of the resulting complex value.

\subsection{Explicit Enumeration Conversions}
\label{Explicit_Enumeration_Conversions}
\index{conversions!enumeration}
\index{conversions!explicit!enumeration}

Explicit conversions are allowed from any enumerated type to any
integer or real type, \chpl{bool}, or \chpl{string}, and vice versa.

When the target type is an integer type, the value is first converted to its
underlying integer type and then to the target type, following the rules above
for converting between integers.

When the target type is a real or complex type, the value is first converted to
its underlying integer type and then to the target type.

The conversion of an enumerated type to \chpl{imag} is not permitted.

When the target type is \chpl{bool}, the value is first converted to its
underlying integer type.  If the result is zero, the value of the \chpl{bool}
is \chpl{false}; otherwise, it is \chpl{true}.

When the target type is \chpl{string}, the value becomes the name of the
enumerator.  % in the execution character set.

When the source type is \chpl{bool}, enumerators corresponding to the values 0
and 1 in the underlying integer type are selected, corresponding to input values
of \chpl{false} and \chpl{true}, respectively.

%REVIEW: hilde
% As with default values for variables of enumerated types, I am pushing for the
% simplest implementation -- in which the conversion does not actually change
% the stored value.  This means that it may be possible for an enumerated variable
% to assume a value that does not correspond to any of its enumerators.  Further
% encouragement to always supply a default clause in your switch statements!

When the source type is a real or integer type, the value is converted to the
target type's underlying integer type.  

The conversion from \chpl{complex} and \chpl{imag} types to an enumerated type is not
permitted.

When the source type is string, the enumerator whose name matches value of the input
string is selected.  If no such enumerator exists, a runtime error occurs.

\subsection{Explicit Class Conversions}
\label{Explicit_Class_Conversions}
\index{conversions!class}
\index{conversions!explicit!class}

An expression of static class type \chpl{C} can be explicitly
converted to a class type \chpl{D} provided that \chpl{C} is derived
from \chpl{D} or \chpl{D} is derived from \chpl{C}.

When at run time the source expression refers to an instance of
\chpl{D} or it subclass, its value is not changed.
Otherwise, or when the source expression is \chpl{nil},
the result of the conversion is \chpl{nil}.

\subsection{Explicit Record Conversions}
\label{Explicit_Record_Conversions}
\index{conversions!records}
\index{conversions!explicit!records}

An expression of record type \chpl{C} can be explicitly converted to
another record type \chpl{D} provided that \chpl{C} is derived
from \chpl{D}.  There are no explicit record conversions that are not
also implicit record conversions.

\subsection{Explicit Range Conversions}
\label{Explicit_Range_Conversions}
\index{conversions!range}
\index{conversions!explicit!range}

An expression of stridable range type can be explicitly converted
to an unstridable range type, changing the stride to 1 in the process.

\subsection{Explicit Domain Conversions}
\label{Explicit_Domain_Conversions}
\index{conversions!domain}
\index{conversions!explicit!domain}

An expression of stridable domain type can be explicitly converted
to an unstridable domain type, changing all strides to 1 in the process.

\subsection{Explicit Type to String Conversions}
\label{Explicit_Type_to_String_Conversions}
\index{conversions!type to string}
\index{conversions!explicit!type to string}

A type expression can be explicitly converted to a \chpl{string}. The resultant
\chpl{string} is the name of the type.

\begin{chapelexample}{explicit-type-to-string.chpl}
For example:
\begin{chapel}
var x: real(64) = 10.0;
writeln(x.type:string);
\end{chapel}
\begin{chapeloutput}
real(64)
\end{chapeloutput}
This program will print out the string \chpl{"real(64)"}.
\end{chapelexample}
