\sekshun{Types}
\label{Types}
\index{types}

Chapel is a statically typed language with a rich set of types.  These
include a set of predefined primitive types, enumerated types,
structured types (classes, records, unions, tuples),
data parallel types (ranges, domains, arrays), and synchronization
types (sync, single).

The syntax of a type is as follows:

\begin{syntax}
type-specifier:
  primitive-type
  enum-type
  structured-type
  dataparallel-type
  synchronization-type
\end{syntax}

Programmers can define their own enumerated types, classes, records,
unions, and type aliases using type declaration statements:

\begin{syntax}
type-declaration-statement:
  enum-declaration-statement
  class-declaration-statement
  record-declaration-statement
  union-declaration-statement
  type-alias-declaration-statement
\end{syntax}

These statements are defined in Sections \rsec{Enumerated_Types},
\rsec{Class_Declarations}, \rsec{Record_Declarations},
\rsec{Union_Declarations}, and \rsec{Type_Aliases}, respectively.

\section{Primitive Types}
\label{Primitive_Types}
\index{types!primitive}

%REVIEW: vass: we should explain what separates primitive types from the rest.
The primitive types are: \chpl{void}, \chpl{bool},
\chpl{int}, \chpl{uint}, \chpl{real}, \chpl{imag}, \chpl{complex},
and \chpl{string}.  They are defined
in this section.

The primitive types are summarized by the following syntax:
\begin{syntax}
primitive-type:
  `void'
  `bool' primitive-type-parameter-part[OPT]
  `int' primitive-type-parameter-part[OPT]
  `uint' primitive-type-parameter-part[OPT]
  `real' primitive-type-parameter-part[OPT]
  `imag' primitive-type-parameter-part[OPT]
  `complex' primitive-type-parameter-part[OPT]
  `string'

primitive-type-parameter-part:
  ( integer-parameter-expression )

integer-parameter-expression:
  expression
\end{syntax}

If present, the parenthesized \sntx{integer-parameter-expression} must
evaluate to a compile-time constant of integer type.  See~\rsec{Compile-Time_Constants}

\begin{openissue}
There is an expectation of future support for larger bit width
primitive types depending on a platform's native support for those
types.
\end{openissue}

\subsection{The Void Type}
\label{The_Void_Type}
\index{void@\chpl{void}}
\index{types!void@\chpl{void}}

The \chpl{void} type is used to represent the lack of a value, for
example when a function has no arguments and/or no return type.  

There may be storage associated with a value of type \chpl{void}, in which
case its lifetime obeys the same rules as a value of type \chpl{int}.

\subsection{The Bool Type}
\label{The_Bool_Type}
\index{bool@\chpl{bool}}
\index{types!bool@\chpl{bool}}

Chapel defines a logical data type designated by the symbol
\chpl{bool} with the two predefined values \chpl{true} and
\chpl{false}.  This default boolean type is stored using an
implementation-defined number of bits.  A particular number of bits
can be specified using a parameter value following the \chpl{bool}
keyword, such as \chpl{bool(8)} to request an 8-bit boolean value.
Legal sizes are 8, 16, 32, and 64 bits.

%% The relational operators return values of \chpl{bool} type and the
%% logical operators operate on values of \chpl{bool} type.

Some statements require expressions of \chpl{bool} type and Chapel
supports a special conversion of values to \chpl{bool} type when used
in this context~(\rsec{Implicit_Statement_Bool_Conversions}).

\subsection{Signed and Unsigned Integral Types}
\label{Signed_and_Unsigned_Integral_Types}
\index{uint@\chpl{uint}}
\index{int@\chpl{int}}
\index{types!uint@\chpl{uint}}
\index{types!int@\chpl{int}}

The integral types can be parameterized by the number of bits used to
represent them.  Valid bit-sizes are 8, 16, 32, and 64.  The default signed integral type, \chpl{int}, and the
default unsigned integral type, \chpl{uint} correspond to 
\chpl{int(64)} and \chpl{uint(64)} respectively.

% BLC: I either don't understand this or don't believe it's true:
%
%, are treated as distinct types (for
%the purposes of type resolution) but behave like 

The integral types and their ranges are given in the following table:

\begin{center}
\begin{tabular}{|l|r|r|}
\hline
{\bf Type} & {\bf Minimum Value} & {\bf Maximum Value} \\
\hline
{\tt int(8)} & -128 & 127 \\
{\tt uint(8)} & 0 & 255 \\
{\tt int(16)} & -32768 & 32767 \\
{\tt uint(16)} & 0 & 65535 \\
{\tt int(32)} & -2147483648 & 2147483647 \\
{\tt uint(32)} & 0 & 4294967295 \\
{\tt int(64)}, {\tt int} & -9223372036854775808 & 9223372036854775807 \\
{\tt uint(64)}, {\tt uint} & 0 & 18446744073709551615 \\
\hline
\end{tabular}
\end{center}

The unary and binary operators that are pre-defined over the integral
types operate with 32- and 64-bit precision.  Using these operators on
integral types represented with fewer bits results in an implicit
conversion to the corresponding 32-bit types
according to the rules defined in~\rsec{Implicit_Conversions}.


\subsection{Real Types}
\label{Real_Types}
\index{real@\chpl{real}}
\index{types!real@\chpl{real}}

Like the integral types, the real types can be parameterized by the
number of bits used to represent them.  The default real
type, \chpl{real}, is 64 bits.  The real types that are supported are
machine-dependent, but usually include \chpl{real(32)} (single
precision) and \chpl{real(64)} (double precision) following the IEEE
754 standard.  

\subsection{Imaginary Types}
\label{Imaginary_Types}
\index{imaginary@\chpl{imaginary}}
\index{types!imaginary@\chpl{imaginary}}

The imaginary types can be parameterized by the number of bits used to
represent them.  The default imaginary type, \chpl{imag}, is 64 bits.
The imaginary types that are supported are machine-dependent, but
usually include \chpl{imag(32)} and \chpl{imag(64)}.

\begin{rationale}
The imaginary type is included to avoid numeric instabilities and
under-optimized code stemming from always converting real values to
complex values with a zero imaginary part.
\end{rationale}

\subsection{Complex Types}
\label{Complex_Types}
\index{complex@\chpl{complex}}
\index{types!complex@\chpl{complex}}

Like the integral and real types, the complex types can be
parameterized by the number of bits used to represent them.  A complex
number is composed of two real numbers so the number of bits used to
represent a complex is twice the number of bits used to represent the
real numbers.  The default complex type, \chpl{complex}, is 128 bits;
it consists of two 64-bit real numbers.  The complex types that are
supported are machine-dependent, but usually
include \chpl{complex(64)} and \chpl{complex(128)}.

The real and imaginary components can be accessed via the methods
\chpl{re} and \chpl{im}.  The type of these components is real.
See~\rsec{Math} for math routines for complex types.

\begin{example}
Given a complex number \chpl{c} with the value \chpl{3.14+2.72i}, the
expressions \chpl{c.re} and \chpl{c.im} refer to \chpl{3.14}
and \chpl{2.72} respectively.
\end{example}

\subsection{The String Type}
\label{The_String_Type}
\index{string@\chpl{string}}
\index{types!string@\chpl{string}}

Strings are a primitive type designated by the symbol \chpl{string}
comprised of ASCII characters.  Their length is unbounded.
See~\rsec{Standard} for routines for manipulating strings.


\begin{openissue}
There is an expectation of future support for fixed-length strings.
\end{openissue}

\begin{openissue}
There is an expectation of future support for different character
sets, possibly including internationalization.
\end{openissue}


\section{Enumerated Types}
\label{Enumerated_Types}
\index{enumerated types}
\index{types!enumerated}

Enumerated types are declared with the following syntax:

\begin{syntax}
enum-declaration-statement:
  `enum' identifier { enum-constant-list }

enum-constant-list:
  enum-constant
  enum-constant , enum-constant-list[OPT]

enum-constant:
  identifier init-part[OPT]

init-part:
  = expression
\end{syntax}

The enumerated type can then be referenced by its name, as summarized
by the following syntax:

\begin{syntax}
enum-type:
  identifier
\end{syntax}

An enumerated type defines a set of named constants that can be
referred to via a member access on the enumerated type.
These constants are treated as parameters of integral type.  Each
enumerated type is a distinct type. If the \sntx{init-part} is
omitted, the \sntx{enum-constant} has an integral value one higher
than the previous \sntx{enum-constant} in the enum, with the first
having the value \chpl{1}.

\begin{chapelexample}{enum.chpl}
The code
\begin{chapel}
enum statesman { Aristotle, Roosevelt, Churchill, Kissinger }
\end{chapel}
defines an enumerated type with four constants.  The function
\begin{chapel}
proc quote(s: statesman) {
  select s {
    when statesman.Aristotle do
       writeln("All paid jobs absorb and degrade the mind.");
    when statesman.Roosevelt do
       writeln("Every reform movement has a lunatic fringe.");
    when statesman.Churchill do
       writeln("A joke is a very serious thing.");
    when statesman.Kissinger do
       { write("No one will ever win the battle of the sexes; ");
         writeln("there's too much fraternizing with the enemy."); }
  }
}
\end{chapel}
\begin{chapelnoprint}
for s in statesman.Aristotle..statesman.Kissinger do
  quote(s:statesman);
\end{chapelnoprint}
\begin{chapeloutput}
All paid jobs absorb and degrade the mind.
Every reform movement has a lunatic fringe.
A joke is a very serious thing.
No one will ever win the battle of the sexes; there's too much fraternizing with the enemy.
\end{chapeloutput}
outputs a quote from the given statesman.  Note that enumerated
constants must be prefixed by the enumerated type and a dot.
\end{chapelexample}


\section{Structured Types}
\label{Structured_Types}
\index{types!structured}

The structured types are summarized by the following syntax:

\begin{syntax}
structured-type:
  class-type
  record-type
  union-type
  tuple-type
\end{syntax}
% in README.firstClassFns: function-type

Classes are discussed in \rsec{Classes}.  Records are discussed
in \rsec{Records}.  Unions are discussed in \rsec{Unions}.  Tuples are
discussed in \rsec{Tuples}.

\subsection{Class Types}
\label{Types_Class_Types}

% TODO: the first sentence does not make strict sense.
% Overall, it does not feel like this paragraph has any value here.
The class type defines a type that contains variables and constants,
called fields, and functions, called methods.  Classes are defined
in~\rsec{Classes}.  The class type can also contain type aliases and
parameters.  Such a class is generic and is defined
in~\rsec{Generic_Types}.

\subsection{Record Types}
\label{Types_Record_Types}

The record type is similar to a class type; the primary difference is
that a record is a value rather than a reference.  Records are defined
in~\rsec{Records}.

\subsection{Union Types}
\label{Types_Union_Types}

The union type defines a type that contains one of a set of variables.
Like classes and records, unions may also define methods.  Unions are
defined in~\rsec{Unions}.

\subsection{Tuple Types}
\label{Types_Tuple_Types}

A tuple is a light-weight record that consists of one or more
anonymous fields.  If all the fields are of the same type, the tuple
is homogeneous.  Tuples are defined in~\rsec{Tuples}.

\section{Data Parallel Types}
\label{Data_Parallel_Types}
\index{types!dataparallel}

The data parallel types are summarized by the following syntax:

\begin{syntax}
dataparallel-type:
  range-type
  domain-type
  mapped-domain-type
  array-type
  index-type
\end{syntax}

Ranges and their index types are discussed in \rsec{Ranges}.
Domains and their index types are discussed in \rsec{Domains}.
Arrays are discussed in \rsec{Arrays}.

\subsection{Range Types}
\label{Types_Range_Types}

A range defines an integral sequence of some integral type.  Ranges
are defined in \rsec{Ranges}.

\subsection{Domain, Array, and Index Types}
\label{Domain_and_Array_Types}

A domain defines a set of indices. An array defines a set of
elements that correspond to the indices in its domain.
A domain's indices can be of any type.
Domains, arrays, and their index
types are defined in \rsec{Domains} and \rsec{Arrays}.

\section{Synchronization Types}
\label{Synchronization_Types}
\index{types!synchronization}

The synchronization types are summarized by the following syntax:

\begin{syntax}
synchronization-type:
  sync-type
  single-type
\end{syntax}

Sync and single types are discussed in \rsec{Synchronization_Variables}.

\section{Type Aliases}
\label{Type_Aliases}
\index{types!aliases}

Type aliases are declared with the following syntax:
\begin{syntax}
type-alias-declaration-statement:
  `config'[OPT] `type' type-alias-declaration-list ;
  external-type-alias-declaration-statement

type-alias-declaration-list:
  type-alias-declaration
  type-alias-declaration , type-alias-declaration-list

type-alias-declaration:
  identifier = type-specifier
  identifier
\end{syntax}
A type alias is a symbol that aliases the type specified in the
\sntx{type-part}.  A use of a type alias has the same meaning as using
the type specified by \sntx{type-part} directly.

If the keyword \chpl{config} precedes the keyword \chpl{type}, the
type alias is called a configuration type alias.  Configuration type
aliases can be set at compilation time via compilation flags or other
implementation-defined means.  The \chpl{type-specifier} in the
program is ignored if the type-alias is alternatively set.

If the keyword \chpl{extern} precedes the \chpl{type} keyword, the type alias is
external.  The declared type name is used by Chapel for type resolution, but no
type alias is generated by the backend.  See the chapter on interoperability
(\rsec{Interoperability}) for more information on external types.

The \sntx{type-part} is optional in the definition of a class or
record.  Such a type alias is called an unspecified type
alias. Classes and records that contain type aliases, specified or
unspecified, are generic~(\rsec{Type_Aliases_in_Generic_Types}).

\begin{openissue}
There is on going discussion on whether a type alias is a new
type or simply an alias.  The former should enable redefinition of
default values, identity elements, etc.
%hilde
% Would inheritance work?
\end{openissue}
