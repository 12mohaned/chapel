\sekshun{Iterators}
\label{Iterators}
\index{functions!iterators}
\index{iterators}

An \emph{iterator} is a function that can generate, or \emph{yield}, multiple values (consecutively or in parallel) via its yield statements.

\begin{openissue}
The parallel iterator story is under development.  It is expected that
the specification will be expanded regarding parallel iterators soon.
\end{openissue}

\section{Iterator Definitions}
\label{Iterator_Function_Definitions}
\index{iterators!definition}

The syntax to declare an iterator is given
by:
\begin{syntax}
iterator-declaration-statement:
  privacy-specifier[OPT] `iter' iterator-name argument-list[OPT] return-intent[OPT] return-type[OPT] where-clause[OPT]
  iterator-body

iterator-name:
  identifier

iterator-body:
  block-statement
  yield-statement
\end{syntax}

The syntax of an iterator declaration is similar to a procedure declaration, with
some key differences:
\begin{itemize}
\item The keyword \chpl{iter} is used instead of the keyword \chpl{proc}.
\item The name of the iterator cannot overload any operator.
\item \chpl{yield} statements may appear in the body of an iterator, but not in
a procedure.
\item A \chpl{return} statement in the body of an iterator is not allowed to have an expression.
\end{itemize}

\section{The Yield Statement}
\label{The_Yield_Statement}
\index{yield@\chpl{yield}}
\index{iterators!yield@\chpl{yield}}

The yield statement can only appear in iterators.  The syntax of the
yield statement is given by
\begin{syntax}
yield-statement:
  `yield' expression ;
\end{syntax}

When an iterator is executed and a \chpl{yield} is encountered, the value of the yield
expression is returned.  However, the state of execution of the iterator is
saved.  On its next invocation, execution resumes from the point immediately
following that \chpl{yield} statement and with the saved state of execution.
A yield statement in a variable iterator must contain an lvalue expression.

When a \chpl{return} is encountered, the iterator finishes without yielding another
index value.  The \chpl{return} statements appearing in an iterator are not
permitted to have a return expression.
An iterator also completes after the last
statement in the iterator is executed.
An iterator need not contain any yield statements.
% TODO: the current implementation requires at least one yield

% TODO specify how the return type is established/checked for an iterator,
% analogously to \label{Return_Types}.

\section{Iterator Calls}
\label{Iterator_Calls}
\index{iterators!calls}

Iterators are invoked using regular call expressions:
\begin{syntax}
iteratable-call-expression:
  call-expression
\end{syntax}

All details of iterator calls, including argument passing, function
resolution, the use of parentheses versus brackets to delimit the parameter
list, and so on,
are identical to procedure calls as described in Chapter~\ref{Functions}.

However, the result of an iterator call depends upon its context, as described below.

\subsection{Iterators in For and Forall Loops}
\label{Iterators_in_For_and_Forall_Loops}
\index{iterators!in for loops}
\index{iterators!in forall loops}

When an iterator is accessed via a for or forall loop, the iterator is
evaluated alongside the loop body in an interleaved manner.  For each
iteration, the iterator yields a value and the body is executed.

\subsection{Iterators as Arrays}
\label{Iterators_as_Arrays}
\index{iterators!and arrays}

If an iterator function is captured into a new variable declaration or
assigned to an array, the iterator is iterated over in total and the
expression evaluates to a one-dimensional arithmetic array that
contains the values returned by the iterator on each iteration.
\begin{chapelexample}{as-arrays.chpl}
Given this iterator
\begin{chapel}
iter squares(n: int): int {
  for i in 1..n do
    yield i * i;
}
\end{chapel}
\begin{chapelpost}
writeln(squares(5));
\end{chapelpost}
the expression \chpl{squares(5)} evaluates to
\begin{chapelprintoutput}{}
1 4 9 16 25
\end{chapelprintoutput}
\end{chapelexample}

\subsection{Iterators and Generics}
\label{Iterators_and_Generics}
\index{iterators!and generics}

An iterator call expression can be passed to a generic function argument that
has neither a declared type nor default value
(\rsec{Formal_Arguments_without_Types}).
In this case the iterator is passed without being evaluated.
Within the generic function the corresponding formal argument
can be used as an iterator, e.g. in for loops.
The arguments to the iterator call expression, if any, are evaluated
at the call site, i.e. prior to passing the iterator to the generic function.

\subsection{Recursive Iterators}
\label{Recursive_Iterators}
\index{iterators!recursive}

Recursive iterators are allowed. A recursive iterator invocation is
typically made by iterating over it in a loop.


\begin{chapelexample}{recursive.chpl}
A post-order traversal of a tree data structure could be written like this:
\begin{chapelnoprint}
class Tree {
  var data: string;
  var left, right: Tree;
  proc ~Tree() {
    if left then delete left;
    if right then delete right;
  }
}

var tree = new Tree("a", new Tree("b"), new Tree("c", new Tree("d"), new Tree("e")));
\end{chapelnoprint}
\begin{chapel}
iter postorder(tree: Tree): string {
  if tree != nil {
    for child in postorder(tree.left) do
      yield child;
    for child in postorder(tree.right) do
      yield child;
    yield tree.data;
  }
}
\end{chapel}
\begin{chapelnoprint}
proc Tree.writeThis(x: Writer)
{
  var first = true;
  for node in postorder(tree) {
    if first then first = false;
      else x.write(" ");
    write(node);
  }
}
writeln("Tree Data");
writeln(tree);
delete tree;
\end{chapelnoprint}
By contrast, using calls \chpl{postorder(tree.left)}
and \chpl{postorder(tree.right)} as stand-alone statements would
result in generating temporary arrays containing the outcomes of these
recursive calls, which would then be discarded.
\begin{chapeloutput}
Tree Data
b d e c a
\end{chapeloutput}
\end{chapelexample}

\section{Parallel Iterators}
\label{Parallel_Iterators}
\index{parallel iterators}
\index{iterators!parallel}

Iterators used in explicit forall-statements or -expressions must be
parallel iterators.  Reductions, scans and promotion over serial
iterators will be serialized.

Parallel iterators are defined for standard constructs in Chapel such
as ranges, domains, and arrays, thereby allowing these constructs to
be used with forall-statements and -expressions.

The left-most iteratable expression in a forall-statement or
-expression determines the number of tasks, the iterations each task
executes, and the locales on which these tasks execute.  For ranges,
default domains, and default arrays, these values can be controlled
via configuration constants~(\rsec{data_parallel_knobs}).

Domains and arrays defined using distributed domain maps will
typically implement forall loops with multiple tasks on multiple
locales.  For ranges, default domains, and default arrays, all tasks
are executed on the current locale.

A more detailed definition of parallel iterators is forthcoming.
