\sekshun{Iterators}
\label{Iterators}
\index{iterators}

An iterator is a function that conceptually returns a sequence of
values rather than simply a single value.  Classes can be viewed as
iterators if they implement a structural iterator interface.

\subsection{Iterator Functions}
\label{Iterator_Functions}
\index{iterator@\chpl{iterator}}

The syntax of an iterator declaration is identical to that of a
function declaration except that the keyword \chpl{def} is replaced
with the keyword \chpl{iterator}.  The body of the iterator may
include yield statements alongside return statements.  When a yield is
encountered, the value is returned, but the iterator is not finished
evaluating.  It will continue from the point after the yield and can
yield or return more values.  When a return is encountered, the value
is returned and the iterator finishes.  An iterator also completes
after the last statement in the iterator function is executed.

\subsection{The Yield Statement}
\label{The_Yield_Statement}
\index{yield@\chpl{yield}}

Yield statement can only appear in iterators.  The syntax of the yield
statement is given by
\begin{syntax}
yield-statement:
  `yield' expression ;
\end{syntax}

\subsection{Iterator Calls}
\label{Iterator_Calls}

Iterator functions can be called within for or forall loops, in which
case they are executed in an interleaved manner with the body of the
loop, or can be called in any expression context, in which case they
evaluate to a sequence of values.

\subsubsection{Iterators in For and Forall Loops}
\label{Iterators_in_For_and_Forall_Loops}

When an iterator is accessed via a for or forall loop, the iterator is
evaluated alongside the loop body in an interleaved manner.  For each
iteration, the iterator yields a value and the body is executed.

\subsubsection{Iterators as Sequences}
\label{Iterators_as_Sequences}
\index{iterators!and sequences}

If an iterator function is accessed outside of the context of a for or
forall loop iterator expression, then the iterator is iterated over in
total and the expression evaluates to a sequence that contains the
values returned by the iterator on each iteration.
\begin{example}
Given an iterator
\begin{chapel}
iterator squares(n: int): int {
  for i in 1..n do
    yield i * i;
}
\end{chapel}
the expression \chpl{squares(5)} evaluates to the sequence \chpl{(/1, 4, 9, 16, 25/)}.
\end{example}
