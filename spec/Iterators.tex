\sekshun{Iterators}
\label{Iterators}
\index{iterators}

An iterator is a function that conceptually returns a sequence of
values rather than simply a single value.  Classes can be viewed as
iterators if they implement a structural iterator interface.

\subsection{Iterator Functions}
\label{Iterator_Functions}
\index{iterator@\chpl{iterator}}

The syntax of a function declaration is identical to that of a
function declaration except that the keyword \chpl{def} is replaced
with the keyword \chpl{iterator}.  The body of the iterator may
include yield statements alongside return statements.  When a yield is
encountered, the value is returned, but the iterator is not finished
evaluating.  It will continue from the point after the yield and can
yield or return more values.  When a return is encountered, the value
is returned and the iterator finishes.  An iterator also completes
after the last statement in the iterator function is executed.

\subsection{The Yield Statement}
\label{The_Yield_Statement}
\index{yield@\chpl{yield}}

Yield statement can only appear in iterators.  The syntax of the yield
statement is given by
\begin{syntax}
yield-statement:
  `yield' expression ;
\end{syntax}

\subsection{Iterator Calls}
\label{Iterator_Calls}

Iterator functions can be called within for or forall loops, in which
case they are executed in an interleaved manner with the body of the
loop, or can be called in any expression context, in which case they
evaluate to a sequence of values.

\subsubsection{Iterators in For and Forall Loops}
\label{Iterators_in_For_and_Forall_Loops}

When an iterator is accessed via a for or forall loop, the iterator is
evaluated alongside the loop body in an interleaved manner.  For each
iteration, the iterator yields a value and the body is executed.

\subsubsection{Iterators as Sequences}
\label{Iterators_as_Sequences}
\index{iterators!and sequences}

If an iterator function is accessed outside of the context of a for or
forall loop iterator expression, then the iterator is iterated over in
total and the expression evaluates to a sequence that contains the
values returned by the iterator on each iteration.
\begin{example}
Given an iterator
\begin{chapel}
iterator squares(n: int): int {
  for i in 1..n do
    yield i * i;
}
\end{chapel}
the expression \chpl{squares(5)} evaluates to the sequence \chpl{(/1, 4, 9, 16, 25/)}.
\end{example}

\subsection{The Structural Iterator Interface}
\label{Iterator_Interface}
\index{iterators!structural interface}

There is a structural interface that allows a class or record to be
treated as if it were an iterator.  The iterator interface is
important for user-defined distributions.

\begin{implementation}
This section describes the current structural iterator interface.
This does not yet support optimized iteration for rank greater than
one.  As such, this iterator interface is incomplete.
\end{implementation}

The iterator interface defines iteration over a class or record by a
cursor of some type, called the cursor type.  The values returned by
the iterator are of a possibly different type, called the value type.
A class or record \chpl{classType} supports the iterator interface if
it defines the following functions for cursor type \chpl{cursorType}
and value type \chpl{valueType}:
\begin{chapel}
def classType.getHeadCursor(): cursorType
def classType.getNextCursor(cursor: cursorType): cursorType
def classType.getValue(cursor: cursorType): valueType;
def classType.isValidCursor?(cursor: cursorType): bool
\end{chapel}

Iteration over a class or record \chpl{C} of type \chpl{classType}
defined by
\begin{chapel}
for i in C do
  ; // body of loop
\end{chapel}
is equivalent to the following loop:
\begin{chapel}
var cursor = C.getHeadCursor();
if C.isValidCursor?(cursor) then do {
  var i = C.getValue(cursor);
  ; // body of loop
  cursor = C.getNextCursor(cursor);
} while C.isValidCursor?(cursor);
\end{chapel}
