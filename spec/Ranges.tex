\sekshun{Ranges}
\label{Ranges}
\index{ranges}

Ranges contain a sequence of values of integral type that can be
specified with a low bound $l$, a high bound $h$, and a stride $s$.
If the stride is negative, the values contained by the range are $h,
h-s, h-2s, h-3s, ...$ such that all of the values in the sequence are
greater than $l$.  If the stride is positive, the values contained by
the range are $l, l+s, l+2s, l+3s, ...$ such that all of the values in
the sequence are less than $h$.

\index{ranges!literals}
A range is specified by the syntax
\begin{syntax}
range-literal:
  expression .. expression
\end{syntax}
The first expression is taken to be the lower bound, the second
expression is taken to be the upper bound.  The stride of the
range is 1 and can be modified with the \chpl{by}
operator.

\index{ranges!integral element type}
The element type of the range type is determined by the
type of the low and high bound.  It is
either \chpl{int}, \chpl{uint}, \chpl{int(64)}, or \chpl{uint64}.  The
type is determined by conceptually adding the low and high bounds
together.

\subsection{Strided Ranges}
\label{Strided_Ranges}
\index{ranges!strided}

The \chpl{by} operator can be applied to any range to create a strided
range.  It is predefined over a range and an integer to yield a new
range that is strided by the integer.  Striding of strided ranges is
accomplished by multiplying the strides.

\subsection{Range Assignment}
\label{Range_Assignment}
\index{ranges!assignment}

\subsection{Querying the Bounds and Stride of a Range}
\index{low@\chpl{low}}
\index{high@\chpl{high}}
\index{stride@\chpl{stride}}

\begin{protohead}
def range.low: eltType
def range.high: eltType
def range.stride: int
\end{protohead}
\begin{protobody}
These routines respectively return the low bound, the high bound, and
the stride of the range.  The type of the returned low and high bound
is the element type of the range.
\end{protobody}

\subsection{Unbounded Ranges}
\label{Unbounded_Ranges}
\index{ranges!unbounded}

An unbounded range is specified by the syntax
\begin{syntax}
unbounded-range-literal:
  expression ..
  .. expression
\end{syntax}

Unbounded ranges can be iterated over with zipper iteration and their
shape conforms to the shape of the other iterators they are being
iterated over with.
\begin{example}
The code
\begin{chapel}
for i in (1..5, 3..) do
  write(i);
\end{chapel}
produces the output ``(1, 3)(2, 4)(3, 5)(4, 6)(5, 7)''.
\end{example}

It is an error to zip an unbounded range with a range that does not
have the same sign stride.

Unbounded ranges can be used to index into ranges, domains, arrays,
and strings.  In thesse cases, the elided bound conforms to the
expression being indexed.
\begin{status}
Indexing by unbounded ranges is not yet supported.
\end{status}
