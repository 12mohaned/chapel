\sekshun{Parallelism and Synchronization}
\label{Parallelism_and_Synchronization}

Chapel is an explicitly parallel programming language. Parallelism is
introduced into a program by the user with three constructs,
\chpl{forall}, \chpl{begin}, and \chpl{cobegin}.  Synchronization is
provided with \emph{synchronizing variables} and atomic statements. To
avoid any unintended implications, we use the terms \emph{computation}
and \emph{sub-computation} to refer to distinct, concurrently
executing portions of the program.


\subsection{Implicit Parallelism}
\label{Implicit_Parallelism}


\subsubsection{Parallel Expressions}
\label{Parallel_Expressions}

\subsection{Forall Expressions} %% SJD: Added this section
\label{Forall_Expressions}

A sequence expression preceded by an iterator expression can be
executed in parallel. The sub-expressions are evaluated once for each
iteration in the list and these sub-expressions can be evaluated
concurrently. The syntax is:
\begin{syntax}
sequence-statement:
   [index-list `in' iterator-list] expression
\end{syntax}

\begin{example}
\begin{chapel}
[i in S] f(i);
\end{chapel}
The function \chpl{f} is evaluated once for each value in \chpl{S}. \chpl{S}
may be a sequence, array, or domain.
\end{example}

\subsubsection{Parallel Array Assignment}
\label{Parallel_Array_Assignment}

The semantics of array assignments is that each array element is
assigned in parallel. If there are duplicate values in the sequence,
then the order of the element-wise assignments is arbitrary.

\begin{example}
\begin{chapel}
X(S) = Y;
\end{chapel}
X is an array and S a sequence of index values. The execution of all X(S$_i$)=Y
occurs concurrently. The compiler and runtime will decide how many computations
are used to execute the array assignment.
\end{example}


\subsubsection{Ordered Expressions}
\label{Ordered_Expressions}

The \chpl{ordered} keyword can be used as an unary operator to
suppress parallel execution of a sequence expression that can involve
side-effects to memory. The \chpl{ordered} keyword does not inhibit
parallelism within the sub-expression. The syntax is:
\begin{syntax}
ordered-expression:
   `ordered' expression
\end{syntax}

\begin{example}
\begin{chapel}
ordered [i in S] f(i) 
\end{chapel}
In this example \chpl{f} is a function and \chpl{S} is a
sequence. Each instance of \chpl{f(i)} is executed once for each value
in \chpl{S} and in sequence order. The \chpl{ordered} constraint does
not propagate to inhibit parallelism within \chpl{f}.
\end{example}

\subsection{Parallel Statements}
\label{Parallel_Statements}



\subsubsection{Forall}
\label{Forall}

The first form of explicitly creating parallelism is with the forall
loop. The forall loop is a variant of for loop that allows concurrent
execution of the loop body. The for loop is described
in~\rsec{The_For_Loop}. The forall loop syntax is:
\begin{syntax}
forall-statement:
   `forall' index-list `in' iterator-list `do' statement
   `forall' index-list `in' iterator-list block-level-statement
\end{syntax}

The forall loop evaluates the loop body once for each element in the
sequence returned by the iterator list. The evaluation of each of
these loop bodies can be executed concurrently and is considered a
separate computation. Similarly, the iterator-list may be executed
concurrently with the statements. The keyword \chpl{ordered} which is
later described in~\rsec{Ordered_Forall}) can be used to constrain
the parallelism to generate a partial order on the sequence.

Control continues with the statement following the forall loop only
after all statement instances have been completely evaluated.
Control transfers such as \chpl{goto}, \chpl{break}, \chpl{continue},
and \chpl{return} are not permitted either into or out of the body of
a forall loop.  However, \chpl{yield} statements are permitted.

\begin{example}
\begin{chapel}
forall i in 1..10000 do
  a(i) = b(i);
\end{chapel}
In this example the user has stated that the element-wise assignments can
execute concurrently. The compiler and runtime will decide how many
computations are used to assign all 10,000 elements.
\end{example}

\subsubsection{Alternative Forall Loop Syntax}
\label{Alternative_Forall_Loop_Syntax}

The forall loop may be specified with a more concise syntax given by
another alternative:
\begin{syntax}
for-statement:
  [ index-list `in' iterator-list ] statement
\end{syntax}
The semantics of this alternative syntax are unchanged.

\begin{example}
The following statements are equivalent:
\begin{chapel}
forall i in 1..n do
  sum += i;
\end{chapel}
\begin{chapel}
[i in 1..n] sum += i;
\end{chapel}
\end{example}

\subsubsection{Ordered Forall}
\label{Ordered_Forall}

The \chpl{ordered} keyword can be used to constrain the parallelism
within the iterator-list generation to that specified by the iterator.
The syntax is:
\begin{syntax}
forall-statement:
   ordered `forall' index-list `in' iterator-list `do' statement
   ordered `forall' index-list `in' iterator-list block-level-statement
\end{syntax}

The \chpl{ordered} constraint only affects order of evaluation only
when the iterator-list is an iterator. There is no effect if the
iterator-list is a sequence value.

By default a forall loop allows complete concurrent evaluation within
the iterator-list and with evaluation of the statements. When
the \chpl{ordered} keyword is present, the only concurrency between
statements is the concurrency that is explicitly specified in the
iterator itself. This allows an iterator to not only define a sequence of
values, but to impose a partial order on that sequence.

% \footnote{It might be tempting to generate a warning when the
% keyword is ignored but those warnings may be spurious in a generic
% context where we are reusing an abstract function definition.}

\begin{example}
\begin{chapel}
forall i in 1..10000 do
  a(i) = b(i);

def jam() {
  yield me;
  cobegin {
  }
}
\end{chapel}
In this example the user has stated that the element-wise assignments can
execute concurrently. The compiler and runtime will decide how many
computations are used to assign all 10,000 elements.
\end{example}


\subsubsection{Cobegin}
\label{Cobegin}

The cobegin statement spawns a computation for each of the statements
in the block-statement that it immediately precedes.  When each of
these computations finish, control continues with the next statement.
The syntax for the cobegin statement is given by
\begin{syntax}
cobegin-statement:
  `cobegin' block-statement
\end{syntax}

The second form of creating parallelism is with the \chpl{cobegin}
statement. The \chpl{cobegin} statement syntax is
\begin{syntax}
cobegin-statement:
  `cobegin' block-statement
\end{syntax}
Each statement in the \chpl{block-statement} is executed concurrently
and is considered a separate computation. Control continues after all
of the statements within the \chpl{block-statement} have been
evaluated. As with \chpl{forall}, control transfers are not permitted
either into or out of the body of a \chpl{cobegin} statement. As
with \chpl{forall}, \chpl{yield} statements are allowed.

Variables declared in the \chpl{cobegin} compound statment are {\em single
assignment variables}~(\rsec{Single_Assignment_Variables}). 

\subsubsection{Begin}
\label{Begin}

The begin statement spawns a computation to execute the statement
embedded in it.  Control continues simultaneously with the next
statement.  The syntax for the begin statement is given by
\begin{syntax}
begin-statement:
  `begin' statement
\end{syntax}

The third form of creating parallelism is with the \chpl{begin}
statement. The \chpl{begin} statement syntax is
\begin{syntax}
begin-statement:
  `begin' statement
\end{syntax}
A second computation is created to execute \chpl{statement}.
This second computation is executed concurrently with the balance of
the initiating computation which continues with the statement
following the \chpl{begin}. Using \chpl{begin} is an unstructured way to 
create a new computation executed only for its side-effects.

Control transfers in to or out of \chpl{<statement>} are prohibited. Unlike
\chpl{forall} and \chpl{cobegin}, \chpl{yield} statement are not permitted.

% including \chpl{return} 

\subsubsection{Serial}
\label{Serial}

The \chpl{serial} statement can be used to control the degree of parallelism.
The serial statement is used to control the parallelism encountered
dynamically in the execution.  The syntax for the serial statement is
given by
\begin{syntax}
serial-statement:
  `serial' expression block-level-statement
\end{syntax}
where \chpl{exprression} evaluates to a \chpl{bool}.  Regardless of
that value, \chpl{block-level-statement} is evaluated. If
the \chpl{expression} evaluates to \chpl{true}, any dynamically
encountered \chpl{forall} or \chpl{cobegin} statement is executed
sequentially within that computation.



\subsection{Parallel Iterators}
\label{Parallel_Iterators}

\begin{example}
\begin{chapel}      
class Tree {
  var is_leaf : bool;
  var left    : Tree;
  var right   : Tree;
}

iterator Tree.walk() {
  if (is_leaf) then
    yield this;
  else
    serial (height <= 10) cobegin {
      yield left.walk();
      yield right.walk();
    }
}
\end{chapel}
In this example, a recursive tree iterator uses the tree's height to
determine when concurrency should be created.
\end{example}

% This iterator could be used in conjunction with an \chpl{ordered
% forall} to aggregate work to avoid parallelism overhead.\footnote{My
% expectation is that we clone functions that may be executed in a
% serial context so we can avoid the overhead of testing and suppressing
% parallelism.}

\subsection{Implicit Synchronization on Variables}
\label{Implicit_Synchronization_on_Variables}

This section describes {\em synchronized variables} which are used as
a mechanism for coordinating accesses to shared variables. There are
two kinds of synchronized variables, \chpl{single} and \chpl{sync}
variable, whose semantics include execution order and memory
consistency requirements.



\subsubsection{Single Assignment Variables}
\label{Single_Assignment_Variables}

Single assignment variable can only be assigned once during their
dynamic lifetime.  Any reference to the single assignment variable
before it is defined causes the computation's execution to be
suspended until the variable is assigned. Otherwise, the reference
proceeds as with normal variables.  After a single assigment variable is
assigned, all computations with pending references are resumed in an
unspecified order. The syntax of a single assignment variable is
\begin{chapel}
[var] <symbol>: single <type> [= <expr>];
\end{chapel}
The keyword \chpl{single} is used as a type modifier. 

\begin{example}
\begin{chapel}
class Tree {
  var is_leaf : bool;
  var left    : Tree;
  var right   : Tree;
  var value   : int;
}

def Tree.sum() {
  if (is_leaf) then return value;
  var x:single int;
  begin x = left.sum();
  var y = right.sum();
  return x+y;
}
\end{chapel}
While a \chpl{cobegin} might be a more suitable formulation, this
fragment creates an asynchronous computation to compute the sum of the
left sub-tree while the main computation continues with the right
sub-tree. The final reference to variable x will be delayed until the
assignment to x completes and that value will be used as a summand.
\end{example}

When a \chpl{single} variable has an initializer, the evaluation of
that initializer is implicitly performed as an asynchronous
computation. 
\begin{example}
\begin{chapel}
def Tree.sum() {
  if (is_leaf) then return value;
  var x: single int = left.sum;
  var y = right.sum();
  return x+y;
}
\end{chapel}
This is equivalent to the above where the declaration and assignment are
separated.
\end{example}

Any variable declaration in a \chpl{cobegin} is implicitly treated as
a \chpl{single} variable for references in other statements of the
\chpl{cobegin}.  The above example might then be written as:
\begin{example}
\begin{chapel}
def Tree.sum() {
  if (is_leaf) then return value;
  var z;
  cobegin {
    var x = left.sum();
    var y = right.sum();
    z = x+y;
  }
  return z;
}
\end{chapel}
The computation with assignment to z waits for the other computations 
to assign to x and y before it reference x and y in order to assign to z.
\end{example}

\subsubsection{Sync Variables}
\label{Sync_Variables}

A synchronized variable generalizes the single assignment variable to
permit multiple assignments over its lifetime. Synchronized variables
are declared with the type attribute \chpl{sync}. The synchronized
variable syntax is
\begin{chapel}
[var] <symbol>: sync <type> [= <expr>];
\end{chapel}
A sync variable is logically either {\em full} or {\em empty}. When it
is empty, computations that attempt to read that variable are delayed
until it becomes full by the next assignment to it which atomically
changes the state to full. When the variable is full, a variable
reference consumes the value and atomically transitions the state back
to empty. If there are more than one computation waiting, one is
non-deterministically selected to receive the value resume execution.
The other computations continue to wait for the next
assignment. Otherwise, the state transistions to empty. 

If a computation attempts to assign to a synchronized variable that is
full, the computation is suspended and the assignment is delayed. When
the variable becomes empty, the computation is resumed and the
assignment proceeds, transistioning the state back to full. If there
are multiple computations attempting such an assignment, one is
non-deterministically selected to proceed and the other computations
continue to wait until the sync variable is emptied again.

\chpl{sync} variables allow a sequence of values to be communicated
between computations using a single shared variable. They also can be
used as building blocks for more traditional synchronization
primitives such as semaphores and monitors.

\subsubsection{Synchronization Variables of Record Type}
\label{Synchronization_Variables_of_Record_Type}

A \chpl{single} variable can have \chpl{record} type but in this case
all fields must be assigned at the same time. Assignments to
individual fields are not permitted.

A \chpl{sync} variable can hold a value of \chpl{record} type but in
this case the entire record must be read and written. Reads and writes
of individual non-\chpl{const} fields are not permitted.



\subsubsection{Synchronization Variables of Class Type}
\label{Synchronization_Variables_of_Class_Type}

This is a stub.  This portion of the document does not exist.

\subsubsection{Functions on Synchronization Variables}
\label{Functions_on_Synchronization_Variables}

Synchronized variables support a number of special methods. These
include the following on synchronized variable s
\begin{chapel}
purge(s)      // force s to be empty
writeFE(s, v) // wait for full, assign s=v, and leave empty
writeXF(s, v) // no wait, assign s=v, and leave full
writeXE(s, v) // no wait, assign s=v, and leave empty
readFF(s)     // wait full, leave full, and return s's value
readXF(s)     // no wait, leave full, and return s's value
readXX(s)     // no wait, leave F/E unchanged, and return s's value
\end{chapel}

\subsection{Memory Consistency}
\label{Memory_Consistency}

\paragraph{Memory Consistency}
The Chapel implementation permits holding the values of variables in
alternate locations which we will refer to as {\em caches}.  The
implementation may {\em prefetch} by copying a value from a variable
to a cache before an access to that variable is executed. It may store
a value into a cache after it is assigned but not write it to the
variable in which case we say the cache is {\em dirty}.
 
There are restrictions on caching to ensure effective communication
between concurrent computations. These restrictions are defined in
terms of {\em input} and {\em output synchronization points}.

An input synchronization point occurs when a computation performs an
operation, such as reading a synchronized variable, that may cause
that computation to be delayed. This also includes waiting for
sub-computations associated with {\tt forall} and \chpl{cobegin}
statements. Any cached value of a variable that might have been
modified by concurrent computation must be treated as invalid at the
input synchronization point.  This is called {\em invalidating} the
cache.

An output synchronization point occurs when a computation performs an
operation that may enable another computation to begin or resume
execution. This includes starting a \chpl{forall} or \chpl{cobegin} or
accessing a synchronized variable.  Any dirty value of a variable that
might be accessed by another computation must be {\em flushed} from
cache back to the location of the variable.

Chapel does not guarantee communication of values between concurrent
computations unless there is appropriate synchronization to coordinate
the producer and consumer.

Because of their role in coordinating threads, synchronized variables
can not generally be cached.

\subsection{Atomic Transactions}
\label{Atomic_Transactions}

\noindent @@@\hrulefill \\
The atomic statement creates an atomic transaction around a statement
or block of statements.  The syntax for the atomic statement is given
by
\begin{syntax}
atomic-statement:
  `atomic' block-level-statement
\end{syntax}
\noindent @@@\hrulefill \\

An {\em atomic transaction} is a region of the program that appears to
execute as if all other computations in the program are suspended. It
is indicated as:
\begin{chapel}
atomic <<block>>;
\end{chapel}
where \chpl{<<block>>} is executed as if it were serialized by a
\chpl{serial} statement.

The behavior of an atomic statement is defined operationally in terms
of acquiring ownership of variables. Each variable in the program may
have an {\em owning} computation. Inside an atomic statement, before
each variable reference, the current computation attempts to acquire
ownership of the variable. If the variable is owned by another
computation, a deadlock might occur if the computation naively
waits. At such a point the implementation may {\em abort} the
statement. This means we release ownership of all variables owned by
the current computation without changing any of their values, and we
restart execution at the beginning of \chpl{<<block>>}.  Visible
side-effects to variables are delayed until \chpl{<<block>>}
completes. At this point we say the atomic statement {\em commits},
and all variables it owns are released and any modifications to
variables are made.

The implementation of the atomic statement must ensure forward
progress, but the details of how ownership information is maintained
and when computations are aborted is not defined here.

Here is an example of an atomic transaction:
\begin{chapel}
var found = false;
atomic {
  if(head == obj) {
    found = true;
    head = obj.next;
  } else  {
    var last = head;
    while(last != null) {
    if (last.next == obj) {
      found = true;
      last.next = object.next;
      break;
    }
    last = last.next;
  }
}
\end{chapel}
Inside the atomic statement is a simple sequential implementation of
removing a particular object denoted by \chpl{obj} from a singly
linked list.  This is an operation that is well-defined, assuming only
one computation is attempting it at a time. The atomic statement
insures that, for example, the value of \chpl{head} does not change
after it is first in the first comparison and subsequently read to
initialize \chpl{last}. The variables eventually owned by this
computation are \chpl{found}, \chpl{head}, \chpl{obj}, and the various
\chpl{next} fields on examined objects.

The effect of an atomic statement extends into called functions. Thus
if we have some method associated with a list that removes an object,
that method may not be parallel safe but could be invoked inside an
atomic statement for safety:
\begin{chapel}
var found;
atomic found = head.remove(obj);
\end{chapel}

Chapel defines three operations that are used to optimize atomic
statements to avoid overheads.  They are the \chpl{_invariant},
\chpl{_private}, and \chpl{_release} operations. The first two are
similar in that they identify variables that are either known to be
invariant over a program interval, or known to not be accessed by any
other computation. In either case, the protocol to assert ownership
and delay side-effects can be avoided. The \chpl{_release} operator
identifies a variable currently owned by the computation that has not
been modified and will not be accessed again. The current
computation's ownership is revoked and another computation can assert
ownership.

These operations might be used to tune the above code to reduce
overheads. For example:
\begin{chapel}
var found;
atomic {
  _private found = false;	   
  if (head == _invariant obj) {
    found = true;
    head = _invariant obj.next;
  } else {
    var last = _release head;
    while(last != null) {
      if(last.next == obj) {
        found = true;
        last.next = _invariant obj.next;
        break;
      }
      last = _release last.next;
    }
  }
}
\end{chapel}
These avoid overheads on the computation's private variables
\chpl{obj} and \chpl{found}.  It also allows a list-removal operation
to be pipelined by releasing ownership of the list head and various
link fields as soon as they will no longer be referenced. The
assertion that \chpl{obj.next} is invariant reflects the fact that no
other variables need to be acquired to allow the transaction to
complete, not that the field is actually invariant. The implementation
is free to prove these attributes and apply them automatically. For
example, we would expect the implementation to identify \chpl{last} as
a private variable without programmer assertion.

Chapel allows variables to have an \chpl{atomic} attribute to allow
optimization of ownership. For example:
\begin{chapel}
class ObjType {
  atomic var next : ObjType;
  ...
}
\end{chapel}
This declaration treats a definitions of such fields as if they are in
short atomic sections. Thus, a statement of the form:
\begin{chapel}
e1.x = e2;
\end{chapel}
where \chpl{x} is atomic would be evaluated as:
\begin{chapel}
var t1 = e1;
var t2 = e2;
atomic (_invariant t1).x = _invariant t2;
\end{chapel}

The declaration also encourages the implementation to allocate any
extra storage needed to maintain ownership adjacent to the variable
itself, avoiding the overhead of mapping the variable to sparsely
maintained information.


\subsubsection{Strong Atomicity}
\label{Strong_Atomicity}

This is a stub.  This portion of the document does not exist.

\subsubsection{Weak Atomicity}
\label{Weak_Atomicity}

This is a stub.  This portion of the document does not exist.
