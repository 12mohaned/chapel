\sekshun{Notation}
\label{Notation}

Special notations are used in this specification to denote Chapel code
and to denote Chapel syntax.

Chapel code is represented with a fixed-width font where keywords are
bold and comments are italicised.
\begin{example}
\begin{chapel}
for i in D do   // iterate over domain D
  writeln(i);   // output indices in D
\end{chapel}
\end{example}

Chapel syntax is represented with standard syntax notation in which
productions define the syntax.  A production is defined in terms of
terminal and non-terminal symbols.  Non-terminal symbols are
italicised.  The complete syntax defines all of the non-terminal
symbols in terms of one another and terminal symbols.

A production of a non-terminal is a multi-line construct.  The first
line shows the name of the non-terminal that is being defined followed
by a colon.  The next lines before an empty line define the
alternative productions to define the non-terminal.
\begin{example}
The production
\begin{syntax}
bool-literal:
  `true'
  `false'
\end{syntax}
defines \sntx{bool-literal} to be either the symbol \sntx{`true'} or
\sntx{`false'}.
\end{example}
In the event that a single line of a definition needs to break across
multiple lines of text, more indentation is used to indicate that it
is a continuation of the same alternative production.

As a short-hand for cases where there are many alternatives that
define one symbol, the first line of the definition of the
non-terminal may be followed by ``one of'' to indicate that the single
line in the production defines alternatives for each symbol.
\begin{example}
The production
\begin{syntax}
unary-operator: one of
  + - ~ !
\end{syntax}
is equivalent to
\begin{syntax}
unary-operator:
  +
  -
  ~
  !
\end{syntax}
\end{example}

As a short-hand to indicate an optional symbol in the definition of a
production, the subscript ``opt'' is suffixed to the symbol.
\begin{example}
The production
\begin{syntax}
formal:
  formal-tag identifier formal-type[OPT] default-expression[OPT]
\end{syntax}
is equivalent to
\begin{syntax}
formal:
  formal-tag identifier formal-type default-expression
  formal-tag identifier formal-type
  formal-tag identifier default-expression
  formal-tag identifier
\end{syntax}
\end{example}
