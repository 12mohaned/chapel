The third form of creating parallelism is with the \chpl{begin}
statement. The \chpl{begin} statement syntax is
\begin{chapel}
begin <statement>;
\end{chapel}
A second computation is created to execute \chpl{<statement>}.
This second computation is executed concurrently with the balance of
the initiating computation which continues with the statement
following the \chpl{begin}. Using \chpl{begin} is an unstructured way to 
create a new computation executed only for its side-effects.

Control transfers in to or out of \chpl{<statement>} are prohibited. Unlike
\chpl{forall} and \chpl{cobegin}, \chpl{yield} statement are not permitted.

% including \chpl{return} 
