The first form of explicitly creating parallelism is with the forall
loop. The forall loop is a variant of for loop that allows concurrent
execution of the loop body. The for loop is described in
Section~\ref{The_For_Loop}. The forall loop syntax is:
\begin{syntax}
forall-statement:
   `forall' index-list `in' iterator-list `do' statement
   `forall' index-list `in' iterator-list block-level-statement
\end{syntax}

The forall loop evaluates the loop body once for each element in the
sequence returned by the iterator list. The evaluation of each of
these loop bodies can be executed concurrently and is considered a
separate computation. Similarly, the iterator-list may be executed
concurrently with the statements. The keyword \chpl{ordered} which is
later described in Section~\ref{Ordered Forall}) can be used to constrain
the parallelism to generate a partial order on the sequence.

Control continues with the statement following the forall loop only
after all statement instances have been completely evaluated.
Control transfers such as \chpl{goto}, \chpl{break}, \chpl{continue},
and \chpl{return} are not permitted either into or out of the body of
a forall loop.  However, \chpl{yield} statements are permitted.

\begin{example}
\begin{chapel}
forall i in 1..10000 do
  a(i) = b(i);
\end{chapel}
In this example the user has stated that the element-wise assignments can
execute concurrently. The compiler and runtime will decide how many
computations are used to assign all 10,000 elements.
\end{example}
