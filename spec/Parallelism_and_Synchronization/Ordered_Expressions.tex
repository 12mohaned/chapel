The \chpl{ordered} keyword can be used as an unary operator to
suppress parallel execution of a sequence expression that can involve
side-effects to memory. The \chpl{ordered} keyword does not inhibit
parallelism within the sub-expression. The syntax is:
\begin{syntax}
ordered-expression:
   `ordered' expression
\end{syntax}

\begin{example}
\begin{chapel}
ordered [i in S] f(i) 
\end{chapel}
In this example \chpl{f} is a function and \chpl{S} is a
sequence. Each instance of \chpl{f(i)} is executed once for each value
in \chpl{S} and in sequence order. The \chpl{ordered} constraint does
not propagate to inhibit parallelism within \chpl{f}.
\end{example}
