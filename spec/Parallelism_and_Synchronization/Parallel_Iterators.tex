\begin{example}
\begin{chapel}      
class Tree {
  var is_leaf : bool;
  var left    : Tree;
  var right   : Tree;
}

iterator Tree.walk() {
  if (is_leaf) then
    yield this;
  else
    serial (height <= 10) cobegin {
      yield left.walk();
      yield right.walk();
    }
}
\end{chapel}
In this example, a recursive tree iterator uses the tree's height to
determine when concurrency should be created.
\end{example}

% This iterator could be used in conjunction with an \chpl{ordered
% forall} to aggregate work to avoid parallelism overhead.\footnote{My
% expectation is that we clone functions that may be executed in a
% serial context so we can avoid the overhead of testing and suppressing
% parallelism.}
