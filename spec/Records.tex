\sekshun{Records}
\label{Records}
\index{records}

A record is a data structure that is similar to a class except it has
value semantics, similar to primitive types.  Value semantics mean that
assignment, argument passing and function return values are by default
all done by copying.  Value semantics also imply that a variable of
record type is associated with only one piece of storage and has only one
type throughout its lifetime.  Storage is allocated for a variable of
record type when the variable declaration is executed, and the record
variable is also initialized at that time. When the record variable goes
out of scope, or at the end of the program if it is a global, it is
deinitialized and its storage is deallocated.

A record declaration statement creates a record
type~\rsec{Record_Declarations}.  A variable of record type contains all
and only the fields defined by that type (\rsec{Record_Types}).  Value
semantics imply that the type of a record variable is known at compile
time (i.e. it is statically typed).

A record can be created using the \chpl{new} operator, which allocates
storage, initializes it via a call to a record constructor, and returns
it.  A record is also created upon a variable declaration of a record
type.

A record type is generic if it contains generic fields.  Generic record types
are discussed in detail in~\rsec{Generic_Types}.

\section{Record Declarations}
\label{Record_Declarations}
\index{records!declarations}
\index{declarations!records}
\index{record@\chpl{record}}

A record type is defined with the following syntax:
\begin{syntax}
record-declaration-statement:
  simple-record-declaration-statement
  external-record-declaration-statement

simple-record-declaration-statement:
  `record' identifier { record-statement-list }

record-statement-list:
  record-statement
  record-statement record-statement-list

record-statement:
  variable-declaration-statement
  method-declaration-statement
  type-declaration-statement
  empty-statement
\end{syntax}

A \sntx{record-declaration-statement} defines a new type symbol specified
by the identifier. As in a class declaration, the body of a record declaration
can contain variable, method, and type declarations.

If a record declaration contains a type alias or parameter field, or it
contains a variable or constant field without a specified type and
without an initialization expression, then it declares a generic record
type.  Generic record types are described in~\rsec{Generic_Types}.

If the \chpl{extern} keyword appears before the \chpl{record} keyword, then an
external record type is declared. An external record is used within Chapel
for type and field resolution, but no corresponding backend definition is
generated.  It is presumed that the definition of an external record is supplied
by a library or the execution environment.  See the chapter on interoperability
(\rsec{Interoperability}) for more information on external records.

\begin{future}
Privacy controls for classes and records are currently not specified,
as discussion is needed regarding its impact on inheritance, for
instance.
\end{future}

\subsection{Record Types}
\label{Record_Types}
\index{records!record types}
\index{records!types}
\index{types!records}

A record type specifier simply names a record type, using
the following syntax:
\begin{syntax}
record-type:
  identifier
  identifier ( named-expression-list )
\end{syntax}
A record type specifier may appear anywhere a type specifier is permitted.

For non-generic records, the record name by itself is sufficient to specify the
type.  Generic records must be instantiated to serve as a fully-specified
type, for example to declare a variable.  This is done with
type constructors, which are defined in Section~\ref{Type_Constructors}.

\subsection{Record Fields}
\label{Record_Fields}
\index{records!fields}
\index{fields!records}

Variable declarations within a record type declaration define fields within that
record type.  The presence of at least one parameter field causes the record
type to become generic.  Variable fields define the storage associated with a
record.

\begin{chapelexample}{defineActorRecord.chpl}
The code
\begin{chapel}
record ActorRecord {
  var name: string;
  var age: uint;
}
\end{chapel}
\begin{chapeloutput}
\end{chapeloutput}
defines a new record type called \chpl{ActorRecord} that has two fields: the
string field \chpl{name} and the unsigned integer field \chpl{age}.  The data
contained by a record of this type is exactly the same as that contained by
an instance of the \chpl{Actor} class defined in the preceding
chapter~\rsec{Class_Fields}.
\end{chapelexample}

\subsection{Record Methods}
\label{Record_Methods}
\index{records!methods}
\index{methods!records}

A record method is a function or iterator that is bound to a record.
See the methods section~\rsec{Methods} for more information about
methods.

Note that the receiver of a record method is passed by \chpl{ref} or
\chpl{const ref} intent by default, depending on whether or not
\chpl{this} is modified in the body of the method.

\subsection{Nested Record Types}
\label{Nested_Record_Types}
\index{nested records}
\index{records!nested}

Record type declarations may be nested within other class, record and union
declarations.  Methods defined in a nested record type may access fields
declared in the containing aggregate type either implicitly, or explicitly by
means of an \chpl{outer} reference.

\section{Record Variable Declarations}
\label{Record_Variable_Declarations}
\index{records!variable declarations}
\index{variables!records}

A record variable declaration is a variable declaration using a record type.
When a variable of record type is declared, storage is allocated sufficient to
store all of the fields defined in that record type.

In the context of a class or record or union declaration, the fields are
allocated within the object as if they had been declared individually.  In this
sense, records provide a way to group related fields within a containing class
or record type.

In the context of a function body, a record variable declaration
causes storage to be allocated sufficient to store all of the fields in that
record type.  The record variable is initialized through a call to its
default initializer.  The default initializer for a record is defined in the
same way as the default initializer for a class (\rsec{Default_Initialization}).

\subsection{Storage Allocation}
\label{Record_Storage}
\index{records!allocation}

Storage for a record variable directly contains the data associated
with the fields in the record, in the same manner as variables
of primitive types directly contain the primitive values.
Record storage is reclaimed when the record variable goes out of scope.
No additional storage for a record is allocated or reclaimed.
Field data of one variable's record is not shared with data
of another variable's record.

\subsection{Record Initialization}
\label{Record_Initialization}
\index{records!initialization}
\index{initialization!record}

A variable of a record type declared without an initialization expression
is initialized through a call to the record's default initializer,
passing no arguments.  The default initializer for a record is defined in
the same way as the default initializer for a class
(\rsec{Default_Initialization}).

To construct a record as an expression,
i.e. without binding it to a variable, the \chpl{new} operator is
required.  In this case, storage is allocated and reclaimed as for a record
variable declaration (\rsec{Record_Storage}), except that the temporary record
goes out of scope at the end of the enclosing statement.
The constructors for a record are
defined in the same way as those for a class (\rsec{Class_Constructors}).

\begin{rationale}

The \chpl{new} keyword disambiguates types from values. This is needed
because of the syntactic similarity between constructors and type
specifiers for classes and records.

\end{rationale}

\begin{chapelexample}{recordCreation.chpl}
The program
\begin{chapel}
record TimeStamp {
  var time: string = "1/1/1011";
}

var timestampDefault: TimeStamp;                  // use the default for 'time'
var timestampCustom = new TimeStamp("2/2/2022");  // ... or a different one
writeln(timestampDefault);
writeln(timestampCustom);

var idCounter = 0;
record UniqueID {
  var id: int;
  proc UniqueID() { idCounter += 1; id = idCounter; }
}

writeln(new UniqueID());  // create and use a record value without a variable
writeln(new UniqueID());
\end{chapel}
\begin{chapelcompopts}
--no-warn-constructors
\end{chapelcompopts}
produces the output
\begin{chapelprintoutput}{}
(time = 1/1/1011)
(time = 2/2/2022)
(id = 1)
(id = 2)
\end{chapelprintoutput}
The variable \chpl{timestampDefault} is initialized with \chpl{TimeStamp}'s
default initializer. The expression \chpl{new TimeStamp} creates a record that
is assigned to \chpl{timestampCustom}.  It effectively
initializes \chpl{timestampCustom} via a call to the constructor with desired
arguments. The records created with \chpl{new UniqueID()} are discarded after
they are used.
\end{chapelexample}

As with classes, the user can provide his own constructors
(\rsec{User_Defined_Constructors}).  If any user-defined constructors are
supplied, the default initializer cannot be called directly.

\subsection{Record Deinitializer}
\label{Record_Deinitializer}
\index{records!deinitializer}
\index{deinitializer!records}

A record author may specify additional actions to be performed before record storage is
reclaimed by defining a record deinitializer.  A record deinitializer is a method named
\chpl{deinit()}.  A record deinitializer takes no arguments
(aside from the implicit \chpl{this} argument).  If defined, the deinitializer is called
on a record object after it goes out of scope and before its memory is reclaimed.

% TODO: The above ambiguous language is intended to allow optimizations involving extending
% the lifetime of an object.  However, we leave unspecified the means by which a user may
% demand avid running of the deinitializer and reclamation of memory (as in C\#).  We need to
% specify this so the above language can be tightened up for that case.

% For now, the actual lifetime of a record object is under the control of the compiler.  For
% example, as an optimization, ownership of an object may be transferred between variables with
% non-overlapping lifetimes.  When this happens, there will be no observable deinitialization of
% one of those variables.  The compiler may also choose to insert temporary copies e.g. of
% record formals or of a record return value.

% The compiler guarantees that every record constructor call will have exactly one
% corresponding record deinitializer call.  However, the exact number of constructor-deinitializer
% pairs is determined by the compiler, and may also be influenced by various compiler
% options.

\begin{chapelexample}{recordDeinitializer.chpl}
\begin{chapel}
class C { var x: int; } // A class with nonzero size.
// If the class were empty, whether or not its memory was reclaimed
// would not be observable.

// Defines a record implementing simple memory management.
record R {
  var c: C;
  proc R() { c = new C(0); }
  proc deinit() { delete c; c = nil; }
}

proc foo()
{
  var r: R; // Initialized using default constructor.
  writeln(r);
  // r will go out of scope here.
  // Its deinitializer will be called to free the C object it contains.
}

foo();
\end{chapel}
\begin{chapelcompopts}
--no-warn-constructors
\end{chapelcompopts}
\begin{chapeloutput}
(c = {x = 0})

====================
Leaked Memory Report
==============================================================
Number of leaked allocations
           Total leaked memory (bytes)
                      Description of allocation
==============================================================
==============================================================
\end{chapeloutput}
\begin{chapelexecopts}
--memLeaksByType
\end{chapelexecopts}
\end{chapelexample}


\section{Record Arguments}
\label{Record_Arguments}
\index{records!arguments}
\index{arguments!records}

When records are copied into or out of a function's formal argument,
the copy is performed consistently with the semantics described for
record assignment (\rsec{Record_Assignment}).

\begin{chapelexample}{paramPassing.chpl}
The program
\begin{chapel}
record MyColor {
  var color: int;
}
proc printMyColor(in mc: MyColor) {
  writeln("my color is ", mc.color);
  mc.color = 6;   // does not affect the caller's record
}
var mc1: MyColor;        // 'color' defaults to 0
var mc2: MyColor = mc1;  // mc1's value is copied into mc2
mc1.color = 3;           // mc1's value is modified
printMyColor(mc2);       // mc2 is not affected by assignment to mc1
printMyColor(mc2);       // ... or by assignment in printMyColor()

proc modifyMyColor(inout mc: MyColor, newcolor: int) {
  mc.color = newcolor;
}
modifyMyColor(mc2, 7);   // mc2 is affected because of the 'inout' intent
printMyColor(mc2);
\end{chapel}
produces
\begin{chapelprintoutput}{}
my color is 0
my color is 0
my color is 7
\end{chapelprintoutput}
The assignment to \chpl{mc1.color} affects only the record stored
in \chpl{mc1}. The record in \chpl{mc2} is not affected by
the assignment to \chpl{mc1} or by the assignment in \chpl{printMyColor}.
\chpl{mc2} is affected by the assignment in \chpl{modifyMyColor}
because the intent \chpl{inout} is used.
\end{chapelexample}

\section{Record Field Access}
\label{Record_Field_Access}
\index{records!field access}
\index{field access}

A record field is accessed the same way as a class field
(\rsec{Class_Field_Accesses}).  When a field access is used as an
rvalue, the value of that field is returned.  When it is used as
an lvalue, the value of the record field is updated.

Accessing a parameter or type field returns a parameter or type,
respectively. Also, parameter and type fields can be accessed from
an instantiated record type in addition to from a record value.


\subsection{Field Getter Methods}
\label{Field_Getter_Methods}
\index{records!getters}

As in classes, field accesses are performed via getter methods
(\rsec{Getter_Methods}).  By default, these methods simply return a reference to
the specified field (so they can be written as well as read).  The user may
redefine these as needed.

\section{Record Method Calls}
\label{Record_Method_Access}
\index{records!method calls}
\index{method calls}

Record method calls are written the same way as other method calls
(\rsec{Method_Calls}). Unlike class methods, record methods are
always resolved at compile time.

\section{Common Operations}

\subsection{Record Assignment}
\label{Record_Assignment}
\index{records!assignment}

A variable of record type may be updated by assignment.  The compiler
provides a default assignment operator for each record type \chpl{R}
having the signature:

\begin{chapel}
proc =(ref lhs:R, rhs) : void ;
\end{chapel}

In it, the value of each field of the record on the right-hand side is assigned
to the corresponding field of the record on the left-hand side. It is
a type error if the left-hand side and the right-hand side do not have
the same set of field names. It is also a type error if two fields with
the same name do not have assignable types.

The compiler-provided assignment operator may be overridden as described
in \ref{Assignment_Statements}.

The following example demonstrates record assignment.
\begin{chapelexample}{assignment.chpl}
\begin{chapel}
record R {
  var i: int;
  var x: real;
  proc print() { writeln("i = ", this.i, ", x = ", this.x); }
}
var A: R;
A.i = 3;
A.print();	// "i = 3, x = 0.0"

var C: R;
A = C;
A.print();	// "i = 0, x = 0.0"

C.x = 3.14;
A.print();	// "i = 0, x = 0.0"
\end{chapel}
\begin{chapeloutput}
i = 3, x = 0.0
i = 0, x = 0.0
i = 0, x = 0.0
\end{chapeloutput}
Prior to the first call to \chpl{R.print}, the record \chpl{A} is created and
initialized to all zeroes.  Then, its \chpl{i} field is set to \chpl{3}.
For the second call to \chpl{R.print}, the record \chpl{C} is created assigned
to \chpl{A}.  Since \chpl{C} is default-initialized to all zeroes, those zero
values overwrite both values in \chpl{A}.

The next clause demonstrates that \chpl{A} and \chpl{C} are distinct entities,
rather than two references to the same object.  Assigning \chpl{3.14}
to \chpl{C.x} does not affect the \chpl{x} field in \chpl{A}.
\end{chapelexample}

\subsection{Default Comparison Operators}
\label{Record_Comparison_Operators}
\index{records!equality}
\index{records!inequality}
\index{records!==@\chpl{==}}
\index{records!"!=@\chpl{"\"!=}}
\index{== (record)@\chpl{==} (record)}
\index{"!= (record)@\chpl{"\"!=} (record)}
Default functions to overload \chpl{==} and \chpl{\!=} are defined for
records if none are explicitly defined.
The default implementation of \chpl{==} applies \chpl{==} to each
field of the two argument records and reduces the result with
the \chpl{&&} operator.  The default implementation of \chpl{\!=}
applies \chpl{\!=} to each field of the two argument records and
reduces the result with the \chpl{||} operator.

\section{Differences between Classes and Records}
\label{Class_and_Record_Differences}
\index{records!differences with classes}

The key differences between records and classes are listed below.

\subsection{Declarations}
\label{Declaration_Differences}
\index{records!declarations!differences with classes}

Syntactically, class and record type declarations are identical, except that
they begin with the \chpl{class} and \chpl{record} keywords, respectively.
In contrast to classes, records do not support inheritance.

\subsection{Storage Allocation}
\label{Storage_Allocation_Differences}
\index{classes!allocation}
\index{records!allocation}

For a variable of record type, storage necessary to contain the data fields
has a lifetime equivalent to the scope in which it is declared.  No two record
variables share the same data.  It is not necessary to call \chpl{new} to create
a record.

By contrast, a class variable contains only a reference to a
class instance.  A class instance is created through a call to its \chpl{new}
operator.  Storage for a class instance, including storage for
the data associated with the fields in the class, is allocated and reclaimed
separately from variables referencing that instance.  The same class instance
can be referenced by multiple class variables.

\subsection{Assignment}
\label{Assignment_Differences}
\index{classes!assignment}
\index{records!assignment}

Assignment to a class variable is performed by reference, whereas assignment to
a record is performed by value.  When a variable of class type is assigned to
another variable of class type, they both become names for the same object.  In
contrast, when a record variable is assigned to another record variable, then
contents of the source record are copied into the target record field-by-field.

When a variable of class type is assigned to a record, matching fields (matched
by name) are copied from the class instance into the corresponding record
fields.  Subsequent changes to the fields in the target record have no effect
upon the class instance.

Assignment of a record to a class variable is not permitted.

\subsection{Arguments}
\label{Argument_Differences}
\index{classes!arguments}
\index{records!arguments}

Record arguments use the \chpl{const ref} intent by default - in contrast
to class arguments which pass by \chpl{const in} intent by default.

Similarly, the \chpl{this} receiver argument is passed by \chpl{const in} by
default for class methods. In contrast, it is passed by \chpl{ref} or
\chpl{const ref} by default for record methods.

\subsection{No {\em nil} Value}
\index{nil@\chpl{nil}!not provided for records}

Records do not provide a counterpart of the \chpl{nil} value.  A variable of
record type is associated with storage throughout its lifetime, so \chpl{nil}
has no meaning with respect to records.

\subsection{The {\em delete} operator}
\label{Record_Delete_Illegal}
\index{records!delete illegal}
\index{delete!illegal for records}

Calling \chpl{delete} on a record is illegal.

%REVIEW: we could discuss this:
%An explicit call to \chpl{delete} with a record argument has no effect.  The
%compiler may treat this as a hint that the record should not be accessed later
%within its scope and diagnose that case.

\subsection{Default Comparison Operators}
\label{Comparison_Operator_Differences}
\index{classes!comparison}
\index{records!comparison}

For records, the compiler will supply default comparison operators if
they are not supplied by the user.  In contrast, the user cannot redefine
\chpl{==} and \chpl{!=} for classes.  The default comparison operators
for a record examine the arguments' fields, while the comparison
operators for classes check whether the l.h.s. and r.h.s. refer to the
same class instance or are both \chpl{nil}.
