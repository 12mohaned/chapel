\sekshun{Unions}
\label{Unions}
\index{unions}

Unions have the semantics of records, however, only one field in the
union can contain data at any particular point in the program's
execution.  Unions are safe so that an access to a field that does not
contain data is a runtime error.  When a union is constructed, it is
in an unset state so that no field contains data.

\subsection{Union Types}
\label{Union_Types}

The syntax of a union type is summarized as follows:
\begin{syntax}
union-type:
  identifier
\end{syntax}
The union type is specified by the name of the union type.  This
simplification from class and record types is possible because generic
unions are not supported.

\subsection{Union Declarations}
\label{Union_Declarations}

\index{union@\chpl{union}}
A union is defined with the following syntax:
\begin{syntax}
union-declaration-statement:
  `union' identifier { union-statement-list }

union-statement-list:
  union-statement
  union-statement union-statement-list

union-statement:
  type-declaration-statement
  function-declaration-statement
  variable-declaration-statement
  empty-statement
\end{syntax}

\subsubsection{Union Fields}
\label{Union_Fields}
\index{unions!fields}

Union fields are accessed in the same way that record fields are
accessed.  It is a runtime error to access a field that is not
currently set.

Union fields should not be specified with initialization expressions.

\subsection{Union Assignment}
\label{Union_Assignment}
\index{unions!assignment}

Union assignment is by value.  The field set by the union on the
right-hand side of the assignment is assigned to the union on the
left-hand side of the assignment and this same field is marked as set.

\subsection{The Type Select Statement and Unions}
\label{The_Type_Select_Statement_and_Unions}
\index{unions!type select@\chpl{type select}}

The type-select statement can be applied to unions to access the
fields in a safe way by determining the type of the union.
