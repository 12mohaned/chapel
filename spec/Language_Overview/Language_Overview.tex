The current dominant parallel programming paradigm is characterized by a localized
view of the computation combined with explicit control
over message passing, as exemplified by a combination
of Fortran or C/C++ with MPI. Such a fragmented memory
model provides the programmer with full control over data
distribution and communication, at the expense of productivity,
conciseness, and clarity.

Chapel is a new parallel programming language being developed by Cray Inc. 
Chapel strives to improve the programmability of parallel computer systems
by providing a higher level of expression 
than current parallel languages do and by improving the separation between 
algorithmic expression and data structure implementation details. 

Chapel supports a global-view parallel programming model at a high level by 
supporting abstractions for data parallelism, task parallelism, and nested parallelism. 
It supports optimization for the locality of data and computation in the program 
via abstractions for data distribution and data-driven placement of subcomputations. 
It supports code reuse and generality via object-oriented concepts and generic 
programming features. While Chapel borrows concepts from many preceding languages, 
its parallel concepts are most closely based on ideas from High-Performance Fortran 
(HPF), ZPL, and the Cray MTA's extensions to Fortran/C. 


