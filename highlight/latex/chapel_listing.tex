\lstdefinelanguage{chapel}
  {
    morekeywords={
      align, as, atomic,
      begin, bool, borrowed, break, by, bytes
      catch, class, cobegin, coforall, complex, config, const, continue,
      defer, delete, dmapped, do, domain,
      else, enum, except, export, extern,
      false, for, forall, forwarding,
      if, imag, in, index, inline, inout, int, iter,
      label, lambda, let, lifetime, local, locale,
      module,
      new, nil, noinit,
      on, only, opaque, otherwise, out, override, owned,
      param, pragma, primitive, private, proc, prototype, public,
      range, real, record, reduce, ref, require, return,
      scan, select, serial, shared, single, sparse, string, subdomain, sync,
      then, this, throw, throws, true, try, type,
      uint, union, unmanaged, use,
      var,
      when, where, while, with,
      yield,
      zip
    },
    sensitive=false,
    mathescape=true,
    morecomment=[l]{//},
    morecomment=[s]{/*}{*/},
    morestring=[b]",
}

\lstset{
    basicstyle=\footnotesize\ttfamily,
    keywordstyle=\bfseries,
    commentstyle=\em,
    showstringspaces=false,
    flexiblecolumns=false,
    numbers=left,
    numbersep=5pt,
    numberstyle=\tiny,
    numberblanklines=false,
    stepnumber=0,
    escapeinside={(*}{*)},
    language=chapel,
  }

%\newcommand{\chpl}[1]{\lstinline[language=chapel,basicstyle=\ttfamily,keywordstyle=\bfseries]!#1!}
\newcommand{\chpl}[1]{\lstinline[language=chapel,basicstyle=\small\ttfamily,keywordstyle=]!#1!}
\newcommand{\varname}[1]{\emph{#1}}
\newcommand{\typename}[1]{\emph{#1}}
\newcommand{\fnname}[1]{\chpl{#1}}

\lstnewenvironment{chapel}{\lstset{language=chapel,xleftmargin=2pc,stepnumber=0}}{}
\lstnewenvironment{invisible}{\lstset{language=chapel,xleftmargin=2pc,stepnumber=0,keywordstyle=\bfseries\color{white},basicstyle=\small\ttfamily\color{white}}}{}
\lstnewenvironment{chapel0}{\lstset{language=chapel,stepnumber=0}}{}

\lstnewenvironment{numberedchapel}{\lstset{language=chapel,xleftmargin=15pt,stepnumber=1}}{}

\lstnewenvironment{chapelcode}{\lstset{language=chapel,stepnumber=1}}{}

% Uses the same listing style as the {chapel} environment, but keyword
% formatting is turned off.  The argument is ignored in LaTeX
% but used to name the .good file during test extraction.
% The argument must be supplied but may be empty.
% If empty it defaults to null, which signals the test extractor to 
% autogenerate the .good file name as ``<test_name>.good''.
\lstnewenvironment{chapelprintoutput}[1]
  {\lstset{language=chapel,xleftmargin=2pc,stepnumber=0,keywordstyle=}}{}

\lstnewenvironment{commandline}{\lstset{keywordstyle=,xleftmargin=2pc}}{}

\lstnewenvironment{protohead}{\lstset{language=chapel,xleftmargin=0pc,belowskip=-10pt,stepnumber=0}}{}

\newenvironment{protobody}{\begin{description}\item[\quad\quad] }{\end{description}}
