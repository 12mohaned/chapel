 \documentclass[times,10pt]{article}
%\documentclass{sig-alternate}
\usepackage{latex8}
\usepackage{times}
\usepackage{tabularx}
\usepackage{fancyvrb}
\usepackage{xspace}
\usepackage{amsmath}
\usepackage{multicols}
\usepackage{graphicx}
\usepackage{wrapfig}
\usepackage{algorithm}
\usepackage{algorithmic}
\usepackage[subnum]{cases}
%\usepackage[boxed]{algorithm2e}
\newcommand{\theHalgorithm}{\arabic{algorithm}}
% used to control vertical space around wrapfigs
\newlength{\saveintextsep}
\setlength{\saveintextsep}{\intextsep}
% Set up \ifpdf test for DVI vs. PDF support
\newif\ifpdf
\ifx\pdfoutput\undefined
    \pdffalse
\else
    \pdftrue
\fi
\usepackage{epsfig}
\usepackage[numbers,sort&compress]{natbib}
\setlength{\bibsep}{0pt}
%%% How graphics handled various with .PS or .PDF output
% \ifpdf
%     \usepackage[pdftex]{graphicx}
%     \pdfcompresslevel=9
% \else
%     \usepackage{graphicx}
%     \DeclareGraphicsExtensions{.epsi}
% \fi
%\usepackage{subfigure}
\usepackage{calc}
% Set \HalfWidth to help zplace two graphics side by side
\newlength{\HalfWidthGap}
\setlength\HalfWidthGap{0.25in}
\newlength{\HalfWidth}
\setlength\HalfWidth{0.5\textwidth - 0.5\HalfWidthGap}
\usepackage{url}
% \myurl makes URL text smaller...
\newcommand\myurl{\begingroup \urlstyle{same}\footnotesize\Url}
% Troublesome URLs (such as those containing underscores) may not work inside
% \url{} command:  instead, define them as macros here, and refer to them in
% .bib file inside '{}' to pass verbatim to latex.
\urldef{\urlvia}\myurl{http://developer.intel.com/design/servers/vi/developer/ia_imp_guide.htm}
\usepackage[
    bookmarks,
    bookmarksopen=true,
    bookmarksnumbered=true,
    breaklinks=true,
    pdfpagemode=UseOutlines,
    colorlinks=true,
    linkcolor=blue,
    citecolor=blue,
    pagecolor=blue,
    urlcolor=blue]{hyperref}  % should come last after all other usepackages
\ifpdf
\else
\hypersetup{colorlinks=false,breaklinks=true} % disable color links for postscript output
\fi
% provide a list of all files used
%\listfiles
% Macros to gulp large vertical space after figures
\newcommand{\GulpFig}{\vspace{-12pt}}
% macros for 'put' and 'get' (\xspace prevent \PUT from gobbling space)
\newcommand{\GET}{`get'\xspace}
\newcommand{\GETS}{`gets'\xspace}
\newcommand{\PUT}{`put'\xspace}
\newcommand{\PUTS}{`puts'\xspace}
\newcommand{\nx}{\mathsf{NX}}
\newcommand{\ny}{\mathsf{NY}}
\newcommand{\nz}{\mathsf{NZ}}
\newcommand{\prefetch}{\mathsf{\eta}}
\newcommand{\threads}{\mathsf{THREADS}}
\newcommand{\mythread}{\mathsf{MYTHREAD}}
\newcommand{\pencils}{\textit{Pencils}~}
\newcommand{\slabs}{\textit{Slabs}~}
\newcommand{\exchange}{\textit{Exchange}~}
\newcommand{\slabsnospace}{\textit{Slabs}}
\newcommand{\exchangenospace}{\textit{Exchange}}
\newcommand{\spmvgatherbulk}{\textit{SPMV Bulk-Gather}~}
\newcommand{\spmvgather}{\textit{SPMV Ga\-ther}~}
\newcommand{\spmvscatter}{\textit{SPMV Sca\-tter}~}
\newcommand{\alglabelref}[1]{#1(Alg.\ref{#1})}
%\newcommand{\threads}{\mathsf{P}}
\newcommand{\threadcols}{\mathsf{threadcols}}
% It's very easy to forget the comma after i.e. or .e.g., or to forget to stop
% the extra space after et al., etc.
\newcommand{\ie}{i.e.,\xspace}
\newcommand{\eg}{e.g.,\xspace}
\newcommand{\etal}{et al.\@\xspace}
\newcommand{\etc}{etc.\@\xspace}
%defs for theorems, etc.
%\newtheorem{theorem}{Theorem}
%\newtheorem{lemma}[theorem]{Lemma}
%\newtheorem{proposition}[theorem]{Proposition}
%\newtheorem{claim}[theorem]{Claim}
%\newtheorem{corollary}[theorem]{Corollary}
%\newtheorem{definition}[theorem]{Definition}
%\newtheorem{problem}[theorem]{Problem Definition}
%\newenvironment{proof}{{\bf Proof:}}{\hfill\rule{2mm}{2mm}}
\newcommand{\tablabel}[1]{\label{tab:#1}}
\newcommand{\tabref}[1]{Table~\ref{tab:#1}}
\newcommand{\figlabel}[1]{\label{fig:#1}}
\newcommand{\figref}[1]{Figure~\ref{fig:#1}}
\newcommand{\eqqlabel}[1]{\label{eq:#1}}
\newcommand{\eqqref}[1]{\eqref{eq:#1}}
\newcommand{\alglabel}[1]{\label{a:#1}}
\newcommand{\algref}[1]{Algorithm~\ref{a:#1}}

\newcommand{\jacquard}{Op\-te\-ron/In\-fini\-band~}
\newcommand{\bassi}{Pow\-er5\-/Fed\-er\-a\-tion~}

%%%%%%%%%%%%%%%%%%%%%%%%%%%%%%%%%%%%%%%%%%%%%%%%%%%%%%%%%%%%%%%%%%%%%%%%%%%%%
%   Begin the document.                             %
%%%%%%%%%%%%%%%%%%%%%%%%%%%%%%%%%%%%%%%%%%%%%%%%%%%%%%%%%%%%%%%%%%%%%%%%%%%%%
\begin{document}

\newcommand{\gasnet}[0]{GASNet\ }





%%%%%%%%%%%%%%%%%%%%%%%%%%%%%%%%%%%%%%%%%%%%%%%%%%%%%%%%%%%%%%%%%%%%%%%%%%%%%
%   Title                                   %
\title{\gasnet Teams Specification and Design Document}
\author{Rajesh Nishtala, Dan Bonachea, Paul Hargrove\\
    \emph{rajeshn@cs.berkeley.edu, bonachea@cs.berkeley.edu, PHHargrove@lbl.gov} \\\\
    Computer Science Division, University of California at Berkeley \\ 
    Future Technologies Group, Lawrence Berkeley National Lab }
\maketitle
%%%%%%%%%%%%%%%%%%%%%%%%%%%%%%%%%%%%%%%%%%%%%%%%%%%%%%%%%%%%%%%%%%%%%%%%%%%%%


\section{Introduction}
In a variety of applications the use of collective communication is very
important for both expressing the communication pattern of the application as
well as achieving the best performance. However, one of the fundamental aspects
that makes the collectives useful is the ability to use them on a subset (or
\textit{team}) of the processors rather than all the processors in the program.
This document will focus on the design and implementation of these teams within
\gasnet as well as the API that exposes them to applications or libraries that
use \gasnet. 

Since \gasnet allows hierarchical declaration of nodes and images we must be
careful about the terminology we use and how we compare our implementation with
MPI's. Henceforth, we will use ``node" to refer to a \gasnet process. Each of
these \gasnet nodes can potentially contain many  \gasnet images (which can be
implemented on top of different threads if one chooses). The distinction allows
one to perform shared memory communication optimizations across various images
within the same node since they will share the same address space. In MPI
parlance this hierarchy is not explicit and thus all the threads of control
visibile to the application are at the same level which we will call ``tasks."
Our teams will consist of \gasnet images rather than nodes since we want
different threads in the same process to be a part of different teams. We will
allow images to have their own relative ranks within different teams but the
\gasnet node identifier will remain fixed throughout the life of the program. 

\section{Design Goals}
\begin{itemize}
\item One of the important and fundamental characteristics of \gasnet that
needs to be addressed in the design of teams is the ability to have multiple
``images" of \gasnet on the same \gasnet process, thus allowing one to perform
shared memory/intra-node optimizations for collectives whenever possible. This
ability will need to also be effectively exposed in the design of the teams. 

\item Each team of images will need to be sufficiently isolated from the others
so that the communication and synchronization mechanisms will never interfere
with those from another team. This will allow libraries to be written on top of
\gasnet that take teams as arguments and be guaranteed that operations on one
team will be isolated from operations performed on another team provided that
they do not share members. If they do, then special care must be done at the
application level.

\item The team creation and destruction must be abstract and expressive enough
to easily and quickly build up whatever subset of processors (GASNet images) the user wishes.

\item The teams must fit seamlessly into the existing \gasnet collectives
framework so that collective performance on all the images in a \gasnet program
is not hindered by the addition of teams. 

\end{itemize}

\section{Similarities and Differences with MPI}
The problem of communication with a subset of the processors is certainly not a
new one and has been around in the MPI specification for many years. In MPI
parlance the notion of teams is expressed as communicators. Many of the ideas
presented in the API for MPI communicators (and groups) are orthogonal to their
use of two-sided communication and their design of the collectives and thus a
lot of these ideas are applicable here. However, there are some important
differences that will need to be addressed as well. 

\subsection{Similarities}
\begin{itemize}
\item \textbf{Communicators:} As stated above, the notion of an MPI
communicator is very similar to our concept of teams, modulo some of the
differences stated below. A team, like a communicator, is a set of \gasnet
images along with a preallocated set of buffer space and communication meta
data (e.g. a distributed buffer space manager) to allow fast collective
communication across the various members of the team. 

\item \textbf{Groups:} We will also adopt the idea of an MPI Group into
\gasnet. The primary function of a \gasnet group is to allow easy team
construction. The process of creating a team is an expensive process due to the
setup of all the required meta-data. Since a group merely an ordered set of
images without the additional meta data necessary for communication and
synchronization their construction and modification can be done using much
simpler, and therefore more efficient, operations. Thus the general model, like
MPI, will be to build up a group based on the different operations provided in
the API and then construct a team around a group once the group has been
finalized. 
  
%\item \textbf{API:} Our initial API for both teams and groups have been
%heavily adopted from the MPI. However as we will show there are a few
%differences that need to be addressed. 

\end{itemize}

\subsection{Differences}
\begin{itemize}
\item \textbf{\gasnet Images:} As stated above the biggest difference
(especially in the implementation) that needs to be addressed is the difference
between a node an image and how these relate to the teams. MPI does not have
this problem since there is no notion of a hierarchical structure between MPI
tasks; they are all at the same level. 

\item \textbf{Scratch Space:} \gasnet has an explicit segment which allows one
to take advantage of some very important communication optimizations (e.g.
RDMA) in one-sided operations. Since we wish to leverage some of these same
optimizations in the collectives, the collectives (and thus teams) will need to
reclaim some of the space that has been allocated to the \gasnet client to provide
the best possible performance. The auxiliary scratch space for the initial
GASNET\_TEAM\_ALL will be handled directly within \gasnet so  that it isn't
visible to the end user, however further team construction will necessitate the
\gasnet client explicitly managing these buffers.

%Note that since \gasnet is indented as a compiler target, this will probably
%not be exposed to final end users and will only be relevant to writers of a
%runtime

\item \textbf{Usage:} Another important and distinct difference is that MPI
allows one to use a communicator for isolation of point-to-point messaging operations such as sends and
receives. GASNet is a one-sided communication system that lacks such two-sided
message passing operations, and the teams are not relevant for one-sided
point-to-point communication. \gasnet teams are only relevant for use in the 
collectives library.  Thus each image will have
one globally unique name for point-to-point communication. The translation
routines that are provided so that one can specify the root for rooted
collectives relative to the other images in the team. 

\end{itemize}


\newcommand{\group}[0]{gasnet\_group\_t}
\newcommand{\grouparg}[0]{gasnet\_group\_t\ g}


\newcommand{\team}[0]{gasnet\_team\_t}
\newcommand{\teamarg}[0]{gasnet\_team\_t\ t}

\newcommand{\node}[0]{gasnet\_node\_t}
\newcommand{\nodearg}[0]{gasnet\_node\_t\ n}

\newcommand{\image}[0]{gasnet\_image\_t}
\newcommand{\imagearg}[0]{gasnet\_image\_t\ i}

\section{Groups}
A \gasnet group is defined as an ordered set of \gasnet images that will take
part in the construction of a \gasnet team. A group is designed to be simple
and easy to construct based on other groups. All these operations are
non-collective. The following is a list of the operations that will operate on
and work with a \group. 

Every operation that returns a \group value logically creates a new group
object that should later be reclaimed by calling $gasnet\_group\_free$.

\begin{itemize}

\item Type \image \\
An unsigned integral type capable of representing the largest number of images
permitted in the current \gasnet implementation.

\item Type \group \\
An abstract type representing ordered set of GASNet images, where each image is 
assigned a unique, 0-based index of type \image, also referred to as the {\it rank} of the image within the group.
In a \group representing $N$ images, the rank indexes shall comprise the set $\{0, ... , (N-1)\}$.

\item Constant \texttt{\image\ GASNET\_IMAGE\_UNDEFINED } \\ 
A constant defined for an invalid image number. 

\item GlobalVar: \texttt{\group\ GASNET\_GROUP\_EMPTY} \\ 
A predefined group representing an empty set of images, to aid in group creation. 

\item \texttt{\image\ gasnet\_group\_size(\grouparg)}\\ 
Returns the total number of images in a \texttt{\group} $g$. 

\item \texttt{\image\ gasnet\_group\_size\_on\_node(\grouparg, \nodearg)} \\ 
Returns the number of images residing on node $n$ that are a member of group $g$. 

\item \texttt{\image\ gasnet\_group\_my\_image(\grouparg)} \\ 
Returns the image rank in $g$ of the calling thread, or \texttt{GASNET\_IMAGE\_UNDEFINED} 
if the image corresponding to the calling thread is not a member of $g$.

\item \texttt{void gasnet\_group\_translate\_images(\group\ A, \group\ B,\\ 
              \image\ *input\_vec, \image\ *output\_vec, \image\ vec\_len)} \\ 
Given a set of input images in $input\_vec$ from group $A$ the function will
translate them to their corresponding ranks in $B$ and write the result into
$output\_vec$. If some images do
not exist in node $B$, the corresponding array slot will be filled with
GASNET\_IMAGE\_UNDEFINED.

\item \texttt{\node\ gasnet\_group\_image\_to\_node(\grouparg, \imagearg)} \\ 
Given a relative rank $i$ in $g$, the function returns the node $n$ that
contains the image. 

\item \texttt{void gasnet\_group\_node\_to\_images(\grouparg, \\ 
              \nodearg, \image\ *outimages)} \\ 
Fills the output array $outimages$ with the ranks of the images in $g$ that reside on node $n$. Use
\texttt{gasnet\_group\_size\_on\_node} to get the count of how many elements
this function will return. 

\item \texttt{enum\{GASNET\_GROUP\_IDENT , GASNET\_GROUP\_SIMILAR, GASNET\_GROUP\_UNEQUAL\} \\ 
           gasnet\_group\_compare(\group\ a, \group\ b)} \\ 
This function returns one of three values:  
\begin{enumerate}
\item If $a$ and $b$ represent the same set of images with the same rank ordering,
the function will return $GASNET\_GROUP\_IDENT$. 

\item If $a$ and $b$ represent the same set of images but the
ranks of the images are different then the function returns
$GASNET\_GROUP\_SIMILAR$.

\item Otherwise the function will return $GASNET\_GROUP\_UNEQUAL$.

\end{enumerate}

\item \texttt{\group\ gasnet\_group\_from\_team(\teamarg)} \\ Returns a
\texttt{\group} that represents the images in $t$. 

\item \texttt{\group\ gasnet\_group\_incl(\grouparg, \image\ * in\_images, \image\ image\_count)} \\ 
Returns a group that consists of the $image\_count$ images
in group $g$ with ranks $in\_images[0], ... , in\_images[n-1]$; the image with rank i in the returned group is the
image with rank $in\_images[i]$ in $g$. Each of the $image\_count$ elements of $in\_images$ must be a valid rank
in group $g$ and all elements must be distinct, or else the program is erroneous. If $image\_count = 0$,
then newgroup is $GASNET\_GROUP\_EMPTY$. This function can, for instance, be used to reorder
the elements of a group.

\item \texttt{\group\ gasnet\_group\_excl(\grouparg, \image\ * in\_images, \image\ image\_count)} \\ 
Returns a \texttt{\group} that removes the images with ranks contained in $in\_images$
from $g$. The relative ordering of the remaining images from $g$ will not change. 
Each of the $image\_count$ elements of $in\_images$
must be a valid rank in $g$ and all elements must be distinct; otherwise, the program is
erroneous. If $image\_count = 0$, then the returned group is identical to $g$.


\item \texttt{\group\ gasnet\_group\_diff(\group\ A, \group\ B)} \\ 
Returns a \texttt{\group} that is the set difference of $A$ and $B$. The
resultant set will be all the images of $A$ that are not in $B$, ordered by
their relative rankings in $A$. 

\item \texttt{\group\ gasnet\_group\_intersect(\group\ A, \group\ B)} \\ 
Returns a \texttt{\group} that is the set intersection of $A$ and $B$. The
images are ordered according to their ordering in $A$. 

\item \texttt{\group\ gasnet\_group\_union(\group\ A, \group\ B)} \\ 
Returns a \texttt{\group} that is the set union of $A$ and $B$.  The images of
$A$ are placed before the images of $B$ in the ordering. 

\item \texttt{size\_t gasnet\_group\_to\_node\_scratch\_size(\grouparg)} \\ 
Given a group $g$ the function returns the amount of node-local scratch space 
the calling thread would need to provide in order to construct a team from $g$
using $gasnet\_team\_from\_group$. Guaranteed to return zero if the image 
associated with the calling thread is not a member of $g$.

\item \texttt{size\_t gasnet\_split\_to\_node\_scratch\_size(\teamarg, \image\ color, \image\ relrank)} \\ 
This is the only group-like function that is a collective call - it must be called collectively by
all images comprising team $t$.
The call returns the amount of node-local space the calling thread would need to provide in order to
perform a split of the team $t$ with the corresponding $color$ and $relrank$
arguments that one would use for the \texttt{gasnet\_team\_split()} function.
If there are multiple teams that will be created
on the same node, the function will return a valid size for the scratch space
for one representative image from each of the teams that will reside on the
node. A nonzero return value indicates that this image must provide the indicated
amount of scratch space for the corresponding team split. 
Guaranteed to return zero to any thread which passes $GASNET\_IMAGE\_UNDEFINED$ as $relrank$.

%\item GroupRangeIncl (maybe: scalability) ???
%\item GroupRangeExcl (maybe: scalability) ???
\item \texttt{void gasnet\_group\_free(\grouparg)} \\ 
This function will perform the required housekeeping to reclaim any resources
associated with group $g$.  

\end{itemize}



\section{Team Operators}
All team constructor operations are collectively invoked over a parent team.
All team constructor operations require local scratch space to be provided,
with a size indicated by the corresponding group operations. The scratch space
provided must reference valid memory residing within the bounds of the local GASNet segment
(for $GASNET\_SEGMENT\_EVERYTHING$, any valid local memory location is suitable).
The provided scratch space is consumed by the team construction routine, 
and must not be accessed by the client throughout the lifetime of the created team.
\begin{itemize}

\item GlobalVar: \texttt{GASNET\_TEAM\_ALL}  \\ 
Initial team that all images are a part of. Comparable to MPI\_COMM\_WORLD. 

\item \texttt{\image\ gasnet\_team\_my\_image(\teamarg)} \\ 
Given a team the function returns a relative rank of the calling image within
$t$. If the calling image is not contained in $t$ then
GASNET\_IMAGE\_UNDEFINED is returned. 

\item \texttt{\image\ gasnet\_team\_size(\teamarg)} \\ 
Returns the number of images in $t$.

\item \texttt{\team\ gasnet\_team\_split(\teamarg, \\ 
              \image\ color, \image\ relrank, void *local\_scratch\_space)} \\ 
This function must be called collectively by all images comprising team $t$.
This function splits the team $t$ based on the $color$ and $relrank$ arguments.
All images that call the function with the same $color$ argument will be
separated into the same team. The $relrank$ argument specifies the relative
position of the image within the new team. If multiple images call the split
function with the same color and relrank argument the results are undefined.
Note that an image is responsible for allocating the local\_scratch\_space if the
corresponding call to \texttt{gasnet\_split\_to\_node\_scratch\_size()} returned a
nonzero value. One can pass GASNET\_IMAGE\_UNDEFINED to the $relrank$ to
indicate that the image is not to be part of any team. 

Note this function can be used to duplicate an existing team $t$, by having all
images pass $color = 0$ and $relrank = gasnet\_team\_my\_image(t)$.

\item \texttt{\team\ gasnet\_team\_from\_group(\teamarg, \\\grouparg, void\ *local\_scratch\_space)} \\ 
This function must be called collectively by all images comprising team $t$.
This function will create a child team of $t$ from the group $g$ with the given
scratch space.  The results are undefined if all members of $g$ are not part of
$t$. 

%\item (TeamSplit is sufficient for Team Duplicate)

\item \texttt{void* gasnet\_team\_free(\teamarg)} \\ 
This function must be called collectively by all images comprising team $t$.
Free a team by cleaning up all associated resources with the team. Once the
free routine returns it is safe to free the scratch space that was passed in
to create the team. As a convenience, the original argument scratch space
argument is returned to each image to facilitate reclaiming the associated storage.

%\item No TeamCompare (can just do pointer comparison on Teams)
\end{itemize}




\section{Notes to Self About Implementation.}
\begin{enumerate}
\item Keep Team Hierarchy
\item Reference Count Groups
\end{enumerate}

\end{document}
